% !TEX root = ../main.tex
\section{Background}
Due to the open-endedness of characters in Yuadrem (see page \pageref{sec::classlessdnd} for more details), backgrounds as they are defined in the official D\&D5e books are not compatible with this book.
An alternative is proposed, where your character's background complements their kin, giving them a unique feat, and benefits when learning feats related to their background.

To choose religions and learned languages, refer to their respective sections in pages \pageref{ssec::religions} and \pageref{ssec::languages}.
These are not complete lists by any means, but rather the most common in the civilized world. %, and you are welcome to talk with your DM about including religions and languages as they may seem fit in your campaign.

\pagebreak

% For your character's background, you can choose any from some of the official D\&D5e books which fit best with your character.
% The specific books from which you are welcome to pick a background are Acquisitions Incorporated, Ghosts of Saltmarsh, the Player's Handbook, Mythic Odysseys of Theros, and Tomb of Annihilation.

% The only change you need to apply is to convert any initial money to the local currency of the region where you're playing.
% To do this, convert the GP to the local currency using the currencies table in page \pageref{sec::currency} using the selling value.

\textbf{TODO: EXPLAIN THE MECHANICS}
% TODO: Add available feats at the end of each subsection, it will make life easier to the player.

% TEMPLATE:
% \subsection*{NAME} \label{ssec::name}
%     DESCRIPTION
%     \subparagraph{Competences} A, plus your choice of one from among B, C, or D.
%     \subparagraph{Equipment}
%     \subsubsection{FEATURE}

\subsection*{Acolyte} \label{ssec::acolyte}
    Clerics, cultists, fanatics, and priests all share a devotion to something larger than themselves.
    Acolytes are shaped by their experience in temples or other religious communities.
    Their study of the history and tenets of their faith and their relationships to temples, shrines, or hierarchies affect their mannerisms and ideals.
    \subparagraph{Competences} Religion, plus your choice of two from among History, Insight, Persuasion, or a language of your choice.
    \subparagraph{Equipment} A holy symbol representative of your religion, a prayer book or prayer wheel, 5 sticks of incense, vestments, and a set of common clothes.
    \subsubsection{Shelter of the Faithful} \label{feat::shelterofthefaithful}
        You command the respect of those who share your faith, and you can perform the religious ceremonies of your deity.
        You and your companions can expect to receive free healing and care at a temple, shrine, or other established presence of your faith, though you must provide any material components needed for spells.
        Those who share your religion will support you (but only you) at a modest lifestyle.

        You might also have ties to a specific temple dedicated to your chosen deity or pantheon, and you have a residence there.
        This could be the temple where you used to serve, if you remain on good terms with it, or a temple where you have found a new home.
        While near your temple, you can call upon the priests for assistance, provided the assistance you ask for is not hazardous and you remain in good standing with your temple.
    % ========================================================================== %

\subsection*{Artisan} \label{ssec::artisan} % TODO: FIX COMPETENCES
    Crafters, guild artisans, and engineers all fall into this category.
    You are a member of an artisan's guild, skilled in a particular field and closely associated with other artisans.
    You are a well-established part of the mercantile world, freed by talent and wealth from the constraints of a feudal social order.
    You learned your skills as an apprentice to a master artisan, under the sponsorship of your guild, until you became a master in your own right.
    \subparagraph{Competences} A set of artisan's tools of your choice, plus your choice of two from among A, B, C, or D.
    \subparagraph{Equipment} A set of artisan's tools with which you are proficient, a maker's mark chisel used to mark your handiwork with the symbol of your guild, and a set of traveler's clothes.
    \subsubsection{Experienced Crafter} \label{feat::experiencedcrafter}
        Knowing the local trade like the back of your hand, you know where to find the best ingredients for the cheapest prices in a settlement in which you are familiar.
        You can buy components and ingredients related to your craft for half their normal price, and they are always of at least fine quality.
        You gain familiarity with a settlement as part of a long rest.
    % ========================================================================== %

% Gate Urchin, Adopted, Haunted One, Urchin, Charlatan
% Black Fist Double Agent, Secret Identity, Spy, Augen Trust
\subsection*{Criminal} \label{ssec::criminal}
    Bandits, pirates, spies, and thieves all share a penchant for crime.
    You are an experienced criminal with a history of breaking the law.
    You have spent a lot of time among other criminals and still have contacts within the criminal underworld.
    \subparagraph{Competences} Deception, plus your choice of two from among Intimidation, Sleight of Hand, Stealth, or a set of Thieves' Tools.
    \subparagraph{Equipment} A crowbar, a black cloak with a hood, and a knife.
    \subsubsection{Criminal Contact}
        You have a reliable and trustworthy contact who acts as your liaison to a network of other criminals.
        You know how to get messages to and from your contact, even over great distances; specifically, you know the local messengers, corrupt caravan masters, and seedy sailors who can deliver messages for you.
    % ========================================================================== %

\subsection*{Entertainer} \label{ssec::entertainer}
    Acrobats, athletes, gladiators, and performers all share a passion for self-improvement and entertainment. % Considering athletes here is a bit shit but is the best I can do tbh.
    You thrive in front of an audience.
    You know how to entrance them, entertain them, and even inspire them.
    Whatever techniques you use, your art is their life.
    \subparagraph{Competences} Performance, plus your choice of two from among Acrobatics, Athletics, a musical instrument, or land vehicles.
    \subparagraph{Equipment} A musical instrument, an inexpensive weapon, a bronze discus, or a leather ball, a lucky charm or past trophy, and costume clothes.
    \subsubsection{By Popular Demand}
        You can always find a place to perform, may it be an inn, a tavern, an arena, a field, or even in a noble's court.
        At such a place, you receive free lodging and food of a modest or comfortable standard (depending on the quality of the establishment), as long as you perform each daily.
        In addition, your performance makes you something of a local figure.
        When strangers recognize you in a town where you have performed, they typically take a liking to you.
    % ========================================================================== %

% City Watch, Knight, Inquisitor, Investigator
\subsection*{Guard} \label{ssec::guard}
    DESCRIPTION
    \subparagraph{Competences} A, plus your choice of two from among B, C, D, or E.
    \subparagraph{Equipment}
    \subsubsection{FEATURE}
    % ========================================================================== %

% Earthspur Miner
\subsection*{Laborer} \label{ssec::laborer}
    DESCRIPTION
    \subparagraph{Competences} A, plus your choice of two from among B, C, D, or E.
    \subparagraph{Equipment}
    \subsubsection{FEATURE}
    % ========================================================================== %

\subsection*{Merchant} \label{ssec::merchant}
    Caravan masters, shopkeepers, and traders all focus on one thing: coin.
    They don't craft items themselves but earn a living by buying and selling the works of others (or the raw materials artisans need to practice their craft).
    Perhaps you transported goods from one place to another, by ship, wagon, or caravan, or bought them from traveling traders and sold them in your own little shop.
    \subparagraph{Competences} Persuasion, plus your choice of two from among Investigation, a set of navigator's tools, one language, or land vehicles.
    \subparagraph{Equipment} A mule and cart, a merchant's scale, a set of fine clothes, and a belt containing 10 agnomas (nickel coins, see page \pageref{sec::currency}).
    \subsubsection{Supply Chain} \label{feat::supplychain}
        From your time as a merchant, you retain connections with wholesalers, suppliers, and other merchants and entrepreneurs.
        You can call upon these connections when looking for items or information.
        In locations where you are not familiar, you are able to quickly establish this web of connections as part of a long rest.
    % ========================================================================== %

% Courtier, Inheritor, Mulmaster Aristocrat, Noble
\subsection*{Noble} \label{ssec::noble}
    You understand wealth, power, and privilege.
    You carry a noble title, and your family owns land, collects taxes, and wields significant political influence.
    You might be a pampered aristocrat unfamiliar with work or discomfort, a former merchant just elevated to the nobility, or a champion who gained status via knighthood.
    Or you could be an honest, hard-working landowner who cares deeply about the people who live and work on your land, keenly aware of your responsibility to them.
    \subparagraph{Competences} History, plus your choice of two from among Deception, Persuasion, Religion, or a language of your choice.
    \subparagraph{Equipment} A set of fine clothes, and signet ring, and a purse containing 25 agnomas.
    \subsubsection{Position of Priviledge}
        Thanks to your noble birth, people are inclined to think the best of you.
        You are welcome in high society, and people assume you have the right to be wherever you are.
        The common folk make every effort to accommodate you and avoid your displeasure, and other people of high birth treat you as a member of the same social sphere.
        You can secure an audience with a local noble if you need to.
    % ========================================================================== %

% Harborfolk, Sailor, Fisher, Marine, Shipwright
\subsection*{Sailor} \label{ssec::sailor}
    DESCRIPTION
    \subparagraph{Competences} A, plus your choice of two from among B, C, D, or E.
    \subparagraph{Equipment}
    \subsubsection{FEATURE}
    % ========================================================================== %

\subsection*{Scholar} \label{ssec::scholar}
    Sages, scholars, students, and scientists are all passionate about one or many fields of study.
    Scholars are defined by their extensive studies, and their characteristics reflect this life of study.
    Devoted to scholarly pursuits, a scholar values knowledge highly --- sometimes in its own right, sometimes as a means toward other ideals.
    \subparagraph{Competences} Investigation, plus your choice of two from among Arcana, History, Medicine, or one language.
    \subparagraph{Equipment} A bottle of black ink, a quill, a small knife, a letter from a dead colleague posing a question you have not yet been able to answer, and a set of common clothes.
    \subsubsection{Researcher} \label{feat::researcher}
        When you attempt to learn or recall a piece of lore, if you do not know that information, you often know where and from whom you can obtain it.
        Usually, this information comes from a library, scriptorium, university, or a sage or other learned person or creature.
        Your DM might rule that the knowledge you seek is secreted away in an almost inaccessible place, or that it simply cannot be found.
    % ========================================================================== %

% Soldier, Mercenary Veteran, Bounty Hunter
\subsection*{Soldier} \label{ssec::soldier}
    DESCRIPTION
    \subparagraph{Competences} A, plus your choice of two from among B, C, D, or E.
    \subparagraph{Equipment}
    \subsubsection{FEATURE}
    % ========================================================================== %

\subsection*{Outlander} \label{ssec::outlander}
    Either by choice or by birthright, you have spent much of your life in the wilds, far from the comforts of civilization.
    You've witnessed the migration of herds larger than forests, survived weather more extreme than any city-dweller could comprehend, and enjoyed the solitude of being the only thinking creature for miles in any direction.
    The wilds are in your blood, whether you were a nomad, an explorer, a recluse, a hunter-gatherer, or a hermit.
    Even in places where you don't know the specific features of the terrain, you know the ways of the wild.
    \subparagraph{Competences} Survival, plus your choice of two from among Animal Handling, Athletics, Nature, or a musical instrument.
    \subparagraph{Equipment} A staff, a hunting trap, a trophy from an animal you killed, and a set of traveler's clothes.
    \subsubsection{Wanderer}
        You have an excellent memory for maps and geography, and you can always recall the general layout of terrain, settlements, and other features around you.
        In addition, you can find food and fresh water for yourself and up to five other people each day, provided that the land offers berries, small game, water, and so forth.
    % ========================================================================== %

% Gambler
% Faction Agent, House Agent

% Strength
%     .2 Athletics
% Dexterity
%     .1 Acrobatics
%     .1 Sleight of Hand
%     .1 Stealth
% Intelligence
%     .1 Arcana
%     !2 History
%     !1 Investigation
%     .1 Nature
%     !1 Religion
% Wisdom
%     .1 Animal Handling
%     .2 Insight
%     .1 Medicine
%     .0 Perception
%     !0 Survival
% Charisma
%     !1 Deception
%     .1 Intimidation
%     !0 Performance
%     !2 Persuasion
% !2 Tools
% .2 Instrument
% .4 Languages
% .2 Vehicles (land)
% .0 Vehicles (water)
% .0 Vehicles (air)

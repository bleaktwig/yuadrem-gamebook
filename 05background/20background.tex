% !TEX root = ../main.tex
\section{Background}
Due to the open-endedness of characters in Yuadrem (see page \pageref{sec::classlessdnd} for more details), backgrounds as they are defined in the official D\&D5e books are not compatible with this book.
An alternative is proposed, where your character's background complements their kin, giving them a unique feat, and benefits when learning feats related to their background.

To choose religions and learned languages, refer to their respective sections in pages \pageref{ssec::religions} and \pageref{ssec::languages}.
These are not complete lists by any means, but rather the most common in the civilized world. %, and you are welcome to talk with your DM about including religions and languages as they may seem fit in your campaign.

\pagebreak

% For your character's background, you can choose any from some of the official D\&D5e books which fit best with your character.
% The specific books from which you are welcome to pick a background are Acquisitions Incorporated, Ghosts of Saltmarsh, the Player's Handbook, Mythic Odysseys of Theros, and Tomb of Annihilation.

% The only change you need to apply is to convert any initial money to the local currency of the region where you're playing.
% To do this, convert the GP to the local currency using the currencies table in page \pageref{sec::currency} using the selling value.

\textbf{TODO: EXPLAIN THE MECHANICS}

\subsection*{NAME} \label{ssec::name}
    DESCRIPTION
    \subparagraph{Competences} A, plus your choice of one from among B, C, or D.
    \subparagraph{Equipment}
    \subsubsection{FEATURE}

\subsection*{Acolyte} \label{ssec::acolyte}
    Clerics, cultists, fanatics, and priests all share a devotion to something larger than themselves.
    Acolytes are shaped by their experience in temples or other religious communities.
    Their study of the history and tenets of their faith and their relationships to temples, shrines, or hierarchies affect their mannerisms and ideals.
    \subparagraph{Competences} Religion, plus your choice of one from among History, Insight, and Persuasion.
    \subparagraph{Equipment} A holy symbol representative of your religion, a prayer book or prayer wheel, 5 sticks of incense, vestments, and a set of common clothes.
    \subsubsection{Shelter of the Faithful} \label{feat::shelterofthefaithful}
        You command the respect of those who share your faith, and you can perform the religious ceremonies of your deity.
        You and your companions can expect to receive free healing and care at a temple, shrine, or other established presence of your faith, though you must provide any material components needed for spells.
        Those who share your religion will support you (but only you) at a modest lifestyle.

        You might also have ties to a specific temple dedicated to your chosen deity or pantheon, and you have a residence there.
        This could be the temple where you used to serve, if you remain on good terms with it, or a temple where you have found a new home.
        While near your temple, you can call upon the priests for assistance, provided the assistance you ask for is not hazardous and you remain in good standing with your temple.
    % ========================================================================== %

\subsection*{Artisan} \label{ssec::artisan}
    Crafters, guild artisans, and engineers all fall into this category.
    You are a member of an artisan's guild, skilled in a particular field and closely associated with other artisans.
    You are a well-established part of the mercantile world, freed by talent and wealth from the constraints of a feudal social order.
    You learned your skills as an apprentice to a master artisan, under the sponsorship of your guild, until you became a master in your own right.
    \subparagraph{Competences} A set of artisan's tools of your choice, plus your choice of one from among Insight, Persuasion, or History.
    \subparagraph{Equipment} A set of artisan's tools with which you are proficient, a maker's mark chisel used to mark your handiwork with the symbol of your guild, and a set of traveler's clothes.
    \subsubsection{Experienced Crafter} \label{feat::experiencedcrafter}
        Knowing the local trade like the back of your hand, you know where to find the best ingredients for the cheapest prices in a settlement in which you are familiar.
        You can buy components and ingredients related to your craft for half their normal price, and they are always of at least fine quality.
        You gain familiarity with a settlement as part of a long rest.
    % ========================================================================== %

% Charlatan, Criminal, Pirate, Smuggler, Iron Route Bandit
\subsection*{Criminal} \label{ssec::criminal}
    DESCRIPTION
    \subparagraph{Competences} A, plus your choice of one from among B, C, or D.
    \subparagraph{Equipment}
    \subsubsection{FEATURE}
    % ========================================================================== %

% Athlete, Gladiator, Entertainer, Folk Hero
\subsection*{Entertainer} \label{ssec::entertainer}
    DESCRIPTION
    \subparagraph{Competences} A, plus your choice of one from among B, C, or D.
    \subparagraph{Equipment}
    \subsubsection{FEATURE}
    % ========================================================================== %

% \subsection*{Guard} \label{ssec::guard}
%     DESCRIPTION
%     \subparagraph{Competences} A, plus your choice of one from among B, C, or D.
%     \subparagraph{Equipment}
%     \subsubsection{FEATURE}
%     % ========================================================================== %

% Hermit, Nomad, Far Traveler
\subsection*{Hermit} \label{ssec::hermit}
    DESCRIPTION
    \subparagraph{Competences} A, plus your choice of one from among B, C, or D.
    \subparagraph{Equipment}
    \subsubsection{FEATURE}
    % ========================================================================== %

% Earthspur Miner
\subsection*{Laborer} \label{ssec::laborer}
    DESCRIPTION
    \subparagraph{Competences} A, plus your choice of one from among B, C, or D.
    \subparagraph{Equipment}
    \subsubsection{FEATURE}
    % ========================================================================== %

\subsection*{Merchant} \label{ssec::merchant}
    Caravan masters, shopkeepers, and traders all focus on one thing: coin.
    They don't craft items themselves but earn a living by buying and selling the works of others (or the raw materials artisans need to practice their craft).
    Perhaps you transported goods from one place to another, by ship, wagon, or caravan, or bought them from traveling traders and sold them in your own little shop.
    \subparagraph{Competences} Persuasion, plus your choice of one from among Investigation, a set of navigator's tools, or one language.
    \subparagraph{Equipment} A mule and cart, a merchant's scale, a set of fine clothes, and a belt containing 10 agnomas (nickel coins, see page \pageref{sec::currency}).
    \subsubsection{Supply Chain} \label{feat::supplychain}
        From your time as a merchant, you retain connections with wholesalers, suppliers, and other merchants and entrepreneurs.
        You can call upon these connections when looking for items or information.
        In locations where you are not familiar, you are able to quickly establish this web of connections as part of a long rest.
    % ========================================================================== %

% Harborfolk, Sailor, Fisher, Marine, Shipwright
\subsection*{Sailor} \label{ssec::sailor}
    DESCRIPTION
    \subparagraph{Competences} A, plus your choice of one from among B, C, or D.
    \subparagraph{Equipment}
    \subsubsection{FEATURE}
    % ========================================================================== %

\subsection*{Scholar} \label{ssec::scholar}
    Sages, scholars, students, and scientists are all passionate about one or many fields of study.
    Scholars are defined by their extensive studies, and their characteristics reflect this life of study.
    Devoted to scholarly pursuits, a scholar values knowledge highly --- sometimes in its own right, sometimes as a means toward other ideals.
    \subparagraph{Competences} History, plus your choice of one from among Arcana, Insight, or one language.
    \subparagraph{Equipment} A bottle of black ink, a quill, a small knife, a letter from a dead colleague posing a question you have not yet been able to answer, and a set of common clothes.
    \subsubsection{Researcher} \label{feat::researcher}
        When you attempt to learn or recall a piece of lore, if you do not know that information, you often know where and from whom you can obtain it.
        Usually, this information comes from a library, scriptorium, university, or a sage or other learned person or creature.
        Your DM might rule that the knowledge you seek is secreted away in an almost inaccessible place, or that it simply cannot be found.
    % ========================================================================== %

% City Watch, Knight, Knight of the Order, Soldier
\subsection*{Soldier} \label{ssec::soldier}
    DESCRIPTION
    \subparagraph{Competences} A, plus your choice of one from among B, C, or D.
    \subparagraph{Equipment}
    \subsubsection{FEATURE}
    % ========================================================================== %

% Tribe Member, Outlander
\subsection*{Outlander} \label{ssec::outlander}
    DESCRIPTION
    \subparagraph{Competences} A, plus your choice of one from among B, C, or D.
    \subparagraph{Equipment}
    \subsubsection{FEATURE}
    % ========================================================================== %

% Black Fist Double Agent, Secret Identity, Spy, Augen Trust
% Courtier, Inheritor, Mulmaster Aristocrat, Noble
% Gate Urchin, Adopted, Haunted One
% Inquisitor, Investigator
% Mercenary Veteran, Bounty Hunter
% Urchin
% Gambler
% Faction Agent, House Agent

% Strength
%     Athletics
% Dexterity
%     Acrobatics
%     Sleight of Hand
%     Stealth
% Intelligence
%     Arcana
%     Investigation
%     Nature
% Wisdom
%     Animal Handling
%     Insight
%     Medicine
%     Perception
%     Survival
% Charisma
%     Deception
%     Intimidation
%     Performance

% !TEX root = ../main.tex
\section{Alchemy} \label{sec::alchemy} % Blue tide, water, drill
Born from the blue tide, the alchemy doctrine focuses on the properties of materials and how to manipulate them for the alchemist's benefit.
An alchemist alters the world via their zuan, a pearl-like object that acts as the medium for their spells (see page \pageref{item::zuan}).
The intrinsic link between an alchemist and their zuan allows them to easily reshape its composition and properties via changing its tidal alignment.

\subparagraph{Nuagal Affinity}
    (see page \pageref{medium::dissolve})
    By dissolving the humidity in their zuan, alchemists are able to evaporate the liquid and emit streams of gas from it, often to deadly effect.
    Alchemists increment the blue alignment of their zuan to do this, increasing the range of the gas emitted by some of their spells.

\subparagraph{Kegal Affinity}
    (see page \pageref{medium::condense})
    By condensing the humidity around their zuan, alchemists increase the water content in it, spilling it upon casting spells.
    To do this, the indigo alignment of their zuan is incremented, allowing the effect of certain spells to linger over time.

\subparagraph{Fagal Affinity}
    (see page \pageref{medium::volatilize})
    Any spell cast by an alchemist increases the inherent volatility of their zuan.
    Unlike other alignments, an alchemist cannot increase volatility out of nothing, but only exchange other alignments for it.
    Silver alignment is inherently chaotic, and the more a zuan is aligned to Fagal the bigger the chance is for it to explode into catastrophe.

\subsection*{Deathtouching} \label{ssec::deathtouching}
    \textit{The deadliest of poisons can be the most beneficial panacea in the right hands.}

    The use of alchemical methods to produce potent poisons isn't uncommon, but Pharika's disciples were the first ones to perfect this craft.
    By infusing their weapons with arsenic, they toxify their victims causing heinous effects.
    Not all they do is harm though, and are known to use this same arsenic in small amounts to heal all sorts of ailments.

    The Deathtouchers spellcasting school was formed by alchemists who sought to understand Pharika's duality by mixing arsenic into their zuans.
    Many, however, join the school only in the pursuit of their toxins, and the lack of regulations has allowed the practice to be used by specialist criminals and assassins.

    When you choose this spellcasting school, you learn the \textbf{Toxic Cuts} nuagal affinity (see page \pageref{medium::toxiccuts}), as well as the \textbf{Pharika's Execution} spell (see page \pageref{spell::pharikasexecution}).
    Neither the medium nor the spell count towards your known mediums and spells.

\subsection*{Shellpoaching} \label{ssec::shellpoaching}
    \textit{An integral part of life, who would've though cooking could be used for something more than substenance.}

    Nobody questions that alchemy was created by the ets, yet the coming of the foreigner kins showed that this didn't mean that the ets knew everything.
    This is embodied by the school of Shellpoaching.
    The tortles, newly welcome to Yuadrem, brought with them the art of obtaining alchemical compounds not through complex formulas and reaction, but through cooking alone.

    Shellpoachers use the shells of long-dead turtles (and sometimes tortles) to poach food.
    They believe that in the act of not allowing the water to fully boil, they'll bring out new alchemical properties in their compounds without losing the natural properties of their reagents.

    When you choose this spellcasting school, you learn the \textbf{Bond} kegal affinity (see page \pageref{medium::bond}), as well as the \textbf{Chain Healing} spell (see page \pageref{spell::chainhealing}).
    Neither the medium nor the spell count towards your known mediums and spells.

\subsection*{Warchanting} \label{ssec::warchanting}
    \textit{Like Mogis and Iroas, we shall fight.
    Like Gelthul and Ethul, we shall die.}

    Little is known about the treb gats of the south, except for their restlessness in combat.
    While some of this might be inherent to their kin, a unique combination of brews and magic enhances this characteristic.
    Alchemists in charge of preparing these brews are called warchanters, and they are an essential component of Kofos' ever-razing armies.

    While many secrets of this school remain obscure, even the dull imitation that Seteshan alchemists have learned over the years provides great strength to the armies of the green city.
    Recklessly they handle the volatility of their zuan, passing it to a willing creature so that it can benefit from its oft unused potential.

    When you choose this spellcasting school, you learn the \textbf{Rage} fagal affinity (see page \pageref{medium::rage}), as well as the \textbf{Rhelath's Dance} spell (see page \pageref{spell::rhelathsdance}).
    Neither the medium nor the spell count towards your known mediums and spells.

    % Bloodhorn Minotaurs
    % Named for their blood-caked horns, the Bloodhorn minotaurs have ragged claws to supplement their charges and gores. Gleeful in their brutality, they slaughter and devour any intruders they encounter in the badlands, and particularly value the bone marrow of young humans. They take pride in their overlarge, razor-sharp horns.

    % Felhide Minotaurs
    % The notoriously dour Felhide minotaurs are descended from the warlord Thyrogog of the Ashlands. The Theriad recounts the brute's defeat and the loss of his great axe, Goremaster. Viewing Thyrogog's defeat as a divine sign, the warlord's descendants retreated into the Ashlands.
    % Burial rites among the Felhide minotaurs involve devouring those who fell in battle, to remove their shame from memory and fuel the survivors' revenge. Should another scavenger reach a fallen Felhide before the rest of the band can eat the dead minotaur's remains, the minotaurs mobilize to track down as much of their dead comrade's body as possible.

    % Ragegore Minotaurs
    % Ragegore minotaurs are the most ferocious of their kind, deeply infected by the bloodlust of Mogis. Ragegores never withdraw from a battle, entering a frenzy of furious delight at the sight of an enemy's blood. While in the heat of battle, a Ragegore minotaur seems to feel no pain and barely notices wounds that would kill a human. Some Ragegores have been known to fall dead immediately at the cessation of battle, their life sustained only by their fury.

\newpage

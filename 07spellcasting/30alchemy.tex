% !TEX root = ../main.tex
\section{Alchemy} \label{sec::alchemy}
Born from the blue tide, the alchemy doctrine focuses on the properties of materials and how to manipulate them for the alchemist's benefit.
An alchemist alters the world via their zuan, a pearl-like object that acts as the medium for their spells (see page \pageref{item::zuan}).
The intrinsic link between an alchemist and their zuan allows them to easily reshape its composition and properties via changing its tidal alignment.

% Zuan can be moved up to 4 meters as a free action.

\subparagraph{Nuagal Affinity}
    (see page \pageref{medium::dissolve})
    By dissolving the humidity in their zuan, alchemists are able to evaporate the liquid and emit streams of gas from it, often to deadly effect.
    Alchemists increment the blue alignment of their zuan to do this, increasing the range of the gas emitted by some of their spells.

\subparagraph{Kegal Affinity}
    (see page \pageref{medium::condense})
    By condensing the humidity around their zuan, alchemists increase the water content in it, spilling it upon casting spells.
    To do this, the indigo alignment of their zuan is incremented, allowing the effect of certain spells to linger over time.

\subparagraph{Fagal Affinity}
    (see page \pageref{medium::volatilize})
    Any spell cast by an alchemist increases the inherent volatility of their zuan.
    Unlike other alignments, an alchemist cannot increase volatility out of nothing, but only exchange other alignments for it.
    The silver alignment is chaotic, and the more a zuan is aligned to Fagal the bigger the chance is for it to explode into catastrophe.

\subsection*{Warchanting} \label{ssec::warchanting}
    \textit{Like Mogis and Iroas, we shall fight.
    Like Gelthul and Ethul, we shall die.}

    Little is known about the treb gats of the south, except for their restlessness in combat.
    While some of this might be inherent to their kin, a unique combination of brews and magic enhances this characteristic.
    Alchemists in charge of preparing these brews are called warchanters, and they are an essential component of Kofos' ever-razing armies.

    While many secrets of this school remain obscure, even the dull imitation that Seteshan alchemists have learned over the years provides great strength to the armies of the green city.
    Recklessly they handle the volatility of their zuan, passing it to a willing creature so that it can benefit from its oft unused potential.

    When you choose this spellcasting school, you learn the \textbf{Rage} fagal affinity (see page \pageref{medium::rage}), as well as the \textbf{Rhelath's Dance} spell (see page \pageref{spell::rhelathsdance}).
    Neither the medium nor the spell count towards your known mediums and spells.

\newpage

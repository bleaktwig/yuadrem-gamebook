% !TEX root = ../main.tex
\section{Meals} \label{sec::meals}
    Unless otherwise specified, the prices listed for meals are for a feast, which feeds up to 8 creatures.
    Divide this by 8 for the price of one meal.

    A meal is consumed during the day, and its benefits (if any) last for the whole day.
    After cooked, the benefits associated to a meal last for one day, after which the food spoils.
    You can only be affected by one meal at the same time.

    \begin{table*}[b]%
        \begin{DndTable}[width=\linewidth, header=Meals]{Xlccccc}
            \textbf{Meal} & \textbf{Rarity} & \textbf{Mats.} & \textbf{Total Cost} & \textbf{Tools} & \textbf{Weight} & \textbf{Source} \\
            Loaf of Bread           & Mundane   & 1 &      8 fobs    & COO & ---  & PHB   158 \\
            Bead of Nourishment     & Plain     & 1 &      1 agnoma  & COO & ---  & XGE   136 \\
            Bead of Refreshment     & Plain     & 1 &      1 agnoma  & COO & ---  & XGE   136 \\
            Dancing Monkey Fruit    & Plain     & 3 &      5 agnomas & COO & ---  & TOA   205 \\
            Essence of Dreamlily    & Plain     & 1 &      2 agnomas & COO & ---  & ERLW  244 \\
            Explosive Seed          & Plain     & 3 &      5 agnomas & COO & ---  & EGW   225 \\
            Hunk of Cheese          & Plain     & 1 &     80 fobs    & COO & ---  & PHB   158 \\
            Menga Leaves            & Plain     & 1 &      2 agnomas & COO & ---  & TOA   205 \\
            Rations (1 meal)        & Plain     & 1 &     50 fobs    & COO & 1 kg & PHB   153 \\
            Sinda Berries           & Plain     & 3 &      5 agnomas & COO & ---  & TOA   205 \\
            Chunk of Meat           & Common    & 1 &     24 agnomas & COO & ---  & PHB   158 \\
            Olisuba Leaf            & Common    & 3 &     50 agnomas & COO & ---  & EGW    70 \\
            Ryath Root              & Common    & 3 &     50 agnomas & COO & ---  & TOA   205 \\
            Wildroot                & Common    & 1 &     25 agnomas & COO & ---  & TOA   205 \\
            Zabou                   & Common    & 1 &     10 agnomas & COO & ---  & TOA   205 \\
            Black Sap               & Uncommon  & 4 &    300 agnomas & COO & ---  & EGW   152 \\
            Blight Ichor            & Uncommon  & 2 &    200 agnomas & COO & ---  & EGW   152 \\
            Dust of Deliciousness   & Uncommon  & 4 &    300 agnomas & COO & ---  & EGW   267 \\
            Dust of Sneezing        & Uncommon  & 8 &    500 agnomas & COO & ---  & DMG   166 \\
            Guardian Broth          & Uncommon  & 8 &    500 agnomas & COO & ---  & TCE   128 \\
            Pineapple Honeyed Ham   & Uncommon  & 3 &    250 agnomas & COO & ---  & ---       \\
            Soothsalts              & Uncommon  & 1 &    150 agnomas & COO & ---  & EGW   152 \\
            Thassa's Olive Salad    & Uncommon  & 8 &    500 agnomas & COO & ---  & ---       \\
            Vedbread                & Uncommon  & 8 &    500 agnomas & COO & ---  & ---       \\
            High Harvest Puree      & Rare      & 8 &  5,000 agnomas & COO & ---  & ---       \\
            Krig's Stuffed Potatoes & Rare      & 8 &  5,000 agnomas & COO & ---  & ---       \\
            Rose-scented Truffles   & Rare      & 8 &  5,000 agnomas & COO & ---  & ---       \\
            Wyvern Salmon           & Rare      & 8 &  5,000 agnomas & COO & ---  & ---       \\
            Drakefire Chili         & Very Rare & 8 & 50,000 agnomas & COO & ---  & ---       \\
            Living Armor            & Very Rare & 8 & 50,000 agnomas & COO & ---  & ERLW  278
        \end{DndTable}
    \end{table*}

    \paragraph{Bead of Nourishment}
        This spongy, flavorless, gelatinous bead dissolves on your tongue and provides as much nourishment as 1 day of rations.
    \paragraph{Bead of Refreshment}
        This spongy, flavorless, gelatinous bead dissolves in liquid, transforming up to a pint of the liquid into fresh, cold drinking water.
        The bead has no effect on harmful substances such as poison.
    \paragraph{Black Sap}
        This tarry substance harvested from the dark boughs of the death's head willow is a powerful intoxicant.
        It can be smoked as a concentrate or injected directly into the bloodstream.
        A creature subjected to a dose of black sap cannot be charmed or frightened for 8 hours.
        For each dose of black sap consumed, a creature must succeed on a DC 15 Constitution saving throw or become poisoned for 24 hours---an effect that is cumulative with multiple doses.
    \paragraph{Blight Ichor}
        This bitter chartreuse concoction is distilled from a fungus native to the Blightshore badlands.
        The sickly green liqueur harbors potent psychedelic properties.
        Provided it is neither a construct nor undead, a creature subjected to a dose of blight ichor gains advantage on Intelligence and Wisdom checks, as well as vulnerability to psychic damage, for 1 hour.
        For each dose of blight ichor consumed, the creature must succeed on a DC 15 Constitution saving throw or become poisoned for 24 hours.
        A fiend subjected to a dose of blight ichor gains advantage on all Dexterity checks and is immune to the frightened condition for 1 hour.
    \paragraph{Dancing Monkey Fruit}
        After being draped in honey and fermented for 10 days, this apple produces enough juice to fill a vial.
        Any creature that eats a dancing monkey fruit or drinks its juice must succeed on a DC 14 Constitution saving throw or begin a comic dance that lasts for 1 minute.
        Humanoids that can't be poisoned are immune to this effect.

        The dancer must use one action on each of its turns to dance without leaving its space.
        In addition, it cannot move, has disadvantage on attack rolls and Dexterity saving throws, and other creatures have advantage on attack rolls against it.
        Each time it takes damage, the dancer can repeat the saving throw, ending the effect on itself on a success.
        When the dancing effect ends, the humanoid suffers the poisoned condition for 1 hour.
    \paragraph{Drakefire Chili}
        This chili is made of drake meat mixed with onion, tomatoes, and a generous portion of pepper.
        After eating this meal, a creature gains the ability to discharge jets of energy from its mouth for 8 hours.
        A creature can fire this jet once every 5 minutes.
        The creature firing the jet chooses the effect from the following options:
        \begin{itemize}
            \item \textbf{Acid Jet}.
                The creature discharges acid in a line 60 meters long and 1 meter wide.
                Each creature in that line must make a DC 15 Dexterity saving throw, taking 4d10 acid damage on a failed save, or half as much damage on a successful one.
                In addition, a creature that fails its saving throw takes 2d10 acid damage at the start of each of its turns; a creature can end this damage by using two actions to wash off the acid with a pint or more of water.
            \item \textbf{Fire Jet}.
                The creature discharges fire in a line 60 meters long and 1 meter wide.
                Each creature in the area must make a DC 15 Dexterity saving throw, taking 6d10 fire damage on a failed save, or half as much damage on a successful one.
                The fire ignites any flammable objects in the area that aren't being worn or carried.
            \item \textbf{Frost Shot}.
                The creature discharges a ball of frost to a point it can see within 240 meters.
                The ball then expands to form a 6-meter-radius sphere centered on that point.
                Each creature in that area must make a DC 15 Constitution saving throw.
                On a failed save, a creature takes 4d10 cold damage, and its speed is reduced by 2 meters for 1 minute.
                On a successful save, the creature takes half as much damage, and its speed isn't reduced.
                A creature whose speed is reduced by this effect can repeat the save at the end of each of its turns, ending the effect on itself on a success.
            \item \textbf{Lightning Shot}.
                The creature shoots a ball of lightning to a point it can see within 240 meters.
                The lightning then expands to form a 4-meter-radius sphere centered on that point.
                Each creature in that area must make a DC 15 Dexterity saving throw, taking 6d10 lightning damage on a failed save, or half as much damage on a successful one.
                Creatures wearing metal armor have disadvantage on the save.
            \item \textbf{Poison Spray}.
                The creature expels poison gas in a 12-meter cone.
                Each creature in that area must make a DC 15 Constitution saving throw.
                On a failed save, the creature takes 4d10 poison damage and is poisoned for 1 minute.
                On a successful save, the creature takes half as much damage and isn't poisoned.
                A creature poisoned in this way can repeat the saving throw at the end of each of its turns, ending the effect on itself on a success.
        \end{itemize}
    \paragraph{Dust of Deliciousness}
        This reddish brown dust can be sprinkled over any edible substance to add an earthly flavor to it.
        The dust also dulls the eater's senses: anyone eating food treated with this dust has disadvantage on Wisdom ability checks and Wisdom saving throws for 1 hour.
    \paragraph{Dust of Sneezing}
        Found in a small container, this powder resembles very fine sand and gives a faint smell of turmeric.
        You can either add this dust to up to 8 meals or use an action to throw a handful of it into the air, creating a 12-meter cloud of dust.
        Any creature that consumes or breathes the dust must succeed on a DC 15 Constitution saving throw or become unable to breathe while sneezing uncontrollably.
        A creature affected in this way is incapacitated and suffocating.
        As long as it is conscious, a creature can repeat the saving throw at the end of each of its turns, ending the effect on it on a success.
    \paragraph{Essence of Dreamlily}
        Dreamlily is a tasteless psychoactive liquid that can be added to foods to have them grant additional effects.
        Though dreamlily isn't illegal if used for medicinal purposes, it's heavily taxed, and thus most dreamlily is smuggled in and sold on the black market.
        Consuming dreamlily causes disorienting euphoria and brings about remarkable resistance to pain.

        A creature under the effects of dreamlily is poisoned for 1 hour.
        While poisoned in this way, the creature is immune to fear, and the first time it drops to 0 hit points without being killed outright, it drops to 1 hit point instead.
    \paragraph{Explosive Seed}
        This acorn-sized hazelnut contains a small amount of blasting powder.
        An explosive seed can be thrown up to 6 meters as an action, detonating on impact.
        Each creature within a meter of the exploding seed must make a DC 10 Dexterity saving throw, taking 1d8 bludgeoning damage on a failed save, or half as much damage on a successful one.
    \paragraph{Guardian Broth}
        Accompanied with melted cheeses, this chunky tomato broth is the perfect breakfast to prepare oneself for a day of hardship.
        For 8 hours after eating the broth, you can use your reaction to turn a critical hit against you into a normal hit instead.
		The effect of the meal fades when you use this ability.
    \paragraph{High Harvest Puree}
        Simply the smell of this squash and garlic puree is enough to brighten any adventurer's day.
        For 8 hours after eating this meal, you are inmune to harmful gases such as those created by the \textbf{Cloudkill} (page \pageref{spell::cloudkill}) spell, inhaled poisons, etc.
        In addition, you can exhale a gust of wind for the duration, as if you had cast the \textbf{Gust} spell (see page \pageref{spell::gust}).
    \paragraph{Krig's Stuffed Potatoes}
        These potatoes are filled with a combination of meat, herbs, and spices, to be then covered in a thick fabric and cooked in an oven.
        Upon eating it, a creature gains an effect of their choice for 8 hours.
        The available effects are:
        \begin{itemize}
            \item \textbf{Mark of Courage}.
                The creature is immune to the frightened condition.
            \item \textbf{Mark of the Sentinel}.
                The creature can see invisible creatures while they are within 4 meters of it and within its line of sight.
            \item \textbf{Mark of the Shield}.
                As an action, the creature can activate the effects of the meal.
                For a minute, it gains a +1 bonus to its AC.
            \item \textbf{Mark of the Sword}.
                As an action, the creature can activate the effects of the meal.
                For a minute, it gains a +1 bonus to its attack rolls.
        \end{itemize}
    \paragraph{Living Armor $\odot$} \label{item::livingarmor}
        This hideous ooze is formed from black chitin, beneath which veins pulse and red sinews glisten.
        To attune to this item, you must consume it as a meal, which is enough food for the day.
        After eaten, tendrils burrow and reach from inside you, merging with your armor.
        While you remain attuned to this item, you have a +1 bonus to your AC, and you have resistance to necrotic, poison, and psychic damage.

        \textbf{Symbiotic Nature}.
        The armor can't be removed from you while you're attuned to it.
        The only way to end your attunement to the armor is by drinking an oozekiller vial (see page \pageref{item::oozekiller}).
        The armor requires fresh blood be fed to it.
        Immediately after you finish any long rest, you must either feed half of your remaining Hit Dice to the armor (round up) or take 1 level of exhaustion.
    \paragraph{Menga Leaves}
        The dried leaves of a menga bush can be ground, dissolved in a liquid, heated, and ingested.
        A creature that ingests 1 ounce of menga leaves in this fashion regains 1 hit point.
        A creature that ingests more than 5 ounces of menga leaves in a 24-hour period gains no additional benefit and must succeed on a DC 11 Constitution saving throw or fall unconscious for 1 hour.
        The unconscious creature awakens if it takes at least 5 damage on one turn.
        A healthy menga bush usually has 2d6 ounces of leaves.
        Once picked, the leaves require 1 day to dry out before they can confer any benefit.
    \paragraph{Olisuba Leaf}
        These dried leaves of the Olisuba tree, when steeped to make a tea, can help a body recover from strenuous activity.
        If you drink a dose of Olisuba tea during a short rest, your exhaustion level is reduced by 2 instead of 1 at the end of that long rest.
    \paragraph{Pineapple Honeyed Ham}
        This cooked ham can be eaten as a side dish or over bread and pineaple gravy as a full meal.
        After eating this meal, a creature gains a +1 bonus to saving throws for 8 hours.
    \paragraph{Rose-scented Truffles}
        These truffles are mixed with equal portions of honey and cocoa.
        Upon serving, they are bathed in an odorous rose syrup, giving them their characteristic fragance.
        The syrup comes in one of five colors, decided by the chef upon cooking.
        For 8 hours after eating a truffle, a creature gains resistance to damage of the type associated with the syrup.
        If the creature would take more than 10 damage of this type from a single source (after applying the resistance), the effect is lost.

        During its turn after losing the effect, the creature can use an action to blow out a rose-smelling breath.
        The creature produces a 4-meter cone of acid, lightning, poison, fire, or cold, as dictated by the syrup's damage type.
        Each creature in the cone must make a DC 15 Constitution saving throw, taking 3d10 damage of the appropiate type on a failed save, or half as much on a successful one.

        \begin{DndTable}[width=\linewidth, header=Chromatic Roses]{ll}
            \textbf{Item} & \textbf{Damage Type} \\
            Black Rose    & Acid                 \\
            Blue Rose     & Lightning            \\
            Green Rose    & Poison               \\
            Red Rose      & Fire                 \\
            White Rose    & Cold
        \end{DndTable}
    \paragraph{Ryath Root}
        Any creature that ingests a dried ryath root gains 2d4 temporary hit points.
        A creature that consumes more than one ryath root in a 24-hour period must succeed on a DC 13 Constitution saving throw or suffer the poisoned condition for 1 hour.
    \paragraph{Sinda Berries}
		These berries are dark brown and bitter.
        A full-grown sinda berry bush has 4d6 berries growing on it.
        A bush plucked of all its berries grows new berries in 1d4 months.
        Picked berries lose their freshness and efficacy after 24 hours.
		Fresh sinda berries can be eaten raw or crushed and added to a drink to dull the bitterness.
        A creature that consumes at least ten fresh sinda berries gains advantage on saving throws against disease and poison for the next 24 hours.
    \paragraph{Soothsalts}
		Soothsalts are derived from a naturally occurring crystalline substance discovered throughout the wilds of the Katajthon canyon.
        The crimson crystals have been mined from cavernous veins and found within smaller geode formations.
        Soothsalts are consumed orally in lozenge-sized doses, and frequent users can be identified by the telltale crimson stain around their mouths.
        A creature subjected to a dose of soothsalts gains advantage on all Intelligence checks for 8 hours.
		For each dose of soothsalts consumed, the creature must succeed on a DC 15 Constitution saving throw or gain one level of exhaustion---an effect which is cumulative with multiple doses.
    \paragraph{Thassa's Olive Salad}
        This simple mediterranean salad combines the flavors of sea greens, pomegranate molasses, green olives, and grape tomatoes.
        After mixing, it is infused with a blessing from Thassa via its traditional black pepper and sea salt seasoning.
        After eating it, a creature gains the ability to breathe normally underwater for 8 hours.
    \paragraph{Vedbread}
        The next time you see a creature within 10 minutes after eating this mushroom bun, you become charmed by that creature for 1 hour.
    \paragraph{Wildroot}
        Introducing the juice of a wildroot into a poisoned creature's bloodstream (for example, by rubbing it on an open wound) rids the creature of the poisoned condition.
        Once used in this way, a wildroot loses this property.
    \paragraph{Wyvern Salmon}
        Cooked with shallot, thyme, and red wine, this salmon is the stuff of legends.
        After eating this meal, a creature is cured of any poison affecting it, becomes inmune to poison, and makes all Wisdom saving throws with advantage.
        In addition, its hit point maximum also increases by 2d10, and it gains the same number of hit points.
        These benefits last for 8 hours.
    \paragraph{Zabou}
        Zabou mushrooms feed on offal and the rotting wood of dead trees.
        If handled carefully, a zabou can be picked or uprooted without causing it to release its spores.
        If crushed or struck, a zabou releases its spores in a 2-meter-radius sphere.
        A zabou can also be hurled up to 6 meter away or dropped like a grenade, releasing its cloud of spores on impact.
        Any creature in that area must succeed on a DC 10 Constitution saving throw or be poisoned for 1 minute.
        The poisoned creature's skin itches for the duration.
        The creature can repeat the saving throw at the end of each of its turns, ending the effect on itself on a success."

\newpage~\newpage

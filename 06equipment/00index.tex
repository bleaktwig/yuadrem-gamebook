% !TEX root = ../main.tex
\chapter{Equipment} \label{ch::equipment}
\section*{Currency} \label{sec::currency}
    \subparagraph{Fob}
        Copper coin of little value.
        Translates to ``common''.
    \subparagraph{Agnoma}
        Nickel coin worth 100 fobs.
        Named after Agnomakhos, the original et ruler of Mephetis.

    In addition to coinage, trade usually occurs in the exchange of trade goods.
    A comprehensive list of trade goods is given in the Trade Goods table below.

    \begin{table*}[b]%
        \begin{DndTable}[width=\linewidth, header=Trade Goods]{Xcccc}
            \textbf{Item} & \textbf{Rarity} & \textbf{Total Cost} & \textbf{Weight} & \textbf{Source} \\
            Chicken                  & Mundane  &     2 fobs    & ---  & PHB 157 \\
            Flour                    & Mundane  &     4 fobs    & 1 kg & PHB 157 \\
            Canvas (1 sq. mt.)       & Mundane  &    10 fobs    & ---  & PHB 157 \\
            Iron                     & Mundane  &    20 fobs    & 1 kg & PHB 157 \\
            Cotton Cloth (1 sq. mt.) & Mundane  &    50 fobs    & ---  & PHB 157 \\
            Salt                     & Mundane  &    10 fobs    & 1 kg & PHB 157 \\
            Wheat                    & Mundane  &     2 fobs    & 1 kg & PHB 157 \\
            Cooper                   & Plain    &     1 agnoma  & 1 kg & PHB 157 \\
            Goat                     & Plain    &     1 agnoma  & ---  & PHB 157 \\
            Ginger                   & Plain    &     2 agnomas & 1 kg & PHB 157 \\
            Cinnamon                 & Plain    &     4 agnomas & 1 kg & PHB 157 \\
            Linen (1 sq. mt.)        & Plain    &     5 agnomas & ---  & PHB 157 \\
            Cloves                   & Plain    &     6 agnomas & 1 kg & PHB 157 \\
            Pepper                   & Plain    &     4 agnomas & 1 kg & PHB 157 \\
            Pig                      & Plain    &     3 agnomas & ---  & PHB 157 \\
            Sheep                    & Plain    &     2 agnomas & ---  & PHB 157 \\
            Cow                      & Common   &    10 agnomas & ---  & PHB 157 \\
            Ox                       & Common   &    15 agnomas & ---  & PHB 157 \\
            Saffron                  & Common   &    30 agnomas & 1 kg & PHB 157 \\
            Silk (1 sq. mt.)         & Common   &    10 agnomas & ---  & PHB 157 \\
            Silver                   & Common   &    10 agnomas & 1 kg & PHB 157 \\
            Gold                     & Uncommon &   100 agnomas & 1 kg & PHB 157 \\
            Platinum                 & Rare     & 1,000 agnomas & 1 kg & PHB 157
        \end{DndTable}
    \end{table*}

\section*{Materials and Labor} \label{sec::materialsandlabor}
    Any item in Yuadrem that can be bought can be crafted as well by a suficiently skilled artisan.
    Just like items, materials are separated by rarity, with increased rarity having increased cost.
    In addition, the labor required to craft rarer items is naturally harder, and thus is more expensive.
    This relation is shown in the Materials and Labor Costs table.

    Different items require a different number of materials.
    In addition to their final cost, items listed in the following sections also show the required number of materials needed to craft them, and the relevant set of artisan's tools needed to craft them.

    \begin{DndTable}[width=\linewidth, header=Materials and Labor Costs]{llll}
        \textbf{Rarity} & \textbf{Material Price} & \textbf{Labor Price} \\
        Mundane         &      5 fobs             &     10 fobs    \\
        Plain           &      1 agnoma           &      2 agnomas \\
        Common          &     10 agnomas          &     20 agnomas \\
        Uncommon        &     50 agnomas          &    100 agnomas \\
        Rare            &    500 agnomas          &  1,000 agnomas \\
        Very Rare       &  5,000 agnomas          & 10,000 agnomas \\
        Legendary       & 25,000 agnomas          & 50,000 agnomas
    \end{DndTable}

    % \subsection*{Foraging} \label{ssec::foraging}
    % NOTE. I could write a bit about foraging and obtaining materials.

\section*{Items} \label{sec::items}
    Items marked with the $\odot$ symbol require attunement to be used.

    \newpage

    % !TEX root = ../main.tex
\section{Adventuring Gear} \label{sec::adventuringgear}
    Adventuring gear includes various miscellaneous items commonly carried and used by commoners and adventurers.
    % NOTE. Maybe separate containers from rest of gear inside the table for simplicity.

    \begin{table*}[b]%
        \begin{DndTable}[width=\linewidth, header=Adventuring Gear]{Xcccccc}
            \textbf{Item} & \textbf{Rarity} & \textbf{Mats.} & \textbf{Total Cost} & \textbf{Tools} & \textbf{Weight} & \textbf{Source} \\
            Basket                & Mundane  &  6  &    40 fobs    & WEA &  1 kg   & PHB 153 \\
            Blanket               & Mundane  &  8  &    50 fobs    & WEA &  1.5 kg & PHB 150 \\
            Bucket                & Mundane  &  1  &     5 fobs    & CAR &  1 kg   & PHB 153 \\
            Candle                & Mundane  & --- &     1 fob     & --- & ---     & PHB 151 \\
            Chalk                 & Mundane  & --- &     1 fob     & --- & ---     & PHB 150 \\
            Ink Pen               & Mundane  &  1  &     2 fobs    & TIN & ---     & PHB 150 \\
            Iron Spike            & Mundane  &  1  &    10 fobs    & SMI &  0.2 kg & PHB 150 \\
            Jug                   & Mundane  &  2  &    20 fobs    & GLA &  2 kg   & PHB 153 \\
            Ladder                & Mundane  &  1  &    10 fobs    & CAR & 12.5 kg & PHB 150 \\
            Lamp                  & Mundane  &  8  &    50 fobs    & TIN &  0.5 kg & PHB 152 \\
            Mess Kit              & Mundane  &  2  &    20 fobs    & SMI &  0.5 kg & PHB 152 \\
            Paper (one sheet)     & Mundane  & --- &    20 fobs    & --- & ---     & PHB 150 \\
            Parchment (one sheet) & Mundane  & --- &    10 fobs    & --- & ---     & PHB 150 \\
            Pitcher               & Mundane  &  1  &     2 fobs    & WOO &  2 kg   & PHB 153 \\
            Piton                 & Mundane  &  1  &     5 fobs    & SMI & ---     & PHB 150 \\
            Pole (3 meters)       & Mundane  &  1  &     5 fobs    & CAR &  3.5 kg & PHB 150 \\
            Pouch                 & Mundane  &  8  &    50 fobs    & LEA &  0.5 kg & PHB 153 \\
            Sack                  & Mundane  &  1  &     1 fob     & WEA & ---     & PHB 153 \\
            Sealing Wax           & Mundane  & --- &    50 fobs    & --- & ---     & PHB 150 \\
            Signal Whistle        & Mundane  &  1  &     5 fobs    & SMI & ---     & PHB 150 \\
            Soap                  & Mundane  & --- &     2 fobs    & --- & ---     & PHB 150 \\
            Tankard               & Mundane  &  1  &    15 fobs    & CAR &  0.5 kg & PHB 153 \\
            Tinderbox             & Mundane  &  8  &    50 fobs    & MAS &  0.5 kg & PHB 153 \\
            Torch                 & Mundane  &  1  &     1 fob     & CAR &  0.5 kg & PHB 153 \\
            Waterskin             & Mundane  &  2  &    20 fobs    & LEA &  2.5 kg & PHB 153 \\
            Whetstone             & Mundane  &  1  &     1 fob     & MAS &  0.5 kg & PHB 150 \\
            Abacus                & Plain    &  1  &     2 agnomas & CAR &  1 kg   & PHB 150 \\
            Backpack              & Plain    &  1  &     2 agnomas & LEA &  2.5 kg & PHB 153 \\
            Ball Bearings (1,000) & Plain    &  1  &     1 agnomas & SMI &  1 kg   & PHB 151 \\
            Barrel                & Plain    &  3  &     5 agnomas & CAR & 35 kg   & PHB 153 \\
            Bedroll               & Plain    &  1  &     1 agnoma  & WEA &  3.5 kg & PHB 150 \\
            Bell                  & Plain    &  1  &     1 agnoma  & SMI & ---     & PHB 150 \\
            Block and Tackle      & Plain    &  1  &     1 agnoma  & TIN &  2.5 kg & PHB 151 \\
            Bottle                & Plain    &  1  &     3 agnomas & GLA &  1 kg   & PHB 153 \\
            Bullseye Lantern      & Plain    &  8  &    10 agnomas & TIN &  1 kg   & PHB 152 \\
            Caltrops (20)         & Plain    &  1  &     1 agnoma  & SMI &  1 kg   & PHB 151 \\
            Chain (2 mt)          & Plain    &  3  &     5 agnomas & SMI &  5 kg   & PHB 151 \\
            Chest                 & Plain    &  3  &     5 agnomas & CAR & 12.5 kg & PHB 153 \\
            Crowbar               & Plain    &  1  &     2 agnomas & SMI &  2.5 kg & PHB 151
        \end{DndTable}
    \end{table*}
    \begin{table*}[b]%
        \begin{DndTable}[width=\linewidth, header=Adventuring Gear (cont.)]{Xcccccc}
            Cup                   & Plain    &  1  &     3 agnomas & GLA & ---     & --- \\
            Fishing Tackle        & Plain    &  1  &     1 agnoma  & TIN &  2 kg   & PHB 151 \\
            Full Keg              & Plain    &  2  &     4 agnomas & CAR & 15 kg   & --- \\
            Grappling Hook        & Plain    &  1  &     2 agnomas & SMI &  2 kg   & PHB 150 \\
            Hammer                & Plain    &  1  &     1 agnoma  & SMI &  1.5 kg & PHB 150 \\
            Hempen Rope           & Plain    & --- &     1 agnoma  & --- &  5 kg   & PHB 153 \\
            Hooded Lantern        & Plain    &  3  &     5 agnomas & TIN &  1 kg   & PHB 152 \\
            Hunting Trap          & Plain    &  3  &     5 agnomas & TIN & 12.5 kg & PHB 152 \\
            Ink (0.025 lt.)       & Plain    & --- &    10 agnomas & --- & ---     & PHB 150 \\
            Iron Pot              & Plain    &  1  &     2 agnomas & SMI &  5 kg   & PHB 153 \\
            Lock                  & Plain    &  8  &    10 agnomas & TIN &  0.5 kg & PHB 152 \\
            Manacles              & Plain    &  1  &     2 agnomas & TIN &  3 kg   & PHB 152 \\
            Map or Scroll Case    & Plain    &  1  &     1 agnoma  & LEA &  0.5 kg & PHB 151 \\
            Merchant's Scale      & Plain    &  3  &     5 agnomas & TIN &  1.5 kg & PHB 153 \\
            Miner's Pick          & Plain    &  1  &     2 agnomas & SMI &  5 kg   & PHB 150 \\
            Portable Ram          & Plain    &  2  &     4 agnomas & CAR & 17.5 kg & PHB 153 \\
            Shovel                & Plain    &  1  &     2 agnomas & SMI &  2.5 kg & PHB 150 \\
            Silk Rope (10 mt.)    & Plain    &  8  &    10 agnomas & WEA &  2.5 kg & PHB 153 \\
            Sledgehammer          & Plain    &  1  &     2 agnomas & SMI &  5 kg   & PHB 150 \\
            Small Keg             & Plain    &  1  &     3 agnomas & CAR &  2 kg   & --- \\
            Mirror                & Plain    &  3  &     5 agnomas & GLA & ---     & PHB 150 \\
            Two-Person Tent       & Plain    &  1  &     2 agnomas & WEA & 10 kg   & PHB 153 \\
            Book                  & Common   &  1  &    25 agnomas & CAL &  2.5 kg & PHB 151 \\
            Hourglass             & Common   &  1  &    25 agnomas & GLA &  0.5 kg & PHB 150 \\
            Magnifying Glass      & Common   &  8  &   100 agnomas & GLA & ---     & PHB 152 \\
            Spyglass              & Uncommon & 16  & 1,000 agnomas & GLA &  0.5 kg & PHB 153
        \end{DndTable}
    \end{table*}

    \paragraph{Barrel}
        A barrel can hold 150 liters of liquid.
    \paragraph{Basket}
        A basket holds 60 liters or 20 kg of gear.
    \paragraph{Block and Tackle}
        A set of pulleys with a cable threaded through them and a hook to attach to objects, a block and tackle allows you to hoist up to four times the weight you can normally lift.
    \paragraph{Book}
        A book might contain poetry, historical accounts, information pertaining to a particular field of lore, diagrams and notes on gat contraptions, or just about anything else that can be represented using text or pictures.
    \paragraph{Bottle}
        A bottle holds 1 liter of liquid.
    \paragraph{Bucket}
        A bucket holds 12 liters.
    \paragraph{Bullseye Lantern}
        A bullseye lantern casts bright light in a 12-meter cone and dim light for an additional 12 meters.
        Once lit, it burns for 6 hours on a flask of oil.
    \paragraph{Caltrops}
        As two action, you can spread a single bag of caltrops to cover a 1-meter-square area.
        Any creature that enters the area must succeed on a DC 15 Dexterity saving throw or stop moving and take 1 piercing damage.
        Until the creature regains at least 1 hit point, its walking speed is reduced by 2 meters.
        A creature moving through the area at half speed doesn't need to make the saving throw.
    \paragraph{Candle}
        For 1 hour, a candle sheds bright light in a 1-meter radius and dim light for an additional meter.
    \paragraph{Chain}
        A chain has 10 hit points.
        It can be burst with a successful DC 20 Strength check.
    \paragraph{Chest}
        A chest holds 350 liters of gear.
    \paragraph{Crowbar}
        Using a crowbar grants advantage to Strength checks where the crowbar's leverage can be applied.
    \paragraph{Fishing Tackle}
        This kit includes a wooden rod, silken line, corkwood bobbers, steel hooks, lead sinkers, velvet lures, and narrow netting.
    \paragraph{Full Keg}
        A full keg can hold 60 liters of liquid.
    \paragraph{Hempen Rope}
        Rope, whether made of hemp or silk, has 2 hit points and can be burst with a DC 17 Strength check.
    \paragraph{Hooded Lantern}
        A hooded lantern casts bright light in a 6-meter radius and dim light for an additional 6 meters.
        Once lit, it burns for 6 hours on a flask of oil.
        Using two actions, you can lower the hood, reducing the light to dim light in a 1-meter radius.
    \paragraph{Hunting Trap}
        When you use two actions to set it, this trap forms a saw-toothed steel ring that snaps shut when a creature steps on a pressure plate in the center.
        The trap is affixed by a heavy chain to an immobile object, such as a tree or a spike driven into the ground.
        A creature that steps on the plate must succeed on a DC 13 Dexterity saving throw or take 1d4 piercing damage and stop moving.
        Thereafter, until the creature breaks free of the trap, its movement is limited by the length of the chain (typically less than a meter long).
        A creature can use two actions to make a DC 13 Strength check, freeing itself or another creature within its reach on a success.
        Each failed check deals 1 piercing damage to the trapped creature.
    \paragraph{Iron Pot}
        An iron pot holds 4 liters of liquid.
    \paragraph{Jug}
        A jug holds 1 liter of liquid.
    \paragraph{Lamp}
        A lamp casts bright light in a 3-meter radius and dim light for an additional 6 meters.
        Once lit, it burns for 6 hours on a flask of oil.
    \paragraph{Lock}
        A key is provided with the lock.
        Without the key, a creature proficient with thieves' tools can pick this lock with a successful DC 15 Dexterity check.
        Your DM may decide that better locks are available for higher prices.
    \paragraph{Magnifying Glass}
        This lens allows a closer look at small objects.
        It is also useful as a substitute for flint and steel when starting fires.
        Lighting a fire with a magnifying glass requires light as bright as sunlight to focus, tinder to ignite, and about 5 minutes for the fire to ignite.
        A magnifying glass grants advantage on any ability check made to appraise or inspect an item that is small or highly detailed.
    \paragraph{Manacles}
        These metal restraints can bind a Small or Medium creature.
        Escaping the manacles requires a successful DC 20 Dexterity check.
        Breaking them requires a successful DC 20 Strength check.
        Each set of manacles comes with one key.
        Without the key, a creature proficient with thieves' tools can pick the manacles' lock with a successful DC 15 Dexterity check.
        Manacles have 15 hit points.
    \paragraph{Map or Scroll Case}
        This cylindrical leather case can hold up to ten rolled-up sheets of paper or five rolled-up sheets of parchment.
    \paragraph{Mess Kit}
        This tin box contains a cup and simple cutlery.
        The box clamps together, and one side can be used as a cooking pan and the other as a plate or shallow bowl.
    \paragraph{Portable Ram}
        You can use a portable ram to break down doors.
        When doing so, you gain a +4 bonus on the Strength check.
        One other character can help you use the ram, giving you advantage on this check.
    \paragraph{Pouch}
        A leather pouch can hold up to 20 sling bullets or 50 blowgun needles, among other things.
        A pouch can hold up to 6 liters or 3 kilograms of gear.
    \paragraph{Sack}
        A sack can hold up to 30 liters or 15 kilograms of gear.
    \paragraph{Small Keg}
        A small, portable keg can hold 4 liters of liquid.
    \paragraph{Spyglass}
        Objects viewed through a spyglass are magnified to up to eight times their size.
    \paragraph{Tankard}
        A tankard holds half a liter of liquid.
    \paragraph{Tinderbox}
        This small container holds flint, fire steel, and tinder (usually dry cloth soaked in light oil) used to kindle a fire.
        Using it to light a torch---or anything else with abundant, exposed fuel---takes two actions.
        Lighting any other fire takes 1 minute.
    \paragraph{Torch}
        A torch burns for 1 hour, providing bright light in a 4-meter radius and dim light for an additional 4 meters.
        If you make a melee attack with a burning torch and hit, it deals 1 fire damage.
    \paragraph{Waterskin}
        A waterskin can hold up to 2 liters of liquid.
\newpage~\newpage

    % !TEX root = ../main.tex
\section{Ammunition} \label{sec::ammunition}
\begin{table*}[b]%
    \begin{DndTable}[width=\linewidth, header=Ammunition]{Xcccccc}
        \textbf{Item} & \textbf{Rarity} & \textbf{Mats.} & \textbf{Cost} & \textbf{Tools} & \textbf{Weight} & \textbf{Source} \\
        Arrow                       & Mundane   &  1 &     15 fobs    & WOO    & 25 gr  & PHB 150 \\
        Arrows (20)                 & Mundane   & 18 &      1 agnoma  & WOO    & 0.5 kg & PHB 150 \\
        Blowgun Needle              & Mundane   &  1 &     15 fobs    & WOO    & 25 gr  & PHB 150 \\
        Blowgun Needles (20)        & Mundane   & 18 &      1 agnoma  & WOO    & 0.5 kg & PHB 150 \\
        Crossbow Bolt               & Mundane   &  1 &     15 fobs    & WOO    & 50 gr  & PHB 150 \\
        Crossbow Bolts (20)         & Mundane   & 18 &      1 agnoma  & WOO    & 1 kg   & PHB 150 \\
        Sling Bullet                & Mundane   &  1 &     15 fobs    & WOO    & 50 gr  & PHB 150 \\
        Sling Bullets (20)          & Mundane   & 18 &      1 agnoma  & WOO    & 1 kg   & PHB 150 \\
        Crossbow Bolt Case          & Plain     &  1 &      3 agnomas & CAR    & 0.5 kg & PHB 151 \\
        Quiver                      & Plain     &  1 &      3 agnomas & LEA    & 0.5 kg & PHB 153 \\
        Bullet                      & Common    &  1 &     30 agnomas & TIN    & 100 g  & DMG 268 \\
        Bullets (10)                & Common    &  8 &    100 agnomas & TIN    & 1 kg   & DMG 268 \\
        +1 Ammunition (20)          & Uncommon  &  4 &    300 agnomas & $\ast$ & $\ast$ & DMG 150 \\
        Barbed Ammunition (20)      & Uncommon  &  4 &    300 agnomas & $\ast$ & $\ast$ & ---     \\
        Destructive Ammunition (20) & Uncommon  &  4 &    300 agnomas & $\ast$ & $\ast$ & XGE  78 \\
        Nylea's Quiver              & Uncommon  &  8 &    500 agnomas & LEA    & 1 kg   & DMG 189 \\
        Reinforced Ammunition (20)  & Uncommon  &  4 &    300 agnomas & $\ast$ & $\ast$ & XGE 139 \\
        Walloping Ammunition (20)   & Uncommon  &  4 &    300 agnomas & $\ast$ & $\ast$ & XGE 139 \\
        +2 Ammunition (20)          & Rare      &  4 &  3,000 agnomas & $\ast$ & $\ast$ & DMG 150 \\
        Cacophonous Ammunition      & Rare      &  1 &  1,500 agnomas & $\ast$ & $\ast$ & ---     \\
        Fulgurating Ammunition (5)  & Rare      &  1 &  1,500 agnomas & $\ast$ & $\ast$ & MOT 198 \\
        Stunning Ammunition (5)     & Rare      &  1 &  1,500 agnomas & $\ast$ & $\ast$ & MOT 198 \\
        +3 Ammunition (20)          & Very Rare &  4 & 30,000 agnomas & $\ast$ & $\ast$ & DMG 150 \\
        Slaying Ammunition (4)      & Very Rare &  4 & 30,000 agnomas & $\ast$ & $\ast$ & DMG 152
    \end{DndTable}
\end{table*}

$\ast$ \textit{Required tool and weight depends on the type of ammunition.}

\paragraph{+1/+2/+3 Ammunition}
    You have a +1/+2/+3 bonus to attack and damage rolls made with this piece of ammunition.
    These pieces of ammunition break normally.
\paragraph{Barbed Ammunition}
    A creature attacked by a piece of barbed ammunition takes 1d4 necrotic damage at the beginning of its next turn.
    This effect doesn't stack.
\paragraph{Cacophonous Ammunition}
    Upon landing, a wave of thunderous force sweeps out from this projectile.
    Each creature in a 3-meters sphere originating from its landing location must make a DC 13 Constitution saving throw, taking 2d8 thunder damage and being pushed 2 meters away on a failed save.
    Unsecured objects that are completely within the area of effect are automatically pushed 2 meters away from the landing location.

    In addition, if you hit a creature with this projectile it must succeed on a DC 13 Constitution saving throw, taking 3d10 lightning damage on a failed save, or half as much damage on a successful save.

    This piece of ammunition breaks upon landing.
\paragraph{Crossbow Bolt Case}
    This wooden case can hold up to twenty crossbow bolts.
\paragraph{Destructive Ammunition}
    Coated with heavy steel, this piece of ammunition is unusually effective when used to break objects.
    Whenever a piece of destructive ammunition hits an object, the hit is a critical hit.
\paragraph{Fulgurating Ammunition}
    On a hit, this projectile deals an extra 2d8 lightning damage to its target.
    All other creatures within 2 meters of the target must each succeed on a DC 15 Constitution saving throw or take 1d8 thunder damage.
\paragraph{Quiver}
    A quiver can hold up to 20 arrows.
\paragraph{Nylea's Quiver}
    This quiver seems to be larger on the inside than what it looks from the outside.
    Each of the quiver's three compartments allows the quiver to hold numerous items while never weighing more than 1 kg.
    The shortest compartment can hold up to sixty arrows, bolts, or similar objects.
    The midsize compartment holds up to eighteen javelins or similar objects.
    The longest compartment holds up to six long objects, such as bows, quarterstaffs, or spears.

    You can draw any item the quiver contains as if doing so from a regular quiver or scabbard.
\paragraph{Reinforced Ammunition}
    This piece of ammunition doesn't break when used in combat.
    It can however be broken by succeeding on a DC 15 Athletics check.
\paragraph{Slaying Ammunition}
    A slaying ammunition is a specially designed weapon meant to slay a particular kind of creature.
    If a creature belonging to the type, race, or group associated with a slaying ammunition takes damage from it, the creature must make a DC 17 Constitution saving throw, taking an extra 6d10 piercing damage on a failed save, or half as much extra damage on a successful one.

    Once an arrow of slaying deals its extra damage to a creature, it breaks.
\paragraph{Stunning Ammunition}
    On a hit, this projectile deals an extra 1d10 force damage, and the target must roll a DC 13 Constitution saving throw, becoming stunned until the end of your next turn on a failed save.
\paragraph{Walloping Ammunition}
    This ammunition packs a wallop.
    A creature hit by the ammunition must succeed on a DC 10 Strength saving throw or be knocked prone.
\newpage~\newpage

    % !TEX root = ../main.tex
\section{Armor and Shields} \label{sec::armorandshields}
\begin{table*}[b]%
    \begin{DndTable}[width=\linewidth, header=Armor]{Xccccccccc}
        \textbf{Armor} & \textbf{Rarity} & \textbf{AC} & \textbf{Properties} & \textbf{Mats.} & \textbf{Cost} & \textbf{Tools} & \textbf{Weight} & \textbf{Source} \\
        \multicolumn{9}{l}{\hspace{0.5cm}\textit{Light Armor}} \\
        Padded Armor       & Plain     & 11 & Cloth, Noisy     &  3 &       5 agnomas & LEA       &  4 kg   & PHB 144   \\
        Leather Armor      & Plain     & 11 & Leather          &  8 &      10 agnomas & LEA       &  5 kg   & PHB 144   \\
        Studded Armor      & Common    & 12 & Leather          &  3 &      50 agnomas & LEA       &  6.5 kg & PHB 144   \\
        Glamoured Armor    & Rare      & 13 & Leather          &  8 &   5,000 agnomas & LEA + WEA &  6.5 kg & DMG 172   \\
        Hunter's Coat      & Very Rare & 12 & Cloth, Leather   &  8 &  50,000 agnomas & LEA + WEA &  5 kg   & EGW 267   \\
        \multicolumn{9}{l}{\hspace{0.5cm}\textit{Medium Armor}} \\
        Hide Armor         & Plain     & 12 & Leather          &  8 &      10 agnomas & LEA       & 6 kg    & PHB 144   \\
        Chain Shirt        & Common    & 13 & Chain            &  4 &      50 agnomas & SMI + WEA & 10 kg   & PHB 144   \\
        Scale Mail         & Common    & 14 & Composite, Noisy &  4 &      50 agnomas & LEA + SMI & 22.5 kg & PHB 144   \\
        Breastplate        & Uncommon  & 14 & Plate            &  6 &     400 agnomas & LEA + SMI & 10 kg   & PHB 145   \\
        Half Plate Armor   & Uncommon  & 15 & Plate, Noisy     & 13 &     750 agnomas & LEA + SMI & 20 kg   & PHB 145   \\
        Laminar Armor      & Uncommon  & 14 & Composite        &  6 &     400 agnomas & CAR + LEA & 30 kg   & ---       \\
        Woven Iron Shirt   & Uncommon  & 13 & Chain            &  8 &     500 agnomas & SMI + WEA & 10 kg   & DMG 182   \\
        Nidhogg Mantle     & Rare      & 15 & Composite, Noisy &  8 &   5,000 agnomas & LEA + WEA & 20 kg   & VGM  81   \\
        Woven Steel Shirt  & Rare      & 14 & Chain            &  8 &   5,000 agnomas & SMI + WEA & 10 kg   & DMG 168   \\
        Wyvern Mail        & Very Rare & 15 & Composite, Noisy &  8 &  50,000 agnomas & LEA + WEA & 22.5 kg & DMG 165   \\
        \multicolumn{9}{l}{\hspace{0.5cm}\textit{Heavy Armor}} \\
        Ring Mail          & Common    & 14 & Chain, Noisy     &  1 &      30 agnomas & LEA + SMI & 20 kg   & PHB 145   \\
        Chain Mail         & Common    & 16 & Chain, Noisy     &  6 &      80 agnomas & SMI + WEA & 27.5 kg & PHB 145   \\
        Splint Armor       & Uncommon  & 17 & Composite, Noisy &  2 &     200 agnomas & LEA + SMI & 30 kg   & PHB 145   \\
        Plate Armor        & Rare      & 18 & Plate, Noisy     &  1 &   1,500 agnomas & LEA + SMI & 32.5 kg & PHB 145   \\
        Demon Armor        & Very Rare & 19 & Plate, Noisy     &  8 &  50,000 agnomas & SMI + WEA & 32.5 kg & DMG 165   \\
        Thulkrakan Plate   & Very Rare & 20 & Plate, Noisy     &  8 &  50,000 agnomas & LEA + SMI & 32.5 kg & DMG 167   \\
        Invulnerable Armor & Legendary & 18 & Plate, Noisy     &  8 & 250,000 agnomas & LEA + SMI & 32.5 kg & DMG 152   \\
        Tizerus' Chain     & Legendary & 19 & Chain, Noisy     &  8 & 250,000 agnomas & SMI + WEA & 27.5 kg & DMG 167   \\
        Obsidian Plate     & Legendary & 20 & Plate, Noisy     &  8 & 250,000 agnomas & LEA + MAS & 32.5 kg & BGDIA 224
    \end{DndTable}
\end{table*}

\subsection*{Armor Properties} \label{ssec::armorproperties}
    \subparagraph{Bulwark}
        The armor covers you so completely that your movement is hindered.
        You do not add your Dexterity modifier to Dexterity saving throws.
    \subparagraph{Chain}
        The armor is so flexible that it bends under strong blows, absorbing some of their strength.
        You gain resistance to the damage made by critical hits from bludgeoning, slashing, an piercing damage.
    \subparagraph{Cloth}
        This armor is light and comfortable.
        You can rest normally while wearing it.
    \subparagraph{Composite}
        The numerous overlapping pieces of this armor protect you from piercing attacks.
        You gain resistance to piercing damage.
    \subparagraph{Leather}
        The thick second skin of the armor disperses blunt force.
        You gain resistance to bludgeoning damage.
    \subparagraph{Noisy}
        The armor is also loud and likely to alert others of your presence, giving you disadvantage on Dexterity (Stealth) checks.
    \subparagraph{Plate}
        The sturdy plate provides no purchase for a cutting edge.
        You gain resistance to slashing damage.
\subsection*{Light Armor} \label{ssec::lightarmor}
    Made from supple and thin materials, light armor favors agile adventurers since it offers some protection without sacrificing mobility.
    If you wear light armor, you add your Dexterity modifier to the base number from your armor type to determine your Armor Class.

    \paragraph{Glamoured Studded Leather}
        This armor has the normal appearance of a set of clothing.
        While crafting it, its maker(s) decide what it looks like, including color, style, and accessories, but the armor retains its normal bulk and weight.
    \paragraph{Hunter's Coat $\odot$}
        This coat has 3 charges.
        When you hit a creature with an attack and that creature doesn't have all its hit points, you can expend 1 charge to deal an extra 1d10 necrotic damage to the target.
        The coat regains 1d3 expended charges daily at dawn.

        The breastplate and shoulder protectors of this armor are made of leather that has been stiffened by being boiled in oil.
        The rest of the armor is made of softer and more flexible materials.
    \paragraph{Leather Armor}
        The breastplate and shoulder protectors of this armor are made of leather that has been stiffened by being boiled in oil.
        The rest of the armor is made of softer and more flexible materials.
    \paragraph{Padded Armor}
        Padded armor consists of quilted layers of cloth and batting.
    \paragraph{Studded Armor}
        Made from tough but flexible leather, studded leather is reinforced with close-set rivets or spikes.
\subsection*{Medium Armor} \label{ssec::mediumarmor}
    Medium armor offers more protection than light armor, but it also impairs movement more.
    If you wear medium armor, you add your Dexterity modifier, to a maximum of +2, to the base number from your armor to determine your Armor Class.

    \paragraph{Breastplate}
        This armor consists of a fitted metal chest piece worn with supple leather.
        Although it leaves the legs and arms relatively unprotected, this armor provides good protection for the wearer's vital organs while leaving the wearer relatively unencumbered.
    \paragraph{Chain Shirt}
        Made of interlocking metal rings, a chain shirt is worn between layers of clothing.
        This armor offers modest protection to the wearer's upper body and allows the sound of the rings rubbing against one another to be muffled by outer layers.
    \paragraph{Half Plate Armor}
        Half plate consists of shaped metal plates that cover most of the wearer's body.
        It does not include leg protection beyond simple greaves that are attached with leather straps.
    \paragraph{Hide Armor}
        This crude armor consists of thick furs and pelts.
    \paragraph{Laminar Armor}
        This armor consists of laminae of small wood slabs woven together using strong leather.
        The strong wood provides a solid defense against piercing attacks, albeit weighing much more than similar armors.
    \paragraph{Nidhogg Mantle}
        Crafted from nidhogg scales, this carapace-like armor encases portions of the wearer's shoulders, neck, and chest.
        A creature wearing this armor can breathe normally in any environment, and has advantage on saving throws against harmful gases (such as those created by a cloudkill spell, a stinking cloud spell, inhaled poisons, and the breath weapons of some creatures).
    \paragraph{Scale Mail}
        This armor consists of a coat, leggings, and a separate skirt of leather covered with overlapping pieces of metal, much like the scales of a fish.
        The suit includes gauntlets.
    \paragraph{Woven Iron Shirt}
        By using thin iron threads, a woven iron shirt provides good protection while remaining light.
        This armor can be worn under normal clothes, becoming effectively invisible.
    \paragraph{Woven Steel Shirt}
        By using thin steel threads, a woven steel shirt provides good protection while remaining light.
        This armor can be worn under normal clothes, becoming effectively invisible.
    \paragraph{Wyvern Mail $\odot$}
        This armor is made from wyvern scales and the carefully skinned hide of a wyvern.
        While wearing this armor, you have resistance to slashing and poison damage.
\subsection*{Heavy Armor} \label{ssec::heavyarmor}
    Of all the armor categories, heavy armor offers the best protection.
    These suits of armor cover the entire body and are designed to stop a wide range of attacks.
    % Only proficient warriors can manage their weight and bulk.
    Heavy armor doesn't let you add your Dexterity modifier to your Armor Class, but it also doesn't penalize you if your Dexterity modifier is negative.

    The armor reduces your speed by 2 meters unless you have a Strength score equal to or higher than 15.

    \paragraph{Armor of Invulnerability $\odot$}
        You have resistance to physical damage while you wear this armor.
    \paragraph{Chain Mail}
        Made of interlocking metal rings, chain mail includes a layer of quilted fabric worn underneath the mail to prevent chafing and to cushion the impact of blows.
        The suit includes gauntlets.
    \paragraph{Demon Armor $\odot$}
        This armor earned its denomination from the grim spikes and scythes that adorn its design.
        While wearing this armor, the armor's clawed gauntlets turn unarmed strikes with your hands into weapons that deal slashing damage, with a +1 bonus to attack and damage rolls and a damage die of 1d8.
    \paragraph{Obsidian Wyvern Plate $\odot$}
        You gain a +2 bonus to AC and resistance to poison damage while you wear this armor.
        In addition, you gain advantage on ability checks and saving throws made to avoid or end the grappled condition on yourself.
    \paragraph{Plate Armor}
        Plate consists of shaped, interlocking metal plates to cover the entire body.
        A suit of plate includes gauntlets, heavy leather boots, a visored helmet, and thick layers of padding underneath the armor.
        Buckles and straps distribute the weight over the body.
    \paragraph{Ring Mail}
        This armor is leather armor with heavy rings sewn into it.
        The rings help reinforce the armor against blows from swords and axes.
        Ring mail is inferior to chain mail, and it's usually worn only by those who can't afford better armor.
    \paragraph{Splint Armor}
        This armor is made of narrow vertical strips of metal riveted to a backing of leather that is worn over cloth padding.
        Flexible chain mail protects the joints.
    \paragraph{Thulkrakan Plate}
        While wearing this armor, if an effect moves you against your will along the ground, you can use your reaction to reduce the distance you are moved by up to 2 meters.
    \paragraph{Tizerus' Chain $\odot$}
        While wearing this armor, you are immune to fire damage, and you can stand on and walk across molten rock as if it were solid ground.
\subsection*{Shields} \label{ssec::shields}
    \begin{table*}[b]%
        \begin{DndTable}[width=\linewidth, header=Armor]{Xccccccccc}
            \textbf{Shield} & \textbf{Rarity} & \textbf{AC} & \textbf{Type} & \textbf{Mats.} & \textbf{Cost} & \textbf{Tools} & \textbf{Weight} & \textbf{Source} \\
            Buckler              & Plain     & +1     & Light  & 3 &       5 agnomas & LEA + SMI &  1 kg  & ---       \\
            Kite Shield          & Plain     & +2     & Medium & 8 &      10 agnomas & CAR       &  3 kg  & PHB 144   \\
            Tower Shield         & Common    & +3     & Heavy  & 3 &      50 agnomas & LEA + CAR & 10 kg  & ---       \\
            +1 Shield            & Uncommon  & $\ast$ & $\ast$ & 8 &     500 agnomas & $\ast$    & $\ast$ & DMG 200   \\
            Sentinel Shield      & Uncommon  & +1     & Light  & 8 &     500 agnomas & LEA + SMI &  2 kg  & DMG 199   \\
            +2 Shield            & Rare      & $\ast$ & $\ast$ & 8 &   5,000 agnomas & $\ast$    & $\ast$ & DMG 200   \\
            Alert Shield         & Rare      & +4     & Medium & 8 &   5,000 agnomas & CAR + SMI &  3 kg  & CoS 68    \\
            Arrow Catcher        & Rare      & +3     & Heavy  & 8 &   5,000 agnomas & CAR + TIN & 10 kg  & DMG 152   \\
            Battering Shield     & Rare      & +4     & Heavy  & 8 &   5,000 agnomas & CAR + SMI & 12 kg  & EGW 266   \\
            Pariah's Shield      & Rare      & +2     & Medium & 8 &   5,000 agnomas & LEA + SMI &  5 kg  & GGR 180   \\
            Shield of Attraction & Rare      & +1     & Light  & 8 &   5,000 agnomas & SMI + TIN &  3 kg  & DMG 200   \\
            +3 Shield            & Very Rare & $\ast$ & $\ast$ & 8 &  50,000 agnomas & $\ast$    & $\ast$ & DMG 200   \\
            Animated Shield      & Very Rare & +2     & Medium & 8 &  50,000 agnomas & CAR + TIN &  3 kg  & DMG 151   \\
            Spellguard Shield    & Very Rare & +3     & Heavy  & 8 &  50,000 agnomas & CAR + SMI & 10 kg  & DMG 201   \\
            Uven Shield          & Very Rare & +3     & Heavy  & 8 &  50,000 agnomas & LEA + MAS & 20 kg  & WDMM 299  \\
            Hidden Lord Shield   & Legendary & +5     & Heavy  & 8 & 250,000 agnomas & SMI + TIN & 12 kg  & BGDIA 225
        \end{DndTable}
    \end{table*}

    A shield is made from wood or metal and is carried in one hand.
    You can benefit from only one shield at a time.

    Shields are separated into three weight classes: small, medium, and heavy:
    \subparagraph{Small Shields}
        Light and small, these shields grant a bonus of +1 to AC.
        In addition, donning or doffing them requires one action instead of two.
    \subparagraph{Medium Shields}
        Standard and sturdy, these shields grant a bonus of +2 to AC.
    \subparagraph{Heavy Shields}
        These shields are bulky and unwieldy, but grant a +3 bonus to AC.
        In addition, they grant you half cover against ranged attacks.
        Due to their size, the wearer's move action requires 2 actions instead of one.

    $\ast$ \textit{AC bonus, type, required tools and weight all depend on the weight class of the shield.}

    \paragraph{+1/+2/+3 Shield}
        You have a +1/+2/+3 bonus to AC on top of the normal AC granted by the shield.
    \paragraph{Animated Shield $\odot$}
        While holding this shield, you can use an action to cause it to animate.
        The shield leaps into the air and hovers in your space to protect you as if you were wielding it, leaving your hands free.
        The shield remains animated for 1 minute, until you use an action to end this effect, or until you are incapacitated or die, at which point the shield falls to the ground or into your hand if you have one free.
    \paragraph{Alert Shield}
        The shield is emblazoned with a stylized silver dragon.
        The shield grants a +2 bonus to initiative if the bearer isn't incapacitated.
    \paragraph{Arrow Catcher $\odot$}
        You gain a +2 bonus to AC against ranged attacks while you wield this shield.
        This bonus is in addition to the shield's normal bonus to AC.
        In addition, whenever an attacker makes a ranged attack against a target within 1 meter of you, you can use your reaction to become the target of the attack instead.
    \paragraph{Battering Shield $\odot$}
        This iron tower shield has 3 charges, and it regains 1d3 expended charges daily at dawn.
        If you are holding the shield and push a creature within your reach at least 1 meter away, you can expend 1 charge to push that creature an additional 2 meters, knock it prone, or both.
    \paragraph{Buckler}
        This small metal shield acts more as a reinforced gauntlet than a shield, but its light weight makes it ideal for mobile combatants.
    \paragraph{Hidden Lord Shield $\odot$}
        This hellish shield is twisted into a fiendish face that resembles a devil of some sort.
        The shield has the following properties:
        \begin{itemize}
            \item \textbf{Tizerus' Friend}.
            You gain resistance to fire damage.
            \item \textbf{Engulfed in Flame}.
            The shield has 3 charges.
            You can expend 1 charge to cast fireball (see page \pageref{spell::fireball}) on a point you choose within range, or 2 charges to cast wall of fire (see page \pageref{spell::firestorm}) on three points within range (save DC 21 for each).
            The wall of fire spell lasts for 1 minute (no concentration required).
            The shield regains all expended charges daily at dawn.
            \item \textbf{Hellish Visage}.
            Anytime during your turn, you can choose to radiate an aura of dread for 1 minute.
            Any creature hostile to you that starts its turn within 4 meters of the shield must make a DC 18 Wisdom saving throw.
            On a failed save, the creature is frightened until the start of its next turn.
            On a successful save, the creature is immune to this power of the shield for the next 24 hours.
            Once you use this power, you can't use it again until the next dawn.
        \end{itemize}
    \paragraph{Kite Shield}
        This wooden shield provides solid defense against most attacks.
    \paragraph{Pariah's Shield $\odot$}
        Akhoash Oromai soldiers consider it an honor to bear this shield, even knowing that it might be the last honor they receive.
        The front of the shield is sculpted to depict a grieving face.

        You gain a +1 bonus to AC for every two allies within 1 meter of you (up to a maximum of +3) while you wield this shield.
        This bonus is in addition to the shield's normal bonus to AC.

        When a creature you can see within 1 meter of you takes damage, you can use your reaction to take that damage, instead of the creature taking it.
    \paragraph{Sentinel Shield}
        While holding this light shield, you have advantage on initiative rolls and Wisdom (Perception) checks.
        The shield is emblazoned with a symbol of an eye.
    \paragraph{Shield of Attraction $\odot$}
        While holding this metal shield, you have resistance to damage from ranged weapon attacks.

        Whenever a ranged weapon attack is made against a target within 2 meters of you, you become the target instead.
    \paragraph{Spellguard Shield $\odot$}
        While holding this shield, you have advantage on saving throws against spells and other magical effects, and spell attacks have disadvantage against you.
    \paragraph{Tower Shield}
        This heavy wooden shield is ideal for watchers and defenders that rarely need to move in combat.
    \paragraph{Uven Shield $\odot$}
        Stolen from the strange designs of the Krudzalian giants, this stone shield has a rune burned into its outward-facing side of unknown meaning.

        While holding the shield, you benefit from the following properties:
        \begin{itemize}
            \item \textbf{Winter's Friend}.
            You are immune to cold damage.
            \item \textbf{Deadly Rebuke}.
            Immediately after a creature hits you with a melee attack, you can use your reaction to deal 3d6 necrotic damage to that creature.
            \item \textbf{Bane}.
            You can cast the bane spell from the shield (save DC 17).
            The spell does not require concentration and lasts for 1 minute.
            Once you cast the spell from the shield, you can't do so again until you finish a short rest.
            \item \textbf{Gift of Vengeance}.
            You can transfer the shield's magic to a weapon by tracing the uven rune on the weapon with one finger.
            The transfer takes one long rest of work that requires the two items to be within 1 meter of each other.
            At the end, the shield is destroyed, and the rune is etched or burned into the chosen weapon.
            This weapon becomes a rare magic item that requires attunement.
            It has the properties of a +1 weapon.
            The bonus increases to +3 when the weapon is used against one type of creature, chosen by you at the time of the weapon's creation.
        \end{itemize}
\newpage~\newpage

    % !TEX root = ../main.tex
\section{Brews} \label{sec::brews}
    The prices listed for brews are for a small keg, which fits 8 tankards worth of brew.
    Divide this by 8 for the price of a tankard, or by 16 for the price of a cup.
    Prices consider only the brew itself, not the medium of transportation.

    It takes one minute to drink a brew and gain its benefits, which can be done at the end of a short rest.
    You can only be affected by one brew at the same time.
    % After drinking a brew a creature rolls a DC 8 Constitution saving throw.
    % On a failure, it empties the content of the brew

    \begin{table*}[b]%
        \begin{DndTable}[width=\linewidth, header=Brews]{Xlccccc}
            \textbf{Brew} & \textbf{Rarity} & \textbf{Mats.} & \textbf{Total Cost} & \textbf{Tools} & \textbf{Weight} & \textbf{Source} \\
            Ale                      & Mundane   & 2 &     20 fobs    & BRE &  4 kg   & PHB 158 \\
            Common Wine              & Plain     & 1 &      3 agnomas & BRE &  4 kg   & PHB 158 \\
            Ale of Heroism           & Common    & 1 &     30 agnomas & BRE &  4 kg   & --- \\
            Dozing Wine              & Common    & 1 &     30 agnomas & BRE &  4 kg   & --- \\
            Exalted Beer             & Common    & 1 &     30 agnomas & BRE &  4 kg   & --- \\
            Expeditious Brew         & Common    & 1 &     30 agnomas & BRE &  4 kg   & --- \\
            Laughing Ale             & Common    & 1 &     30 agnomas & BRE &  4 kg   & --- \\
            Beer of Vitality         & Uncommon  & 1 &    150 agnomas & BRE &  4 kg   & --- \\
            Brew of Absorption       & Uncommon  & 1 &    150 agnomas & BRE &  4 kg   & --- \\
            Brew of Giant's Strength & Uncommon  & 1 &    150 agnomas & BRE &  4 kg   & --- \\
            Calming Wine             & Uncommon  & 1 &    150 agnomas & BRE &  4 kg   & --- \\
            Dead Man's Ale           & Uncommon  & 1 &    150 agnomas & BRE &  4 kg   & --- \\
            Fine Wine                & Uncommon  & 2 &    200 agnomas & BRE &  4 kg   & PHB 158 \\
            Madman Wine              & Uncommon  & 1 &    150 agnomas & BRE &  4 kg   & --- \\
            Speaker's Whiskey        & Uncommon  & 1 &    150 agnomas & BRE &  4 kg   & --- \\
            Thassa's Mockery         & Uncommon  & 1 &    150 agnomas & BRE &  4 kg   & --- \\
            Breathkiller             & Rare      & 1 &  1,500 agnomas & BRE &  4 kg   & --- \\
            Brew of Clairvoyance     & Rare      & 1 &  1,500 agnomas & BRE &  4 kg   & --- \\
            Brew of Invulnerability  & Rare      & 1 &  1,500 agnomas & BRE &  4 kg   & --- \\
            Brew of Letargy          & Rare      & 1 &  1,500 agnomas & BRE &  4 kg   & --- \\
            Brew of Maximum Power    & Rare      & 1 &  1,500 agnomas & BRE &  4 kg   & EGW 268 \\
            Mindblower               & Rare      & 1 &  1,500 agnomas & BRE &  4 kg   & --- \\
            Soulcatcher (Bottle)     & Rare      & 8 &  5,000 agnomas & BRE &  0.5 kg & --- \\
            Brainbender              & Very Rare & 1 & 15,000 agnomas & BRE &  4 kg   & --- \\
            Brew of Vitality         & Very Rare & 1 & 15,000 agnomas & BRE &  4 kg   & --- \\
            Brew of Speed            & Very Rare & 1 & 15,000 agnomas & BRE &  4 kg   & --- \\
            Erebos' Ooze (Jar)       & Very Rare & 8 & 50,000 agnomas & BRE &  1 kg   & ERLW 278 \\
            Kickstarter (Bottle)     & Very Rare & 8 & 50,000 agnomas & BRE &  0.5 kg & --- \\
            Dancer's Whiskey         & Legendary & 1 & 75,000 agnomas & BRE &  4 kg   & --- \\
            Groundshaker             & Legendary & 1 & 75,000 agnomas & BRE &  4 kg   & --- \\
            Purphoros' Maker         & Legendary & 1 & 75,000 agnomas & BRE &  4 kg   & --- \\
            Seer's Vodka             & Legendary & 1 & 75,000 agnomas & BRE &  4 kg   & ---
        \end{DndTable}
    \end{table*}

    \paragraph{Ale of Heroism} % Heroism
        This ale is honey-flavored, reminding you of those most important to you.
        A creature who drinks this is imbued with bravery.
        For an hour, the creature is immune to being frightened and gains 1 temporary hit point at the start of each of its turns.
        After the hour passes, the creature loses all temporary hit points gained from this brew.
    \paragraph{Beer of Vitality} % Aid
        This beer emboldens creatures with toughness and resolve.
        The drinker's hit point maximum and current hit points increase by 5 for 1 hour.
    \paragraph{Brainbender} % Confusion
        This brew is a nut-flavored greenish liquid.
        The liquid assaults and twists creatures' minds, spawning delusions and provoking uncontrolled action.
        Each creature that drinks the brew must succeed on a DC 16 Wisdom saving throw or be affected by it for an hour.

        An affected target can't take reactions and must roll a d10 at the start of each of its turns to determine its behavior for that turn.

        \begin{DndTable}[width=\linewidth, header=Confusion Behavior]{lX}
            \textbf{d10} & \textbf{Behavior} \\
            1            & The creature uses all its movement to move in a random direction. To determine the direction, roll a d8 and assign a direction to each die face.
            The creature doesn't take any other action this turn. \\
            2-6	         & The creature doesn't move or take actions this turn. \\
            7-8	         & The creature uses its action to make a melee attack against a randomly determined creature within its reach.
            If there is no creature within its reach, the creature does nothing this turn. \\
            9-10         & The creature can act and move normally.
        \end{DndTable}

        For every minute that passes, an affected creature can make a DC 16 Wisdom saving throw.
        If it succeeds, this effect ends for that creature.
    \paragraph{Breathkiller} % Stinking Cloud
        This atrocity of a brew is a yellow muddy liquid that leaves a terrible itching sensation upon passing through your throat.
        After drinking it, a creature becomes surrounded by a 6-meter-radius sphere of yellow, nauseating gas centered on itself.
        The cloud spreads around corners, and its area is heavily obscured.
        The cloud lingers in the air for an hour after the brew is drunk.

        Each creature that is completely within the cloud at the start of its turn must make a DC 14 Constitution saving throw against poison.
        On a failed save, the creature spends its action that turn retching and reeling.
        Creatures that don't need to breathe or are immune to poison automatically succeed on this saving throw.
        This effect lasts for an hour.
    \paragraph{Brew of Absorption} % Absorb Elements + Dragon's breath of the absorbed element
        This drink seems to remain in your throat after drinking it, ready to return to the world on command.
        When the drinker takes acid, cold, fire, lightning, or poison damage within an hour of drinking the brew, it can use its reaction to gain resistance against the attack.
        As part of the same reaction, it can then exhale energy in a 3 meter cone.
        Each creature in the area must make a DC 12 Dexterity saving throw, taking 3d6 damage of the absorbed type on a failed save, or half as much damage on a successful one.
        The drinker can only use this ability once during the brew's duration.
    \paragraph{Brew of Clairvoyance} % Clairvoyance
        An eyeball bobs in this yellowish vodka but vanishes as the brew is drunk.
        Upon drinking this brew, you create an invisible sensor within 2 kilometers in a location familiar to you (a place you have visited or seen before) or in an obvious location that is unfamiliar to you (such as behind a door, around a corner, or in a grove of trees).
        The sensor remains in place for the duration, and it can't be attacked or otherwise interacted with.

        You can see and hear through the sensor as if you were in its space.

        A creature that can see the sensor (such as a creature with truesight) sees a luminous, intangible orb about the size of your fist.

        The sensor remains in position for 1 hour, after which it disappears.
    \paragraph{Brew of Giant Strength}
        This brew's transparent liquid has floating in it a sliver of fingernail from a cyclops.
        When you drink it, your Strength score changes to 21 for 1 hour.
        The potion has no effect on you if your Strength is equal to or greater than that score.
    \paragraph{Brew of Invulnerability}
        This syrupy liquid looks like liquefied iron.
        For an hour after drinking it, you have resistance to all physical damage.
    \paragraph{Brew of Lethargy} % Slow
        This thick whiskey seems to flow slower than naturally possible.
        After drinking it, a creature must succeed on a DC 14 Wisdom saving throw or be affected by this brew for an hour.

        An affected target's speed is halved, it takes a -2 penalty to AC and Dexterity saving throws, and it can't use reactions.
        On its turn, it has only 2 actions instead of the normal 3.

        If the creature attempts to cast a spell with a casting time of 2 or more actions, roll a d20.
        On an 11 or higher, the spell doesn't take effect until the creature's next turn, and the creature must use its actions on that turn to complete the spell.
        If it can't, the spell is wasted.

        A creature affected by this brew makes another DC 14 Wisdom saving throw for each minute that passes.
        On a successful save, the effect ends for it.
    \paragraph{Brew of Maximum Power}
        This glowing purple liquid smells of sugar and plum, but it has a muddy taste.

        The first time you cast a damage-dealing spell of 4th level or lower within an hour after drinking the brew, instead of rolling dice to determine the damage dealt, you can instead use the highest number possible for each die.
    \paragraph{Brew of Speed} % Haste
        This black liquid flows faster than what seems reasonable.
        For an hour, the drinker's speed is doubled, it gains a +2 bonus to AC, it has advantage on Dexterity saving throws, and it gains an additional action on each of its turns.

        When the effect ends, the target can't move or take actions for a minute, as a wave of lethargy sweeps over it.
        Additionally, it gains one level of exhaustion.
    \paragraph{Brew of Vitality} % Potion of Vitality
        The brew's crimson liquid regularly pulses with dull light, calling to mind a heartbeat.
        When you drink this brew, it removes any exhaustion you are suffering and cures any disease or poison affecting you.
        For the next 24 hours, you regain the maximum number of hit points for any Hit Die you spend.
    \paragraph{Calming Wine} % Calm Emotions
        This wine's fruity flavor seems to calm emotions on command.
        Any creature who drinks this rolls a DC 12 Charisma saving throw.
        A creature can choose to fail this saving throw if it wishes.
        If a creature fails, two effects occur with a duration of an hour:
        \begin{itemize}
            \item Any effect causing a target to be charmed or frightened is suppressed.
            When this effect ends, any suppressed effect resumes, provided that its duration has not expired in the meantime.
            \item The target feels indifferent about creatures that it is hostile toward.
            This indifference ends if the target is attacked or harmed by a spell or if it witnesses any of its friends being harmed.
            When the effect ends, the creature becomes hostile again, unless the DM rules otherwise.
        \end{itemize}

        A creature affected by this brew makes another DC 12 Charisma saving throw for every minute that passes.
        On a successful save, the effect ends for it.
    \paragraph{Dancer's Whiskey} % Otto's Irresistible Dance
        This fuzzy pink brew feels like small explosions happening inside your mouth.
        After finishing the brew, the drinker begins a comic dance in place: shuffling, tapping its feet, and capering for an hour.

        A dancing creature must use all its movement to dance without leaving its space and has disadvantage on Dexterity saving throws and attack rolls.
        While the target is affected by this brew, other creatures have advantage on attack rolls against it.
        At the end of every minute that passes, a dancing creature makes a DC 18 Wisdom saving throw to regain control of itself.
        On a successful save, the effect ends.
    \paragraph{Dead Man's Ale} % Blindness/Deafness
        This brew looks like common ale, hiding a terrible effect within.
        After drinking it, the creature rolls a DC 12 Constitution saving throw.
        If it fails, the target is blinded for an hour.

        A creature affected by this brew can makes another DC 12 Constitution saving throw at the end of every minute that passes.
        On a successful save, the effect ends for it.
    \paragraph{Dozing Wine} % Sleep
        This wine is so comforting that it sends creatures into slumber.
        Any creature who drinks this must succeed on a DC 10 Constitution saving throw.
        On a failure, the creature falls unconscious for an hour, until the sleeper takes damage, or someone uses an action to shake or slap the sleeper awake.

        A creature affected by this brew makes another DC 10 Constitution saving throw at the end of every minute that passes.
        On a successful save, the effect ends for it.
    \paragraph{Erebos' Ooze (Jar) $\odot$} \label{item::erebosooze} % Kyrzin's Ooze
        This opalescent, symbiotic goo comes sealed in a jar and slowly shifts and moves, as if endlessly exploring the jar's interior.
        You can attune to this item by drinking the contents of the jar, unlocking the following properties.

        \begin{itemize}
            \item \textbf{Resistant} While attuned to Erebos' ooze, you have resistance to poison and acid damage, and you're immune to the poisoned condition.
            \item \textbf{Amorphous} Using two actions, you can speak a command word and cause your body to assume the amorphous qualities of an ooze.
            For the next minute, you (along with any equipment you're wearing or carrying) can move through a space as narrow as 1 inch wide without squeezing.
            Once you use this property, it can't be used again until the next dawn.
            \item \textbf{Acid Breath} Using two actions, you can exhale black acid in a 6-meter line that is 1 meters wide.
            Each creature in that line must make a DC 15 Dexterity saving throw, taking 36 (8d8) acid damage on a failed save, or half as much damage on a successful one.
            Once you use this property, it can't be used again until the next dawn.
            \item \textbf{Symbiotic Nature} The ooze can't be removed from you while you're attuned to it, and you can't voluntarily end your attunement to it.
            The only way to remove the ooze is by drinking an \textbf{Oozekiller} potion (see page \pageref{item::oozekiller}).
        \end{itemize}

        If you die while the ooze is inside you, it bursts out and engulfs you, turning your corpse into a black pudding.
    \paragraph{Exalted Beer} % Bless
        The sweet flavor of this beer truly is a blessing.
        After drinking it, the creature can add a d4 to any attack roll or saving throw it makes for one hour.
    \paragraph{Expeditious Brew} % Expeditious Retreat
        This brew feels like caustic acid falling through your throat.
        After drinking it, the creature gains a bonus of 2 meters to its movement speed for one hour.
    \paragraph{Groundshaker} % Investiture of Stone
        Small bits of stone float aimlessly in this amber root beer.
        For an hour after drinking it, bits of rock spread across your body, and you gain the following benefits:
        \begin{itemize}
            \item You have resistance to bludgeoning, piercing, and slashing damage from nonmagical attacks.
            \item You can use your action to create a small earthquake on the ground in a 3-meter radius centered on you.
            Other creatures on that ground must succeed on a DC 18 Dexterity saving throw or be knocked prone.
            \item You can move across difficult terrain made of earth or stone without spending extra movement.
            You can move through solid earth or stone as if it was air and without destabilizing it, but you can't end your movement there.
            If you do so, you are ejected to the nearest unoccupied space, this effect ends, and you are stunned until the end of your next turn.
        \end{itemize}
    \paragraph{Kickstarter (Bottle)} % Awaken
        This color-changing thick liquid is said to contain the very essence of the tides.
        Any Huge or smaller beast with an intelligence of 3 or less can receive the effect of this potion.
        The target gains an Intelligence of 10.

        The awakened beast is charmed by you for 30 days or until you or your companions do anything harmful to it.
        When the charmed condition ends, the awakened creature chooses whether to remain friendly to you, based on how you treated it while it was charmed.

        After this period passes, the creature needs a qualar to remain sentient, and it suffers Dementia effects normally (See page \pageref{ssec::dementia}).
    \paragraph{Laughing Brew} % Tasha's Hideous Laughter
        This fuzzy and bubbly drink seems to vibrate with astounding speed.
        After drinking it, the creature perceives everything as hilariously funny and falls into fits of laughter.
        The creature must succeed on a DC 10 Wisdom saving throw or fall prone, becoming incapacitated and unable to stand up for up to an hour.
        A creature with an Intelligence score of 4 or less isn't affected.

        A creature affected by this brew makes another DC 10 Wisdom saving throw at the end of every minute and when it takes damage.
        The target has advantage on the saving throw if it's triggered by damage.
        On a success, the effect ends.
    \paragraph{Madman Wine} % Crown of Madness
        This wine looks normal, but has a somewhat odd flavor.
        A creature who drinks this must succeed on a DC 12 Wisdom saving throw or become charmed for an hour.
        While charmed, madness glows in the creature's eyes.

        The charmed creature must use its action before moving on each of its turns to make a melee attack against a creature other than itself within 1 meter of it chosen randomly.
        If no creatures are within reach, it moves towards the closest creature to attack it.

        The target can make a DC 12 Wisdom saving throw at the end of each minute that passes.
        On a success, the effect ends.
    \paragraph{Mindblower} % Speak with Plants + Hallucinatory Terrain
        This brew is a clear liquid indistinguishable from water.
        Upon drinking it, you can communicate with any plant within 6 meters of you and issue it simple commands for an hour.
        You can question plants about events in the spell's area within the past day, gaining information about creatures that have passed, weather, and other circumstances.

        You can also turn difficult terrain caused by plant growth (such as thickets and undergrowth) into ordinary terrain that lasts for the duration.
        Or you can turn ordinary terrain where plants are present into difficult terrain that lasts for the duration, causing vines and branches to hinder pursuers, for example.

        Plants might be able to perform other tasks on your behalf, at the DM's discretion. The spell doesn't enable plants to uproot themselves and move about, but they can freely move branches, tendrils, and stalks.

        If a plant creature is in the area, you can communicate with it as if you shared a common language, but you gain no magical ability to influence it.

        This spell can cause the plants created by the entangle spell to release a restrained creature.

        While under the effect of this brew, all terrain looks hard to traverse, and your movement speed is halved.
    \paragraph{Purphoros' Maker} % Flesh to Stone
        Upon drinking this unassuming wine, a creature begins to turn into stone.
        If the drinker's body is made of flesh, the creature must make a DC 18 Constitution saving throw.
        On a failed save, it is restrained as its flesh begins to harden.
        On a successful save, the creature isn't affected.

        A creature restrained by this effect must make another DC 18 Constitution saving throw at the end of each of its turns.
        If it successfully saves against this brew three times, the effect ends.
        If it fails its saves three times, it is turned to stone and subjected to the petrified condition until the effect is removed.
        The successes and failures don't need to be consecutive; keep track of both until the target collects three of a kind.

        If the creature is physically broken while petrified, it suffers from similar deformities if it reverts to its original state.
    \paragraph{Seer's Vodka} % True Seeing
        Reflections in this clear vodka seem out of place, as if the turbulent liquid displaced light in turbulent ways.
        This brew gives the drinker the ability to see things as they actually are.
        For an hour, the creature has truesight, and notices secrets hidden by magic out to a range of 24 meters.
    \paragraph{Soulcatcher} % Revivify
        As two actions, you can force a creature that has died within the last minute to drink this liquid.
        That creature returns to life with 1 hit point.
        This spell can't return to life a creature that has died of old age, nor can it restore any missing body parts.
    \paragraph{Speaker's Whiskey} % Zone of Truth on a target
        This strong whiskey seems to pick at your brain, forcing you to feel remorse about your misdeeds.
        Any creature who drinks this must make a DC 12 Charisma saving throw.
        On a failed save, a creature can't speak a deliberate lie for an hour.
        A skilled brewer knows whether each creature succeeds or fails on its saving throw.

        An affected creature is aware of the effect and can thus avoid answering questions to which it would normally respond with a lie.
        Such creatures can be evasive in its answers as long as it remains within the boundaries of the truth.
    \paragraph{Thassa's Mockery} % Water Breathing
        This transparent vodka has fish scales floating in it, which must be drunk with the brew to gain its effect.
        This drinks grants a creature the ability to breathe underwater for an hour.
        Affected creatures also retain their normal mode of respiration.
\newpage

    % !TEX root = ../main.tex
\section{Clothing} \label{sec::clothing}
    \begin{table*}[b]%
        \begin{DndTable}[width=\linewidth, header=Clothing and Accessories]{Xlccccc}
            \textbf{Item} & \textbf{Rarity} & \textbf{Mats.} & \textbf{Total Cost} & \textbf{Tools} & \textbf{Weight} & \textbf{Source} \\
            \multicolumn{7}{l}{\hspace{0.5cm}\textit{Clothes}} \\
            Common Clothes               & Mundane   & 8 &      50 fobs    & WEA & 1.5 kg & PHB   150 \\
            Costume Clothes              & Plain     & 3 &       5 agnomas & WEA & 2 kg   & PHB   150 \\
            Robes                        & Plain     & 1 &       1 agnoma  & WEA & 2 kg   & PHB   150 \\
            Traveler's Clothes           & Plain     & 1 &       2 agnomas & WEA & 2 kg   & PHB   150 \\
            Warm Clothing                & Plain     & 8 &      10 agnomas & WEA & 2.5 kg & IDRotF 20 \\
            Boots of False Tracks        & Common    & 8 &     100 agnomas & COB & 1 kg   & XGE   136 \\
            Fine Clothes                 & Common    & 1 &      30 agnomas & WEA & 3 kg   & PHB   150 \\
            Boggart Boots                & Uncommon  & 8 &     500 agnomas & COB & ---    & DMG   155 \\
            Boots of Springing           & Uncommon  & 8 &     500 agnomas & COB & ---    & DMG   156 \\
            Winter Boots                 & Uncommon  & 8 &     500 agnomas & COB & ---    & DMG   156 \\
            Boots of Speed               & Rare      & 8 &   5,000 agnomas & COB & ---    & DMG   155 \\
            Robe of Scintillating Colors & Very Rare & 8 &  50,000 agnomas & WEA & ---    & DMG   194 \\
            Robe of the Archmagi         & Legendary & 8 & 250,000 agnomas & WEA & ---    & DMG   194 \\
            \multicolumn{7}{l}{\hspace{0.5cm}\textit{Accessories}} \\
            Athlete's Gloves             & Uncommon  & 8 &     500 agnomas & LEA & ---    & DMG   172 \\
            Bracers of Archery           & Uncommon  & 8 &     500 agnomas & LEA & ---    & DMG   156 \\
            Cloak of Protection          & Uncommon  & 8 &     500 agnomas & LEA & ---    & DMG   159 \\
            Cloak of Stealth             & Uncommon  & 8 &     500 agnomas & LEA & ---    & DMG   158 \\
            Cloak of Thassa              & Uncommon  & 8 &     500 agnomas & LEA & ---    & DMG   159 \\
            Gloves of Snaring            & Uncommon  & 8 &     500 agnomas & WEA & ---    & DMG   172 \\
            Eyes of Charming             & Uncommon  & 8 &     500 agnomas & GLA & ---    & DMG   168 \\
            Eyes of Minute Seeing        & Uncommon  & 8 &     500 agnomas & GLA & ---    & DMG   168 \\
            Eyes of the Eagle            & Uncommon  & 8 &     500 agnomas & GLA & ---    & DMG   168 \\
            Hat of Disguise              & Uncommon  & 8 &     500 agnomas & LEA & ---    & DMG   173 \\
            Helm of Communication        & Uncommon  & 8 &     500 agnomas & JEW & ---    & DMG   174 \\
            Mask of Adaptation           & Uncommon  & 8 &     500 agnomas & WEA & ---    & DMG   182 \\
            Thieves' Gloves              & Uncommon  & 8 &     500 agnomas & WEA & ---    & DMG   172 \\
            Belt of Giant's Strength     & Rare      & 8 &   5,000 agnoams & LEA & ---    & DMG   155 \\
            Bracers of Defense           & Rare      & 8 &   5,000 agnomas & LEA & ---    & DMG   156 \\
            Cape of the Mountebank       & Rare      & 8 &   5,000 agnomas & WEA & ---    & DMG   157 \\
            Cloak of Displacement        & Rare      & 8 &   5,000 agnomas & LEA & ---    & DMG   158 \\
            Cloak of Marsetkind          & Rare      & 8 &   5,000 agnomas & LEA & ---    & DMG   159 \\
            Living Gloves                & Rare      & 6 &   4,000 agnomas & ALC & ---    & ERLW  278 \\
            Watchful Helm                & Rare      & 8 &   5,000 agnomas & JEW & ---    & CM    183 \\
            Cloak of Arasta              & Very Rare & 8 &  50,000 agnomas & LEA & ---    & DMG   158 \\
            % Helm of Fiend Command        & Very Rare & 8 &  50,000 agnomas & SMI & 1 kg   & BGDIA 223 \\
            % Illusionist's Bracers        & Very Rare & 8 &  50,000 agnoams & JEW & ---    & GGR   178 \\
            Wings of Flying              & Very Rare & 8 &  50,000 agnomas & WEA & ---    & DMG   214 \\
            Cloak of Invisibility        & Legendary & 8 & 250,000 agnomas & LEA & ---    & DMG   158 \\
        \end{DndTable}
    \end{table*}
    \begin{table*}[b]%
        \begin{DndTable}[width=\linewidth, header=Clothing and Accessories (Cont.)]{Xlccccc}
            \multicolumn{7}{l}{\hspace{0.5cm}\textit{Jewelry}} \\
            Amulet                       & Plain     & 3 &       5 agnomas & JEW & 0.5 kg & PHB   151 \\
            Signet Ring                  & Plain     & 3 &       5 agnomas & JEW & ---    & PHB   150 \\
            Amulet of the Closed Eye     & Uncommon  & 8 &     500 agnomas & JEW & 0.5 kg & DMG   150 \\
            Brooch of Shielding          & Uncommon  & 8 &     500 agnomas & GLA & ---    & DMG   156 \\
            Periapt of Health            & Uncommon  & 8 &     500 agnomas & JEW & 0.5 kg & DMG   184 \\
            Amulet of Health             & Rare      & 8 &   5,000 agnomas & JEW & 0.5 kg & DMG   150 \\
            Necklace of Fireballs        & Rare      & 8 &   5,000 agnomas & JEW & 0.5 kg & DMG   182 \\
            Ring of Animal Influence     & Rare      & 8 &   5,000 agnomas & JEW & ---    & DMG   189 \\
            Ring of Evasion              & Rare      & 8 &   5,000 agnomas & JEW & ---    & DMG   191 \\
            Ring of Resistance           & Rare      & 8 &   5,000 agnomas & JEW & ---    & DMG   192 \\
            Ring of Salvation            & Rare      & 8 &   5,000 agnomas & JEW & ---    & EGW   269 \\
            Ring of Spell Storing        & Rare      & 8 &   5,000 agnomas & JEW & ---    & DMG   192 \\
            Ring of the Ram              & Rare      & 8 &   5,000 agnomas & JEW & ---    & DMG   193 \\
            Ring of Spell Turning        & Legendary & 8 & 250,000 agnomas & JEW & ---    & DMG   193 \\
            Talisman of the Sphere       & Legendary & 8 & 250,000 agnoams & JEW & 0.5 kg & DMG   207 \\
            \multicolumn{7}{l}{\hspace{0.5cm}\textit{Qualars}} \\
            Qualar of Mind Shielding     & Uncommon  & 8 &     500 agnomas & WOO & ---    & DMG   191 \\
            Qualar of Thoughts           & Uncommon  & 8 &     500 agnomas & WOO & ---    & DMG   181 \\
            Qualar of Truth Telling      & Uncommon  & 8 &     500 agnomas & WOO & ---    & WDH   192 \\
            Qualar of Intellect          & Rare      & 8 &   5,000 agnomas & WOO & ---    & DMG   173
        \end{DndTable}
    \end{table*}

    \paragraph{Amulet of Health $\odot$}
        While wearing this amulet, your Constitution ability score increases by 4, up to a limit of 24.
    \paragraph{Amulet of the Closed Eye $\odot$}
        While you are wearing this amulet, you can't be targeted by scrying magic or be perceived by any incorporeal sensors.
    \paragraph{Athlete's Gloves $\odot$}
        While wearing these gloves, climbing and swimming don't cost you extra movement, and you gain a +5 bonus to Strength (Athletics) checks made to climb or swim.
    \paragraph{Belt of Giant's Strength $\odot$}
        While wearing this belt, your Strength ability score increases by 4, up to a limit of 24.
    \paragraph{Boggart Boots}
        While you wear these boots, your steps make no sound, regardless of the surface you are moving across. You also have advantage on Dexterity (Stealth) checks that rely on moving silently.
    \paragraph{Boots of False Tracks}
        By using two actions to adjust the soles, you can adjust the size and shape of your tracks to make them look like those of a wild animal.
    \paragraph{Boots of Speed $\odot$}
        While you wear these boots, you can use an action and click the boots' heels together.
        If you do, the boots double your walking speed, and any creature that makes an opportunity attack against you has disadvantage on the attack roll.
        If you click your heels together again, you end the effect.

        When the boots' property has been used for a total of 10 minutes, the boots ceases to function until you finish a short rest.
    \paragraph{Boots of Springing $\odot$}
        While you wear these boots, your walking speed becomes 6 meters, unless your walking speed is higher, and your speed isn't reduced if you are encumbered or wearing heavy armor.
        In addition, you can jump three times the normal distance, though you can't jump farther than your remaining movement would allow.
    \paragraph{Bracers of Archery $\odot$}
        While wearing these bracers, you gain a +2 bonus to damage rolls on ranged attacks made with weapons of the bow and crossbow types.
    \paragraph{Bracers of Defense $\odot$}
        While wearing these bracers, you gain a +2 bonus to AC if you are wearing no armor and using no shield.
    \paragraph{Brooch of Shielding $\odot$}
        While wearing this brooch, you have resistance against force damage.
    \paragraph{Cape of the Mountebank}
        This cape smells faintly of brimstone.
        While wearing it, you can use it to teleport up to 100 meters using two actions.

        You can bring along objects as long as their weight doesn't exceed what you can carry.
        You can also bring one willing creature of your size or smaller who is carrying gear up to its carrying capacity.
        The creature must be within a meter of you when you cast this spell.

        If you would arrive in a place already occupied by an object or a creature, you and any creature traveling with you each take 4d6 force damage, and the spell fails to teleport you.

        This property of the cape can't be used again until the next dawn.

        When you disappear, you leave behind a cloud of smoke, and you appear in a similar cloud of smoke at your destination.
        The smoke lightly obscures the space you left and the space you appear in, and it dissipates at the end of your next turn. A light or stronger wind disperses the smoke.
    \paragraph{Cloak of Arasta $\odot$}
        This fine garment is made of black silk interwoven with faint silvery threads.
        While wearing it, you gain the following benefits:
        \begin{itemize}
            \item You have resistance to poison damage.
            \item You have a climbing speed equal to your walking speed.
            \item You can move up, down, and across vertical surfaces and upside down along ceilings, while leaving your hands free.
            \item You can't be caught in webs of any sort and can move through webs as if they were difficult terrain.
            \item You can use an action to cast the web spell (save DC 13).
            The web created by the spell fills twice its normal area.
            Once used, this property of the cloak can't be used again until the next dawn.
        \end{itemize}
    \paragraph{Cloak of Displacement $\odot$}
        While you wear this cloak, it projects an illusion that makes you appear to be standing in a place near your actual location, causing any creature to have disadvantage on attack rolls against you.
        If you take damage, the property ceases to function until the start of your next turn.
        This property is suppressed while you are incapacitated, restrained, or otherwise unable to move.
    \paragraph{Cloak of Invisibility $\odot$}
        While wearing this cloak, you can pull its hood over your head to cause yourself to become invisible.
        While you are invisible, anything you are carrying or wearing is invisible with you.
        You become visible when you cease wearing the hood.
        Pulling the hood up or down requires an action.

        Deduct the time you are invisible, in increments of 1 minute, from the cloak's maximum duration of 2 hours.
        After 2 hours of use, the cloak ceases to function.
        For every uninterrupted period of 12 hours the cloak goes unused, it regains 1 hour of duration.
    \paragraph{Cloak of Marsetkind $\odot$}
        While wearing this cloak, you have advantage on Dexterity (Stealth) checks.
        In an area of dim light or darkness, you can grip the edges of the cloak with both hands and use it to fly at a speed of 8 meters.
        If you ever fail to grip the cloak's edges while flying in this way you lose this flying speed.
    \paragraph{Cloak of Protection $\odot$}
        You gain a +1 bonus to AC and saving throws while you wear this cloak.
    \paragraph{Cloak of Stealth $\odot$}
        While you wear this cloak with its hood up, Wisdom (Perception) checks made to see you have disadvantage, and you have advantage on Dexterity (Stealth) checks made to hide, as the cloak's color shifts to camouflage you.
        Pulling the hood up or down requires an action.
    \paragraph{Cloak of Thassa $\odot$}
        While wearing this cloak with its hood up, you can breathe underwater, and you have a swimming speed of 12 meters.
        Pulling the hood up or down requires an action.
    \paragraph{Eyes of Charming $\odot$}
        These crystal lenses fit over the eyes.
        They have 3 charges.
        While wearing them, you can expend 1 charge as an action to cast the charm person spell (see page \pageref{spell::charmperson}) with a save DC of 13 on a humanoid within 6 meters of you, provided that you and the target can see each other.
        The lenses regain all expended charges daily at dawn.
    \paragraph{Eyes of Minute Seeing}
        These crystal lenses fit over the eyes.
        While wearing them, you can see much better than normal out to a range of 30 cm.
        You have advantage on Intelligence (Investigation) checks that rely on sight while searching an area or studying an object within that range.
    \paragraph{Eyes of the Eagle $\odot$}
        These crystal lenses fit over the eyes.
        While wearing them, you have advantage on Wisdom (Perception) checks that rely on sight.
        In conditions of clear visibility, you can make out details of even extremely distant creatures and objects as small as 60 cm. across.
    \paragraph{Gloves of Snaring $\odot$}
        These gloves seem to almost meld into your hands when you don them.
        When a ranged weapon attack hits you while you're wearing them, you can use your reaction to reduce the damage by 1d10 + your Dexterity modifier, provided that you have a free hand.
        If you reduce the damage to 0, you can catch the missile if it is small enough for you to hold in that hand.
    \paragraph{Hat of Disguise $\odot$}
        While wearing this hat, you can use two actions to cast the disguise self spell from it at will.
        The spell ends if the hat is removed.
    \paragraph{Helm of Communication $\odot$}
        Wearing this copper helm brings the constant sensation of a light breeze on the wearer's head.
        While wearing this helm, you can use two actions to cast the detect thoughts spell (save DC 13) from it.
        As long as you maintain concentration on the spell, you can use an action to send a message to a creature you are focused on.
        It can reply --- using an action to do so --- while your focus on it continues.

        While focusing on a creature with detect thoughts, you can use an action to cast the suggestion spell (save DC 13) from the helm on that creature.
        Once used, the suggestion property can't be used again until the next dawn.
    % \paragraph{Helm of Fiend Command $\odot$}
    %     While wearing this bulky, eyelass helmet, you can see out of it as though you weren't wearing it.
    %     In addition, you know the exact location and type of all fiends within a kilometer of you.
    %     You can telepathically communicate with a fiend within range, or you can broadcast your thoughts to all fiends within range.
    %     The fiends receiving your broadcasted thoughts have no special means of replying to them.
    %
    %     The helm has 3 charges.
    %     Using two actions, you can expend 1 charge to cast dominate monster with save DC 21, which affects fiends only.
    %     If a fiend can see you when you cast this spell on it, the fiend knows you tried to charm it.
    %     The helm regains all its charges 24 hours after its last charge is expended.
    % \paragraph{Illusionist's Bracers $\odot$}
    %     While wearing these intricate bracers, whenever you cast a cantrip, you can use an action on the same turn to cast that cantrip a second time.
    %     You can do this only once per turn.
    \paragraph{Living Gloves $\odot$}
        These symbiotic gloves --- made of thin chitin and sinew --- pulse with a life of their own.
        To attune to them, you must wear them for the entire attunement period, during which the gloves bond with your skin.

        While attuned to these gloves, you double your proficiency bonus with one of the following proficiencies (your choice when you attune to the gloves):
        \begin{itemize}
            \item Sleight of Hand,
            \item thieves' tools,
            \item one kind of artisan's tools of your choice, or
            \item one kind of musical instrument of your choice.
        \end{itemize}

        \paragraph{Symbiotic Nature}
        The gloves can't be removed from you while you're attuned to them, and you can't voluntarily end your attunement to them.
        The only way to end your attunement to the gloves is by pouring Oozekiller (see page \pageref{item::oozekiller}) on them, which destroys the gloves.
    \paragraph{Mask of Adaptation $\odot$}
        While wearing this necklace, you can breathe normally in any environment, and you have advantage on saving throws made against harmful gases and vapors (such as cloudkill and stinking cloud effects, and inhaled poisons).
    \paragraph{Periapt of Health}
        You have advantage on saving throws made to avoid contracting a disease while you wear this pendant.
        If you are infected with a disease, the effects of the disease are slower, and the time it takes for its effect to take place are doubled.
    \paragraph{Qualar of Mind Shielding $\odot$}
        While carrying this qualar, you are immune to magic that allows other creatures to read your thoughts, determine whether you are lying, know your tidal alignment, or know your creature type.
        Creatures can telepathically communicate with you only if you allow it.
    \paragraph{Qualar of Intellect $\odot$}
        While wearing this qualar, your Intelligence ability score increases by 4, up to a limit of 24.
    \paragraph{Qualar of Thoughts $\odot$}
        The qualar has 3 charges.
        While wearing it, you can use two actions and expend 1 charge to cast the detect thoughts spell with save DC 13 from it.
        The qualar regains 1d3 expended charges daily at dawn.
    \paragraph{Qualar of Truth Telling $\odot$}
        While wearing this ring, you have advantage on Wisdom (Insight) checks to determine whether someone is lying to you.
    \paragraph{Necklace of Fireballs}
        This necklace has 1d6 + 3 beads hanging from it. You can use an action to detach a bead and throw it up to 12 meters away.
        When it reaches the end of its trajectory, the bead detonates as a 3rd-level fireball spell with save DC 15.

        You can hurl multiple beads, or even the whole necklace, as one action.
        When you do so, increase the level of the fireball by 1 for each bead beyond the first.
    \paragraph{Ring of Animal Influence}
        This bone ring has 3 charges, and it regains 1d3 expended charges daily at dawn.
        While wearing the ring, you can use an action to expend 1 of its charges to cast one of the following spells with save DC 13:
        \begin{itemize}
            \item Animal friendship,
            \item Fear, targeting only beasts that have an Intelligence of 3 or lower.
        \end{itemize}
    \paragraph{Ring of Evasion $\odot$}
        This ring has 3 charges, and it regains 1d3 expended charges daily at dawn.
        When you fail a Dexterity saving throw while wearing it, you can use your reaction to expend 1 of its charges to succeed on that saving throw instead.
    \paragraph{Ring of Resistance $\odot$}
        You have resistance to one damage type while wearing this ring.
        The gem in the ring indicates the type.
        \begin{DndTable}[width=\linewidth, header=Resistance Gems]{cll}
            \textbf{d10} & \textbf{Damage Type} & \textbf{Gem} \\
            1            & Acid                 & Pearl        \\
            2            & Cold                 & Tourmaline   \\
            3            & Fire                 & Garnet       \\
            4            & Force                & Sapphire     \\
            5            & Lightning            & Citrine      \\
            6            & Necrotic             & Jet          \\
            7            & Poison               & Amethyst     \\
            8            & Psychic              & Jade         \\
            9            & Radiant              & Topaz        \\
            10           & Thunder              & Spinel
        \end{DndTable}
    \paragraph{Ring of Salvation $\odot$}
        If you die while wearing this gray crystal ring, it breathes life back to you, waking you up.
        You have a number of hit points equal to 3d6 + your Constitution modifier.
        If your hit point maximum is lower than the number of hit points you regain, your hit point maximum rises to a similar amount.
        If you have any levels of exhaustion, reduce your level of exhaustion by 1.
        Once the ring is used, it turns to dust and is destroyed.
    \paragraph{Ring of Spell Storing $\odot$}
        This ring stores spells cast into it, holding them until the attuned wearer uses them.
        The ring can store up to 5 levels worth of spells at a time.

        Any creature can cast a spell of 1st through 5th level into the ring by touching the ring as the spell is cast.
        The spell has no effect, other than to be stored in the ring.
        If the ring can't hold the spell, the spell is expended without effect.
        The level of the slot used to cast the spell determines how much space it uses.

        While wearing this ring, you can cast any spell stored in it.
        The spell uses the slot level, spell save DC, spell attack bonus, and spellcasting ability of the original caster, but is otherwise treated as if you cast the spell.
        The spell cast from the ring is no longer stored in it, freeing up space.
    \paragraph{Ring of Spell Turning $\odot$}
        While wearing this ring, you have advantage on saving throws against any spell that targets only you (not in an area of effect).
        In addition, if you roll a 20 for the save and the spell is 7th level or lower, the spell has no effect on you and instead targets the caster, using the slot level, spell save DC, attack bonus, and spellcasting ability of the caster.
    \paragraph{Ring of the Ram $\odot$}
        This ring has 3 charges, and it regains 1d3 expended charges daily at dawn.
        While wearing the ring, you can use an action to expend 1 to 3 of its charges to attack one creature you can see within 12 meters of you.
        The ring produces a spectral ram's head and makes its attack roll with a +7 bonus.
        On a hit, for each charge you spend, the target takes 2d10 force damage and is pushed 1 meter away from you.

        Alternatively, you can expend 1 to 3 of the ring's charges as an action to try to break an object you can see within 12 meters of you that isn't being worn or carried.
        The ring makes a Strength check with a +5 bonus for each charge you spend.
    \paragraph{Robe of Scintillating Colors $\odot$}
        This robe has 3 charges, and it regains 1d3 expended charges daily at dawn.
        While you wear it, you can use two actions and expend 1 charge to cause the garment to display a shifting pattern of dazzling hues until the end of your next turn.
        During this time, the robe sheds bright light in a 6-meter radius and dim light for an additional 6 meters.
        Creatures that can see you have disadvantage on attack rolls against you.
        In addition, any creature in the bright light that can see you when the robe's power is activated must succeed on a DC 15 Wisdom saving throw or become stunned until the effect ends.
    \paragraph{Robe of the Magi $\odot$}
        This elegant garment is made from exquisite cloth of white, gray, or black and adorned with silvery runes.
        The robe's color corresponds to the tidal alignment of the wearer.

        You gain these benefits while wearing the robe:
        \begin{itemize}
            \item If you aren't wearing armor, your base Armor Class is 15 + your Dexterity modifier.
            \item You have advantage on saving throws against spell and other magical effects.
            \item Your spell save DC and spell attack bonus each increase by 2.
        \end{itemize}
    \paragraph{Talisman of the Sphere $\odot$}
        When you make an Intelligence (Arcana) check to control a sphere of annihilation while you are holding this talisman, you double your proficiency bonus on the check.
        In addition, when you start your turn with control over a sphere of annihilation, you can use an action to levitate it 2 meters plus a number of additional meters equal to 2 times your Intelligence modifier.
    \paragraph{Thieves' Glovers}
        These gloves are invisible while worn.
        While wearing them, you gain a +5 bonus to Dexterity (Sleight of Hand) checks and Dexterity checks made to pick locks.
    \paragraph{Warm Clothing}
        This outfit consists of a heavy fur coat or cloak over layers of wool clothing, as well as a fur-lined hat or hood, goggles, and fur-lined leather boots and gloves.

        As long as this clothing remains dry, its wearer automatically succeeds on saving throws against the effects of extreme cold.
    \paragraph{Watchful Helm $\odot$}
        While you wear this helm, you gain a +1 bonus to AC and remain aware of your surroundings even while you're asleep, and you have advantage on Wisdom (Perception) checks that rely on sight.
    \paragraph{Wings of Flying $\odot$}
        While wearing this cloak, you can use two actions to sprout bird wings on your back.
        This effect lasts for one hour or until you use an action to end it.
        The wings give you a flying speed of 12 meters.
        When they disappear, you can't use them again for 1d12 hours.
    \paragraph{Winter Boots $\odot$}
        These furred boots are snug and feel quite warm.
        While you wear them, you gain the following benefits:
        \begin{itemize}
            \item You have resistance to cold damage.
            \item You ignore difficult terrain created by ice or snow.
            \item You can tolerate temperatures as low as -50 degrees Celsius without any additional protection.
            If you wear heavy clothes, you can tolerate temperatures as low as -70 degrees Celsius.
        \end{itemize}
\newpage~\newpage

    % !TEX root = ../main.tex
\section{Food, Drink, and Lodging} \label{sec::fooddrinkandlodging}
\begin{table*}[b]%
    \begin{DndTable}[width=\linewidth, header=Food and Lodging]{Xlccccc}
        \textbf{Food or Service} & \textbf{Rarity} & \textbf{Mats.} & \textbf{Total Cost} & \textbf{Tools} & \textbf{Weight} & \textbf{Source} \\
        \multicolumn{7}{l}{\hspace{0.5cm}\textit{Inn Stay (per day)}} \\
        Squalid              & ---       & --- &     7 fobs    & --- & ---  & PHB 158 \\
        Poor                 & ---       & --- &    10 fobs    & --- & ---  & PHB 158 \\
        Modest               & ---       & --- &    50 fobs    & --- & ---  & PHB 158 \\
        Comfortable          & ---       & --- &    80 fobs    & --- & ---  & PHB 158 \\
        Wealthy              & ---       & --- &     2 agnomas & --- & ---  & PHB 158 \\
        Aristocratic         & ---       & --- &     4 agnomas & --- & ---  & PHB 158 \\
        \multicolumn{7}{l}{\hspace{0.5cm}\textit{Services}} \\
        Coach cab between towns (per km) & --- & --- &  2 fobs    & --- & --- & PHB 159 \\
        Coach cab within a city          & --- & --- &  1 fob     & --- & --- & PHB 159 \\
        Skilled hireling (per day)       & --- & --- &  2 agnomas & --- & --- & PHB 159 \\
        Untrained Hireling (per day)     & --- & --- & 20 fobs    & --- & --- & PHB 159 \\
        Messenger (per km)               & --- & --- &  1 fob     & --- & --- & PHB 159 \\
        Road or gate toll                & --- & --- &  1 fob     & --- & --- & PHB 159 \\
        Ship's passage (per km)          & --- & --- &  5 fobs    & --- & --- & PHB 159 \\
        \multicolumn{7}{l}{\hspace{0.5cm}\textit{Food (per day)}} \\
        Animal Feed          & Mundane   & 1 &       5 fobs    & COO & 5 kg & PHB 157 \\
        Squalid              & Mundane   &  1  &     3 fobs    & COO & ---  & PHB 158 \\
        Poor                 & Mundane   &  1  &     6 fobs    & COO & ---  & PHB 158 \\
        Modest               & Mundane   &  1  &    30 fobs    & COO & ---  & PHB 158 \\
        Comfortable          & Mundane   &  1  &    50 fobs    & COO & ---  & PHB 158 \\
        Wealthy              & Mundane   &  1  &    80 fobs    & COO & ---  & PHB 158 \\
        Aristocratic         & Plain     &  1  &     2 agnomas & COO & ---  & PHB 158 \\
        Banquet (per person) & Common    & 1   &    10 agnomas & COO & ---  & PHB 158 \\
    \end{DndTable}
\end{table*}

\subsection*{Lifestyle Expenses}
    At the beginning of a week or month in a town or city, you choose a lifestyle choice and pay the price to sustain that lifestyle.
    The prices are listed per day, so multiply the cost by 6 for the cost of one week or by 24 for the cost of one month.
    % Your lifestyle might change from one period to the next, based on the funds you have at your disposal, or you might maintain the same lifestyle throughout your character's career.

    Your lifestyle choice can have consequences.
    Maintaining a wealthy lifestyle might help you make contacts with the rich and powerful, though you run the risk of attracting thieves.
    Likewise, living frugally might help you avoid criminals, but you are unlikely to make powerful connections.

    \paragraph{Wretched}
        You live in inhumane conditions.
        With no place to call home, you shelter wherever you can, sneaking into barns, huddling in old crates, and relying on the good graces of people better off than you.
        % A wretched lifestyle presents abundant dangers.
        % Violence, disease, and hunger follow you wherever you go.
        % Other wretched people covet your armor, weapons, and adventuring gear, which represent a fortune by their standards.
        % You are beneath the notice of most people.
    \paragraph{Squalid}
        You live in a leaky stable, a mud-floored hut just outside town, or a vermin-infested boarding house in the worst part of town.
        You have shelter from the elements, but you live in a desperate and often violent environment, in places rife with disease, hunger, and misfortune.
        % You are beneath the notice of most people, and you have few legal protections.
        % Most people at this lifestyle level have suffered some terrible setback.
        % They might be disturbed, marked as exiles, or suffer from disease.
    \paragraph{Poor}
        A poor lifestyle means going without the comforts available in a stable community.
        Simple food and lodgings, threadbare clothing, and unpredictable conditions result in a sufficient, though probably unpleasant, experience.
        Your accommodations might be a room in a flophouse or in the common room above a tavern.
        % You benefit from some legal protections, but you still have to contend with violence, crime, and disease.
        % People at this lifestyle level tend to be unskilled laborers, costermongers, peddlers, thieves, mercenaries, and other disreputable types.
    \paragraph{Modest}
        A modest lifestyle keeps you out of the slums and ensures that you can maintain your equipment.
        You live in an older part of town, renting a room in a boarding house, inn, or temple.
        You don't go hungry or thirsty, and your living conditions are clean, if simple.
        % Ordinary people living modest lifestyles include soldiers with families, laborers, students, priests, and the like.
    \paragraph{Comfortable}
        Choosing a comfortable lifestyle means that you can afford nicer clothing and can easily maintain your equipment.
        You live in a small cottage in a middle-class neighborhood or in a private room at a fine inn.
        You associate with merchants, skilled tradespeople, and military officers.
    \paragraph{Wealthy}
        Choosing a wealthy lifestyle means living a life of luxury, though you might not have achieved the social status associated with the old money of nobility or royalty.
        You live a lifestyle comparable to that of a highly successful merchant, a favored servant of the royalty, or the owner of a few small businesses.
        You have respectable lodgings, usually a spacious home in a good part of town or a comfortable suite at a fine inn.
        % You likely have a small staff of servants.
    \paragraph{Aristocratic}
        You live a life of plenty and comfort.
        You move in circles populated by the most powerful people in the community.
        You have excellent lodgings, perhaps a townhouse in the nicest part of town or rooms in the finest inn.
        % You dine at the best restaurants, retain the most skilled and fashionable tailor, and have servants attending to your every need.
        % You receive invitations to the social gatherings of the rich and powerful, and spend evenings in the company of politicians, guild leaders, high priests, and nobility.
        % You must also contend with the highest levels of deceit and treachery.

    % The wealthier you are, the greater the chance you will be drawn into political intrigue as a pawn or participant.

    % !TEX root = ../main.tex
\section{Meals} \label{sec::meals}
    Unless otherwise specified, the prices listed for meals are for a feast, which feeds up to 8 creatures.
    Divide this by 8 for the price of one meal.

    A meal is consumed during the day, and its benefits (if any) last for the whole day.
    After cooked, the benefits associated to a meal last for one day, after which the food spoils.
    You can only be affected by one meal at the same time.

    \begin{table*}[b]%
        \begin{DndTable}[width=\linewidth, header=Meals]{Xlccccc}
            \textbf{Meal} & \textbf{Rarity} & \textbf{Mats.} & \textbf{Total Cost} & \textbf{Tools} & \textbf{Weight} & \textbf{Source} \\
            Loaf of Bread           & Mundane   & 1 &      8 fobs    & COO & ---  & PHB   158 \\
            Bead of Nourishment     & Plain     & 1 &      1 agnoma  & COO & ---  & XGE   136 \\
            Bead of Refreshment     & Plain     & 1 &      1 agnoma  & COO & ---  & XGE   136 \\
            Dancing Monkey Fruit    & Plain     & 3 &      5 agnomas & COO & ---  & TOA   205 \\
            Essence of Dreamlily    & Plain     & 1 &      2 agnomas & COO & ---  & ERLW  244 \\
            Explosive Seed          & Plain     & 3 &      5 agnomas & COO & ---  & EGW   225 \\
            Hunk of Cheese          & Plain     & 1 &     80 fobs    & COO & ---  & PHB   158 \\
            Menga Leaves            & Plain     & 1 &      2 agnomas & COO & ---  & TOA   205 \\
            Rations (1 meal)        & Plain     & 1 &     50 fobs    & COO & 1 kg & PHB   153 \\
            Sinda Berries           & Plain     & 3 &      5 agnomas & COO & ---  & TOA   205 \\
            Chunk of Meat           & Common    & 1 &     24 agnomas & COO & ---  & PHB   158 \\
            Olisuba Leaf            & Common    & 1 &     30 agnomas & COO & ---  & EGW    70 \\
            Ryath Root              & Common    & 1 &     30 agnomas & COO & ---  & TOA   205 \\
            Wildroot                & Common    & 1 &     25 agnomas & COO & ---  & TOA   205 \\
            Zabou                   & Common    & 1 &     10 agnomas & COO & ---  & TOA   205 \\
            Black Sap               & Uncommon  & 4 &    300 agnomas & COO & ---  & EGW   152 \\
            Blight Ichor            & Uncommon  & 2 &    200 agnomas & COO & ---  & EGW   152 \\
            Dust of Deliciousness   & Uncommon  & 4 &    300 agnomas & COO & ---  & EGW   267 \\
            Dust of Sneezing        & Uncommon  & 1 &    150 agnomas & COO & ---  & DMG   166 \\
            Guardian Broth          & Uncommon  & 1 &    150 agnomas & COO & ---  & TCE   128 \\
            Pineapple Honeyed Ham   & Uncommon  & 3 &    250 agnomas & COO & ---  & ---       \\
            Soothsalts              & Uncommon  & 1 &    150 agnomas & COO & ---  & EGW   152 \\
            Thassa's Olive Salad    & Uncommon  & 1 &    150 agnomas & COO & ---  & ---       \\
            Vedbread                & Uncommon  & 1 &    150 agnomas & COO & ---  & ---       \\
            High Harvest Puree      & Rare      & 1 &  1,500 agnomas & COO & ---  & ---       \\
            Krig's Stuffed Potatoes & Rare      & 1 &  1,500 agnomas & COO & ---  & ---       \\
            Rose-scented Truffles   & Rare      & 1 &  1,500 agnomas & COO & ---  & ---       \\
            Wyvern Salmon           & Rare      & 1 &  1,500 agnomas & COO & ---  & ---       \\
            Drakefire Chili         & Very Rare & 1 & 15,000 agnomas & COO & ---  & ---       \\
            Living Armor            & Very Rare & 8 & 50,000 agnomas & COO & ---  & ERLW  278
        \end{DndTable}
    \end{table*}

    \paragraph{Bead of Nourishment}
        This spongy, flavorless, gelatinous bead dissolves on your tongue and provides as much nourishment as 1 day of rations.
    \paragraph{Bead of Refreshment}
        This spongy, flavorless, gelatinous bead dissolves in liquid, transforming up to a pint of the liquid into fresh, cold drinking water.
        The bead has no effect on harmful substances such as poison.
    \paragraph{Black Sap}
        This tarry substance harvested from the dark boughs of the death's head willow is a powerful intoxicant.
        It can be smoked as a concentrate or injected directly into the bloodstream.
        A creature subjected to a dose of black sap cannot be charmed or frightened for 8 hours.
        For each dose of black sap consumed, a creature must succeed on a DC 15 Constitution saving throw or become poisoned for 24 hours---an effect that is cumulative with multiple doses.
    \paragraph{Blight Ichor}
        This bitter chartreuse concoction is distilled from a fungus native to the Blightshore badlands.
        The sickly green liqueur harbors potent psychedelic properties.
        Provided it is neither a construct nor undead, a creature subjected to a dose of blight ichor gains advantage on Intelligence and Wisdom checks, as well as vulnerability to psychic damage, for 1 hour.
        For each dose of blight ichor consumed, the creature must succeed on a DC 15 Constitution saving throw or become poisoned for 24 hours.
        A fiend subjected to a dose of blight ichor gains advantage on all Dexterity checks and is immune to the frightened condition for 1 hour.
    \paragraph{Dancing Monkey Fruit}
        After being draped in honey and fermented for 10 days, this apple produces enough juice to fill a vial.
        Any creature that eats a dancing monkey fruit or drinks its juice must succeed on a DC 14 Constitution saving throw or begin a comic dance that lasts for 1 minute.
        Humanoids that can't be poisoned are immune to this effect.

        The dancer must use one action on each of its turns to dance without leaving its space.
        In addition, it cannot move, has disadvantage on attack rolls and Dexterity saving throws, and other creatures have advantage on attack rolls against it.
        Each time it takes damage, the dancer can repeat the saving throw, ending the effect on itself on a success.
        When the dancing effect ends, the humanoid suffers the poisoned condition for 1 hour.
    \paragraph{Drakefire Chili}
        This chili is made of drake meat mixed with onion, tomatoes, and a generous portion of pepper.
        After eating this meal, a creature gains the ability to discharge jets of energy from its mouth for 8 hours.
        A creature can fire this jet once every 5 minutes.
        The creature firing the jet chooses the effect from the following options:
        \begin{itemize}
            \item \textbf{Acid Jet}.
                The creature discharges acid in a line 60 meters long and 1 meter wide.
                Each creature in that line must make a DC 15 Dexterity saving throw, taking 4d10 acid damage on a failed save, or half as much damage on a successful one.
                In addition, a creature that fails its saving throw takes 2d10 acid damage at the start of each of its turns; a creature can end this damage by using two actions to wash off the acid with a pint or more of water.
            \item \textbf{Fire Jet}.
                The creature discharges fire in a line 60 meters long and 1 meter wide.
                Each creature in the area must make a DC 15 Dexterity saving throw, taking 6d10 fire damage on a failed save, or half as much damage on a successful one.
                The fire ignites any flammable objects in the area that aren't being worn or carried.
            \item \textbf{Frost Shot}.
                The creature discharges a ball of frost to a point it can see within 240 meters.
                The ball then expands to form a 6-meter-radius sphere centered on that point.
                Each creature in that area must make a DC 15 Constitution saving throw.
                On a failed save, a creature takes 4d10 cold damage, and its speed is reduced by 2 meters for 1 minute.
                On a successful save, the creature takes half as much damage, and its speed isn't reduced.
                A creature whose speed is reduced by this effect can repeat the save at the end of each of its turns, ending the effect on itself on a success.
            \item \textbf{Lightning Shot}.
                The creature shoots a ball of lightning to a point it can see within 240 meters.
                The lightning then expands to form a 4-meter-radius sphere centered on that point.
                Each creature in that area must make a DC 15 Dexterity saving throw, taking 6d10 lightning damage on a failed save, or half as much damage on a successful one.
                Creatures wearing metal armor have disadvantage on the save.
            \item \textbf{Poison Spray}.
                The creature expels poison gas in a 12-meter cone.
                Each creature in that area must make a DC 15 Constitution saving throw.
                On a failed save, the creature takes 4d10 poison damage and is poisoned for 1 minute.
                On a successful save, the creature takes half as much damage and isn't poisoned.
                A creature poisoned in this way can repeat the saving throw at the end of each of its turns, ending the effect on itself on a success.
        \end{itemize}
    \paragraph{Dust of Deliciousness}
        This reddish brown dust can be sprinkled over any edible substance to add an earthly flavor to it.
        The dust also dulls the eater's senses: anyone eating food treated with this dust has disadvantage on Wisdom ability checks and Wisdom saving throws for 1 hour.
    \paragraph{Dust of Sneezing}
        Found in a small container, this powder resembles very fine sand and gives a faint smell of turmeric.
        You can either add this dust to up to 8 meals or use an action to throw a handful of it into the air, creating a 12-meter cloud of dust.
        Any creature that consumes or breathes the dust must succeed on a DC 15 Constitution saving throw or become unable to breathe while sneezing uncontrollably.
        A creature affected in this way is incapacitated and suffocating.
        As long as it is conscious, a creature can repeat the saving throw at the end of each of its turns, ending the effect on it on a success.
    \paragraph{Essence of Dreamlily}
        Dreamlily is a tasteless psychoactive liquid that can be added to foods to have them grant additional effects.
        Though dreamlily isn't illegal if used for medicinal purposes, it's heavily taxed, and thus most dreamlily is smuggled in and sold on the black market.
        Consuming dreamlily causes disorienting euphoria and brings about remarkable resistance to pain.

        A creature under the effects of dreamlily is poisoned for 1 hour.
        While poisoned in this way, the creature is immune to fear, and the first time it drops to 0 hit points without being killed outright, it drops to 1 hit point instead.
    \paragraph{Explosive Seed}
        This acorn-sized hazelnut contains a small amount of blasting powder.
        An explosive seed can be thrown up to 6 meters as an action, detonating on impact.
        Each creature within a meter of the exploding seed must make a DC 10 Dexterity saving throw, taking 1d8 bludgeoning damage on a failed save, or half as much damage on a successful one.
    \paragraph{Guardian Broth}
        Accompanied with melted cheeses, this chunky tomato broth is the perfect breakfast to prepare oneself for a day of hardship.
        For 8 hours after eating the broth, you can use your reaction to turn a critical hit against you into a normal hit instead.
		The effect of the meal fades when you use this ability.
    \paragraph{High Harvest Puree}
        Simply the smell of this squash and garlic puree is enough to brighten any adventurer's day.
        For 8 hours after eating this meal, you are inmune to harmful gases such as those created by the \textbf{Cloudkill} (page \pageref{spell::cloudkill}) spell, inhaled poisons, etc.
        In addition, you can exhale a gust of wind for the duration, as if you had cast the \textbf{Gust} spell (see page \pageref{spell::gust}).
    \paragraph{Krig's Stuffed Potatoes}
        These potatoes are filled with a combination of meat, herbs, and spices, to be then covered in a thick fabric and cooked in an oven.
        Upon eating it, a creature gains an effect of their choice for 8 hours.
        The available effects are:
        \begin{itemize}
            \item \textbf{Mark of Courage}.
                The creature is immune to the frightened condition.
            \item \textbf{Mark of the Sentinel}.
                The creature can see invisible creatures while they are within 4 meters of it and within its line of sight.
            \item \textbf{Mark of the Shield}.
                As an action, the creature can activate the effects of the meal.
                For a minute, it gains a +1 bonus to its AC.
            \item \textbf{Mark of the Sword}.
                As an action, the creature can activate the effects of the meal.
                For a minute, it gains a +1 bonus to its attack rolls.
        \end{itemize}
    \paragraph{Living Armor $\odot$} \label{item::livingarmor}
        This hideous ooze is formed from black chitin, beneath which veins pulse and red sinews glisten.
        To attune to this item, you must consume it as a meal, which is enough food for the day.
        After eaten, tendrils burrow and reach from inside you, merging with your armor.
        While you remain attuned to this item, you have a +1 bonus to your AC, and you have resistance to necrotic, poison, and psychic damage.

        \textbf{Symbiotic Nature}.
        The armor can't be removed from you while you're attuned to it.
        The only way to end your attunement to the armor is by drinking an oozekiller vial (see page \pageref{item::oozekiller}).
        The armor requires fresh blood be fed to it.
        Immediately after you finish any long rest, you must either feed half of your remaining Hit Dice to the armor (round up) or take 1 level of exhaustion.
    \paragraph{Menga Leaves}
        The dried leaves of a menga bush can be ground, dissolved in a liquid, heated, and ingested.
        A creature that ingests 1 ounce of menga leaves in this fashion regains 1 hit point.
        A creature that ingests more than 5 ounces of menga leaves in a 24-hour period gains no additional benefit and must succeed on a DC 11 Constitution saving throw or fall unconscious for 1 hour.
        The unconscious creature awakens if it takes at least 5 damage on one turn.
        A healthy menga bush usually has 2d6 ounces of leaves.
        Once picked, the leaves require 1 day to dry out before they can confer any benefit.
    \paragraph{Olisuba Leaf}
        These dried leaves of the Olisuba tree, when steeped to make a tea, can help a body recover from strenuous activity.
        If you drink a dose of Olisuba tea during a short rest, your exhaustion level is reduced by 2 instead of 1 at the end of that long rest.
    \paragraph{Pineapple Honeyed Ham}
        This cooked ham can be eaten as a side dish or over bread and pineaple gravy as a full meal.
        After eating this meal, a creature gains a +1 bonus to saving throws for 8 hours.
    \paragraph{Rose-scented Truffles}
        These truffles are mixed with equal portions of honey and cocoa.
        Upon serving, they are bathed in an odorous rose syrup, giving them their characteristic fragance.
        The syrup comes in one of five colors, decided by the chef upon cooking.
        For 8 hours after eating a truffle, a creature gains resistance to damage of the type associated with the syrup.
        If the creature would take more than 10 damage of this type from a single source (after applying the resistance), the effect is lost.

        During its turn after losing the effect, the creature can use an action to blow out a rose-smelling breath.
        The creature produces a 4-meter cone of acid, lightning, poison, fire, or cold, as dictated by the syrup's damage type.
        Each creature in the cone must make a DC 15 Constitution saving throw, taking 3d10 damage of the appropiate type on a failed save, or half as much on a successful one.

        \begin{DndTable}[width=\linewidth, header=Chromatic Roses]{ll}
            \textbf{Item} & \textbf{Damage Type} \\
            Black Rose    & Acid                 \\
            Blue Rose     & Lightning            \\
            Green Rose    & Poison               \\
            Red Rose      & Fire                 \\
            White Rose    & Cold
        \end{DndTable}
    \paragraph{Ryath Root}
        Any creature that ingests a dried ryath root gains 2d4 temporary hit points.
        A creature that consumes more than one ryath root in a 24-hour period must succeed on a DC 13 Constitution saving throw or suffer the poisoned condition for 1 hour.
    \paragraph{Sinda Berries}
		These berries are dark brown and bitter.
        A full-grown sinda berry bush has 4d6 berries growing on it.
        A bush plucked of all its berries grows new berries in 1d4 months.
        Picked berries lose their freshness and efficacy after 24 hours.
		Fresh sinda berries can be eaten raw or crushed and added to a drink to dull the bitterness.
        A creature that consumes at least ten fresh sinda berries gains advantage on saving throws against disease and poison for the next 24 hours.
    \paragraph{Soothsalts}
		Soothsalts are derived from a naturally occurring crystalline substance discovered throughout the wilds of the Katajthon canyon.
        The crimson crystals have been mined from cavernous veins and found within smaller geode formations.
        Soothsalts are consumed orally in lozenge-sized doses, and frequent users can be identified by the telltale crimson stain around their mouths.
        A creature subjected to a dose of soothsalts gains advantage on all Intelligence checks for 8 hours.
		For each dose of soothsalts consumed, the creature must succeed on a DC 15 Constitution saving throw or gain one level of exhaustion---an effect which is cumulative with multiple doses.
    \paragraph{Thassa's Olive Salad}
        This simple mediterranean salad combines the flavors of sea greens, pomegranate molasses, green olives, and grape tomatoes.
        After mixing, it is infused with a blessing from Thassa via its traditional black pepper and sea salt seasoning.
        After eating it, a creature gains the ability to breathe normally underwater for 8 hours.
    \paragraph{Vedbread}
        The next time you see a creature within 10 minutes after eating this mushroom bun, you become charmed by that creature for 1 hour.
    \paragraph{Wildroot}
        Introducing the juice of a wildroot into a poisoned creature's bloodstream (for example, by rubbing it on an open wound) rids the creature of the poisoned condition.
        Once used in this way, a wildroot loses this property.
    \paragraph{Wyvern Salmon}
        Cooked with shallot, thyme, and red wine, this salmon is the stuff of legends.
        After eating this meal, a creature is cured of any poison affecting it, becomes inmune to poison, and makes all Wisdom saving throws with advantage.
        In addition, its hit point maximum also increases by 2d10, and it gains the same number of hit points.
        These benefits last for 8 hours.
    \paragraph{Zabou}
        Zabou mushrooms feed on offal and the rotting wood of dead trees.
        If handled carefully, a zabou can be picked or uprooted without causing it to release its spores.
        If crushed or struck, a zabou releases its spores in a 2-meter-radius sphere.
        A zabou can also be hurled up to 6 meter away or dropped like a grenade, releasing its cloud of spores on impact.
        Any creature in that area must succeed on a DC 10 Constitution saving throw or be poisoned for 1 minute.
        The poisoned creature's skin itches for the duration.
        The creature can repeat the saving throw at the end of each of its turns, ending the effect on itself on a success."

\newpage~\newpage

    % !TEX root = ../main.tex
\section{Mounts} \label{sec::mounts}
    Mounting or dismounting a creature requires two actions.

    \begin{table*}[b]%
        \begin{DndTable}[width=\linewidth, header=Mounts]{Xlcccc}
            \textbf{Mount} & \textbf{Rarity} & \textbf{Speed} & \textbf{Carrying Capacity} & \textbf{Total Cost} & \textbf{Source} \\
            Axebeak       & Common   &  8 mt & 210 kg &  50 agnomas & IDRotF  20 \\
            Beakdog       & Common   & 10 mt & 100 kg &  50 agnomas & ---        \\
            Camel         & Common   & 10 mt & 240 kg &  50 agnomas & PHB    157 \\
            Donkey        & Common   &  8 mt & 210 kg &   8 agnomas & PHB    157 \\
            Draft Horse   & Common   &  8 mt & 270 kg &  25 agnomas & PHB    157 \\
            Mastiff       & Common   &  8 mt & 100 kg &  25 agnomas & PHB    157 \\
            Mule          & Common   &  8 mt & 210 kg &   8 agnomas & PHB    157 \\
            Pony          & Common   &  8 mt & 110 kg &  30 agnomas & PHB    157 \\
            Riding Horse  & Common   & 12 mt & 240 kg &  75 agnomas & PHB    157 \\
            Sled Dog      & Common   & 12 mt & 180 kg &  50 agnomas & IDRotF  20 \\
            Elephant      & Uncommon &  8 mt & 660 kg & 200 agnomas & PHB    157 \\
            Elk Bird      & Uncommon & 12 mt & 150 kg & 250 agnomas & ---
            % TODO. Drake
            % TODO. Skyjek roc
            % TODO. Wyvern
        \end{DndTable}
    \end{table*}

    % TODO. Add ``bestiary'' entries for all of these.
\newpage~\newpage

    % !TEX root = ../main.tex
\section{Poisons} \label{sec::poisons}
    Poisons are small amounts of dangerous liquids contained in a vial.
    Poisons have varied methods of application, which is the medium needed to administer the poison to a victim.

    Using two actions, you can coat one slashing or piercing weapon or three pieces of ammunition with poison.
    Poison can remain effective in a weapon for 1 minute after coating or until the weapon strucks a target three times.

    % TODO. Upgradable feat with poisoner's kit: improve basic poison.
    % TODO. Feat with poisoner's kit: change a poison's method of application.
    % TODO. Feat with poisoner's kit: three applications of potions on melee weapons before fading instead of one.
    \begin{table*}[b]%
        \begin{DndTable}[width=\linewidth, header=Poisons]{Xccccccc}
            \textbf{Poison} & \textbf{Rarity} & \textbf{Method} & \textbf{Mats.} & \textbf{Total Cost} & \textbf{Tools} & \textbf{Weight} & \textbf{Source} \\
            Basic Poison          & Common    & Injury   &  8 &    100 agnomas & POI & --- & PHB 153 \\
            Assassin's Blood      & Uncommon  & Ingested &  1 &    150 agnomas & POI & --- & DMG 258 \\
            Burnt Othur Fumes     & Uncommon  & Inhaled  &  8 &    500 agnomas & POI & --- & DMG 258 \\
            Carrion Crawler Mucus & Uncommon  & Contact  &  3 &    200 agnomas & POI & --- & DMG 258 \\
            Essence of Ether      & Uncommon  & Inhaled  &  5 &    300 agnomas & POI & --- & DMG 258 \\
            Sprouting Poison      & Uncommon  & Injury   &  3 &    200 agnomas & POI & --- & DMG 258 \\
            Malice                & Uncommon  & Inhaled  &  4 &    250 agnomas & POI & --- & DMG 258 \\
            Oil of Taggit         & Uncommon  & Contact  &  6 &    400 agnomas & POI & --- & DMG 258 \\
            Pale Tincture         & Uncommon  & Ingested &  4 &    250 agnomas & POI & --- & DMG 258 \\
            Serpent Venom         & Uncommon  & Injury   &  3 &    200 agnomas & POI & --- & DMG 258 \\
            Torpor                & Uncommon  & Ingested & 10 &    600 agnomas & POI & --- & DMG 258 \\
            Truth Serum           & Uncommon  & Ingested &  1 &    150 agnomas & POI & --- & DMG 258 \\
            Midnight Tears        & Rare      & Ingested &  1 &  1,500 agnomas & POI & --- & DMG 258 \\
            Nidhogg Poison        & Rare      & Injury   &  2 &  2,000 agnomas & POI & --- & DMG 258 \\
            Wyvern Poison         & Rare      & Injury   &  1 &  1,200 agnomas & POI & --- & DMG 258 \\
            Erebos' Whisper       & Very Rare & Inhaled  &  2 & 20,000 agnomas & POI & --- & VRGR 83
        \end{DndTable}
    \end{table*}

    \paragraph{Assassin's Poison}
        A creature subjected to this poison must make a DC 10 Constitution saving throw.
        On a failed save, it takes 6 (1d12) poison damage and is poisoned for 24 hours.
        On a successful save, the creature takes half damage and isn't poisoned.
    \paragraph{Basic Poison} \label{item::basicpoison}
        You can use the poison in this vial to coat one slashing or piercing weapon or up to three pieces of ammunition.
        Applying the poison takes two actions.
        A creature hit by the poisoned weapon or ammunition must make a DC 10 Constitution saving throw or take 1d4 poison damage.
        Once applied, the poison retains potency for 1 minute before drying.
    \paragraph{Burnt Othur Fumes}
        A creature subjected to this poison must succeed on a DC 13 Constitution saving throw or take 10 (3d6) poison damage, and must repeat the saving throw at the start of each of its turns.
        On each successive failed save, the character takes 3 (1d6) poison damage.
        After three successful saves, the poison ends.
    \paragraph{Carrion Crawler Mucus}
        This poison must be harvested from a dead or incapacitated carrion crawler.
        A creature subjected to this poison must succeed on a DC 13 Constitution saving throw or be poisoned for 1 minute.
        The poisoned creature is paralyzed.
        The creature can repeat the saving throw at the end of each of its turns, ending the effect on itself on a success.
    \paragraph{Essence of Ether}
        A creature subjected to this poison must succeed on a DC 15 Constitution saving throw or become poisoned for 8 hours.
        The poisoned creature is unconscious.
        The creature wakes up if it takes damage or if another creature takes two actions to shake it awake.
    \paragraph{Erebos' Whisper}
        A creature subjected to this poison must succeed on a DC 18 Constitution saving throw or experience a terrible nightmare the next time they sleep.
        Echoes of a phantasmal monstrosity spawn a nightmare that lasts the duration of the target's sleep and prevents the target from gaining any benefit from that rest.
        In addition, when the target wakes up, it takes 3d6 psychic damage.
    \paragraph{Sprouting Poison}
        This poison is typically made only in a place far removed from sunlight.
        A creature subjected to this poison must succeed on a DC 13 Constitution saving throw or be poisoned for 1 hour.
        If the saving throw fails by 5 or more, the creature is also unconscious while poisoned in this way.
        The creature wakes up if it takes damage or if another creature takes two actions to shake it awake.
    \paragraph{Malice}
        A creature subjected to this poison must succeed on a DC 15 Constitution saving throw or become poisoned for 1 hour.
        The poisoned creature is blinded.
    \paragraph{Midnight Tears}
        A creature that ingests this poison suffers no effect until the stroke of midnight.
        If the poison has not been neutralized before then, the creature must succeed on a DC 17 Constitution saving throw, taking 31 (9d6) poison damage on a failed save, or half as much damage on a successful one.
    \paragraph{Oil of Taggit}
        A creature subjected to this poison must succeed on a DC 13 Constitution saving throw or become poisoned for 24 hours.
        The poisoned creature is unconscious.
        The creature wakes up if it takes damage.
    \paragraph{Pale Tincture}
        A creature subjected to this poison must succeed on a DC 16 Constitution saving throw or take 3 (1d6) poison damage and become poisoned.
        The poisoned creature must repeat the saving throw every 24 hours, taking 3 (1d6) poison damage on a failed save.
        Until this poison ends, the damage the poison deals can't be healed by any means.
        After seven successful saving throws, the effect ends and the creature can heal normally.
    \paragraph{Serpent Venom}
        This poison must be harvested from a dead or incapacitated poisonous snake.
        A creature subjected to this poison must succeed on a DC 11 Constitution saving throw, taking 10 (3d6) poison damage on a failed save, or half as much damage on a successful one.
    \paragraph{Torpor}
        A creature subjected to this poison must succeed on a DC 15 Constitution saving throw or become poisoned for 4d6 hours.
        The poisoned creature is incapacitated.
    \paragraph{Truth Serum}
        A creature subjected to this poison must succeed on a DC 11 Constitution saving throw or become poisoned for 1 hour.
        The poisoned creature can't knowingly speak a lie.
    \paragraph{Wyvern Poison}
        This poison must be harvested from a dead or incapacitated wyvern.
        A creature subjected to this poison must make a DC 15 Constitution saving throw, taking 24 (7d6) poison damage on a failed save, or half as much damage on a successful one.
\newpage~\newpage

    % !TEX root = ../main.tex
\section{Potions} \label{sec::potions}
    Potions are small amounts of potent liquid contained in a flask or vial.
    A creature can drink a potion using two actions, or it can throw the container up to 4/12 meters as an action.
    This is considered a ranged weapon and any feat that applies to thrown weapons applies to it.

    A flask contains the same amount of liquid as 4 vials.
    \begin{table*}[b]%
        \begin{DndTable}[width=\linewidth, header=Potions]{Xlccccc}
            \textbf{Potion} & \textbf{Rarity} & \textbf{Mats.} & \textbf{Cost} & \textbf{Tools} & \textbf{Weight} & \textbf{Source} \\
            Flask                              & Mundane   & 1 &      15 fobs    & GLA       & 0.5 kg. & PHB 153 \\
            Oil (Flask)                        & Mundane   & 1 &      15 fobs    & ALC       & 0.5 kg. & PHB 152 \\
            Soap                               & Mundane   & 1 &      15 fobs    & ALC       & ---     & PHB 150 \\
            Perfume (Vial)                     & Plain     & 3 &       5 agnomas & ALC       & ---     & PHB 150 \\
            Acid (Vial)                        & Common    & 1 &      30 agnomas & ALC       & ---     & PHB 148 \\
            Alchemist's Fire (Flask)           & Common    & 1 &      30 agnomas & ALC       & 0.5 kg. & PHB 148 \\
            Antitoxin (Vial)                   & Common    & 1 &      30 agnomas & ALC       & ---     & PHB 151 \\
            Caustic Brew (Flask)               & Common    & 1 &      30 agnomas & ALC       & 0.5 kg. & --- \\
            Potion of Healing (Flask)          & Common    & 1 &      30 agnomas & ALC       & 0.5 kg. & DMG 187 \\
            Potion of Sickness (Flask)         & Common    & 1 &      30 agnomas & ALC       & 0.5 kg. & --- \\
            Vial                               & Common    & 1 &       3 agnomas & GLA       & ---     & PHB 153 \\
            Bloodwell Vial +1                  & Uncommon  & 8 &     500 agnomas & ALC + GLA & ---     & TCE 122 \\
            Bottled Acid Arrow (Flask)         & Uncommon  & 1 &     150 agnomas & ALC       & 0.5 kg. & --- \\
            Coldblood (Vial)                   & Uncommon  &$\ast$ & 500 agnomas & ALC       & ---     & --- \\
            Oil of Slipperiness (Vial)         & Uncommon  & 1 &     150 agnomas & ALC       & ---     & DMG 184 \\
            Potion of Greater Healing (Flask)  & Uncommon  & 1 &     150 agnomas & ALC       & 0.5 kg. & DMG 187 \\
            Potion of Resistance (Flask)       & Uncommon  & 1 &     150 agnomas & ALC       & 0.5 kg. & DMG 188 \\
            Bloodwell Vial +2                  & Rare      & 8 &   5,000 agnomas & ALC + GLA & ---     & TCE 122 \\
            Bottled Fire (Flask)               & Rare      & 1 &   1,500 agnomas & ALC       & 0.5 kg. & --- \\
            Elixir of Health (Flask)           & Rare      & 1 &   1,500 agnomas & ALC       & 0.5 kg. & DMG 168 \\
            Flask of Mass Healing (Flask)      & Rare      & 1 &   1,500 agnomas & ALC       & 0.5 kg. & --- \\
            Oil of Burning (Vial)              & Rare      & 1 &   1,500 agnomas & ALC       & ---     & --- \\
            Potion of Aqueous Form (Vial)      & Rare      & 1 &   1,500 agnomas & ALC       & ---     & MOT 197 \\
            Potion of Superior Healing (Flask) & Rare      & 1 &   1,500 agnomas & ALC       & 0.5 kg. & DMG 187 \\
            Bloodwell Vial +3                  & Very Rare & 8 &  50,000 agnomas & ALC + GLA & ---     & TCE 122 \\
            Bottled Blight (Vial)              & Very Rare & 1 &  15,000 agnomas & ALC       & ---     & --- \\
            Flask of Weakness (Flask)          & Very Rare & 1 &  15,000 agnomas & ALC       & 0.5 kg. & --- \\
            Healing Cloud (Flask)              & Very Rare & 1 &  15,000 agnomas & ALC       & 0.5 kg. & --- \\
            Liquid Death (Flask)               & Very Rare & 1 &  15,000 agnomas & ALC       & 0.5 kg. & --- \\
            Oil of Immolation (Vial)           & Very Rare & 1 &  15,000 agnomas & ALC       & ---     & --- \\
            Oil of Sharpness (Vial)            & Very Rare & 1 &  15,000 agnomas & ALC       & ---     & DMG 184 \\
            Oozekiller (Vial)                  & Very Rare & 1 &  15,000 agnomas & ALC       & ---     & --- \\
            Potion of Contagion (Flask)        & Very Rare & 1 &  15,000 agnomas & ALC       & 0.5 kg. & --- \\
            Potion of Supreme Healing (Flask)  & Very Rare & 1 &  15,000 agnomas & ALC       & 0.5 kg. & DMG 187 \\
            Bottled Tizerus (Flask)            & Legendary & 1 &  75,000 agnomas & ALC       & 0.5 kg. & --- \\
            Potion of Condensed Life (Flask)   & Legendary & 1 &  75,000 agnomas & ALC       & 0.5 kg. & --- \\
            Potion of Regeneration (Vial)      & Legendary & 1 &  75,000 agnomas & ALC       & ---     & --- \\
            Universal Solvent (Vial)           & Legendary & 8 & 250,000 agnomas & ALC       & ---     & --- \\
            Vial of Pain (Vial)                & Legendary & 1 &  75,000 agnomas & ALC       & ---     & ---
        \end{DndTable}
    \end{table*}
    $\ast$ Coldblood is not commonly sold, and can only be gathered from a live uman using Alchemist's supplies.

    \paragraph{Acid (Vial)}
        As an action, you can splash the contents of this vial onto a creature within 1 meter of you or throw the vial up to 4/12 meters, shattering it on impact.
        In either case, make a ranged attack against a creature or object, treating the acid as an improvised weapon.
        On a hit, the target takes 2d6 acid damage.
    \paragraph{Alchemist's Fire (Flask)}
        This sticky, adhesive fluid ignites when exposed to air.
        As an action, you can throw this flask up to 4/12 meters, shattering it on impact.
        Make a ranged attack against a creature or object, treating the alchemist's fire as an improvised weapon.
        On a hit, the target takes 1d4 fire damage at the start of each of its turns.
        A creature can end this damage by using two actions to make a DC 10 Dexterity check to extinguish the flames.
    \paragraph{Antitoxin (Vial)}
        A creature that drinks this vial of liquid gains advantage on saving throws against poison for 1 hour.
        It confers no benefit to undead or constructs.
    \paragraph{Bloodwell Vial $\odot$}
        To attune to this vial, you must fill it with coldblood and add a few drops of your own blood.
        As long as this vial is close to you, you gain a bonus to spell attack rolls and to the saving throw DCs of your spells as specified on the item's name.

        This vial has 100 charges.
        Every time a spell's effect is improved by this vial, you expend one charge.
        When all these charges are expended, your attunement with the item is lost.
    \paragraph{Bottled Acid Arrow (Flask)} % Melf's acid arrow.
        Upon being opened, a shimmering green arrow streaks from the bottle opening in a line and bursts in a spray of acid.
        Make attack against a target within 18 meters.
        On a hit, the target takes 4d4 acid damage immediately and 2d4 acid damage at the end of its next turn.
        On a miss, the arrow splashes the target with acid for half as much of the initial damage and no damage at the end of its next turn.
    \paragraph{Bottled Blight (Vial)} % 5th-level Blight.
        Upon being opened, this bottle releases a sickening gas that drains moisture and vitality.
        The gas fills a 1-meter square before dispersing.

        Any creature standing in this square must make a DC 16 Constitution saving throw.
        It takes 9d8 necrotic damage on a failed save, or half as much damage on a successful one.
        This potion has no effect on undead or constructs.

        If you target a plant creature or a magical plant, it makes the saving throw with disadvantage, and the potion deals maximum damage to it.

        If you target a nonmagical plant that isn't a creature, such as a tree or shrub, it doesn't make a saving throw, it simply withers and dies.
    \paragraph{Bottled Fire (Flask)} % Fireball.
        Upon breaking, a bright spark blossoms with a low roar into an explosion of flame.
        Each creature in a 4-meter-radius sphere centered on that point must make a Dexterity saving throw with DC 14.
        A target takes 8d6 fire damage on a failed save, or half as much damage on a successful one.

        The fire spreads around corners.
        It ignites flammable objects in the area that aren't being worn or carried.
    \paragraph{Bottled Tizerus (Flask)} % Delayed Blast Fireball
        This flask contains a tiny spark, patiently waiting to be awakened.
        You can shake the bottle to activate this spark.
        For the following minute, the bead continues to grow ever more violent.
        Upon breaking or when the minute passes, it blossoms with a low roar into an explosion of flame that spreads around corners.
        Each creature in a 4-meter radius sphere centered on that point must make a DC 18 Dexterity saving throw.
        A creature takes fire damage equal to the total accumulated damage on a failed save, or half as much damage on a successful one.

        The potion's base damage is 12d6.
        If at the end of your turn the bead has not yet detonated, the damage increases by 1d6.

        The glowing bead can be taken from the flask without causing it to explode.
        The creature touching it must make a DC 12 Dexterity saving throw.
        On a failed save, the spark erupts in flame.
        On a successful save, the creature can throw the bead up to 8 meters.
        When it strikes a creature or a solid object, the bead explodes.

        The fire damages objects in the area and ignites flammable objects that aren't being worn or carried.
    \paragraph{Caustic Brew (Flask)} % Tasha's Caustic Brew
        Upon being shaken and breaking, this flask explodes in a radius 2 meters.
        Each creature in the cloud must succeed on a DC 10 Dexterity saving throw or be covered in acid for a minute or until a creature uses two actions to scrape or wash the acid off itself or another creature.
        A creature covered in the acid takes 2d4 acid damage at start of each of its turns.
    \paragraph{Elixir of Health (Flask)}
        When you drink this potion, it cures any disease afflicting you, and it removes the blinded, deafened, paralyzed, and poisoned conditions.
        The clear red liquid has tiny bubbles of light in it.
    \paragraph{Flask of Mass Healing (Flask)} % Mass healing word.
        Upon breaking, all creatures within a 4-meter radius regain hit points equal to 1d4 + 3.
        This potion has no effect on undead or constructs.
    \paragraph{Flask of Weakness (Flask)} % Elemental Bane.
        This flask comes in 5 flavors: acid, cold, fire, lightning, and thunder.
        Upon being opened, this flask releases a noxious gas in a 10-meter square.
        Any creature who breathes the gas must succeed on a DC 16 Constitution saving throw or be affected by a weakening effect for a minute.
        The first time each turn the affected creature takes damage of the potion's type, the target takes an extra 2d6 damage of the type.
        Moreover, the target loses any resistance to that damage until the effect ends.
    \paragraph{Healing Cloud (Flask)} % Mass Cure Wounds.
        Upon being opened, this potion releases a wave of healing energy in the form of a flowery smell.
        Any creature standing in a 6-meter radius sphere centered on the potion's location regains hit points equal to 3d8 + 4.
        This potion has no effect on undead or constructs.
    \paragraph{Liquid Death (Flask)} % Cloudkill.
        Upon breaking, this flask releases a 6-meter radius sphere of poisonous, yellow-green fog.
        The fog spreads around corners.
        It lasts for 10 minutes or until strong wind disperses the fog, ending the effect.
        Its area is heavily obscured.

        When a creature enters the fog's area for the first time on a turn or starts its turn there, that creature must make a DC 16 Constitution saving throw.
        The creature takes 5d8 poison damage on a failed save, or half as much damage on a successful one.
        Creatures are affected even if they hold their breath or don't need to breathe.

        % The fog moves 3 meters away from you at the start of each of your turns, rolling along the surface of the ground.
        % The vapors, being heavier than air, sink to the lowest level of the land, even pouring down openings.
    \paragraph{Oil (Flask)}
        As an action, you can splash the oil in this flask onto a creature within 1 meter of you or throw it up to 4 meters, shattering it on impact.
        Make a ranged attack against a target creature or object, treating the oil as an improvised weapon.
        On a hit, the target is covered in oil.
        The oil dries after one minute, and for this duration the target gains vulnerability to fire damage.
        You can also pour a flask of oil on the ground to cover a 1.5-meter-square area, provided that the surface is level.
        If lit, the oil burns for 2 rounds and deals 5 fire damage to any creature that enters the area or ends its turn in the area.
        A creature can take this damage only once per turn.
    \paragraph{Oil of Burning (Vial)} % Flame arrows.
        This brown sticky oil quickly heats up when accelerated.
        By drenching an arrow or bolt in the oil as an action, the piece of ammunition gains an additional 1d6 fire damage on hit.
        The oil exaporates on a piece of ammunition when it hits or misses.
        Using two actions, you can pour the liquid into up to twelve pieces of ammunition at once.
    \paragraph{Oil of Immolation (Vial)} % Immolation.
        Upon being opened, the vial explodes into all consuming flames in a 1-meter radius sphere.
        Any creature within this sphere is wreathed by flames.
        It must make a DC 16 Dexterity saving throw.
        It takes 8d6 fire damage on a failed save, or half as much damage on a successful one.
        On a failed save, the target also burns for 1 minute.
        The burning target sheds bright light in a 9-meter radius and dim light for an additional 6 meters.
        At the end of each of its turns, the target repeats the saving throw.
        It takes 4d6 fire damage on a failed save, and the effect ends on a successful one.
        These flames can't be extinguished by other means.

        If damage from this spell kills a target, the target is turned to ash.
    \paragraph{Oil of Sharpness (Vial)}
        This clear, gelatinous oil sparkles with tiny, ultrathin silver shards.
        The oil can coat one slashing or piercing weapon or up to 5 pieces of slashing or piercing ammunition.
        Applying the oil takes 1 minute.
        For 1 hour, the coated item has a +3 bonus to attack and damage rolls.
    \paragraph{Oil of Slipperiness (Vial)} \label{item::oilofslipperiness} % Freedom of Movement + Grease
        This sticky black unguent is thick and heavy in the container, but it flows quickly when poured.
        The oil can cover a Medium or smaller creature, along with the equipment it's wearing and carrying (one additional vial is required for each size category above Medium).
        Applying the oil takes 10 minutes.

        The affected creature's movement is unaffected by difficult terrain for 8 hours, and spells and other magical effects can neither reduce the target's speed nor cause the target to be paralyzed or restrained for the same duration.
        The target can also spend 1 meter of movement to automatically escape from nonmagical restraints, such as manacles or a creature that has it grappled.
        Finally, being underwater imposes no penalties on the target's movement or attacks.

        Alternatively, the oil can be poured on the ground as two actions, where it covers a 3-meter square.
        Slick grease covers the ground in the square which is turned into difficult terrain for 8 hours.
        Each creature standing in the area when you pour the liquid must succeed on a Dexterity saving throw or fall prone.
        A creature that enters the area or ends its turn there must also succeed on a Dexterity saving throw or fall prone.
    \paragraph{Oozekiller} \label{item::oozekiller}
        This potion contains a clear liquid which can clean any surface instantly.
        When poured on an ooze, it takes 14d6 necrotic damage.

        This potion can be drunk to end attunement with the Erebos' Ooze (see page \pageref{item::erebosooze}) and the Living Armor (see page \pageref{item::livingarmor}) items, destroying the ooze in the process.
    \paragraph{Potion of Aqueous Form (Vial)}
        When you drink this potion, you transform into a pool of water.
        You return to your true form after 10 minutes or if you are incapacitated or die.
        You're under the following effects while in this form:
        \begin{itemize}
            \item \textbf{Liquid Movement.} You have a swimming speed of 6 meters.
            You can move over or through other liquids.
            You can enter and occupy the space of another creature.
            You can rise up to your normal height, and you can pass through even Tiny openings.
            You extinguish nonmagical flames in any space you enter.
            \item \textbf{Watery Resilience.} You have resistance to physical damage.
            You also have advantage on Strength, Dexterity, and Constitution saving throws.
            \item \textbf{Limitations}. You can't talk, attack, cast spells, or activate magic items.
            Any objects you were carrying or wearing meld into your new form and are inaccessible, though you continue to be affected by anything you're wearing, such as armor.
        \end{itemize}
    \paragraph{Potion of Condensed Life (Flask)} % Heal
        Upon drinking, a surge of energy washes through the creature, causing it to regain 80 hit points.
        This potion also ends blindness, deafness, and any diseases affecting the target.
        This spell has no effect on constructs or undead.
    \paragraph{Potion of Contagion (Flask)} % Contagion.
        This flask inflicts disease, and comes in 6 flavors, detailed below.
        A creature touched by the contents of this flask is poisoned.

        At the end of each of the poisoned target's turns, the target must make a DC 16 Constitution saving throw.
        If the target succeeds on three of these saves, it is no longer poisoned, and the effect ends.
        If the target fails three of these saves, the target is no longer poisoned, but is infected by the potion's disease.
        The target is subjected to the disease for the a full week.

        Since this spell induces a natural disease in its target, any effect that removes a disease or otherwise ameliorates a disease's effects apply to it.

        \begin{itemize}
            \item \textbf{Blinding Sickness} Pain grips the creature's mind, and its eyes turn milky white.
            The creature has disadvantage on Wisdom checks and Wisdom saving throws and is blinded.
            \item \textbf{Filth Fever} A raging fever sweeps through the creature's body.
            The creature has disadvantage on Strength checks, Strength saving throws, and attack rolls that use Strength.
            \item \textbf{Fleshrot} The creature's flesh decays.
            The creature has disadvantage on Charisma checks and vulnerability to all damage.
            \item \textbf{Mindfire} The creature's mind becomes feverish.
            The creature has disadvantage on Intelligence checks and Intelligence saving throws, and the creature behaves as if under the effects of the confusion spell during combat.
            \item \textbf{Seizure} The creature is overcome with shaking.
            The creature has disadvantage on Dexterity checks, Dexterity saving throws, and attack rolls that use Dexterity.
            \item \textbf{Slimy Doom} The creature begins to bleed uncontrollably.
            The creature has disadvantage on Constitution checks and Constitution saving throws.
            In addition, whenever the creature takes damage, it is stunned until the end of its next turn.
        \end{itemize}
    \paragraph{Potion of Greater Healing (Flask)}
        You regain 4d4 + 4 hit points when you drink this potion.
        The potion's red liquid glimmers when agitated.
    \paragraph{Potion of Healing (Flask)}
        You regain 2d4 + 2 hit points when you drink this potion.
        The potion's red liquid glimmers when agitated.
    \paragraph{Potion of Regeneration (Vial)} % Regenerate
        This potion's contents stimulate a creature's natural healing ability.
        The target regains 4d8 + 15 hit points.
        Lasting for an hour, the target regains 1 hit point at the start of each of its turns (10 hit points each minute).

        The target's severed body members (fingers, legs, tails, and so on), if any, are restored after 2 minutes.
        Any body part larger than a finger causes the creature to suffer one level of exhaustion upon regeneration.
    \paragraph{Potion of Resistance (Flask)}
        This potion comes in 5 flavors.
        When you drink this potion, you gain resistance to the potion's damage type for 1 hour.
        The damage types available are acid, cold, fire, lightning, or thunder.
    \paragraph{Potion of Sickness (Flask)} % Ray of sickness.
        Upon breaking, a raw of sickening greenish energy lashes out toward the closest creature within 12 meters.
        The target takes 2d8 poison damage and must make a Constitution saving throw with DC 10.
        On a failed save, it is also poisoned until the end of your next turn.
    \paragraph{Potion of Superior Healing (Flask)}
        You regain 8d4 + 8 hit points when you drink this potion. The potion's red liquid glimmers when agitated.
    \paragraph{Potion of Supreme Healing (Flask)}
        You regain 10d4 + 20 hit points when you drink this potion.
        The potion's red liquid glimmers when agitated.
    \paragraph{Universal Solvent (Vial)} \label{item::universalsolvent}
        This tube holds milky liquid with a strong alcohol smell.
        You can use an action to pour the contents of the tube onto a surface within reach.
        The liquid instantly dissolves up to a 50 centimeters square of adhesive it touches, including sovereign glue (see page \pageref{item::sovereignglue}).
    \paragraph{Vial of Pain (Vial)} % Harm
        This vial contains a virulent disease which harms any creature who touches its contents.
        The target must make a DC 18 Constitution saving throw.
        On a failed save, it takes 14d6 necrotic damage, or half as much damage on a successful save.
        The damage can't reduce the target's hit points below 1.
        If the target fails the saving throw, its hit point maximum is reduced for 1 hour by an amount equal to the necrotic damage it took.
        Any effect that removes a disease allows a creature's hit point maximum to return to normal before that time passes.
\newpage

    % !TEX root = ../main.tex
\section{Tools \& Kits} \label{sec::toolsandkits}
\begin{table*}[b]%
    \begin{DndTable}[width=\linewidth, header=Tools \& Kits]{Xcccccc}
        \textbf{Item} & \textbf{Rarity} & \textbf{Mats.} & \textbf{Total Cost} & \textbf{Tools} & \textbf{Weight} & \textbf{Source} \\
        \multicolumn{7}{l}{\hspace{0.5cm}\textit{Artisan's Tools}} \\
        Calligrapher Supplies & Plain  & 8 & 10 agnomas & TIN       & 2.5 kg & PHB 154 \\
        Carpenter's Tools     & Plain  & 6 &  8 agnomas & CAR + SMI & 3 kg   & PHB 154 \\
        Cobbler's Tools       & Plain  & 3 &  5 agnomas & SMI + WEA & 2.5 kg & PHB 154 \\
        Cook's Utensils       & Plain  & 1 &  1 agnoma  & SMI       & 4 kg   & PHB 154 \\
        Leatherworker's Tools & Plain  & 3 &  5 agnomas & SMI       & 2.5 kg & PHB 154 \\
        Mason's Tools         & Plain  & 8 & 10 agnomas & SMI       & 4 kg   & PHB 154 \\
        Painter's Supplies    & Plain  & 8 & 10 agnomas & WOO       & 2.5 kg & PHB 154 \\
        Potter's Tools        & Plain  & 8 & 10 agnomas & TIN       & 1.5 kg & PHB 154 \\
        Weaver's Tools        & Plain  & 1 &  1 agnoma  & TIN       & 2.5 kg & PHB 154 \\
        Woodcarver's Tools    & Plain  & 1 &  1 agnoma  & SMI       & 2.5 kg & PHB 154 \\
        Alchemist's Supplies  & Common & 3 & 50 agnomas & GLA + MAS & 4 kg   & PHB 154 \\
        Brewer's Supplies     & Common & 1 & 20 agnomas & SMI + MAS & 4.5 kg & PHB 154 \\
        Cartographer's Tools  & Common & 1 & 15 agnomas & SMI + TIN & 3 kg   & PHB 154 \\
        Glassblower's Tools   & Common & 1 & 30 agnomas & SMI       & 2.5 kg & PHB 154 \\
        Jeweler's Tools       & Common & 1 & 25 agnomas & SMI       & 1 kg   & PHB 154 \\
        Navigator's Tools     & Common & 1 & 25 agnomas & TIN       & 1 kg   & PHB 154 \\
        Smith's Tools         & Common & 1 & 20 agnomas & SMI       & 4 kg   & PHB 154 \\
        Thieves' Tools        & Common & 1 & 25 agnomas & SMI + TIN & 0.5 kg & PHB 154 \\
        Tinker's Tools        & Common & 3 & 50 agnomas & TIN       & 5 kg   & PHB 154 \\
    \end{DndTable}
\end{table*}
\begin{table*}[b]%
    \begin{DndTable}[width=\linewidth, header=Tools \& Kits (cont.)]{Xcccccc}
        \multicolumn{7}{l}{\hspace{0.5cm}\textit{Kits}} \\
        Healer's Kit   & Plain  & 3 &  5 agnomas & ALC + WEA & 1.5 kg & PHB 151 \\
        Herbalism Kit  & Plain  & 3 &  5 agnomas & MAS + SMI & 1.5 kg & PHB 154 \\
        Climber's Kit  & Common & 1 & 25 agnomas & SMI + WEA & 6 kg   & PHB 151 \\
        Forgery Kit    & Common & 1 & 15 agnomas & CAL + TIN & 2.5 kg & PHB 154 \\
        Poisoner's Kit & Common & 3 & 50 agnomas & GLA + MAS & 1 kg   & PHB 154 \\
        \multicolumn{7}{l}{\hspace{0.5cm}\textit{Musical Instruments}} \\
        Drum      & Plain  &  4 &  6 agnomas & CAR + LEA & 1.5 kg & PHB 154 \\
        Flute     & Plain  &  1 &  2 agnomas & WOO       & 0.5 kg & PHB 154 \\
        Horn      & Plain  &  1 &  3 agnomas & ---       & 1 kg   & PHB 154 \\
        Pan Flute & Plain  & 10 & 12 agnomas & WOO       & 1 kg   & PHB 154 \\
        Shawm     & Plain  &  1 &  2 agnomas & WOO       & 0.5 kg & PHB 154 \\
        Bagpipes  & Common &  1 & 30 agnomas & LEA       & 3 kg   & PHB 154 \\
        Dulcimer  & Common &  1 & 25 agnomas & CAR       & 5 kg   & PHB 154 \\
        Lute      & Common &  2 & 35 agnomas & LEA + WOO & 1 kg   & PHB 154 \\
        Lyre      & Common &  1 & 30 agnomas & WOO       & 1 kg   & PHB 154 \\
        Viol      & Common &  1 & 30 agnomas & WOO       & 0.5 kg & PHB 154 \\
        \multicolumn{7}{l}{\hspace{0.5cm}\textit{Gaming Kits}} \\
        Dice Set         & Mundane & 1 & 10 fobs    & WOO & --- & PHB 154 \\
        Chess Set        & Plain   & 1 &  1 agnoma  & WOO & --- & PHB 154 \\
        Huathem Card Set & Plain   & 8 & 10 agnomas & CAL & --- & --- \\
        Skull Card Set   & Plain   & 1 &  3 agnomas & CAL & --- & ---
    \end{DndTable}
\end{table*}

\paragraph{Climber's Kit}
        A climber's kit includes pitons, boot tips, gloves, and a harness.
        You can use the climber's kit as two actions to anchor yourself; when you do, you can't fall more than 5 meters from the point where you anchored yourself, and you can't climb more than 5 meters away from that point without undoing the anchor.

        You don't need proficiency with this kit to use it effectively.
\paragraph{Forgery Kit}
        This small box contains a variety of papers and parchments, pens and inks, seals and sealing wax, gold and silver leaf, and other supplies necessary to create convincing forgeries of physical documents.
        A forgery kit is designed to duplicate documents and to make it easier to copy a person's seal or signature.

        You need proficiency with this kit to use it effectively.
\paragraph{Healer's Kit}
        This kit is a leather pouch containing bandages, salves, and splints.
        The kit has ten uses.
        Using two actions, you can expend one use of the kit to stabilize a creature that has 0 hit points, without needing to make a Wisdom (Medicine) check.

        More uses can be drawn from this set, all of which require proficiency with it.
\paragraph{Herbalism Kit}
        This kit contains a variety of instruments such as clippers, mortar and pestle, and pouches and vials used by herbalists to identify plants and poisons.

        This kit requires proficiency to be used effectively.
\paragraph{Poisoner's Kit}
        A poisoner's kit includes glass vials, a mortar and pestle, chemicals, and a glass stirring rod.
\newpage~\newpage~\newpage

    % !TEX root = ../main.tex
\section{Vehicles \& Harnesses} \label{sec::vehiclesandharnesses}
    \begin{table*}[b]%
        \begin{DndTable}[width=\linewidth, header=Vehicles and Saddles]{Xlccccc}
            \textbf{Vehicle} & \textbf{Rarity} & \textbf{Mats.} & \textbf{Total Cost} & \textbf{Tools} & \textbf{Weight} & \textbf{Source} \\
            Stabling (per day) & ---      & ---    &     50 fobs    & --- & ---      & PHB 157 \\
            \multicolumn{7}{l}{\hspace{0.5cm}\textit{Saddles, Tack, and Harnesses}} \\
            Barding            & $\ast$   & $\ast$ & $\ast$      & $\ast$ & $\ast$   & PHB 157 \\
            Bit and Bridle     & Plain    &  1     &      2 agnomas & LEA &   0.5 kg & PHB 157 \\
            Saddlebags         & Plain    &  2     &      4 agnomas & LEA &   4 kg   & PHB 157 \\
            Pack Saddle        & Plain    &  3     &      5 agnomas & LEA &   7.5 kg & PHB 157 \\
            Riding Saddle      & Plain    &  8     &     10 agnomas & LEA &  12.5 kg & PHB 157 \\
            Military Saddle    & Common   &  1     &     30 agnomas & LEA &  15 kg   & PHB 157 \\
            Exotic Saddle      & Common   &  4     &     60 agnomas & LEA &  20 kg   & PHB 157 \\
            Cavalier Saddle    & Uncommon &  8     &    500 agnomas & LEA &  20 kg   & DMG 199 \\
            \multicolumn{7}{l}{\hspace{0.5cm}\textit{Land Vehicles}} \\
            Cart               & Plain    & 13     &     15 agnomas & CAR & 100 kg   & PHB 157 \\
            Carriage           & Common   &  8     &    100 agnomas & CAR & 300 kg   & PHB 157 \\
            Chariot            & Common   & 13     &    150 agnomas & CAR &  50 kg   & PHB 157 \\
            Sled               & Common   &  1     &     20 agnomas & CAR & 150 kg   & PHB 157 \\
            Wagon              & Common   &  2     &     35 agnomas & CAR & 200 kg   & PHB 157 \\
            \multicolumn{7}{l}{\hspace{0.5cm}\textit{Water Vehicles}} \\
            Rowboat            & Common   &  3     &     50 agnomas & CAR &  50 kg   & DMG 119 \\
            Keelboat           & Uncommon & 58     &  3,000 agnomas & CAR & ---      & DMG 119 \\
            Galley             & Rare     & 58     & 30,000 agnomas & CAR & ---      & DMG 119 \\
            Longship           & Rare     & 18     & 10,000 agnomas & CAR & ---      & DMG 119 \\
            Sailing Ship       & Rare     & 18     & 10,000 agnomas & CAR & ---      & DMG 119 \\
            Warship            & Rare     & 48     & 25,000 agnomas & CAR & ---      & DMG 119
        \end{DndTable}
    \end{table*}

    \paragraph{Barding}
        Barding is armor designed to protect an animal's head, neck, chest, and body.
        Any type of armor shown on the armor tables in this chapter can be purchased as barding.
        The cost is four times the equivalent armor made for umanoids, and it weighs twice as much.
    \paragraph{Cavalier Saddle}
        While in this saddle on a mount, you can't be dismounted against your will if you're conscious, and attack rolls against the mount have disadvantage.
    \paragraph{Exotic Saddle}
        An exotic saddle is required for riding any aquatic or flying mount.
    \paragraph{Galley}
        A galley has a speed of 6 km/h, and a carrying capacity of 150 tons.
        It requires a crew of 80 to run effectively, and it has an AC of 15, HP of 500, and a damage threshold of 20.
    \paragraph{Keelboat}
        A keelboat has a speed of 1.5 km/h, and a carrying capacity of half a ton plus 6 passengers at most.
        It requires a crew of 1, and it has an AC of 15, HP of 100, and a damage threshold of 10.

        Keelboats and rowboats are used on lakes and rivers.
        If going downstream, add the speed of the current (typically 5 kilometers per hour) to the speed of the vehicle.
        These vehicles can't be rowed against any significant current, but they can be pulled upstream by draft animals on the shores.
    \paragraph{Longship}
        A longship has a speed of 5 km/h, and a carrying capacity of 10 tons plus a cargo of 150 passengers.
        It requires a crew of 40, and it has an AC of 15, HP of 300, and a damage threshold of 15.
    \paragraph{Military Saddle}
        A military saddle braces the rider, helping you keep your seat on an active mount in battle.
        It gives you advantage on any check you make to remain mounted.
    \paragraph{Rowboat}
        A rowboat has a speed of 2.5 km/h, and a carrying capacity of 3 passengers at most.
        It requires a crew of 1, and it has an AC of 11, and HP of 50.

        Keelboats and rowboats are used on lakes and rivers.
        If going downstream, add the speed of the current (typically 5 kilometers per hour) to the speed of the vehicle.
        These vehicles can't be rowed against any significant current, but they can be pulled upstream by draft animals on the shores.
        A rowboat weighs 50 kilograms, in case adventurers carry it over land.
    \paragraph{Sailing Ship}
        A sailing ship has a speed of 3 km/h, and a carrying capacity of 100 tons plus a cargo of 20 passengers.
        It requires a crew of 20, and it has an AC of 15, HP of 300, and a damage threshold of 15.
    \paragraph{Warship}
        A warship has a speed of 4 km/h, and a carrying capacity of 200 tons plus a cargo of 60 passengers.
        It requires a crew of 60, and it has an AC of 15, HP of 500, and a damage threshold of 20.
\newpage

    % !TEX root = ../main.tex
\section{Weapons} \label{sec::weapons}
% Some items are listed twice. This is because both proficiencies work for that item.

\begin{table*}[b]%
    \begin{DndTable}[width=\linewidth, header=Weapons (1/4)]{Xccccccccc}
        \textbf{Weapon} & \textbf{Rarity} & \textbf{Damage} & \textbf{Properties} & \textbf{Mats.} & \textbf{Cost} & \textbf{Tools} & \textbf{Weight} & \textbf{Source} \\
        +1 Weapon                  & Uncommon  & $\ast$ +1       & $\ast$                 & 8 &     500 agnomas & $\ast$    & $\ast$    & DMG   213 \\
        Weapon of Warning          & Uncommon  & $\ast$          & $\ast$                 & 8 &     500 agnomas & $\ast$    & $\ast$    & DMG   213 \\
        +2 Weapon                  & Rare      & $\ast$ +2       & $\ast$                 & 8 &   5,000 agnomas & $\ast$    & $\ast$    & DMG   213 \\
        Weapon of Certain Death    & Rare      & $\ast$          & $\ast$                 & 8 &   5,000 agnomas & $\ast$    & $\ast$    & EGW   270 \\
        Vicious Weapon             & Rare      & $\ast$          & $\ast$                 & 8 &   5,000 agnomas & $\ast$    & $\ast$    & DMG   209 \\
        +3 Weapon                  & Very Rare & $\ast$ +3       & $\ast$                 & 8 &  50,000 agnomas & $\ast$    & $\ast$    & DMG   213 \\
        \multicolumn{9}{l}{\hspace{0.5cm}\textit{Simple Melee Weapons}} \\
        Club                       & Mundane   & 1d4      blud.  & L                      & 1 &      10 fobs    & WOO       &  1 kg     & PHB   146 \\
        Greatclub                  & Mundane   & 1d8      blud.  & 2H                     & 2 &      20 fobs    & CAR       &  5 kg     & PHB   146 \\
        Javelin                    & Mundane   & 1d6      pier.  & T (6/24)               & 8 &      50 fobs    & WOO       &  1 kg     & PHB   146 \\
        Quarterstaff               & Mundane   & 1d6      blud.  & V                      & 2 &      20 fobs    & WOO       &  1 kg     & PHB   146 \\
        Dagger                     & Plain     & 1d4      pier.  & F, L, T (4/12)         & 1 &       2 agnomas & SMI       &  0.5 kg   & PHB   146 \\
        Handaxe                    & Plain     & 1d6      slash. & L, T (4/12)            & 3 &       5 agnomas & SMI       &  1 kg     & PHB   146 \\
        Light hammer               & Plain     & 1d4      blud.  & L, T (4/12)            & 1 &       2 agnomas & SMI       &  1 kg     & PHB   146 \\
        Mace                       & Plain     & 1d6      blud.  & ---                    & 3 &       5 agnomas & SMI       &  2 kg     & PHB   146 \\
        Sickle                     & Plain     & 1d4      slash. & L                      & 1 &       1 agnoma  & SMI       &  1 kg     & PHB   146 \\
        Spear                      & Plain     & 1d6      pier.  & T (4/12), V            & 1 &       1 agnomas & SMI       &  1.5 kg   & PHB   146 \\
        Blood Spear                & Uncommon  & 1d6      pier.  & T (4/12), V            & 8 &     500 agnomas & SMI       &  1.5 kg   & CoS   221 \\
        Lightning Javelin          & Uncommon  & 1d6      pier.  & T (6/24)               & 8 &     500 agnomas & SMI       &  1 kg     & DMG   178 \\
        Blacksaw                   & Rare      & 1d4      pier.  & F, L, T (4/12)         & 8 &   5,000 agnomas & SMI       &  0.5 kg   & WDMM   86 \\
        Luck Weapon                & Rare      & $\dagger$ + 1   & $\dagger$              & 8 &   5,000 agnomas & $\dagger$ & $\dagger$ & DMG   179 \\
        Smiting Mace (+1)          & Rare      & 1d6 + 1  blud.  & ---                    & 8 &   5,000 agnomas & SMI       &  2 kg     & DMG   179 \\
        Terror Mace                & Rare      & 1d6      blud.  & ---                    & 8 &   5,000 agnomas & SMI       &  2 kg     & DMG   180 \\
        Venom Dagger (+1)          & Rare      & 1d4 + 1  pier.  & F, L, T (4/12)         & 8 &   5,000 agnomas & POI       &  0.5 kg   & DMG   161 \\
        Nidhogg Tooth (+1)         & Rare      & 1d4 + 1  pier.  & F, L, T (4/12)         & 8 &   5,000 agnomas & WOO       &  0.5 kg   & RoT    94 \\
        Tinderstrike (+2)          & Legendary & 1d4 + 2  pier.  & F, L, T (4/12)         & 8 & 250,000 agnomas & SMI       &  0.5 kg   & PotA  224 \\
        \multicolumn{9}{l}{\hspace{0.5cm}\textit{Simple Ranged Weapons}} \\
        Dart                       & Mundane   & 1d4      pier.  & F, T (4/12)            & 1 &       5 fobs    & WOO       & ---       & PHB  146 \\
        Sling                      & Mundane   & 1d4      blud.  & A (6/24)               & 1 &      10 fobs    & LEA       & ---       & PHB  146 \\
        Light crossbow             & Common    & 1d8      pier.  & A (16/64)              & 1 &      25 agnomas & TIN       &  2.5 kg   & PHB  146 \\
        Shortbow                   & Common    & 1d6      pier.  & A (16/64), 2H          & 1 &      25 agnomas & CAR       &  1 kg     & PHB  146 \\
        Seeker Dart                & Uncommon  & 1d4      pier.  & F, T (4/12)            & 1 &     150 agnomas & WOO       & ---       & PotA 223 \\
        Storm Boomerang            & Uncommon  & 1d4      blud.  & T (12/24)              & 8 &     500 agnomas & WOO       & ---       & PotA 223 \\
        Luck Weapon (+1)           & Rare      & $\dagger$ + 1   & $\dagger$              & 8 &   5,000 agnomas & $\dagger$ & $\dagger$ & DMG  179 \\
        Two-Birds Sling (+1)       & Rare      & 1d4 + 1  blud.  & A (6/24)               & 8 &  50,000 agnomas & LEA       & ---       & MOT  198 \\
    \end{DndTable}
\end{table*}
\begin{table*}[b]%
    \begin{DndTable}[width=\linewidth, header=Weapons (2/4)]{Xccccccccc}
        \multicolumn{9}{l}{\hspace{0.5cm}\textit{Axes}} \\
        Battleaxe                  & Plain     & 1d8      slash. & V                      & 8 &      10 agnomas & SMI       &  2 kg     & PHB   146 \\
        Handaxe                    & Plain     & 1d6      slash. & L, T (4/12)            & 3 &       5 agnomas & SMI       &  1 kg     & PHB   146 \\
        War pick                   & Plain     & 1d8      pier.  & ---                    & 3 &       5 agnomas & SMI       &  1 kg     & PHB   146 \\
        Greataxe                   & Common    & 1d12     slash. & H, 2H                  & 1 &      30 agnomas & SMI       &  3.5 kg   & PHB   146 \\
        Berserker Axe (+1)         & Uncommon  & $\dagger$ + 1   & $\dagger$              & 8 &     500 agnomas & SMI       & $\dagger$ & DMG   155 \\
        Giant Slayer (+1)          & Rare      & $\dagger$ + 1   & $\dagger$              & 8 &   5,000 agnomas & MAS       & $\dagger$ & DMG   172 \\
        Woodcutter (+1)      & Rare      & $\dagger$ + 1   & $\dagger$              & 8 &   5,000 agnomas & SMI       & $\dagger$ & WBtW  214 \\
        Bloodaxe (+2)              & Very Rare & 1d12 + 2 slash. & H, 2H                  & 8 &  50,000 agnomas & SMI       &  3.5 kg   & EGW   266 \\
        Fane-Eater (+3)            & Legendary & 1d8 + 3  slash. & V                      & 8 & 250,000 agnomas & SMI       &  2 kg     & BGDIA 223 \\
        Ironfang (+2)              & Legendary & 1d8 + 2  pier.  & ---                    & 8 & 250,000 agnomas & SMI       &  1 kg     & PotA  224 \\
        \multicolumn{9}{l}{\hspace{0.5cm}\textit{Bows}} \\
        Composite Bow              & Common    & 1d6      pier.  & A (16/64), S, 2H       & 6 &      75 agnomas & TIN       &  1.5 kg   & ---       \\
        Greatbow                   & Common    & 1d10     pier.  & A (30/120), S, H, 2H   & 6 &      75 agnomas & CAR       &  3 kg     & ---       \\
        Shortbow                   & Common    & 1d6      pier.  & A (16/64), 2H          & 1 &      25 agnomas & CAR       &  1 kg     & PHB   146 \\
        Longbow                    & Common    & 1d8      pier.  & A (30/120), H, 2H      & 3 &      50 agnomas & CAR       &  1 kg     & PHB   146 \\
        Sunbow (+2)                & Rare      & 1d8 + 2  pier.  & A (30/120), H, 2H      & 8 &   5,000 agnomas & CAR       &  1 kg     & GGR   181 \\
        Oathbow                    & Very Rare & 1d8      pier.  & A (30/120), H, 2H      & 8 &  50,000 agnomas & CAR       &  1 kg     & DMG   183 \\
        \multicolumn{9}{l}{\hspace{0.5cm}\textit{Crossbows and Firearms}} \\
        Hand crossbow              & Common    & 1d6      pier.  & A (6/24), L, Load      & 6 &      75 agnomas & TIN       &  1.5 kg   & PHB  146 \\
        Heavy crossbow             & Common    & 1d10     pier.  & A (20/80), Load, 2H    & 3 &      50 agnomas & TIN       &  9 kg     & PHB  146 \\
        Light crossbow             & Common    & 1d8      pier.  & A (16/64), Load        & 1 &      25 agnomas & TIN       &  2.5 kg   & PHB  146 \\
        Musket                     & Uncommon  & 1d12     pier.  & A (8/24), M, Load, 2H  & 8 &     500 agnomas & TIN       &  5 kg     & DMG  268 \\
        Pistol                     & Uncommon  & 1d10     pier.  & A (6/18), M, Load      & 3 &     250 agonmas & TIN       &  1.5 kg   & DMG  268 \\
        Pyroconverger              & Uncommon  & 4d6      fire   & S, H, 2H               & 8 &     500 agnomas & TIN       &  6 kg     & GGR  180 \\
        Repeater                   & Uncommon  & 1d8      pier.  & A (16/64), Load        & 8 &     500 agnomas & TIN       &  2.5 kg   & OotA 224 \\
        Corpse Slayer (+1)         & Rare      & $\dagger$ + 1   & $\dagger$              & 8 &   5,000 agnomas & TIN       & $\dagger$ & EGW  266 \\
        \multicolumn{9}{l}{\hspace{0.5cm}\textit{Curved Swords}} \\
        Curved Greatsword          & Common    & 1d10     slash. & H, 2H                  & 6 &      75 agnomas & SMI       &  3 kg     & ---      \\
        Twin Scimitar              & Common    & 2d4      slash. & F, S, 2H               & 8 &     100 agnomas & SMI       &  3 kg     & ERLW  21 \\
        Falchion                   & Common    & 1d6      slash. & V                      & 1 &      30 agnomas & SMI       &  2 kg     & ---      \\
        Scimitar                   & Common    & 1d6      slash. & F, L                   & 1 &      25 agnomas & SMI       &  1.5 kg   & PHB  146 \\
        Flame Tongue               & Rare      & $\dagger$       & $\dagger$              & 8 &   5,000 agnomas & POT       & $\dagger$ & DMG  170 \\
        Frost Brand                & Very Rare & $\dagger$       & $\dagger$              & 8 &  50,000 agnomas & POT       & $\dagger$ & DMG  171 \\
        Speed Scimitar (+2)        & Very Rare & 1d6  + 2 slash. & F, L                   & 8 &  50,000 agnomas & SMI       &  1.5 kg   & DMG  199 \\
        Grungsplitter (+2)         & Legendary & 1d10 + 2 slash. & H, 2H                  & 8 & 250,000 agnomas & SMI       &  3.5 kg   & PotA 224 \\
    \end{DndTable}
\end{table*}
\begin{table*}[b]%
    \begin{DndTable}[width=\linewidth, header=Weapons (3/4)]{Xccccccccc}
        \multicolumn{9}{l}{\hspace{0.5cm}\textit{Hammers}} \\
        Light hammer               & Plain     & 1d4      blud.  & L, T (4/12)            & 1 &       2 agnomas & SMI       &  1 kg     & PHB   146 \\
        Mace                       & Plain     & 1d6      blud.  & ---                    & 3 &       5 agnomas & SMI       &  2 kg     & PHB   146 \\
        Maul                       & Common    & 2d6      blud.  & H, 2H                  & 1 &      10 agnomas & SMI       &  5 kg     & PHB   146 \\
        Morningstar                & Common    & 1d8      pier.  & ---                    & 1 &      15 agnomas & SMI       &  2 kg     & PHB   146 \\
        Warhammer                  & Common    & 1d8      blud.  & V                      & 1 &      15 agnomas & SMI       &  1 kg     & PHB   146 \\
        Destructive Weapon         & Uncommon  & $\ast$          & $\ast$                 & 8 &     500 agnomas & $\ast$    & $\ast$    & XGE    78 \\
        Smiting Mace (+1)          & Rare      & 1d6 + 1  blud.  & ---                    & 8 &   5,000 agnomas & SMI       &  2 kg     & DMG   179 \\
        Terror Mace                & Rare      & 1d6      blud.  & ---                    & 8 &   5,000 agnomas & SMI       &  2 kg     & DMG   180 \\
        Duskcrusher                & Very Rare & 1d8      blud.  & V                      & 8 &  50,000 agnomas & WOO       &  1 kg     & EGW   266 \\
        Thulkrakan Thrower (+3)    & Very Rare & 1d8 + 3  blud.  & T (4/12), V            & 8 &  50,000 agnomas & SMI       &  1 kg     & DMG   167 \\
        Matalotok (+3)             & Legendary & 1d8 + 3 blud.   & V                      & 8 & 250,000 agnomas & SMI       &  2 kg     & BGDIA 224 \\
        \multicolumn{9}{l}{\hspace{0.5cm}\textit{Polearms}} \\
        Glaive                     & Common    & 1d10     slash. & H, R, 2H               & 1 &      30 agnomas & SMI       &  3 kg     & PHB 146 \\
        Poleaxe                    & Common    & 1d10     slash. & H, R, 2H               & 1 &      20 agnomas & SMI       &  3 kg     & PHB 146 \\
        Horseman's Pick            & Common    & 1d10     pier.  & H, R, 2H               & 1 &      30 agnomas & SMI       &  3 kg     & ---     \\
        Lucerne                    & Common    & 1d10     blud.  & H, R, 2H               & 1 &      30 agnomas & SMI       &  3.5 kg   & ---     \\
        Lifestealer (+2)           & Very Rare & 1d10 + 2 pier.  & H, R, 2H               & 8 &  50,000 agnomas & SMI       &  3 kg     & DMG 183 \\
        Vorpal Glaive (+3)         & Legendary & 1d10 + 3 slash. & H, R, 2H               & 8 & 250,000 agnomas & POT       &  3 kg     & DMG 209 \\
        \multicolumn{9}{l}{\hspace{0.5cm}\textit{Rapiers}} \\
        Rapier                     & Common    & 1d8      pier.  & F                      & 1 &      25 agnomas & SMI       &  1 kg     & PHB 146 \\
        Estoc                      & Common    & 1d8      slash. & F                      & 1 &      20 agnomas & SMI       &  1.5 kg   & ---     \\
        Serpent's Fang             & Rare      & 1d8      slash. & F                      & 8 &   5,000 agnomas & SMI       &  1.5 kg   & CM   98 \\
        Tartyx Blade (+1)          & Rare      & $\dagger$ + 1   & $\dagger$              & 8 &   5,000 agnomas & SMI       & $\dagger$ & EGW 265 \\
        Vampire's Kiss             & Rare      & 1d8      pier.  & F                      & 8 &   5,000 agnomas & SMI       &  1 kg     & DMG 206 \\
        Estoc of Sharpness         & Very Rare & 8        slash. & F                      & 8 &  50,000 agnomas & POT       &  1.5 kg   & DMG 206 \\
        Defender (+3)              & Legendary & $\dagger$ + 3   & $\dagger$              & 8 & 250,000 agnomas & SMI       & $\dagger$ & DMG 164 \\
        \multicolumn{9}{l}{\hspace{0.5cm}\textit{Spears}} \\
        Javelin                    & Mundane   & 1d6      pier.  & T (6/24)               & 8 &      50 fobs    & WOO       &  1 kg     & PHB   146 \\
        Lance                      & Common    & 1d12     pier.  & R, S                   & 1 &      10 agnomas & SMI       &  9 kg     & PHB   146 \\
        Pike                       & Plain     & 1d10     pier.  & H, R, 2H               & 3 &       5 agnomas & CAR       &  9 kg     & PHB   146 \\
        Spear                      & Plain     & 1d6      pier.  & T (4/12), V            & 1 &       1 agnomas & SMI       &  1.5 kg   & PHB   146 \\
        Trident                    & Plain     & 1d6      pier.  & T (4/12), V            & 3 &       5 agnomas & SMI       &  2 kg     & PHB   146 \\
        Blood Spear                & Uncommon  & 1d6      pier.  & T (4/12), V            & 8 &     500 agnomas & SMI       &  1.5 kg   & CoS   221 \\
        Lightning Javelin          & Uncommon  & 1d6      pier.  & T (6/24)               & 8 &     500 agnomas & SMI       &  1 kg     & DMG   178 \\
        Backbiting Spear (+2)      & Very Rare & 1d6 + 2  pier.  & T (10/18), V           & 8 &  50,000 agnomas & CAR       &  1.5 kg   & TftYP 229 \\
        Purger (+3)                & Legendary & $\dagger$ + 3   & $\dagger$              & 8 & 250,000 agnomas & MAS       & $\dagger$ & DMG   174 \\
        Windvane (+2)              & Legendary & 1d10     pier.  & F, R, 2H               & 8 & 250,000 agnomas & JEW       & ---       & PotA  224 \\
    \end{DndTable}
\end{table*}
\begin{table*}[b]%
    \begin{DndTable}[width=\linewidth, header=Weapons (4/4)]{Xccccccccc}
        \multicolumn{9}{l}{\hspace{0.5cm}\textit{Straight Swords}} \\
        Backsword                  & Common    & 1d8      slash. & F                      & 3 &      50 agnomas & SMI       &  2.5 kg   & ---       \\
        Broadsword                 & Common    & 1d8      slash. & H, V                   & 1 &      20 agnomas & SMI       &  2 kg     & ---       \\
        Greatsword                 & Common    & 2d6      slash. & H, 2H                  & 3 &      50 agnomas & SMI       &  3 kg     & PHB   146 \\
        Longsword                  & Common    & 1d8      slash. & V                      & 1 &      15 agnomas & SMI       &  1.5 kg   & PHB   146 \\
        Shortsword                 & Common    & 1d6      pier.  & F, L                   & 1 &      10 agnomas & SMI       &  1 kg     & PHB   146 \\
        Macuahuitl (+1)            & Uncommon  & 1d8      blud.  & H, V                   & 8 &     500 agnomas & MAS       &  3 kg     & TftYP  70 \\
        Crusader's Sword (+1)      & Rare      & 1d6 + 1  pier.  & F, L                   & 8 &   5,000 agnomas & SMI       &  1 kg     & CoS    81 \\
        Beast Slayer (+1)          & Rare      & $\dagger$ + 1   & $\dagger$              & 8 &   5,000 agnomas & SMI       & $\dagger$ & DMG   166 \\
        Sword of Wounding          & Rare      & 1d8      slash. & V                      & 8 &   5,000 agnomas & SMI       &  1.5 kg   & DMG   207 \\
        Dancing Sword              & Very Rare & $\dagger$       & $\dagger$              & 8 &  50,000 agnomas & SMI       & $\dagger$ & DMG   161 \\
        Magekiller Backsword (+3)  & Legendary & 1d8 + 3  slash. & F                      & 8 & 250,000 agnomas & SMI       &  1.5 kg   & MTF    89 \\
        Waythe (+1)                & Legendary & 2d6 + 1  slash. & H, 2H                  & 8 & 250,000 agnomas & SMI       &  3 kg     & TftYP 229 \\
        \multicolumn{9}{l}{\hspace{0.5cm}\textit{Exotic Weapons}} \\
        Net                        & Plain     & ---             & S, T (2/3)             & 1 &       1 agnoma  & LEA       &  1.5 kg   & PHB   146 \\
        Whip                       & Plain     & 1d4      slash. & F, R                   & 1 &       2 agnomas & LEA       &  1.5 kg   & PHB   146 \\
        Blowgun                    & Common    & 1        pier.  & A (5/20)               & 1 &      10 agnomas & WOO       &  0.5 kg   & PHB   146 \\
        Flail                      & Common    & 1d8      blud.  & ---                    & 1 &      10 agnomas & SMI       &  1 kg     & PHB   146 \\
        Gunpowder Horn             & Common    & 3d6      fire   & S                      & 2 &      35 agnomas & TIN       &  1 kg     & DMG   268 \\
        Bomb                       & Uncommon  & 3d6      fire   & S, T (4/12)            & 1 &     150 agnomas & TIN       &  0.5 kg   & DMG   268 \\
        Gunpowder Keg              & Uncommon  & 7d6      fire   & S                      & 3 &     250 agnomas & TIN       & 10 kg     & DMG   268 \\
        Devotee's Censer           & Rare      & 1d8      blud.  & ---                    & 8 &   5,000 agnomas & SMI       &  1 kg     & TCE   126 \\
        Dyrrn's Tentacle Whip (+2) & Very Rare & 1d4 + 2  slash. & F, R                   & 8 &  50,000 agnomas & ALC       &  1.5 kg   & ERLW  276
    \end{DndTable}
\end{table*}

$\ast$ \textit{This weapon's properties, cost, and required toolset depend on the mundane version of the weapon in question.}

$\dagger$ \textit{This weapon is a variant.
Its base weapon can be any weapon in its category.}

\subsection*{Weapon Properties} \label{ssec::weaponproperties}
    \subparagraph{Ammunition (A)}
        You can use a weapon that has the ammunition property to make a ranged attack only if you have ammunition to fire from the weapon.
        Each time you attack with the weapon, you expend one piece of ammunition.
        Drawing the ammunition from a quiver, case, or other container is part of the attack.
        Loading a one-handed weapon requires a free hand.
        At the end of the battle, you can recover half your expended ammunition by taking a minute to search the battlefield.

        If you use a weapon that has the ammunition property to make a melee attack, you treat the weapon as an improvised weapon.
        A sling must be loaded to deal any damage when used in this way.
    \subparagraph{Finesse (F)}
        When making an attack with a finesse weapon, you use your choice of your Strength or Dexterity modifier for the attack and damage rolls.
        You must use the same modifier for both rolls.
    \subparagraph{Heavy (H)}
        Creatures that are Small or Tiny have disadvantage on attack rolls with heavy weapons.
        A heavy weapon's size and bulk make it too large for a Small or Tiny creature to use effectively.

        In addition, the multiple attack penalty is increased to 6 for weapons with this property.
    \subparagraph{Light (L)}
        A light weapon is small and easy to handle, making it ideal for use when fighting with two weapons.

        In addition, the multiple attack penalty is reduced to 4 for weapons with this property.
    \subparagraph{Loading (Load)}
        After attacking with this weapon, you need to use one action to reload it before attacking again.
        You must have one free hand to reload the weapon.
    \subparagraph{Misfiring (M)}
        Whenever you make an attack roll, and the dice roll is equal to or lower than the weapon's misfire score, the weapon misfires.
        The attack misses, and the weapon cannot be used again until you spend two actions to try and repair it.
        To repair your weapon, you must make a successful Tinker's Tools check (DC equal to 8 + the weapon's misfire score).
        If your check fails, the weapon is broken and must be mended as part of a long rest at a quarter of its cost.
        Creatures who use a weapon without being proficient increase its misfire score by 1.
    \subparagraph{Range}
        A weapon that can be used to make a ranged Attack has a range in parentheses after the Ammunition or thrown property.
        The range lists two numbers.
        The first is the weapon's normal range in meters, and the second indicates the weapon's long range.
        When attacking a target beyond normal range, you have disadvantage on the Attack roll.
        You can't Attack a target beyond the weapon's long range.
    \subparagraph{Reach (R)}
        This weapon adds 1 meter to your reach when you attack with it, as well as when determining your reach for opportunity attacks with it.
    \subparagraph{Special (S)}
        A weapon with the special property has unusual rules governing its use, explained in the weapon's description (see ``Special Weapons'' later in this section).
    \subparagraph{Thrown (T)}
        If a weapon has the thrown property, you can throw the weapon to make a ranged attack.
        If the weapon is a melee weapon, you use the same ability modifier for that attack roll and damage roll that you would use for a melee attack with the weapon.
        For example, if you throw a handaxe, you use your Strength, but if you throw a dagger, you can use either your Strength or your Dexterity, since the Dagger has the finesse property.
    % \subparagraph{Trick}
        % This weapon requires the Flexible Fighter 1 feat to use.
        % These weapons can be transformed into alternate forms and have different capabilities in this transformed state.
        % The weapons have two values for damage and properties, one for their normal form, and one for the transformed one.
    \subparagraph{Two-Handed (2H)}
        This weapon requires two hands to attack with it.
        This property is relevant only when you attack with the weapon, not when you simply hold it.
    \subparagraph{Versatile (V)}
        This weapon can be used with one or two hands.
        A damage value in parentheses appears with the property --- the damage when the weapon is used with two hands to make a melee attack.
\subsection*{Improvised Weapons} \label{ssec::improvisedweapons}
    Sometimes characters don't have their weapons and have to attack with whatever is at hand.
    An improvised weapon includes any object you can wield in one or two hands, such as broken glass, a table leg, a frying pan, a wagon wheel, or a dead creature.

    Often, an improvised weapon is similar to an actual weapon and can be treated as such.
    For example, a table leg is akin to a club.
    At the DM's option, a character proficient with a weapon can use a similar object as if it were that weapon and use their proficiency bonus with that weapon.

    An object that bears no resemblance to a weapon deals 1d4 damage (the DM assigns a damage type appropriate to the object).
    If a character uses a ranged weapon to make a melee attack, or throws a melee weapon that does not have the thrown property, it also deals 1d4 damage.
    An improvised thrown weapon has a normal range of 4 meters and a long range of 12 meters.
\subsection*{Weapon List} \label{ssec::weaponlist}
    \paragraph{+1/+2/+3 Weapon}
        You have a +1/+2/+3 bonus to attack and damage rolls made with this weapon.
    \paragraph{Backbiting Spear $\odot$}
        This spear's handle sports an intricate and impossible-looking design.
        When you throw it, it deals one extra die of damage on a hit.
        After you throw it and it hits or misses, it flies back to your hand immediately.

        Whenever you roll a 1 on an thrown attack roll using this weapon, the weapon bends or flies to hit you in the back.
        Make a new attack roll with advantage against your own AC.
        If the result is a hit, you take damage as if you had attacked yourself with the spear.
    \paragraph{Beast Slayer}
        When you hit a beast with this weapon, the beast takes an extra 3d6 damage of the weapon's type.
        For the purpose of this weapon, ``beast'' refers to any creature with the beast type.
    \paragraph{Berserker Axe $\odot$}
        While you are attuned to this weapon, your hit point maximum increases by 1 for each level you have attained.

        Whenever a hostile creature damages you while the axe is in your possession, you must succeed on a DC 15 Wisdom saving throw or go berserk.
        While berserk, you must use your actions each round to attack the creature nearest to you with the axe, moving to attack the next nearest creature after you fell your current target.
        If you have multiple possible targets, you attack one at random.
        You are berserk until you start your turn with no creatures within 12 meters of you that you can see or hear.
    \paragraph{Blacksaw $\odot$}
        This dagger has a saw-toothed edge and a black pearl nested in its pommel.
        A creature attuned to it gains blindsight out to a range of 6 meters.
    \paragraph{Blood Spear $\odot$}
        When you hit with a melee attack using this spear and reduce the target to 0 hit points, you gain 2d6 temporary hit points.
    \paragraph{Bloodaxe $\odot$}
        This axe is forged from a dark, rust-colored metal originally designed for the barbarian Grog Strongjaw.

        This greataxe deals an extra 1d6 necrotic damage to creatures that aren't constructs.
        If you reduce such a creature to 0 hit points with an attack using this axe, you gain 10 temporary hit points.
    \paragraph{Bomb}
        As an action, a character can light this bomb and throw it at a point up to 4/12 meters away.
        Each creature within 1 meter of that point must succeed on a DC 12 Dexterity saving throw or take 3d6 fire damage.
    \paragraph{Composite Bow}
        Instead of your Dexterity bonus, you add your Strength bonus to ranged attack and damage rolls made with a composite bow.
    \paragraph{Corpse Slayer}
        This special crossbow or firearm uses a peculiar mechanism to heat up its ammunition up while firing it.
        When you hit a fiend with an attack using this weapon, the attack deals an extra 1d8 damage of the weapon's type, and the creature has disadvantage on saving throws until the start of your next turn.
    \paragraph{Crusader's Sword $\odot$}
        This shortsword's design shows a purpose to fight evil.
        The sword has the following properties:
        \begin{itemize}
            \item The sword continually sheds bright light in a 3-meter radius and dim light for an additional 3 meters.
            Only by destroying the sword can this light be extinguished.
            \item While attuned to the weapon, the sword's wielder can activate an aura of boldness.
            For one minute, this sword awakens boldness in every friendly creature within 6 meters of you.
            While in this aura, each of these creatures (including you) deals an extra 1d4 radiant damage when it hits with a weapon attack.
            Once used, this property of the sword can't be used again until the next dawn.
        \end{itemize}
    \paragraph{Dancing Sword $\odot$}
        You can use an action to toss this magic sword into the air and speak the command word.
        When you do so, the sword begins to hover, flies up to 6 meters, and attacks one creature of your choice within 1 meter of it.
        The sword uses your attack roll and ability score modifier to damage rolls.

        While the sword hovers, you can use an action to cause it to fly up to 6 meters to another spot within 6 meters of you.
        As part of the same action, you can cause the sword to attack one creature within 1 meter of it.

        After a minute or when you call it as a free action, the sword flies up to 6 meters and tries to return to your hand.
        If you have no hand free, it falls to the ground at your feet.
        If the sword has no unobstructed path to you, it moves as close to you as it can and then falls to the ground.
        It also ceases to hover if you grasp it or move more than 6 meters away from it.
    \paragraph{Defender $\odot$}
        The first time you attack with this sword on each of your turns, you can transfer some or all of the sword's bonus to your Armor Class, instead of using the bonus on any attacks that turn.
        For example, you could reduce the bonus to your attack and damage rolls to +1 and gain a +2 bonus to AC.
        The adjusted bonuses remain in effect until the start of your next turn, although you must hold the sword to gain a bonus to AC from it.
    \paragraph{Destructive Weapon}
        Coated with heavy steel, this hammer is unusually effective when used to break objects.
        Whenever this weapon hits an object, the hit is a critical hit.
    \paragraph{Devotee's Censer $\odot$}
        The rounded head of this flail is perforated with tiny holes, arranged in symbols and patterns.
        When you hit with an attack using this flail, the target takes an extra 1d8 radiant damage.

        As an action, you can speak the command word to cause the flail to emanate a thin cloud of incense out to 2 meters for 1 minute.
        At the start of each of your turns, you and any other creatures in the incense each regain 1d4 hit points.
        This property can't be used again until the next dawn.
    \paragraph{Twin Scimitar}
        If you attack with a Twin Scimitar as part of the Attack action on your turn, you can use another action immediately after to make a melee attack with it.
        This attack deals 1d4 slashing damage on a hit, instead of 2d4.
        Additionally, this attack is unaffected by the multiple attack penalty and doesn't add to it.
    \paragraph{Duskcrusher $\odot$}
        This item takes the form of a leather-wrapped wooden rod emblazoned with the symbol of Heliod, god of the sun.
        While grasping the rod, you can use an action to cause a warhammer head of crackling radiance to spring into existence.
        The warhammer's radiant head emits bright light in a 3-meter radius and dim light for an additional 3 meters.
        The light is sunlight.
        You can use an action to make the radiant head disappear.

        While the radiant head is active, you gain a +2 bonus to attack and damage rolls made with this weapon, and attacks with the weapon deal radiant damage instead of bludgeoning damage.
        A fiend hit by the weapon takes an extra 1d8 radiant damage.

        While you are holding Duskcrusher and its radiant head is active, you can use two actions to cast the sunbeam spell (see page \pageref{spell::sunbeam}) with save DC 15 from the weapon, and this action can't be used again until the next dawn.
    \paragraph{Dyrrn's Tentacle Whip $\odot$}
        This long, whip-like strand of tough muscle bears a sharp stinger at one end.
        To attune to this symbiotic weapon, you wrap the whip around your wrist for the entire attunement period, during which time the whip painfully embeds its tendrils into your arm.

        Attack rolls made against aberrations with this weapon have disadvantage.
        A creature hit by this weapon takes an extra 1d6 psychic damage.
        When you roll a 20 on an attack roll with this weapon, the target is stunned until the end of its next turn.

        As an action, you can sheathe the whip by causing it to retract into your arm, or draw the whip out of your arm again.

        \textbf{Symbiotic Nature}.
        The whip can't be removed from you while you're attuned to it, and you can't voluntarily end your attunement to it.
    \paragraph{Estoc of Sharpness $\odot$}
        When you attack a creature or object with this black ceramic estoc and hit, maximize your weapon damage dice against the target.

        When you attack a creature with this weapon and roll a 20 on the attack roll, that target takes an extra 14 slashing damage.
        In addition, all minor injuries inflicted with this sword are converted into major injuries.
    \paragraph{Fane-Eater $\odot$}
        Born of bone and steel, this battleaxe exhudes an unholy aura.

        If you attack a creature with this weapon and roll a 20 on the attack roll, the creature takes an extra 2d8 necrotic damage, and you regain a number of hit points equal to the necrotic damage taken.
    \paragraph{Flame Tongue $\odot$}
        You can use an action to speak this sword's command word, causing flames to erupt from the blade.
        These flames shed bright light in a 8-meter radius and dim light for an additional 8 meters.
        While the sword is ablaze, it deals an extra 2d6 fire damage to any target it hits.
        The flames last until you use an action to speak the command word again or until you drop or sheathe the sword.
    \paragraph{Frost Brand $\odot$}
        When you hit with an attack using this sword, the target takes an extra 1d6 cold damage.
        In addition, while you hold the sword, you have resistance to fire damage.

        In freezing temperatures, the blade sheds bright light in a 2-meter radius and dim light for an additional 2 meters.

        When you draw this weapon, you extinguish all nonmagical flames within 6 meters of you.
        This property can be used no more than once per hour.
    \paragraph{Giant Slayer}
        When you hit a giant with this weapon, the giant takes an extra 2d6 damage of the weapon's type and must succeed on a DC 15 Strength saving throw or fall prone.
        For the purpose of this weapon, ``giant'' refers to any creature with the giant type.
    \paragraph{Greatbow}
        Due to its size, all attack rolls made with a greatbow are made with disadvantage if you are not mounted.
    \paragraph{Grungsplitter $\odot$}
        A mighty curved greatsword designed long ago by the gat king Torhild Flametongue, Grungsplitter is a battered weapon that appears unremarkable at first glance.
        You gain the following benefits while holding this weapon:
        \begin{itemize}
            \item When you roll a 20 on an attack roll with this weapon against a grung, that grung must succeed on a DC 17 Constitution saving throw or drop to 0 hit points.
            \item You can't be surprised by grungs while you're not incapacitated.
            You are also aware when grungs are within 24 meters of you and aren't behind total cover, although you don't know their location.
            \item You and any of your friends within 30 feet of you can't be frightened while you're not incapacitated.
        \end{itemize}
    \paragraph{Gunpowder Horn}
        Setting fire to a a horn full of gunpowder can cause it to explode, dealing 3d6 fire damage to creatures within 2 meters of it.
        A successful DC 12 Dexterity saving throw halves the damage.
        Setting fire to an ounce of gunpowder causes it to flare for 1 round, shedding bright light in a 6-meter radius and dim light for an additional 6 meters.
    \paragraph{Gunpowder Keg}
        Setting fire to a keg full of gunpowder can cause it to explode, dealing 7d6 fire damage to creatures within 2 meters of it.
        A successful DC 12 Dexterity saving throw halves the damage.
        Setting fire to an ounce of gunpowder causes it to flare for 1 round, shedding bright light in a 6-meter radius and dim light for an additional 6 meters.
    \paragraph{Ironfang $\odot$}
        A war pick forged from a single piece of iron, Ironfang has a fang-like head inscribed with ancient runes.
        When you hit with this weapon, the target takes an extra 1d8 thunder damage.

        \textbf{Earth Mastery}.
        You gain the following benefits while you hold Ironfang:
        \begin{itemize}
            \item You have resistance to acid damage.
            \item You have tremorsense out to a range of 12 meters.
            \item You can sense the presence of precious metals and stones within 12 meters of you, but not their exact location.
        \end{itemize}

        \textbf{Shatter}.
        Ironfang has 3 charges.
        You can use your action to expend 1 charge and cast the 2nd-level version of shatter (DC 17).
        Ironfang regains 1d3 expended charges daily at dawn.
    \paragraph{Lance}
        You have disadvantage when you use a lance to attack a target within 1 meter of you.
        Also, a lance requires two hands to wield when you aren't mounted.
    \paragraph{Lifestealer $\odot$}
        This horseman's pick has 1d8 + 1 charges.
        If you score a critical hit against a creature that has fewer than 100 hit points, it must succeed on a DC 15 Constitution saving throw or be slain instantly as the pick tears its life force from its body (a construct is immune to this effect).
        The pick loses 1 charge if the creature is slain.
        When the pick has no charges remaining, it loses this property.
    \paragraph{Repeater}
        This light crossbow is fitted with a cartridge that can hold up to six crossbow bolts.
        It automatically reloads after firing until the cartridge runs out of ammunition.
        Reloading the cartridge takes an action.
    \paragraph{Lightning Javelin}
        When you hurl this javelin and speak its command word, it transforms into a bolt of lightning, forming a line 1 meter wide that extends out from you to a target within 24 meters.
        Each creature in the line excluding you and the target must make a DC 13 Dexterity saving throw, taking 4d6 lightning damage on a failed save, and half as much damage on a successful one.
        The lightning bolt turns back into a javelin when it reaches the target.
        Make a ranged weapon attack against the target.
        On a hit, the target takes damage from the javelin plus 4d6 lightning damage.

        The javelin's property can't be used again until the next dawn.
        In the meantime, the javelin can still be used as a weapon.
    \paragraph{Luck Weapon $\odot$}
        While this weapon is on your person, you gain a +1 bonus to saving throws.

        If this weapon is on your person, you can call on its luck (no action required) to reroll one attack roll, ability check, or saving throw you dislike.
        You must use the second roll.
        This property can't be used again until the next dawn.
    \paragraph{Macuahuitl}
        This broadsword is made of laminated wood, and inset with jagged teeth of obsidian.
    \paragraph{Magekiller Backsword $\odot$}
        This silvered backsword is shaped in an intricate and ornate design.
        While you hold the sword, you have advantage on Intelligence, Wisdom, and Charisma saving throws, you are immune to being charmed, and you have resistance to psychic damage.
        In addition, if you score a critical hit with it against a creature, the weapon's damage becomes psychic for the attack.
    \paragraph{Matalotok $\odot$}
        You are immune to cold damage while holding this warhammer.
        Whenever it deals damage to a creature, the hammer radiates a burst of intense cold in a 6-meter-radius sphere.
        Each creature in that area takes 3d6 cold damage.
    \paragraph{Musket}
        This weapon's misfire score is 1.
    \paragraph{Net}
        A Large or smaller creature hit by a net is restrained until it is freed.
        A net has no effect on creatures that are formless, or creatures that are Huge or larger.
        A creature can use its action to make a DC 10 Strength check, freeing itself or another creature within its reach on a success.
        Dealing 5 slashing damage to the net (AC 10) also frees the creature without harming it, ending the effect and destroying the net.
        You can only make one attack with a net per turn.
    \paragraph{Nidhogg Tooth}
        Fashioned from the tooth of a nidhogg, this dagger's handle is leather-wrapped and has no crossguard.
        On a hit with this weapon, the target takes an extra 1d6 acid damage.
    \paragraph{Oathbow $\odot$}
        When you use this weapon to make a ranged attack, you can, as a command phrase, say, ``Swift death to you who have wronged me''.
        The target of your attack becomes your sworn enemy until it dies or until dawn seven days later.
        You can have only one such sworn enemy at a time.
        When your sworn enemy dies, you can choose a new one after the next dawn.

        When you make a ranged attack roll with this weapon against your sworn enemy, you have advantage on the roll.
        In addition, your target gains no benefit from cover, other than total cover, and you suffer no disadvantage due to long range.
        If the attack hits, your sworn enemy takes an extra 3d6 piercing damage.

        While your sworn enemy lives, you have disadvantage on attack rolls with all other weapons.
    \paragraph{Pistol}
        This weapon's misfire score is 1.
    \paragraph{Purger $\odot$}
        This fiery stone spear was designed to erradicate the red pox.
        When you hit a fiend with this weapon, that creature takes an extra 2d10 fire damage.

        As two actions, you can touch a wound inflicted by a fiend on a willing creature.
        The creature takes 1 fire damage, and any effect associated to the red pox is purged from it.
    \paragraph{Pyroconverger}
        A Pyroconverger is an Drer-made flamethrower.
        It carries a risk of malfunction each time you use it.

        As an action, you can cause the Pyroconverger to project fire in a 2-meter cone.
        Each creature in that area must make a DC 13 Dexterity saving throw, taking 4d6 fire damage on a failed save, or half as much damage on a successful one.

        Each time you use the Pyroconverger, roll a d10 and add the number of times you have used it since your last long rest.
        If the total is 11 or higher, the Pyroconverger malfunctions: you take 4d6 fire damage, and you can't use the Pyroconverger again until you finish a long rest.
    \paragraph{Speed Scimitar $\odot$}
        This weapon's multiple attack penalty is reduced by 1.
    \paragraph{Seeker Dart}
        This small dart is decorated with designs like windy spirals that span the length of its shaft.

        When you hurl this dart, it seeks out a target of your choice within 24 meters of you.
        You must have seen the target before, but you don't need to see it now.
        If the target isn't within range or if there is no clear path to it, the dart falls to the ground, destroyed and wasted.
        Otherwise, winds guide the dart instantly through the air to the target.
        The dart can pass though openings as narrow as 2 cm wide and can change direction to fly around corners.

        When the dart reaches its target, the target must succeed on a DC 16 Dexterity saving throw or take 1d4 piercing damage and 3d4 lightning damage.
        The dart breaks on impact.
    \paragraph{Serpent's Fang}
        This single-edged estoc is made from the scrimshawed fang of a giant serpent.
        The weapon deals an extra 1d10 poison damage to any target it hits.
    \paragraph{Smiting Mace}
        This mace gains a +2 bonus to attack and damage rolls made to attack a construct.

        When you roll a 20 on an attack roll made with this weapon, the target takes an extra 7 bludgeoning damage, or an extra 14 bludgeoning damage if it's a construct.
        If a construct has 25 hit points or fewer after taking this damage, it is destroyed.
    \paragraph{Storm Boomerang}
        This boomerang is a ranged weapon carved from griffon bone.
        On a hit, the boomerang deals 1d4 bludgeoning damage and 3d4 thunder damage, and the target must succeed on a DC 10 Constitution saving throw or be stunned until the end of its next turn.
        On a miss, the boomerang returns to the thrower's hand.

        Once the boomerang deals thunder damage to a target, the weapon loses its ability to deal thunder damage and its ability to stun a target.
        These properties return after an hour.
    \paragraph{Sunbow $\odot$}
        This poplar bow is carved in a design that resembles Khrusor, Heliod's spear.

        As an action, you can infuse an arrow with sunlight.
        Using another action, you can shoot this arrow to a point of your choice within the weapon's range.
        Upon hitting a target, the weapon vanishes in an explosion, and each creature in a 4-meter-radius sphere centered on that point must make a DC 15 Dexterity saving throw, taking 6d6 fire damage on a failed save, or half as much damage on a successful one.
        You can't infuse an arrow to explode again until you finish a short rest.
    \paragraph{Sword of Wounding $\odot$}
        Hit points lost to this longsword's damage can be regained only through a short or long rest, rather than by regeneration, magic, or any other means.

        Once per turn, when you hit a creature with an attack using this weapon, you can wound the target.
        At the start of each of the wounded creature's turns, it takes 1d4 necrotic damage for each time you've wounded it, and it can then make a DC 15 Constitution saving throw, ending the effect of all such wounds on itself on a success.
        Alternatively, the wounded creature, or a creature within 1 meter of it, can use two actions to make a DC 15 Wisdom (Medicine) check, ending the effect of such wounds on it on a success.
    \paragraph{Tartyx Blade $\odot$}
        The black blade of this sword is crafted from a mysterious alloy.
        \begin{itemize}
            \item \textbf{Dark Blessing}.
            While holding this sword, you can use an action to give yourself 1d4 + 4 temporary hit points.
            This property can't be used again until the next dusk.
            \item \textbf{Disheartening Strike}.
            When you hit a creature with an attack using this weapon, you can fill the target with unsettling dread: the target has disadvantage on the next saving throw it makes before the end of your next turn.
            The creature ignores this effect if it's immune to the frightened condition.
            Once you use this property, you can't do so again until the next dusk.
        \end{itemize}
    \paragraph{Terror Mace $\odot$}
        This weapon has 3 charges.
        While holding it, you can use two actions and expend 1 charge to release a wave of terror.
        Each creature of your choice in a 6-meter radius extending from you must succeed on a DC 15 Wisdom saving throw or become frightened of you for 1 minute.
        While it is frightened in this way, a creature must spend its turns trying to move as far away from you as it can, and it can't willingly move to a space within 6 meters of you.
        It also can't take reactions.
        For its action it can use only the Dash action or try to escape from an effect that prevents it from moving.
        If it has nowhere it can move, the creature can use the Dodge action.
        At the end of each of its turns, a creature can repeat the saving throw, ending the effect on itself on a success.

        The mace regains 1d3 expended charges daily at dawn.
    \paragraph{Thulkrakan Thrower $\odot$}
        When you hit with a ranged attack using this weapon, it deals an extra 1d8 damage or, if the target is a giant, 2d8 damage.
        Immediately after the attack, the weapon flies back to your hand.
    \paragraph{Tinderstrike $\odot$}
        A flint dagger, Tinderstrike is uncommonly sharp, and sparks cascade off its edge whenever it strikes something solid.
        Its handle is always warm to the touch, and the blade smolders for 1d4 minutes after it is used to deal damage.

        When you hit with this dagger, the target takes an extra 2d6 fire damage.
        In addition, you have resistance to fire damage while you hold Tinderstrike.
    \paragraph{Two-Birds Sling}
        When you make a ranged attack with this sling and hit a target, you can cause the ammunition to ricochet toward a second target within 2 meters of the first, and then make a ranged attack against the second target.
    \paragraph{Vampire's Kiss $\odot$}
        When you attack a creature with this rapier and roll a 20 on the attack roll, that target takes an extra 10 necrotic damage if it isn't a construct.
        You also gain 10 temporary hit points.
    \paragraph{Venom Dagger}
        You can use two actions to cause thick, black poison to coat the blade.
        The poison remains for 1 minute or until an attack using this weapon hits a creature.
        That creature must succeed on a DC 15 Constitution saving throw or take 2d10 poison damage and become poisoned for 1 minute.
        The dagger can't be used this way again until the next dawn.
    \paragraph{Vicious Weapon}
        When you roll a 20 with this weapon, the target takes an extra 7 damage of the weapon's type.
    \paragraph{Vorpal Glaive $\odot$}
        This glaive was originally commissioned by Grimtooth, a legendary bounty hunter.

        This weapon ignores resistance to slashing damage.
        In addition, when you attack a creature and roll a 20 on the attack roll, the creature takes an extra 6d8 slashing damage from the hit.
        In addition, if this critical hit inflicts a major injury, you can choose for the roll to be a 20, decapitating the creature.
    \paragraph{Waythe $\odot$}
        Waythe is a greatsword with an undulating blade.
        When you hit a creature of the giant type with Waythe, the giant takes an extra 2d6 slashing damage, and it must succeed on a DC 15 Strength saving throw or fall prone.
    \paragraph{Weapon of Certain Death}
        When you damage a creature with an attack using this magic weapon, the target can't regain hit points until the start of your next turn.
    \paragraph{Weapon of Warning $\odot$}
        This weapon warns you of danger.
        While the weapon is on your person, you have advantage on initiative rolls.
        In addition, you and any of your companions within 6 meters of you can't be surprised, except when incapacitated by something other than sleep.
        The weapon awakens you and your companions within range if any of you are sleeping naturally when combat begins.
    \paragraph{Woodcutter}
        When you use this axe to make an attack against a plant (an ordinary plant or a creature with the Plant type) or a wooden object, the attack deals an extra 2d6 slashing damage on a hit.
    \paragraph{Windvane $\odot$}
        A silver pike, Windvane has dark sapphires on the filigreed surface of its polished head.
        Held by its shining haft, the weapon feels insubstantial, as if clutching a cool, gently flowing breeze.

        When you hit with Windvane, the target takes an extra 1d6 lightning damage.
        In addition, you have resistance to lightning damage when you hold this weapon.

% SPECIAL WEAPONS:
% * Huge Krudzal weapons used to fight giants
% * Drer fire weapons
% * Sulia's large blast weapons (cannons and large handcannons)
% * Kaldrathal's more refined flintlock pistols
% * Mercury weapons from frostburn umans

% WEAPON CATEGORIES:
% Knives:
%   simple    : throwing dagger (L) 1d4, kukri 1d6 (only when thrown, less range)
%   1-handed  : parrying dagger 1
% Straight Swords:
%   versatile : bastard sword 1d8-1d10
%   2-handed  : claymore (H) 1d12, zweihander (H) 2d6, flamberge (H) 1d10
% Curved Swords:
%   simple    : cutlass 1d6
%   1-handed  : sabre 1d8, khopesh 1d6
%   2-handed  : katana 1d10
% Light Swords:
%   simple    : mail breaker 1d4
%   1-handed  : basket hilted sword 1d8
% Axes:
%   versatile : broad axe 1d8-1d10 (H)
%   2-handed  : bearded axe 1d10
% Hammer:
%   1-handed  : boomerang 1d4
% Flails:
%   1-handed  : nunchaku (L) 1d4, ball-and-chain 1d6
% Pistol:
%   1-handed  : flintlock R1 1d10
% "Newer" Firearms:
%   1-handed  : palm pistol (L) R1 1d8, pepperbox R6 1d10
%   2-handed  : blunderbuss R1 2d8, bad news R1 2d12
% Fire-breathers (Drer) - misfiring damages you:
%   2-handed  : firelance 1d12, hand mortar 2d8, flamespewer 1d6*
% Giant-slayers (Krudzal) - no extra attack:
%   2-handed  : dreihander 2d12, greatmaul 4d6, greatchisel 2d8*, hand ballista 2d10
%       dreihander    - https://mortalshell.wiki.fextralife.com/Martyr's+Blade
%       greatmaul     - https://mortalshell.wiki.fextralife.com/Smoldering+Mace
%       greatchisel   - https://mortalshell.wiki.fextralife.com/Hammer+and+Chisel
%       hand ballista - https://mortalshell.wiki.fextralife.com/Ballistazooka - action to shoot, action to reload
% Trick weapons: take 3 or 4 from here https://bloodborne.wiki.fextralife.com/Weapons.
% Exotic:
%   2-handed  : swordspear* (H) 1d10

% Take firearms from https://www.dndbeyond.com/subclasses/gunslinger

\newpage~\newpage

    % !TEX root = ../main.tex
\section{Wondrous Items} \label{sec::wondrousitems}
    \begin{table*}[b]%
        \begin{DndTable}[width=\linewidth, header=Wondrous Items]{Xcccccc}
            \textbf{Item} & \textbf{Rarity} & \textbf{Mats.} & \textbf{Total Cost} & \textbf{Tools} & \textbf{Weight} & \textbf{Source} \\
            Component Pouch                  & Common    & 1 &      25 agnomas & LEA & 1 kg   & PHB   151 \\
            Gem of Brightness                & Common    & 8 &     100 agnomas & JEW & 0.5 kg & DMG   171 \\
            Spell Scroll (Cantrip)           & Common    & 1 &      20 agnomas & CAL & ---    & DMG   199 \\
            Spell Scroll (1st-level)         & Common    & 2 &      30 agnomas & CAL & ---    & DMG   200 \\
            Spellbook                        & Common    & 3 &      50 agnomas & CAL & 1.5 kg & PHB   153 \\
            Decanter of Endless Water        & Uncommon  & 8 &     500 agnomas & GLA & 2 kg   & DMG   161 \\
            Fiend Sensor                     & Uncommon  & 4 &     300 agnomas & JEW & ---    & ---       \\
            Eversmoking Bottle               & Uncommon  & 8 &     500 agnomas & GLA & 0.5 kg & DMG   168 \\
            Firetrap                         & Uncommon  & 4 &     250 agnomas & GLA & 1 kg   & ---       \\
            Goggles of Night                 & Uncommon  & 8 &     500 agnomas & GLA & ---    & DMG   172 \\
            Pearl of Power                   & Uncommon  & 8 &     500 agnomas & JEW & ---    & DMG   184 \\
            Pharika's Ointment               & Uncommon  & 8 &     500 agnomas & COO & ---    & DMG   179 \\
            Spell Scroll (2nd-level)         & Uncommon  & 1 &     150 agnomas & CAL & ---    & DMG   201 \\
            Spell Scroll (3rd-level)         & Uncommon  & 2 &     200 agnomas & CAL & ---    & DMG   202 \\
            Stone of Good Luck               & Uncommon  & 8 &     500 agnomas & MAS & ---    & DMG   205 \\
            Stone of Ill Luck                & Uncommon  & 8 &     500 agnomas & MAS & ---    & TftYP 229 \\
            Zuan                             & Uncommon  & 1 &     150 agnomas & ALC & ---    & ---       \\
            Crystal Ball                     & Rare      & 8 &   5,000 agnomas & GLA & 1.5 kg & DMG   159 \\
            Feather Token: Anchor            & Rare      & 8 &   5,000 agnomas & JEW & ---    & DMG   188 \\
            Feather Token: Swan              & Rare      & 8 &   5,000 agnomas & CAR & ---    & DMG   188 \\
            Figurine of Wondrous Power       & Rare      & 8 &   5,000 agnomas & MAS & ---    & DMG   170 \\
            Folding Boat                     & Rare      & 4 &   3,000 agnomas & CAR & 2 kg   & DMG   170 \\
            Frithling Glass Home             & Rare      & 8 &   5,000 agnomas & GLA & 0.5 kg & ---       \\
            Gem of Seeing                    & Rare      & 8 &   5,000 agnomas & JEW & 0.5 kg & DMG   172 \\
            Horseshoes of Speed              & Rare      & 8 &   5,000 agnomas & SMI & ---    & DMG   175 \\
            Jamming Shackles                 & Rare      & 8 &   5,000 agnomas & CAR & ---    & DMG   165 \\
            Kruphix Stone (Rare)             & Rare      & 8 &   5,000 agnomas & MAS & ---    & DMG   176 \\
            Rod of Leadership                & Rare      & 8 &   5,000 agnomas & GLA & 1 kg   & DMG   197 \\
            Spell Scroll (4th-level)         & Rare      & 1 &   1,500 agnomas & CAL & ---    & DMG   203 \\
            Spell Scroll (5th-level)         & Rare      & 2 &   2,000 agnomas & CAL & ---    & DMG   204
        \end{DndTable}
    \end{table*}
    \begin{table*}[b]%
        \begin{DndTable}[width=\linewidth, header=Wondrous Items (Cont.)]{Xcccccc}
            Crystal of Absorption            & Very Rare & 8 &  50,000 agnomas & GLA & 1 kg.  & DMG   195 \\
            Crystalline Chronicle            & Very Rare & 8 &  50,000 agnomas & GLA & 1.5 kg & TCE   124 \\
            Kruphix Stone (Very Rare)        & Very Rare & 8 &  50,000 agnomas & MAS & ---    & DMG   176 \\
            Manual of Bodily Health          & Very Rare & 8 &  50,000 agnomas & CAL & 2.5 kg & DMG   180 \\
            Manual of Gainful Exercise       & Very Rare & 8 &  50,000 agnomas & CAL & 2.5 kg & DMG   180 \\
            Manual of Golems                 & Very Rare & 8 &  50,000 agnomas & CAL & 2.5 kg & DMG   180 \\
            Manual of Quickness of Action    & Very Rare & 8 &  50,000 agnomas & CAL & 2.5 kg & DMG   181 \\
            Rod of Alertness                 & Very Rare & 8 &  50,000 agnomas & GLA & 1 kg   & DMG   196 \\
            Shard of Nuagal                  & Very Rare & 8 &  50,000 agnomas & GLA & 1 kg   & DMG   197 \\
            Spell Scroll (6th-level)         & Very Rare & 1 &  15,000 agnomas & CAL & ---    & DMG   205 \\
            Spell Scroll (7th-level)         & Very Rare & 2 &  20,000 agnomas & CAL & ---    & DMG   206 \\
            Spell Scroll (8th-level)         & Very Rare & 4 &  30,000 agnomas & CAL & ---    & DMG   207 \\
            Tome of Clear Thought            & Very Rare & 8 &  50,000 agnomas & CAL & 2.5 kg & DMG   209 \\
            Tome of Leadership and Influence & Very Rare & 8 &  50,000 agnomas & CAL & 2.5 kg & DMG   209 \\
            Tome of Understanding            & Very Rare & 8 &  50,000 agnomas & CAL & 2.5 kg & DMG   209 \\
            Crystal Ball of Mind Reading     & Legendary & 6 & 200,000 agnomas & GLA & 1.5 kg & DMG   159 \\
            Crystal Ball of Telepathy        & Legendary & 6 & 200,000 agnomas & GLA & 1.5 kg & DMG   159 \\
            Crystal Ball of True Seeing      & Legendary & 6 & 200,000 agnomas & GLA & 1.5 kg & DMG   159 \\
            Kruphix Stone (Legendary)        & Legendary & 8 & 250,000 agnomas & MAS & ---    & DMG   176 \\
            Pearl of Nightfall               & Legendary & 8 & 250,000 agnomas & JEW & ---    & EGW   268 \\
            Rod of Might                     & Legendary & 8 & 250,000 agnomas & GLA & 1.5 kg & DMG   196 \\
            Sovereign Glue                   & Legendary & 8 & 250,000 agnomas & COO & ---    & DMG   200 \\
            Spell Scroll (9th-level)         & Legendary & 1 &  75,000 agnomas & CAL & ---    & DMG   208 \\
            Tome of the Stilled Tongue       & Legendary & 8 & 250,000 agnomas & CAL & 2.5 kg & DMG   208
        \end{DndTable}
    \end{table*}

    \paragraph{Component Pouch}
        A component pouch is a small, watertight leather belt pouch that has compartments to hold all the material components and other special items you need to cast your spells, except for those components that have a specific cost (as indicated in a spell's description).
    \paragraph{Crystal Ball $\odot$}
        Made by wandering zaloths, this item is infused with their strange magic.

        This crystal ball is about 15 centimeters in diameter.
        While touching it, you can cast the scrying spell (see page \pageref{spell::scrying}) with save DC 17 with it.
    \paragraph{Crystal of Absorption $\odot$}
        Designed by zaloths, this item is infused with their strange magic.

        While holding this crystal, you can use your reaction to absorb a spell that is targeting only you and not with an area of effect.
        The absorbed spell's effect is canceled, and the spell's energy --- not the spell itself --- is stored in the crystal.
        The energy has the same level as the spell when it was cast.
        The crystal can absorb and store up to 50 spell points worth of spells over the course of its existence.
        Once the crystal absorbs 50 spell points, it can't absorb more.
        If you are targeted by a spell that the crystal can't store, the crystal has no effect on that spell.

        When you become attuned to the crystal, you know how many spell points the crystal has absorbed over the course of its existence, and how many spell points it currently has stored.

        If you are a spellcaster holding the crystal, you can convert energy stored in it into spell points to cast spells you know.
        You can use the spell points stored in the crystal instead of your normal spell points, but otherwise cast the spell as normal.

        A newly found crystal has 1d10 spell points stored in it already.
        A crystal that can no longer absorb spells and has no spell points remaining becomes nonmagical.
    \paragraph{Crystal Ball of Mind Reading $\odot$}
        Made by wandering zaloths, this item is infused with their strange magic.

        This crystal ball is about 15 centimeters in diameter.
        While touching it, you can cast the scrying spell (see page \pageref{spell::scrying}) with save DC 17 with it.

        You can use an action to cast the detect thoughts spell (see page \pageref{spell::detectthoughts}) with a save DC of 17 while you are scrying with the crystal ball, targeting creatures you can see within 30 feet of the spell's sensor.
        You don't need to concentrate on this detect thoughts to maintain it during its duration, but it ends if scrying ends.
    \paragraph{Crystal Ball of Telepathy $\odot$}
        Made by wandering zaloths, this item is infused with their strange magic.

        This crystal ball is about 15 centimeters in diameter.
        While touching it, you can cast the scrying spell (see page \pageref{spell::scrying}) with save DC 17 with it.

        While scrying with the crystal ball, you can communicate telepathically with creatures you can see within 6 meters of the spell's sensor.
        You can also use an action to cast the suggestion spell (see page \pageref{spell::suggestion}) with save DC 17 through the sensor on one of those creatures.
        You don't need to concentrate on this suggestion to maintain it during its duration, but it ends if scrying ends.
        Once used, the suggestion power of the crystal ball can't be used again until the next dawn.
    \paragraph{Crystal Ball of True Seeing $\odot$}
        Made by wandering zaloths, this item is infused with their strange magic.

        This crystal ball is about 15 centimeters in diameter.
        While touching it, you can cast the scrying spell (see page \pageref{spell::scrying}) with save DC 17 with it.

        While scrying with the crystal ball, you have truesight with a radius of 24 meters centered on the spell's sensor.
    \paragraph{Crystalline Chronicle $\odot$}
        Made by wandering zaloths, this item is infused with their strange magic.
        You can only attune to this item if you have taken the spellcaster feat (page \pageref{feat::spellcaster}).

        An etched crystal sphere the size of a grapefruit hums faintly and pulses with irregular flares of inner light.
        The crystal contains the following spells: detect thoughts (see page \pageref{spell::detectthoughts}), intellect fortress (see page \pageref{spell::intellectfortress}), Rary's telepathic bond (see page \pageref{spell::rarystelepathicbond}), sending (see page \pageref{spell::sending}), telekinesis (see page \pageref{spell::telekinesis}), Tasha's mind whip (see page \pageref{spell::tashasmindwhip}), and Tenser's floating disk (see page \pageref{spell::tensersfloatingdisc}).
        While attuned to this item you know all these spells, and your knowledge of them fades when you end your attunement with it.

        While you are holding the crystal, you can use it as a spellcasting focus for these spells, and you know the mage hand (see page \pageref{spell::magehand}), mind sliver (see page \pageref{spell::mindsliver}), and message (see page \pageref{spell::message}) cantrips if you don't already know them.

        The crystal has 3 charges, and it regains 1d3 expended charges daily at dawn.
        You can use the charges to cast a spell given by this item without verbal, somatic, or material components of up to 100 agnomas.
    \paragraph{Decanter of Endless Water}
        This decanter can contain up to 8 liters of water, which is compressed to extreme lengths when inside.
        Upon being opened, a 6-meter long geyser gushes forth from the decanter.
        As an action while holding the decanter, you can aim the geyser at a creature you can see within 6 meters of you.
        The target must succed on a DC 12 Strength saving throw or take 1d4 bludgeoning damage and fall prone.
        Instead of a creature, you can target an object that isn't being worn or carried and that weighs no more than 100 kg.
        The object is either knocked over or pushed up to 3 meters away from you.

        To use this ability the decanter must be filled with water, a process that takes half a minute.
    \paragraph{Eversmoking Bottle}
        Smoke leaks from the lead-stoppered mouth of this glass bottle, which weighs 0.5 kg.
        When you use two actions to remove the stopper, a cloud of thick smoke pours out in a 12-meter radius from the bottle.
        The cloud's area is heavily obscured.
        Each minute the bottle remains open and within the cloud, the radius increases by 2 meters until it reaches its maximum radius of 24 meters.

        The cloud persists as long as the bottle is open.
        Closing the bottle requires an action.
        Once the bottle is closed, the cloud disperses after 10 minutes.
        A moderate wind can also disperse the smoke after 1 minute, and a strong wind can do so after 1 round.
    \paragraph{Firetrap}
        A firetrap is an spherical glass container about the size of a bottle.
        A flamespeaker can use a firetrap as a spellcasting focus, and is the required material component to cast the \textbf{Screaming Bead} spell (see page \pageref{spell::screamingbead}).
    \paragraph{Folding Boat}
        This object appears as a wooden box that measures 30 cm long, 15 cm wide, and 15 cm deep.
        It weighs 2 kilograms and floats.
        It can be opened to store items inside.

        Using two actions, the box can be unfolded into a boat 3 meters long, 1 meter wide, and 60 cm deep.
        The boat has one pair of oars, a mast, and a lateen sail.
        The boat can hold up to four Medium creatures comfortably.

        Using two actions, the folding boat can fold back into a box, provided that no creatures are aboard.
    \paragraph{Fiend Sensor}
        This small ruby glows slightly when it is within 24 meters of a fiend.
    \paragraph{Figurine of Wondrous Power}
        This figurin is a statuette of a beast small enough to fit in a pocket.
        If you use an action to throw the figurine to a point on the ground within 12 meters of you, the figurine becomes a living creature.
        If the space where the creature would appear is occupied by other creatures or objects, or if there isn't enough space for the creature, the figurine doesn't become a creature.

        The creature is friendly to you and your companions.
        It understands your languages and obeys your spoken commands.
        If you issue no commands, the creature defends itself but takes no other actions.

        The creature exists for a duration specific to each figurine.
        At the end of the duration, the creature reverts to its figurine form.
        It reverts to a figurine early if it drops to 0 hit points or if you use an action to speak the command word again while touching it.
        When the creature becomes a figurine again, its property can't be used again until a week has passed.

        \begin{itemize}
            \item \textbf{Bronze Griffon}.
            This bronze statuette is of a griffon rampant.
            It can become a griffon for up to 6 hours.
            \item \textbf{Golden Lions}.
            These gold statuettes of lions are always created in pairs.
            You can use one figurine or both simultaneously.
            Each can become a lion for up to 1 hour.
            \item \textbf{Ivory Goats}.
            These ivory statuettes of goats are always created in sets of three.
            Each goat looks unique and functions differently from the others.
            Their properties are as follows:
            \begin{itemize}
                \item The goat of traveling can become a Large goat with the same statistics as a riding horse.
                It lasts for 24 hours.
                \item The goat of travail becomes a giant goat for up to 3 hours.
                Once it has been used, it can't be used again until 30 days have passed.
                \item The goat of terror becomes a giant goat for up to 3 hours.
                The goat can't attack, but you can remove its horns and use them as weapons.
                One horn becomes a +1 lance, and the other becomes a +2 longsword.
                Removing a horn requires an action, and the weapons disappear and the horns return when the goat reverts to figurine form.

                In addition, the goat radiates a 6-meter-radius aura of terror while you are riding it.
                Any creature hostile to you that starts its turn in the aura must succeed on a DC 15 Wisdom saving throw or be frightened of the goat for 1 minute, or until the goat reverts to figurine form.
                The frightened creature can repeat the saving throw at the end of each of its turns, ending the effect on itself on a success. Once it successfully saves against the effect, a creature is immune to the goat's aura for the next 24 hours.
                Once the figurine has been used, it can't be used again until 15 days have passed.
            \end{itemize}
            \item \textbf{Onyx Dog}.
            This onyx statuette of a dog can become a mastiff for up to 6 hours.
            The mastiff has an Intelligence of 8.
            The dog has darkvision out to a range of 12 meters and can see invisible creatures and objects within that range.
            \item \textbf{Silver Raven}.
            This silver statuette of a raven can become a raven for up to 12 hours.
            While in raven form, you can attach a letter to the raven's leg and tell it a location you know within 12 hours of travel, and the raven will go there to relay the message.
        \end{itemize}
    \paragraph{Frithling Glass Home}
        This colored glass bottle weights 0.5 kg.
        The bottle can serve as a comfortable home for up to 8 frithlings (see page \pageref{creature::frithling}) as long as the frithlings get along with each other.
        The frithlings are thankful towards you for giving them a home and taking them in your adventures, and can perform tasks asked by you within their ability, and will ocassionally give you favors or gifts for your friendship.
        The frithlings need to eat and drink, each consuming 1/8th of the rations and doses of water that a normal-sized humanoid needs.

        If you fail to provide the frithlings basic needs or constantly put them in danger they will abandon you, usually without warning.
    \paragraph{Gem of Brightness}
        This blue gem sheds bright light in a 6-meter radius and dim light for an additional 6 meters.
    \paragraph{Gem of Seeing}
        This gem is attached to a short wooden pole.
        While peering through the gem, you gain truesight out to a range of 24 meters.
    \paragraph{Goggles of Night}
        While wearing these dark lenses, you have darkvision out to a range of 12 meters.
        If you already have darkvision, wearing the goggles increases its range by 12 meters.
    \paragraph{Horseshoes of Speed}
        These iron horseshoes come in a set of four.
        While all four shoes are affixed to the hooves of a horse or similar creature, they increase the creature's walking speed by 6 meters.
    \paragraph{Jamming Shackles}
        You can use an action to place these shackles on an incapacitated creature.
        The shackles adjust to fit a creature of Small to Large size.
        In addition to serving as mundane manacles, the shackles prevent a creature bound by them from using any method of movement, including teleportation or travel to a different plane of existence.

        You and any creature you designate when you use the shackles can use an action to remove them.
        Once every 30 days, the bound creature can make a DC 30 Strength (Athletics) check.
        On a success, the creature breaks free and destroys the shackles.
    \paragraph{Kruphix Stone $\odot\odot$}
        Many types of Kruphix stones exist, each type a distinct combination of shape and color.

        When you use an action to toss one of these stones into the air, the stone orbits your head at a distance of 30 centimeters and confers a benefit to you.
        Thereafter, another creature must use an action to grasp or net the stone to separate it from you, either by making a successful attack roll against AC 24 or a successful DC 24 Dexterity (Acrobatics) check.
        You can use an action to seize and stow the stone, ending its effect.

        A stone has AC 24, 10 hit points, and resistance to all damage.
        It is considered to be an object that is being worn while it orbits your head.

        Attuning to a Kruphix stone requires two attunement slots.
        You can carry any number of Kruphix stones using the same attunement slots.
        You cannot attune to more than one Kruphix stone of the same type at once.

        Rare Kruphix stones:
        \begin{itemize}
            \item \textbf{Awareness}.
            You can't be surprised while this dark blue rhomboid orbits your head.
            \item \textbf{Protection}.
            You gain a +1 bonus to AC while this dusty rose prism orbits your head.
            \item \textbf{Reserve}.
            While this stone orbits your head, you can cast any spell stored in it.
            The spell uses the slot level, spell save DC, spell attack bonus, and spellcasting ability of the original caster, but is otherwise treated as if you cast the spell.
            The spell cast from the stone is no longer stored in it, freeing up space.
            \item \textbf{Sustenance}.
            As long as this clear spindle spends at least 4 hours under direct, unobstructed sunlight, you don't need to eat or drink while it orbits your head.
        \end{itemize}

        Very rare Kruphix stones:
        \begin{itemize}
            \item \textbf{Absorption}.
            While this pale lavender ellipsoid orbits your head, you can use your reaction to cancel a spell of 4th level or lower cast by a creature you can see and targeting only you.

            Once the stone has canceled 20 levels of spells, it burns out and turns dull gray, losing its magic.
            If you are targeted by a spell whose level is higher than the number of spell levels the stone has left, the stone can't cancel it.
            \item \textbf{Agility}.
            Your Dexterity score increases by 2, to a maximum of 22, while this deep red sphere orbits your head.
            \item \textbf{Fortitude}.
            Your Constitution score increases by 2, to a maximum of 22, while this pink rhomboid orbits your head.
            \item \textbf{Insight}.
            Your Wisdom score increases by 2, to a maximum of 22, while this incandescent blue sphere orbits your head.
            \item \textbf{Intellect}.
            Your Intelligence score increases by 2, to a maximum of 22, while this marbled scarlet and blue sphere orbits your head.
            \item \textbf{Leadership}.
            Your Charisma score increases by 2, to a maximum of 22, while this marbled pink and green sphere orbits your head.
            \item \textbf{Might}.
            Your Strength score increases by 2, to a maximum of 22, while this pale blue rhomboid orbits your head.
        \end{itemize}

        Legendary Kruphix stones:
        \begin{itemize}
            \item \textbf{Greater Absorption}.
            While this marbled lavender and green ellipsoid orbits your head, you can use your reaction to cancel a spell of 8th level or lower cast by a creature you can see and targeting only you.

            Once the stone has canceled 50 levels of spells, it burns out and turns dull gray, losing its magic.
            If you are targeted by a spell whose level is higher than the number of spell levels the stone has left, the stone can't cancel it.
            \item \textbf{Proficiency}.
            Your proficiency bonus increases by 1 while this pale green prism orbits your head.
            \item \textbf{Regeneration}.
            You regain 5 hit points at the end of each hour this pearly white spindle orbits your head, provided that you have at least 1 hit point.
        \end{itemize}
    \paragraph{Manual of Bodily Health}
        This book contains health and diet tips from all around the world.
        If you spend 6 long rests studying the book's contents and practicing its guidelines, your Constitution score increases by 2, as does your maximum for that score.
        A creature can benefit from this manual only once in its life.
    \paragraph{Manual of Gainful Exercise}
        This book describes fitness exercises from the harshest mountains of the north.
        If you spend 6 long rests studying the book's contents and practicing its guidelines, your Strength score increases by 2, as does your maximum for that score.
        A creature can benefit from this manual only once in its life.
    \paragraph{Manual of Golems}
        This tome contains information and incantations necessary to make a golem.
        To decipher and use the manual, you must be a spellcaster with at least 60 spell points.
        A creature that can't use a manual of golems and attempts to read it takes 6d6 psychic damage.

        To create a golem, you must spend 10 long rests working without interruption with the manual at hand and resting no more than 8 hours per day.
        You must also pay 50,000 agnomas to purchase supplies.
        The golem becomes animate when the ashes of the manual are sprinkled on it.
        It is under your control, and it understands and obeys your spoken commands.
    \paragraph{Manual of Quickness of Action}
        This book contains coordination and balance exercises worthy of the most elegant of the circus-folk.
        If you spend 6 long rests studying the book's contents and practicing its guidelines, your Dexterity score increases by 2, as does your maximum for that score.
        A creature can benefit from this manual only once in its life.
    \paragraph{Pearl of Nightfall $\odot$}
        Used to summon night, this 15-centimeter-diameter, jet-black orb is cold to the touch.
        You can spend 10 minutes to activate it, causing the area within 15 kilometer of it at the moment of activation to become night even if it is daytime.
        This night lasts for 24 hours, until you cancel it as an action, or until your attunement to the pearl ends.
        Once used, the pearl can't be used again for 24 hours.
    \paragraph{Pearl of Power $\odot$}
        While this pearl is on your person, you can use an action to regain 5 spell points.
        Once you have used the pearl, it can't be used again until the next dawn.
    \paragraph{Pharika's Ointment}
        This glass jar, 7 centimeters in diameter, contains 5 doses of a thick mixture that smells faintly of aloe.

        Taking up one minute, one dose of the ointment can be swallowed or applied to the skin.
        The creature that receives it regains 2d8 + 2 hit points and ceases to be poisoned.
    \paragraph{Rod of Alertness $\odot$}
        This glass rod has a flanged head.
        It has the following properties.
        \begin{itemize}
            \item \textbf{Alertness}
            While holding the rod, you have advantage on Wisdom (Perception) checks and on rolls for initiative.
            \item \textbf{Protective Aura}
            Using two actions, you can plant the haft end of the rod in the ground, whereupon the rod's head sheds bright light in a 12-meter radius and dim light for an additional 12 meters.
            While in that bright light, you and any creature that is friendly to you gain a +1 bonus to AC and saving throws and can sense the location of any invisible hostile creature that is also in the bright light.
            The rod's head stops glowing and the effect ends after 10 minutes, or when a creature uses an action to pull the rod from the ground.
            This property can't be used again until the next dawn.
        \end{itemize}
    \paragraph{Rod of Leadership $\odot$}
        Said to be designed by king Olag for Khedrat's top military personnel, this rod can command respect from anyone around you.
        You can use two actions to present the rod and command obedience from each creature of your choice that you can see within 24 meters of you.
        Each target must succeed on a DC 15 Wisdom saving throw or be charmed by you for 8 hours.
        While charmed in this way, the creature regards you as its trusted leader.
        If harmed by you or your companions, or commanded to do something contrary to its nature, a target ceases to be charmed in this way.
        The rod can't be used in this way again until the next dawn.
    \paragraph{Rod of Might $\odot$}
        This glass rod has a straight and unassuming shape, and it functions as a mace that grants a +3 bonus to attack and damage rolls made with it.
        An item of intricate design, you can transform this rod into four different forms as an action.
        As another action you can transform the rod back to its original form or to another of its special forms.
        \begin{itemize}
            \item \textbf{Spear}.
            The rod's head folds down, a spear point springs from the rod's tip, and the rod's handle lengthens into a 1.8-meter haft, transforming the rod into a spear that grants a +3 bonus to attack and damage rolls made with it.
            \item \textbf{Ladder}.
            The rod transforms into a climbing pole up to 10 meters long, as you specify.
            In surfaces as hard as granite, a spike at the bottom and three hooks at the top anchor the pole.
            Horizontal bars 8 centimeters long fold out from the sides, 30 centimeters apart, forming a ladder.
            The pole can bear up to 2,000 kg.
            More weight or lack of solid anchoring causes the rod to revert to its normal form.
            \item \textbf{Ram}.
            The rod transforms into a handheld battering ram and grants its user a +10 bonus to Strength checks made to break through doors, barricades, and other barriers.
            \item \textbf{Compass}.
            The rod assumes or remains in its normal form and indicates magnetic north (Nothing happens if this function of the rod is used in a location that has no magnetic north).
        \end{itemize}

        In addition, you can add one of the following effect to this rod's attacks by expending an extra action:
        \begin{itemize}
            \item \textbf{Drain Life}.
            When you hit a creature with a melee attack using the rod, you can force the target to make a DC 17 Constitution saving throw.
            On a failure, the target rakes an extra 4d6 necrotic damage, and you regain a number of hit points equal to half that necrotic damage.
            This property can't be used again until the next dawn.
            \item \textbf{Paralyze}.
            When you hit a creature with a melee attack using the rod, you can force the target to make a DC 17 Strength saving throw.
            On a failure, the target is paralyzed for 1 minute.
            The target can repeat the saving throw at the end of each of its turns, ending the effect on a success.
            This property can't be used again until the next dawn.
            \item \textbf{Terrify}.
            While holding the rod, you can use an action to force each creature you can see within 6 meters of you to make a DC 17 Wisdom saving throw.
            On a failure, a target is frightened of you for 1 minute.
            A frightened target can repeat the saving throw at the end of each of its turns, ending the effect on itself on a success.
            This property can't be used again until the next dawn.
        \end{itemize}
    \paragraph{Shard of Nuagal}
        The origins of these shards are unknown, and only a master glassblower can evoke the following effects from it.

        While holding this shard, you can use an action to activate it.
        The shard then instantly transports you and up to 199 other willing creatures you can see to an island in Nuagal, the blue moon.
        The island is a tranquil garden, and contains enough water and food to sustain its visitors.

        For each hour spent in the island, a visitor regains hit points as if it had spent 1 hit die.
        Visitors remain in the island for up to 6 days.

        When the time runs out or you use an action to end it, all visitors reappear in the location they occupied when you activated the shard, or an unoccupied space nearest that location.
        The shard loses its power when it is used in this way.
    \paragraph{Sovereign Glue} \label{item::sovereignglue}
        This viscous, milky-white substance can form a permanent adhesive bond between any two objects.
        It must be stored in a jar or flask that has been coated inside with oil of slipperiness (see page \pageref{item::oilofslipperiness}).
        A container contains 7 uses.

        One use of the glue can cover a 50 centimeters square surface.
        The glue takes 1 minute to set.
        Once it has done so, the bond it creates can be broken only by the application of universal solvent (see page \pageref{item::universalsolvent}).
    \paragraph{Spell Scroll} \label{item::spellscroll}
        A spell scroll bears the words of a single spell, written as a cipher.
        You can read the scroll and cast its spell without providing any material components.
        Casting the spell by reading the scroll requires the spell's normal casting time.
        Once the spell is cast, the words on the scroll fade, and it crumbles to dust.
        If the casting is interrupted, the scroll is not lost.

        To cast the spell, you must make an ability check using your spellcasting ability to determine whether you cast it successfully.
        The DC depends on the scroll rarity.
        If you already know the spell, you don't need to succeed on this DC.
        On a failed check, the spell disappears from the scroll with no other effect.

        Once the spell is cast, the words on the scroll fade, and the scroll itself crumbles to dust.

        The DC and attack bonus from spell scrolls are specified in the spell scroll table.

        A spell on a spell scroll can be copied just as spells in spellbooks can be copied.
        A creature proficient with calligrapher's supplies can copy a spell from a spell scroll.
        The copier must succeed on an Intelligence (Arcana) check with a DC equal to the Casting DC of the scroll.
        If the check succeeds, the spell is successfully copied.
        Whether the check succeeds or fails, the spell scroll is destroyed.

        When writing a spell scroll, you decide the language in which is written among the languages you know.
        To read a spell scroll, you must speak the language in which it's written.

        \begin{DndTable}[width=\linewidth, header=Spell Scrolls]{Xccc}
            \textbf{Level} & \textbf{Casting DC} & \textbf{Spell Save DC} & \textbf{Attack Bonus} \\
            Cantrip & 10 & 13 &  +5 \\
            1       & 11 & 13 &  +5 \\
            2       & 12 & 13 &  +5 \\
            3       & 13 & 15 &  +7 \\
            4       & 14 & 15 &  +7 \\
            5       & 15 & 17 &  +9 \\
            6       & 16 & 17 &  +9 \\
            7       & 17 & 18 & +10 \\
            8       & 18 & 18 & +10 \\
            9       & 19 & 19 & +11
        \end{DndTable}
    \paragraph{Spellbook} \label{item::spellbook}
        A spellbook is a leather-bound tome with 100 blank vellum pages suitable for recording spells.
        A creature capable of spellcasting can store any spell they know in the book.
        As part of a long rest, the creature can change its known spells with any set of spells written on the book, as long as they fulfill all the requirements to cast each spell.

        As part of a short rest, a creature proficient with calligrapher's supplies can attempt to copy one spell from one spellbook to another, or from a spell scroll to a spellbook.
        The Intelligence (Arcana) DC required to do this is the same as the Casting DC specified in the spell scrolls table.
    \paragraph{Stone of Good Luck $\odot$}
        While this polished agate is on your person, you gain a +1 bonus to ability checks and saving throws.
    \paragraph{Stone of Ill Luck $\odot$}
        This polished agate appears to be a stone of good luck to anyone who tries to identify it, and it confers that item's property while on your person.

        While this item is on your person, you take a -2 penalty to ability checks and saving throws.
        The DM secretly applies this penalty, assuming you are adding the item's bonus.
    \paragraph{Tome of Clear Thought}
        This book contains memory and logic exercises worthy of the great oth scholars of yore.
        If you spend 6 long rests studying the book's contents and practicing its guidelines, your Intelligence score increases by 2, as does your maximum for that score.
        A creature can benefit from this manual only once in its life.
    \paragraph{Tome of Leadership and Influence}
        This book contains guidelines for influencing and charming others.
        If you spend 6 long rests studying the book's contents and practicing its guidelines, your Charisma score increases by 2, as does your maximum for that score.
        A creature can benefit from this manual only once in its life.
    \paragraph{Tome of the Stilled Tongue $\odot$}
        When you attune to this item, you can use it as a spellbook.
        In addition, while holding the tome, you can use one action to cast a spell you have written in this tome, without expending spell points or using any verbal or somatic component.
        Once used, this property of the tome can't be used again until the next dawn.

        While attuned to this book, only you can read the spells written on it; all other creatures see random scribbles.
        When you end your attunement with the item, all words written fade from it.
    \paragraph{Tome of Understanding}
        This book contains intuition and insight exercises.
        If you spend 6 long rests studying the book's contents and practicing its guidelines, your Wisdom score increases by 2, as does your maximum for that score.
        A creature can benefit from this manual only once in its life.
    \paragraph{Zuan $\odot$} \label{item::zuan}
        A zuan is a pearl infused by an alchemist to act as the medium for their spells.
        Once you attune to a zuan, you form a sympathetic bond with it, allowing you to move it through the air just by willing it.
        As a free action on your turn, you can move your zuan in any direction up to 6 meters.
        You zuan cannot be more than 24 meters away from you at any time.


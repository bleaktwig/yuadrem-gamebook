% !TEX root = ../main.tex
\subsection*{Potions} \label{ssec::potions}
Potions are small amounts of potent liquid contained in a flask or vial.
A creature can drink a potion using two actions, or it can throw the container up to 4/12 meters as an action.
This is considered a ranged weapon and any feat that applies to thrown weapons applies to it.

A flask can contain the same amount of liquid as 4 vials.

\begin{table*}[b]%
    \begin{DndTable}[width=\linewidth, header=Potions]{llll}
        \textbf{Item}                      & \textbf{Rarity} & \textbf{Cost}  & \textbf{Weigtht} \\
        Acid (Vial)                        & Common          &     25 agnomas & ---              \\
        Alchemist's Fire (Flask)           & Common          &     50 agnomas & 0.5 kg.          \\
        Antitoxin (Vial)                   & Common          &     50 agnomas & ---              \\
        Caustic Brew (Flask)               & Common          &     25 agnomas & 0.5 kg.          \\
        Flask                              & Common          &      2 fobs    & 0.5 kg.          \\
        Oil (Flask)                        & Common          &       10 fobs  & 0.5 kg.          \\
        Perfume (Vial)                     & Common          &      5 agnomas & ---              \\
        Potion of Healing (Flask)          & Common          &     50 agnomas & 0.5 kg.          \\
        Potion of Sickness (Flask)         & Common          &     25 agnomas & 0.5 kg.          \\
        Soap                               & Common          &      2 fobs    & ---              \\
        Vial                               & Common          &      1 agnoma  & ---              \\
        Bloodwell Vial +1                  & Uncommon        &    200 agnomas & ---              \\
        Bottled Acid Arrow (Flask)         & Uncommon        &    100 agnomas & 0.5 kg.          \\
        Coldblood (Vial)                   & Uncommon        &    100 agnomas & ---              \\
        Oil of Slipperiness (Vial)         & Uncommon        &    100 agnomas & ---              \\
        Potion of Greater Healing (Flask)  & Uncommon        &    100 agnomas & 0.5 kg.          \\
        Potion of Resistance (Flask)       & Uncommon        &    100 agnomas & 0.5 kg.          \\
        Bloodwell Vial +2                  & Rare            &  2,000 agnomas & ---              \\
        Bottled Fire (Flask)               & Rare            &  1,000 agnomas & 0.5 kg.          \\
        Elixir of Health (Flask)           & Rare            &  1,000 agnomas & 0.5 kg.          \\
        Flask of Mass Healing (Flask)      & Rare            &  1,000 agnomas & 0.5 kg.          \\
        Oil of Burning (Vial)              & Rare            &  1,000 agnomas & ---              \\
        Potion of Aqueous Form (Vial)      & Rare            &  1,000 agnomas & ---              \\
        Potion of Superior Healing (Flask) & Rare            &  1,000 agnomas & 0.5 kg.          \\
        Bloodwell Vial +3                  & Very Rare       & 20,000 agnomas & ---              \\
        Bottled Blight (Vial)              & Very Rare       & 10,000 agnomas & ---              \\
        Flask of Weakness (Flask)          & Very Rare       & 10,000 agnomas & 0.5 kg.          \\
        Healing Cloud (Flask)              & Very Rare       & 10,000 agnomas & 0.5 kg.          \\
        Liquid Death (Flask)               & Very Rare       & 10,000 agnomas & 0.5 kg.          \\
        Oil of Immolation (Vial)           & Very Rare       & 10,000 agnomas & ---              \\
        Oil of Sharpness (Vial)            & Very Rare       & 10,000 agnomas & ---              \\
        Oozekiller (Vial)                  & Very Rare       & 10,000 agnomas & ---              \\
        Potion of Contagion (Flask)        & Very Rare       & 10,000 agnomas & 0.5 kg.          \\
        Potion of Supreme Healing (Flask)  & Very Rare       & 10,000 agnomas & 0.5 kg.          \\
        Bottled Tizerus (Flask)            & Legendary       & 50,000 agnomas & 0.5 kg.          \\
        Potion of Condensed Life (Flask)   & Legendary       & 50,000 agnomas & 0.5 kg.          \\
        Potion of Regeneration (Vial)      & Legendary       & 50,000 agnomas & ---              \\
        Vial of Pain (Vial)                & Legendary       & 50,000 agnomas & ---
    \end{DndTable}
\end{table*}

\paragraph{Acid (Vial)}
    As an action, you can splash the contents of this vial onto a creature within 1 meter of you or throw the vial up to 4 meters, shattering it on impact.
    In either case, make a ranged attack against a creature or object, treating the acid as an improvised weapon.
    On a hit, the target takes 2d6 acid damage.
\paragraph{Alchemist's Fire (Flask)}
    This sticky, adhesive fluid ignites when exposed to air.
    As an action, you can throw this flask up to 4 meters, shattering it on impact.
    Make a ranged attack against a creature or object, treating the alchemist's fire as an improvised weapon.
    On a hit, the target takes 1d4 fire damage at the start of each of its turns.
    A creature can end this damage by using its action to make a DC 10 Dexterity check to extinguish the flames.
\paragraph{Antitoxin (Vial)}
    A creature that drinks this vial of liquid gains advantage on saving throws against poison for 1 hour.
    It confers no benefit to undead or constructs.
\paragraph{Bloodwell Vial $\odot$}
    To attune to this vial, you must fill it with coldblood and add a few drops of your own blood.
    As long as this vial is close to your spellcasting focus, you gain a bonus to spell attack rolls and to the saving throw DCs of your spells as specified on the item's name.

    This vial has 100 charges.
    When all these charges are expended, your attunement with the item is lost.
\paragraph{Bottled Acid Arrow (Flask)} % Melf's acid arrow.
    Upon being opened, a shimmering green arrow streaks from the bottle opening in a line and bursts in a spray of acid.
    Make attack against a target within 18 meters.
    On a hit, the target takes 4d4 acid damage immediately and 2d4 acid damage at the end of its next turn.
    On a miss, the arrow splashes the target with acid for half as much of the initial damage and no damage at the end of its next turn.
\paragraph{Bottled Blight (Vial)} % 5th-level Blight.
    Upon being opened, this bottle releases a sickening gas that drains moistury and vitality.
    The gas fills a 1.5-meter square before dispersing.

    Any creature standing in this square must make a DC 16 Constitution saving throw.
    It takes 9d8 necrotic damage on a failed save, or half as much damage on a successful one.
    This potion has no effect on undead or constructs.

    If you target a plant creature or a magical plant, it makes the saving throw with disadvantage, and the potion deals maximum damage to it.

    If you target a nonmagical plant that isn't a creature, such as a tree or shrub, it doesn't make a saving throw, it simply withers and dies.
\paragraph{Bottled Fire (Flask)} % Fireball.
    Upon breaking, a bright spark blossoms with a low roar into an explosion of flame.
    Each creature in a 6-meter-radius sphere centered on that point must make a Dexterity saving throw with DC 14.
    A target takes 8d6 fire damage on a failed save, or half as much damage on a successful one.

    The fire spreads around corners.
    It ignites flammable objects in the area that aren't being worn or carried.
\paragraph{Bottled Tizerus (Flask)} % Delayed Blast Fireball
    This flask contains a tiny spark, patiently waiting to be awakened.
    You can shake the bottle to activate this spark.
    For the following minute, the bead continues to grow ever more violent.
    Upon breaking or when the minute passes, it blossoms with a low roar into an explosion of flame that spreads around corners.
    Each creature in a 6-meter radius sphere centered on that point must make a DC 18 Dexterity saving throw.
    A creature takes fire damage equal to the total accumulated damage on a failed save, or half as much damage on a successful one.

    The potions's base damage is 12d6.
    If at the end of your turn the bead has not yet detonated, the damage increases by 1d6.

    The glowing bead can be taken from the flask without causing it to explode.
    The creature touching it must make a DC 12 Dexterity saving throw.
    On a failed save, the spark erupts in flame.
    On a successful save, the creature can throw the bead up to 8 meters.
    When it strikes a creature or a solid object, the bead explodes.

    The fire damages objects in the area and ignites flammable objects that aren't being worn or carried.
\paragraph{Caustic Brew (Flask)} % Tasha's Caustic Brew
    Upon being shaken and breaking, this flask explodes in a radius 2 meters.
    Each creature in the cloud must succeed on a DC 10 Dexterity saving throw or be covered in acid for a minute or until a creature uses two actions to scrape or wash the acid off itself or another creature.
    A creature covered in the acid takes 2d4 acid damage at start of each of its turns.
\paragraph{Elixir of Health (Flask)}
    When you drink this potion, it cures any disease afflicting you, and it removes the blinded, deafened, paralyzed, and poisoned conditions.
    The clear red liquid has tiny bubbles of light in it.
\paragraph{Flask of Mass Healing (Flask)} % Mass healing word.
    Upon breaking, all creatures within a 6-meter radius regain hit points equal to 1d4 + 3.
    This potion has no effect on undead or constructs.
\paragraph{Flask of Weakness (Flask)} % Elemental Bane.
    This flask comes in 5 flavors: acid, cold, fire, lightning, and thunder.
    Upon being opened, this flask releases a noxious gas in a 10-meter square.
    Any creature who breathes the gas must succeed on a DC 16 Constitution saving throw or be affected by a weakening effect for a minute.
    The first time each turn the affected creature takes damage of the potion's type, the target takes an extra 2d6 damage of the type.
    Moreover, the target loses any resistance to that damage until the effect ends.
\paragraph{Healing Cloud (Flask)} % Mass Cure Wounds.
    Upon being opened, this potion releases a wave of healing energy in the form of a flowery smell.
    Any creature standing in a 9-meter radius sphere centered on the potion's location regains hit points equal to 3d8 + 4.
    This potion has no effect on undead or constructs.
\paragraph{Liquid Death (Flask)} % Cloudkill.
    Upon breaking, this flask releases a 6-meter radius sphere of poisonous, yellow-green fog.
    The fog spreads around corners.
    It lasts for 10 minutes or until strong wind disperses the fog, ending the effect.
    Its area is heavily obscured.

    When a creature enters the fog's area for the first time on a turn or starts its turn there, that creature must make a DC 16 Constitution saving throw.
    The creature takes 5d8 poison damage on a failed save, or half as much damage on a successful one.
    Creatures are affected even if they hold their breath or don't need to breathe.

    % The fog moves 3 meters away from you at the start of each of your turns, rolling along the surface of the ground.
    % The vapors, being heavier than air, sink to the lowest level of the land, even pouring down openings.
\paragraph{Oil (Flask)}
    As an action, you can splash the oil in this flask onto a creature within 1 meter of you or throw it up to 4 meters, shattering it on impact.
    Make a ranged attack against a target creature or object, treating the oil as an improvised weapon.
    On a hit, the target is covered in oil.
    The oil dries after one minute, and for this duration the target gains vulnerability to fire damage.
    You can also pour a flask of oil on the ground to cover a 1.5-meter-square area, provided that the surface is level.
    If lit, the oil burns for 2 rounds and deals 5 fire damage to any creature that enters the area or ends its turn in the area.
    A creature can take this damage only once per turn.
\paragraph{Oil of Burning (Vial)} % Flame arrows.
    This brown sticky oil quickly heats up when accelerated.
    By drenching an arrow or bolt in the oil as an action, the piece of ammunition gains an additional 1d6 fire damage on hit.
    The oil evapores on a piece of ammunition when it hits or misses.
    Using two actions, you can pour the liquid into up to twelve pieces of ammunition at once.
\paragraph{Oil of Immolation (Vial)} % Immolation.
    Upon being opened, the vial explodes into all consuming flames in a 1.5-meter radius sphere.
    Any creature within this sphere is wreathed by flames.
    It must make a DC 16 Dexterity saving throw.
    It takes 8d6 fire damage on a failed save, or half as much damage on a successful one.
    On a failed save, the target also burns for 1 minute.
    The burning target sheds bright light in a 9-meter radius and dim light for an additional 6 meters.
    At the end of each of its turns, the target repeats the saving throw.
    It takes 4d6 fire damage on a failed save, and the effect ends on a successful one.
    These flames can't be extinguished by other means.

    If damage from this spell kills a target, the target is turned to ash.
\paragraph{Oil of Sharpness (Vial)}
    This clear, gelatinous oil sparkles with tiny, ultrathin silver shards.
    The oil can coat one slashing or piercing weapon or up to 5 pieces of slashing or piercing ammunition.
    Applying the oil takes 1 minute.
    For 1 hour, the coated item has a +3 bonus to attack and damage rolls.
\paragraph{Oil of Slipperiness (Vial)} % Freedom of Movement + Grease
    This sticky black unguent is thick and heavy in the container, but it flows quickly when poured.
    The oil can cover a Medium or smaller creature, along with the equipment it's wearing and carrying (one additional vial is required for each size category above Medium).
    Applying the oil takes 10 minutes.

    The affected creature's movement is unaffected by difficult terrain for 8 hours, and spells and other magical effects can neither reduce the target's speed nor cause the target to be paralyzed or restrained for the same duration.
    The target can also spend 1 meter of movement to automatically escape from nonmagical restraints, such as manacles or a creature that has it grappled.
    Finally, being underwater imposes no penalties on the target's movement or attacks.

    Alternatively, the oil can be poured on the ground as two actions, where it covers a 3-meter square.
    Slick grease covers the ground in the square which is turned into difficult terrain for 8 hours.
    Each creature standing in the area when you pour the liquid must succeed on a Dexterity saving throw or fall prone.
    A creature that enters the area or ends its turn there must also succeed on a Dexterity saving throw or fall prone.
\paragraph{Oozekiller} \label{item::oozekiller}
    This potion contains a clear liquid which can clean any surface instantly.
    When poured on an ooze, it takes 14d6 necrotic damage.

    This potion can be drunk to end attunement with the Erebos' Ooze item (see page \pageref{item::erebosooze}), destroying the ooze in the process.
\paragraph{Potion of Aqueous Form (Vial)}
    When you drink this potion, you transform into a pool of water.
    You return to your true form after 10 minutes or if you are incapacitated or die.
    You're under the following effects while in this form:
    \begin{itemize}
        \item \textbf{Liquid Movement.} You have a swimming speed of 6 meters.
        You can move over or through other liquids.
        You can enter and occupy the space of another creature.
        You can rise up to your normal height, and you can pass through even Tiny openings.
        You extinguish nonmagical flames in any space you enter.
        \item \textbf{Watery Resilience.} You have resistance to physical damage.
        You also have advantage on Strength, Dexterity, and Constitution saving throws.
        \item \textbf{Limitations}. You can't talk, attack, cast spells, or activate magic items.
        Any objects you were carrying or wearing meld into your new form and are inaccessible, though you continue to be affected by anything you're wearing, such as armor.
    \end{itemize}
\paragraph{Potion of Condensed Life (Flask)} % Heal
    Upon drinking, a surge of energy washes through the creature, causing it to regain 80 hit points.
    This potion also ends blindness, deafness, and any diseases affecting the target.
    This spell has no effect on constructs or undead.
\paragraph{Potion of Contagion (Flask)} % Contagion.
    This flask inflicts disease, and comes in 6 flavors, detailed below.
    A creature touched by the contents of this flask is poisoned.

    At the end of each of the poisoned target's turns, the target must make a DC 16 Constitution saving throw.
    If the target succeeds on three of these saves, it is no longer poisoned, and the effect ends.
    If the target fails three of these saves, the target is no longer poisoned, but is infected by the potion's disease.
    The target is subjected to the disease for the a full week.

    Since this spell induces a natural disease in its target, any effect that removes a disease or otherwise ameliorates a disease's effects apply to it.

    \paragraph{Blinding Sickness} Pain grips the creature's mind, and its eyes turn milky white.
    The creature has disadvantage on Wisdom checks and Wisdom saving throws and is blinded.
    \paragraph{Filth Fever} A raging fever sweeps through the creature's body.
    The creature has disadvantage on Strength checks, Strength saving throws, and attack rolls that use Strength.
    \paragraph{Fleshrot} The creature's flesh decays.
    The creature has disadvantage on Charisma checks and vulnerability to all damage.
    \paragraph{Mindfire} The creature's mind becomes feverish.
    The creature has disadvantage on Intelligence checks and Intelligence saving throws, and the creature behaves as if under the effects of the confusion spell during combat.
    \paragraph{Seizure} The creature is overcome with shaking.
    The creature has disadvantage on Dexterity checks, Dexterity saving throws, and attack rolls that use Dexterity.
    \paragraph{Slimy Doom} The creature begins to bleed uncontrollably.
    The creature has disadvantage on Constitution checks and Constitution saving throws.
    In addition, whenever the creature takes damage, it is stunned until the end of its next turn.
\paragraph{Potion of Greater Healing (Flask)}
    You regain 4d4 + 4 hit points when you drink this potion.
    The potion's red liquid glimmers when agitated.
\paragraph{Potion of Healing (Flask)}
    You regain 2d4 + 2 hit points when you drink this potion.
    The potion's red liquid glimmers when agitated.
\paragraph{Potion of Regeneration (Vial)} % Regenerate
    This potion's contents stimulate a creature's natural healing ability.
    The target regains 4d8 + 15 hit points.
    Lasting for an hour, the target regains 1 hit point at the start of each of its turns (10 hit points each minute).

    The target's severed body members (fingers, legs, tails, and so on), if any, are restored after 2 minutes.
    Any bodypart larger than a finger causes the creature to suffer one level of exhaustion upon regeneration.
\paragraph{Potion of Resistance (Flask)}
    This potion comes in 5 flavors.
    When you drink this potion, you gain resistance to the potion's damage type for 1 hour.
    The damage types available are acid, cold, fire, lightning, or thunder.
\paragraph{Potion of Sickness (Flask)} % Ray of sickness.
    Upon breaking, a raw of sickening greenish energy lashes out toward the closest creature within 12 meters.
    The target takes 2d8 poison damage and must make a Constitution saving throw with DC 10.
    On a failed save, it is also poisoned until the end of your next turn.
\paragraph{Potion of Superior Healing (Flask)}
    You regain 8d4 + 8 hit points when you drink this potion. The potion's red liquid glimmers when agitated.
\paragraph{Potion of Supreme Healing (Flask)}
    You regain 10d4 + 20 hit points when you drink this potion.
    The potion's red liquid glimmers when agitated.
\paragraph{Vial of Pain (Vial)} % Harm
    This vial contains a virulent disease which harms any creature who touches its contents.
    The target must make a DC 18 Constitution saving throw.
    On a failed save, it takes 14d6 necrotic damage, or half as much damage on a successful save.
    The damage can't reduce the target's hit points below 1.
    If the target fails the saving throw, its hit point maximum is reduced for 1 hour by an amount equal to the necrotic damage it took.
    Any effect that removes a disease allows a creature's hit point maximum to return to normal before that time passes.

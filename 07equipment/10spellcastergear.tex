% !TEX root = ../main.tex
\section{Spellcaster Gear} \label{sec::spellcastergear}
    \textbf{TODO.}

    \begin{table*}[b]%
        \begin{DndTable}[width=\linewidth, header=Spellcasting Foci]{Xlccccc}
            \textbf{Item} & \textbf{Rarity} & \textbf{Mats.} & \textbf{Total Cost} & \textbf{Tools} & \textbf{Weight} & \textbf{Source} \\
            \multicolumn{7}{l}{\hspace{0.5cm}\textit{Spellcasting Foci}} \\
            Crystal                  & Plain     & 8 &      10 agnomas & JEW & 0.5 kg & PHB 151 \\
            Rod                      & Plain     & 8 &      10 agnomas & WOO & 1 kg   & PHB 151 \\
            Staff                    & Plain     & 3 &       5 agnomas & WOO & 2 kg   & PHB 151 \\
            Blowpipe                 & Common    & 1 &      20 agnomas & SMI & 1 kg   & --- \\
            Rod of Leadership        & Rare      & 8 &   5,000 agnomas & GLA & 1 kg   & DMG 197 \\
            Rod of Alertness         & Very Rare & 8 &  50,000 agnomas & GLA & 1 kg   & DMG 196 \\
            Rod of Might             & Legendary & 8 & 250,000 agnomas & GLA & 1.5 kg & DMG 196 \\
            \multicolumn{7}{l}{\hspace{0.5cm}\textit{Utilities}} \\
            Component Pouch          & Common    & 1 &      25 agnomas & LEA & 1 kg   & PHB 151 \\
            Spell Scroll (Cantrip)   & Common    & 1 &      20 agnomas & CAL & ---    & DMG 199 \\
            Spell Scroll (1st-level) & Common    & 2 &      30 agnomas & CAL & ---    & DMG 200 \\
            Spellbook                & Common    & 3 &      50 agnomas & CAL & 1.5 kg & PHB 153 \\
            Spell Scroll (2nd-level) & Uncommon  & 1 &     150 agnomas & CAL & ---    & DMG 201 \\
            Spell Scroll (3rd-level) & Uncommon  & 2 &     200 agnomas & CAL & ---    & DMG 202 \\
            Firetrap                 & Uncommon  & 4 &     250 agnomas & GLA & 1 kg   & --- \\
            Pearl of Power           & Uncommon  & 8 &     500 agnomas & JEW & ---    & DMG 184 \\
            Spell Scroll (4th-level) & Rare      & 1 &   1,500 agnomas & CAL & ---    & DMG 203 \\
            Spell Scroll (5th-level) & Rare      & 2 &   2,000 agnomas & CAL & ---    & DMG 204 \\
            Crystal of Absorption    & Very Rare & 8 &  50,000 agnomas & GLA & 1 kg.  & DMG 195 \\
            Crystalline Chronicle    & Very Rare & 8 &  50,000 agnomas & GLA & 1.5 kg & TCE 124 \\
            Spell Scroll (6th-level) & Very Rare & 1 &  15,000 agnomas & CAL & ---    & DMG 205 \\
            Spell Scroll (7th-level) & Very Rare & 2 &  20,000 agnomas & CAL & ---    & DMG 206 \\
            Spell Scroll (8th-level) & Very Rare & 4 &  30,000 agnomas & CAL & ---    & DMG 207 \\
            Spell Scroll (9th-level) & Legendary & 1 &  75,000 agnomas & CAL & ---    & DMG 208
        \end{DndTable}
    \end{table*}

    \paragraph{Blowpipe}
        Metal blowpipes are of higher quality than the common blowpipe contained in a set of glassblower's tools, and can be used as a quarterstaff.
        In addition, a flamespeaker can use a good quality blowpipe as an arcane focus.
    \paragraph{Component Pouch}
        A component pouch is a small, watertight leather belt pouch that has compartments to hold all the material components and other special items you need to cast your spells, except for those components that have a specific cost (as indicated in a spell's description).
    \paragraph{Crystal of Absorption $\odot$}
        Designed by zaloths, this item is infused with their strange magic.

        While holding this crystal, you can use your reaction to absorb a spell that is targeting only you and not with an area of effect.
        The absorbed spell's effect is canceled, and the spell's energy --- not the spell itself --- is stored in the crystal.
        The energy has the same level as the spell when it was cast.
        The crystal can absorb and store up to 50 levels of energy over the course of its existence.
        Once the crystal absorbs 50 levels of energy, it can't absorb more.
        If you are targeted by a spell that the crystal can't store, the crystal has no effect on that spell.

        When you become attuned to the crystal, you know how many levels of energy the crystal has absorbed over the course of its existence, and how many levels of spell energy it currently has stored.

        If you are a spellcaster holding the crystal, you can convert energy stored in it into spell points to cast spells you know.
        You create a number of spell points equal to the level of energy the crystal has stored instead of your normal spell points, but otherwise cast the spell as normal.
        For example, you can use 5 levels stores in the crystal to cast a 3rd level spell.

        A newly found crystal has 1d10 levels of spell energy stored in it already.
        A crystal that can no longer absorb spell energy and has no energy remaining becomes nonmagical.
    \paragraph{Firetrap}
        A firetrap is an spherical glass container about the size of a bottle.
        A flamespeaker can use a firetrap as a spellcasting focus, and is a required component to cast the whispering ember spell (see page \pageref{spell::whisperingember}).
    \paragraph{Pearl of Power $\odot$}
        While this pearl is on your person, you can use an action to regain 5 spell points.
        Once you have used the pearl, it can't be used again until the next dawn.
    \paragraph{Rod of Alertness $\odot$}
        This glass rod has a flanged head and can be used as a spellcasting focus.
        In addition, it has the following properties.
        \begin{itemize}
            \item \textbf{Alertness}
            While holding the rod, you have advantage on Wisdom (Perception) checks and on rolls for initiative.
            \item \textbf{Protective Aura}
            Using two actions, you can plant the haft end of the rod in the ground, whereupon the rod's head sheds bright light in a 12-meter radius and dim light for an additional 12 meters.
            While in that bright light, you and any creature that is friendly to you gain a +1 bonus to AC and saving throws and can sense the location of any invisible hostile creature that is also in the bright light.
            The rod's head stops glowing and the effect ends after 10 minutes, or when a creature uses an action to pull the rod from the ground.
            This property can't be used again until the next dawn.
        \end{itemize}
    \paragraph{Rod of Leadership $\odot$}
        This rod can be used as a spellcasting focus.

        Said to be designed by king Olag the Immortal for Khedrat's military, this rod can command respect from anyone around you.
        You can use two actions to present the rod and command obedience from each creature of your choice that you can see within 24 meters of you.
        Each target must succeed on a DC 15 Wisdom saving throw or be charmed by you for 8 hours.
        While charmed in this way, the creature regards you as its trusted leader.
        If harmed by you or your companions, or commanded to do something contrary to its nature, a target ceases to be charmed in this way.
        The rod can't be used in this way again until the next dawn.
    \paragraph{Rod of Might $\odot$} % TODO. Changes pending.
        % TODO. Add some lore.
        This glass rod has a straight and unassuming shape, and it functions as a mace that grants a +3 bonus to attack and damage rolls made with it, in addition to being a spellcasting focus.
        An item of intricate design, you can transform this rod into four different forms using an action.
        As another action you can transform the rod back to its original form or to another of its special forms.
        \begin{itemize}
            \item \textbf{Spear}.
            The rod's head folds down, a spear point springs from the rod's tip, and the rod's handle lengthens into a 1.8-meter haft, transforming the rod into a spear that grants a +3 bonus to attack and damage rolls made with it.
            \item \textbf{Ladder}.
            The rod transforms into a climbing pole up to 10 meters long, as you specify.
            In surfaces as hard as granite, a spike at the bottom and three hooks at the top anchor the pole.
            Horizontal bars 8 centimeters long fold out from the sides, 30 centimeters apart, forming a ladder.
            The pole can bear up to 2,000 kg.
            More weight or lack of solid anchoring causes the rod to revert to its normal form.
            \item \textbf{Ram}.
            The rod transforms into a handheld battering ram and grants its user a +10 bonus to Strength checks made to break through doors, barricades, and other barriers.
            \item \textbf{Compass}.
            The rod assumes or remains in its normal form and indicates magnetic north (Nothing happens if this function of the rod is used in a location that has no magnetic north).
        \end{itemize}

        In addition, you can add one of the following effect to this rod's attacks by expending an action:
        \begin{itemize}
            \item \textbf{Drain Life}.
            When you hit a creature with a melee attack using the rod, you can force the target to make a DC 17 Constitution saving throw.
            On a failure, the target rakes an extra 4d6 necrotic damage, and you regain a number of hit points equal to half that necrotic damage.
            This property can't be used again until the next dawn.
            \item \textbf{Paralyze}.
            When you hit a creature with a melee attack using the rod, you can force the target to make a DC 17 Strength saving throw.
            On a failure, the target is paralyzed for 1 minute.
            The target can repeat the saving throw at the end of each of its turns, ending the effect on a success.
            This property can't be used again until the next dawn.
            \item \textbf{Terrify}.
            While holding the rod, you can use an action to force each creature you can see within 6 meters of you to make a DC 17 Wisdom saving throw.
            On a failure, a target is frightened of you for 1 minute.
            A frightened target can repeat the saving throw at the end of each of its turns, ending the effect on itself on a success.
            This property can't be used again until the next dawn.
        \end{itemize}
    \paragraph{Spell Scroll} \label{item::spellscroll}
        A spell scroll bears the words of a single spell, written as a cipher.
        You can read the scroll and cast its spell without providing any material components.
        Casting the spell by reading the scroll requires the spell's normal casting time.
        Once the spell is cast, the words on the scroll fade, and it crumbles to dust.
        If the casting is interrupted, the scroll is not lost.

        To cast the spell, you must make an ability check using your spellcasting ability to determine whether you cast it successfully.
        The DC depends on the scroll rarity.
        If you already know the spell, you don't need to succeed on this DC.
        On a failed check, the spell disappears from the scroll with no other effect.

        Once the spell is cast, the words on the scroll fade, and the scroll itself crumbles to dust.

        The DC and attack bonus from spell scrolls are specified in the spell scroll table.

        A spell on a spell scroll can be copied just as spells in spellbooks can be copied.
        A creature proficient with calligrapher's supplies can copy a spell from a spell scroll.
        The copier must succeed on an Intelligence (Arcana) check with a DC equal to the Casting DC of the scroll.
        If the check succeeds, the spell is successfully copied.
        Whether the check succeeds or fails, the spell scroll is destroyed.

        When writing a spell scroll, you decide the language in which is written among the languages you know.
        To read a spell scroll, you must speak the language in which it's written.

        \begin{DndTable}[width=\linewidth, header=Spell Scrolls]{Xccc}
            \textbf{Level} & \textbf{Casting DC} & \textbf{Spell Save DC} & \textbf{Attack Bonus} \\
            Cantrip & 10 & 13 &  +5 \\
            1       & 11 & 13 &  +5 \\
            2       & 12 & 13 &  +5 \\
            3       & 13 & 15 &  +7 \\
            4       & 14 & 15 &  +7 \\
            5       & 15 & 17 &  +9 \\
            6       & 16 & 17 &  +9 \\
            7       & 17 & 18 & +10 \\
            8       & 18 & 18 & +10 \\
            9       & 19 & 19 & +11
        \end{DndTable}
    \paragraph{Spellbook} \label{item::spellbook}
        A spellbook is a leather-bound tome with 100 blank vellum pages suitable for recording spell.
        A creature capable of spellcasting can store any spell they know in the book.
        As part of a long rest, the creature can change its known spells with any set of spells written on the book, as long as they fulfill all the requirements to cast the spell.

        As part of a short rest, a creature proficient with calligrapher's supplies can attempt to copy a spell from one spellbook to another, or from a spell scroll to a spellbook.
        The Intelligence (Arcana) DC requiredto do this is the same as the Casting DC specified in the spell scrolls table.
\newpage

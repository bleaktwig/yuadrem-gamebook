% !TEX root = ../main.tex
\subsection*{Consumables} \label{ssec::consumables}
% TODO. The effect of two kinds of food or brews do not stack, but the effect of a brew, a potion, and food can stack together.

\subsubsection{Food} \label{ssec::food}
    \begin{table*}[t]%
        \begin{DndTable}[width=\linewidth, header=Food]{Xlccccc}
            \textbf{Food} & \textbf{Rarity} & \textbf{Mats.} & \textbf{Total Cost} & \textbf{Tools} & \textbf{Weight} & \textbf{Source} \\
            Animal Feed (ration)  &  &  &   5 fobs    & COO & 5 kg   & PHB 157 \\
            Chunk of Meat         &  &  &   3 agnomas & COO & ---    & PHB 158 \\
            Hunk of Cheese        &  &  &  10 fobs    & COO & ---    & PHB 158 \\
            Loaf of Bread         &  &  &   2 fobs    & COO & ---    & PHB 158 \\
            Mess Kit              &  &  &  20 fobs    & SMI & 0.5 kg & PHB 152 \\
        \end{DndTable}
    \end{table*}

    \paragraph{Mess Kit}
        This tin box contains a cup and simple cutlery.
        The box clamps together, and one side can be used as a cooking pan and the other as a plate or shallow bowl.

    % Food and water are separate!

    % Ioun's Stone Pig % Random Ioun strong --- randomized when drinking.
    % Heroes Feast % Heroes Feast spell.
\newpage~\newpage
\subsection*{Brews} \label{ssec::brews} % TODO. Brews require a TON of balance and to reference the correct sources.
    The prices listed for brews are for a small keg, which fits 8 tankards worth of brew.
    Divide this by 8 for the price of a tankard, or by 16 for the price of a cup.
    Prices consider only the brew itself, not the medium of transportation.

    It takes one minute to drink a brew and gain its benefits, which can be done at the end of a short rest.
    A creature cannot benefit from more than one brew at the same time.
    % After drinking a brew a creature rolls a DC 8 Constitution saving throw.
    % On a failure, it empties the content of the brew
    \begin{table*}[t]%
        \begin{DndTable}[width=\linewidth, header=Brews]{Xlccccc}
            \textbf{Brew} & \textbf{Rarity} & \textbf{Mats.} & \textbf{Total Cost} & \textbf{Tools} & \textbf{Weight} & \textbf{Source} \\
            Ale                      & Mundane   & 2 &      20 fobs    & BRE &  4 kg.   & PHB 158 \\
            Jug                      & Mundane   & 2 &      20 fobs    & GLA &  2 kg.   & --- \\
            Tankard                  & Mundane   & 1 &      15 fobs    & CAR &  0.5 kg. & PHB 153 \\
            Barrel                   & Plain     & 3 &       5 agnomas & CAR & 35 kg.   & PHB 153 \\
            Bottle                   & Plain     & 1 &       3 agnomas & GLA &  1 kg.   & PHB 153 \\
            Cup                      & Plain     & 1 &       3 agnomas & GLA & ---      & --- \\
            Common Wine              & Plain     & 1 &       3 agnomas & BRE &  4 kg.   & PHB 158 \\
            Full Keg                 & Plain     & 2 &       4 agnomas & CAR & 15 kg.   & --- \\
            Small Keg                & Plain     & 1 &       3 agnomas & CAR &  2 kg.   & --- \\
            Ale of Heroism           & Common    & 3 &      50 agnomas & BRE &  4 kg.   & --- \\
            Dozing Wine              & Common    & 3 &      50 agnomas & BRE &  4 kg.   & --- \\
            Exalted Beer             & Common    & 3 &      50 agnomas & BRE &  4 kg.   & --- \\
            Expeditious Brew         & Common    & 3 &      50 agnomas & BRE &  4 kg.   & --- \\
            Laughing Ale             & Common    & 3 &      50 agnomas & BRE &  4 kg.   & --- \\
            Beer of Vitality         & Uncommon  & 3 &     250 agnomas & BRE &  4 kg.   & --- \\
            Brew of Absorption       & Uncommon  & 3 &     250 agnomas & BRE &  4 kg.   & --- \\
            Brew of Giant's Strength & Uncommon  & 3 &     250 agnomas & BRE &  4 kg.   & --- \\
            Calming Wine             & Uncommon  & 3 &     250 agnomas & BRE &  4 kg.   & --- \\
            Dead Man's Ale           & Uncommon  & 3 &     250 agnomas & BRE &  4 kg.   & --- \\
            Fine Wine                & Uncommon  & 2 &     200 agnomas & BRE &  4 kg.   & PHB 158 \\
            Madman Wine              & Uncommon  & 3 &     250 agnomas & BRE &  4 kg.   & --- \\
            Speaker's Whiskey        & Uncommon  & 3 &     250 agnomas & BRE &  4 kg.   & --- \\
            Thassa's Mockery         & Uncommon  & 3 &     250 agnomas & BRE &  4 kg.   & --- \\
            Breathkiller             & Rare      & 3 &   2,500 agnomas & BRE &  4 kg.   & --- \\
            Brew of Clairvoyance     & Rare      & 3 &   2,500 agnomas & BRE &  4 kg.   & --- \\
            Brew of Invulnerability  & Rare      & 3 &   2,500 agnomas & BRE &  4 kg.   & --- \\
            Brew of Letargy          & Rare      & 3 &   2,500 agnomas & BRE &  4 kg.   & --- \\
            Brew of Maximum Power    & Rare      & 3 &   2,500 agnomas & BRE &  4 kg.   & EGW 268 \\
            Mindblower               & Rare      & 3 &   2,500 agnomas & BRE &  4 kg.   & --- \\
            Soulcatcher (Bottle)     & Rare      & 8 &   5,000 agnomas & BRE &  0.5 kg. & --- \\
            Brainbender              & Very Rare & 3 &  25,000 agnomas & BRE &  4 kg.   & --- \\
            Brew of Vitality         & Very Rare & 3 &  25,000 agnomas & BRE &  4 kg.   & --- \\
            Brew of Speed            & Very Rare & 3 &  25,000 agnomas & BRE &  4 kg.   & --- \\
            Erebos' Ooze (Jar)       & Very Rare & 8 &  50,000 agnomas & BRE &  1 kg.   & ERLW 278 \\
            Kickstarter (Bottle)     & Very Rare & 8 &  50,000 agnomas & BRE &  0.5 kg. & --- \\
            Dancer's Whiskey         & Legendary & 3 & 125,000 agnomas & BRE &  4 kg.   & --- \\
            Groundshaker             & Legendary & 3 & 125,000 agnomas & BRE &  4 kg.   & --- \\
            Purphoros' Maker         & Legendary & 3 & 125,000 agnomas & BRE &  4 kg.   & --- \\
            Seer's Vodka             & Legendary & 3 & 125,000 agnomas & BRE &  4 kg.   & ---
        \end{DndTable}
    \end{table*}

    \paragraph{Ale of Heroism} % Heroism
        This ale is honey-flavored, reminding you of those most important to you.
        A creature who drinks this is imbued with bravery.
        For an hour, the creature is immune to being frightened and gains 1 temporary hit point at the start of each of its turns.
        After the hour passes, the creature loses all temporary hit points gained from this brew.
    \paragraph{Barrel}
        A barrel can hold 150 liters of liquid.
    \paragraph{Beer of Vitality} % Aid
        This beer emboldens creatures with toughness and resolve.
        The drinker's hit point maximum and current hit points increase by 5 for 8 hours.
    \paragraph{Bottle}
        A bottle holds 1 liter of liquid.
    \paragraph{Brainbender} % Confusion
        This brew is a nut-flavored greenish liquid.
        The liquid assaults and twists creatures' minds, spawning delusions and provoking uncontrolled action.
        Each creature that drinks the brew must succeed on a DC 16 Wisdom saving throw or be affected by it for an hour.

        An affected target can't take reactions and must roll a d10 at the start of each of its turns to determine its behavior for that turn.

        \begin{DndTable}[width=\linewidth, header=Confusion Behaviour]{lX}
            \textbf{d10} & \textbf{Behaviour} \\
            1            & The creature uses all its movement to move in a random direction. To determine the direction, roll a d8 and assign a direction to each die face.
            The creature doesn't take any other action this turn. \\
            2-6	         & The creature doesn't move or take actions this turn. \\
            7-8	         & The creature uses its action to make a melee attack against a randomly determined creature within its reach.
            If there is no creature within its reach, the creature does nothing this turn. \\
            9-10         & The creature can act and move normally.
        \end{DndTable}

        Every 10 minutes, an affected creature can make a DC 16 Wisdom saving throw.
        If it succeeds, this effect ends for that it.
    \paragraph{Breathkiller} % Stinking Cloud
        This atrocity of a brew is a yellow muddy liquid that leaves a terrible itching sensation upon passing through your throat.
        After drinking it, a creature becomes surrounded by a 6-meter-radius sphere of yellow, nauseating gas centered on itself.
        The cloud spreads around corners, and its area is heavily obscured.
        The cloud lingers in the air for an hour after the brew is drunk.

        Each creature that is completely within the cloud at the start of its turn must make a DC 14 Constitution saving throw against poison.
        On a failed save, the creature spends its action that turn retching and reeling.
        Creatures that don't need to breathe or are immune to poison automatically succeed on this saving throw.
    \paragraph{Brew of Absorption} % Absorb Elements + Dragon's breath of the absorbed element
        This drink seems to remain in your throat after drinking it, ready to return to the world on command.
        When the drinker takes acid, cold, fire, lightning, or poison damage within 8 hours of drinking the brew, it can use its reaction to gain resistance against the attack.
        As part of the same reaction, it can then exhale energy in a 4.5 meter cone.
        Each creature in the area must make a DC 12 Dexterity saving throw, taking 3d6 damage of the absorbed type on a failed save, or half as much damage on a successful one.
        The drinker can only use this ability once during the brew's duration.
    \paragraph{Brew of Clairvoyance} % Clairvoyance
        An eyeball bobs in this yellowish vodka but vanishes as the brew is drunk.
        Upon drinking this brew, you create an invisible sensor within 2 kilometers in a location familiar to you (a place you have visited or seen before) or in an obvious location that is unfamiliar to you (such as behind a door, around a corner, or in a grove of trees).
        The sensor remains in place for the duration, and it can't be attacked or otherwise interacted with.

        You can see and hear through the sensor as if you were in its space.

        A creature that can see the sensor (such as a creature with truesight) sees a luminous, intangible orb about the size of your fist.

        The sensor remains in position for 8 hours, after which it disappears.
    \paragraph{Brew of Giant Strength}
        This brew's transparent liquid has floating in it a sliver of fingernail from a cyclops.
        When you drink it, your Strength score changes to 21 for 1 hour.
        The potion has no effect on you if your Strength is equal to or greater than that score.
    \paragraph{Brew of Invulnerability}
        This syrupy liquid looks like liquefied iron.
        For an hour after drinking it, you have resistance to all physical damage.
    \paragraph{Brew of Letargy} % Slow
        This thick whiskey seems to flow slower than naturally possible.
        After drinking it, a creature must succeed on a DC 14 Wisdom saving throw or be affected by this brew for an hour.

        An affected target's speed is halved, it takes a -2 penalty to AC and Dexterity saving throws, and it can't use reactions.
        On its turn, it has only 2 actions instead of the normal 3.

        If the creature attempts to cast a spell with a casting time of 2 or more actions, roll a d20.
        On an 11 or higher, the spell doesn't take effect until the creature's next turn, and the creature must use its actions on that turn to complete the spell.
        If it can't, the spell is wasted.

        A creature affected by this brew makes another DC 14 Wisdom saving throw every 10 minutes.
        On a successful save, the effect ends for it.
    \paragraph{Brew of Maximum Power}
        This glowing purple liquid smells of sugar and plum, but it has a muddy taste.

        The first time you cast a damage-dealing spell of 4th level or lower within an hour after drinking the brew, instead of rolling dice to determine the damage dealt, you can instead use the highest number possible for each die.
    \paragraph{Brew of Speed} % Haste
        This black liquid flows faster than what seems reasonable.
        For an hour, the drinker's speed is doubled, it gains a +2 bonus to AC, it has advantage on Dexterity saving throws, and it gains an additional action on each of its turns.

        When the effect ends, the target can't move or take actions for a minute, as a wave of lethargy sweeps over it.
    \paragraph{Brew of Vitality} % Potion of Vitality
        The brew's crimson liquid regularly pulses with dull light, calling to mind a heartbeat.
        When you drink this brew, it removes any exhaustion you are suffering and cures any disease or poison affecting you.
        For the next 24 hours, you regain the maximum number of hit points for any Hit Die you spend.
    \paragraph{Calming Wine} % Calm Emotions
        This wine's fruity flavor seems to calm emotions on command.
        Any creature who drinks this rolls a DC 12 Charisma saving throw.
        A creature can choose to fail this saving throw if it wishes.
        If a creature fails, two effects occur with a duration of an hour:
        \begin{itemize}
            \item Any effect causing a target to be charmed or frightened is suppressed.
            When this effect ends, any suppressed effect resumes, provided that its duration has not expired in the meantime.
            \item The target feels indifferent about creatures that it is hostile toward.
            This indifference ends if the target is attacked or harmed by a spell or if it witnesses any of its friends being harmed.
            When the effect ends, the creature becomes hostile again, unless the DM rules otherwise.
        \end{itemize}

        A creature affected by this brew makes another DC 12 Charisma saving throw every 10 minutes.
        On a successful save, the effect ends for it.
    \paragraph{Dancer's Whiskey} % Otto's Irresistible Dance
        This fuzzy pink brew feels like small explosions happening inside your mouth.
        After finishing the brew, the drinker begins a comic dance in place: shuffling, tapping its feet, and capering for an hour.

        A dancing creature must use all its movement to dance without leaving its space and has disadvantage on Dexterity saving throws and attack rolls.
        While the target is affected by this brew, other creatures have advantage on attack rolls against it.
        Every 10 minutes, a dancing creature makes a DC 18 Wisdom saving throw to regain control of itself.
        On a successful save, the effect ends.
    \paragraph{Dead Man's Ale} % Blindness/Deafness
        This brew looks like common ale, hiding a terrible effect within.
        After drinking it, the creature rolls a DC 12 Constitution saving throw.
        If it fails, the target is blinded for an hour.

        A creature affected by this brew can makes another DC 12 Constitution saving throw every 10 minutes.
        On a successful save, the effect ends for it.
    \paragraph{Dozing Wine} % Sleep
        This wine is so comforting that it sends creatures into slumber.
        Any creature who drinks this must succeed on a DC 10 Constitution saving throw.
        On a failure, the creature falls unconscious for an hour, until the sleeper takes damage, or someone uses an action to shake or slap the sleeper awake.

        A creature affected by this brew makes another DC 10 Constitution saving throw every 10 minutes.
        On a successful save, the effect ends for it.
    \paragraph{Erebos' Ooze (Jar) $\odot$} \label{item::erebosooze} % Kyrzin's Ooze
        This opalescent, symbiotic goo comes sealed in a jar and slowly shifts and moves, as if endlessly exploring the jar's interior.
        You can attune to this item by drinking the contents of the jar, unlocking the following properties.

        \paragraph{Resistant} While attuned to Erebos' ooze, you have resistance to poison and acid damage, and you're immune to the poisoned condition.
        \paragraph{Amorphous} Using two actions, you can speak a command word and cause your body to assume the amorphous qualities of an ooze.
        For the next minute, you (along with any equipment you're wearing or carrying) can move through a space as narrow as 1 inch wide without squeezing.
        Once you use this property, it can't be used again until the next dawn.
        \paragraph{Acid Breath} Using two actions, you can exhale acid in a 6-meter line that is 1 meters wide.
        Each creature in that line must make a DC 15 Dexterity saving throw, taking 36 (8d8) acid damage on a failed save, or half as much damage on a successful one.
        Once you use this property, it can't be used again until the next dawn.
        \paragraph{Symbiotic Nature} The ooze can't be removed from you while you're attuned to it, and you can't voluntarily end your attunement to it.
        The only way to remove the ooze is by drinking an Oozekiller potion (see page \pageref{item::oozekiller}).

        If you die while the ooze is inside you, it bursts out and engulfs you, turning your corpse into a black pudding.
    \paragraph{Exalted Beer} % Bless
        The sweet flavor of this beer truly is a blessing.
        After drinking it, the creature can add a d4 to any attack roll or saving throw it makes for one hour.
    \paragraph{Expeditious Brew} % Expeditious Retreat
        This brew feels like caustic acid falling through your throat.
        After drinking it, the creature gains a bonus of 2 meters to its movement speed for one hour.
    \paragraph{Full Keg}
        A full keg can hold 60 liters of liquid.
    \paragraph{Groundshaker} % Investiture of Stone
        Small bits of stone float aimlessly in this amber root beer.
        For an hour after drinking it, bits of rock spread across your body, and you gain the following benefits:
        \begin{itemize}
            \item You have resistance to bludgeoning, piercing, and slashing damage from nonmagical attacks.
            \item You can use your action to create a small earthquake on the ground in a 4.5-meter radius centered on you.
            Other creatures on that ground must succeed on a DC 18 Dexterity saving throw or be knocked prone.
            \item You can move across difficult terrain made of earth or stone without spending extra movement.
            You can move through solid earth or stone as if it was air and without destabilizing it, but you can't end your movement there.
            If you do so, you are ejected to the nearest unoccupied space, this effect ends, and you are stunned until the end of your next turn.
        \end{itemize}
    \paragraph{Jug}
        A jug holds 1 liter of liquid.
    \paragraph{Kickstarter (Bottle)} % Awaken
        This color-changing thick liquid is said to contain the very essence of the tides.
        Any Huge or smaller beast with an intelligence of 3 or less can receive the effect of this potion.
        The target gains an Intelligence of 10.

        The awakened beast is charmed by you for 30 days or until you or your companions do anything harmful to it.
        When the charmed condition ends, the awakened creature chooses whether to remain friendly to you, based on how you treated it while it was charmed.

        After this period passes, the creature needs a qualar to remain sentient, and it suffers Dementia effects normally (See page \pageref{ssec::dementia}).
    \paragraph{Laughing Brew} % Tasha's Hideous Laughter
        This fuzzy and bubbly drink seems to vibrate with astounding speed.
        After drinking it, the creature perceives everything as hilariously funny and falls into fits of laughter.
        The creature must succeed on a DC 10 Wisdom saving throw or fall prone, becoming incapacitated and unable to stand up for up to an hour.
        A creature with an Intelligence score of 4 or less isn't affected.

        A creature affected by this brew makes another DC 10 Wisdom saving throw every 10 minutes and when it takes damage.
        The target has advantage on the saving throw if it's triggered by damage.
        On a success, the effect ends.
    \paragraph{Madman Wine} % Crown of Madness
        This wine looks normal, but has a somewhat odd flavor.
        A creature who drinks this must succeed on a DC 12 Wisdom saving throw or become charmed for a minute.
        While charmed, madness glows in the creature's eyes.

        The charmed creature must use its action before moving on each of its turns to make a melee attack against a creature other than itself within 1 meter of it chosen randomly.
        If no creatures are within reach, it moves towards the closest creature to attack it.

        The target can make a DC 12 Wisdom saving throw at the end of each of its turns.
        On a success, the effect ends.
    \paragraph{Mindblower} % Speak with Plants + Hallucinatory Terrain
        This brew is a clear liquid indistinguishable from water.
        Upon drinking it, you can communicate with any plant within 6 meters of you and issue it simple commands for 8 hours.
        You can question plants about events in the spell's area within the past day, gaining information about creatures that have passed, weather, and other circumstances.

        You can also turn difficult terrain caused by plant growth (such as thickets and undergrowth) into ordinary terrain that lasts for the duration.
        Or you can turn ordinary terrain where plants are present into difficult terrain that lasts for the duration, causing vines and branches to hinder pursuers, for example.

        Plants might be able to perform other tasks on your behalf, at the DM's discretion. The spell doesn't enable plants to uproot themselves and move about, but they can freely move branches, tendrils, and stalks.

        If a plant creature is in the area, you can communicate with it as if you shared a common language, but you gain no magical ability to influence it.

        This spell can cause the plants created by the entangle spell to release a restrained creature.

        While under the effect of this brew, all terrain looks hard to traverse, and your movement speed is halved.
    \paragraph{Purphoros' Maker} % Flesh to Stone
        Upon drinking this unassuming wine, a creature begins to turn into stone.
        If the drinker's body is made of flesh, the creature must make a DC 18 Constitution saving throw.
        On a failed save, it is restrained as its flesh begins to harden.
        On a successful save, the creature isn't affected.

        A creature restrained by this effect must make another DC 18 Constitution saving throw at the end of each of its turns.
        If it successfully saves against this brew three times, the effect ends.
        If it fails its saves three times, it is turned to stone and subjected to the petrified condition until the effect is removed.
        The successes and failures don't need to be consecutive; keep track of both until the target collects three of a kind.

        If the creature is physically broken while petrified, it suffers from similar deformities if it reverts to its original state.
    \paragraph{Seer's Vodka} % True Seeing
        Reflections in this clear vodka seem out of place, as if the turbulent liquid displaced light in turbulent ways.
        This brew gives the drinker the ability to see things as they actually are.
        For 8 hours, the creature has truesight, and notices secrets hidden by magic out to a range of 24 meters.
    \paragraph{Soulcatcher} % Revivify
        As two actions, you can force a creature that has died within the last minute to drink this liquid.
        That creature returns to life with 1 hit point.
        This spell can't return to life a creature that has died of old age, nor can it restore any missing body parts.
    \paragraph{Small Keg}
        A small, portable keg can hold 4 liters of liquid.
    \paragraph{Speaker's Whiskey} % Zone of Truth on a target
        This strong whiskey seems to pick at your brain, forcing you to feel remorse about your misdeeds.
        Any creature who drinks this must make a DC 12 Charisma saving throw.
        On a failed save, a creature can't speak a deliberate lie for an hour.
        A skilled brewer knows whether each creature succeeds or fails on its saving throw.

        An affected creature is aware of the effect and can thus avoid answering questions to which it would normally respond with a lie.
        Such creatures can be evasive in its answers as long as it remains within the boundaries of the truth.
    \paragraph{Tankard}
        A tankard holds half a liter of liquid.
    \paragraph{Thassa's Mockery} % Water Breathing
        This transparent vodka has fish scales floating in it, which must be drinked with the brew to gain its effect.
        This drinks grants a creature the ability to breathe underwater for 8 hours.
        Affected creatures also retain their normal mode of respiration.
\pagebreak

\subsubsection{Potions} \label{ssec::potions} % TODO. Potions require some balancing and to reference the correct sources.
    Potions are small amounts of potent liquid contained in a flask or vial.
    A creature can drink a potion using two actions, or it can throw the container up to 4/12 meters as an action.
    This is considered a ranged weapon and any feat that applies to thrown weapons applies to it.

    A flask can contain the same amount of liquid as 4 vials.

    \begin{table*}[t]%
        \begin{DndTable}[width=\linewidth, header=Potions]{Xlccccc}
            \textbf{Potion} & \textbf{Rarity} & \textbf{Mats.} & \textbf{Total Cost} & \textbf{Tools} & \textbf{Weight} & \textbf{Source} \\
            Flask                              & Mundane   & 1 &      15 fobs    & GLA       & 0.5 kg. & PHB 153 \\
            Oil (Flask)                        & Mundane   & 1 &      15 fobs    & ALC       & 0.5 kg. & PHB 152 \\
            Soap                               & Mundane   & 1 &      15 fobs    & ALC       & ---     & PHB 150 \\
            Perfume (Vial)                     & Plain     & 3 &       5 agnomas & ALC       & ---     & PHB 150 \\
            Acid (Vial)                        & Common    & 1 &      30 agnomas & ALC       & ---     & PHB 148 \\
            Alchemist's Fire (Flask)           & Common    & 3 &      50 agnomas & ALC       & 0.5 kg. & PHB 148 \\
            Antitoxin (Vial)                   & Common    & 3 &      50 agnomas & ALC       & ---     & PHB 151 \\
            Caustic Brew (Flask)               & Common    & 3 &      50 agnomas & ALC       & 0.5 kg. & --- \\
            Potion of Healing (Flask)          & Common    & 1 &      30 agnomas & ALC       & 0.5 kg. & DMG 187 \\
            Potion of Sickness (Flask)         & Common    & 3 &      50 agnomas & ALC       & 0.5 kg. & --- \\
            Vial                               & Common    & 1 &       3 agnomas & GLA       & ---     & PHB 153 \\
            Bloodwell Vial +1                  & Uncommon  & 8 &     500 agnomas & ALC + GLA & ---     & TCE 122 \\
            Bottled Acid Arrow (Flask)         & Uncommon  & 3 &     250 agnomas & ALC       & 0.5 kg. & --- \\
            Coldblood (Vial)                   & Uncommon  &$\ast$ & 500 agnomas & ALC       & ---     & --- \\
            Oil of Slipperiness (Vial)         & Uncommon  & 3 &     250 agnomas & ALC       & ---     & DMG 184 \\
            Potion of Greater Healing (Flask)  & Uncommon  & 1 &     150 agnomas & ALC       & 0.5 kg. & DMG 187 \\
            Potion of Resistance (Flask)       & Uncommon  & 3 &     250 agnomas & ALC       & 0.5 kg. & DMG 188 \\
            Bloodwell Vial +2                  & Rare      & 8 &   5,000 agnomas & ALC + GLA & ---     & TCE 122 \\
            Bottled Fire (Flask)               & Rare      & 3 &   2,500 agnomas & ALC       & 0.5 kg. & --- \\
            Elixir of Health (Flask)           & Rare      & 3 &   2,500 agnomas & ALC       & 0.5 kg. & DMG 168 \\
            Flask of Mass Healing (Flask)      & Rare      & 3 &   2,500 agnomas & ALC       & 0.5 kg. & --- \\
            Oil of Burning (Vial)              & Rare      & 3 &   2,500 agnomas & ALC       & ---     & --- \\
            Potion of Aqueous Form (Vial)      & Rare      & 3 &   2,500 agnomas & ALC       & ---     & MOT 197 \\
            Potion of Superior Healing (Flask) & Rare      & 1 &   1,500 agnomas & ALC       & 0.5 kg. & DMG 187 \\
            Bloodwell Vial +3                  & Very Rare & 8 &  50,000 agnomas & ALC + GLA & ---     & TCE 122 \\
            Bottled Blight (Vial)              & Very Rare & 3 &  25,000 agnomas & ALC       & ---     & --- \\
            Flask of Weakness (Flask)          & Very Rare & 3 &  25,000 agnomas & ALC       & 0.5 kg. & --- \\
            Healing Cloud (Flask)              & Very Rare & 3 &  25,000 agnomas & ALC       & 0.5 kg. & --- \\
            Liquid Death (Flask)               & Very Rare & 3 &  25,000 agnomas & ALC       & 0.5 kg. & --- \\
            Oil of Immolation (Vial)           & Very Rare & 3 &  25,000 agnomas & ALC       & ---     & --- \\
            Oil of Sharpness (Vial)            & Very Rare & 3 &  25,000 agnomas & ALC       & ---     & DMG 184 \\
            Oozekiller (Vial)                  & Very Rare & 3 &  25,000 agnomas & ALC       & ---     & --- \\
            Potion of Contagion (Flask)        & Very Rare & 3 &  25,000 agnomas & ALC       & 0.5 kg. & --- \\
            Potion of Supreme Healing (Flask)  & Very Rare & 1 &  15,000 agnomas & ALC       & 0.5 kg. & DMG 187 \\
            Bottled Tizerus (Flask)            & Legendary & 3 & 125,000 agnomas & ALC       & 0.5 kg. & --- \\
            Potion of Condensed Life (Flask)   & Legendary & 3 & 125,000 agnomas & ALC       & 0.5 kg. & --- \\
            Potion of Regeneration (Vial)      & Legendary & 3 & 125,000 agnomas & ALC       & ---     & --- \\
            Vial of Pain (Vial)                & Legendary & 3 & 125,000 agnomas & ALC       & ---     & ---
        \end{DndTable}
    \end{table*}
    $\ast$ Coldblood is not commonly sold, and can only be gathered from a live uman using Alchemist's supplies.

    \paragraph{Acid (Vial)}
        As an action, you can splash the contents of this vial onto a creature within 1 meter of you or throw the vial up to 4 meters, shattering it on impact.
        In either case, make a ranged attack against a creature or object, treating the acid as an improvised weapon.
        On a hit, the target takes 2d6 acid damage.
    \paragraph{Alchemist's Fire (Flask)}
        This sticky, adhesive fluid ignites when exposed to air.
        As an action, you can throw this flask up to 4 meters, shattering it on impact.
        Make a ranged attack against a creature or object, treating the alchemist's fire as an improvised weapon.
        On a hit, the target takes 1d4 fire damage at the start of each of its turns.
        A creature can end this damage by using its action to make a DC 10 Dexterity check to extinguish the flames.
    \paragraph{Antitoxin (Vial)}
        A creature that drinks this vial of liquid gains advantage on saving throws against poison for 1 hour.
        It confers no benefit to undead or constructs.
    \paragraph{Bloodwell Vial $\odot$}
        To attune to this vial, you must fill it with coldblood and add a few drops of your own blood.
        As long as this vial is close to your spellcasting focus, you gain a bonus to spell attack rolls and to the saving throw DCs of your spells as specified on the item's name.

        This vial has 100 charges.
        Every time a spell's effect is improved by this vial, you expend one charge.
        When all these charges are expended, your attunement with the item is lost.
    \paragraph{Bottled Acid Arrow (Flask)} % Melf's acid arrow.
        Upon being opened, a shimmering green arrow streaks from the bottle opening in a line and bursts in a spray of acid.
        Make attack against a target within 18 meters.
        On a hit, the target takes 4d4 acid damage immediately and 2d4 acid damage at the end of its next turn.
        On a miss, the arrow splashes the target with acid for half as much of the initial damage and no damage at the end of its next turn.
    \paragraph{Bottled Blight (Vial)} % 5th-level Blight.
        Upon being opened, this bottle releases a sickening gas that drains moistury and vitality.
        The gas fills a 1.5-meter square before dispersing.

        Any creature standing in this square must make a DC 16 Constitution saving throw.
        It takes 9d8 necrotic damage on a failed save, or half as much damage on a successful one.
        This potion has no effect on undead or constructs.

        If you target a plant creature or a magical plant, it makes the saving throw with disadvantage, and the potion deals maximum damage to it.

        If you target a nonmagical plant that isn't a creature, such as a tree or shrub, it doesn't make a saving throw, it simply withers and dies.
    \paragraph{Bottled Fire (Flask)} % Fireball.
        Upon breaking, a bright spark blossoms with a low roar into an explosion of flame.
        Each creature in a 6-meter-radius sphere centered on that point must make a Dexterity saving throw with DC 14.
        A target takes 8d6 fire damage on a failed save, or half as much damage on a successful one.

        The fire spreads around corners.
        It ignites flammable objects in the area that aren't being worn or carried.
    \paragraph{Bottled Tizerus (Flask)} % Delayed Blast Fireball
        This flask contains a tiny spark, patiently waiting to be awakened.
        You can shake the bottle to activate this spark.
        For the following minute, the bead continues to grow ever more violent.
        Upon breaking or when the minute passes, it blossoms with a low roar into an explosion of flame that spreads around corners.
        Each creature in a 6-meter radius sphere centered on that point must make a DC 18 Dexterity saving throw.
        A creature takes fire damage equal to the total accumulated damage on a failed save, or half as much damage on a successful one.

        The potions's base damage is 12d6.
        If at the end of your turn the bead has not yet detonated, the damage increases by 1d6.

        The glowing bead can be taken from the flask without causing it to explode.
        The creature touching it must make a DC 12 Dexterity saving throw.
        On a failed save, the spark erupts in flame.
        On a successful save, the creature can throw the bead up to 8 meters.
        When it strikes a creature or a solid object, the bead explodes.

        The fire damages objects in the area and ignites flammable objects that aren't being worn or carried.
    \paragraph{Caustic Brew (Flask)} % Tasha's Caustic Brew
        Upon being shaken and breaking, this flask explodes in a radius 2 meters.
        Each creature in the cloud must succeed on a DC 10 Dexterity saving throw or be covered in acid for a minute or until a creature uses two actions to scrape or wash the acid off itself or another creature.
        A creature covered in the acid takes 2d4 acid damage at start of each of its turns.
    \paragraph{Elixir of Health (Flask)}
        When you drink this potion, it cures any disease afflicting you, and it removes the blinded, deafened, paralyzed, and poisoned conditions.
        The clear red liquid has tiny bubbles of light in it.
    \paragraph{Flask of Mass Healing (Flask)} % Mass healing word.
        Upon breaking, all creatures within a 6-meter radius regain hit points equal to 1d4 + 3.
        This potion has no effect on undead or constructs.
    \paragraph{Flask of Weakness (Flask)} % Elemental Bane.
        This flask comes in 5 flavors: acid, cold, fire, lightning, and thunder.
        Upon being opened, this flask releases a noxious gas in a 10-meter square.
        Any creature who breathes the gas must succeed on a DC 16 Constitution saving throw or be affected by a weakening effect for a minute.
        The first time each turn the affected creature takes damage of the potion's type, the target takes an extra 2d6 damage of the type.
        Moreover, the target loses any resistance to that damage until the effect ends.
    \paragraph{Healing Cloud (Flask)} % Mass Cure Wounds.
        Upon being opened, this potion releases a wave of healing energy in the form of a flowery smell.
        Any creature standing in a 9-meter radius sphere centered on the potion's location regains hit points equal to 3d8 + 4.
        This potion has no effect on undead or constructs.
    \paragraph{Liquid Death (Flask)} % Cloudkill.
        Upon breaking, this flask releases a 6-meter radius sphere of poisonous, yellow-green fog.
        The fog spreads around corners.
        It lasts for 10 minutes or until strong wind disperses the fog, ending the effect.
        Its area is heavily obscured.

        When a creature enters the fog's area for the first time on a turn or starts its turn there, that creature must make a DC 16 Constitution saving throw.
        The creature takes 5d8 poison damage on a failed save, or half as much damage on a successful one.
        Creatures are affected even if they hold their breath or don't need to breathe.

        % The fog moves 3 meters away from you at the start of each of your turns, rolling along the surface of the ground.
        % The vapors, being heavier than air, sink to the lowest level of the land, even pouring down openings.
    \paragraph{Oil (Flask)}
        As an action, you can splash the oil in this flask onto a creature within 1 meter of you or throw it up to 4 meters, shattering it on impact.
        Make a ranged attack against a target creature or object, treating the oil as an improvised weapon.
        On a hit, the target is covered in oil.
        The oil dries after one minute, and for this duration the target gains vulnerability to fire damage.
        You can also pour a flask of oil on the ground to cover a 1.5-meter-square area, provided that the surface is level.
        If lit, the oil burns for 2 rounds and deals 5 fire damage to any creature that enters the area or ends its turn in the area.
        A creature can take this damage only once per turn.
    \paragraph{Oil of Burning (Vial)} % Flame arrows.
        This brown sticky oil quickly heats up when accelerated.
        By drenching an arrow or bolt in the oil as an action, the piece of ammunition gains an additional 1d6 fire damage on hit.
        The oil evapores on a piece of ammunition when it hits or misses.
        Using two actions, you can pour the liquid into up to twelve pieces of ammunition at once.
    \paragraph{Oil of Immolation (Vial)} % Immolation.
        Upon being opened, the vial explodes into all consuming flames in a 1.5-meter radius sphere.
        Any creature within this sphere is wreathed by flames.
        It must make a DC 16 Dexterity saving throw.
        It takes 8d6 fire damage on a failed save, or half as much damage on a successful one.
        On a failed save, the target also burns for 1 minute.
        The burning target sheds bright light in a 9-meter radius and dim light for an additional 6 meters.
        At the end of each of its turns, the target repeats the saving throw.
        It takes 4d6 fire damage on a failed save, and the effect ends on a successful one.
        These flames can't be extinguished by other means.

        If damage from this spell kills a target, the target is turned to ash.
    \paragraph{Oil of Sharpness (Vial)}
        This clear, gelatinous oil sparkles with tiny, ultrathin silver shards.
        The oil can coat one slashing or piercing weapon or up to 5 pieces of slashing or piercing ammunition.
        Applying the oil takes 1 minute.
        For 1 hour, the coated item has a +3 bonus to attack and damage rolls.
    \paragraph{Oil of Slipperiness (Vial)} % Freedom of Movement + Grease
        This sticky black unguent is thick and heavy in the container, but it flows quickly when poured.
        The oil can cover a Medium or smaller creature, along with the equipment it's wearing and carrying (one additional vial is required for each size category above Medium).
        Applying the oil takes 10 minutes.

        The affected creature's movement is unaffected by difficult terrain for 8 hours, and spells and other magical effects can neither reduce the target's speed nor cause the target to be paralyzed or restrained for the same duration.
        The target can also spend 1 meter of movement to automatically escape from nonmagical restraints, such as manacles or a creature that has it grappled.
        Finally, being underwater imposes no penalties on the target's movement or attacks.

        Alternatively, the oil can be poured on the ground as two actions, where it covers a 3-meter square.
        Slick grease covers the ground in the square which is turned into difficult terrain for 8 hours.
        Each creature standing in the area when you pour the liquid must succeed on a Dexterity saving throw or fall prone.
        A creature that enters the area or ends its turn there must also succeed on a Dexterity saving throw or fall prone.
    \paragraph{Oozekiller} \label{item::oozekiller}
        This potion contains a clear liquid which can clean any surface instantly.
        When poured on an ooze, it takes 14d6 necrotic damage.

        This potion can be drunk to end attunement with the Erebos' Ooze item (see page \pageref{item::erebosooze}), destroying the ooze in the process.
    \paragraph{Potion of Aqueous Form (Vial)}
        When you drink this potion, you transform into a pool of water.
        You return to your true form after 10 minutes or if you are incapacitated or die.
        You're under the following effects while in this form:
        \begin{itemize}
            \item \textbf{Liquid Movement.} You have a swimming speed of 6 meters.
            You can move over or through other liquids.
            You can enter and occupy the space of another creature.
            You can rise up to your normal height, and you can pass through even Tiny openings.
            You extinguish nonmagical flames in any space you enter.
            \item \textbf{Watery Resilience.} You have resistance to physical damage.
            You also have advantage on Strength, Dexterity, and Constitution saving throws.
            \item \textbf{Limitations}. You can't talk, attack, cast spells, or activate magic items.
            Any objects you were carrying or wearing meld into your new form and are inaccessible, though you continue to be affected by anything you're wearing, such as armor.
        \end{itemize}
    \paragraph{Potion of Condensed Life (Flask)} % Heal
        Upon drinking, a surge of energy washes through the creature, causing it to regain 80 hit points.
        This potion also ends blindness, deafness, and any diseases affecting the target.
        This spell has no effect on constructs or undead.
    \paragraph{Potion of Contagion (Flask)} % Contagion.
        This flask inflicts disease, and comes in 6 flavors, detailed below.
        A creature touched by the contents of this flask is poisoned.

        At the end of each of the poisoned target's turns, the target must make a DC 16 Constitution saving throw.
        If the target succeeds on three of these saves, it is no longer poisoned, and the effect ends.
        If the target fails three of these saves, the target is no longer poisoned, but is infected by the potion's disease.
        The target is subjected to the disease for the a full week.

        Since this spell induces a natural disease in its target, any effect that removes a disease or otherwise ameliorates a disease's effects apply to it.

        \paragraph{Blinding Sickness} Pain grips the creature's mind, and its eyes turn milky white.
        The creature has disadvantage on Wisdom checks and Wisdom saving throws and is blinded.
        \paragraph{Filth Fever} A raging fever sweeps through the creature's body.
        The creature has disadvantage on Strength checks, Strength saving throws, and attack rolls that use Strength.
        \paragraph{Fleshrot} The creature's flesh decays.
        The creature has disadvantage on Charisma checks and vulnerability to all damage.
        \paragraph{Mindfire} The creature's mind becomes feverish.
        The creature has disadvantage on Intelligence checks and Intelligence saving throws, and the creature behaves as if under the effects of the confusion spell during combat.
        \paragraph{Seizure} The creature is overcome with shaking.
        The creature has disadvantage on Dexterity checks, Dexterity saving throws, and attack rolls that use Dexterity.
        \paragraph{Slimy Doom} The creature begins to bleed uncontrollably.
        The creature has disadvantage on Constitution checks and Constitution saving throws.
        In addition, whenever the creature takes damage, it is stunned until the end of its next turn.
    \paragraph{Potion of Greater Healing (Flask)}
        You regain 4d4 + 4 hit points when you drink this potion.
        The potion's red liquid glimmers when agitated.
    \paragraph{Potion of Healing (Flask)}
        You regain 2d4 + 2 hit points when you drink this potion.
        The potion's red liquid glimmers when agitated.
    \paragraph{Potion of Regeneration (Vial)} % Regenerate
        This potion's contents stimulate a creature's natural healing ability.
        The target regains 4d8 + 15 hit points.
        Lasting for an hour, the target regains 1 hit point at the start of each of its turns (10 hit points each minute).

        The target's severed body members (fingers, legs, tails, and so on), if any, are restored after 2 minutes.
        Any bodypart larger than a finger causes the creature to suffer one level of exhaustion upon regeneration.
    \paragraph{Potion of Resistance (Flask)}
        This potion comes in 5 flavors.
        When you drink this potion, you gain resistance to the potion's damage type for 1 hour.
        The damage types available are acid, cold, fire, lightning, or thunder.
    \paragraph{Potion of Sickness (Flask)} % Ray of sickness.
        Upon breaking, a raw of sickening greenish energy lashes out toward the closest creature within 12 meters.
        The target takes 2d8 poison damage and must make a Constitution saving throw with DC 10.
        On a failed save, it is also poisoned until the end of your next turn.
    \paragraph{Potion of Superior Healing (Flask)}
        You regain 8d4 + 8 hit points when you drink this potion. The potion's red liquid glimmers when agitated.
    \paragraph{Potion of Supreme Healing (Flask)}
        You regain 10d4 + 20 hit points when you drink this potion.
        The potion's red liquid glimmers when agitated.
    \paragraph{Vial of Pain (Vial)} % Harm
        This vial contains a virulent disease which harms any creature who touches its contents.
        The target must make a DC 18 Constitution saving throw.
        On a failed save, it takes 14d6 necrotic damage, or half as much damage on a successful save.
        The damage can't reduce the target's hit points below 1.
        If the target fails the saving throw, its hit point maximum is reduced for 1 hour by an amount equal to the necrotic damage it took.
        Any effect that removes a disease allows a creature's hit point maximum to return to normal before that time passes.
\newpage

\subsubsection{Poisons}
    % NOTE: Coating a weapon or 3 pieces of ammunition in poison costs two actions.
    \begin{table*}[t]%
        \begin{DndTable}[width=\linewidth, header=Poisons]{Xlccccc}
            \textbf{Poison} & \textbf{Rarity} & \textbf{Method} & \textbf{Mats.} & \textbf{Total Cost} & \textbf{Tools} & \textbf{Weight} & \textbf{Source} \\
            Assassin's Blood      &  & Ingested &  &   150 agnomas & POI & --- & DMG 258 \\
            Basic Poison (vial)   &  & Injury   &  &   100 agnomas & POI & --- & PHB 153 \\
            Burnt Othur Fumes     &  & Inhaled  &  &   500 agnomas & POI & --- & DMG 258 \\
            Carrion Crawler Mucus &  & Contact  &  &   200 agnomas & POI & --- & DMG 258 \\
            Essence of Ether      &  & Inhaled  &  &   300 agnomas & POI & --- & DMG 258 \\
            Sprouting Poison      &  & Injury   &  &   200 agnomas & POI & --- & DMG 258 \\
            Malice                &  & Inhaled  &  &   250 agnomas & POI & --- & DMG 258 \\
            Midnight Tears        &  & Ingested &  & 1,500 agnomas & POI & --- & DMG 258 \\
        \end{DndTable}
    \end{table*}

    \paragraph{Assassin's Poison}
        A creature subjected to this poison must make a DC 10 Constitution saving throw.
        On a failed save, it takes 6 (1d12) poison damage and is poisoned for 24 hours.
        On a successful save, the creature takes half damage and isn't poisoned.
    \paragraph{Basic Poison}
        You can use the poison in this vial to coat one slashing or piercing weapon or up to three pieces of ammunition.
        Applying the poison takes two actions.
        A creature hit by the poisoned weapon or ammunition must make a DC 10 Constitution saving throw or take 1d4 poison damage.
        Once applied, the poison retains potency for 1 minute before drying.
    \paragraph{Burnt Othur Fumes}
        A creature subjected to this poison must succeed on a DC 13 Constitution saving throw or take 10 (3d6) poison damage, and must repeat the saving throw at the start of each of its turns.
        On each successive failed save, the character takes 3 (1d6) poison damage.
        After three successful saves, the poison ends.
    \paragraph{Carrion Crawler Mucus}
        This poison must be harvested from a dead or incapacitated carrion crawler.
        A creature subjected to this poison must succeed on a DC 13 Constitution saving throw or be poisoned for 1 minute.
        The poisoned creature is paralyzed.
        The creature can repeat the saving throw at the end of each of its turns, ending the effect on itself on a success.
    \paragraph{Essence of Ether}
        A creature subjected to this poison must succeed on a DC 15 Constitution saving throw or become poisoned for 8 hours.
        The poisoned creature is unconscious.
        The creature wakes up if it takes damage or if another creature takes two actions to shake it awake.
    \paragraph{Sprouting Poison}
        This poison is typically made only in a place far removed from sunlight.
        A creature subjected to this poison must succeed on a DC 13 Constitution saving throw or be poisoned for 1 hour.
        If the saving throw fails by 5 or more, the creature is also unconscious while poisoned in this way.
        The creature wakes up if it takes damage or if another creature takes two actions to shake it awake.
    \paragraph{Malice}
        A creature subjected to this poison must succeed on a DC 15 Constitution saving throw or become poisoned for 1 hour.
        The poisoned creature is blinded.
    \paragraph{Midnight Tears}
        A creature that ingests this poison suffers no effect until the stroke of midnight.
        If the poison has not been neutralized before then, the creature must succeed on a DC 17 Constitution saving throw, taking 31 (9d6) poison damage on a failed save, or half as much damage on a successful one.

% !TEX root = ../main.tex
\subsection*{Tools \& Foci} \label{ssec::toolsandfoci}
\subsubsection{Artisan's Tools}
    \begin{table*}[t]%
        \begin{DndTable}[width=\linewidth, header=Artisan's Tools]{Xlccccc}
            \textbf{Toolset} & \textbf{Rarity} & \textbf{Mats.} & \textbf{Total Cost} & \textbf{Tools} & \textbf{Weight} & \textbf{Source} \\
            Alchemist's Supplies  &  &  & 50 agnomas & GLA + MAS & 4 kg   & PHB 154 \\
            Brewer's Supplies     &  &  & 20 agnomas & SMI + MAS & 4.5 kg & PHB 154 \\
            Calligrapher Supplies &  &  & 10 agnomas & TIN       & 2.5 kg & PHB 154 \\
            Carpenter's Tools     &  &  &  8 agnomas & CAR + SMI & 3 kg   & PHB 154 \\
            Cartographer's Tools  &  &  & 15 agnomas & SMI + TIN & 3 kg   & PHB 154 \\
            Cobbler's Tools       &  &  &  5 agnomas & SMI + WEA & 2.5 kg & PHB 154 \\
            Cook's Utensils       &  &  &  1 agnoma  & SMI       & 4 kg   & PHB 154 \\
        \end{DndTable}
    \end{table*}

    \paragraph{Alchemist's Supplies} \label{item::alchemistssupplies}
        This toolset consists of a moor's head, an alembic, a retort, a mortal and pestle, a stirring rod, and one or more flasks and vials.
    \paragraph{Brewer's Supplies}
        These supplies consist of a boil kettle, a mortar and pestle, a mash tun, and usually one or more containers for the ale and liquors produced.
    \paragraph{Calligrapher's Supplies}
        These tools include parchment paper, ink, quills, a ruler, and a paintbrush.
    \paragraph{Carpenter's Tools}
        A set of carpenter's tools include a ruler, a framing square, a bag of nails, one or more clamps, a chisel, a carving knife, a framing hammer, a level, and a saw.
    \paragraph{Cartographer's Tools}
        These tools include a compass, a ruler, a triangle, a pen, callipers, and a bunch of parchment.
    \paragraph{Cobbler's Tools}
        A good set of cobbler's tools include and awl, a closing block, a gouge, a hammer, a last, a lachet, a paring knife, and a bunch of leather, needle and thread.
    \paragraph{Cook's Utensils}
        Among other various utilities, this toolset contains a kettle, a frying pan, a pot, and a set of spoons, knives, bowls, and plates.

\subsubsection{Gaming Kits}
    \begin{table*}[t]%
        \begin{DndTable}[width=\linewidth, header=Kits]{Xlccccc}
            \textbf{Kit} & \textbf{Rarity} & \textbf{Mats.} & \textbf{Total Cost} & \textbf{Tools} & \textbf{Weight} & \textbf{Source} \\
            Chess Set        &  &  &  1 agnoma  & WOO & --- & PHB 154 \\
            Dice Set         &  &  & 10 fobs    & WOO & --- & PHB 154 \\
            Huathem Card Set &  &  & 10 agnomas & CAL & --- & --- \\
            Skull Card Set   &  &  &  3 agnomas & CAL & --- & --- \\
        \end{DndTable}
    \end{table*}

\subsubsection{Kits}
    % Kits are similar to tools in the sense that they require proficiency, but aren't dedicated to crafting shit.
    \begin{table*}[t]%
        \begin{DndTable}[width=\linewidth, header=Kits]{Xlccccc}
            \textbf{Kit} & \textbf{Rarity} & \textbf{Mats.} & \textbf{Total Cost} & \textbf{Tools} & \textbf{Weight} & \textbf{Source} \\
            Climber's Kit         &  &  & 25 agnomas & SMI + WEA & 6 kg   & PHB 151 \\
            Forgery Kit           &  &  & 15 agnomas & CAL + TIN & 2.5 kg & PHB 154 \\
        \end{DndTable}
    \end{table*}

    \paragraph{Climber's Kit}
        A climber's kit includes pitons, boot tips, gloves, and a harness.
        You can use the climber's kit as two actions to anchor yourself; when you do, you can't fall more than 5 meters from the point where you anchored yourself, and you can't climb more than 5 meters away from that point without undoing the anchor.

        You don't need proficiency with this kit to use it effectively.
    \paragraph{Forgery Kit}
        This small box contains a variety of papers and parchments, pens and inks, seals and sealing wax, gold and silver leaf, and other supplies necessary to create convincing forgeries of physical documents.
        A forgery kit is designed to duplicate documents and to make it easier to copy a person's seal or signature.

        You need proficiency with this kit to use it effectively.

\subsubsection{Musical Instruments}
    \begin{table*}[t]%
        \begin{DndTable}[width=\linewidth, header=Musical Instruments]{Xlccccc}
            \textbf{Instrument} & \textbf{Rarity} & \textbf{Mats.} & \textbf{Total Cost} & \textbf{Tools} & \textbf{Weight} & \textbf{Source} \\
            Bagpipes &  &  & 30 agnomas & LEA       & 3 kg   & PHB 154 \\
            Drum     &  &  &  6 agnomas & CAR + LEA & 1.5 kg & PHB 154 \\
            Dulcimer &  &  & 25 agnomas & CAR       & 5 kg   & PHB 154 \\
            Flute    &  &  &  2 agnomas & WOO       & 0.5 kg & PHB 154 \\
        \end{DndTable}
    \end{table*}

\subsubsection{Spellcasting Foci and Utilities}
    \begin{table*}[t]%
        \begin{DndTable}[width=\linewidth, header=Spellcasting Foci]{Xlccccc}
            \textbf{Item} & \textbf{Rarity} & \textbf{Mats.} & \textbf{Total Cost} & \textbf{Tools} & \textbf{Weight} & \textbf{Source} \\
            Component Pouch &  &  &  25 agnomas & LEA       & 1 kg & PHB 151 \\
            Spellbook       &  &  & 100 agnomas & CAL + LEA &
        \end{DndTable}
    \end{table*}

    \paragraph{Component Pouch}
        A component pouch is a small, watertight leather belt pouch that has compartments to hold all the material components and other special items you need to cast your spells, except for those components that have a specific cost (as indicated in a spell's description).
    \paragraph{Spellbook} \label{item::spellbook}
        \textbf{TODO}.
        % Store unlimited number of spells.
        % Change prepared spells after a short rest.

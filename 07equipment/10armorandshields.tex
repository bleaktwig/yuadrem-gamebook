% !TEX root = ../main.tex
\subsection*{Armor and Shields} \label{ssec::armorandshields}
% https://2e.aonprd.com/Armor.aspx
% https://2e.aonprd.com/Shields.aspx
\subsubsection{Armor Properties} \label{ssec::armorproperties}
    \subparagraph{Bulwark}
        The armor covers you so completely that your movement is hindered.
        You do not add your Dexterity modifier to Dexterity saving throws.
    \subparagraph{Chain}
        The armor is so flexible that it bends under strong blows, absorbing some of their strength.
        You gain resistance to the damage made by critical hits from bludgeoning, slashing, an piercing damage.
    \subparagraph{Cloth}
        This armor is light and comfortable.
        You can rest normally while wearing it.
    \subparagraph{Composite}
        The numerous overlapping pieces of this armor protect you from piercing attacks.
        You gain resistance to piercing damage.
    \subparagraph{Leather}
        The thick second skin of the armor disperses blunt force.
        You gain resistance to bludgeoning damage.
    \subparagraph{Noisy}
        The armor is also loud and likely to alert others of your presence, giving you disadvantage on Dexterity (Stealth) checks.
    \subparagraph{Plate}
        The sturdy plate provides no purchase for a cutting edge.
        You gain resistance to slashing damage.
\subsubsection{Light Armor} \label{ssec::lightarmor}
    Made from supple and thin materials, light armor favors agile adventurers since it offers some protection without sacrificing mobility.
    If you wear light armor, you add your Dexterity modifier to the base number from your armor type to determine your Armor Class.

    \begin{table*}[t]%
        \begin{DndTable}[width=\linewidth, header=Light Armor]{Xlclcccccc}
            \textbf{Armor} & \textbf{Rarity} & \textbf{AC} & \textbf{Properties} & \textbf{Mats.} & \textbf{Total Cost} & \textbf{Tools} & \textbf{Weight} & \textbf{Source} \\
            Padded Armor          & Plain     & 11 & Cloth, Noisy   & 3 &      5 agnomas & LEA       & 4 kg   & PHB 144 \\
            Leather Armor         & Plain     & 11 & Leather        & 8 &     10 agnomas & LEA       & 5 kg   & PHB 144 \\
            Studded Leather Armor & Common    & 12 & Leather        & 3 &     50 agnomas & LEA       & 6.5 kg & PHB 144 \\
            Glamoured Armor       & Rare      & 13 & Leather        & 8 &  5,000 agnomas & LEA + WEA & 6.5 kg & DMG 172 \\
            Hunter's Coat         & Very Rare & 12 & Cloth, Leather & 8 & 50,000 agnomas & LEA + WEA & 5 kg   & EGW 267
        \end{DndTable}
    \end{table*}

    \paragraph{Padded Armor}
        Padded armor consists of quilted layers of cloth and batting.
    \paragraph{Leather Armor}
        The breastplate and shoulder protectors of this armor are made of leather that has been stiffened by being boiled in oil.
        The rest of the armor is made of softer and more flexible materials.
    \paragraph{Studded Leather Armor}
        Made from tough but flexible leather, studded leather is reinforced with close-set rivets or spikes.
    \paragraph{Glamoured Studded Leather}
        This armor has the normal appearance of a set of clothing.
        While crafting it, its maker(s) decide what it looks like, including color, style, and accessories, but the armor retains its normal bulk and weight.
    \paragraph{Hunter's Coat}
        This coat has 3 charges.
        When you hit a creature with an attack and that creature doesn't have all its hit points, you can expend 1 charge to deal an extra 1d10 necrotic damage to the target.
        The coat regains 1d3 expended charges daily at dawn.

        The breastplate and shoulder protectors of this armor are made of leather that has been stiffened by being boiled in oil.
        The rest of the armor is made of softer and more flexible materials.
\newpage
\subsubsection{Medium Armor} \label{ssec::mediumarmor}
    Medium armor offers more protection than light armor, but it also impairs movement more.
    If you wear medium armor, you add your Dexterity modifier, to a maximum of +2, to the base number from your armor type to determine your Armor Class.

    \begin{table*}[t]%
        \begin{DndTable}[width=\linewidth, header=Medium Armor]{Xlclcccccc}
            \textbf{Armor} & \textbf{Rarity} & \textbf{AC} & \textbf{Properties} & \textbf{Mats.} & \textbf{Total Cost} & \textbf{Tools} & \textbf{Weight} & \textbf{Source} \\
            Hide Armor        & Plain     & 12 & Leather                 &  8 &     10 agnomas & LEA       & 6 kg    & PHB 144 \\
            Chain Shirt       & Common    & 13 & Chain                   &  4 &     50 agnomas & SMI + WEA & 10 kg   & PHB 144 \\
            Scale Mail        & Common    & 14 & Composite, Noisy        &  4 &     50 agnomas & LEA + SMI & 22.5 kg & PHB 144 \\
            Breastplate       & Uncommon  & 14 & Plate                   &  6 &    400 agnomas & LEA + SMI & 10 kg   & PHB 144 \\
            Half Plate Armor  & Uncommon  & 15 & Plate, Noisy            & 13 &    750 agnomas & LEA + SMI & 20 kg   & PHB 144 \\
            Laminar Armor     & Uncommon  & 14 & Composite               &  6 &    400 agnomas & CAR + LEA & 30 kg   & --- \\
            Woven Iron Shirt  & Uncommon  & 13 & Chain                   &  8 &    500 agnomas & SMI + WEA & 10 kg   & DMG 182 \\
            Nidhogg Mantle    & Rare      & 15 & Composite, Noisy        &  8 &  5,000 agnomas & LEA + WEA & 20 kg   & VGM  81 \\
            Woven Steel Shirt & Rare      & 14 & Chain                   &  8 &  5,000 agnomas & SMI + WEA & 10 kg   & DMG 168 \\
            Wyvern Mail       & Very Rare & 15 & Composite, Plate, Noisy &  8 & 50,000 agnomas & LEA + WEA & 22.5 kg & DMG 165
        \end{DndTable}
    \end{table*}

    \paragraph{Breastplate}
        This armor consists of a fitted metal chest piece worn with supple leather.
        Although it leaves the legs and arms relatively unprotected, this armor provides good protection for the wearer's vital organs while leaving the wearer relatively unencumbered.
    \paragraph{Chain Shirt}
        Made of interlocking metal rings, a chain shirt is worn between layers of clothing.
        This armor offers modest protection to the wearer's upper body and allows the sound of the rings rubbing against one another to be muffled by outer layers.
    \paragraph{Half Plate Armor}
        Half plate consists of shaped metal plates that cover most of the wearer's body.
        It does not include leg protection beyond simple greaves that are attached with leather straps.
    \paragraph{Hide Armor}
        This crude armor consists of thick furs and pelts.
    \paragraph{Laminar Armor}
        This armor consists of laminae of small wood slabs woven together using strong leather.
        The strong wood provides a solid defense against piercing attacks, albeit weighing much more than similar armors.
    \paragraph{Scale Mail}
        This armor consists of a coat, leggings, and a separate skirt of leather covered with overlapping pieces of metal, much like the scales of a fish.
        The suit includes gauntlets.
    \paragraph{Woven Iron Shirt}
        By using thin iron threads, a woven iron shirt provides good protection while remaining light.
        This armor can be worn under normal clothes, becoming effectively invisible.
    \paragraph{Nidhogg Mantle}
        Crafted from nidhogg scales, this carapace-like armor encases portions of the wearer's shoulders, neck, and chest.
        A creature wearing this armor can breathe normally in any environment, and has advantage on saving throws against harmful gases (such as those created by a cloudkill spell, a stinking cloud spell, inhaled poisons, and the breath weapons of some creatures).
    \paragraph{Woven Steel Shirt}
        By using thin steel threads, a woven steel shirt provides good protection while remaining light.
        This armor can be worn under normal clothes, becoming effectively invisible.
    \paragraph{Wyvern Mail}
        This armor is made from wyvern scales and the carefully skinned hide of a wyvern.
        While wearing this armor, you have resistance to poison damage.
\newpage
\subsubsection{Small Shields} \label{ssec::smallshields}
This type of shield grants a +1 bonus to your AC.

\subsubsection{Medium Shields} \label{ssec::mediumshields}
This type of shield grants a +2 bonus to your AC.

\subsubsection{Heavy Shields} \label{ssec::heavyshields}
This type of shield grants a +3 bonus to your AC.
In addition, it grants you half cover against ranged attacks, but the move action requires 2 actions for you.

% !TEX root = ../main.tex
\chapter{Equipment} \label{ch::equipment}
\section{Currency} \label{sec::currency}
\subparagraph{Fob}
    Copper coin of little value.
    Translates to ``common''.
\subparagraph{Agnoma}
    Nickel coin worth 100 fobs.
    Named after Agnomakhos, the original et ruler of Mephetis.

\section{Materials and Labor} \label{sec::materialsandlabor}
    Any item in Yuadrem that can be bought can be crafted as well by a suficiently skilled artisan.
    Just like items, materials are separated by rarity, with increased rarity having increased cost.
    In addition, the labor required to craft rarer items is naturally harder, and thus is more expensive.
    This relation is shown in the Materials and Labor Costs table.

    Different items require a different number of materials.
    In addition to their final cost, items listed in the following sections also show the required number of materials needed to craft them, and the relevant set of artisan's tools needed to craft them.

    % NOTE. Maybe I'll add this. Maybe I won't. IDK yet.
    % Due to their scarcity in the world, some items require materials more expensive than what is common for their rarity.
    % These are marked by an $\ast$ symbol, and their material price is specified in their description.

    \begin{DndTable}[width=\linewidth, header=Materials and Labor Costs]{llll}
        \textbf{Rarity} & \textbf{Material Price} & \textbf{Labor Price} \\
        Mundane         &      5 fobs             &     10 fobs    \\
        Plain           &      1 agnoma           &      2 agnomas \\
        Common          &     10 agnomas          &     20 agnomas \\
        Uncommon        &     50 agnomas          &    100 agnomas \\
        Rare            &    500 agnomas          &  1,000 agnomas \\
        Very Rare       &  5,000 agnomas          & 10,000 agnomas \\
        Legendary       & 25,000 agnomas          & 50,000 agnomas
    \end{DndTable}

    \subsection*{Foraging} \label{ssec::foraging}
    % TODO. I should write a bit about foraging and obtaining materials.

\section{Items} \label{sec::items}
    Items marked with the $\odot$ symbol require attunement to be used.

    \newpage~\newpage

    % TODO. UPDATE PRICES BASED ON THE LIST.
    % !TEX root = ../main.tex
\subsection*{Ammunition} \label{ssec::ammunition}
\begin{table*}[b]%
    \begin{DndTable}[width=\linewidth, header=Ammunition]{Xcccccc}
        \textbf{Item} & \textbf{Rarity} & \textbf{Mats.} & \textbf{Total Cost} & \textbf{Tools} & \textbf{Weight} & \textbf{Source} \\
        Arrow                       & Mundane   &  1 &     15 fobs    & WOO    & 25 gr  & PHB 150 \\
        Arrows (20)                 & Mundane   & 18 &      1 agnoma  & WOO    & 0.5 kg & PHB 150 \\
        Blowgun Needle              & Mundane   &  1 &     15 fobs    & WOO    & 25 gr  & PHB 150 \\
        Blowgun Needles (20)        & Mundane   & 18 &      1 agnoma  & WOO    & 0.5 kg & PHB 150 \\
        Crossbow Bolt               & Mundane   &  1 &     15 fobs    & WOO    & 50 gr  & PHB 150 \\
        Crossbow Bolts (20)         & Mundane   & 18 &      1 agnoma  & WOO    & 1 kg   & PHB 150 \\
        Sling Bullet                & Mundane   &  1 &     15 fobs    & WOO    & 50 gr  & PHB 150 \\
        Sling Bullets (20)          & Mundane   & 18 &      1 agnoma  & WOO    & 1 kg   & PHB 150 \\
        Crossbow Bolt Case          & Plain     &  1 &      3 agnomas & CAR    & 0.5 kg & PHB 151 \\
        Quiver                      & Plain     &  1 &      3 agnomas & LEA    & 0.5 kg & PHB 153 \\
        Bullet                      & Common    &  1 &     30 agnomas & TIN    & 100 g  & DMG 268 \\
        Bullets (10)                & Common    &  8 &    100 agnomas & TIN    & 1 kg   & DMG 268 \\
        +1 Ammunition (20)          & Uncommon  &  4 &    300 agnomas & $\ast$ & $\ast$ & DMG 150 \\
        Barbed Ammunition (20)      & Uncommon  &  4 &    300 agnomas & $\ast$ & $\ast$ & ---     \\
        Destructive Ammunition (20) & Uncommon  &  4 &    300 agnomas & $\ast$ & $\ast$ & XGE  78 \\
        Reinforced Ammunition (20)  & Uncommon  &  4 &    300 agnomas & $\ast$ & $\ast$ & XGE 139 \\
        Walloping Ammunition (20)   & Uncommon  &  4 &    300 agnomas & $\ast$ & $\ast$ & XGE 139 \\
        +2 Ammunition (20)          & Rare      &  4 &  3,000 agnomas & $\ast$ & $\ast$ & DMG 150 \\
        Cacophonous Ammunition      & Rare      &  1 &  1,500 agnomas & $\ast$ & $\ast$ & ---     \\
        Fulgurating Ammunition (5)  & Rare      &  1 &  1,500 agnomas & $\ast$ & $\ast$ & MOT 198 \\
        Stunning Ammunition (5)     & Rare      &  1 &  1,500 agnomas & $\ast$ & $\ast$ & MOT 198 \\
        +3 Ammunition (20)          & Very Rare &  4 & 30,000 agnomas & $\ast$ & $\ast$ & DMG 150 \\
        Slaying Ammunition (4)      & Very Rare &  4 & 30,000 agnomas & $\ast$ & $\ast$ & DMG 152
    \end{DndTable}
\end{table*}

$\ast$ \textit{Required tool and weight depends on the type of ammunition.
Refer to the mundane or common version of the ammunition.}

\paragraph{+1/+2/+3 Ammunition}
    You have a +1/+2/+3 bonus to attack and damage rolls made with this piece of ammunition.
    These pieces of ammunition break following the normal rules for losing ammunition.
\paragraph{Barbed Ammunition}
    A creature attacked by a piece of barbed ammunition takes 1d4 necrotic damage at the beginning of its next turn.
    This effect doesn't stack.
\paragraph{Cacophonous Ammunition}
    Upon landing, a wave of thunderous force sweeps out from this projectile.
    Each creature in a 3-meters sphere originating from its landing location must make a DC 13 Constitution saving throw, taking 2d8 thunder damage and being pushed 2 meters away on a failed save.
    Unsecured objects that are completely within the area of effect are automatically pushed 2 meters away from the landing location.

    In addition, if you hit a creature with this projectile it must succeed on a DC 13 Constitution saving throw, taking 3d10 lightning damage on a failed save, or half as much damage on a successful save.

    This piece of ammunition breaks upon landing.
\paragraph{Crossbow Bolt Case}
    This wooden case can hold up to twenty crossbow bolts.
\paragraph{Destructive Ammunition}
    Coated with heavy steel, this piece of ammunition is unusually effective when used to break objects.
    Whenever of a piece of destructive ammunition hits an object, the hit is a critical hit.
\paragraph{Fulgurating Ammunition}
    On a hit, this projectile deals an extra 2d8 lightning damage to its target.
    All other creatures within 2 meters of the target must each succeed on a DC 15 Constitution saving throw or take 1d8 thunder damage.
\paragraph{Quiver}
    A quiver can hold up to 20 arrows.
\paragraph{Reinforced Ammunition}
    This piece of ammunition doesn't break when used in combat.
    It can however be broken by succeeding on a DC 15 Athletics check.
\paragraph{Slaying Ammunition}
    A slaying ammunition is a specially designed weapon meant to slay a particular kind of creature.
    If a creature belonging to the type, race, or group associated with a slaying ammunition takes damage from it, the creature must make a DC 17 Constitution saving throw, taking an extra 6d10 piercing damage on a failed save, or half as much extra damage on a successful one.

    Once an arrow of slaying deals its extra damage to a creature, it breaks.
\paragraph{Stunning Ammunition}
    On a hit, this projectile deals an extra 1d10 force damage, and the target must roll a DC 13 Constitution saving throw, becoming stunned until the end of your next turn on a failed save.
\paragraph{Walloping Ammunition}
    This ammunition packs a wallop.
    A creature hit by the ammunition must succeed on a DC 10 Strength saving throw or be knocked prone.
\pagebreak

    % !TEX root = ../main.tex
\subsection*{Armor and Shields} \label{ssec::armorandshields}
% https://2e.aonprd.com/Armor.aspx
% https://2e.aonprd.com/Shields.aspx
% NOTE: Heavy shields give you half cover against ranged attacks, but you lose 3 meters of your movemens speed while they are equipped.

\subsection*{Armor Properties} \label{ssec::armorproperties}
    \subparagraph{Bulwark} The armor covers you so completely that your movement is hindered.
    You do not add your Dexterity modifier to Dexterity saving throws.

    \subparagraph{Chain} The armor is so flexible that it bends under strong blows, absorbing some of their strength.
    You gain resistance to the damage made by critical hits from bludgeoning, slashing, an piercing damage.

    \subparagraph{Cloth} This armor is light and comfortable.
    You can rest normally while wearing it.

    \subparagraph{Composite} The numerous overlapping pieces of this armor protect you from piercing attacks.
    You gain resistance to piercing damage.

    \subparagraph{Leather} The thick second skin of the armor disperses blunt force.
    You gain resistance to bludgeoning damage.

    \subparagraph{Noisy} The armor is also loud and likely to alert others of your presence, giving you disadvantage on Dexterity (Stealth) checks.

    \subparagraph{Plate} The sturdy plate provides no purchase for a cutting edge.
    You gain resistance to slashing damage.

    % !TEX root = ../main.tex
\section*{Clothing} \label{sec::clothing}
    \begin{table*}[b]%
        \begin{DndTable}[width=\linewidth, header=Clothing and Accessories]{Xlccccc}
            \textbf{Item} & \textbf{Rarity} & \textbf{Mats.} & \textbf{Total Cost} & \textbf{Tools} & \textbf{Weight} & \textbf{Source} \\
            \multicolumn{7}{l}{\hspace{0.5cm}\textit{Clothes}} \\
            Common Clothes        & Mundane  & 8 &  50 fobs    & WEA & 1.5 kg & PHB   150 \\
            Costume Clothes       & Plain    & 3 &   5 agnomas & WEA & 2 kg   & PHB   150 \\
            Robes                 & Plain    & 1 &   1 agnoma  & WEA & 2 kg   & PHB   150 \\
            Traveler's Clothes    & Plain    & 1 &   2 agnomas & WEA & 2 kg   & PHB   150 \\
            Warm Clothing         & Plain    & 8 &  10 agnomas & WEA & 2.5 kg & IDRotF 20 \\
            Fine Clothes          & Common   & 1 &  30 agnomas & WEA & 3 kg   & PHB   150 \\
            \multicolumn{7}{l}{\hspace{0.5cm}\textit{Accessories}} \\
            \multicolumn{7}{l}{\hspace{0.5cm}\textit{Jewelry}} \\
            Amulet                & Plain    & 3 &   5 agnomas & JEW & 0.5 kg & PHB   151 \\
            Signet Ring           & Plain    & 3 &   5 agnomas & JEW & ---    & PHB   150 \\
            Brooch of Shielding   & Uncommon & 8 & 500 agnomas & GLA & ---    & DMG   156 \\
            \multicolumn{7}{l}{\hspace{0.5cm}\textit{Qualars}} \\
            \multicolumn{7}{l}{\hspace{0.5cm}\textit{Misc.}} \\
            Eyes of Charming      & Uncommon & 8 & 500 agnomas & GLA & ---    & DMG   168 \\
            Eyes of Minute Seeing & Uncommon & 8 & 500 agnomas & GLA & ---    & DMG   168 \\
            Eyes of the Eagle     & Uncommon & 8 & 500 agnomas & GLA & ---    & DMG   168
        \end{DndTable}
    \end{table*}

    \paragraph{Brooch of Shielding $\odot$}
        While wearing this brooch, you have resistance against force damage.
    \paragraph{Eyes of Charming $\odot$}
        These crystal lenses fit over the eyes.
        They have 3 charges.
        While wearing them, you can expend 1 charge as an action to cast the charm person spell (see page \pageref{spell::charmperson}) with a save DC of 13 on a humanoid within 6 meters of you, provided that you and the target can see each other.
        The lenses regain all expended charges daily at dawn.
    \paragraph{Eyes of Minute Seeing}
        These crystal lenses fit over the eyes.
        While wearing them, you can see much better than normal out to a range of 30 cm.
        You have advantage on Intelligence (Investigation) checks that rely on sight while searching an area or studying an object within that range.
    \paragraph{Eyes of the Eagle $\odot$}
        These crystal lenses fit over the eyes.
        While wearing them, you have advantage on Wisdom (Perception) checks that rely on sight.
        In conditions of clear visibility, you can make out details of even extremely distant creatures and objects as small as 60 cm. across.
    \paragraph{Warm Clothing}
        This outfit consists of a heavy fur coat or cloak over layers of wool clothing, as well as a fur-lined hat or hood, goggles, and fur-lined leather boots and gloves.

        As long as this clothing remains dry, its wearer automatically succeeds on saving throws against the effects of extreme cold.
\newpage~\newpage

    % !TEX root = ../main.tex
\section*{Food, Drink, and Lodging} \label{sec::fooddrinkandlodging}
\begin{table*}[b]%
    \begin{DndTable}[width=\linewidth, header=Food and Lodging]{Xlccccc}
        \textbf{Food or Service} & \textbf{Rarity} & \textbf{Mats.} & \textbf{Total Cost} & \textbf{Tools} & \textbf{Weight} & \textbf{Source} \\
        \multicolumn{7}{l}{\hspace{0.5cm}\textit{Inn Stay (per day)}} \\
        Squalid              & ---       & --- &     7 fobs    & --- & ---  & PHB 158 \\
        Poor                 & ---       & --- &    10 fobs    & --- & ---  & PHB 158 \\
        Modest               & ---       & --- &    50 fobs    & --- & ---  & PHB 158 \\
        Comfortable          & ---       & --- &    80 fobs    & --- & ---  & PHB 158 \\
        Wealthy              & ---       & --- &     2 agnomas & --- & ---  & PHB 158 \\
        Aristocratic         & ---       & --- &     4 agnomas & --- & ---  & PHB 158 \\
        \multicolumn{7}{l}{\hspace{0.5cm}\textit{Services}} \\
        Coach cab between towns (per km) & --- & --- &  2 fobs    & --- & --- & PHB 159 \\
        Coach cab within a city          & --- & --- &  1 fob     & --- & --- & PHB 159 \\
        Skilled hireling (per day)       & --- & --- &  2 agnomas & --- & --- & PHB 159 \\
        Untrained Hireling (per day)     & --- & --- & 20 fobs    & --- & --- & PHB 159 \\
        Messenger (per km)               & --- & --- &  1 fob     & --- & --- & PHB 159 \\
        Road or gate toll                & --- & --- &  1 fob     & --- & --- & PHB 159 \\
        Ship's passage (per km)          & --- & --- &  5 fobs    & --- & --- & PHB 159 \\
        \multicolumn{7}{l}{\hspace{0.5cm}\textit{Food (per day)}} \\
        Animal Feed          & Mundane   & 1 &       5 fobs    & COO & 5 kg & PHB 157 \\
        Squalid              & Mundane   &  1  &     3 fobs    & COO & ---  & PHB 158 \\
        Poor                 & Mundane   &  1  &     6 fobs    & COO & ---  & PHB 158 \\
        Modest               & Mundane   &  1  &    30 fobs    & COO & ---  & PHB 158 \\
        Comfortable          & Mundane   &  1  &    50 fobs    & COO & ---  & PHB 158 \\
        Wealthy              & Mundane   &  1  &    80 fobs    & COO & ---  & PHB 158 \\
        Aristocratic         & Plain     &  1  &     2 agnomas & COO & ---  & PHB 158 \\
        Banquet (per person) & Common    & 1   &    10 agnomas & COO & ---  & PHB 158 \\
    \end{DndTable}
\end{table*}

\subsection*{Lifestyle Expenses}
    At the beginning of a week or month in a town or city, you choose a lifestyle choice and pay the price to sustain that lifestyle.
    The prices are listed per day, so multiply the cost by 6 for the cost of one week or by 24 for the cost of one month.
    % Your lifestyle might change from one period to the next, based on the funds you have at your disposal, or you might maintain the same lifestyle throughout your character's career.

    Your lifestyle choice can have consequences.
    Maintaining a wealthy lifestyle might help you make contacts with the rich and powerful, though you run the risk of attracting thieves.
    Likewise, living frugally might help you avoid criminals, but you are unlikely to make powerful connections.

    \paragraph{Wretched}
        You live in inhumane conditions.
        With no place to call home, you shelter wherever you can, sneaking into barns, huddling in old crates, and relying on the good graces of people better off than you.
        % A wretched lifestyle presents abundant dangers.
        % Violence, disease, and hunger follow you wherever you go.
        % Other wretched people covet your armor, weapons, and adventuring gear, which represent a fortune by their standards.
        % You are beneath the notice of most people.
    \paragraph{Squalid}
        You live in a leaky stable, a mud-floored hut just outside town, or a vermin-infested boarding house in the worst part of town.
        You have shelter from the elements, but you live in a desperate and often violent environment, in places rife with disease, hunger, and misfortune.
        % You are beneath the notice of most people, and you have few legal protections.
        % Most people at this lifestyle level have suffered some terrible setback.
        % They might be disturbed, marked as exiles, or suffer from disease.
    \paragraph{Poor}
        A poor lifestyle means going without the comforts available in a stable community.
        Simple food and lodgings, threadbare clothing, and unpredictable conditions result in a sufficient, though probably unpleasant, experience.
        Your accommodations might be a room in a flophouse or in the common room above a tavern.
        % You benefit from some legal protections, but you still have to contend with violence, crime, and disease.
        % People at this lifestyle level tend to be unskilled laborers, costermongers, peddlers, thieves, mercenaries, and other disreputable types.
    \paragraph{Modest}
        A modest lifestyle keeps you out of the slums and ensures that you can maintain your equipment.
        You live in an older part of town, renting a room in a boarding house, inn, or temple.
        You don't go hungry or thirsty, and your living conditions are clean, if simple.
        % Ordinary people living modest lifestyles include soldiers with families, laborers, students, priests, and the like.
    \paragraph{Comfortable}
        Choosing a comfortable lifestyle means that you can afford nicer clothing and can easily maintain your equipment.
        You live in a small cottage in a middle-class neighborhood or in a private room at a fine inn.
        You associate with merchants, skilled tradespeople, and military officers.
    \paragraph{Wealthy}
        Choosing a wealthy lifestyle means living a life of luxury, though you might not have achieved the social status associated with the old money of nobility or royalty.
        You live a lifestyle comparable to that of a highly successful merchant, a favored servant of the royalty, or the owner of a few small businesses.
        You have respectable lodgings, usually a spacious home in a good part of town or a comfortable suite at a fine inn.
        % You likely have a small staff of servants.
    \paragraph{Aristocratic}
        You live a life of plenty and comfort.
        You move in circles populated by the most powerful people in the community.
        You have excellent lodgings, perhaps a townhouse in the nicest part of town or rooms in the finest inn.
        % You dine at the best restaurants, retain the most skilled and fashionable tailor, and have servants attending to your every need.
        % You receive invitations to the social gatherings of the rich and powerful, and spend evenings in the company of politicians, guild leaders, high priests, and nobility.
        % You must also contend with the highest levels of deceit and treachery.

    % The wealthier you are, the greater the chance you will be drawn into political intrigue as a pawn or participant.
\subsection*{Meals} \label{ssec::meals}
    Unless otherwise specified, the prices listed for brews are for a feast, which feeds up to 8 creatures.
    Divide this by 8 for the price of one meal.

    A meal is consumed during the day, and its benefits (if any) last for the whole day.
    After cooked, the benefits associated to a meal last for one day, after which the food spoils.
    You can only be affected by one meal at the same time.

    \begin{table*}[b]%
        \begin{DndTable}[width=\linewidth, header=Meals]{Xlccccc}
            \textbf{Meal} & \textbf{Rarity} & \textbf{Mats.} & \textbf{Total Cost} & \textbf{Tools} & \textbf{Weight} & \textbf{Source} \\
            Loaf of Bread    & Mundane & 1 &  8 fobs    & COO & ---  & PHB 158 \\
            Hunk of Cheese   & Plain   & 1 & 80 fobs    & COO & ---  & PHB 158 \\
            Rations (1 meal) & Plain   & 1 & 50 fobs    & COO & 1 kg & PHB 153 \\
            Chunk of Meat    & Common  & 1 & 24 agnomas & COO & ---  & PHB 158
        \end{DndTable}
    \end{table*}

    % Food: Chili that gives Investiture of Flame

    % Ioun's Stone Pig % Random Ioun strong --- randomized when drinking.
    % Heroes Feast % Heroes Feast spell.
\newpage~\newpage
\subsection*{Brews} \label{ssec::brews}
    % TODO. Brews require a TON of balance and to reference the correct sources.
    The prices listed for brews are for a small keg, which fits 8 tankards worth of brew.
    Divide this by 8 for the price of a tankard, or by 16 for the price of a cup.
    Prices consider only the brew itself, not the medium of transportation.

    It takes one minute to drink a brew and gain its benefits, which can be done at the end of a short rest.
    You can only be affected by one brew at the same time.
    % After drinking a brew a creature rolls a DC 8 Constitution saving throw.
    % On a failure, it empties the content of the brew

    \begin{table*}[b]%
        \begin{DndTable}[width=\linewidth, header=Brews]{Xlccccc}
            \textbf{Brew} & \textbf{Rarity} & \textbf{Mats.} & \textbf{Total Cost} & \textbf{Tools} & \textbf{Weight} & \textbf{Source} \\
            Ale                      & Mundane   & 2 &      20 fobs    & BRE &  4 kg   & PHB 158 \\
            Common Wine              & Plain     & 1 &       3 agnomas & BRE &  4 kg   & PHB 158 \\
            Ale of Heroism           & Common    & 3 &      50 agnomas & BRE &  4 kg   & --- \\
            Dozing Wine              & Common    & 3 &      50 agnomas & BRE &  4 kg   & --- \\
            Exalted Beer             & Common    & 3 &      50 agnomas & BRE &  4 kg   & --- \\
            Expeditious Brew         & Common    & 3 &      50 agnomas & BRE &  4 kg   & --- \\
            Laughing Ale             & Common    & 3 &      50 agnomas & BRE &  4 kg   & --- \\
            Beer of Vitality         & Uncommon  & 3 &     250 agnomas & BRE &  4 kg   & --- \\
            Brew of Absorption       & Uncommon  & 3 &     250 agnomas & BRE &  4 kg   & --- \\
            Brew of Giant's Strength & Uncommon  & 3 &     250 agnomas & BRE &  4 kg   & --- \\
            Calming Wine             & Uncommon  & 3 &     250 agnomas & BRE &  4 kg   & --- \\
            Dead Man's Ale           & Uncommon  & 3 &     250 agnomas & BRE &  4 kg   & --- \\
            Fine Wine                & Uncommon  & 2 &     200 agnomas & BRE &  4 kg   & PHB 158 \\
            Madman Wine              & Uncommon  & 3 &     250 agnomas & BRE &  4 kg   & --- \\
            Speaker's Whiskey        & Uncommon  & 3 &     250 agnomas & BRE &  4 kg   & --- \\
            Thassa's Mockery         & Uncommon  & 3 &     250 agnomas & BRE &  4 kg   & --- \\
            Breathkiller             & Rare      & 3 &   2,500 agnomas & BRE &  4 kg   & --- \\
            Brew of Clairvoyance     & Rare      & 3 &   2,500 agnomas & BRE &  4 kg   & --- \\
            Brew of Invulnerability  & Rare      & 3 &   2,500 agnomas & BRE &  4 kg   & --- \\
            Brew of Letargy          & Rare      & 3 &   2,500 agnomas & BRE &  4 kg   & --- \\
            Brew of Maximum Power    & Rare      & 3 &   2,500 agnomas & BRE &  4 kg   & EGW 268 \\
            Mindblower               & Rare      & 3 &   2,500 agnomas & BRE &  4 kg   & --- \\
            Soulcatcher (Bottle)     & Rare      & 8 &   5,000 agnomas & BRE &  0.5 kg & --- \\
            Brainbender              & Very Rare & 3 &  25,000 agnomas & BRE &  4 kg   & --- \\
            Brew of Vitality         & Very Rare & 3 &  25,000 agnomas & BRE &  4 kg   & --- \\
            Brew of Speed            & Very Rare & 3 &  25,000 agnomas & BRE &  4 kg   & --- \\
            Erebos' Ooze (Jar)       & Very Rare & 8 &  50,000 agnomas & BRE &  1 kg   & ERLW 278 \\
            Kickstarter (Bottle)     & Very Rare & 8 &  50,000 agnomas & BRE &  0.5 kg & --- \\
            Dancer's Whiskey         & Legendary & 3 & 125,000 agnomas & BRE &  4 kg   & --- \\
            Groundshaker             & Legendary & 3 & 125,000 agnomas & BRE &  4 kg   & --- \\
            Purphoros' Maker         & Legendary & 3 & 125,000 agnomas & BRE &  4 kg   & --- \\
            Seer's Vodka             & Legendary & 3 & 125,000 agnomas & BRE &  4 kg   & --- \\
        \end{DndTable}
    \end{table*}

    \paragraph{Ale of Heroism} % Heroism
        This ale is honey-flavored, reminding you of those most important to you.
        A creature who drinks this is imbued with bravery.
        For an hour, the creature is immune to being frightened and gains 1 temporary hit point at the start of each of its turns.
        After the hour passes, the creature loses all temporary hit points gained from this brew.
    \paragraph{Beer of Vitality} % Aid
        This beer emboldens creatures with toughness and resolve.
        The drinker's hit point maximum and current hit points increase by 5 for 8 hours.
    \paragraph{Brainbender} % Confusion
        This brew is a nut-flavored greenish liquid.
        The liquid assaults and twists creatures' minds, spawning delusions and provoking uncontrolled action.
        Each creature that drinks the brew must succeed on a DC 16 Wisdom saving throw or be affected by it for an hour.

        An affected target can't take reactions and must roll a d10 at the start of each of its turns to determine its behavior for that turn.

        \begin{DndTable}[width=\linewidth, header=Confusion Behaviour]{lX}
            \textbf{d10} & \textbf{Behaviour} \\
            1            & The creature uses all its movement to move in a random direction. To determine the direction, roll a d8 and assign a direction to each die face.
            The creature doesn't take any other action this turn. \\
            2-6	         & The creature doesn't move or take actions this turn. \\
            7-8	         & The creature uses its action to make a melee attack against a randomly determined creature within its reach.
            If there is no creature within its reach, the creature does nothing this turn. \\
            9-10         & The creature can act and move normally.
        \end{DndTable}

        Every 10 minutes, an affected creature can make a DC 16 Wisdom saving throw.
        If it succeeds, this effect ends for that it.
    \paragraph{Breathkiller} % Stinking Cloud
        This atrocity of a brew is a yellow muddy liquid that leaves a terrible itching sensation upon passing through your throat.
        After drinking it, a creature becomes surrounded by a 6-meter-radius sphere of yellow, nauseating gas centered on itself.
        The cloud spreads around corners, and its area is heavily obscured.
        The cloud lingers in the air for an hour after the brew is drunk.

        Each creature that is completely within the cloud at the start of its turn must make a DC 14 Constitution saving throw against poison.
        On a failed save, the creature spends its action that turn retching and reeling.
        Creatures that don't need to breathe or are immune to poison automatically succeed on this saving throw.
    \paragraph{Brew of Absorption} % Absorb Elements + Dragon's breath of the absorbed element
        This drink seems to remain in your throat after drinking it, ready to return to the world on command.
        When the drinker takes acid, cold, fire, lightning, or poison damage within 8 hours of drinking the brew, it can use its reaction to gain resistance against the attack.
        As part of the same reaction, it can then exhale energy in a 4.5 meter cone.
        Each creature in the area must make a DC 12 Dexterity saving throw, taking 3d6 damage of the absorbed type on a failed save, or half as much damage on a successful one.
        The drinker can only use this ability once during the brew's duration.
    \paragraph{Brew of Clairvoyance} % Clairvoyance
        An eyeball bobs in this yellowish vodka but vanishes as the brew is drunk.
        Upon drinking this brew, you create an invisible sensor within 2 kilometers in a location familiar to you (a place you have visited or seen before) or in an obvious location that is unfamiliar to you (such as behind a door, around a corner, or in a grove of trees).
        The sensor remains in place for the duration, and it can't be attacked or otherwise interacted with.

        You can see and hear through the sensor as if you were in its space.

        A creature that can see the sensor (such as a creature with truesight) sees a luminous, intangible orb about the size of your fist.

        The sensor remains in position for 8 hours, after which it disappears.
    \paragraph{Brew of Giant Strength}
        This brew's transparent liquid has floating in it a sliver of fingernail from a cyclops.
        When you drink it, your Strength score changes to 21 for 1 hour.
        The potion has no effect on you if your Strength is equal to or greater than that score.
    \paragraph{Brew of Invulnerability}
        This syrupy liquid looks like liquefied iron.
        For an hour after drinking it, you have resistance to all physical damage.
    \paragraph{Brew of Letargy} % Slow
        This thick whiskey seems to flow slower than naturally possible.
        After drinking it, a creature must succeed on a DC 14 Wisdom saving throw or be affected by this brew for an hour.

        An affected target's speed is halved, it takes a -2 penalty to AC and Dexterity saving throws, and it can't use reactions.
        On its turn, it has only 2 actions instead of the normal 3.

        If the creature attempts to cast a spell with a casting time of 2 or more actions, roll a d20.
        On an 11 or higher, the spell doesn't take effect until the creature's next turn, and the creature must use its actions on that turn to complete the spell.
        If it can't, the spell is wasted.

        A creature affected by this brew makes another DC 14 Wisdom saving throw every 10 minutes.
        On a successful save, the effect ends for it.
    \paragraph{Brew of Maximum Power}
        This glowing purple liquid smells of sugar and plum, but it has a muddy taste.

        The first time you cast a damage-dealing spell of 4th level or lower within an hour after drinking the brew, instead of rolling dice to determine the damage dealt, you can instead use the highest number possible for each die.
    \paragraph{Brew of Speed} % Haste
        This black liquid flows faster than what seems reasonable.
        For an hour, the drinker's speed is doubled, it gains a +2 bonus to AC, it has advantage on Dexterity saving throws, and it gains an additional action on each of its turns.

        When the effect ends, the target can't move or take actions for a minute, as a wave of lethargy sweeps over it.
    \paragraph{Brew of Vitality} % Potion of Vitality
        The brew's crimson liquid regularly pulses with dull light, calling to mind a heartbeat.
        When you drink this brew, it removes any exhaustion you are suffering and cures any disease or poison affecting you.
        For the next 24 hours, you regain the maximum number of hit points for any Hit Die you spend.
    \paragraph{Calming Wine} % Calm Emotions
        This wine's fruity flavor seems to calm emotions on command.
        Any creature who drinks this rolls a DC 12 Charisma saving throw.
        A creature can choose to fail this saving throw if it wishes.
        If a creature fails, two effects occur with a duration of an hour:
        \begin{itemize}
            \item Any effect causing a target to be charmed or frightened is suppressed.
            When this effect ends, any suppressed effect resumes, provided that its duration has not expired in the meantime.
            \item The target feels indifferent about creatures that it is hostile toward.
            This indifference ends if the target is attacked or harmed by a spell or if it witnesses any of its friends being harmed.
            When the effect ends, the creature becomes hostile again, unless the DM rules otherwise.
        \end{itemize}

        A creature affected by this brew makes another DC 12 Charisma saving throw every 10 minutes.
        On a successful save, the effect ends for it.
    \paragraph{Dancer's Whiskey} % Otto's Irresistible Dance
        This fuzzy pink brew feels like small explosions happening inside your mouth.
        After finishing the brew, the drinker begins a comic dance in place: shuffling, tapping its feet, and capering for an hour.

        A dancing creature must use all its movement to dance without leaving its space and has disadvantage on Dexterity saving throws and attack rolls.
        While the target is affected by this brew, other creatures have advantage on attack rolls against it.
        Every 10 minutes, a dancing creature makes a DC 18 Wisdom saving throw to regain control of itself.
        On a successful save, the effect ends.
    \paragraph{Dead Man's Ale} % Blindness/Deafness
        This brew looks like common ale, hiding a terrible effect within.
        After drinking it, the creature rolls a DC 12 Constitution saving throw.
        If it fails, the target is blinded for an hour.

        A creature affected by this brew can makes another DC 12 Constitution saving throw every 10 minutes.
        On a successful save, the effect ends for it.
    \paragraph{Dozing Wine} % Sleep
        This wine is so comforting that it sends creatures into slumber.
        Any creature who drinks this must succeed on a DC 10 Constitution saving throw.
        On a failure, the creature falls unconscious for an hour, until the sleeper takes damage, or someone uses an action to shake or slap the sleeper awake.

        A creature affected by this brew makes another DC 10 Constitution saving throw every 10 minutes.
        On a successful save, the effect ends for it.
    \paragraph{Erebos' Ooze (Jar) $\odot$} \label{item::erebosooze} % Kyrzin's Ooze
        This opalescent, symbiotic goo comes sealed in a jar and slowly shifts and moves, as if endlessly exploring the jar's interior.
        You can attune to this item by drinking the contents of the jar, unlocking the following properties.

        \paragraph{Resistant} While attuned to Erebos' ooze, you have resistance to poison and acid damage, and you're immune to the poisoned condition.
        \paragraph{Amorphous} Using two actions, you can speak a command word and cause your body to assume the amorphous qualities of an ooze.
        For the next minute, you (along with any equipment you're wearing or carrying) can move through a space as narrow as 1 inch wide without squeezing.
        Once you use this property, it can't be used again until the next dawn.
        \paragraph{Acid Breath} Using two actions, you can exhale acid in a 6-meter line that is 1 meters wide.
        Each creature in that line must make a DC 15 Dexterity saving throw, taking 36 (8d8) acid damage on a failed save, or half as much damage on a successful one.
        Once you use this property, it can't be used again until the next dawn.
        \paragraph{Symbiotic Nature} The ooze can't be removed from you while you're attuned to it, and you can't voluntarily end your attunement to it.
        The only way to remove the ooze is by drinking an Oozekiller potion (see page \pageref{item::oozekiller}).

        If you die while the ooze is inside you, it bursts out and engulfs you, turning your corpse into a black pudding.
    \paragraph{Exalted Beer} % Bless
        The sweet flavor of this beer truly is a blessing.
        After drinking it, the creature can add a d4 to any attack roll or saving throw it makes for one hour.
    \paragraph{Expeditious Brew} % Expeditious Retreat
        This brew feels like caustic acid falling through your throat.
        After drinking it, the creature gains a bonus of 2 meters to its movement speed for one hour.
    \paragraph{Groundshaker} % Investiture of Stone
        Small bits of stone float aimlessly in this amber root beer.
        For an hour after drinking it, bits of rock spread across your body, and you gain the following benefits:
        \begin{itemize}
            \item You have resistance to bludgeoning, piercing, and slashing damage from nonmagical attacks.
            \item You can use your action to create a small earthquake on the ground in a 4.5-meter radius centered on you.
            Other creatures on that ground must succeed on a DC 18 Dexterity saving throw or be knocked prone.
            \item You can move across difficult terrain made of earth or stone without spending extra movement.
            You can move through solid earth or stone as if it was air and without destabilizing it, but you can't end your movement there.
            If you do so, you are ejected to the nearest unoccupied space, this effect ends, and you are stunned until the end of your next turn.
        \end{itemize}
    \paragraph{Kickstarter (Bottle)} % Awaken
        This color-changing thick liquid is said to contain the very essence of the tides.
        Any Huge or smaller beast with an intelligence of 3 or less can receive the effect of this potion.
        The target gains an Intelligence of 10.

        The awakened beast is charmed by you for 30 days or until you or your companions do anything harmful to it.
        When the charmed condition ends, the awakened creature chooses whether to remain friendly to you, based on how you treated it while it was charmed.

        After this period passes, the creature needs a qualar to remain sentient, and it suffers Dementia effects normally (See page \pageref{ssec::dementia}).
    \paragraph{Laughing Brew} % Tasha's Hideous Laughter
        This fuzzy and bubbly drink seems to vibrate with astounding speed.
        After drinking it, the creature perceives everything as hilariously funny and falls into fits of laughter.
        The creature must succeed on a DC 10 Wisdom saving throw or fall prone, becoming incapacitated and unable to stand up for up to an hour.
        A creature with an Intelligence score of 4 or less isn't affected.

        A creature affected by this brew makes another DC 10 Wisdom saving throw every 10 minutes and when it takes damage.
        The target has advantage on the saving throw if it's triggered by damage.
        On a success, the effect ends.
    \paragraph{Madman Wine} % Crown of Madness
        This wine looks normal, but has a somewhat odd flavor.
        A creature who drinks this must succeed on a DC 12 Wisdom saving throw or become charmed for a minute.
        While charmed, madness glows in the creature's eyes.

        The charmed creature must use its action before moving on each of its turns to make a melee attack against a creature other than itself within 1 meter of it chosen randomly.
        If no creatures are within reach, it moves towards the closest creature to attack it.

        The target can make a DC 12 Wisdom saving throw at the end of each of its turns.
        On a success, the effect ends.
    \paragraph{Mindblower} % Speak with Plants + Hallucinatory Terrain
        This brew is a clear liquid indistinguishable from water.
        Upon drinking it, you can communicate with any plant within 6 meters of you and issue it simple commands for 8 hours.
        You can question plants about events in the spell's area within the past day, gaining information about creatures that have passed, weather, and other circumstances.

        You can also turn difficult terrain caused by plant growth (such as thickets and undergrowth) into ordinary terrain that lasts for the duration.
        Or you can turn ordinary terrain where plants are present into difficult terrain that lasts for the duration, causing vines and branches to hinder pursuers, for example.

        Plants might be able to perform other tasks on your behalf, at the DM's discretion. The spell doesn't enable plants to uproot themselves and move about, but they can freely move branches, tendrils, and stalks.

        If a plant creature is in the area, you can communicate with it as if you shared a common language, but you gain no magical ability to influence it.

        This spell can cause the plants created by the entangle spell to release a restrained creature.

        While under the effect of this brew, all terrain looks hard to traverse, and your movement speed is halved.
    \paragraph{Purphoros' Maker} % Flesh to Stone
        Upon drinking this unassuming wine, a creature begins to turn into stone.
        If the drinker's body is made of flesh, the creature must make a DC 18 Constitution saving throw.
        On a failed save, it is restrained as its flesh begins to harden.
        On a successful save, the creature isn't affected.

        A creature restrained by this effect must make another DC 18 Constitution saving throw at the end of each of its turns.
        If it successfully saves against this brew three times, the effect ends.
        If it fails its saves three times, it is turned to stone and subjected to the petrified condition until the effect is removed.
        The successes and failures don't need to be consecutive; keep track of both until the target collects three of a kind.

        If the creature is physically broken while petrified, it suffers from similar deformities if it reverts to its original state.
    \paragraph{Seer's Vodka} % True Seeing
        Reflections in this clear vodka seem out of place, as if the turbulent liquid displaced light in turbulent ways.
        This brew gives the drinker the ability to see things as they actually are.
        For 8 hours, the creature has truesight, and notices secrets hidden by magic out to a range of 24 meters.
    \paragraph{Soulcatcher} % Revivify
        As two actions, you can force a creature that has died within the last minute to drink this liquid.
        That creature returns to life with 1 hit point.
        This spell can't return to life a creature that has died of old age, nor can it restore any missing body parts.
    \paragraph{Speaker's Whiskey} % Zone of Truth on a target
        This strong whiskey seems to pick at your brain, forcing you to feel remorse about your misdeeds.
        Any creature who drinks this must make a DC 12 Charisma saving throw.
        On a failed save, a creature can't speak a deliberate lie for an hour.
        A skilled brewer knows whether each creature succeeds or fails on its saving throw.

        An affected creature is aware of the effect and can thus avoid answering questions to which it would normally respond with a lie.
        Such creatures can be evasive in its answers as long as it remains within the boundaries of the truth.
    \paragraph{Thassa's Mockery} % Water Breathing
        This transparent vodka has fish scales floating in it, which must be drinked with the brew to gain its effect.
        This drinks grants a creature the ability to breathe underwater for 8 hours.
        Affected creatures also retain their normal mode of respiration.
\newpage

    % !TEX root = ../main.tex
\subsection*{Mounts and Vehicles} \label{ssec::mountsandvehicles}
\subsubsection{Mounts}
    Mounting or dismounting a creature requires two actions.

    \begin{table*}[t]%
        \begin{DndTable}[width=\linewidth, header=Adventuring Gear]{Xlcccc}
            \textbf{Mount} & \textbf{Rarity} & \textbf{Speed} & \textbf{Carrying Capacity} & \textbf{Total Cost} & \textbf{Source} \\
            Beakdog       & Common   & 10 mt & 100 kg &  50 agnomas & --- \\
            Donkey        & Common   &  8 mt & 210 kg &   8 agnomas & PHB 157 \\
            Draft Horse   & Common   &  8 mt & 270 kg &  25 agnomas & PHB 157 \\
            Mastiff       & Common   &  8 mt & 100 kg &  25 agnomas & PHB 157 \\
            Mule          & Common   &  8 mt & 210 kg &   8 agnomas & PHB 157 \\
            Elephant      & Uncommon &  8 mt & 660 kg & 200 agnomas & PHB 157 \\
            Elk Bird      & Uncommon & 12 mt & 150 kg & 250 agnomas & --- \\
        \end{DndTable}
    \end{table*}

\subsubsection{Vehicles \& Harnesses}
    \begin{table*}[t]%
        \begin{DndTable}[width=\linewidth, header=Adventuring Gear]{Xlccccc}
            \textbf{Vehicle} & \textbf{Rarity} & \textbf{Mats.} & \textbf{Total Cost} & \textbf{Tools} & \textbf{Weight} & \textbf{Source} \\
            Cart            & Plain    & 13 &     15 agnomas & CAR & 100 kg & PHB 157 \\
            Military Saddle & Common   &  1 &     30 agnomas & LEA &  15 kg & PHB 157 \\
            Exotic Saddle   & Common   &  4 &     60 agnomas & LEA &  20 kg & PHB 157 \\
            Carriage        & Common   &  8 &    100 agnomas & CAR & 300 kg & PHB 157 \\
            Chariot         & Common   & 13 &    150 agnomas & CAR &  50 kg & PHB 157 \\
            Keelboat        & Uncommon & 58 &  3,000 agnomas & CAR & ---    & DMG 119 \\
            Galley          & Rare     & 58 & 30,000 agnomas & CAR & ---    & DMG 119 \\
            Longship        & Rare     & 18 & 10,000 agnomas & CAR & ---    & DMG 119 \\
        \end{DndTable}
    \end{table*}

    \paragraph{Exotic Saddle}
        An exotic saddle is required for riding any aquatic or flying mount.
    \paragraph{Galley}
        A galley has a speed of 6 km/h, and a carrying capacity of 150 tons.
        It requires a crew of 80 to run effectively, and it has an AC of 15, HP of 500, and a damage threshold of 20.
    \paragraph{Keelboat}
        A keelboat has a speed of 1.5 km/h, and a carrying capacity of half a ton plus 6 passengers at most.
        It requires a crew of 1, and it has an AC of 15, HP of 100, and a damage threshold of 10.

        Keelboats and rowboats are used on lakes and rivers.
        If going downstream, add the speed of the current (typically 5 kilometers per hour) to the speed of the vehicle.
        These vehicles can't be rowed against any significant current, but they can be pulled upstream by draft animals on the shores.
    \paragraph{Longship}
        A longship has a speed of 5 km/h, and a carrying capacity of 10 tons plus a cargo of 150 passengers.
        It requires a crew of 40, and it has an AC of 15, HP of 300, and a damage threshold of 15.
    \paragraph{Military Saddle}
        A military saddle braces the rider, helping you keep your seat on an active mount in battle.
        It gives you advantage on any check you make to remain mounted.

    % % !TEX root = ../main.tex
\section*{Adventuring Gear}
    Adventuring gear includes various miscellaneous items commonly carried and used by commoners and adventurers.
    % NOTE. Maybe separate containers from rest of gear inside the table for simplicity.

    \begin{table*}[b]%
        \begin{DndTable}[width=\linewidth, header=Adventuring Gear]{Xcccccc}
            \textbf{Item} & \textbf{Rarity} & \textbf{Mats.} & \textbf{Total Cost} & \textbf{Tools} & \textbf{Weight} & \textbf{Source} \\
            Basket                & Mundane  &  6  &    40 fobs    & WEA &  1 kg   & PHB 153 \\
            Blanket               & Mundane  &  8  &    50 fobs    & WEA &  1.5 kg & PHB 150 \\
            Bucket                & Mundane  &  1  &     5 fobs    & CAR &  1 kg   & PHB 153 \\
            Candle                & Mundane  & --- &     1 fob     & --- & ---     & PHB 151 \\
            Chalk                 & Mundane  & --- &     1 fob     & --- & ---     & PHB 150 \\
            Ink Pen               & Mundane  &  1  &     2 fobs    & TIN & ---     & PHB 150 \\
            Iron Spike            & Mundane  &  1  &    10 fobs    & SMI &  0.2 kg & PHB 150 \\
            Jug                   & Mundane  &  2  &    20 fobs    & GLA &  2 kg   & PHB 153 \\
            Ladder                & Mundane  &  1  &    10 fobs    & CAR & 12.5 kg & PHB 150 \\
            Lamp                  & Mundane  &  8  &    50 fobs    & TIN &  0.5 kg & PHB 152 \\
            Mess Kit              & Mundane  &  2  &    20 fobs    & SMI &  0.5 kg & PHB 152 \\
            Paper (one sheet)     & Mundane  & --- &    20 fobs    & --- & ---     & PHB 150 \\
            Parchment (one sheet) & Mundane  & --- &    10 fobs    & --- & ---     & PHB 150 \\
            Pitcher               & Mundane  &  1  &     2 fobs    & WOO &  2 kg   & PHB 153 \\
            Piton                 & Mundane  &  1  &     5 fobs    & SMI & ---     & PHB 150 \\
            Pole (3 meters)       & Mundane  &  1  &     5 fobs    & CAR &  3.5 kg & PHB 150 \\
            Pouch                 & Mundane  &  8  &    50 fobs    & LEA &  0.5 kg & PHB 153 \\
            Sack                  & Mundane  &  1  &     1 fob     & WEA & ---     & PHB 153 \\
            Sealing Wax           & Mundane  & --- &    50 fobs    & --- & ---     & PHB 150 \\
            Signal Whistle        & Mundane  &  1  &     5 fobs    & SMI & ---     & PHB 150 \\
            Soap                  & Mundane  & --- &     2 fobs    & --- & ---     & PHB 150 \\
            Tankard               & Mundane  &  1  &    15 fobs    & CAR &  0.5 kg & PHB 153 \\
            Tinderbox             & Mundane  &  8  &    50 fobs    & MAS &  0.5 kg & PHB 153 \\
            Torch                 & Mundane  &  1  &     1 fob     & CAR &  0.5 kg & PHB 153 \\
            Waterskin             & Mundane  &  2  &    20 fobs    & LEA &  2.5 kg & PHB 153 \\
            Whetstone             & Mundane  &  1  &     1 fob     & MAS &  0.5 kg & PHB 150 \\
            Abacus                & Plain    &  1  &     2 agnomas & CAR &  1 kg   & PHB 150 \\
            Backpack              & Plain    &  1  &     2 agnomas & LEA &  2.5 kg & PHB 153 \\
            Ball Bearings (1,000) & Plain    &  1  &     1 agnomas & SMI &  1 kg   & PHB 151 \\
            Barrel                & Plain     & 3 &      5 agnomas & CAR & 35 kg   & PHB 153 \\
            Bedroll               & Plain    &  1  &     1 agnoma  & WEA &  3.5 kg & PHB 150 \\
            Bell                  & Plain    &  1  &     1 agnoma  & SMI & ---     & PHB 150 \\
            Block and Tackle      & Plain    &  1  &     1 agnoma  & TIN &  2.5 kg & PHB 151 \\
            Bottle                & Plain     & 1 &      3 agnomas & GLA &  1 kg   & PHB 153 \\
            Bullseye Lantern      & Plain    &  8  &    10 agnomas & TIN &  1 kg   & PHB 152 \\
            Caltrops (20)         & Plain    &  1  &     1 agnoma  & SMI &  1 kg   & PHB 151 \\
            Chain (2 mt)          & Plain    &  3  &     5 agnomas & SMI &  5 kg   & PHB 151 \\
            Chest                 & Plain    &  3  &     5 agnomas & CAR & 12.5 kg & PHB 153 \\
            Crowbar               & Plain    &  1  &     2 agnomas & SMI &  2.5 kg & PHB 151
        \end{DndTable}
    \end{table*}
    \begin{table*}[b]%
        \begin{DndTable}[width=\linewidth, header=Adventuring Gear (cont.)]{Xcccccc}
            Cup                   & Plain     & 1 &       3 agnomas & GLA & ---     & --- \\
            Fishing Tackle        & Plain    &  1  &     1 agnoma  & TIN &  2 kg   & PHB 151 \\
            Full Keg              & Plain     & 2 &       4 agnomas & CAR & 15 kg   & --- \\
            Grappling Hook        & Plain    &  1  &     2 agnomas & SMI &  2 kg   & PHB 150 \\
            Hammer                & Plain    &  1  &     1 agnoma  & SMI &  1.5 kg & PHB 150 \\
            Hempen Rope           & Plain    & --- &     1 agnoma  & --- &  5 kg   & PHB 153 \\
            Hooded Lantern        & Plain    &  3  &     5 agnomas & TIN &  1 kg   & PHB 152 \\
            Hunting Trap          & Plain    &  3  &     5 agnomas & TIN & 12.5 kg & PHB 152 \\
            Ink (0.025 lt.)       & Plain    & --- &    10 agnomas & --- & ---     & PHB 150 \\
            Iron Pot              & Plain    &  1  &     2 agnomas & SMI &  5 kg   & PHB 153 \\
            Lock                  & Plain    &  8  &    10 agnomas & TIN &  0.5 kg & PHB 152 \\
            Manacles              & Plain    &  1  &     2 agnomas & TIN &  3 kg   & PHB 152 \\
            Map or Scroll Case    & Plain    &  1  &     1 agnoma  & LEA &  0.5 kg & PHB 151 \\
            Merchant's Scale      & Plain    &  3  &     5 agnomas & TIN &  1.5 kg & PHB 153 \\
            Miner's Pick          & Plain    &  1  &     2 agnomas & SMI &  5 kg   & PHB 150 \\
            Portable Ram          & Plain    &  2  &     4 agnomas & CAR & 17.5 kg & PHB 153 \\
            Shovel                & Plain    &  1  &     2 agnomas & SMI &  2.5 kg & PHB 150 \\
            Silk Rope (10 mt.)    & Plain    &  8  &    10 agnomas & WEA &  2.5 kg & PHB 153 \\
            Sledgehammer          & Plain    &  1  &     2 agnomas & SMI &  5 kg   & PHB 150 \\
            Small Keg             & Plain     & 1 &       3 agnomas & CAR &  2 kg   & --- \\
            Mirror                & Plain    &  3  &     5 agnomas & GLA & ---     & PHB 150 \\
            Two-Person Tent       & Plain    &  1  &     2 agnomas & WEA & 10 kg   & PHB 153 \\
            Book                  & Common   &  1  &    25 agnomas & CAL &  2.5 kg & PHB 151 \\
            Hourglass             & Common   &  1  &    25 agnomas & GLA &  0.5 kg & PHB 150 \\
            Magnifying Glass      & Common   &  8  &   100 agnomas & GLA & ---     & PHB 152 \\
            Spyglass              & Uncommon & 16  & 1,000 agnomas & GLA &  0.5 kg & PHB 153
        \end{DndTable}
    \end{table*}

    \paragraph{Barrel}
        A barrel can hold 150 liters of liquid.
    \paragraph{Basket}
        A basket holds 60 liters or 20 kg of gear.
    \paragraph{Block and Tackle}
        A set of pulleys with a cable threaded through them and a hook to attach to objects, a block and tackle allows you to hoist up to four times the weight you can normally lift.
    \paragraph{Book}
        A book might contain poetry, historical accounts, information pertaining to a particular field of lore, diagrams and notes on gat contraptions, or just about anything else that can be represented using text or pictures.
    \paragraph{Bottle}
        A bottle holds 1 liter of liquid.
    \paragraph{Bucket}
        A bucket holds 12 liters.
    \paragraph{Bullseye Lantern}
        A bullseye lantern casts bright light in a 12-meter cone and dim light for an additional 12 meters.
        Once lit, it burns for 6 hours on a flask of oil.
    \paragraph{Caltrops}
        As two action, you can spread a single bag of caltrops to cover a 1-meter-square area.
        Any creature that enters the area must succeed on a DC 15 Dexterity saving throw or stop moving and take 1 piercing damage.
        Until the creature regains at least 1 hit point, its walking speed is reduced by 2 meters.
        A creature moving through the area at half speed doesn't need to make the saving throw.
    \paragraph{Candle}
        For 1 hour, a candle sheds bright light in a 1-meter radius and dim light for an additional meter.
    \paragraph{Chain}
        A chain has 10 hit points.
        It can be burst with a successful DC 20 Strength check.
    \paragraph{Chest}
        A chest holds 350 liters of gear.
    \paragraph{Crowbar}
        Using a crowbar grants advantage to Strength checks where the crowbar's leverage can be applied.
    \paragraph{Fishing Tackle}
        This kit includes a wooden rod, silken line, corkwood bobbers, steel hooks, lead sinkers, velvet lures, and narrow netting.
    \paragraph{Full Keg}
        A full keg can hold 60 liters of liquid.
    \paragraph{Hempen Rope}
        Rope, whether made of hemp or silk, has 2 hit points and can be burst with a DC 17 Strength check.
    \paragraph{Hooded Lantern}
        A hooded lantern casts bright light in a 6-meter radius and dim light for an additional 6 meters.
        Once lit, it burns for 6 hours on a flask of oil.
        Using two actions, you can lower the hood, reducing the light to dim light in a 1-meter radius.
    \paragraph{Hunting Trap}
        When you use two actions to set it, this trap forms a saw-toothed steel ring that snaps shut when a creature steps on a pressure plate in the center.
        The trap is affixed by a heavy chain to an immobile object, such as a tree or a spike driven into the ground.
        A creature that steps on the plate must succeed on a DC 13 Dexterity saving throw or take 1d4 piercing damage and stop moving.
        Thereafter, until the creature breaks free of the trap, its movement is limited by the length of the chain (typically less than a meter long).
        A creature can use two actions to make a DC 13 Strength check, freeing itself or another creature within its reach on a success.
        Each failed check deals 1 piercing damage to the trapped creature.
    \paragraph{Iron Pot}
        An iron pot holds 4 liters of liquid.
    \paragraph{Jug}
        A jug holds 1 liter of liquid.
    \paragraph{Lamp}
        A lamp casts bright light in a 3-meter radius and dim light for an additional 6 meters.
        Once lit, it burns for 6 hours on a flask of oil.
    \paragraph{Lock}
        A key is provided with the lock.
        Without the key, a creature proficient with thieves' tools can pick this lock with a successful DC 15 Dexterity check.
        Your DM may decide that better locks are available for higher prices.
    \paragraph{Magnifying Glass}
        This lens allows a closer look at small objects.
        It is also useful as a substitute for flint and steel when starting fires.
        Lighting a fire with a magnifying glass requires light as bright as sunlight to focus, tinder to ignite, and about 5 minutes for the fire to ignite.
        A magnifying glass grants advantage on any ability check made to appraise or inspect an item that is small or highly detailed.
    \paragraph{Manacles}
        These metal restraints can bind a Small or Medium creature.
        Escaping the manacles requires a successful DC 20 Dexterity check.
        Breaking them requires a successful DC 20 Strength check.
        Each set of manacles comes with one key.
        Without the key, a creature proficient with thieves' tools can pick the manacles' lock with a successful DC 15 Dexterity check.
        Manacles have 15 hit points.
    \paragraph{Map or Scroll Case}
        This cylindrical leather case can hold up to ten rolled-up sheets of paper or five rolled-up sheets of parchment.
    \paragraph{Mess Kit}
        This tin box contains a cup and simple cutlery.
        The box clamps together, and one side can be used as a cooking pan and the other as a plate or shallow bowl.
    \paragraph{Portable Ram}
        You can use a portable ram to break down doors.
        When doing so, you gain a +4 bonus on the Strength check.
        One other character can help you use the ram, giving you advantage on this check.
    \paragraph{Pouch}
        A leather pouch can hold up to 20 sling bullets or 50 blowgun needles, among other things.
        A pouch can hold up to 6 liters or 3 kilograms of gear.
    \paragraph{Sack}
        A sack can hold up to 30 liters or 15 kilograms of gear.
    \paragraph{Small Keg}
        A small, portable keg can hold 4 liters of liquid.
    \paragraph{Spyglass}
        Objects viewed through a spyglass are magnified to up to eight times their size.
    \paragraph{Tankard}
        A tankard holds half a liter of liquid.
    \paragraph{Tinderbox}
        This small container holds flint, fire steel, and tinder (usually dry cloth soaked in light oil) used to kindle a fire.
        Using it to light a torch---or anything else with abundant, exposed fuel---takes two actions.
        Lighting any other fire takes 1 minute.
    \paragraph{Torch}
        A torch burns for 1 hour, providing bright light in a 4-meter radius and dim light for an additional 4 meters.
        If you make a melee attack with a burning torch and hit, it deals 1 fire damage.
    \paragraph{Waterskin}
        A waterskin can hold up to 2 liters of liquid.
\newpage~\newpage

    % % !TEX root = ../main.tex
\subsection*{Tools \& Foci} \label{ssec::toolsandfoci}
\subsubsection{Artisan's Tools}
    \begin{table*}[t]%
        \begin{DndTable}[width=\linewidth, header=Artisan's Tools]{Xlccccc}
            \textbf{Toolset} & \textbf{Rarity} & \textbf{Mats.} & \textbf{Total Cost} & \textbf{Tools} & \textbf{Weight} & \textbf{Source} \\
            Alchemist's Supplies  &  &  & 50 agnomas & GLA + MAS & 4 kg   & PHB 154 \\
            Brewer's Supplies     &  &  & 20 agnomas & SMI + MAS & 4.5 kg & PHB 154 \\
            Calligrapher Supplies &  &  & 10 agnomas & TIN       & 2.5 kg & PHB 154 \\
            Carpenter's Tools     &  &  &  8 agnomas & CAR + SMI & 3 kg   & PHB 154 \\
            Cartographer's Tools  &  &  & 15 agnomas & SMI + TIN & 3 kg   & PHB 154 \\
            Cobbler's Tools       &  &  &  5 agnomas & SMI + WEA & 2.5 kg & PHB 154 \\
            Cook's Utensils       &  &  &  1 agnoma  & SMI       & 4 kg   & PHB 154 \\
        \end{DndTable}
    \end{table*}

    \paragraph{Alchemist's Supplies} \label{item::alchemistssupplies}
        This toolset consists of a moor's head, an alembic, a retort, a mortal and pestle, a stirring rod, and one or more flasks and vials.
    \paragraph{Brewer's Supplies}
        These supplies consist of a boil kettle, a mortar and pestle, a mash tun, and usually one or more containers for the ale and liquors produced.
    \paragraph{Calligrapher's Supplies}
        These tools include parchment paper, ink, quills, a ruler, and a paintbrush.
    \paragraph{Carpenter's Tools}
        A set of carpenter's tools include a ruler, a framing square, a bag of nails, one or more clamps, a chisel, a carving knife, a framing hammer, a level, and a saw.
    \paragraph{Cartographer's Tools}
        These tools include a compass, a ruler, a triangle, a pen, callipers, and a bunch of parchment.
    \paragraph{Cobbler's Tools}
        A good set of cobbler's tools include and awl, a closing block, a gouge, a hammer, a last, a lachet, a paring knife, and a bunch of leather, needle and thread.
    \paragraph{Cook's Utensils}
        Among other various utilities, this toolset contains a kettle, a frying pan, a pot, and a set of spoons, knives, bowls, and plates.

\subsubsection{Gaming Kits}
    \begin{table*}[t]%
        \begin{DndTable}[width=\linewidth, header=Kits]{Xlccccc}
            \textbf{Kit} & \textbf{Rarity} & \textbf{Mats.} & \textbf{Total Cost} & \textbf{Tools} & \textbf{Weight} & \textbf{Source} \\
            Chess Set        &  &  &  1 agnoma  & WOO & --- & PHB 154 \\
            Dice Set         &  &  & 10 fobs    & WOO & --- & PHB 154 \\
            Huathem Card Set &  &  & 10 agnomas & CAL & --- & --- \\
            Skull Card Set   &  &  &  3 agnomas & CAL & --- & --- \\
        \end{DndTable}
    \end{table*}

\subsubsection{Kits}
    % Kits are similar to tools in the sense that they require proficiency, but aren't dedicated to crafting shit.
    \begin{table*}[t]%
        \begin{DndTable}[width=\linewidth, header=Kits]{Xlccccc}
            \textbf{Kit} & \textbf{Rarity} & \textbf{Mats.} & \textbf{Total Cost} & \textbf{Tools} & \textbf{Weight} & \textbf{Source} \\
            Climber's Kit         &  &  & 25 agnomas & SMI + WEA & 6 kg   & PHB 151 \\
            Forgery Kit           &  &  & 15 agnomas & CAL + TIN & 2.5 kg & PHB 154 \\
        \end{DndTable}
    \end{table*}

    \paragraph{Climber's Kit}
        A climber's kit includes pitons, boot tips, gloves, and a harness.
        You can use the climber's kit as two actions to anchor yourself; when you do, you can't fall more than 5 meters from the point where you anchored yourself, and you can't climb more than 5 meters away from that point without undoing the anchor.

        You don't need proficiency with this kit to use it effectively.
    \paragraph{Forgery Kit}
        This small box contains a variety of papers and parchments, pens and inks, seals and sealing wax, gold and silver leaf, and other supplies necessary to create convincing forgeries of physical documents.
        A forgery kit is designed to duplicate documents and to make it easier to copy a person's seal or signature.

        You need proficiency with this kit to use it effectively.

\subsubsection{Musical Instruments}
    \begin{table*}[t]%
        \begin{DndTable}[width=\linewidth, header=Musical Instruments]{Xlccccc}
            \textbf{Instrument} & \textbf{Rarity} & \textbf{Mats.} & \textbf{Total Cost} & \textbf{Tools} & \textbf{Weight} & \textbf{Source} \\
            Bagpipes &  &  & 30 agnomas & LEA       & 3 kg   & PHB 154 \\
            Drum     &  &  &  6 agnomas & CAR + LEA & 1.5 kg & PHB 154 \\
            Dulcimer &  &  & 25 agnomas & CAR       & 5 kg   & PHB 154 \\
            Flute    &  &  &  2 agnomas & WOO       & 0.5 kg & PHB 154 \\
        \end{DndTable}
    \end{table*}

\subsubsection{Spellcasting Foci and Utilities}
    \begin{table*}[t]%
        \begin{DndTable}[width=\linewidth, header=Spellcasting Foci]{Xlccccc}
            \textbf{Item} & \textbf{Rarity} & \textbf{Mats.} & \textbf{Total Cost} & \textbf{Tools} & \textbf{Weight} & \textbf{Source} \\
            Component Pouch &  &  &  25 agnomas & LEA       & 1 kg & PHB 151 \\
            Spellbook       &  &  & 100 agnomas & CAL + LEA &
        \end{DndTable}
    \end{table*}

    \paragraph{Component Pouch}
        A component pouch is a small, watertight leather belt pouch that has compartments to hold all the material components and other special items you need to cast your spells, except for those components that have a specific cost (as indicated in a spell's description).
    \paragraph{Spellbook} \label{item::spellbook}
        \textbf{TODO}.
        % Store unlimited number of spells.
        % Change prepared spells after a short rest.

    % % !TEX root = ../main.tex
\subsection*{Trinkets and Qualars} \label{ssec::trinketsandqualars}
\textbf{TODO.}
\newpage~\newpage

    % % !TEX root = ../main.tex
\subsection*{Weapons} \label{ssec::weapons}
% SPECIAL WEAPONS:
% * Huge Krudzal weapons used to fight giants
% * Drer fire weapons
% * Sulia's large blast weapons (cannons and large handcannons)
% * Kaldrathal's more refined flintlock pistols
% * Mercury weapons from frostburn umans

% heavy shields grant +3 AC and half cover against ranged attacks, but impose a movement speed debuff:
% tower shield: +3 AC, -10 movement speed
% bulwark: +4 AC, -20 movement speed

% WEAPON CATEGORIES:
% Knives:
%   simple    : dagger (L) 1d4, throwing dagger (L) 1d4, kukri 1d6 (only when thrown, less range)
%   1-handed  : parrying dagger 1
% Straight Swords:
%   simple    : shortsword 1d6
%   1-handed  : broadsword (H) 1d8, backsword 1d8
%   versatile : bastard sword 1d8-1d10
%   2-handed  : longsword 1d10, claymore (H) 1d12, zweihander (H) 2d6, flamberge (H) 1d10, greatsword (H) 2d6
% Curved Swords:
%   simple    : sickle (L) 1d4, cutlass 1d6
%   1-handed  : scimitar (L) 1d6, sabre 1d8, khopesh 1d6
%   versatile : falchion (H) 1d8-1d10
%   2-handed  : katana 1d10, double-bladed scimitar 2d4
% Light Swords:
%   simple    : mail breaker 1d4
%   1-handed  : rapier 1d8, estoc/tuck 1d8, basket hilted sword 1d8
% Axes:
%   simple    : handaxe (L) 1d6
%   versatile : battleaxe 1d8-1d10, broad axe 1d8-1d10 (H)
%   2-handed  : bearded axe 1d10, greataxe (H) 1d12
% Hammer:
%   simple    : club (L) 1d4, light hammer (L) 1d4, mace 1d6
%   1-handed  : warhammer 1d8-1d10, boomerang 1d4
%   2-handed  : greatclub (H) 1d8, maul (H) 2d6, lucerne 1d10
% Pick:
%   simple    : kama (L) 1d6
%   1-handed  : war pick 1d8
%   2-handed  : horseman's pick 1d10
% Staves:
%   simple    : quarterstaff 1d6-1d8
% Spears:
%   simple    : javelin 1d6, trident 1d6-1d8
%   versatile : spear 1d8-1d10, lance 1d12
%   2-handed  : pike (H) 1d10
% Halberds:
%   2-handed  : glaive (H) 1d10, poleaxe (H) 1d10, lucerne (H) 1d0
% Flails:
%   1-handed  : nunchaku (L) 1d4, ball-and-chain 1d6
% Whip:
%   1-handed  : whip 1d4, bullwhip (1d4 slashing but 6m reach), qilinbian (1d6 slashing, finesse, reach - 6mt)
% Bow:
%   2-handed  : shortbow 1d6, composite bow 1d8, longbow (H) 1d8, yumi (H) 1d10 - disadvantage on less than 6 meters
% Crossbow:
%   1-handed  : hand crossbow (L) R1 1d6
%   2-handed  : light crossbow R1 1d8, repeating crossbow R3 1d8, heavy crossbow (H) R1 1d10
% Pistol:
%   1-handed  : flintlock R1 1d10, pistol (change name) R4 1d8
% Musket:
%   2-handed  : musket R1 1d12
% "Newer" Firearms:
%   1-handed  : palm pistol (L) R1 1d8, pepperbox R6 1d10
%   2-handed  : blunderbuss R1 2d8, bad news R1 2d12
% Fire-breathers (Drer) - misfiring damages you:
%   1-handed  : bomb 3d6
%   2-handed  : firelance 1d12, hand mortar 2d8, flamespewer 1d6*
% Giant-slayers (Krudzal) - no extra attack:
%   2-handed  : dreihander 2d12, greatmaul 4d6, greatchisel 2d8*, hand ballista 2d10
%       dreihander    - https://mortalshell.wiki.fextralife.com/Martyr's+Blade
%       greatmaul     - https://mortalshell.wiki.fextralife.com/Smoldering+Mace
%       greatchisel   - https://mortalshell.wiki.fextralife.com/Hammer+and+Chisel
%       hand ballista - https://mortalshell.wiki.fextralife.com/Ballistazooka - action to shoot, action to reload
% Trick weapons: take 3 or 4 from here https://bloodborne.wiki.fextralife.com/Weapons.
% Exotic:
%   1-handed  :
%   versatile :
%   2-handed  : swordspear* (H) 1d10
%   ranged    : blowgun 1, net -
% Special (no specific talents, generally used for flavor):
%   simple    : shortspear 1d6, dart 1d4, sling 1d4
% Footnotes:
% * swordspears are essentially fancy glaive, but require proficiency with spears and swords due to having a different combat style.

% Take firearms from https://www.dndbeyond.com/subclasses/gunslinger .

\subsection*{Weapon Properties} \label{ssec::weaponproperties}
    \subparagraph{Ammunition} You can use a weapon that has the ammunition property to make a ranged attack only if you have ammunition to fire from the weapon.
    Each time you attack with the weapon, you expend one piece of ammunition.
    Drawing the ammunition from a quiver, case, or other container is part of the attack.
    Loading a one-handed weapon requires a free hand.
    At the end of the battle, you can recover half your expended ammunition by taking a minute to search the battlefield.

    If you use a weapon that has the ammunition property to make a melee attack, you treat the weapon as an improvised weapon.
    A sling must be loaded to deal any damage when used in this way.

    \subparagraph{Explosive} Upon a hit, everything within 5 ft of the target must make a Dexterity saving throw (DC equal to 8 + your proficiency bonus + your Dexterity modifier) or suffer 1d8 fire damage.
    If the weapon misses, the ammunition fails to detonate, or bounces away harmlessly before doing so.

    % \subparagraph{Great} This weapon requires the Great Weapon Fighter feat to be used.

    \subparagraph{Heavy} Creatures that are Small or Tiny have disadvantage on attack rolls with heavy weapons.
    A heavy weapon's size and bulk make it too large for a Small or Tiny creature to use effectively.

    In addition, the Multiple Attack Penalty is increased to 7 for weapons with this property..

    \subparagraph{Light} TODO.

    In addition, the Multiple Attack Penalty is reduced to 3 for weapons with this property.

    \subparagraph{Loading} Because of the time required to load this weapon, you can fire only one piece of ammunition from it when you use an action, bonus action, or reaction to fire it, regardless of the number of attacks you can normally make.

    \subparagraph{Misfiring} Whenever you make an attack roll, and the dice roll is equal to or lower than the weapon’s Misfire score, the weapon misfires.
    The attack misses, and the weapon cannot be used again until you spend an action to try and repair it.
    To repair your weapon, you must make a successful Tinker’s Tools check (DC equal to 8 + misfire score).
    If your check fails, the weapon is broken and must be mended out of combat at a quarter of its cost.
    Creatures who use a weapon without being proficient increase its misfire score by 1.

    \subparagraph{Reloading} The weapon can be fired a number of times equal to its Reload score before you must spend 1 attack or 1 action to reload.
    You must have one free hand to reload the weapon.

    \subparagraph{Trick} This weapon requires the Flexible Fighter 1 feat to use.
    These weapons can be transformed into alternate forms and have different capabilities in this transformed state.
    The weapons have two values for damage and properties, one for their normal form, and one for the transformed one.

    \subparagraph{Two-Handed} This weapon requires two hands to use.
    This property is relevant only when you attack with the weapon, not when you simply hold it.


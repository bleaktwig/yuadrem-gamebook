% !TEX root = ../main.tex
\section{3-Action System} \label{sec::3actionsystem}
Instead of the movement, action, and bonus action system of D\&D 5e, this book uses Pathfinder 2e's 3-action system.
Simply put, you can use up to three Action Points (AP) in a turn.
Additionally, one free action and one reaction per round are available.

The number of AP used by an action is denoted by the $\circ$ symbol.
Reactions are denoted by the $\otimes$ symbol.

The standard actions and reactions available to you are:

\subsubsection{Attack $\circ$} \label{act::attack}
    \textit{Perform a melee or ranged attack with your weapon.}

    You can use more than one Attack action on your turn, but each additional attack after the first becomes less accurate.
    This is reflected by a multiple attack penalty that starts at -5 on the second attack, increases to -10 on the third, -15 on the fourth, etc.
    This penalty resets at the end of your turn.

    Some conditions give advantage on the attack: attacks against blinded, paralyzed, petrified, restrained, stunned, or unconscious targets; melee attacks against prone targets; attacks by invisible or hidden attackers.

    Some conditions give disadvantage on the attack: attacks against invisible or hidden targets; ranged attacks against prone targets; attacks by blinded, frightened, poisoned, or restrained attackers.
\subsubsection{Cast a Spell $\ast$} \label{act::castaspell}
    \textit{Cast a spell with a casting time of 1 or more actions.}

    The target of a spell must be within the spell's range.
    To target something, you must have a clear path to it, so it can't be behind total cover.

    Spells with material components do not consume the material unless explicitly stated.
    Unless the cost of a material is given, you can assume that the cost is negligible and the material is simply available in a component pouch.

    Some spells require you to maintain concentration in order to keep their magic active.
    If you lose concentration, such a spell ends.
    You lose concentration on a spell if you cast another spell that requires concentration or when you are incapacitated.
    Each time you take damage, you must make a Constitution saving throw to maintain your concentration.
    The DC equals 10 or half the damage you take, whichever number is higher.
\subsubsection{Climb onto a Bigger Creature $\circ\circ$}  \label{act::climbontoabiggercreature}
    \textit{Hop atop your target.}

    You can treat a suitably large opponent as terrain for the purpose of jumping onto its back or clinging to a limb.
    After making any ability checks necessary to get into position and onto the larger creature, you use an action to make a Strength (Athletics) or Dexterity (Acrobatics) check contested by the target's Dexterity (Acrobatics) check.
    If you win the contest, you successfully move into the target creature's space.
    You move with the target and have advantage on attack rolls against it.

    You can move around within the larger creature's space, treating it as difficult terrain.
    The larger creature's ability to attack you depends on your location, and is left to the DMs discretion.
    The larger creature can dislodge you as an action—knocking you off, scraping you against a wall, or grabbing and throwing you—by making a Strength (Athletics) check contested by the your Strength (Athletics) or Dexterity (Acrobatics) check.
    You choose which ability to use.

\pagebreak

\subsubsection{Disarm $\circ\circ$} \label{act::disarm}
    \textit{Force your opponent to unequip their weapon.}

    You can use an action to knock a weapon or another item from a target's grasp.
    You make an attack roll contested by the target's Strength (Athletics) check or Dexterity (Acrobatics) check.
    If you win the contest, the attack causes no damage or other ill effect, but the defender drops the item.

    You have disadvantage on the attack roll if your target is holding the item with two or more hands.
    The target has advantage on its ability check if it is larger than you, or disadvantage if it is smaller.
\subsubsection{Disengage $\circ\circ$} \label{act::disengage}
    \textit{Your movement doesn't provoke opportunity attacks for the rest of the turn.}
\subsubsection{Dodge $\circ\circ$} \label{act::dodge}
    \textit{Focus entirely on avoiding attacks.}

    Until the start of your next turn, any attack roll made against you has disadvantage if you can see the attacker, and you make Dexterity saving throws with advantage.

    You lose this benefit if you are incapacitated or if your speed drops to 0.
\subsubsection{Equip Shield $\circ\circ$} \label{act::equipshield}
    \textit{Equip or unequip a shield.}

    A shield takes two actions to equip or unequip.

    Armor takes several minutes to equip or unequip.
\subsubsection{Escape $\circ$} \label{act::escape}
    \textit{Escape a grapple.}

    To escape a grapple, you must succeed on a Strength (Athletics) or Dexterity (Acrobatics) check contested by the grappler's Strength (Athletics) check.

    Escaping other conditions that restrain you (such as manacles) may require a Dexterity or Strength check, as specified by the condition.
\subsubsection{Grapple $\circ$} \label{act::grapple}
    \textit{Attempt to grab a creature or wrestle with it.}

    The target of your grapple must be no more than one size larger than you, and it must be within your reach.

    Using at least one free hand, you try to seize the target by making a grapple check, a Strength (Athletics) check contested by the target's Strength (Athletics) or Dexterity (Acrobatics) check (the target chooses the ability to use).

    If you succeed, you subject the target to the grappled condition (its speed is set to 0).
\subsubsection{Help $\circ$} \label{act::help}
    \textit{Grant an ally within 1 meter advantage on an ability check or attack.}

    The target gains advantage on the next ability check it makes to perform the task you are helping with.
    Alternatively, the target gains advantage on the next attack roll against against a creature within 1 meter of you.

    The advantage lasts until the start of your next turn, and you can use this action only once per turn.
\subsubsection{Hide $\circ\circ$} \label{act::hide}
    \textit{Attempt to hide.}

    You can't hide from a creature that can see you.
    You must have total cover, be in a heavily obscured area, be invisible, or otherwise block the enemy's vision.

    If you make noise (such as shouting a warning or knocking over a vase), you give away your position.

    When you try to hide, make a Dexterity (Stealth) check and note the result.
    Until you are discovered or you stop hiding, that check's total is contested by the Wisdom (Perception) check of any creature that actively searches for signs of your presence.

    A creature notices you even if it isn't searching unless your Stealth check is higher than its Passive Perception.

    Out of combat, you may also use a Dexterity (Stealth) check for acts like concealing yourself from enemies, slinking past guards, slipping away without being noticed, or sneaking up on someone without being seen or heard.
\subsubsection{Identify a Spell $\circ/\otimes$} \label{act::identifyaspell}
    \textit{Attempt to understand a casted spell.}

    Sometimes you may want to identify a spell that someone else is casting or that was already cast.
    To do so, you can use your reaction to identify a spell as it's being cast, or use an action on your turn to identify a spell by its effect after it is cast.

    If you perceived the casting, the spell's effect, or both, you can make an Intelligence (Arcana) check with the reaction or action.
    The DC equals 10 + the spell's level.
    If the spell casted is from a school of magic in which you have competence, the check is made with advantage.
    Some spells aren't associated with any school, such as when a monster uses its Innate Spellcasting trait.

    This Intelligence (Arcana) check represents the fact that identifying a spell requires a quick mind and familiarity with the theory and practice of casting.
    This is true even for a character whose spellcasting ability is Wisdom or Charisma.
    Being able to cast spells doesn't by itself make you adept at deducing exactly what others are doing when they cast their spells.
\subsubsection{Improvise $\circ$} \label{act::improvise}
    \textit{Perform any action you can imagine.}

    When you describe an action not detailed elsewhere in the rules, the DM tells you whether that action is possible and what kind of roll you need to make, if any, to determine success or failure.
\subsubsection{Move $\circ$} \label{act::move}
    \textit{Move up to your movement speed.}

    You can interrupt the move action to take another action, continuing your move afterwards.

    If you have more than one speed, such as your walking speed and a flying speed, you can switch back and forth between your speeds during your move.
    Whenever you switch, subtract the distance you've already moved from the new speed.

    You can move through a nonhostile creature's space.
    You can move through a hostile creature's space only if the creature is at least two sizes larger or smaller than you.
    Another creature's space is difficult terrain for you.
    Whether a creature is a friend or an enemy, you can't willingly end your move in its space.

    Climbing, swimming, and crawling cost an additional 1.5 m per 1.5 m travelled, unless you have an specific climbing or swimming speed.

    This movement can include a high jump and/or a long jump, following the normal rules for each.
\subsubsection{Overrun $\circ$} \label{act::overrun}
    \textit{Force your way through a creature's space.}

    When you try to move through a hostile creature's space, you try to force your way through by overrunning the hostile creature.
    As an action, you make a Strength (Athletics) check contested by the hostile creature's Strength (Athletics) check.
    You have advantage on this check if you are larger than the hostile creature, or disadvantage if you are smaller.
    If you win the contest, you can move through the hostile creature's space once this turn.
\subsubsection{Ready $\ast$} \label{act::ready}
    \textit{Choose a trigger and a response reaction.}

    First, you decide what perceivable circumstance will trigger your reaction.

    Then, you choose the action you will take in response to that trigger, or you choose to move up to your speed in response to it.

    When the trigger occurs, you can either take your reaction right after the trigger finishes or ignore the trigger.

    When you ready a spell, you cast it as normal but hold its energy, which you release with your reaction when the trigger occurs.
    Holding onto the spell's magic requires concentration.

    The cost of this action is equal to the cost of the readied action, so 1 for the attack action, 2 for the Search action, etc.
\subsubsection{Reload $\circ$} \label{act::reload}
    \textit{Reload a crossbow or firearm.}

    Some weapons require more than one action to reload.
    This is specified in the weapon itself.
\subsubsection{Search $\circ\circ$} \label{act::search}
    \textit{Devote your attention to finding something.}

    Depending on the nature of your search, the DM might have you make a Wisdom (Perception) check or an Intelligence (Investigation) check.
\subsubsection{Shove $\circ$} \label{act::shove}
    \textit{Shove a creature, either to knock it prone or push it to the side.}

    The target of your shove must be no more than one size larger than you, and it must be within your reach.

    You make a Strength (Athletics) check contested by the target's Strength (Athletics) or Dexterity (Acrobatics) check (the target chooses the ability to use).

    If you win the contest, you either knock the target prone or push it 1 meter to the side (your choice).
\subsubsection{Stabilize a Creature $\circ\circ$} \label{act::stabilizeacreature}
    \textit{Stop a dying creature from needing to make death saving throws.}

    Make a Wisdom (Medicine) check with DC 10.

    On a success, the creature is stable and no longer needs to make death saving throws.

    A stable creature regains 1 hit point after 1d4 hours.
\subsubsection{Stand Up $\circ$} \label{act::standup}
    \textit{End the prone condition on yourself by standing up.}
\subsubsection{Tumble $\circ$} \label{act::tumble}
    \textit{Weave past a hostile creature.}

    You can try to tumble through a hostile creature's space, ducking and weaving past the opponent.
    As an action, you make a Dexterity (Acrobatics) check contested by the hostile creature's Dexterity (Acrobatics) check.
    If you win the contest, you can move through the hostile creature's space once this turn.
\subsubsection{Use Object $\circ\circ$} \label{act::useobject}
    \textit{Interact with a second object or use special object abilities.}

    You can interact with one object for free during your turn (such as drawing a weapon or opening a door).
    If you want to interact with a second object, use this action.

    When an object requires your action for its use, you also take this action.
\subsubsection{Cast a Spell $\otimes$} \label{act::castaspellreact}
    \textit{Cast a spell with a casting time of 1 reaction.}

    Trigger: specified by the spell.

    For further details, see the Cast a spell action.
\subsubsection{Readied Action $\otimes$} \label{act::readiedaction}
    \textit{Execute the reaction specified by your Ready action.}

    Trigger: specified by your Ready action.
\subsubsection{Opportunity Attack $\otimes$} \label{act::opportunityattack}
    \textit{You can rarely move heedlessly past your foes without putting yourself in danger.}

    Trigger: enemy creature leaves your reach.

    Make one melee attack against the provoking creature.

    The attack interrupts the provoking creature's movement, occurring right before the creature leaves your reach.

    Creatures don't provoke an opportunity attack when they teleport or when someone or something moves them without using their movement, action, or reaction.

    Additional conditions for opportunity attacks are included as an optional rule in page \pageref{rule::opportunityattacks}.

\subsection*{Learnable Actions} \label{ssec::learnableableactions}
    Some actions are not available by default, but instead require learning a feat to be obtained.
    These are listed in this subsection, along with links to the feats required to learn them.
    The \textbf{Adaptable Fighter} (page \pageref{feat::adaptablefighter}) and \textbf{Dynamic Fighter} (page \pageref{feat::dynamicfighter}) feats allow you to learn an action of your choice.

    \subsubsection{Aim $\circ$} \label{act::aim}
        You give yourself advantage on your next attack roll, which can be made until the end of your next turn.

        This action can be obtained from the \textbf{Deadly Precision} (page \pageref{feat::deadlyprecision}) and the \textbf{Eye for Weakness} (page \pageref{feat::eyeforweakness}) feats.
    \subsubsection{Block $\otimes$} \label{act::block}
        You can use your reaction to prepare yourself against one melee or ranged attack.
        You increase your AC by 1d6 against the attack.
        If the attack does hit you, you take half damage from it (rounded down).

        This reaction can be obtained from the \textbf{Stalwart Stance} feat (page \pageref{feat::stalwartstance}).
    \subsubsection{Distracting Strike $\circ\circ$} \label{act::distractingstrike}
        You use your action to distract a creature with an attack, giving your allies an opening.
        You roll damage normally, and the next attack roll against the target by an creature other than you has advantage if the attack is made before the start of your next turn.

        This action can be obtained from the \textbf{Flourished Duelist} (page \pageref{feat::flourishedduelist}) feat.
    \subsubsection{Quick Draw $\otimes$} \label{act::quickdraw}
        You can make an attack of opportunity with an equipped thrown or ranged weapon when a creature moves out of a 9-meter radius around you.

        This reaction can be obtained from the \textbf{Quick Draw} (page \pageref{feat::quickdraw}) and the \textbf{Quick Shot} (page \pageref{feat::quickshot}) feats.
    \subsubsection{Mark $\circ\circ$} \label{act::mark}
        You choose a creature you can see within 18 meters and mark it as your quarry.
        For up to one hour, you deal an extra 1d6 damage to the target whenever you hit it with a weapon attack, and you have advantage on any Intelligence (Investigation), Wisdom (Perception), and Wisdom (Survival) checks you make to find it.
        You can use this action a number of times equal to your Wisdom modifier, and you restore all expended uses on a short rest.

        This action can be obtained from the \textbf{Tracker} (page \pageref{feat::tracker}) and the \textbf{Mark} (page \pageref{feat::mark}) feats.
    \subsubsection{Parry $\otimes$} \label{act::parry}
        When another creature attacks you with a melee attack, you can use your reaction to increase your AC by 1d6 + your Dexterity modifier against the attack.

        This reaction can be obtained from the \textbf{Parrying Stance} (page \pageref{feat::parryingstance}), the \textbf{Swift Deflection} (page \pageref{feat::swiftdeflection}), and the \textbf{Sword Breaker} (page \pageref{feat::swordbreaker}) feats.
    \subsubsection{Push $\circ$} \label{act::push}
        You try to push a target back.
        The target must be no more than one size larger than you and be within 1 meter of you.
        You make a Strength (Athletics) check contested by the target's Strength (Athletics) or Dexterity (Acrobatics) check (the target chooses the ability to use).
        If you win, you push the target up to 3 meters away from you.
        If you win by 10 or more, the target is also knocked prone.

        This action can be obtained from the \textbf{Forest Defender} (page \pageref{feat::forestdefender}) and \textbf{Stalwart Shield} (page \pageref{feat::stalwartshield}) feats.
    \subsubsection{Reckless Attack $\circ$} \label{act::recklessattack}
        You make a special melee weapon attack against a creature using a weapon with the heavy property.
        Apply a -5 penalty to the attack roll.
        If the attack hits, you add +10 to the attack's damage.

        This action can be obtained from the \textbf{Giant Slayer} (page \pageref{feat::giantslayer}), the \textbf{Heavy Hitter} (page \pageref{feat::heavyhitter}), and the \textbf{Reckless Attack} (page \pageref{feat::recklessattack}) feats.
    \subsubsection{Reckless Shot $\circ$} \label{act::recklessshot}
        You make a special ranged weapon attack against a creature with a ranged weapon.
        Apply a -5 penalty to the attack roll.
        If the attack hits, you add +10 to the attack's damage.

        This action can be obtained from the \textbf{Reckless Shot} (page \pageref{feat::recklessshot}) feat.
    \subsubsection{Riposte $\otimes$} \label{act::riposte}
        When a creature misses you with a melee attack, you can use your reaction to make a melee weapon attack against the creature.

        This action can be obtained from the \textbf{Appropiate Response} (page \pageref{feat::appropiateresponse}) feat.
    \subsubsection{Sneak Attack $\circ\circ$} \label{act::sneakattack}
        You know how to strike subtly and exploit a foe's distraction.
        You do an attack that deal an extra 2d6 damage to one creature you hit with an attack if you have advantage on the attack roll.
        The attack must use a finesse or a ranged weapon.

        You don't need advantage on the attack roll if another enemy of the target is within 5 feet of it, that enemy isn't incapacitated, and you don't have disadvantage on the attack roll.
        You can use this action only once per turn.

        You can take this action various times.
        Each time after the first increases the number of d6s rolled by one.

        This action can be learned and improved from the \textbf{Hidden Striker} (page \pageref{feat::hiddenstriker}), the \textbf{Purposeful Strike} (page \pageref{feat::purposefulstrike}) and the \textbf{Stab-a-Lung} (page \pageref{feat::stabalung}) feats.
    \subsubsection{Steal $\circ\circ$} \label{act::steal}
        As an action, you can make a Dexterity (Sleight of Hand) check contested by a creature's Wisdom (Perception) to plant something on someone else, conceal an object on a creature, lift a purse, or take something from a pocket.
        You can do this in the middle of an encounter.

        This action can be learned from the \textbf{Poacher} (page \pageref{feat::poacher}) feat.
    % === UNUSED ============================================================ %
    % \subsubsection{Careless Deflect} \label{tec::carelessdeflect}
    %     When a creature misses you with a melee attack roll, you can use your reaction to cause that attack to hit one creature of your choice, other than the attacker, that you can see within 1 meter of you.
    %
    % \subsubsection{Commander's Strike} \label{tec::commandersstrike}
    %     When you take the Attack action on your turn, you can choose to attack only once and use a bonus action to direct one of your companions to strike.
    %     When you do so, choose a friendly creature who can see or hear you.
    %     That creature can immediately use its reaction to make one weapon attack, adding a d8 to the attack's damage roll.
    %
    % \subsubsection{Disarming Attack} \label{tec::disarmingattack}
    %     You use your action to attempt to disarm the target with an attack, forcing it to drop one item of your choice that it's holding.
    %     The target must make a Strength saving throw with a DC of 8 + your proficiency bonus + your Strength modifier.
    %     On a failed save, it drops the object you choose.
    %     The object lands at its feet.
    %
    % \subsubsection{Evasive Footwork} \label{tec::evasivefootwork}
    %     When you move, you can use a bonus action to add a d6 to your AC until you stop moving.
    %
    % \subsubsection{Goading Attack} \label{tec::goadingattack}
    %     You use your action to attack a creature and goad it into attacking you.
    %     The target must make a Wisdom saving throw with a DC of 8 + your proficiency bonus + your Charisma modifier.
    %     On a failed save, the target has disadvantage on all attack rolls against targets other than you until the end of your next turn.
    %
    % \subsubsection{Injure} \label{tec::injure}
    %     As an action, you can attack with the sole purpose of injuring a creature.
    %     Roll your weapon attack normally.
    %     On a hit, you deal only half damage, but the creature rolls on the minor injury chart.
    %     If the attack is a critical hit, the creature rolls on the major injury chart.
    %
    % \subsubsection{Maneuvering Attack} \label{tec::maneuveringattack}
    %     When you hit a creature with a melee attack, you can use your bonus action to maneuver one of your comrades into a more advantageous position.
    %     You choose a friendly creature who can see or hear you.
    %     That creature can use its reaction to move up to half its speed without provoking opportunity attacks from the target of your attack.
    %
    % \subsubsection{Menacing Attack} \label{tec::menacingattack}
    %     When you hit a creature with a weapon attack, you can use your bonus action to attempt to frighten the target.
    %     The target must make a Wisdom saving throw of a DC of 8 + your proficiency bonus + your Charisma modifier.
    %     On a failed save, it is frightened of you until the end of your next turn.
    %
    % \subsubsection{Precise Attack} \label{tec::preciseattack}
    %     When you make a weapon attack roll against a creature, you can use your bonus action to add a d8 to the attack roll.
    %     You can use this technique before or after making the attack roll, but before any effects of the attack are applied.
    %
    % \subsubsection{Rapid Repair} \label{tec::rapidrepair}
    %     You can attempt to repair a misfired (but not broken) firearm as a bonus action.
    %
    % \subsubsection{Violent Shot} \label{tec::violentshot}
    %     When you make a firearm attack against a creature, you can tune your weapon to enhance the volatility of the attack.
    %     The attack gains a +2 to the firearm's misfire score.
    %     If the attack hits, you can roll one additional weapon damage die.
    % \subsubsection{Predetermined Fate} \label{mtec::predeterminedfate}
    %     Whenever you make a saving throw and fail, you can reroll it and take the second result.
    %
    % \subsubsection{Quivering Palm} \label{mtec::quiveringpalm}
    %     This is the ultimate technique of the SCHOOL school, and can only be learned after you learn all other martial arts techniques.
    %     When you hit a creature with an unarmed strike, you can use the quivering palm technique.
    %     The creature must make a Constitution saving throw or take 10d10 bludgeoning damage, taking half damage on a success.
    %     You can use this technique only once per short rest.

\newpage~\newpage

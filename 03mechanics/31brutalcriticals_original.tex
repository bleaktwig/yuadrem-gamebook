% !TEX root = ../main.tex
\subsection*{Brutal Criticals} \label{ssec::brutalcriticals}

Presented here are optional rules that expand the scope of critical hits.
Introducing a series of critical hit charts that vary based on the damage type of the attack, this module adds an additional element of uncertainty, suspense, and surprise to combat.

\subsubsection{Revisions to Critical Hits}
    \begin{itemize}
        \item When you land a critical hit on a creature, roll the damage dice twice and add them together as normal.
        Then, roll a d20 and use the corresponding result on the critical hit chart determined by the damage type of your attack.
        \item When you score a critical hit with an attack that does two or more types of damage, choose one of those damage types and roll on that critical chart.
        \item Traits and feats like Savage Attacks and Brutal Critical continue to work as written.
    \end{itemize}

\subsubsection{Injuries \& Insanity}
    Critical hits can cause minor and major injuries.
    When a creature suffers a minor injury it can be healed by someone competent with a healer's kit over the course of a long rest.

    Major injuries and Insanities will not heal on their own, and count as a new Flaw (see page \pageref{ssec::flaws}) when they are applied.
    The text of certain injuries will specify other ways this injuries might resolve.

\subsubsection{Damage Charts}
    \begin{DndTable}[width=\linewidth, header=Bludgeoning]{cX}
        \textbf{d20} & \textbf{Effect} \\
        1     & Roll damage as normal. \\
        2-3   & Roll damage as normal and the next attack against the creature has advantage. \\
        4-6   & Do not roll your damage dice, instead deal the maximum result possible with those dice. \\
        7-8   & Do not roll your damage dice, instead deal the maximum result possible with those dice and push the creature up to 15 feet in any direction. \\
        9-11  & Roll your damage dice twice and add them together. \\
        12-13 & Roll damage dice twice and add them together and the creature is knocked prone. \\
        14-16 & Roll damage dice twice and add them together, push the creature up to 15 feet away, and the creature is knocked prone. \\
        17-18 & Deal the maximum amount of damage from your normal damage dice then roll your damage dice and add the result. Then roll on the Minor Injury chart. If the creature is wearing heavy armor roll on the Major Injury chart instead. \\
        19    & Deal the twice maximum result of your damage dice and roll on the major injury chart. \\
        20    & Deal the maximum result of your damage dice twice, the creature is stunned until the end of your next turn, and roll on the major injury chart.
    \end{DndTable}

    \begin{DndTable}[width=\linewidth, header=Piercing]{cX}
        \textbf{d20} & \textbf{Effect} \\
        1     & Roll damage as normal. \\
        2-3   & Roll damage dice twice and use the higher result. \\
        4-6   & Do not roll your damage dice, instead deal the maximum result possible with those dice. \\
        7-8   & Roll your damage dice twice and add them together. \\
        9-11  & Roll your damage dice twice and add them together. \\
        12-13 & Roll twice as many damage dice as normal. Then roll twice on the minor injury chart and use the lower result. \\
        14-16 & Roll your damage dice twice and add them together and roll on the minor injury chart. \\
        17-18 & Roll your damage dice twice and add them together and roll on the major injury chart. \\
        19    & Deal twice the maximum result of your damage dice and roll on the major injury chart. \\
        20    & Deal the maximum result of your damage dice twice, roll on the minor injury chart, and roll on the major injury chart.
    \end{DndTable}

    \begin{DndTable}[width=\linewidth, header=Slashing]{cX}
        \textbf{d20} & \textbf{Effect} \\
        1     & Roll damage as normal. \\
        2-3   & Roll damage as normal and the creature loses 1d6 hit points at the start of its next turn. \\
        4-6   & Do not roll your damage dice, instead deal the maximum result possible with those dice. \\
        7-8   & Roll your damage dice as normal and the creature is bleeding. For the next minute the creature loses 1d4 damage at the start of each of its turns until it uses an action to staunch this wound. \\
        9-11  & Roll your damage dice twice and add them together. \\
        12-13 & Roll your damage dice twice and add them together and the creature is bleeding. For the next minute the creature loses 1d8 hit points at the start of each of its turns until it uses an action to staunch this wound. \\
        14-16 & Roll your damage dice twice and add them together and the creature is bleeding. For the next minute the creature loses 1d12 hit points at the start of each of its turns until it uses an action to staunch this wound. \\
        17-18 & Deal the maximum amount of damage from your normal damage dice then roll your damage dice and add the result. Then roll on the Minor Injury chart. If the creature is wearing light or no armor roll on the Major Injury chart instead. \\
        19    & Deal twice the maximum result of your damage dice and roll on the major injury chart. \\
        20    & Deal the maximum result of your damage dice twice, roll on the major injury chart, and the creature is bleeding. For the next minute the creature loses 2d8 hit points at the start of each of its turns until it uses an action to staunch this wound.
    \end{DndTable}

    \begin{DndTable}[width=\linewidth, header=Acid]{cX}
        \textbf{d20} & \textbf{Effect} \\
        1     & Roll damage as normal. \\
        2-3   & Roll damage as normal and the creature is scarred. While scarred the creature has disadvantage on all Charisma ability checks except Charisma (Intimidation). Being scarred can be healed in the same way as a minor injury. \\
        4-6   & Do not roll your damage dice, instead deal the maximum result possible with those dice. \\
        7-8   & Roll your damage as normal and the creature is disfigured. While disfigured the creature has disadvantage on all Charisma ability checks except Charisma (Intimidation). Being disfigured can be healed in the same way as a major injury. \\
        9-11  & Roll your damage dice twice and add them together. \\
        12-13 & Roll twice as many damage dice as normal and if the creature is wearing armor its AC modifier is reduced by 1 until it can be repaired (for 1/4th the price of new armor of the same type). If the creature is not wearing armor, roll on the minor injury chart. \\
        14-16 & Roll twice as many damage dice as normal and roll on the minor injury chart. Additionally, the creature is disfigured. While disfigured the creature has disadvantage on all Charisma ability checks except Charisma (Intimidation). Being disfigured can be healed in the same way as a major injury. \\
        17-18 & Deal the maximum amount of damage from your normal damage dice then roll your damage dice and add that result. If the creature is wearing armor, roll on the minor injury chart and its AC modifier is reduced by 2 until it can be repaired (for half the price of new armor of the same type). If the creature is not wearing armor, roll on the major injury chart. \\
        19    & Deal twice the maximum result of your damage dice and roll on the major injury chart. \\
        20    & Deal twice the maximum result of your damage dice. If the creature is wearing armor, the armor is destroyed and roll on the major injury chart. If the creature is not wearing armor, roll on the major injury chart and the creature is disfigured. While disfigured the creature has disadvantage on all Charisma ability checks except Charisma (Intimidation). Being disfigured can be healed in the same way as a major injury.
    \end{DndTable}

    \begin{DndTable}[width=\linewidth, header=Cold]{cX}
        \textbf{d20} & \textbf{Effect} \\
        1     & Roll damage as normal. \\
        2-3   & Roll damage as normal and the creature may only move half its movement speed on its next turn. \\
        4-6   & Do not roll your damage dice, instead deal the maximum result possible with those dice. \\
        7-8   & Roll your damage as normal and the creature’s movement speed is 0 until the end of its next turn. \\
        9-11  & Roll twice as many damage dice as normal. \\
        12-13 & Roll twice as many damage dice as normal and the creature is paralyzed until the end of its next turn. \\
        14-16 & Roll twice as many damage dice as normal and the creature is paralyzed until the end of its next turn. If the creature takes damage before the end of its next turn, roll on the minor injury chart.
        17-18 & Deal the maximum amount of damage from your normal damage dice then roll your damage dice and add that result. The creature is paralyzed until the end of its next turn and rolls on the minor injury chart. \\
        19    & Deal twice the maximum result of your damage dice and roll on the major injury chart. \\
        20    & Deal twice the maximum result of your damage dice, roll on the major injury chart, and the creature is paralyzed for the next minute. The creature may attempt a saving throw at the end of each of its turns (DC 16) to end this effect. If it fails this saving throw three times it is frozen solid and becomes a petrified but frozen solid in a block of ice rather than turned to stone.
    \end{DndTable}

    \begin{DndTable}[width=\linewidth, header=Fire]{cX}
        \textbf{d20} & \textbf{Effect} \\
        1     & Roll damage as normal. \\
        2-3   & Roll damage as normal and attack rolls for attacks that deal fire damage have advantage when made against the creature until the end of its next turn. \\
        4-6   & Do not roll your damage dice, instead deal the maximum result possible with those dice. \\
        7-8   & Roll your damage as normal and the creature is on fire. While the creature is on fire it takes 2d4 fire damage at the start of each of its turns. The creature can end this condition by dropping prone and using 5 feet of movement to roll on the ground. \\
        9-11  & Roll twice as many damage dice as normal. \\
        12-13 & Roll twice as many damage dice as normal and the creature is on fire. While the creature is on fire it takes 2d6 fire damage at the start of each of its turns. The creature can end this condition by dropping prone and using 5 feet of movement to roll on the ground. \\
        14-16 & Roll twice as many damage dice as normal and the creature is charred. If the creature has resistance to fire, it loses that resistance. If the creature does not have resistance to fire, it gains vulnerability to fire. Both of these effects can be ended the same as a minor injury. \\
        17-18 & Deal the maximum amount of damage from your normal damage dice then roll your damage dice and add that result then roll on the minor injury chart. Additionally, the creature is on fire. While the creature is on fire it takes 2d6 fire damage at the start of each of its turns. The creature can end this condition by dropping prone and using 5 feet of movement to roll on the ground. \\
        19    & Deal twice the maximum result of your damage dice and roll on the major injury chart. \\
        20    & Deal twice the maximum result of your damage dice and roll on the major injury chart. Additionally, the creature is on fire. While the creature is on fire it takes 2d8 fire damage at the start of each of its turns.
        The creature can end this condition by dropping prone and using 5 feet of movement to roll on the ground.
    \end{DndTable}

    \begin{DndTable}[width=\linewidth, header=Force]{cX}
        \textbf{d20} & \textbf{Effect} \\
        1     & Roll damage as normal. \\
        2-3   & Roll damage as normal and the creature has disadvantage on saving throws against spells until the end of its next turn. \\
        4-6   & Do not roll your damage dice, instead deal the maximum result possible with those dice. \\
        7-8   & Roll your damage as normal and spell attack rolls against the creature have advantage until the end of its next turn. \\
        9-11  & Roll twice as many damage dice as normal. \\
        12-13 & Roll twice as many damage dice as normal and the creature is spellbound until the end of its next turn.
        While spellbound it makes saving throws against spells with disadvantage and spell attack rolls against it have advantage. \\
        14-16 & Roll twice as many damage dice as normal and the creature is spellbound for the next minute.
        While spellbound it makes saving throws against spells with disadvantage and spell attack rolls against it have advantage.
        At the end of each of the creature’s turns it can make an Intelligence saving throw (DC 14) to end this effect. \\
        17-18 & Deal the maximum amount of damage from your normal damage dice then roll your damage dice and add that result then roll on the minor injury chart.
        Additionally, the creature is spellbound for the next minute. While spellbound it makes saving throws against spells with disadvantage and spell attack rolls against it have advantage.
        At the end of each of the creature’s turns it can make an Intelligence saving throw (DC 16) to end this effect. \\
        19    & Deal twice the maximum result of your damage dice and roll on the major injury chart. \\
        20    & Deal twice the maximum result of your damage dice and roll on the major injury chart.
        Additionally, the creature is spellbound for the next minute.
        While spellbound it makes saving throws against spells with disadvantage and spell attack rolls against it have advantage.
        At the end of each of the creature’s turns it can make an Intelligence saving throw (DC 18) to end this effect.
    \end{DndTable}

    \begin{DndTable}[width=\linewidth, header=Lightning]{cX}
        \textbf{d20} & \textbf{Effect} \\
        1     & Roll damage as normal. \\
        2-3   & Roll damage as normal and the creature cannot use reactions until the end of its next turn. \\
        4-6   & Do not roll your damage dice, instead deal the maximum result possible with those dice. \\
        7-8   & Roll your damage as normal and you may choose one other creature within 15 ft. of the victim.
        That creature must succeed on a Dexterity saving throw (DC 14) or take half as much damage. \\
        9-11  & Roll twice as many damage dice as normal. \\
        12-13 & Roll twice as many damage dice as normal and you may choose one other creature within 15 ft. of the victim.
        That creature must succeed on a Dexterity saving throw (DC 18) or take half as much damage. \\
        14-16 & Roll twice as many damage dice as normal and roll on the minor injury chart.
        If the creature is wearing metal armor roll on the major injury chart instead. \\
        17-18 & Deal the maximum amount of damage from your normal damage dice then roll your damage dice and add that result.
        The creature and each creature you choose within 15 ft. of it cannot take reactions until the end of their next turn.
        Then roll on the minor injury chart. \\
        19    & Deal twice the maximum result of your damage dice and roll on the major injury chart. \\
        20    & Deal twice the maximum result of your damage dice and you may choose one other creature within 15 ft. of the victim.
        That creature must succeed on a Dexterity saving throw (DC 20) or take half as much damage.
        Then roll on the major injury chart.
    \end{DndTable}

    \begin{DndTable}[width=\linewidth, header=Necrotic]{cX}
        \textbf{d20} & \textbf{Effect} \\
        1     & Roll damage as normal. \\
        2-3   & Roll damage as normal and the creature cannot regain hit points until the end of its next turn. \\
        4-6   & Do not roll your damage dice, instead deal the maximum result possible with those dice. \\
        7-8   & Roll your damage as normal and the creature’s maximum hit points are reduced by the same amount. \\
        9-11  & Roll twice as many damage dice as normal. \\
        12-13 & Roll twice as many damage dice as normal and the creature’s maximum hit points are reduced by the same amount.
        This reduction can only be healed as part of a long rest. \\
        14-16 & Roll twice as many damage dice as normal and the creature cannot regain hit points for the next minute.
        It may make a saving throw (DC 16) at the end of each of its turns to end this effect. \\
        17-18 & Deal the maximum amount of damage from your normal damage dice then roll your damage dice and add that result.
        The creature’s maximum hit points are reduced by the same amount.
        This reduction can only be healed as part of a long rest.
        Then roll on the minor injury chart. \\
        19    & Deal twice the maximum result of your damage dice and roll on the major injury chart. \\
        20    & Deal twice the maximum result of your damage dice, the creature’s maximum hit points are reduced by the same amount, and the creature cannot regain hit points until the end of its next turn.
        Then roll on the major injury chart.
    \end{DndTable}

    \begin{DndTable}[width=\linewidth, header=Poison]{cX}
        \textbf{d20} & \textbf{Effect} \\
        1     & Roll damage as normal. \\
        2-3   & Roll damage as normal and the creature has disadvantage on its next ability check, attack roll, or saving throw. \\
        4-6   & Do not roll your damage dice, instead deal the maximum result possible with those dice. \\
        7-8   & Roll your damage as normal and the creature has disadvantage on all ability checks, attack rolls, and saving throws until the end of its next turn. \\
        9-11  & Roll twice as many damage dice as normal. \\
        12-13 & Roll twice as many damage dice as normal and the creature is poisoned for the next minute.
        The creature may attempt a saving throw at the end of each of its turns (DC 12) to end this effect. \\
        14-16 & Roll twice as many damage dice as normal and the creature is poisoned for the next minute.
        The creature may attempt a saving throw at the end of each of its turns (DC 16) to end this effect. \\
        17-18 & Deal the maximum amount of damage from your normal damage dice then roll your damage dice and add that result.
        Then roll on the minor injury chart and the creature is poisoned for the next minute.
        The creature may attempt a saving throw at the end of each of its turns (DC 12) to end this effect. \\
        19    & Deal twice the maximum result of your damage dice and roll on the major injury chart. \\
        20    & Deal twice the maximum result of your damage dice, roll on the major injury chart, and the creature is poisoned for the next minute.
        The creature may attempt a saving throw at the end of each of its turns (DC 16) to end this effect.
    \end{DndTable}

    \begin{DndTable}[width=\linewidth, header=Psychic]{cX}
        \textbf{d20} & \textbf{Effect} \\
        1     & Roll damage as normal. \\
        2-3   & Roll your damage dice as normal and you control the creature’s movement on its next turn. \\
        4-6   & Roll your damage dice as normal and the creature cannot differentiate friend from foe until the end of its next turn. \\
        7-8   & Do not roll your damage dice, instead deal the maximum result possible with those dice. \\
        9-11  & Roll twice as many damage dice as normal. \\
        12-13 & Roll twice as many damage dice as normal and you control the creature’s actions on its next turn. \\
        14-16 & Roll twice as many damage dice as normal and roll on the Insanity chart with disadvantage. \\
        17-18 & Deal the maximum amount of damage from your normal damage dice then roll your damage dice and add that result.
        Then roll on the Insanity chart. \\
        19    & Deal twice the maximum result of your damage dice and roll on the Insanity chart. \\
        20    & Deal twice the maximum result of your damage dice and roll on the Insanity chart with advantage.
    \end{DndTable}

    \begin{DndTable}[width=\linewidth, header=Radiant]{cX}
        \textbf{d20} & \textbf{Effect} \\
        1     & Roll damage as normal. \\
        2-3   & Roll damage as normal and the creature cannot willingly move closer to you until the end of its next turn. \\
        4-6   & Do not roll your damage dice, instead deal the maximum result possible with those dice. \\
        7-8   & Roll your damage as normal and the creature is frightened until the end of its next turn. \\
        9-11  & Roll twice as many damage dice as normal. \\
        12-13 & Roll twice as many damage dice as normal and the creature is frightened until the end of its next turn. \\
        14-16 & Roll twice as many damage dice as normal and roll on the minor injury chart.
        Additionally, the creature is frightened for the next minute.
        It can make a Wisdom saving throw (DC 16) at the end of each of its turns to end this effect. \\
        17-18 & Deal the maximum amount of damage for your normal damage dice then roll your damage dice and add that result, then roll on the minor injury chart.
        Additionally, the creature glows for the next minute.
        While glowing it produces bright light up 10 feet and dim light up to 30 feet and all successful attacks against the creature deal an additional 1d4 radiant damage. \\
        19    & Deal twice the maximum result of your damage dice and roll on the major injury chart. \\
        20    & Deal twice the maximum result of your damage dice and roll on the major injury chart.
        Additionally, the creature glows for the next minute. While glowing it produces bright light up 10 feet and dim light up to 30 feet and all successful attacks against the creature deal an additional 1d6 radiant damage.
    \end{DndTable}

    \begin{DndTable}[width=\linewidth, header=Thunder]{cX}
        \textbf{d20} & \textbf{Effect} \\
        1     & Roll damage as normal. \\
        2-3   & Roll damage as normal and the creature is deafened until the end of its next turn. \\
        4-6   & Do not roll your damage dice, instead deal the maximum result possible with those dice. \\
        7-8   & Roll your damage as normal and the creature is deafened for one minute. \\
        9-11  & Roll twice as many damage dice as normal. \\
        12-13 & Roll twice as many damage dice as normal and the creature is stunned until the start of its next turn and deafened for one minute. \\
        14-16 & Roll twice as many damage dice as normal and the creature is deafened permanently.
        Then roll on the minor injury chart. \\
        17-18 & Deal the maximum amount of damage from your normal damage dice then roll your damage dice and add that result.
        The creature stunned until the end of its next turn and deafened permanently.
        Then roll on the minor injury chart. \\
        19    & Deal twice the maximum result of your damage dice and roll on the major injury chart. \\
        20    & Deal twice the maximum result of your damage dice, the creature is deafened permanently, and stunned until the end of its next round.
        Then roll on the major injury chart.
    \end{DndTable}

\subsubsection{Minor and Major Injuries}
    \begin{DndTable}[width=\linewidth, header=Minor and Major Injuries]{cXX}
        \textbf{d20} & \textbf{Minor Injury} & \textbf{Major Injury} \\
        1-3 &
        \textbf{Injured leg}.
        The creature’s movement speed is reduced by 10 ft. &
        \textbf{Crippled leg}.
        The creature’s movement speed is reduced by 10 feet and it has disadvantage on Dexterity saving throws. \\
        4-8 &
        \textbf{Injured arm}.
        Randomly determine one of the creature’s arms.
        That arm cannot be used to hold a shield and the creature has disadvantage on any rolls involving the use of that arm. &
        \textbf{Crippled arm}.
        Randomly determine one of the creature’s arms.
        That arm cannot be used to hold an item and any ability check requiring that arm automatically fails or has disadvantage (DM’s choice). \\
        9-11 &
        \textbf{Multiple injuries}.
        The creature’s maximum hitpoints are reduced by an amount equivalent to half of the damage dealt by the attack. &
        \textbf{Severely wounded}.
        The creature’s maximum hit points are reduced by an amount equivalent to the damage dealt by the attack. \\
        12-16 &
        \textbf{Badly beaten}.
        The creature has disadvantage on Constitution saving throws. &
        \textbf{Edge of death}.
        The creature has disadvantage on Constitution and death saving throws. \\
        17-19 &
        \textbf{Ringing blow}.
        The creature is stunned until the end of its next turn and deafened until it completes a short rest. &
        \textbf{Blinding blow}.
        The creature is blinded. \\
        20    &
        \textbf{Blow to the head}
        The creature is unconscious for 2d12 hours. &
        \textbf{Decapitated}.
        The creature is dead.
    \end{DndTable}

\subsubsection{Insanity}
    \begin{DndTable}[width=\linewidth, header=Insanity]{cX}
        \textbf{d20} & \textbf{Insanity} \\
        1  & \textbf{Synesthesia}.
        You can hear colors, smell sounds, or taste textures. Regardless of the specific manifestation, you have disadvantage on all Perception and Investigation skill checks. \\
        2  & \textbf{Kleptomania}.
        Once per day when you are in a personal residence or marketplace, the DM can call on you to succeed on a Wisdom saving throw (DC 12) or attempt to steal an item of insignificant practical and monetary value. \\
        3  & \textbf{Paranoia}.
        Once per day following an interaction with another creature (including other PCs) the DM can call on you to succeed on a Wisdom saving throw (DC 12) or you suspect that creature is secretly plotting against you. \\
        4  & \textbf{Obsession}.
        Choose a person or personal interest you are obsessed with. Once per day, when you are presented with an opportunity to interact with or learn more about the subject of your obsession the DM can call on you to succeed on a Wisdom saving throw (DC 14) or ignore everything else to obsess over the object of your fascination. \\
        5  & \textbf{Addiction}.
        Choose a behavior or substance you have used. Once per day, when you are presented with an opportunity to do the behavior or use the substance the DM can call on you to succeed on a Wisdom saving throw (DC 14) or ignore everything else to indulge in your vice. \\
        6  & \textbf{Odd Thinking}.
        Once per day when you hear a rational explanation for an event or occurrence, your DM may call on you to succeed on a Wisdom saving throw (DC 12) or you reject the rational explanation for a conspiratorial or fantastical explanation. \\
        7  & \textbf{Narcissism}.
        When you take an action or series of action that doesn’t directly benefit you, you must pass a Wisdom saving throw (DC 11) or you can’t take that action / series of actions.
        If any self-sacrifice on your part would be required the DC of the saving throw is increased to 16. \\
        8  & \textbf{Delusional}.
        When you gain this insanity the DM will tell you a belief that you have. No matter what evidence is presented to the contrary so long as you have this insanity you cannot be persuaded that this belief is not true. \\
        9  & \textbf{Pica}.
        Once per day the DM can call on you to pass a Wisdom saving throw (DC 14) or immediately eat one non-food object (such as dirt, napkins, or a small piece of jewelry) of the DM’s choice. \\
        10 & \textbf{Retrograde amnesia}.
        You forget everything about your personal life prior to the moment you received this insanity. \\
        11 & \textbf{Overwhelmed}.
        If you do not have immunity or resistance to psychic damage, you gain vulnerability to psychic damage.
        If you have resistance to psychic damage, you lose it.
        If you have immunity to psychic damage, you lose it but gain resistance to psychic damage. \\
        12 & \textbf{Anterograde amnesia}.
        Whenever you try to recall a fact you learned since you received this insanity, make a Wisdom saving throw (DC 12).
        If you fail you cannot recall the fact. \\
        13 & \textbf{Dependence}.
        You must pass a Wisdom saving throw (DC 14) to take an action that one or more of your allies disapprove of. \\
        14 & \textbf{Anxious}.
        You have disadvantage on saving throws against being frightened. Additionally, once per day the DM can call on you to succeed a Wisdom saving throw (DC 14) or be frightened by a creature of the DM’s choosing for the next minute. \\
        15 & \textbf{Mute}.
        Whenever you wish to speak allowed (including casting a spell that has verbal components) you must succeed on a Wisdom saving throw (DC 13) to do so. \\
        16 & \textbf{Narcolepsy}.
        You have disadvantage on saving throws against sleeping or unconsciousness.
        Once per day the DM may call on you to succeed on a Constitution saving throw (DC 12) or fall unconscious for one minute or until you take damage or another creature spends their action trying to rouse you. \\
        17 & \textbf{Insomnia}.
        You cannot take long rests and your short rests take 8 hours to complete. \\
        18 & \textbf{Homicidal}.
        After each long rest you must pass a Wisdom saving throw (DC 14) or be overcome with the urge to end the life of a humanoid creature and you cannot benefit from another long rest until you do so. \\
        19 & \textbf{Suicidal}.
        After each long rest you must pass a Wisdom saving throw (DC 12) or make an attempt to end your own life. \\
        20 & \textbf{Catatonia}.
        You are unconscious for 10d10 years.
    \end{DndTable}

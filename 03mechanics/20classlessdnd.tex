% !TEX root = ../main.tex
\section{Classless D\&D} \label{sec::classlessdnd}
\DndDropCapLine{A}{n inch away from death, a band of}
fellows barely manages to escape from a bloodthirsty nidhogg.
Worn and weary, they settle against an outcrop of rocks, taking a moment to catch their breath.

Nightfall comes all too fast, and they improvise a campfire to keep them warm during the night.
Just as they begin their well-deserved rest, they feel an unsettling rumbling beneath their feet.
What they originally thought was a strange-looking rock starts moving, revealing itself to be a troll.
Tired they scramble for arms, a gat grabbing their mace, and an ird her trusty quarterstaff.

\subsection*{Call to Adventure}
    Adventurers leave home as any person would, picking up an assortment of skills and feats on the road.
    Unskilled and untrained, they take their time to become experts in what they do.
    Lest they become prey to a savage beast or an opportunist bandit.

\subsection*{Creating a character}
    Your first step in playing is creating a character of your own.
    You choose a kin, which will determine their physical characteristics and provide them with a set of traits.
    Then, you choose a background, which describes what they did before joining the adventuring life and provides them with one feature.
    Additionally, your character's kin and background will give them access to a specific set of feats.
    You also choose your character's tidal alignment, invent the personality, appearance, and backstory of your character.

    \subsubsection{Ability Scores}
        After choosing your character's kin and background, it's time to generate their ability scores randomly.
        Roll four d6s and record the total of the highest three dice on a piece of scratch paper.
        Do this five more times, so that you have six numbers.

        Now take your six numbers and write each beside one of your character's six abilities to assign scores to Strength, Dexterity, Constitution, Intelligence, Wisdom, and Charisma.
        Afterward, make any changes to your ability scores as a result of your kin choice.

        After assigning your ability scores, determine your ability modifiers.
        This is done by subtracting 10 from the ability score and then dividing the result by 2 (round down).
        Write the modifier next to each of your scores.

    \subsubsection{Hit Points and Hit Dice}
        On creation, your character has a number of hit points equal to 8 + their Constitution modifier.
        Upon leveling up, roll a d8 and add the number rolled + your character's Constitution modifier to their hit point pool.
        Alternatively, you can forgo this roll and simply add 5 + your character's Constitution modifier.

        Your character has a number of hit dice equal to their level.
        By default all these hit dice are d8s, but they can be improved (or worsened) by taking relevant major character advancement, which are described in their own section at page \pageref{ssec::majorcharacteradvancement}.

    \subsubsection{Proficiency Levels}
        Instead of having one particular proficiency bonus, your character has specific proficiency levels in different skills, proficiencies, and saving throws.
        Proficiency levels in skills and proficiencies are gained via feats, while in saving throws they are gained at specific levels, as is listed in the Character Progression Table.

        There are five proficiency levels:
        \subparagraph{Untrained} Completely unskilled in the practice, you have no proficiency bonus.
        \subparagraph{Competent} Some basic experience gives you a +2 proficiency bonus.
        \subparagraph{Skilled} Practice makes perfect, you have a +4 proficiency bonus.
        \subparagraph{Expert} A fully realized professional, you have a +6 proficiency bonus.
        \subparagraph{Legendary} Your skill is lauded, and your crafts acclaimed.
        You have a +12 proficiency bonus.

    % NOTE. It might be cool to find a way to make this look more like the official class summary tables.
    \begin{DndTable}[width=\linewidth, header=Character Progression Table]{ccX}
        \textbf{Level} & \textbf{Required FP} & \textbf{Feature} \\
        1              & 0                    & 2 Saving Throw Improvements \\
        2              & 1                    & Ability Score Improvement \\
        3              & 2                    & Major Character Advancement \\
        4              & 3                    & Ability Score Improvement \\
        5              & 5                    & Saving Throw Improvement \\
        6              & 7                    & Ability Score Improvement \\
        7              & 9                    & - \\
        8              & 11                   & Ability Score Improvement \\
        9              & 13                   & Saving Throw Improvement \\
        10             & 15                   & Ability Score Improvement \\
        11             & 18                   & Major Character Advancement \\
        12             & 21                   & Ability Score Improvement \\
        13             & 24                   & Saving Throw Improvement \\
        14             & 27                   & Ability Score Improvement \\
        15             & 30                   & - \\
        16             & 34                   & Ability Score Improvement \\
        17             & 39                   & Saving Throw Improvement \\
        18             & 44                   & Ability Score Improvement \\
        19             & 51                   & Major Character Advancement \\
        20             & 60                   & Ability Score Improvement
    \end{DndTable}

% !TEX root = ../main.tex
\subsection*{Beyond First Level} \label{ssec::beyondfirstlevel}
\DndDropCapLine{W}{ith one heavy loss, the group}
dispatches the fearsome troll.
They perform a small funerary ritual, to then finally get some rest.
Beaten and tired, they learned their lesson for the day.

% TODO. Add more flavor text or add stuff here and there to fill the page.

\thispagestyle{empty}
\begin{tikzpicture}[remember picture,overlay]
    \node[anchor=south, yshift=-0.10cm] at (current page.south) {\includegraphics[width=\pdfpagewidth]{03mechanics/img/21jungle_adventure}};
\end{tikzpicture}

\subsubsection{Experience}
    What is learned by your character during the game is measured by Experience Points (XP).
    After you finish playing, talk about the session with the whole group and discuss what happened.
    Read the questions in the table below.
    For each of them that you can reply ``yes'' to, your character gets one XP.

    \begin{itemize}
        \item Did you participate in the session?
        \item Did you travel to a new location?
        \item Did you defeat one or more creatures?
        \item Did you loot treasure?
        \item Did you complete a quest?
        \item Did you solve a conflict?
        \item Did you make a new friend, ally, or enemy?
        \item Did you help another PC?
        \item Did your ideals, bonds, or flaws affect any of your decisions during the session?
        \item Did you perform any action related to your kin, background, or origin?
        \item Did you perform an extraordinary action of some kind?
    \end{itemize}

    In case of any discussion or disagreement, the DM always has the last word.
    Additional XP may be awarded when specific story milestones are achieved, but this too is left to the discression of the DM.

    \newpage

\subsubsection{Feat Points}
    As your character adventures, they will gain \textbf{Feat Points} (FP).
    Feat points are spent to buy feats, which can only be done as part of a long rest.
    You can buy multiple feats in one long rest.
    Feats either increase your proficiency level at a skill or a proficiency, or they include a useful feature.

    When you reach 10 XP, reset your XP counter and add one FP, keeping the remaining XP.
    Feat points are used to learn new feats, which include either an increase in a proficiency level or a feature.

    You don't need to spend FP right after earning it, you can save as many as you want for as long as you need.
    The full list of feats starts in the next page.

    A new character has a number of FP equal to 3 plus the number of FP that would be required to reach the character's level if it had leveled normally.

\subsubsection{Leveling Up}
    As your character gains feat points they will gain levels, acquiring hit dice, improving their ability scores and saving throw proficiencies, and gaining major character advancements.
    \begin{itemize}
        \item Ability Score Improvements allow you to increase one of you ability scores by 1 up to a maximum of 20.
        \item Saving Throw Improvements increase your level of proficiency on a saving throw of your choice by one, to a maximum of Expert proficiency.
        \item Major Character Advancements are character-defining improvements which include hit dice upgrades, combat styles, and proficiency improvements.
        They are listed in page \pageref{sec::majorcharacteradvancement}.
    \end{itemize}

    % When your Constitution modifier increases by 1, your hit point maximum increases by 1 for each level you have attained.
    % The same effect happens if you improve your hit dice.
    % Similarly, you decrease your hit point maximum by 1 if you worsen your hit dice.

    \newpage

% !TEX root = ../main.tex
\subsection*{Major Character Advancement} \label{ssec::majorcharacteradvancement}
Upon reaching levels 3, 11, and 19, your character gains major character advancements.
These are similar to feats, but have stronger effects and are more limited.

\subsubsection{Ability Modifier Improvement} \label{mca::abilitymodifierimprovement}
    Increase an ability modifier of your choice by 2.
    In addition, your maximum for that score is now 24.

\subsubsection{Legendary Proficiency} \label{mca::legendaryproficiency}
    Increase your level of proficiency in two skills or proficiencies to legendary, increasing your proficiency bonus on those skills to +12.
    You must have an expert level of proficiency in both.

\subsubsection{Fighting Style} \label{mca::fightingstyle}
    You learn a fighting style of your choice, or upgrade one that you already know.
    The list of fighting styles is available in page \pageref{ssec::fightingstyles}.

\subsubsection{Hit Die Improvement} \label{mca::hitdieimprovement}
    You increase your hit die from a d6 to a d8, a d8 to a d10, or a d10 to a d12.
    Additionally, you gain a number of hit points equal to your level when you take this major character advancement.

\subsubsection{Spellcasting School} \label{mca::spellcastingschool}
    You join a new spellcasting school from a doctrine different to the one you already follow.
    When you do this, you learn one medium from the new doctrine, and the signature medium and spell from the spellcasting school.

    In addition, you can learn any spell from the new doctrine as you normally would learn a new spell.

\subsubsection{Upgraded Spellcasting} \label{mca::upgradedspellcasting}
    You expand your spellcasting ability beyond its normal limit.
    You reduce your hit die from a d8 to a d6, a d10 to a d8, or a d12 to a d10.
    In addition, you lose a number of hit points equal to your level when you take this major character advancement.

    In exchange for this sacrifice, you gain the ability to cast higher level spells and add more spell points to your pool.
    Your total number of spell points and maximum spell level is available in the table below.

    \begin{DndTable}[width=\linewidth, header=Spellcasting Ability]{XX}
        \textbf{Level} & \textbf{Spell Points} \\
         1st           &   4 \\
         2nd           &   6 \\
         3rd           &  14 \\
         4th           &  17 \\
         5th           &  27 \\
         6th           &  32 \\
         7th           &  38 \\
         8th           &  44 \\
         9th           &  57 \\
        10th           &  64 \\
        11th           &  73 \\
        12th           &  73 \\
        13th           &  83 \\
        14th           &  83 \\
        15th           &  94 \\
        16th           &  94 \\
        17th           & 107 \\
        18th           & 114 \\
        19th           & 123 \\
        20th           & 133
    \end{DndTable}

    \paragraph{Requirements} Spellcaster feat (page \pageref{feat::spellcaster}).

% !TEX root = ../main.tex
\subsection*{Fighting Styles} \label{ssec::fightingstyles}
    \subsubsection{Archery 1}
        You gain a +2 bonus to attack rolls you make with ranged weapons.
    \subsubsection{Archery 2}
        Both your normal and long ranges are multiplied by 1.5 for all ranged weapons.
    \subsubsection{Battle Mastery 1}
        You gain a number a Maneuver Dice equal to half your level (rounded down).
        Your Maneuver Dice are d6s, and are special dice that you can use to enhance the effect of certain actions.
        You regain all expended Maneuver Dice at the end of a short rest.

        Whenever you take an action that involves a d20 roll which isn't the Attack or Cast a Spell action, you can choose to roll a Maneuver Die, adding the result of the die to the d20.
    \subsubsection{Battle Mastery 2}
        You increase the die associated to your Maneuver Dice from a d6 to a d8.

        In addition, when you roll for initiative and have no Maneuver Dice remaining, you regain 4 Maneuver Dice.
    \subsubsection{Dueling 1}
        When you are wielding a melee weapon in one hand and no other weapons, you gain a +2 bonus to damage rolls with that weapon.
    \subsubsection{Dueling 2}
        When you are wielding a melee weapon in one hand and no other weapons, you gain a +1 bonus to AC.
    \subsubsection{Great Weapon Fighting 1}
        When you roll a 1 or 2 on a damage die for an attack you make with a melee weapon that you are wielding with two hands, you can reroll the die and must use the new roll, even if the new roll is a 1 or a 2.
        The weapon must have the two-handed or versatile property for you to gain this benefit.
    \subsubsection{Great Weapon Fighting 2}
        You gain a +1 bonus to attack rolls you make with weapons with the two-handed or versatile property.

        In addition, you gain a +1 bonus to attack rolls you make with weapons with the heavy property.
    \subsubsection{Monastic Fighting 1}
        You gain a number of monastic dice equal to half your level (rounded down).
        Your monastic dice are d6s, and are special dice you can spend to use special abilities related to this fighting style.
        You regain all expended monastic dice at the end of a short rest.

        In addition, you learn three abilities when you learn this fighting style:
        \subparagraph{Concentrated Attack} After you hit a creature with an unarmed attack but before you roll damage for the attack, you can spend one monastic die and add the number rolled to the damage done.
        \subparagraph{Patient Defense} You can spend one monastic die and take the Dodge action using only one action.
        When you do this, add the number rolled to your AC until the start of your next turn.
        \subparagraph{Step of the Wind} You can spend one monastic die to take the Disengage action using only one action.
        When you do this, add the 1.5 times the number rolled to your movement speed (in meters) until the end of your turn.
    \subsubsection{Monastic Fighting 2}
        You increase the die associated to your monastic dice from a d6 to a d8.

        In addition, when you roll for initiative and have no monastic dice remaining, you regain 4 monastic dice.
    \subsubsection{Mounted Fighting 1}
        You have advantage on saving throws made to avoid falling off your mount.
        If you fall off your mount and descend no more than 2 meters, you can land on your feet if you're not incapacitated.

        Finally, mounting or dismounting a creature requires one action rather than two for you.

        To take this fighting style, you must be proficient with land vehicles.
    \subsubsection{Mounted Fighting 2}
        Your steed gains a number of hit points equal to half your maximum hit points.
    \subsubsection{Protection 1}
        While you are wearing armor, you gain a +1 bonus to AC.
    \subsubsection{Protection 2}
        You gain a bonus to all Strength, Dexterity, and Constitution saving throws equal to the AC granted by your shield.
    \subsubsection{Thrown Weapon Fighting 1}
        You can draw a weapon that has the thrown property as part of the attack you make with the weapon.
    \subsubsection{Thrown Weapon Fighting 2}
        When you hit with a ranged attack using a thrown weapon, you add a +2 bonus to the damage roll.
    \subsubsection{Two-Weapon Fighting 1}
        You suffer from the multiple attack penalty for each wielded weapon independently --- Attack rolls made with a weapon held in one hand don't impose a penalty on attack rolls made with a weapon held in your other hand.
    \subsubsection{Two-Weapon Fighting 2}
        When you engage in two-weapon fighting, you can add your ability modifier to the damage of the second attack.
    \subsubsection{Unarmed Fighting 1}
        Your unarmed strikes can deal bludgeoning damage equal to 1d6 + your Strength modifier on a hit.

        In addition, the multiple attack penalty is reduced by 1 when you make unarmed attacks.
    \subsubsection{Unarmed Fighting 2}
        If you aren't wielding any weapons or a shield, the damage die from your unarmed strikes becomes a d8.

        In addition, you can deal 1d4 bludgeoning damage to one creature grappled by you at the start of each of your turns.

\newpage

% % !TEX root = ../main.tex
\section{Feats}
For simplicity, feats are separated into six categories: feats related to kin, background, skills, proficiencies, combat, and spellcasting.
Each category is listed in the following pages.
Categories are ordered alfabetically, but each is listed in an easy to follow fashion in their respective sections.

Unless specified otherwise, a feat can only be learned once.

% TODO: Write list of Fighting Styles and ``Casting Styles'' (Based on this: https://standsinthefire.com/2016/05/31/dd-5e-magical-styles/).
% TODO: Combat maneuvers, metamagic options and other **don't have a point system associated to them**. They simply add effects to attacks or spellcasting by using one additional action!
% TODO: Write Combat Maneuvers section.
% TODO: Write list of Metamagic Options.

% !TEX root = ../main.tex
\subsection*{Kin Feats}

% GAT
\subsubsection{Force of Will} \label{feat::forceofwill}
    You can plant yourself in place with all your weight, making you very difficult to move.
    As long as your feet are on the ground, you have advantage on any ability checks or saving throws made to move you or force you to fall prone.
    \paragraph{Requirements} Gat or Quies kin.
\subsubsection{Powerful Build} \label{feat::powerfulbuild}
    You count as one size larger when determining your carrying capacity and the weight you can push, drag, or lift.
    \paragraph{Requirements} Gat or Tortle kin.
\subsubsection{Goat's Strength} \label{feat::goatsstrength}
    You complement your natural strengths with hard training.
    The DC related to the Push action gains a +2 bonus for you.
    Additionally, any attack made with your horns gains a +2 to its attack rolls.

    You can take this feat three times.
    \paragraph{Requirements} Gat kin.
\subsubsection{Efficient Craftsgat} \label{feat::efficientcraftsgat}
    They say that a gat is born with tools in their hands.
    Crafting times with the artisan's tools related to your craftgatship trait are halved for you.
    \paragraph{Requirements} Gat kin.
\subsubsection{Relentless Endurance} \label{feat::relentlessendurance}
    When you are reduced to 0 hit points but not killed outright, you can drop to 1 hit point instead.
    You can't use this feature again until you finish a long rest.
    \paragraph{Requirements} Gat kin.
\subsubsection{Stonecutting} \label{feat::stonecutting}
    Originally designed as miners, all gats have a history in stonecutting.
    Whenever you make an Intelligence (History) check related to the origin of stonework, you are considered of Legendary proficiency in the History skill, adding a proficiency bonus of +12 to the check instead of your normal proficiency bonus in the skill.
\subsubsection{Legendary Craftsgat (2 FP)} \label{feat::legendarycraftsgat}
    Above the average gat, you are legendary at the use of your tools.
    Your proficiency with the artisan's tools related to your craftgatship trait is increased to Legendary, incresing your proficiency bonus to +12 with them.
    \paragraph{Requirements} Gat kin. Expert proficiency with a set of artisan's tools.
% NOVES GAT
\subsubsection{Mountain Born} \label{feat::mountainborn}
    You are acclimated to high altitude.
    You don't suffer the effects associated to the cold or lack of oxygen of any elevation up to 10,000 meters.
    You also have a climbing speed of 6 meters.
    \paragraph{Requirements} Noves Gat or Thulkraka Ird subrace.
\subsubsection{Stone's Endurance} \label{feat::stonesendurance}
    You can focus yourself to occasionally shrug off injury.
    When you take damage, you can use your reaction to roll a d12.
    Add your Constitution modifier to the number rolled, and reduce the damage by that total.
    After you use this trait, you can't use it again until you finish a short rest.
    \paragraph{Requirements} Noves Gat or Juggernaut Quies subrace.
\subsubsection{Gat Resilience (2 FP)} \label{feat::gatresilience}
    The horned kin possesses an almost otherwordly hardiness.
    You have advantage on saving throws against poison and are resistant against poison damage.
    \paragraph{Requirements} Noves Gat subrace.
% BUGHNA GAT
\subsubsection{Hardy} \label{feat::hardy}
    Resilience comes as natural to you as breathing.
    As long as you are not already an expert, you increase your proficiency in the Survival skill.
    Additionally, you can choose to add your Constitution modifier to Survival ability checks instead of your Wisdom modifier.
    \paragraph{Requirements} Bughna Gat or Gannagian Warrior subrace.
\subsubsection{Mirthful Leaps} \label{feat::mirthfulleaps}
    Whenever you make a long or high jump, you can roll a d4 and add the number rolled to the number of meters you cover, even when making a standing jump.
    This extra distance costs movement as normal.
    \paragraph{Requirements} Bughna Gat or Grung kin.
\subsubsection{Exercised Mind (2 FP)} \label{feat::exercisedmind}
    Generational learning has taught you that a qualar is only as strong as your will is.
    You are naturally good staving off dementia.
    You can choose to change the Dementia saving throw (see page \pageref{ssec::dementia}) from Intelligence to Wisdom, and its DC is 12 for you.
    Additionally, if you succeed on the saving throw, you reduce your dementia level by 1.
    \paragraph{Requirements} Bughna Gat subrace.
% TREB GAT
\subsubsection{Relentless Striker} \label{feat::relentlessstriker}
    Your hammering horns are your most valuable weapons, and you are trained to use them as a normal part of your arsenal.
    Melee attack rolls made with your horns don't add to your multiple attack penalty.
    \paragraph{Requirements} Treb Gat subrace.
\subsubsection{Goring Rush} \label{feat::goringrush}
    If you move at least 6 meters towards a creature, you can make one melee attack or use the Push action with your horns against the creature as a free action.
    \paragraph{Requirements} Treb Gat subrace.
\subsubsection{Imposing Presence (2 FP)} \label{feat::imposingpresence}
    You increase your proficiency level in Intimidation or Persuasion to Legendary, increasing your proficiency bonus to +12.
    \paragraph{Requirements} Treb Gat or Cursed Uman subrace. Expert proficiency in the chosen skill.

% IRD
\subsubsection{Perfect Landing} \label{feat::perfectlanding}
    Furthering your natural falling skills, you are trained to handle any fall with ease.
    As long as you are conscious and can freely use your wings, you are immune to fall damage.
    \paragraph{Requirements} Ird kin.
\subsubsection{Songbird} \label{feat::songbird}
    Every so often, you can demonstrate the innate power of your Charisma.
    You may cast the charm person spell a number of times equal to your Charisma modifier (Minimum of one) per short rest.
    This spell does not require any somatic components to cast.

    The DC of this spell is equal to 10 + your Charisma modifier, but you can increase this by 2 twice by taking this feat again a second and third time.

    Additionally, you have advantage in any Charisma (Performance) check that involves whistling.
    \paragraph{Requirements} Ird kin.
\subsubsection{Deadly Grip} \label{feat::deadlygrip}
    While flying, you have advantage on Grapple checks made with your talons.
    \paragraph{Requirements} Ird kin.
\subsubsection{Aerial Defense (2 FP)} \label{feat::aerialdefense}
    Creatures who attack you while you're falling, flying, gliding, or jumping have disadvantage on their attack roll.
    \paragraph{Requirements} Ird kin.
\subsubsection{Wing-Assisted Running} \label{feat::wingassistedrunning}
    You increase your base walking speed by 1.5 meters.
    Additionally, you gain a climbing speed equal to half your flying speed.
    \paragraph{Requirements} Ird kin.
\subsubsection{Nimble Step} \label{feat::nimblestep}
    Opportunity attacks made against you are rolled with disadvantage.
    \paragraph{Requirements} Ird or Zaloth kin.
\subsubsection{Graceful Pass} \label{feat::gracefulpass}
    Using your wings to aid your movement, difficult terrain doesn't cost you extra movement.
    \paragraph{Requirements} Ird or Oth kin.
% QULBABA IRD
\subsubsection{Forest Defender} \label{feat::forestdefender}
    You have received martial training to fight among the branches, and are extremely dangerous to climbing opponents.
    You know the Push action (See page \pageref{act::push}), and the target rolls the saving throw with disadvantage if you climbed, glided, or flew at least 3 meters this turn.
    \paragraph{Requirements} Qulbaba Ird or Marset kin.
\subsubsection{Woodland Hunter} \label{feat::woodlandhunter}
    Your accuracy allows you to treat three-quarters cover as half cover and half cover as no cover.
    \paragraph{Requirements} Qulbaba Ird or Gannagian Hunter subrace.
\subsubsection{Keenest Senses (2 FP)} \label{feat::keenestsenses}
    You increase your proficiency level with Insight, Investigation, or Perception from Expert to Legendary, increasing your proficiency bonus with the skill to +12.
    \paragraph{Requirements} Qulbaba Ird or Boggart subrace.
% THULKRAKA IRD
\subsubsection{Thulkraka Steel} \label{feat::thulkrakasteel}
    You are specially skilled with a set of smith's tools.
    You can craft any metal item for half the normal cost.

    Additionally, if you have access to a forge at an altitude of at least 3,000 meters, you can use the quench-hardening technique unique to your people and craft the item for a quarter its normal cost.
    \paragraph{Requirements} Thulkraka Ird subrace.
\subsubsection{Giant Slayer (2 FP)} \label{feat::giantslayer}
    An apprentice of the giant slayers from the north, your skill with heavy weapons is unparalleled.
    You gain a +2 bonus to attack rolls made with heavy weapons, and learn the Reckless Attack action (see page \pageref{act::recklessattack}).
    \paragraph{Requirements} Thulkraka ird subrace.
% DRATL IRD
\subsubsection{Patient} \label{feat::patient}
    When you react with a readied action, you have advantage on the first die roll you make as part of that action.
    \paragraph{Requirements} Dratl Ird subrace.
\subsubsection{Deadly Precision (2 FP)} \label{feat::deadlyprecision}
    Whenever you have advantage on an attack roll using Dexterity, Intelligence, Wisdom, or Charisma, you can reroll one of the dice once.
    Additionally, you learn the Aim action.
    \paragraph{Requirements} Dratl Ird or Sunstruck Oth subrace.

% MARSET
\subsubsection{Born Climber} \label{feat::bornclimber}
    You can use your tail to grab onto branches, and don't require to pass any ability checks to climb.
    Additionally, you are able to freely use your hands while climbing, and you increase your climbing speed by three meters.
    \paragraph{Requirements} Marset kin.
\subsubsection{Impaling Carapace} \label{feat::impalingcarapace}
    Through careful training, you know how to position your natural armor for deadly effect in combat.
    When you grapple a creature, the target takes 3 piercing damage if your grapple check succeeds.
    If another creature grapples you, it takes 3 piercing damage.
    If you start your turn grappling or being grappled by a creature, it takes 3 piercing damage.
    \paragraph{Requirements} Marset or Tortle kin.
\subsubsection{Lip Reading} \label{feat::lipreading}
    If you can see a creature's mouth while it is speaking a language you understand, you can understand what it's saying by reading its lips.
    \paragraph{Requirements} Marset kin.
\subsubsection{Curl Up (2 FP)} \label{feat::curlup}
    You can use one action to curl up, exposing attackers to a wall of your toughened quills.
    While curled up in this way you cannot move, attack, or cast spells with somatic components, and your base armor class becomes 19.
    You cannot benefit from any Dexterity bonus to armor class while curled up, but you can still use shields.

    Any creature that misses you with a melee attack while you are curled up takes 2d8 + your Dexterity modifier piercing damage from your sharp spines.
    You may uncurl as a free action at any point during your turn.
    \paragraph{Requirements} Marset kin.
\subsubsection{Communal} \label{feat::communal}
    Whenever you make an Intelligence (History) check related to the history of your kin, culture, or community, you are considered of Legendary proficiency in the History skill, adding a proficiency bonus of +12 to the check instead of your normal proficiency bonus in the skill.
    \paragraph{Requirements} Marset kin.
\subsubsection{Seedspeech} \label{feat::seedspeech}
    Through sounds and touch, you can communicate simple ideas to living plants, and are able to interpret their responses as simple language.
    Plants do not perceive the world in terms of sight, but most can feel differences in temperature, describe things that have touched them, as well as hear vibrations that happened around them (including speech).
    \paragraph{Requirements} Marset kin.
\subsubsection{Acid Spit} \label{feat::acidspit}
    By repeatedly feeding on toxic tree leaves, your saliva has become as vile as its poison.
    You can use two actions to spit a 1.5 by 9 meters line of acid.
    Every target in the area must make a Dexterity saving throw, with a DC of 8 + your Constitution modifier.
    A creature takes 2d6 acid damage on a failed save, or half as much damage on a successful one.
    The acid damage increases to 3d6 with your 6th hit die, 4d6 with your 11th, and 5d6 with your 16th.
    You can use this trait a number of times equal to your Constitution modifier (Minimum of 1) per short rest.

    You can take this feat three additional times, increasing the DC by 2 each time.
    \paragraph{Requirements} Marset kin.
\subsubsection{Natural Forager} \label{feat::naturalforager}
    Due to your nature as a gatherer, all foraging DCs are reduced by 5 for you.
    You can also feed on tree gum, but cannot cook meals for other species with this material.
    \paragraph{Requirements} Marset or Gannagian Gatherer kin.
\subsubsection{Nimble and Deadly (2 FP)} \label{feat::nimbleanddeadly}
    You can move through the space of any creaturee that is of a size larger than you.
    When you do so, the creature takes 3 piercing damage.
    \paragraph{Requirements} Marset kin.

% OTH
\subsubsection{Silent Speech} \label{feat::silentspeech}
    You can communicate with other oths, thri-kreen, and other insectoids using Silent Speech.
    This language can only communicate simple ideas, and it does so via a combination of pheromones and thrumming sounds made with antennae.
    \paragraph{Requirements} Oth kin.
\subsubsection{Silk Spinning} \label{feat::silkspinning}
    Dust kin's silk is known for its beauty and strength, and oth's know how to craft clothes and various items with it.
    Your silk counts as a cloth material, and armor made with it has a +1 bonus to its AC.
    You can produce up to one kg of silk per long rest, which you can store for future use.
    \paragraph{Requirements} Oth kin. At least Untrained proficiency with weaver's tools.
\subsubsection{Keen Mind} \label{feat::keenmind}
    You always know which way is north, and you can always know the number of hours left before the next sunrise or sunset.
    \paragraph{Requirements} Oth kin.
\subsubsection{Predestined} \label{feat::predenstined}
    Once per day you can choose to reroll an attack, skill check, or saving throw.
    You can decide to do this after the roll, but before the outcome of the roll has been determined.
    If you are a Chu'ash oth, consider your Fated trait as one additional use of this feat.

    You can learn this trait 3 times, increasing the number of times you can use this feat per day by one each time.
    On the third time, you gain the ability to use this trait to force an enemy to reroll an attack roll made against you.
    \paragraph{Requirements} Oth kin.
\subsubsection{Third Weapon (2 FP)} \label{feat::thirdweapon}
    You can use your two extra arms to hold a weapon with the light property.
    You can attack with this weapon as a free action after you make a melee weapon attack, reducing the attack roll by the apropiate multiple attack penalty.
    \paragraph{Requirements} Oth kin.
\subsubsection{Born Grappler} \label{feat::borngrappler}
    You don't need to have a hand free to grapple, using both your extra arms to do so.
    If you do have a free arm, you roll for the grapple check with advantage.
    You cannot use this feat if you are holding something in one or both of your extra arms.
    \paragraph{Requirements} Oth kin.
% MOONBORN
\subsubsection{Observant} \label{feat::observant}
    You gain a +5 bonus to your passive Wisdom (Perception) and passive Intelligence (Investigation) scores.
    \paragraph{Requirements} Moonborn Oth or Na'anian Tsanek subrace.
\subsubsection{Linguist} \label{feat::linguist}
    You have studied languages and codes, gaining the following benefits:
    \begin{itemize}
        \item You learn a language of your choice.
        \item You can ably create written ciphers.
        Others can't decipher a code you create unless you teach them, they succeed on an Intelligence check (DC equal to 6 + your Intelligence score), or they use magic to decipher it. % TODO: Figure out what spell (probably from Dremshamad) would be able to do this.
    \end{itemize}
    \paragraph{Requirements} Moonborn Oth subrace.
\subsubsection{Magic Initiate (2 FP)} \label{feat::magicinitiate}
    Your learn two cantrips of your choice from one spellcasting doctrine of your choice.
    In addition, choose one 1st-level spell from the same doctrine.
    You learn that spell and can cast it at its lowest level.
    Once you cast it, you must finish a long rest before you can cast it again using this feat.
    Your spellcasting ability for these spells is Intelligence.
    \paragraph{Requirements} Moonborn Oth subrace.
% CHU'ASH
\subsubsection{Fey Touched} \label{feat::feytouched}
    You learn the misty step (see page \pageref{spell:mistystep}) and one 1st-level spell of your choice.
    The spell must be from the Sympathy doctrine.
    You can cast each of these spells without expending a spell slot.
    Once you cast either of these spells in this way, you can't cast that spell in this way again until you finish a short rest.
    You can also cast these spells using spell slots you have of the appropriate level.
    The spells' spellcasting ability is Intelligence.
    \paragraph{Requirements} Chu'ash Oth subrace.
\subsubsection{Bountiful Luck (2 FP)} \label{feat::bountifulluck}
    Your people have extraordinary luck, which you have learned to lend to your companions when you see them falter.
    You're not sure how you do it; you just wish it, and it happens.
    Surely a sign of fortune's favor!

    When an ally you can see within 9 meters of you rolls a 1 on the d20 for an attack roll, an ability check, or a saving throw, you can use your reaction to let the ally reroll the die.
    The ally must use the new roll.

    When you use this ability, you can't use your Fated trait before the end of your next turn.
    \paragraph{Requirements} Chu'ash Oth subrace.
\subsubsection{Lucky} \label{feat::lucky}
    When you roll a 1 on an attack roll, ability check, or saving throw, you can reroll the die and must use the new roll.
    This doesn't count as a use of your Fated trait.
    \paragraph{Requirements} Chu'ash Oth subrace.
% SUNSTRUCK
\subsubsection{Sun-powered} \label{feat::sunpowered}
    By reabsorbing your sweat and taking in dew, you do not need to drink water to survive.
    You gain full sustenance when you use your Photosynthesis trait, including both food and water.
    Additionally, you only need to keep your hindwings for one hour exposed to sunlight or two hours exposed to moonlight to be properly well fed.
    \paragraph{Requirements} Sunstruck Oth subrace.
\subsubsection{Bone Breaker} \label{feat::bonebreaker}
    While gliding, you can attempt to attack a creature with an eviscerating attack.
    Using two actions, you can swoop down up to your movement speed towards a creature you can see, and make a melee weapon attack roll against it.
    If the attack hits, it's a critical hit.
    The attack is tiring, and you can use this feat only once per combat encounter.
    \paragraph{Requirements} Sunstruck Oth subrace.

% NAENK
\subsubsection{Moss Armor} \label{feat::mossarmor}
    Your plant-based framework provides you with unique resilience.
    You have resistance against lightning damage.
    % You have advantage on saving throws against paralysis and are resistant to lightning damage.
    \paragraph{Requirements} Naenk kin.
\subsubsection{Automatic Nutrition} \label{feat::automaticnutrition}
    When in fertile land, you don't need to eat.

    Additionally, you can enter a state of hibernation as part of a short rest, becoming indistinguishable from a large plant.
    While hibernating you don't age, and are aware of your surroundings.
    You cannot use any actions, except to take an action to wake up at will.
    \paragraph{Requirements} Naenk kin.
\subsubsection{Cellular Regeneration} \label{feat::cellularregeneration}
    As an action, you can stimulate your plant cells to rapidly multiply to quickly regenerate wounds.
    You regain 1 hit point, and regain 1 hit point at the start of each of your turns after, until you've restored an amount equal to twice your level.
    After using this trait, you must take a long rest before using it again.
    If you suffer cold, fire, or necrotic damage, this regeneration is cancelled.
    You are also able to regenerate lost limbs, albeit at a very slow pace: it takes you 1d4+2 months to fully recover a lost arm or leg.
    \paragraph{Requirements} Naenk kin.
\subsubsection{Enhanced Claws} \label{feat::enhancedclaws}
    By manipulating your biology, you can improve your claws to become deadly weapons.
    You change the damage die from your claws to the following one, so that a d4 changes into a d6, a d6 into a d8, a d8 into a d10, or a d10 into a d12.

    You can take this feat three times, improving the damage die by the described amount.
    \paragraph{Requirements} Naenk kin.
\subsubsection{Take Root (2 FP)} \label{feat::takeroot}
    As an action, you can plant your feet on the ground, making you impossible to move.
    Any check to move you or knock you prone in any way automatically fails, and you cannot take the move action.
    You can deplant your feet from the ground by expending a second action.

    You can use this ability when riding a Nuen or standing over any plant-based creature.
    \paragraph{Requirements} Naenk kin.
\subsubsection{Nature's Sanctuary} \label{feat::naturessanctuary}
    Creatures of the natural world sense your connection to nature and become hesitant to attack you.
    When a beast or plant creature attacks you, that creature must make a Wisdom saving throw with a DC of 12.
    On a failed save, the creature must choose a different target, or the attack automatically misses.
    On a successful save, the creature is immune to this effect for 24 hours.

    The creature is aware of this effect before it makes its attack against you.
    \paragraph{Requirements} Naenk or Tsanek kin.
\subsubsection{Primeval Awareness} \label{feat::primevalawareness}
    You can use an action to focus your awareness on the region around you.
    For 1 minute, you can sense all creatures that aren't plants or beasts present within 1 kilometer of you.
    This feat doesn't reveal the creatures' exact location or number, but it gives you a hint of their position in relation to you.

    You can use this feature a number of times equal to your Wisdom modifier (minimum of once).
    \paragraph{Requirements} Naenk kin or Boggart subrace.
% GANNAGIAN WARRIOR
\subsubsection{Poisoned Claws} \label{feat::poisonedclaws}
    When you hit a creature with an unarmed strike using your claws, the creature must roll on a DC 12 Constitution saving throw.
    On a failed save, the creature is poisoned.
    The creature can repeat this saving throw at the end of its turns, ending the poisoned conditions on a successful save.
    \paragraph{Requirements} Gannagian Warrior subrace.
\subsubsection{Improved Nuen (2 FP)} \label{feat::improvednuen}
    Your nuens are stronger than average, and you do not half the creature's hit points or hit dice when you raise one.
    Additionally, your nuen grows a thick layer of thorns, gaining the \textbf{Impaling Carapace} feat (see page \pageref{feat::impalingcarapace}).
    \paragraph{Requirements} Gannagian Warrior subrace.
% GANNAGIAN HUNTER
\subsubsection{Improved Plant Camouflage} \label{feat::improvedplantcamouflage}
    You have advantage in Dexterity (Stealth) checks you make, using your mutable form to mimic an object in the environment.
    \paragraph{Requirements} Gannagian Hunter subrace.
\subsubsection{Tireless (2 FP)} \label{feat::tireless}
    As an action, you can give yourself a number of temporary hit points equal to 1d8 + your Constitution modifier.
    You can use this ability a number of times equal to your level, and you regain all expended uses when you finish a long rest.

    In addition, you can decrease one level of exhaustion by resting for an hour.
    \paragraph{Requirements} Gannagian Hunter or Common Uman subrace.
% GANNAGIAN GATHERER
\subsubsection{Smell the Danger} \label{feat::smellthedanger}
    % As long as you are not already a master, you increase your proficiency with the herbalism kit.
    You do not need to roll anything to tell if a plant is poisonous, even if you've never encountered it before.
    \paragraph{Requirements} Gannagian Gatherer subrace.
\subsubsection{Master Gatherer (2 FP)} \label{feat::mastergatherer}
    You increase your proficiency with a herbalism kit to Legendary, increasing your proficiency bonus with it to +12.
    \paragraph{Requirements} Gannagian Gatherer subrace. Expert proficiency with a herbalism kit.
% NA'ANIAN NAENK
\subsubsection{Halo of Spores} \label{feat::haloofspores}
    You are surrounded by invisible, necrotic spores that are harmless until you unleash them on a creature nearby.
    When a creature you can see moves into a space within 3 meters of you or starts its turn there, you can use your reaction to deal 1d4 necrotic damage to that creature unless it succeeds on a Constitution saving throw against a DC of 8 + your Constitution modifier.
    The necrotic damage increases to 1d6 at 6th level, 1d8 at 10th level, and 1d10 at 14th level.
    \paragraph{Requirements} Na'anian Naenk or Na'anian Tsanek subrace.
\subsubsection{Fungal Abode} \label{feat::fungalabode}
    Spending a minute to communicate with the ground, you bring forth a 3 meter dome-shaped hollow fungus from mycelium.
    Nine creatures of Medium size or smaller can fit inside the dome with you.
    The atmosphere inside the fungus is humid, making it slightly uncomfortable for most creatures.
    The interior is dimly lit.

    The dome has an AC of 8 and an HP of 15, and lasts for 8 hours.
    After this time or if the dome is destroyed, it returns to the mycelium.
    \paragraph{Requirements} Tsanek kin or Na'anian Naenk subrace.

% TSANEK
\subsubsection{Enhanced Spores} \label{feat::enhancedspores}
    You can add a +2 bonus to all of your spore effects' spell save DC.
    % Alternatively, you can add your spellcasting modifier if you have one.

    You can take this feat three times.
    \paragraph{Requirements} Tsanek kin.
\subsubsection{Mycelium Connection} \label{feat::myceliumconnection}
    As an action, you can establish a connection with a creature within 9 meters of you.
    Roll a Wisdom (Insight) check contested by the creature's Wisdom (Insight).
    The creature can choose to fail on this save on purpose.
    If you succeed, you know the location of the creature and can telepathically communicate with it for 24 hours.
    The creature knows that it is connected to you, as it feels your presence pulsating in its head, but it cannot end this connection.

    This ability only works while the creature is touching the ground.
    \paragraph{Requirements} Tsanek kin.
\subsubsection{Telepathic} \label{feat::telepathic}
    You can speak telepathically to any creature you can see within 18 meters of you.
    Your telepathic utterances are in a language you know, and the creature understands you only if it knows that language.
    Your communication doesn't give the creature the ability to respond to you telepathically.

    Additionally, you can cast the detect thoughts (see page \pageref{spell::detectthoughts}) spell, requiring no spell slot or components, and you must finish a short rest before you can cast it this way again.
    Your spellcasting ability for the spell is your choice between Intelligence and Wisdom.
    \paragraph{Requirements} Tsanek or Zaloth kin.
\subsubsection{Fungal Body} \label{feat::fungalbody}
    Your normal sight and hearing are extended by the spores around you.
    You gain truesight to a range of 3 meters, and become immune to the blinded and deafened conditions.
    \paragraph{Requirements} Tsanek kin.
\subsubsection{Symbiotic Entity (2 FP)} \label{feat::symbioticentity}
    You gain the ability to channel magic into your spores.
    As 2 actions, you can awaken those spores, and you gain 4 temporary hit points for each level you have.
    While this feature is active, you gain the following benefits:
    \begin{itemize}
        \item Your melee weapon attacks deal an extra 1d6 necrotic damage to any target they hit.
        \item Roll the damage die of any attack involving spores an additional time.
    \end{itemize}
    These benefits last for 10 minutes, or until you lose all these temporary hit points.
    \paragraph{Requirements} Na'anian Naenk or Tsanek kin.
% GANNAGIAN TSANEK
\subsubsection{Narcotic Empowerement} \label{feat::narcoticempowerement}
    From your training as a shaman, you are able to release special chemicals as part of your meditation.
    With an hour of rest, you choose expended spell slots to recover.
    The spell slots can have a combined level that is equal to or less than half your level (rounded up), and none of the slots can be 6th level or higher.
    You can't use this feature again until you finish a short rest.

    For example, when you are 4th-level, you can recover up to two levels worth of spell slots.
    You can recover either a 2nd-level slot or two 1st-level slots.
    \paragraph{Requirements} Gannagian Tsanek subrace.
\subsubsection{Balm of Dreams} \label{feat::balmofdreams}
    You learn how to excrete a healing balm from your fungal growths.
    You have a pool of balm represented by a number of d4s equal to your level.

    As an action, you can choose one creature you can see withint 1.5 meters of you and spend a number of those dice equal to half your level or less.
    Roll the spent dice and add them together.
    The target regains a number of hit points equal to the total.
    The target also gains 1 temporary hit point per die spent.

    This balm is uneffective on tsaneks.
    You regain all expended dice when you finish a short rest.
    \paragraph{Requirements} Gannagian Tsanek subrace.
\subsubsection{Fungal Infestation (2 FP)} \label{feat::fungalinfestation}
    Your spores gain the ability to infest a corpse and animate it.
    If a beast or a humanoid that is Small or Medium dies within 3 meters of you, you can use your reaction to animate it, causing it to stand up immediately with 1 hit point.
    The creature uses the zombie stat block in the Monster Manual.
    It remains animate for 1 hour, after which time it collapses and dies.

    In combat, the zombie's turn comes immediately after yours.
    It obeys your mental commands, and the only action it can take is the Attack action, making one melee attack.

    You can use this feature a number of times equal to your Wisdom modifier (minimum of once), and you regain all expended uses of it when you finish a short rest.
    \paragraph{Requirements} Gannagian Tsanek subrace.
% NA'ANIAN TSANEK
\subsubsection{Benign Growths (2 FP)} \label{feat::benigngrowths}
    As part of a short rest, you can grow a fungal cornucopia on your back.
    The growths correspond to 3 doses, each of which can be consumed by expending 2 actions.
    Each growth has a random special effect, which is decided randomly by rolling a d6 during the short rest taken:
    \begin{DndTable}[width=\linewidth, header=Persona]{cX}
        \textbf{d6} & \textbf{Effect} \\
        1  & \textbf{Healing}. The creature's regains a number of hit points equal to 2d4 + your Constitution modifier. \\
        2  & \textbf{Swiftness}. The creature's walking speed increases by 3 meters for 10 minutes. \\
        3  & \textbf{Resilience}. The creature gains a +1 bonus to AC for 10 minutes. \\
        4  & \textbf{Boldness}. The creature can roll a d4 and add the number rolled to every attack roll and saving throw they make for the next minute. \\
        5  & \textbf{Sporing} The creature gains the \textbf{Pacifying Spores} trait (see page \pageref{kin::tsanek.pacifyingspores}) for 10 minutes. \\
        5  & \textbf{Melding}. The creature can meld with you on your next short rest.
        You both gain the effect associated to melding (see page \pageref{kin::tsanek.meld}).
    \end{DndTable}
    Apart from their effect, each growth comfortably feeds a creature for a day.
    \paragraph{Requirements} Na'anian Tsanek subrace.

% TORTLE
\subsubsection{Reptile Claws} \label{feat::reptileclaws}
    Your claws are natural weapons, which you can use to make unarmed strikes.
    If you hit with them, you deal slashing damage equal to 1d4 + your Strength modifier, instead of the bludgeoning damage normal for an unarmed strike.
    \paragraph{Requirements} Tortle kin.
\subsubsection{Steam Breath} \label{feat::steambreath}
    Some tortles are born with special abilities, which some say hail from the great dragon turtles of old.
    You can use two actions to exhale a 6 meter cone of scalding steam.
    Every target in the area must make a Dexterity saving throw, with a DC of 8 + your Constitution modifier.
    A creature takes 2d6 fire damage on a failed save, or half as much damage on a successful one.
    The fire damage increases to 3d6 with your 6th hit die, 4d6 with your 11th, and 5d6 with your 16th.
    You can use this trait a number of times equal to your Constitution modifier (Minimum of 1) per short rest.
    Being underwater doesn't grant resistance to this damage.

    You can take this feat 3 additional times, increasing the DC by 2 each time.
    \paragraph{Requirements} Tortle kin.
\subsubsection{Reveler (2 FP)} \label{feat::reveler}
    You increase your proficiency level in Performance or Persuasion to Legendary, increasing your proficiency bonus to +12.
    \paragraph{Requirements} Tortle kin. Expert proficiency in the chosen skill.
% Cast spells while hidden in shell (2 FP)

% GRUNG
\subsubsection{Poison Immunity} \label{feat::poisonimmunity}
    You've developed your poison resistance from constant exposure to poisons in your environment and from your own skin.
    You are immune to poison damage and the poisoned condition.
    \paragraph{Requirements} Grung kin.

% UMAN
\subsubsection{NAME} \label{feat::name}
    DESCRIPTION
    \paragraph{Requirements} Uman kin.
\subsubsection{Savage Attacks} \label{feat::savageattacks}
    When you score a critical hit with a melee weapon attack or unarmed strike, you can roll one of the weapon's damage dice one additional time and add it to the extra damage of the critical hit.
    \paragraph{Requirements} Cursed Uman or Dratl Ird subrace.
\subsubsection{Horned Fury} \label{feat::hornedfury}
    Your fury burns tirelessly.
    You gain the following benefits:
    \begin{itemize}
        \item When you hit with an attack using a melee weapon or unarmed strike, you can roll one of the weapon's damage dice an additional time and add it as extra damage of the weapon's damage type.
        Once you use this ability, you can't use it again until you finish a short rest.
        \item Immediately after you use your Relentless Endurance trait, you can use your reaction to make one weapon attack.
    \end{itemize}
    \paragraph{Requirements} Cursed Uman subrace.

% ZALOTH
\subsubsection{Cosmic Omen} \label{feat::cosmicomen}
    It is said that zaloths have a special connection to the cosmos.
    As part of a short rest, you can consult the starts for omens.
    When you do so, roll a die.
    Until you finish your next short rest, you gain access to a special reaction based on whether you rolled an even or an odd number on the die:
    \begin{itemize}
        \item \textbf{Weal (even).} Whenever a creature you can see within 9 meters of you is about to make an attack roll, a saving throw, or an ability check, you can use your reaction to roll a d6 and add the number rolled to the total.
        \item \textbf{Woe (odd).} Whenever a creature you can see within 9 meters of you is about to make an attack roll, a saving throw, or an ability check, you can use your reaction to roll a d6 and subtract the number rolled from the total.
    \end{itemize}
    You can use this ability a number of times equal to your Charisma modifier (minimum of once), and regain all expended uses on a short rest.
    \paragraph{Requirements} Zaloth kin.
\subsubsection{Telekinetic} \label{feat::telekinetic}
    You learn the mage hand (see page \pageref{spell:magehand}) cantrip.
    You can cast it without verbal or somatic components, and you can make the spectral hand invisible.
    If you already know this spell, its range increases by 9 meters when you cast it.
    Its spellcasting ability is the ability increased by this feat.

    You can use the Disarm, Grapple, Help, Search, Shove, and Use Object actions using the hand.
    For the Disarm and Shove actions, you roll Intelligence (Arcana) instead of Strength (Athletics) for the contest.

    You can learn this feat two additional times.
    The second time allows you to use the Grapple action with the hand, rolling Intelligence (Arcana) instead of Strength (Athletics).
    The third time allows you to use the Attack action with the hand, using your Arcana skill as your attack bonus.
    On a hit, the hand's damage is 1d4 + your Intelligence modifier force damage, and this attack is affected from all the normal effects of a melee attack.
    \paragraph{Requirements} Zaloth kin.
\subsubsection{Starry Form} \label{feat::starryform}
    As an action, you take on a starry form.
    While in this form, your body becomes luminous and your joints glimmer like stars.
    This form lasts until the end of your next turn.
    You become partially incorporeal, giving you resistance to bludgeoning, piercing, and slashing damage while in this form.

    You can use this action once per short rest.
\subsubsection{Zaloth Cunning (2 FP)} \label{feat::zalothcunning}
    You have advantage on Intelligence, Wisdom, and Charisma saving throws against magic.
    \paragraph{Requirements} Zaloth kin.
% GALE ZALOTH
% THUNDER ZALOTH
\subsubsection{Fast as Lightning} \label{feat::fastaslightning}
    As an action on your turn, you can teleport up to 18 meters to an unoccupied space you can see.
    Alternatively, you can use this action to teleport one willing creature you touch up to 9 meters to an unoccupied space you can see.
    This movement is accompanied by a roaring sound of thunder that can be heard up to a distance of 36 meters.

    You can use this feature a number of times equal to your Charisma modifier (Minimum of once), and you regain all expended uses of it when you finish a short rest.
    \paragraph{Requirements} Thunder Zaloth subrace.
% ASH ZALOTH
% HAIL ZALOTH

% QUIES
\subsubsection{Integrated Weapon} \label{feat::integratedweapon}
    Your understanding of your own body allows you to modify it with ease.
    As part of a short rest, you can integrate a weapon into your form.
    You can only have one weapon integrated at a time, and you can separate from it as part of a short rest.

    You can equip and unequip your integrated weapon as a free action, and don't provoke opportunity attacks when equipping it.
    When inactive, the weapon is effectively invisible, requiring no check of any kind to conceal.

    You can take this feat two additional times.
    The second time gives you a +1 bonus to damage rolls made with the weapon.
    The third time allows you to make a melee attack as part of the free action used to equip your weapon.
    This attack doesn't cause a multiple attack penalty.
    \paragraph{Requirements} Quies kin.
\subsubsection{Immutable Form} \label{feat::immutableform}
    Embracing your origin, you become protected by the strange designs of the tall kin.
    You are immune to any effect that would alter your form.
    \paragraph{Requirements} Quies kin.
\subsubsection{Long-Limbed (2 FP)} \label{feat::longlimbed}
    You harness your inherited adaptability, learning how to alter your proportions at will.
    When you make a melee attack on your turn, your reach for it is 1.5 meters greater than normal.
    \paragraph{Requirements} Quies kin.
\subsubsection{Designed with a Purpose (2 FP)} \label{feat::designedwithapurpose}
    Following your creator's design, your competence at a particular skill is unparalleled.
    Your proficiency with a skill is increased to Legendary, increasing your proficiency bonus to +12 with it.
    \paragraph{Requirements} Quies Operative subrace. Expert proficiency with a skill.
\subsubsection{Thermal Manipulation} \label{feat::thermalmanipulation}
    The years of hard work have taught you how to manipulate your obsidian extensions, modifying their hardness and properties at will.
    The damage die of your unarmed strikes is increased to a d6.
    Additionally, you can change their damage type to bludgeoning, slashing, fire, or cold as a free action.
    \paragraph{Requirements} Quies Juggernaut subrace.
\subsubsection{Rugged Aspect} \label{feat::ruggedaspect}
    By manipulating the composition of your natural armor, you are able to better shift into specific environments.
    You have advantage on Stealth checks made to hide in rocky and similar terrain.
    \paragraph{Requirements} Quies Slag Worker kin.

% !TEX root = ../main.tex
\subsection*{Background Feats}
% TODO: LIST BACKGROUNDS AND THEIR ASSOCIATED FEATS. IT'LL MAKE THINGS EASIER FOR EVERYONE.

% NOTE: DESIGN RULES FOR BACKGROUND FEATS:
% Each background has 6 feats associated to it.
% One is related to their initial feat, usually related to acquiring shelter from their people, extra food from the wilds, etc.
% At least one is useful in combat.
% One is upgradeable. If related to a DC, use 12, 15, and 18, if related to a die, use d6, d8, d10, and d12.
% One required 2 FP, and is a lv 6 ability from a related class.
% Two are shared with other classes, and are considered outside of the rules previously mentioned, but must only need 1 FP.

% A
\subsubsection{All Eyes on You} \label{feat::alleyesonyou}
    Your accent, mannerisms, figures of speech, and perhaps even your appearance all mark you as foreign.
    You gain the friendly interest of scholars and others intrigued by far-removed lands, to say nothing of everyday folk who are eager to hear stories of your origin.

    You can parley this attention into access to people and places you might not otherwise have, for you and your traveling companions.
    Noble lords, scholars, and merchant princes, to name a few, might be interested in hearing about your homeland.
    \paragraph{Requirements} Outlander background.
\subsubsection{Adept Linguist} \label{feat::adeptlinguist}
    You can communicate with people who don't speak any language you know.
    You must observe the people interacting with one another for at least 1 day, after which you learn a handful of important words, expressions, and gestures --- enough to communicate on a rudimentary level.
    \paragraph{Requirements} Scholar background.
\subsubsection{Artificer's Infusion} \label{feat::artificersinfusion}
    You gain the ability to imbue mundane items with infusions.

    When you gain this feature, pick four infusions to learn, choosing from the Infusions list in page \pageref{ssec::infusions}.
    The description of each of the following infusions details the type of item that can receive it, along with whether the resulting magic item requires attunement.
    Unless an infusion's description says otherwise, you can't learn an infusion more than once.

    Whenever you finish a long rest, you can touch a nonmagical object and imbue it with one of your artificer infusions.
    An infusion works on only certain kinds of objects, as specified in the infusion's description.
    If the item requires attunement, you can attune yourself to it the instant you infuse the item.
    If you decide to attune to the item later, you must do so using the normal process for attunement (see ``Attunement'' in chapter 7 of the Dungeon Master's Guide).

    Your infusion remains in an item indefinitely, but only one object can be infused at a time.
    If you try to exceed your maximum number of infusions, the oldest infusion immediately ends, and then the new infusion applies.

    You can learn this feat 3 times.
    The second and third time, you learn an additional infusion, and increase the number of objects that can be infused at the same time by one.

    \paragraph{Requirements} Artisan background.
% B
\subsubsection{Bardic Inspiration} \label{feat::bardicinspiration}
    You can inspire others through stirring words, actions, or music.
    To do so, you use an action on your turn to choose one creature other than yourself within 18 meters of you who can hear you.
    That creature gains one Bardic Inspiration die, a d6.

    Once within the next 10 minutes, the creature can roll the die and add the number rolled to one ability check, attack roll, or saving throw it makes.
    The creature can wait until after it rolls the d20 before deciding to use the Bardic Inspiration die, but must decide before the DM says whether the roll succeeds or fails.
    Once the Bardic Inspiration die is rolled, it is lost.
    A creature can have only one Bardic Inspiration die at a time.

    You can use this feature a number of times equal to your Charisma modifier (a minimum of once).
    You regain any expended uses when you finish a short rest.

    You can take this feat additional times to improve your Bardic Inspiration die.
    On a second time it becomes a d8, on a third a d10, and on a fourth a d12.

    \paragraph{Requirements} Entertainer background.
\subsubsection{Beyond Mastery (2 FP)} \label{feat::beyondmastery}
    Practicing for your whole life, you can take your skill as an artisan further than most mortals.
    You increase your proficiency with a set of artisan's tools from Expert to Legendary, increasing your proficiency bonus to +12.
    \paragraph{Requirements} Artisan background, Expert proficiency with a set of artisan's tools.
% C
\subsubsection{Commoner's Toughness (2 FP)} \label{feat::commonerstoughness}
    Your hit point maximum increases by an amount equal to half your level (rounded up) when you gain this feat.
    Whenever you gain an odd level thereafter, your hit point maximum increases by an additional hit point.
    \paragraph{Requirements} Laborer background.
\subsubsection{Courtier} \label{feat::courtier}
    Your knowledge of how bureaucracies function lets you gain access to the records and inner workings of any noble court or government you encounter.
    You know who the movers and shakers are, whom to go to for the favors you seek, and what the current intrigues of interest in the group are.
    When encountering a new court, you need to spend at least a short rest among their ranks to map their inner workings, which you may do by performing for them or using your status to gain an invitation to their dwellings.
    \paragraph{Requirements} Entertainer or Noble background.
\subsubsection{Criminal Contact} \label{feat::criminalcontact}
    Based on your connections, you can gain a criminal contact in a settlement as part of a long rest.
    This contact is reliable and trustworthy, and acts as your liaison to a network of other criminals.
    You know how to get messages to and from your contacts, even over great distances; specifically, you know the local messengers, corrupt caravan masters, and seedy sailors who can deliver messages for you.
    \paragraph{Requirements} Criminal background.
\subsubsection{Cunning Action} \label{feat::cunningaction}
    Your quick thinking and agility allow you to move and act quickly.
    Choose an action between Dash, Disengage, or Hide.
    The chosen action only costs you one action to perform instead of two.

    You can take this feat up to three times, selecting a different action each time.
    \paragraph{Requirements} Criminal background.
% D
\subsubsection{Deep Miner} \label{feat::deepminer}
    You are used to navigating the deep places of the earth.
    You never get lost in caves or mines if you have either seen an accurate map of them or have been through them before.
    Furthermore, you are able to scrounge fresh water and food for yourself and as many as five other people each day if you are in a mine or natural caves.
    \paragraph{Requirements} Laborer background.
\subsubsection{Divine Healing (2 FP)} \label{feat::divinehealing}
    When you roll a 1 or 2 on a die related to healing or giving temporary hit points to one or more creatures, you can reroll the die and must use the new roll, even if the new is a 1 or a 2.
    Additionally, whenever you heal a creature (including yourself) you heal yourself by 1.
    \paragraph{Requirements} Acolyte background.
\subsubsection{Divine Inspiration} \label{feat::divineinspiration}
    As an action, you can touch your holy symbol, utter a prayer, and regain one expended spell slot, the level of which can be no higher than 1/5th your level, rounded up.
    You can use this feat once per short rest.
    \paragraph{Requirements} Acolyte background.
\subsubsection{Down Low} \label{feat::downlow}
    After spending a short rest in a settlement, you acquaint yourself with a network of smugglers who are willing to help you out of tight situations.
    While in a town or larger community, you and your companions can stay for free in safe houses.
    Safe houses provide a poor lifestyle.
    While staying at a safe house, you can choose to keep your presence (and that of your companions) a secret.
    \paragraph{Requirements} Criminal or Merchant background.
\subsubsection{Drunken Resilience (2 FP)} \label{feat::drunkenresilience}
    You can drink enough alcohol to knock an elephant.
    You have advantage on saving throws against poison, and you have resistance against poison damage.
    \paragraph{Requirements} Sailor background.
\subsubsection{Dual Personalities (2 FP)} \label{feat::dualpersonalities}
    By carefully managing your connections and your appearance, you have effectively created a second persona for yourself.
    Roll on the Faceless Persona table to determine your persona, or work with the DM to create a persona that's unique to your character and suits the tone of your game.

    \begin{DndTable}[width=\linewidth, header=Persona]{cX}
        \textbf{d10} & \textbf{Faceless Persona}                \\
        1  & A flamboyant spy or brigand.                       \\
        2  & The incarnation of a nation or people.             \\
        3  & A scoundrel with a masked guise.                   \\
        4  & A vengeful spirit.                                 \\
        5  & The manifestation of a deity of your faith.        \\
        6  & One whose beauty is greatly accented using makeup. \\
        7  & An impersonation of another hero.                  \\
        8  & The embodiment of a school of magic.               \\
        9  & A warrior with distinctive armor.                  \\
        10 & A disguise with animalistic or monstrous characteristics, meant to inspire fear.
    \end{DndTable}

    Most of the people who know you do so as your persona.
    Those who seek to learn more about you --- your weaknesses, your origins, your purpose --- find themselves stymied by your disguise.
    Upon donning a disguise and behaving as your persona, you are unidentifiable as your true self.
    By removing your disguise and revealing your true face, you are no longer identifiable as your persona.
    This allows you to change appearances between your two personalities as often as you wish, using one to hide the other or serve as convenient camouflage.
    However, should someone realize the connection between your persona and your true self, your deception might lose its effectiveness.
    \paragraph{Requirements} Criminal background.
% E
\subsubsection{Early Training} \label{feat::earlytraining}
    From an early age, your family procured only the best education for you.
    You learn one Fighting Style (see page \pageref{ssec::fightingstyles}) or one Casting Style (see page \pageref{ssec::castingstyle}) of your choice.
    If you already have a style, the one you choose must be different.
    \paragraph{Requirements} Noble background.
\subsubsection{Expert Stirrer} \label{feat::expertstirrer}
    Years of arguing and interacting at seas have made your insults extraordinarily effective.
    As an action, you can hurl a terrible insult to a creature within 18 meters of you who can hear you and understand your language.
    The creature must roll a DC 12 Wisdom saving throw.
    On a failure, the creature has disadvantage on attack rolls against targets other than you and can't make opportunity attacks against targets other than you.
    This effect lasts for 1 minute, until one of your companions attacks the target or affects it with a spell, or until you and the target are more than 18 meters apart.

    You can use this ability a number of times per short rest equal to your Charisma modifier (Minimum of one).

    You can learn this feat a total of three times, increasing the DC to 15 the second time and to 18 the third.
    \paragraph{Requirements} Sailor background.
% F
\subsubsection{Flash of Genius} \label{feat::flashofgenius}
    You gain the ability to come up with solutions under pressure.
    When you or another creature you can see within 9 meters of you makes an ability check or a saving throw, you can use an action or your reaction to add your Intelligence modifier to the roll.

    You can use this feature a number of times equal to your Intelligence modifier (minimum of once).
    You regain all expended uses when you finish a short rest.
    \paragraph{Requirements} Artisan or Scholar background.
% G
\subsubsection{Guidance} \label{feat::guidance}
    As an action, you give words of encouragement to one willing creature.
    Once during the next minute, the target can roll a d4 and add the number rolled to one ability check of its choice.
    It can roll the die before or after making the ability check.

    Only one creature can be affected by this ability at a time.
    \paragraph{Requirements} Acolyte background
\subsubsection{Guild Membership} \label{feat::guildmembership}
    As an known figure of your trade, you can become a member of a guild associated to it.
    This membership provides you with certain benefits.
    Your fellow guild members will provide you with lodging and food if necessary, and pay for your funeral if needed.
    In some cities and towns, a guildhall offers a central place to meet other members of your profession, which can be a good place to meet potential patrons, allies, or hirelings.

    Guilds often wield tremendous political power.
    If you are accused of a crime, your guild will support you if a good case can be made for your innocence or the crime is justifiable.
    You can also gain access to powerful political figures through the guild, if you are a member in good standing.
    Such connections might require the donation to the guild's coffers.

    To maintain your membership, you must pay dues of 5 agnomas per month to the guild.
    If you miss payments, you must make up back dues to remain in the guild's good graces.
    \paragraph{Requirements} Artisan or Merchant background.
% H
\subsubsection{Harvest the Water} \label{feat::harvestthewater}
    You gain advantage on ability checks made using fishing tackle.
    If you have access to a body of water that sustains marine life, you can maintain a moderate lifestyle while working as a fisher, and you can catch enough food to feed yourself and up to ten other people each day.
    \paragraph{Requirements} Sailor background.
\subsubsection{Historical Knowledge} \label{feat::historicalknowledge}
    When you enter a ruin or dungeon, you can correctly ascertain its original purpose and determine its builders, whether those were gats, oths, ets, thri-kreen, or some other known kin.
    In addition, you can determine the monetary value of art objects more than a century old.
    \paragraph{Requirements} Scholar background.
% I
\subsubsection{I'll Patch It!} \label{feat::illpatchit}
    Provided you have carpenter's tools and wood, you can perform repairs on a water vehicle.
    You don't need to be competent with carpenter's tools to gain this feat.
    When you use this ability, you restore a number of hit points to the hull of a water vehicle equal to twice your level.
    A vehicle cannot be patched by you in this way again until you finish a short rest.
    \paragraph{Requirements} Sailor background.
\subsubsection{Incite Respect} \label{feat::inciterespect}
    Using two actions, you present your holy symbol, and each creature of your choice that can see or hear you within 9 meters of you must succeed on a Wisdom saving throw of DC 12 or be charmed by you until the end of your next turn or until the charmed creature takes any damage.
    You can also cause any of the charmed creatures to drop what they are holding when they fail the saving throw.

    You can use this ability a number of times per short rest equal to your Wisdom modifier (Minimum of one).

    You can learn this feat a total of three times, increasing the DC to 15 the second time and to 18 the third.
    \paragraph{Requirements} Acolyte background
\subsubsection{Inpiring Leader} \label{feat::inspiringleader}
    You can spend 10 minutes inspiring your companions, shoring up their resolve to fight.
    When you do so, choose up to six friendly creatures (which can include yourself) within 9 meters of you who can see or hear you and who can understand you.
    Each creature can gain temporary hit points equal to your level + your Charisma modifier.
    A creature can't gain temporary hit points from this feat again until it has finished a short rest.
    \paragraph{Requirements} Acolyte or Soldier background.
% \subsubsection{Inheritance} \label{feat::inheritance}
%     Choose or randomly determine your inheritance from the possibilities in the table below.
%     Work with your Dungeon Master to come up with details: Why is your inheritance so important, and what is its full story?
%     You might prefer for the DM to invent these details as part of the game, allowing you to learn more about your inheritance as your character does.
%
%     The Dungeon Master is free to use your inheritance as a story hook, sending you on quests to learn more about its history or true nature, or confronting you with foes who want to claim it for themselves or prevent you from learning what you seek.
%     The DM also determines the properties of your inheritance and how they figure into the item's history and importance.
%     For instance, the object might be a minor magic item, or one that begins with a modest ability and increases in potency with the passage of time.
%     Or, the true nature of your inheritance might not be apparent at first and is revealed only when certain conditions are met.
%
%     You can decide whether or when to tell your companions about your inheritance.
%     Rather than attracting attention to yourself, you might want to keep your inheritance a secret until you learn more about what it means to you and what it can do for you.
%
%     \begin{DndTable}[width=\linewidth, header=Inheritance]{cX}
%         \textbf{d8} & \textbf{Object or item}                                  \\
%         1   & A document such as a map, a letter, or a journal                 \\
%         2-3 & a trinket (see "Trinkets" in chapter 5 of the Player's Handbook) \\
%         4   & an article of clothing                                           \\
%         5   & a piece of jewelry                                               \\
%         6   & an arcane book or formulary                                      \\
%         7   & a written story, song, poem, or secret                           \\
%         8   & a tattoo or other body marking
%     \end{DndTable}
%     \paragraph{Requirements} Noble background.
% J
\subsubsection{Jack of All Trades (2 FP)} \label{feat::jackofalltrades}
    You can add a bonus equal to 1/10th of your level, rounded up, to any ability check with which you are not already proficient.
    \paragraph{Requirements} Entertainer background.
% K
\subsubsection{Kept in Style (2 FP)} \label{feat::keptinstyle}
    Your family is known across the realm, and your attitude reflects this.
    Your name and signet are sufficient to cover most of your expenses, since everyone would prefer to be in your family's good graces.

    This advantage enables you and your companions to receive most services for free, as long as the cost doesn't rise above 10 agnomas.
    For services up to 50 agnomas, roll a Charisma (Persuasion) check contested by the target's Wisdom (Insight).
    On a success, you can enjoy the service for free.

    At the DM discretion, this feat may be used for more expensive or rarer services, perhaps with disadvantage on the roll or by promising a reward to the service provider.
    \paragraph{Requirements} Noble background.
\subsubsection{Know your Enemy} \label{feat::knowyourenemy}
    If you spend at least 1 minute observing or interacting with another creature outside combat, you can learn certain information about its capabilities compared to your own.
    The DM tells you if the creature is your equal, superior, or inferior in regard to two of the following characteristics of your choice:
    \begin{itemize}
        \item Strength score
        \item Dexterity score
        \item Constitution score
        \item Armor Class
        \item Current hit points
        \item Total levels
    \end{itemize}
    \paragraph{Requirements} Soldier background.
\subsubsection{Known Crafter} \label{feat::knowncrafter}
    Knowing the local trade like the back of your hand, you know where to find the best ingredients for the cheapest prices in a settlement in which you are familiar.
    You can buy components and ingredients related to your craft for half their normal price, and you can tell the quality of a raw material just by looking at it.
    To use this feat you must first gain familiarity with a settlement as part of a long rest.
    \paragraph{Requirements} Artisan background.
% L
\subsubsection{Land's Stride (2 FP)} \label{feat::landsstride}
    Moving through difficult terrain costs you no extra movement.
    You can also pass through plants without being slowed by them and without taking damage from them if they have thorns, spines, or a similar hazard.

    In addition, you have advantage on saving throws against plants that are magically created or manipulated to impede movement, such those created by the entangle spell.
    \paragraph{Requirements} Outlander background.
\subsubsection{Legalese} \label{feat::legalese}
    Your experience with your local legal system has given you a firm knowledge of the ins and outs of that system.
    Even when the law is not on your side, you can use complex terms like ``ex injuria jus non oritur'' and ``cogitationis poenam nemo patitur'' to frighten people into thinking you know what you're talking about.
    With common folks who don't know any better, you might be able to intimidate or deceive to get favors or special treatment.
    \paragraph{Requirements} Noble or Scholar background.
% \subsubsection{Library Access} \label{feat::libraryaccess}
%     You are a well-known scholar (or good enough at pretending to be one), and have free and easy access to the majority of libraries.
%     This doesn't give you access to repositories of lore that are too valuable or secret to permit anyone immediate access.
%
%     Additionally, you are likely to gain preferential treatment at libraries and universities accross Yuadrem, as professional courtesy to a fellow scholar.
%     \paragraph{Requirements} Scholar background.
% M
\subsubsection{Martial Adept} \label{feat::martialadept}
    You have martial training that allows you to perform special combat maneuvers.
    You learn one maneuver of your choice from among those available in the Combat Maneuvers section (see page \pageref{ssec::combatmaneuvers}).
    If a maneuver you use requires your target to make a saving throw to resist the maneuver's effects, the saving throw DC equals 8 + twice your Strength or Dexterity modifier (your choice).

    % You gain two maneuver points, which are added to any maneuver points you have from other sources.
    % These are used to fuel maneuvers, and are expended when you use one.
    % You regain your expended maneuver points when you finish a short rest.

    You can take this feat 3 times, learning a new maneuver the second and third time.
    \paragraph{Requirements} Soldier background.
\subsubsection{Master Negotiator (2 FP)} \label{feat::masternegotiator}
    An expert trader, haggling is your second nature.
    When you interact with a fellow merchant, you can roll a Charisma (Persuasion) check contested by the merchant's Charisma (Persuasion).
    If you succeed, the price for any item you buy from them is reduced by 25\%.
    \paragraph{Requirements} Merchant background.
% \subsubsection{Mercenary Life} \label{feat::mercenarylife}
%     You know local mercenaries as only someone who has worked with them can.
%     You are able to identify mercenary companies by their emblems, and you know a little about any such company, including who has hired them recently.
%     You can find the taverns and festhalls where mercenaries abide in any area, as long as you speak the language.
%     You can find mercenary work between adventures sufficient to maintain a comfortable lifestyle.
%     \paragraph{Requirements} Soldier background.
\subsubsection{Metamagic Adept} \label{feat::metamagicadept}
    By understanding the relation between a variety of doctrines, you've learned how to exert your will on your spells to alter how they function.
    You learn one metamagic option of your choice (see page \pageref{ssec::metamagic}).
    You can use only one metamagic option on a spell when you cast it, unless the option says otherwise.
    % You gain 2 metamagic points to spend on Metamagic (these points are added to any metamagic points you have from another source but can be used only on Metamagic).
    % You regain all spent metamagic points when you finish a short rest.

    You can take this feat 3 times, learning a new metamagic option the second and third time.
    \paragraph{Requirements} Scholar background and Spellcasting feat.
% N
\subsubsection{Nature's Heritage} \label{feat::naturesheritage}
    You are familiar enough with any wilderness area that you can find twice as much food and water as you normally would when you forage.
    \paragraph{Requirements} Outlander background.
\subsubsection{Never Tell Me the Odds} \label{feat::nevertellmetheodds}
    Odds and probability are your bread and butter.
    During downtime activities that involve games of chance or figuring odds on the best plan, you can get a solid sense of which choice is likely the best one and which opportunities seem too good to be true, at the DM's determination.
    \paragraph{Requirements} Merchant background.
% O
\subsubsection{Official Inquiry} \label{feat::officialinquiry}
    You're experienced at gaining access to people and places to get the information you need.
    Through a combination of fast-talking, determination, and official-looking documentation, you can gain access to a place or an individual related to a something you're investigating.
    Those who aren't involved in your investigation avoid impeding you or pass along your requests.
    \paragraph{Requirements} Soldier background.
% P
\subsubsection{Powerful Build} \label{feat::powerfulbuild}
    Hardened by work, you count as one size larger when determining your carrying capacity and the weight you can push, drag, or lift.
    \paragraph{Requirements} Laborer background.
\subsubsection{Promise of Status} \label{feat::promiseofstatus}
    Using two actions, you shout your family name to inspire fear from your enemies.
    Each creature that can hear and understand you within 9 meters of you must succeed on a DC 12 Charisma saving throw.
    On a failure, the creature is frightened of you until the end of your next turn.
    At the DM's discretion, a creature might alternatively cease fighting altogether or even join your cause, expecting retribution from your family afterwards.

    You can use this ability a number of times per short rest equal to your Charisma modifier (Minimum of one).

    You can learn this feat a total of three times, increasing the DC to 15 the second time and to 18 the third.
    \paragraph{Requirements} Noble background
% Q
% R
\subsubsection{Rakish Audacity} \label{feat::rakishaudacity}
    Your confidence propels you into battle.
    You can give yourself a bonus to your initiative rolls equal to your Charisma modifier.
    \paragraph{Requirements} Entertainer background.
\subsubsection{Resilient} \label{feat::resilient}
    Stout beyond belief, you are living proof of the hardiness of the common folk.
    Choose an ability score.
    You gain competence in saving throws using the chosen ability.

    You can take this feat 3 times, improving your proficiency level with the saving throws of the chosen ability each time.
    \paragraph{Requirements} Laborer background.
\subsubsection{Retainers} \label{feat::retainers}
    You gain the service of three retainers loyal to your family.
    These retainers can be attendants or messengers, and one might be a majordomo.
    One of your retainers can serve as your squire, aiding you in exchange for training on their own path to knighthood.
    Another retainer might be a groom to care for your horse or a servant who polishes your armor and helps you put it on.

    Your retainers can perform mundane tasks for you, but they do not fight for you, follow you into obviously dangerous areas (such as dungeons).
    Your retainers will leave if they are frequently endangered or abused, and it will take your family 1d6 weeks to send you a new group.
    \paragraph{Requirements} Noble background.
\subsubsection{Roving} \label{feat::roving}
    A deft explorer, your movement is unimpeded by water or mountain.
    You can take this feat three times, gaining different abilities each time:
    \begin{itemize}
        \item The first time, you gain a swimming speed equal to your walking speed.
        \item The second time, you can a climbing speed equal to your walking speed.
        \item The third time, your walking speed is increased by 5.
    \end{itemize}
    After gaining the first or second abilities from this feat, Any effect that increases your movement speed also increases your swimming and climbing speed by the same amount.
    \paragraph{Requirements} Outlander background.
\subsubsection{Rustic Hospitality} \label{feat::rustichospitality}
    You've spent your life in company of the masses, and the common folk thank you for it.
    You can find a place to hide, rest, or recuperate among other commoners, unless you have shown yourself to be a danger to them.
    They will shield you from the law or anyone else searching for you, though they will not risk their lives for you.
    For this feat to work, you need to have worked or performed in the settlement at least once.
    \paragraph{Requirements} Entertainer or Laborer background.
% S
\subsubsection{Sailor's Balance} \label{feat::sailorsbalance}
    Used to the ever-present swing of a ship, you have advantage on any ability checks and saving throws made to avoid being moved or being knocked prone.
    \paragraph{Requirements} Sailor background.
\subsubsection{Shelter of the Faithful} \label{feat::shelterofthefaithful}
    Your piety inspires the respect of those who share your faith.
    After performing a religious ceremony of your deity, you and your companions can expect to receive free healing and care at a temple, shrine, or other established presence of your faith, though you must provide any material components needed for spells.
    Those who share your religion will support you (but only you) at a modest lifestyle.

    Additionally, your devotion might rouse members from other religions to look kindly to you.
    After performing a religious ceremony or act of kindness, you can roll an Intelligence (Religion) check contested by the creature's Intelligence (Religion).
    On a success, you gain the benefits from this feat from the creature's creed.
    This check is made with advantage if yours and the target deity share tide, and automatically fails if they are enemy deities from the same pantheon.

    % You might also have ties to a specific temple dedicated to your chosen deity or pantheon, and you have a residence there.
    % This could be the temple where you used to serve, if you remain on good terms with it, or a temple where you have found a new home.
    % While near your temple, you can call upon the priests for assistance, provided the assistance you ask for is not hazardous and you remain in good standing with your temple.
    \paragraph{Requirements} Acolyte background.
\subsubsection{Silver Tongue} \label{feat::silvertongue}
    You are a master at saying the right thing at the right time.
    When you make a Charisma (Persuasion) or Charisma (Deception) check, you can treat a d20 roll of 9 or lower as a 10.
    \paragraph{Requirements} Merchant background.
\subsubsection{Skill Versatility (2 FP)} \label{feat::skillversatility}
    Your life of studies has given you the capacity to quickly learn new subjects.
    You gain two proficiency levels on a skill of your choice.
    This skill must be related to Intelligence or Wisdom.

    After a long rest, you can chance this proficiency to another skill, also related to Intelligence or Wisdom.
    \paragraph{Requirements} Scholar background.
\subsubsection{Soldier's Fortitude (2 FP)} \label{feat::soldiersfortitude}
    Whenever you take the Dodge action in combat, you can spend one hit die to heal yourself.
    Roll the die, add your Constitution modifier, and regain a number of hit points equal to the total (minimum of 1).
    \paragraph{Requirements} Soldier background.
\subsubsection{Song of Rest} \label{feat::songofrest}
    You can use soothing music, oration, or a performance to help revitalize your wounded allies during a short rest.
    If you or any friendly creatures who can see or hear your performance regain hit points by spending Hit Dice at the end of the short rest, each of those creatures regains an extra 1d6 hit points.

    The extra hit points increase when you reach certain levels: to 1d8 at 6th level, to 1d10 at 11th level, and to 1d12 at 16th level.
    \paragraph{Requirements} Entertainer background.
\subsubsection{Steady} \label{feat::steady}
    You can move twice the normal amount of time (up to 16 hours) each day before being subject to the effect of a forced march (see ``Travel Pace'' in chapter 8 of the Player's Handbook).
    \paragraph{Requirements} Outlander or Soldier background.
% T
\subsubsection{The Right Tool for the Job} \label{feat::therighttoolforthejob}
    In times of need, your vast experience allows you to improvise solutions, adapt to any adversity, and overcome insurmountable challenges.
    You learn how to produce exactly the tool you need.
    By gathering resources from any environment, you can create one set of artisan's tools with which you are already competent with.
    This creation requires 1 hour of uninterrupted work, which can coincide with a short or long rest.
    Tools conjured via this feat are flimsy at best, and no reasonable person would buy them.
    \paragraph{Requirements} Artisan background.
% U
\subsubsection{Undercity Paths} \label{feat::undercitypaths}
    You know hidden, underground pathways that you can use to bypass crowds, obstacles, and observation as you move through the city.
    When you aren't in combat, you and companions you lead can travel between any two locations in the city twice as fast as your speed would normally allow.
    The paths of the undercity are haunted by dangers that rarely brave the light of the surface world, so your journey isn't guaranteed to be safe.
    \paragraph{Requirements} Criminal background.
\subsubsection{Unsettling Words} \label{feat::unsettlingwords}
    You can spin words laced with malice that unsettle a creature and cause it to doubt itself.
    Using one action, you can choose one creature you can see within 18 meters of you.
    Roll a d6.
    The creature must subtract the number rolled from the next saving throw it makes before the start of your next turn.
    A creature can only be affected by this once per round.

    You can use this feature a number of times equal to your Charisma modifier (a minimum of once).
    You regain any expended uses when you finish a short rest.

    You can take this feat additional times to improve the die.
    On a second time it becomes a d8, on a third a d10, and on a fourth a d12.
    \paragraph{Requirements} Merchant background.
\subsubsection{Urban Infrastructure} \label{feat::urbaninfrastructure}
    You have a basic knowledge of the structure of buildings, including the stuff behind the walls.
    You can also find blueprints of a specific building in order to learn the details of its construction.
    Such blueprints might provide knowledge of entry points, structural weaknesses, or secret spaces.
    Your access to such information isn't unlimited, and obtaining it can sometimes get you in trouble with the law.
    \paragraph{Requirements} Criminal or Laborer background.
% V
% W
% \subsubsection{Watcher's Eye} \label{feat::watcherseye}
%     Your experience in enforcing the law, and dealing with lawbreakers, gives you a feel for local laws and criminals.
%     You can easily find the local outpost of the watch or a similar organization, and just as easily pick out the dens of criminal activity in a community, although you're more likely to be welcome in the former locations rather than the latter.
%     \paragraph{Requirements} Soldier background.
\subsubsection{Wisdom of the Wilds} \label{feat::wisdomofthewilds}
    Your extended travels have given you a special appreciation for the stillness of the world.
    After finishing a short rest, you gain temporary hit points equal to your level + your Wisdom modifier, and you have advantage on the first ability check you make during the day.
    \paragraph{Requirements} Sailor or Outlander background.
% X
% Y
% Z

% !TEX root = ../main.tex
\addcontentsline{toc}{section}{Skill Feats}
\subsection*{Skill Feats}
% STRENGTH
\subsubsection{NAME} \label{feat::name}
    DESCRIPTION
    \paragraph{Requirements} Strength 12.
\subsubsection{NAME} \label{feat::name}
    DESCRIPTION
    \paragraph{Requirements} Strength 15.
\subsubsection{NAME} \label{feat::name}
    DESCRIPTION
    \paragraph{Requirements} Strength 18.
\subsubsection{NAME (2 FP)} \label{feat::name}
    DESCRIPTION
    \paragraph{Requirements} Strength 21.
% DEXTERITY
\subsubsection{NAME} \label{feat::name}
    DESCRIPTION
    \paragraph{Requirements} Dexterity 12.
\subsubsection{NAME} \label{feat::name}
    DESCRIPTION
    \paragraph{Requirements} Dexterity 15.
\subsubsection{NAME} \label{feat::name}
    DESCRIPTION
    \paragraph{Requirements} Dexterity 18.
\subsubsection{NAME (2 FP)} \label{feat::name}
    DESCRIPTION
    \paragraph{Requirements} Dexterity 21.
% CONSTITUTION
\subsubsection{NAME} \label{feat::name}
    DESCRIPTION
    \paragraph{Requirements} Constitution 12.
\subsubsection{NAME} \label{feat::name}
    DESCRIPTION
    \paragraph{Requirements} Constitution 15.
\subsubsection{Hit die Improvement} \label{feat::hitdieimprovement}
    You increase your hit die from a d6 to a d8, a d8 to a d10, or a d10 to a d12.
    Additionally, you gain a number of hit points equal to your level when you take this feat.
    \paragraph{Requirements} Constitution 18.
\subsubsection{Cling to Life (2 FP)} \label{feat::clingtolife}
    Resilient even in the face of death, you roll death saving throws with advantage.

    In addition, if you are attacked while unconscious roll a Constitution saving throw with a DC equal to 10 or half the damage done.
    If you succeed, you don't suffer a death saving throw failure.
    \paragraph{Requirements} Constitution 21.
% WISDOM
\subsubsection{NAME} \label{feat::name}
    DESCRIPTION
    \paragraph{Requirements} Wisdom 12.
\subsubsection{NAME} \label{feat::name}
    DESCRIPTION
    \paragraph{Requirements} Wisdom 15.
\subsubsection{NAME} \label{feat::name}
    DESCRIPTION
    \paragraph{Requirements} Wisdom 18.
\subsubsection{NAME (2 FP)} \label{feat::name}
    DESCRIPTION
    \paragraph{Requirements} Wisdom 21.
% INTELLIGENCE
\subsubsection{NAME} \label{feat::name}
    DESCRIPTION
    \paragraph{Requirements} Intelligence 12.
\subsubsection{NAME} \label{feat::name}
    DESCRIPTION
    \paragraph{Requirements} Intelligence 15.
\subsubsection{NAME} \label{feat::name}
    DESCRIPTION
    \paragraph{Requirements} Intelligence 18.
\subsubsection{NAME (2 FP)} \label{feat::name}
    DESCRIPTION
    \paragraph{Requirements} Intelligence 21.
% CHARISMA
\subsubsection{NAME} \label{feat::name}
    DESCRIPTION
    \paragraph{Requirements} Charisma 12.
\subsubsection{NAME} \label{feat::name}
    DESCRIPTION
    \paragraph{Requirements} Charisma 15.
\subsubsection{NAME} \label{feat::name}
    DESCRIPTION
    \paragraph{Requirements} Charisma 18.
\subsubsection{NAME (2 FP)} \label{feat::name}
    DESCRIPTION
    \paragraph{Requirements} Charisma 21.
% ATHLETICS alwaysready athlete jogger robust strongbody
\subsubsection{Athlete} \label{feat::athlete}
    You increase your proficiency level in the Athletics skill.
    This feat can be taken three times, or until you reach Expert proficiency on the skill.
\subsubsection{Always Ready} \label{feat::alwaysready}
    Your physical prowess is a marvel to most.
    When you are prone, standing up doesn't provoke attacks of opportunity.
    \paragraph{Requirements} Competent proficiency in the Athletics skill.
\subsubsection{Strong Body} \label{feat::strongbody}
    You count as if you were one size larger for the purpose of determining your carrying capacity.
    \paragraph{Requirements} Skilled proficiency in the Athletics skill.
\subsubsection{Jogger} \label{feat::jogger}
    Your movement speed is increased by 1.5 meters.
    You can take this feat three times, increasing your speed by the same amount each time.

    The second time you take this feat you also gain a swimming speed equal to your moving speed.

    The third time you gain the ability to move twice the normal amount of time (up to 16 hours) each day before being subject to the effect of a forced march (see ``Travel Pace'' in chapter 8 of the Player's Handbook).
    \paragraph{Requirements} Skilled proficiency in the Athletics skill.
\subsubsection{Robust (2 FP)} \label{feat::robust}
    Your hit point maximum increases by an amount equal to your level when you take this feat.
    Whenever you gain a level thereafter, your hit point maximum increases by an additional hit point.
    \paragraph{Requirements} Expert proficiency in the Athletics skill.
% ACROBATICS acrobat aerialist allterrain fleetfooted mobile
\subsubsection{Acrobat} \label{feat::acrobat}
    You increase your proficiency level in the Acrobatics skill.
    This feat can be taken three times, or until you reach Expert proficiency on the skill.
\subsubsection{Aerialist} \label{feat::aerialist}
    You've had an exceptional flexibility from a very young age.
    You can make a running long jump or a running high jump after moving only 1.5 meters or foot, rather than 3 meters.
    \paragraph{Requirements} Competent proficiency in the Acrobatics skill.
\subsubsection{All-Terrain} \label{feat::allterrain}
    Moving through difficult terrain costs you no extra movement.
    \paragraph{Requirements} Skilled proficiency in the Acrobatics skill.
\subsubsection{Mobile} \label{feat::mobile}
    Your movement speed is increased by 1.5 meters.
    You can take this feat three times, increasing your speed by the same amount each time.

    The second time you take this feat you also gain a climbing speed equal to your moving speed.

    The third time you gain the ability to run along vertical surfaces, falling to the ground if you don't finish your move in a horizontal surface.
    \paragraph{Requirements} Skilled proficiency in the Acrobatics skill.
\subsubsection{Fleet-Footed (2 FP)} \label{feat::fleetfooted}
    You reduce the cost of the Disengage action to one action.
    In addition, opportunity attacks made against you are made with disadvantage.
    % In addition, if an attack of opportunity against you misses, you can use your reaction to make a melee attack against the attacker.
    \paragraph{Requirements} Expert proficiency in the Acrobatics skill.
% SLEIGHT OF HAND fasthands quickfingers phantom pickpocket thief
\subsubsection{Quick Fingers} \label{feat::quickfingers}
    You increase your proficiency level in the Sleight of Hand skill.
    This feat can be taken three times, or until you reach Expert proficiency on the skill.
\subsubsection{Pickpocket} \label{feat::pickpocket}
    Other creatures have disadvantage on Intelligence (Investigation) and Wisdom (Perception) checks to see you stealing or notice an object you have stolen.
    \paragraph{Requirements} Competent proficiency in the Sleight of Hand skill.
\subsubsection{Thief} \label{feat::thief}
    While they've given you problems in the past, your quick and sticky fingers have awarded you with many a treasure thorough your life.
    You learn the Steal action (see page \pageref{act::steal}).
    \paragraph{Requirements} Skilled proficiency in the Sleight of Hand skill.
\subsubsection{Fast Hands} \label{feat::fasthands}
    You can learn this feat three times:
    \begin{itemize}
        \item The first, you reduce the action cost of the Use and Object action to one.
        \item The second, you reduce the action cost of the Search action to one.
        \item The last, you reduce the action cost of the Steal action to one.
    \end{itemize}
    \paragraph{Requirements} Skilled proficiency in the Sleight of Hand skill.
\subsubsection{Phantom (2 FP)} \label{feat::phantom} % NOTE: Consider renaming at some point.
    You have taken the time to hone your larcenous skills, making you an accomplished thief.
    If you spend at least one minute observing or interacting with another creature outside combat, you learn about anything valuable the creature has that can be stolen.
    In addition, any Dexterity (Sleight of Hand) checks you make to steal from the creature are made with advantage.
    \paragraph{Requirements} Expert proficiency in the Sleight of Hand skill.
% STEALTH ghost hiddenstriker skulker sly sneak
\subsubsection{Sneak} \label{feat::sneak}
    You increase your proficiency level in the Stealth skill.
    This feat can be taken three times, or until you reach Expert proficiency on the skill.
\subsubsection{Sly} \label{feat::sly}
    One with shadows, you can always escape from the most unfavorable situations.
    You have a natural ease melding with the dark, and intuitively know how to avoid making any noise.
    You have advantage on any Dexterity (Stealth) check if you move no more than half your speed on the same turn.
    \paragraph{Requirements} Competent proficiency in the Stealth skill.
\subsubsection{Skulker} \label{feat::skulker}
    You are an expert at slinking through shadows.
    Missing a ranged attack against a target doesn't alert your target.
    In addition, you can try to hide when you are lightly obscured from the creature from which you are hiding.
    \paragraph{Requirements} Skilled proficiency in the Stealth skill.
\subsubsection{Hidden Striker} \label{feat::hiddenstriker}
    You learn or improve the Sneak Attack action (see page \pageref{act:sneakattack}).
    You can take this feat three times, improving the action's damage by a d6 each time.
    \paragraph{Requirements} Skilled proficiency in the Stealth skill.
\subsubsection{Ghost (2 FP)} \label{feat::ghost}
    You reduce the action cost of the Hide action to one action.
    In addition, when you make a ranged attack against a creature while hidden, roll Dexterity (Stealth) contested against the creaute's Wisdom (Perception).
    If you win, the attack doesn't immediately reveal your position.
    \paragraph{Requirements} Expert proficiency in the Stealth skill.
% ARCANA
\subsubsection{Arcanist} \label{feat::arcanist}
    You increase your proficiency level in the Arcana skill.
    This feat can be taken three times, or until you reach Expert proficiency on the skill.
\subsubsection{NAME} \label{feat::name}
    DESCRIPTION
    \paragraph{Requirements} Competent proficiency in the Arcana skill.
\subsubsection{NAME} \label{feat::name}
    DESCRIPTION
    \paragraph{Requirements} Skilled proficiency in the Arcana skill.
\subsubsection{NAME} \label{feat::name}
    DESCRIPTION - UPGRADABLE
    \paragraph{Requirements} Skilled proficiency in the Arcana skill.
\subsubsection{NAME (2 FP)} \label{feat::name}
    DESCRIPTION
    \paragraph{Requirements} Expert proficiency in the Arcana skill.
% HISTORY advisor cognizant historian sharpenedmind wizenedadvice
\subsubsection{Historian} \label{feat::historian}
    You increase your proficiency level in the History skill.
    This feat can be taken three times, or until you reach Expert proficiency on the skill.
\subsubsection{Cognizant} \label{feat::cognizant}
    From background or perchance you've been given a privileged access to books and teaching thorough your life.
    You have advantage on Intelligence (History) checks to recall any information of your kin and your country of origin.
    \paragraph{Requirements} Competent proficiency in the History skill.
\subsubsection{Sharpened Mind} \label{feat::sharpenedmind}
    You can accurately recall anything you have seen or heard within the past month.
    \paragraph{Requirements} Skilled proficiency in the History skill.
\subsubsection{Wizened Advice} \label{feat::wizenedadvice}
    You can take this feat three times, obtaining different effects each time:
    \begin{itemize}
        \item You increase the range of the Help action to 9 meters, but you can only use the action in this way if the creature can hear and understand you.
        \item When you use the Help action to aid a creature's attack roll, you can make a DC 15 Intelligence (History) check.
        On a success, the creature gains a bonus on the attack roll equal to your proficiency bonus in History, as you share pertinent advice and historical examples.
        \item Emboldened by your words, the creature also gains a bonus equal to half your proficiency bonus in History (rounded down).
    \end{itemize}
    \paragraph{Requirements} Skilled proficiency in the History skill.
\subsubsection{Advisor (2 FP)} \label{feat::advisor}
    As an action, you can give a creature a bonus equal to your proficiency bonus with the History skill to any ability check or saving throw.
    The creature must be able to hear and understand you.

    You can use this ability a number of times equal to your Intelligence modifier (Minimum of one).
    \paragraph{Requirements} Expert proficiency in the History skill.
% INVESTIGATION detective eyeforweakness inquisitive investigative unerringeye
\subsubsection{Investigative} \label{feat::investigative}
    You increase your proficiency level in the Investigation skill.
    This feat can be taken three times, or until you reach Expert proficiency on the skill.
\subsubsection{Inquisitive} \label{feat::inquisitive}
    No detail can miss your analytical mind.
    Your extensive experience and awareness means you can always tell when something's amiss.

    Your inquisitiveness allows you to tell when you are missing a detail or clue in a situation, even after a failed roll.
    This doesn't allow you to roll again, but the lingering uneasiness stays in your mind.
    \paragraph{Requirements} Competent proficiency in the Investigation skill.
\subsubsection{Detective} \label{feat::detective}
    You reduce the action cost of the Search action to one action.
    \paragraph{Requirements} Skilled proficiency in the Investigation skill.
\subsubsection{Unerring Eye} \label{feat::unerringeye}
    You excel at rooting out secrets and unraveling mysteries.
    You can take this feat three times, obtaining different benefits each time:
    \begin{itemize}
        \item You have advantage on any Wisdom (Perception) or Intelligence (Investigation) check if you move no more than half your speed on the same turn.
        \item You gain a +5 bonus to your passive Intelligence (Investigation) score.
        \item Your senses are almost impossible to foil.
        You sense the presence of illusions, any creature not in their original form, and other magic designed to deceive the senses within 9 meters of you, provided you aren't blinded or deafened.
        You sense that an effect is attempting to trick you, but you gain no insight into what is hidden or into its true nature.
    \end{itemize}
    \paragraph{Requirements} Skilled proficiency in the Investigation skill.
\subsubsection{Eye for Weakness (2 FP)} \label{feat::eyeforweakness}
    You learn to exploit a creature's weaknesses by carefully studying its tactics and movement.
    You learn the Aim action.
    When you successfully attack a creature after using this action, you roll the weapon's damage die one additional time.
    \paragraph{Requirements} Expert proficiency in the Investigation skill.
% NATURE deftexplorer experiencedtraveler hideinplainsight naturalawareness naturalist
\subsubsection{Naturalist} \label{feat::naturalist}
    You increase your proficiency level in the Nature skill.
    This feat can be taken three times, or until you reach Expert proficiency on the skill.
\subsubsection{Deft Explorer} \label{feat::deftexplorer}
    Your appreciation for nature has led you to attain an extensive categorical knowledge of plants and animals, giving you a great feat at recognizing and identifying them.

    Provided you are familiar with the local flora and fauna from experience or a guide, you don't need to make an ability check to identify common plants, fungi, animals, and insects.
    You gain familiarity with a region by taking a long rest in it.
    \paragraph{Requirements} Competent proficiency in the Nature skill.
\subsubsection{Natural Awareness} \label{feat::naturalawareness}
    By almost supernatural perception, you can sense the presence of poisons and poisonous creatures within 9 meters of you.

    \paragraph{Requirements} Skilled proficiency in the Nature skill.
\subsubsection{Experienced Traveler} \label{feat::experiencedtraveler}
    You are an unsurpassed explorer, using your uncanny knowledge of nature to understand and survive in any environment.
    You can take this feat multiple times, obtaining different effects each time:
    \begin{itemize}
        \item Difficult Terrain doesn't slow your group's travel.
        \item When you forage, you find twice as much food as you normally would.
        \item While tracking down creatures, you also learn their exact number, their sizes, and how long ago they passed through the area.
    \end{itemize}
    \paragraph{Requirements} Skilled proficiency in the Nature skill.
\subsubsection{Hide in Plain Sight (2 FP)} \label{feat::hideinplainsight}
    While in the wilds, your proficiency bonus with Intelligence (Nature) and Wisdom (Survival) are doubled.

    In addition, you can spend 1 minute creating camouflage for yourself.
    You must have access to fresh mud, dirt, plants, soot, and other naturally occurring materials with which to create your camouflage.

    Once you are camouflaged in this way, you can try to hide by pressing yourself up against a solid surface, such as a tree or wall, that is at least as tall and wide as you are.
    You gain a +10 bonus to Dexterity (Stealth) checks as long as you remain there without moving or taking actions.
    \paragraph{Requirements} Expert proficiency in the Nature skill.
% RELIGION holyfortitude peaceofmind pious studentofthetides theologian
\subsubsection{Theologian} \label{feat::theologian}
    You increase your proficiency level in the Religion skill.
    This feat can be taken three times, or until you reach Expert proficiency on the skill.
\subsubsection{Pious} \label{feat::pious}
    You're a zealot when it comes to your faith.
    Your devotion has led to an incredible understanding of your religion and its history, and no connoted figure goes unnoticed in your prayers.
    You can recall the name and description of every deity and important figure related to a religion of your choice without needing to succeed on any ability check.

    In addition, you have advantage on Charisma (Persuasion) and Charisma (Deception) checks against creatures that share your faith.
    \paragraph{Requirements} Competent proficiency in the Religion skill.
\subsubsection{Peace of Mind} \label{feat::peaceofmind}
    By sharing a prayer with a creature during a minute, you both reduce your stress levels by 1.
    If you and the creature share the same religion, the stress level reduction is increased to 2.
    You can use this ability once per short rest.
    \paragraph{Requirements} Skilled proficiency in the Religion skill.
\subsubsection{Student of the Tides} \label{feat::studentofthetides}
    You can take this feat three times, gaining different benefits each time:
    \begin{itemize}
        \item You can make an ability check with your Intelligence (Religion) modifier contested against a creature's Charisma (Deception) to tell its tidal alignment.
        You must be able to see the creature to use this ability.
        \item You have advantage on Charisma checks against creatures that shares your tidal alignment, and roll on any attempt made to charm such creature with advantage.
        You must know the tidal alignment of the creature to get this benefit.
        \item By knowing its tidal alignment you have a basic understanding of a creature's goals and ideals.
        As an action, you can roll a Charisma (Persuasion) or Intelligence (Religion) check contested with the creature's Wisdom (Insight).
        On a failure, the creature is charmed by you for a minute.
        This charm ends early if you or your companions attack the creature.
    \end{itemize}
    \paragraph{Requirements} Skilled proficiency in the Religion skill.
\subsubsection{Holy Fortitude (2 FP)} \label{feat::holyfortitude}
    Your piousness provides you with an extreme mental fortitude.
    You roll Constitution saving throws made to maintain concentration on a spell with advantage.

    In addition, you can naturally tell when you are affected by an illusion, but don't know the exact nature of it.
    You have advantage on any Intelligence (Investigation) checks made to identify the illusion.
    \paragraph{Requirements} Expert proficiency in the Religion skill.
% ANIMAL HANDLING animalcompanion animalhandler sympathetic tamer zoophilist
\subsubsection{Animal Handler} \label{feat::animalhandler}
    You increase your proficiency level in the Animal Handling skill.
    This feat can be taken three times, or until you reach Expert proficiency on the skill.
\subsubsection{Sympathetic} \label{feat::sympathetic}
    You can intuit what an animal is feeling and what its intentions are without making a Wisdom (Animal Handling) check.
    \paragraph{Requirements} Competent proficiency in the Animal Handling skill.
\subsubsection{Zoophilist} \label{feat::zoophilist}
    Ever since you were a kid, you've always surrounded yourself with animals, pets or otherwise.
    Animals are calmer when they're around you, and you yourself get the same feeling when around them.

    Animals naturally trust you, and you don't need to perform any check to calm an animal or monster that is not already violent toward you.
    \paragraph{Requirements} Skilled proficiency in the Animal Handling skill.
\subsubsection{Tamer} \label{feat::tamer}
    You can learn this feat three times, gaining different abilities each time:
    \begin{itemize}
        \item Using two actions, you can issue a command to a beast within 18 meters of you that can hear you and you haven't attacked or damaged in any way.
        Make a Wisdom (Animal Handling) check contested by the creature's Wisdom (Insight).
        If you succeed, you can issue a general command that the creature will follow during its next turn.
        \item You gain an increased insight when issuing commands to creatures, and decide the actions that the creature will take during its next turn.
        \item After succeeding on the ability check, you can issue subsequent commands to the creature for the next minute by expending one action per turn.
        The effect ends if you or your companions attack the creature.
        You cannot have more than one creature affected by this ability at once.
    \end{itemize}
    \paragraph{Requirements} Skilled proficiency in the Animal Handling skill.
\subsubsection{Animal Companion (2 FP)} \label{feat::animalcompanion}
    You can try to bond with a wild creature that is not violent towards you and has a number of hit dice equal or lower than yours.
    Make a Wisdom (Animal Handling) check contested by the creature's Wisdom (Insight).
    The creature's roll gains a bonus equal to half its number of hit dice (rounded down).
    If you succeed, you can spend 8 hours bonding with the beast, after which it becomes your animal companion.

    The beast rolls initiative as normal and acts on its own volition during its turns.
    On its turns, it will defend itself and you against aggressors.
    You can issue commands to it as an action without needing to make an ability check, and when you do so you can control the beast during its next turn.
    It never requires your command to use its reaction, such as when making an opportunity attack.

    If you are incapacitated or absent, your beast companion will act on its own, focusing on protecting you and itself.

    If you fail on the check, you cannot try to bond with the same beast again.
    \paragraph{Requirements} Expert proficiency in the Animal Handling skill.
% INSIGHT awesomealertness incredibleintuition insightful keenmind uncannyinsight
\subsubsection{Insightful} \label{feat::insightful}
    You increase your proficiency level in the Insight skill.
    This feat can be taken three times, or until you reach Expert proficiency on the skill.
\subsubsection{Keen Mind} \label{feat::keenmind}
    You always know which way is north, and you always know the number of hours left before the next sunrise or sunset.
    \paragraph{Requirements} Competent proficiency in the Insight skill.
\subsubsection{Uncanny Understanding} \label{feat::uncannyinsight}
    You can use one action to try to get uncanny insight about one humanoid you can see within 9 meters of you.
    Make a Wisdom (Insight) check contested by the target's Charisma (Deception) check.
    If your check succeeds, you have advantage on attack rolls and ability checks against the target until the end of your next turn.
    \paragraph{Requirements} Skilled proficiency in the Insight skill.
\subsubsection{Incredible Intuition} \label{feat::incredibleintuition}
    Either by primal intuition or extensive knowledge, you are always in complete awareness of your surroundings.
    You can learn this feat three times, gaining different effects each time:
    \begin{itemize}
        \item You have advantage on any Wisdom (Perception) or Intelligence (Investigation) checks made to detect the presence of secret doors and traps.
        \item You gain a +5 bonus to your passive Wisdom (Insight) score.
        \item It is nigh-impossible to lie to you.
        You make Wisdom (Insight) checks to determine whether a creature is lying with advantage, and you can treat a roll of 9 or lower on the d20 as a 10.
    \end{itemize}
    \paragraph{Requirements} Skilled proficiency in the Insight skill.
\subsubsection{Awesome Alertness (2 FP)} \label{feat::awesomealertness}
    You are permanently aware of danger, and have advantage on initiative rolls.

    In addition, you and any of your companions within 9 meters of you can't be surprised, except when incapacitated by something other than nonmagical sleep.
    You instantly awaken if you are sleeping naturally when combat begins.
    \paragraph{Requirements} Expert proficiency in the Insight skill.
% MEDICINE guardianangel healersinsight medic refinedhealing restoringrest
\subsubsection{Medic} \label{feat::medic}
    You increase your proficiency level in the Medicine skill.
    This feat can be taken three times, or until you reach Expert proficiency on the skill.
\subsubsection{Guardian Angel} \label{feat::guardianangel}
    When you successfully stabilize a dying creature, that creature also regains 1 hit point.
    \paragraph{Requirements} Competent proficiency in the Medicine skill.
\subsubsection{Healer's Insight} \label{feat::healersinsight}
    You can tell when a creature is suffering from a wound or injury without rolling an ability check.
    \paragraph{Requirements} Skilled proficiency in the Medicine skill.
\subsubsection{Restoring Rest} \label{feat::restoringrest}
    During a short rest, you can clean and bind the wounds of up to six willing beasts and humanoids (including yourself).
    On a success, if a creature spends a hit die during this rest, that creature can forgo the roll and instead regain the maximum number of hit points the die can restore.
    A creature can do so only for one die per rest, regardless of how many hit dice it spends.

    You can take this feat two more times, increasing the number of hit dice affected by this ability by one each time.
    \paragraph{Requirements} Skilled proficiency in the Medicine skill.
\subsubsection{Refined Healing (2 FP)} \label{feat::refinedhealing}
    You mastered the physician's arts, and are able to heal any wound or bruise with ease.
    When you heal a creature by any means, the creature regains additional hit points equal to your proficiency bonus with the Medicine skill.
    \paragraph{Requirements} Expert proficiency in the Medicine skill.
% PERCEPTION
\subsubsection{Perceptive} \label{feat::perceptive}
    You increase your proficiency level in the Perception skill.
    This feat can be taken three times, or until you reach Expert proficiency on the skill.
\subsubsection{NAME} \label{feat::name}
    DESCRIPTION
    \paragraph{Requirements} Competent proficiency in the Perception skill.
\subsubsection{NAME} \label{feat::name}
    DESCRIPTION
    \paragraph{Requirements} Skilled proficiency in the Perception skill.
\subsubsection{NAME} \label{feat::name}
    DESCRIPTION - UPGRADABLE
    \paragraph{Requirements} Skilled proficiency in the Perception skill.
\subsubsection{NAME (2 FP)} \label{feat::name}
    DESCRIPTION
    \paragraph{Requirements} Expert proficiency in the Perception skill.
% SURVIVAL
\subsubsection{Survival Proficiency} \label{feat::survivalprof}
    You increase your proficiency level in the Survival skill.
    This feat can be taken three times, or until you reach Expert proficiency on the skill.
\subsubsection{Always on Track} \label{feat::alwaysontrack}
    You and your group can't become lost except by magical means.
    \paragraph{Requirements} Competent proficiency in the Survival skill.
\subsubsection{NAME} \label{feat::name}
    DESCRIPTION
    \paragraph{Requirements} Skilled proficiency in the Survival skill.
\subsubsection{NAME} \label{feat::name}
    DESCRIPTION - UPGRADABLE
    \paragraph{Requirements} Skilled proficiency in the Survival skill.
\subsubsection{NAME (2 FP)} \label{feat::name}
    DESCRIPTION
    \paragraph{Requirements} Expert proficiency in the Survival skill.
% DECEPTION
\subsubsection{Deception Proficiency} \label{feat::deceptionprof}
    You increase your proficiency level in the Deception skill.
    This feat can be taken three times, or until you reach Expert proficiency on the skill.
\subsubsection{NAME} \label{feat::name}
    DESCRIPTION
    \paragraph{Requirements} Competent proficiency in the Deception skill.
\subsubsection{NAME} \label{feat::name}
    DESCRIPTION
    \paragraph{Requirements} Skilled proficiency in the Deception skill.
\subsubsection{NAME} \label{feat::name}
    DESCRIPTION - UPGRADABLE
    \paragraph{Requirements} Skilled proficiency in the Deception skill.
\subsubsection{NAME (2 FP)} \label{feat::name}
    DESCRIPTION
    \paragraph{Requirements} Expert proficiency in the Deception skill.
% INTIMIDATION
\subsubsection{Intimidation Proficiency} \label{feat::intimidationprof}
    You increase your proficiency level in the Intimidation skill.
    This feat can be taken three times, or until you reach Expert proficiency on the skill.
\subsubsection{NAME} \label{feat::name}
    DESCRIPTION
    \paragraph{Requirements} Competent proficiency in the Intimidation skill.
\subsubsection{NAME} \label{feat::name}
    DESCRIPTION
    \paragraph{Requirements} Skilled proficiency in the Intimidation skill.
\subsubsection{NAME} \label{feat::name}
    DESCRIPTION - UPGRADABLE
    \paragraph{Requirements} Skilled proficiency in the Intimidation skill.
\subsubsection{NAME (2 FP)} \label{feat::name}
    DESCRIPTION
    \paragraph{Requirements} Expert proficiency in the Intimidation skill.
% PERFORMANCE
\subsubsection{Performance Proficiency} \label{feat::performanceprof}
    You increase your proficiency level in the Performance skill.
    This feat can be taken three times, or until you reach Expert proficiency on the skill.
\subsubsection{NAME} \label{feat::name}
    DESCRIPTION
    \paragraph{Requirements} Competent proficiency in the Performance skill.
\subsubsection{NAME} \label{feat::name}
    DESCRIPTION
    \paragraph{Requirements} Skilled proficiency in the Performance skill.
\subsubsection{NAME} \label{feat::name}
    DESCRIPTION - UPGRADABLE
    \paragraph{Requirements} Skilled proficiency in the Performance skill.
\subsubsection{NAME (2 FP)} \label{feat::name}
    DESCRIPTION
    \paragraph{Requirements} Expert proficiency in the Performance skill.
% PERSUASION
\subsubsection{Persuasion Proficiency} \label{feat::persuasionprof}
    You increase your proficiency level in the Persuasion skill.
    This feat can be taken three times, or until you reach Expert proficiency on the skill.
\subsubsection{NAME} \label{feat::name}
    DESCRIPTION
    \paragraph{Requirements} Competent proficiency in the Persuasion skill.
\subsubsection{Therapist} \label{feat::therapist}
    A keen healer and convincing persuader, you can heal a creature's insanity as part of a long rest.
    The creature must succeed on a DC 12 Intelligence saving throw for the treatment to work.
    \paragraph{Requirements} Skilled proficiency in the Persuasion.
\subsubsection{NAME} \label{feat::name}
    DESCRIPTION - UPGRADABLE
    \paragraph{Requirements} Skilled proficiency in the Persuasion skill.
\subsubsection{NAME (2 FP)} \label{feat::name}
    DESCRIPTION
    \paragraph{Requirements} Expert proficiency in the Persuasion skill.

% \subsubsection{Reliever} \label{feat::reliever}
% \small{\textcolor{gray}{Prof in Persuasion}}
%
% \normalsize
% You are able to motivate and relieve the most stressed of folks.
% As part of a short rest, you can reduce an ally's stress by one point.
% You can only use this feat once per short rest, and cannot reduce your own stress via this manner.
% \paragraph{REQUIREMENTS} Expert in the Persuasion skill.
% % ========================================================================== %
%
% \subsubsection{Entertainer} \label{feat::entertainer}
% \small{\textcolor{gray}{Prof in Performance}}
%
% \normalsize
% A favorite of the masses, you are an expert at making people laugh and relax.
% During a long rest, you and your allies reduce stress by one additional point.
% \paragraph{REQUIREMENTS} Master in the Performance skill.
%
% % === SKILLS =================================================================== %
% % \subsubsection{Gifted} \label{feat::gifted} %
% % \small{\textcolor{gray}{Arcana}}
%
% % \normalsize
% % By chance or mysterious circumstance, you've gained a special attunement with the arcane arts of the world.
% % It's easy for you to perceive when magic is used around you, and your extensive studies allow you to even identify the spells being cast.
% % \paragraph{RANK 1} You are proficient with the Arcana skill.
% % \paragraph{RANK 2} You learn one harmless cantrip of your choice that doesn't belong to any magic school.
% % \paragraph{RANK 3} You double your proficiency modifier in the Arcana skill.
%
% ============================================================================== %
\subsubsection{Actor} \label{feat::actor}
\small{\textcolor{gray}{Performer}}

\normalsize
You master performance so that you can command any stage.
\paragraph{REQUIREMENTS} Performer 3.
\paragraph{RANK 1} You double your proficiency modifier in the Performance skill.
\paragraph{RANK 2} You have advantage on Charisma (Deception) and Charisma (Performance) checks when trying to pass yourself off as a different person.
\paragraph{RANK 3} Increase your Charisma score by 1, to a maximum of 20.

% ============================================================================== %
\subsubsection{Aerialist} \label{feat::aerialist}
\small{\textcolor{gray}{}}

\normalsize
High above the ground, you defy falls to earth on a daily basis.
\paragraph{REQUIREMENTS} Acrobat 3.
\paragraph{RANK 1} You double your proficiency modifier in the Acrobatics skill.
\paragraph{RANK 2} Your walking speed increases by 1.5 meters.
Additionally, you gain a climbing speed and a swimming speed equal to your walking speed.
\paragraph{RANK 3} You learn the Light as a Feather technique.

% ============================================================================== %
\subsubsection{Educated} \label{feat::educated}
\small{\textcolor{gray}{Science}}

\normalsize
From background or perchance you've been given a privileged access to books and teaching thorough your life.
You have an advanced understanding of the natural laws of the world.
\paragraph{RANK 1} You are proficient with the Science skill.
\paragraph{RANK 2} You have advantage on the Identify a Spell action.
\paragraph{RANK 3} You have an intuitive understanding of how to use all sorts of machinery, and can operate siege weapons and heavy machines without performing any ability checks.

% ============================================================================== %
\subsubsection{Fearsome} \label{feat::fearsome}
\small{\textcolor{gray}{Intimidation}}

\normalsize
You become fearsome to others, and people are scared to even talk about you.
\paragraph{REQUIREMENTS} Menacing 3.
\paragraph{RANK 1} You double your proficiency modifier in the Intimidation skill.
\paragraph{RANK 2} You learn the Demoralize technique.
\paragraph{RANK 3} Increase your Charisma score by 1, to a maximum of 20.

% ============================================================================== %
\subsubsection{Haggler} \label{feat::haggler}
\small{\textcolor{gray}{Persuasion}}

\normalsize
Your abilities at haggling and persuading are known in your local markets, and you are both feared and revered by merchants and traders.
\paragraph{REQUIREMENTS} Persuasive 3.
\paragraph{RANK 1} You double your proficiency modifier in the Persuasion skill.
\paragraph{RANK 2} Your well-honed haggling skills allow you to shave off 10\% off the price of anything, even if you fail a Charisma (Persuasion) check.
\paragraph{RANK 3} Increase your Charisma score by 1, to a maximum of 20.

% ============================================================================== %
\subsubsection{Knowledgeable} \label{feat::knowledgeable}
\small{\textcolor{gray}{Science}}

\normalsize
Your extensive knowledge in the natural sciences gives you access to a plethora of explanations to the phenomena of the world.
\paragraph{REQUIREMENTS} Educated 3.
\paragraph{RANK 1} You double your proficiency modifier in the Science skill.
\paragraph{RANK 2} You learn the Identify technique.
\paragraph{RANK 3} Increase your Intelligence score by 1, to a maximum of 20.

% ============================================================================== %
\subsubsection{Liar} \label{feat::liar}
\small{\textcolor{gray}{Deception}}

\normalsize
You can easily lie to someone to their face without batting an eye.
While you might not always be proud of this skill, you can't deny that it's saved you in countless situations.
\paragraph{RANK 1} You are proficient with the Deception skill.
\paragraph{RANK 2} When you fail to convince someone with a lie, it'll still take them one turn to realize you're lying, provided the lie isn't something absurd.
\paragraph{RANK 3} You learn the Mimic technique.

% ============================================================================== %
\subsubsection{Linguist} \label{feat::linguist}
\small{\textcolor{gray}{Languages}}

\normalsize
Your extensive knowledge in tongues is manifested in an enhanced ability recognizing languages and inventing codes of your own.
\paragraph{RANK 1} You are proficient in the Languages skill.
\paragraph{RANK 2} You have advantage on Charisma (Languages) checks to determine the language that a creature is speaking.
\paragraph{RANK 3} You can ably create ciphers.
Others can't decipher a code you create unless you teach them or they succeed on an Intelligence check (DC equal to your Intelligence score + your proficiency bonus).

% ============================================================================== %
\subsubsection{Lip-reader} \label{feat::lipreader}
\small{\textcolor{gray}{Languages}}

\normalsize
You have an uncanny capacity with languages and understanding speakers even without actually hearing them.
\paragraph{REQUIREMENTS} Linguist 3.
\paragraph{RANK 1} You double your proficiency modifier in the Languages skill.
\paragraph{RANK 2} If you see a creature's mouth while it is speaking a language you understand, you can interpret what it's saying by reading its lips.
\paragraph{RANK 3} Increase your Charisma score by 1, to a maximum of 20.

% ============================================================================== %
\subsubsection{Menacing} \label{feat::menacing}
\small{\textcolor{gray}{Intimidation}}

\normalsize
With good reason or not, people have always been a bit fearful of you.
Your menacing appearance has permitted you to bypass consequence in many situations, allowing you to lead a carefree lifestyle.
\paragraph{RANK 1} You are proficient with the Intimidation skill.
\paragraph{RANK 2} People are naturally more careful around you, and you can get away with petty crimes without repercussion from the law.
\paragraph{RANK 3} You learn the Taunt technique.

% ============================================================================== %
\subsubsection{Mobile} \label{feat::mobile}
\small{\textcolor{gray}{Acrobatics}}

\normalsize
You are exceptionally speedy and agile.
\paragraph{REQUIREMENTS} Lightly Armored 2 and Acrobat 2.
\paragraph{RANK 1} When you use the Dash action, difficult terrain doesn't cost you extra movement.
\paragraph{RANK 2} When you make a melee attack against a creature, you don't provoke opportunity attacks from that creature for the rest of the turn, whether you hit or not.
\paragraph{RANK 3} Your speed increases by 3 meters.

% ============================================================================== %
\subsubsection{Observant} \label{feat::observant}
\small{\textcolor{gray}{Perception}}

\normalsize
Easily distracted, you are always attentive of the details in your surroundings.
\paragraph{RANK 1} You are proficient with the Perception skill.
\paragraph{RANK 2} You can easily spot a person you know in a crowd or an object you're familiar with among many.
\paragraph{RANK 3} Being in a lightly obscured area doesn't impose disadvantage on your Wisdom (Perception) checks if you can both see and hear.

% ============================================================================== %
\subsubsection{Perceptive} \label{feat::perceptive}
\small{\textcolor{gray}{Perception}}

\normalsize
When you focus your mind on something, you are able to quickly perceive even the smallest details and imperfections.
\paragraph{REQUIREMENTS} Observant 3.
\paragraph{RANK 1} You double your proficiency modifier in the Perception skill.
\paragraph{RANK 2} You have a +5 bonus to your passive Wisdom (Perception) score.
\paragraph{RANK 3} Increase your Wisdom score by 1, to a maximum of 20.

% ============================================================================== %
\subsubsection{Performer} \label{feat::performer}
\small{\textcolor{gray}{Performance}}

\normalsize
A natural performer, you've never failed to impress with your refined acting skills.
You have an easy time imitating the demeanor of others, especially of those that you know well.
\paragraph{RANK 1} You are proficient with the Performance skill.
\paragraph{RANK 2} You have advantage on ability checks trying to pass off as another member of your kin, even without a disguise.
\paragraph{RANK 3} You learn the Distract technique.

% ============================================================================== %
\subsubsection{Persuasive} \label{feat::persuasive}
\small{\textcolor{gray}{Persuasion}}

\normalsize
A skilled negotiator and a master of diplomacy, you have an easy time convincing people to do what you need.
\paragraph{RANK 1} You are proficient with the Persuasion skill.
\paragraph{RANK 2} You have advantage on Charisma (Persuasion) checks when trading with a creature.
\paragraph{RANK 3} You learn the Charm technique.

% ============================================================================== %
\subsubsection{Silver-tongued} \label{feat::silvertongued}
\small{\textcolor{gray}{Deception}}

\normalsize
You develop your conversational skill to better deceive others.
\paragraph{REQUIREMENTS} Liar 3.
\paragraph{RANK 1} You double your proficiency modifier in the Deception skill.
\paragraph{RANK 2} You learn the Deceive technique.
\paragraph{RANK 3} Increase your Charisma score by 1, to a maximum of 20.

\subsubsection{Survivor} \label{feat::survivor}
\small{\textcolor{gray}{Survival}}

\normalsize
As if raised by wolves, you are a natural survivor in the most harsh of circumstances.
Your enviable endurance and experience allows you to be particularly effective at building shelter, purifying water, and foraging food.
\paragraph{RANK 1} You are proficient with the Survival skill.
\paragraph{RANK 2} When you forage, you find twice as much food as you normally would.
\paragraph{RANK 3} You have advantage on Wisdom (Survival) checks made to track creatures.

% ============================================================================== %
\subsubsection{Tracker} \label{feat::tracker}
\small{\textcolor{gray}{Survival}}

\normalsize
You have spent enough time hunting to hone your skill to a venerable level.
\paragraph{REQUIREMENTS} Survivor 3.
\paragraph{RANK 1} You double your proficiency modifier in the Survival skill.
\paragraph{RANK 2} You learn the Mark technique.
\paragraph{RANK 3} Increase your Wisdom score by 1, to a maximum of 20.

% !TEX root = ../main.tex
\addcontentsline{toc}{section}{Proficiency Feats}
\subsection*{Proficiency Feats}
% NOTE: DIVIDED INTO THRRE, TOOL PROFICIENCIES, LAND VEHICLES, AND KITS.

\subsection*{Tool Proficiencies}
    You can use a set of Artisan's Tools to craft items as part of a long rest.
    Make an ordered list of the items you want to make.
    The items you have access to depend on your proficiency level with the tools, and are detailed in each tools' associated proficiency feat.
    Each item in the list has an associated DC and cost, which is equal to its DC and cost plus these values for all the previous items on the list.

    The DC of an item depends on its rarity and its associated proficiency.
    They are listed in the tools' proficiency feat.
    % Common items have a DC of 6, Uncommon items have a DC of 9, Rare items a DC of 12, Very Rare a DC of 15, and legendary a DC of 20.

    After you finish the list, make an ability check with the relevant toolset.
    The number you roll determines how far down the list you are able to craft, and the cost you need to pay for components.

% ALCHEMIST'S SUPPLIES alchemist experimentalelixir fromgreattobest potionperuser restorativereagents
\subsubsection{Alchemist} \label{feat::alchemist}
    Increase your level of proficiency with Alchemist's Supplies.
    This feat can be re-taken until you are an expert with the toolset.

    You add your Intelligence modifier to checks with Alchemists' Supplies.

    From mastery with these tools, you have access to all items in the Potions section (see page \pageref{ssec::potions}).
    The cost associated to each item only accounts for the required reagents, not the vial or flask used to contain it.

    Common potions have a DC of 6 to make, Uncommon potions a DC of 9, Rare potions a DC of 12, Very Rare a DC of 15, and Legendary a DC of 18.
    The cost and DC of potions relates to one vial or flask of the substance, as is detailed in the corresponding section.
\subsubsection{Potion Peruser} \label{feat::potionperuser}
    Using two actions, you can identify one potion within 1 meters of you, as if you had tasted it.
    You must see the liquid for this benefit to work.
    \paragraph{Requirements} Competent proficiency with Alchemist's Supplies.
\subsubsection{Restorative Reagents} \label{feat::restorativereagents}
    Whenever a creature drinks a potion you created, the creature gains temporary hit points equal to 2d6 + your proficiency bonus with Alchemist's Supplies.
    \paragraph{Requirements} Skilled proficiency with Alchemist's Supplies.
\subsubsection{From Great to Best} \label{feat::fromgreattobest}
    Over the course of a short rest, you can temporarily improve the potency of one potion of any rarity.
    % To use this benefit, you must have alchemist's supplies with you, and the potion must be within reach.
    If the potion is drunk during the day after the short rest ends, ignore the potion's die roll (if it has any), automatically rolling the maximum possible number.

    You can take this feat three times, increasing the number of potions you can improve as part of the short rest by one each time.
    \paragraph{Requirements} Skilled proficiency with Alchemist's Supplies.
\subsubsection{Experimental Elixir (2 FP)} \label{feat::experimentalelixir}
    Whenever you finish a short rest, you can produce two experimental elixirs in an empty flask you touch.
    Roll on the Experimental Elixir table for the elixir's effect, which is triggered when someone drinks the elixir.
    Using two actions, a creature can drink the elixir or administer it to an incapacitated creature.

    Creating an experimental elixir requires you to have alchemist's supplies on your person, and any elixir you create with this feature lasts until it is drunk or until the end of your next short rest.

    If you gain Legendary proficiency with Alchemist's Supplies, can make one more elixir with this ability.

    \begin{DndTable}[width=\linewidth, header=Experimental Elixir]{lX}
        \textbf{d6} & \textbf{Effect} \\
        1 & \textbf{Healing.}
        The drinker regains a number of hit points equal to 2d4 + half your proficiency bonus with Alchemist's Supplies. \\
        2 & \textbf{Swiftness.}
        The drinker's walking speed increases by 2 meters for 1 hour. \\
        3 & \textbf{Resilience.}
        The drinker gains a +1 bonus to AC for 10 minutes. \\
        4 & \textbf{Boldness.}
        The drinker can roll a d4 and add the number rolled to every attack roll and saving throw they make for the next minute. \\
        5 & \textbf{Flight.}
        The drinker gains a flying speed of 2 meters for 10 minutes. \\
        6 & \textbf{Transformation.}
        The drinker's body is transformed as if by the alter self spell (see page \pageref{spell::alterself}).
        The drinker determines the transformation caused by the spell, the effects of which last for 10 minutes.
    \end{DndTable}
    \paragraph{Requirements} Expert proficiency with Alchemist's Supplies.

% BREWER'S SUPPLIES brewer chug mixeddrinks purify theenhancer
\subsubsection{Brewer} \label{feat::brewer}
    Increase your level of proficiency with Brewer's Supplies.
    This feat can be re-taken until you are an expert with the toolset.

    You add your Constitution modifier to checks with Brewer's Supplies.

    From mastery with these tools, you have access to all items in the Brews sections (see page \pageref{ssec::brews}).
    The cost associated to each item only accounts for the required reagents, not the barrels used to contain the liquid or the tankard used to drink it.

    Common brews have a DC of 6 to make, Uncommon brews a DC of 9, Rare brews a DC of 12, Very Rare a DC of 15, and Legendary a DC of 18.
    Unless specified otherwise in the item's name, the DC of a brew relates to one small keg of the stuff, which fills 8 tankards before being drained.
\subsubsection{Purify} \label{feat::purify}
    Using brewing techniques you can purify dirty water with your Brewer's Supplies.
    As part of a short rest, you can purify doses of water equal to 4 + your proficiency bonus with Brewer's Supplies.
    \paragraph{Requirements} Competent proficiency with Brewer's Supplies.
\subsubsection{Chug!} \label{feat::chug}
    Instead of a minute, you can drink a brew using 3 actions, gaining the normal benefits associated to it.
    \paragraph{Requirements} Skilled proficiency with Brewer's Supplies.
\subsubsection{The Enhancer} \label{feat::theenhancer}
    As part of a short rest, you can quickly brew a tankard worth of The Enhancer, a brew unique to master brewers.
    Upon drinking, the drinker gains the effect associated to the drink for an hour.

    The six variants of The Enhancer are listed.
    \paragraph{Bear's Endurance} The target has advantage on Constitution checks.
    It also gains 2d6 temporary hit points, which are lost when the effect ends.
    \paragraph{Bull's Strength} The target has advantage on Strength checks, and their carrying capacity doubles.
    \paragraph{Cat's Grace} The target has advantage on Dexterity checks.
    It also doesn't take damage from falling 4 meters or less if it isn't incapacitated.
    \paragraph{Eagle's Splendor} The target has advantage on Charisma checks.
    \paragraph{Fox's Cunning} The target has advantage on Intelligence checks.
    \paragraph{Owl's Wisdom} The target has advantage on Wisdom checks.

    You can take this feat two additional times.
    The second, you extend the duration of the effect to 8 hours.

    The third, you extend the duration of the effect to 24 hours, and can brew two tankards worth of The Enhancer per short rest.
    These tankards can be different variants of the brew.
    \paragraph{Requirements} Skilled proficiency with Brewer's Supplies.
\subsubsection{Mixed Drinks (2 FP)} \label{feat::mixeddrinks}
    As part of a short rest, you can mix two brews to combine their effects.
    To do this, you must succeed on a DC 15 check with your Brewer's Supplies.
    On a successful check you combine the brews' effects, and are left with an amount of the combined brew equal to half the sum of the two brews --- the remainder is lost in the mixing process.
    \paragraph{Requirements} Expert proficiency with Brewer's Supplies.

% CALLIGRAPHER'S SUPPLIES
%   - DEXTERITY : SCROLLS
% \subsubsection{Calligrapher} \label{feat::calligrapher}
%     Increase your level of proficiency with Calligrapher's Supplies.
%     This feat can be re-taken until you are an expert with the toolset.
% \subsubsection{NAME} \label{feat::name}
%     DESCRIPTION
%     \paragraph{Requirements} Competent proficiency with Calligrapher's Supplies.
% \subsubsection{NAME} \label{feat::name}
%     DESCRIPTION
%     \paragraph{Requirements} Skilled proficiency with Calligrapher's Supplies.
% \subsubsection{Forger} \label{feat::forger}
%     You gain proficiency with a Forgery kit, using your proficiency bonus with Calligrapher's Supplies for checks using it.
%     You can take this feat two additional times, gaining different effects each time:
%     \begin{itemize}
%         \item The second, ...
%         \item The third, ...
%     \end{itemize}
%     \paragraph{Requirements} Skilled proficiency with Calligrapher's Supplies.
% \subsubsection{NAME (2 FP)} \label{feat::name}
%     DESCRIPTION
%     \paragraph{Requirements} Expert proficiency with Calligrapher's Supplies.

% CARPENTER'S TOOLS
%   - Strength : Wooden Weapons and Staves
\subsubsection{Carpenter} \label{feat::carpenter}
    Increase your level of proficiency with Carpenter's Tools.
    This feat can be re-taken until you are an expert with the toolset.

    You add your Strength modifier to checks with Carpenter's Tools.

    From mastery with these tools, you have access to large items made of wood, like barrels, battering rams, boats, etc.
    The cost associated to each item only accounts for the required wood.

    Common wooden items have a DC of 8 to make, Uncommon items a DC of 12, Rare items a DC of 16, Very Rare a DC of 20, and Legendary a DC of 24.
\subsubsection{NAME} \label{feat::name}
    DESCRIPTION
    \paragraph{Requirements} Competent proficiency with Carpenter's Tools.
\subsubsection{NAME} \label{feat::name}
    DESCRIPTION
    \paragraph{Requirements} Skilled proficiency with Carpenter's Tools.
\subsubsection{NAME (2 FP)} \label{feat::name}
    DESCRIPTION
    \paragraph{Requirements} Expert proficiency with Carpenter's Tools.

% COOKING UTENSILS chef cook sweettreat warmmeal
%   - Constitution : Food
\subsubsection{Cook} \label{feat::cook}
    Increase your level of proficiency with cooking utensils.
    This feat can be re-taken until you are an expert with the toolset.

    You add your Constitution modifier to checks with cooking utensils.

    From mastery with these tools, you have access to all items in the Food section (see page \pageref{ssec::food}).
    The cost associated to each item only accounts for the required reagents.

    Common foods have a DC of 6 to make, Uncommon foods a DC of 9, Rare a DC of 12, Very Rare a DC of 15, and Legendary a DC of 18.
    Unless specified otherwise in the item's name, the DC of food relates to 8 rations worth of the stuff.
\subsubsection{Warm Meal} \label{feat::warmmeal}
    By cooking them, you can make rations more effective.
    As part of a short rest, you can cook a number of rations equal to 4 + your proficiency bonus with Cooking Utensils.
    The number of cooked rations produced is equal to double the original number of rations, but spoil if left untouched for more than 24 hours.
    \paragraph{Requirements} Competent proficiency with Cooking Utensils.
\subsubsection{Sweet Treat} \label{feat::sweettreat}
    With one hour of work or as part of a short rest, you can cook a number of treats equal to your proficiency bonus with Cooking Utensils.
    These special treats last 8 hours after being made.
    A creature can use a bonus action to eat one of those treats to gain temporary hit points equal to your proficiency bonus with Cooking Utensils.
    \paragraph{Requirements} Skilled proficiency with Cooking Utensils.
\subsubsection{Chef (2 FP)} \label{feat::chef}
    As part of a short rest, you can cook special food, provided you have ingredients and Cooking Utensils on hand.
    You can prepare enough of this food for a number of creatures equal to 4 + your proficiency bonus with Cooking Utensils.
    At the end of the short rest, any creature who eats the food and spends one or more Hit Dice to regain hit points regains extra hit points equal to 2d8 + their Constitution modifier.
    \paragraph{Requirements} Expert proficiency with Cooking Utensils.

% GAMING KIT
%   - Intelligence?
% \subsubsection{Player} \label{feat::player}
%     Increase your level of proficiency with Gaming Kits.
%     This feat can be re-taken until you are an expert with the item.
%
%     When you take this feat, choose a specific Gaming Kit with which you are proficient.
%     This doesn't apply for other Gaming Kits, but spending a long rest practicing with one allows you to apply your proficiency bonus to rolls related to it.
% \subsubsection{NAME} \label{feat::name}
%     DESCRIPTION
%     \paragraph{Requirements} Competent proficiency with Gaming kit.
% \subsubsection{NAME} \label{feat::name}
%     DESCRIPTION
%     \paragraph{Requirements} Skilled proficiency with Gaming kit.
% \subsubsection{NAME (2 FP)} \label{feat::name}
%     DESCRIPTION
%     \paragraph{Requirements} Expert proficiency with Gaming kit.

% GLASSBLOWER'S TOOLS fineflask glassblower glassenhancement glassgrenade
\subsubsection{Glassblower} \label{feat::glassblower}
    Increase your level of proficiency with Glassblower's Tools.
    This feat can be re-taken until you are an expert with the toolset.

    You add your Dexterity modifier to checks with Glassblower's Tools.

    From mastery with these tools, you have access to any item made with glass, like vials, cups, and flasks.
    The cost associated to each item only accounts for the required glass.

    Common glass items have a DC of 8 to make, Uncommon items a DC of 12, Rare items a DC of 16, Very Rare a DC of 20, and Legendary a DC of 24.

    % To work with your glassblowers tools you need access to a furnace capable of reaching temperatures of more than 1,500$\degree$ Celcius.
\subsubsection{Glass Grenade} \label{feat::glassgrenade}
    As part of a short rest, you can bundle up discarded glass dust in a bag to craft a makeshift ``glass grenade''.
    You can throw this glass grenade at a range of 4/12 mt., and it violently explodes out in a 1 mt. radius upon hitting the floor.
    All creatures in its radius must succeed on a DC 15 Dexterity saving throw or take 1d4 piercing damage.
    Furthermore, the target loses 1d4 hit points at the start of each of its turns due to the shattered glass in the wound.
    Any creature can take two actions to staunch the wound with a successful DC 12 Wisdom (Medicine) check.

    This effect is not cummulative.
    \paragraph{Requirements} Competent proficiency with Glassblower's Tools.
\subsubsection{Fine Flask} \label{feat::fineflask}
    Flasks and vials you make are especially effective, making a great addition to an already fine liquid.
    Any potion drank from one flask or vial you make has its duration doubled.
    Bloodwell vials made in this fashion have their number of uses doubled.

    In addition, you learn how to make glass lining inside tankards, which adds this same effects to brews drank from them.
    \paragraph{Requirements} Skilled proficiency with Glassblower's Tools.
\subsubsection{Glass Enhancements (2 FP)} \label{feat::glassenhancement}
    You can enhance a weapon's effect by adding glass into the mix.
    As part of a long rest, you can work with a weaponsmith to add one of the following effects to a slashing or piercing weapon:
    \subparagraph{Hollowed Weapon} When a poison is applied to the weapon, the time it remains applied is doubled, and any roll related to it is made with advantage.
    \subparagraph{Grainy Edge} By adding glass grain to a weapons edge it becomes more effective.
    The weapon gains a +1 bonus to attack and damage rolls.
    \subparagraph{Glass Shards} This weapon contains loosely glued shards of glass on its faces.
    On a successful hit, the target takes 1d4 damage on the beginning of each of its turns.
    Any creature can take two actions to staunch the wound with a successful DC 12 Wisdom (Medicine) check.
    This effect is not cummulative.

    A weapon can only have one of these effects applied at the same time.
    \paragraph{Requirements} Expert proficiency with Glassblower's Tools.

% HEALER'S KIT combatmedic healer physician quickmender travellingdoctor
\subsubsection{Healer} \label{feat::healer}
    Increase your level of proficiency with the healer's kit.
    This feat can be re-taken until you are an expert with the kit.

    You add your Wisdom modifier to checks with the healer's kit.
\subsubsection{Quick Mender} \label{feat::quickmender}
    Used to working in the field, you can heal minor injuries as part of a short rest expending one use of your healer's kit.

    A creature healed with this feat also gains a number of temporary hit points equal to its Constitution modifier (minimum of one).
    \paragraph{Requirements} Competent proficiency with healer's kit.
\subsubsection{Physician} \label{feat::physician}
    As an avid physician, you can expend 3 uses of a healer's kit to heal a major injury as part of a long rest.
    The creature must succeed on a DC 12 Constitution saving throw for the treatment to work.

    A creature healed with this feat also gains a number of temporary hit points equal to double its Constitution modifier (minimum of two).
    \paragraph{Requirements} Skilled proficiency with healer's kit.
\subsubsection{Combat Medic} \label{feat::combatmedic}
    You are able to mend wounds quickly and get your allies back in the fight.
    Using two actions, you can spend one use of a healer's kit to tend to a creature and restore 1d6 + your proficiency modifier with the healer's kit hit points to it, plus additional hit points equal to the creature's maximum number of Hit Dice.
    The creature can't regain hit points from this feat again until it finishes a short rest.

    You can take this feat two additional times, increasing the number of dice rolled to 2d6 the second time and 3d6 the third.
    \paragraph{Requirements} Skilled proficiency with Healer's kit.
\subsubsection{Travelling Doctor (2 FP)} \label{feat::travellingdoctor}
    You keep your allies in top shape during your travels.
    The minimum number of hit points you and your allies regain from a hit die roll is equal to their Constitution modifier (minimum of 2).
    \paragraph{Requirements} Expert proficiency with Healer's kit.

% HERBALISM KIT
%   - ??? : Reagents, this includes stuff for alchemy, brewing, cooking, and poisonmaking.
% \subsubsection{Herbalist} \label{feat::herbalist}
%     Increase your level of proficiency with Herbalism Kit.
%     This feat can be re-taken until you are an expert with the toolset.
% \subsubsection{NAME} \label{feat::name}
%     DESCRIPTION
%     \paragraph{Requirements} Competent proficiency with Herbalism Kit.
% \subsubsection{NAME} \label{feat::name}
%     DESCRIPTION % collect reagents as part of a long rest. Roll an ability check. Reagents are worth the number rolled times 3.
%     \paragraph{Requirements} Skilled proficiency with Herbalism Kit.
% \subsubsection{Poisoner} \label{feat::poisoner}
%     You gain proficiency with Poisoner's Kits, using your proficiency bonus with Herbalism Kits for checks using them.
%     You can take this feat two additional times, gaining different effects each time:
%     \begin{itemize}
%         \item The second, your poisons ignore resistance to poison damage.
%         Additionally, you can coat a weapon in poison using one action instead of two.
%         \item The third, the Constitution saving throws of the poisons you make are increased by your proficiency bonus with the Herbalism Kit.
%     \end{itemize}
%     \paragraph{Requirements} Skilled proficiency with Herbalism Kit.
% \subsubsection{NAME (2 FP)} \label{feat::name}
%     DESCRIPTION
%     \paragraph{Requirements} Expert proficiency with Herbalism Kit.

% JEWELER'S TOOLS
%   - Dexterity : Wondrous items
% \subsubsection{Jeweler} \label{feat::jeweler}
%     Increase your level of proficiency with Jeweler's Tools.
%     This feat can be re-taken until you are an expert with the toolset.
% \subsubsection{NAME} \label{feat::name}
%     DESCRIPTION
%     \paragraph{Requirements} Competent proficiency with Jeweler's Tools.
% \subsubsection{NAME} \label{feat::name}
%     DESCRIPTION
%     \paragraph{Requirements} Skilled proficiency with Jeweler's Tools.
% \subsubsection{NAME (2 FP)} \label{feat::name}
%     DESCRIPTION
%     \paragraph{Requirements} Expert proficiency with Jeweler's Tools.

% LEATHERWORKER'S TOOLS
%   - Dexterity : Light armour, hide armour?
% \subsubsection{Leatherworker} \label{feat::leatherworker}
%     Increase your level of proficiency with Leatherworker's Tools.
%     This feat can be re-taken until you are an expert with the toolset.
% \subsubsection{NAME} \label{feat::name}
%     DESCRIPTION
%     \paragraph{Requirements} Competent proficiency with Leatherworker's Tools.
% \subsubsection{NAME} \label{feat::name}
%     DESCRIPTION
%     \paragraph{Requirements} Skilled proficiency with Leatherworker's Tools.
% \subsubsection{Cobbler} \label{feat::cobbler}
%     %   - Dexterity : boots
%     You gain proficiency with Cobbler's Tools, using your proficiency bonus with Leatherworker's Tools for checks using them.
%     You can take this feat two additional times, gaining different effects each time:
%     \begin{itemize}
%         \item The second, ...
%         \item The third, ...
%     \end{itemize}
%     \paragraph{Requirements} Skilled proficiency with Leatherworker's Tools.
% \subsubsection{NAME (2 FP)} \label{feat::name}
%     DESCRIPTION
%     \paragraph{Requirements} Expert proficiency with Leatherworker's Tools.

% MASON'S TOOLS
%   - Strength : Weapons and armour made of stone (e.g. obsidian)
% \subsubsection{Mason} \label{feat::mason}
%     Increase your level of proficiency with Mason's Tools.
%     This feat can be re-taken until you are an expert with the toolset.
% \subsubsection{NAME} \label{feat::name}
%     DESCRIPTION
%     \paragraph{Requirements} Competent proficiency with Mason's Tools.
% \subsubsection{NAME} \label{feat::name}
%     DESCRIPTION
%     \paragraph{Requirements} Skilled proficiency with Mason's Tools.
% \subsubsection{NAME (2 FP)} \label{feat::name}
%     DESCRIPTION
%     \paragraph{Requirements} Expert proficiency with Mason's Tools.

% MUSICAL INSTRUMENTS
%   - ???
% \subsubsection{Musician} \label{feat::musician}
%     Increase your level of proficiency with Musical Instruments.
%     This feat can be re-taken until you are an expert with the toolset.
%
%     When you take this feat, choose a specific instrument with which you are proficient.
%     This doesn't apply for other instruments, but spending a long rest practicing with one allows you to apply your proficiency bonus to rolls related to it.
% \subsubsection{NAME} \label{feat::name}
%     DESCRIPTION
%     \paragraph{Requirements} Competent proficiency with Musical Instruments.
% \subsubsection{NAME} \label{feat::name}
%     DESCRIPTION
%     \paragraph{Requirements} Skilled proficiency with Musical Instruments.
% \subsubsection{NAME (2 FP)} \label{feat::name}
%     DESCRIPTION
%     \paragraph{Requirements} Expert proficiency with Musical Instruments.

% NAVIGATOR'S TOOLS
%   - ???
% \subsubsection{Navigator} \label{feat::navigator}
%     Increase your level of proficiency with Navigator's Tools.
%     This feat can be re-taken until you are an expert with the toolset.
% \subsubsection{NAME} \label{feat::name}
%     DESCRIPTION
%     \paragraph{Requirements} Competent proficiency with Navigator's Tools.
% \subsubsection{NAME} \label{feat::name}
%     DESCRIPTION
%     \paragraph{Requirements} Skilled proficiency with Navigator's Tools.
% \subsubsection{Seafarer} \label{feat::seafarer}
%     You gain proficiency with Water Vehicles, using your proficiency bonus with Navigator's Tools for checks using them.
%     You can take this feat two additional times, gaining different effects each time:
%     \begin{itemize}
%         \item The second, ...
%         \item The third, ... % you cool and increase a ship's speed
%     \end{itemize}
%     \paragraph{Requirements} Skilled proficiency with Navigator's Tools.
% \subsubsection{NAME (2 FP)} \label{feat::name}
%     DESCRIPTION
%     \paragraph{Requirements} Expert proficiency with Navigator's Tools.

% PAINTER'S SUPPLIES
%   - Dexterity : Scrolls
% \subsubsection{Painter} \label{feat::painter}
%     Increase your level of proficiency with Painter's Supplies.
%     This feat can be re-taken until you are an expert with the toolset.
% \subsubsection{NAME} \label{feat::name}
%     DESCRIPTION
%     \paragraph{Requirements} Competent proficiency with Painter's Supplies.
% \subsubsection{NAME} \label{feat::name}
%     DESCRIPTION
%     \paragraph{Requirements} Skilled proficiency with Painter's Supplies.
% \subsubsection{NAME (2 FP)} \label{feat::name}
%     DESCRIPTION
%     \paragraph{Requirements} Expert proficiency with Painter's Supplies.
% \subsubsection{Cartographer} \label{feat::cartographer}
%     You gain proficiency with a Cartographer's Kit, using your proficiency bonus with Painter's Supplies for checks using it.
%     You can take this feat two additional times, gaining different effects each time:
%     \begin{itemize}
%         \item The second, you gain the ability to map the region you've traveled during the day with perfect accuracy as part of a short rest.
%         These maps allow you to retrace your steps without rolling any check, and give you advantage on any ability check to avoid getting lost.
%         Additionally, maping unexplored or uncharted regions can fetch high prices with the right buyer.
%         \item The third, you gain the ability to copy another map as part of a long rest.
%         To do this, you consume materials that cost half the cost of the map you're copying, and must succeed on an ability check using the Cartographer's Kit with a DC proportional to the scale and the level of detail of the map, at the DM's discretion.
%         If you have spent a month of more exploring the region in the map, you make this ability check with advantage.
%     \end{itemize}
%     \paragraph{Requirements} Skilled proficiency with Painter's Supplies.

% POTTER'S TOOLS
%   - Dexterity : ...?
% \subsubsection{Potter} \label{feat::potter}
%     Increase your level of proficiency with Potter's Tools.
%     This feat can be re-taken until you are an expert with the toolset.
% \subsubsection{NAME} \label{feat::name}
%     DESCRIPTION
%     \paragraph{Requirements} Competent proficiency with Potter's Tools.
% \subsubsection{NAME} \label{feat::name}
%     DESCRIPTION
%     \paragraph{Requirements} Skilled proficiency with Potter's Tools.
% \subsubsection{NAME (2 FP)} \label{feat::name}
%     DESCRIPTION
%     \paragraph{Requirements} Expert proficiency with Potter's Tools.

% SMITH'S TOOLS
%   - Strength : Metal weapons, heavy armour, medium armour (except hide)
% \subsubsection{Smith} \label{feat::smith}
%     Increase your level of proficiency with Smith's Tools.
%     This feat can be re-taken until you are an expert with the toolset.
% \subsubsection{NAME} \label{feat::name}
%     DESCRIPTION
%     \paragraph{Requirements} Competent proficiency with Smith's Tools.
% \subsubsection{NAME} \label{feat::name}
%     DESCRIPTION
%     \paragraph{Requirements} Skilled proficiency with Smith's Tools.
% \subsubsection{NAME (2 FP)} \label{feat::name}
%     DESCRIPTION
%     \paragraph{Requirements} Expert proficiency with Smith's Tools.

% Locksmithing Tools
%   - Dexterity : ???
% \subsubsection{Locksmith} \label{feat::locksmith}
%     Increase your level of proficiency with Locksmithing Tools.
%     This feat can be re-taken until you are an expert with the toolset.
%
%     Locksmithing Tools can be used to make checks to open locks or disarm traps, which are rolled after a minute fidgeting with the lock.
% \subsubsection{NAME} \label{feat::name}
%     DESCRIPTION
%     \paragraph{Requirements} Competent proficiency with Locksmithing Tools.
% \subsubsection{NAME} \label{feat::name}
%     DESCRIPTION
%     \paragraph{Requirements} Skilled proficiency with Locksmithing Tools.
% \subsubsection{Thief} \label{feat::thief}
%     You gain proficiency with Thieves' Tools, using your proficiency bonus with Locksmithing Tools for checks using them.
%     Thieves' tools reduce the time it takes you to open a lock or disarm a trap to two actions.
%
%     You can take this feat two additional times, gaining different effects each time:
%     \begin{itemize}
%         \item The second, ...
%         \item The third, the time it takes you to try to disarm a trap or open a lock is reduced from two to one action, provided you use your thieves' tools to do this.
%     \end{itemize}
%     \paragraph{Requirements} Skilled proficiency with Locksmithing Tools.
% \subsubsection{NAME (2 FP)} \label{feat::name}
%     DESCRIPTION
%     \paragraph{Requirements} Expert proficiency with Locksmithing Tools.

% TINKER'S TOOLS
%   - Dexterity : Wondrous items, firearms
% \subsubsection{Tinkerer} \label{feat::tinkerer}
%     Increase your level of proficiency with Tinker's Tools.
%     This feat can be re-taken until you are an expert with the toolset.
% \subsubsection{NAME} \label{feat::name}
%     DESCRIPTION
%     \paragraph{Requirements} Competent proficiency with Tinker's Tools.
% \subsubsection{NAME} \label{feat::name}
%     DESCRIPTION
%     \paragraph{Requirements} Skilled proficiency with Tinker's Tools.
% \subsubsection{NAME} \label{feat::name}
%     DESCRIPTION - UPGRADABLE
%     \paragraph{Requirements} Skilled proficiency with Tinker's Tools.
% \subsubsection{NAME (2 FP)} \label{feat::name}
%     DESCRIPTION
%     \paragraph{Requirements} Expert proficiency with Tinker's Tools.

% VEHICLES (LAND)
% \subsubsection{Rider} \label{feat::rider}
%     Increase your level of proficiency with Land Vehicles.
%     This feat can be re-taken until you are an expert with the proficiency.
%
%     Mounts act on the same initiative as you, but they can only take the Disengage, Dodge, and Move actions.
% \subsubsection{NAME} \label{feat::name}
%     DESCRIPTION
%     \paragraph{Requirements} Competent proficiency with Land Vehicles.
% \subsubsection{NAME} \label{feat::name}
%     DESCRIPTION
%     \paragraph{Requirements} Skilled proficiency with Land Vehicles.
% \subsubsection{Exotic Rider} \label{feat::exoticrider}
%     You gain proficiency with Air Vehicles, using your proficiency bonus with Land Vehicles for checks using them.
%     You can take this feat two additional times, gaining different effects each time:
%     \begin{itemize}
%         \item The second, ...
%         \item The third, ... % Half falling speed and reduce falling damage die from d6s to d4s.
%     \end{itemize}
%     \paragraph{Requirements} Skilled proficiency with Land Vehicles.
% \subsubsection{NAME (2 FP)} \label{feat::name}
%     DESCRIPTION
%     \paragraph{Requirements} Expert proficiency with Land Vehicles.

% WEAVER'S TOOLS
%   - Dexterity : Cloaks, robes, and clothing
% \subsubsection{Weaver} \label{feat::weaver}
%     Increase your level of proficiency with Weaver's Tools.
%     This feat can be re-taken until you are an expert with the toolset.
% \subsubsection{NAME} \label{feat::name}
%     DESCRIPTION
%     \paragraph{Requirements} Competent proficiency with Weaver's Tools.
% \subsubsection{NAME} \label{feat::name}
%     DESCRIPTION
%     \paragraph{Requirements} Skilled proficiency with Weaver's Tools.
% \subsubsection{Disguise Maker} \label{feat::disguisemaker}
%     You gain proficiency with Disguise Kits, using your proficiency bonus with Weaver's Tools for checks using them.
%     You can take this feat two additional times, gaining different effects each time:
%     \begin{itemize}
%         \item The second, you learn to observe the most distinguishing attributes of a person.
%         If you spend one hour observing or interacting with a creature, you can then craft a disguise that can be used to mimic that creature during a long rest.
%         \item The third, you learn how to procure a convincing disguise with stuff you find around you.
%         Taking an hour, you can make a disguise that makes sense with the materials you have around you (at the DM's discretion), which is as effective as one made with a Disguise Kit with similar materials.
%     \end{itemize}
%     \paragraph{Requirements} Skilled proficiency with Weaver's Tools.
% \subsubsection{NAME (2 FP)} \label{feat::name}
%     DESCRIPTION
%     \paragraph{Requirements} Expert proficiency with Weaver's Tools.

% WOODCARVER woodcarver adhocweaponry bonecarver chiseledhandle
\subsubsection{Woodcarver} \label{feat::woodcarver}
    Increase your level of proficiency with woodcarver's tools.
    This feat can be re-taken until you are an expert with the toolset.

    You add your Dexterity modifier to checks with woodcarver's tools.

    From mastery with these tools, you have access to small items made with wood and to apply engravings on wooden items and surfaces.
    These items include arrows, bolts, rods, etc.

    Common wooden items have a DC of 8 to make, Uncommon items a DC of 12, Rare items a DC of 16, Very Rare a DC of 20, and Legendary a DC of 24.
\subsubsection{Ad-hoc Weaponry} \label{feat::adhocweaponry}
    Taking an hour of work, you can create a number of improvised weapons equal to your proficiency bonus with woodcarver's tools.
    The weapons available to you are clubs, daggers, greatclubs, javelins, light hammers, maces, quarterstaves, and spears.
    The weapons quickly lose their effectivity, and can only be used for one combat encounter.

    To use this feat, you must have your woodcarver tools at hand and enough wood to make all the weapons.
    \paragraph{Requirements} Competent proficiency with Woodcarver's Tools.
\subsubsection{Chiseled Handle} \label{feat::chiseledhandle}
    Using your woodcarver's tools, you can chisel the wooden handle of a weapon or shield as part of a short rest.
    Equipment worked in this way fits nicely into their user's hand, and any check that attempts to disarm them is made with disadvantage.
    \paragraph{Requirements} Skilled proficiency with Woodcarver's Tools.
\subsubsection{Bonecarver (2 FP)} \label{feat::bonecarver}
    You can use your woodcarving tools to work with bone.
    Provided you can get the raw materials, you can craft bone charms and qualars as you would craft wooden charms.
    \paragraph{Requirements} Expert proficiency with Woodcarver's Tools.

% NOTE: Dunno where to place these so here, they're your problem now.
% \subsubsection{Linguist} \label{feat::linguist}
% \paragraph{RANK 2} You have advantage on Charisma (Languages) checks to determine the language that a creature is speaking.
% \paragraph{RANK 2} If you see a creature's mouth while it is speaking a language you understand, you can interpret what it's saying by reading its lips.
%
% \subsubsection{Educated} \label{feat::educated}
% \paragraph{RANK 3} You have an intuitive understanding of how to use all sorts of machinery, and can operate siege weapons and heavy machines without performing any ability checks.

% !TEX root = ../main.tex
\addcontentsline{toc}{section}{Combat Feats}
\subsection*{Combat Feats}
TODO: DESCRIPTION

% === ARMOR TYPES ==================================================================================
% UNARMORED
\subsubsection{Unarmored Agility} \label{feat::unarmoredagility}
    As long as you are not wearing any armor, your movement speed increases by 1.5 meters.
\subsubsection{NAME} \label{feat::name}
    DESCRIPTION

    To gain the benefits of this feat, you must not be wearing armor.
\subsubsection{NAME} \label{feat::name}
    DESCRIPTION - UPGRADABLE

    To gain the benefits of this feat, you must not be wearing armor.
\subsubsection{Unarmored Master (2 FP)} \label{feat::unarmoredfighter}
    If you are subjected to an effect that allows you to make a Dexterity saving throw to take only half damage, you can use your reaction to take no damage if you succeed on the saving throw, and half damage if you fail.
    You only get this bonus while wearing no armor or light armor.
% LIGHT
\subsubsection{Lightly Armored} \label{feat::lightlyarmored}
    You learn how to wear light armor in combat.
\subsubsection{Light on your Feet} \label{feat::lightonyourfeet}
    Equipped light armor and cloth accessories don't add to your encumbrance.
    \paragraph{Requirements} Proficiency with Light Armor.
\subsubsection{NAME} \label{feat::name}
    DESCRIPTION

    To gain the benefits of this feat, you must be wearing light armor.
    \paragraph{Requirements} Proficiency with Light Armor.
\subsubsection{NAME} \label{feat::name}
    DESCRIPTION - UPGRADABLE

    To gain the benefits of this feat, you must be wearing light armor.
    \paragraph{Requirements} Proficiency with Light Armor.
\subsubsection{Light Armor Master (2 FP)} \label{feat::lightarmormaster}
    If you aren't incapacitated, you can use your reaction to gain advantage on any Dexterity saving throw you make against a spell or other harmful effect that targets only you.
    In addition, you can use the Dodge action even if your speed is 0.

    To gain the benefits of this feat, you must be wearing light armor.
    \paragraph{Requirements} Proficiency with Light Armor.
% MEDIUM
\subsubsection{Moderately Armored} \label{feat::moderatelyarmored}
    You learn how to wear medium armor in combat.
\subsubsection{Silent Despite Everything} \label{feat::silentdespiteeverything}
    Wearing medium armor doesn't impose disadvantage on your Dexterity (Stealth) checks.
    \paragraph{Requirements} Proficiency with Medium Armor.
\subsubsection{Agile in a Suit} \label{feat::agileinasuit}
    When you wear medium armor, you can add 3, rather than 2, to your AC if you have a Dexterity of 16 or higher.
    \paragraph{Requirements} Proficiency with Medium Armor.
\subsubsection{Vital Protection} \label{feat::vitalprotection}
    While wearing medium armor, you may add a +1 to Strength, Dexterity, and Constitution saving throws in which you don't have any proficiency level.

    You can take this feat two additional times, increasing this bonus by +1 each time.
    \paragraph{Requirements} Proficiency with Medium Armor.
\subsubsection{NAME (2 FP)} \label{feat::name}
    DESCRIPTION

    To gain the benefits of this feat, you must be wearing medium armor.
    \paragraph{Requirements} Proficiency with Medium Armor.
% HEAVY heavilyarmored heavyarmormaster heavyweight imposingfigure tough
\subsubsection{Heavily Armored} \label{feat::heavilyarmored}
    You learn how to wear heavy armor in combat.
\subsubsection{Heavy-Weight} \label{feat::heavyweight}
    While you are wearing heavy armor, you roll with advantage when you take the Shove action.
    You also roll Strength (Athletics) with advantage when you are targeted by the Shove action.
    \paragraph{Requirements} Proficiency with Heavy Armor.
\subsubsection{Imposing Figure} \label{feat::imposingfigure}
    When you make a Charisma (Intimidation) check, you can add a +5 bonus if you are wearing heavy armor.
    \paragraph{Requirements} Proficiency with Heavy Armor.
\subsubsection{Heavy Armor Master} \label{feat::heavyarmormaster}
    While you are wearing heavy armor, bludgeoning, piercing, and slashing damage that you take is reduced by 1.

    You can take this feat two additional times, increasing this defense by 1 each time.
    \paragraph{Requirements} Proficiency with Heavy Armor.
\subsubsection{Tough (2 FP)} \label{feat::tough}
    Your hit point maximum increases by an amount equal to your level when you take this feat.
    Whenever you gain a level thereafter, your hit point maximum increases by an additional hit point.
    \paragraph{Requirements} Proficiency with Heavy Armor.

% === ARMOR PROPERTIES =============================================================================
% BULWARK
\subsubsection{NAME} \label{feat::name}
    DESCRIPTION

    To gain the benefits of this feat, you must be wearing armor with the Bulwark property.
\subsubsection{NAME} \label{feat::name}
    DESCRIPTION

    To gain the benefits of this feat, you must be wearing armor with the Bulwark property.
\subsubsection{NAME (2 FP)} \label{feat::name}
    DESCRIPTION

    To gain the benefits of this feat, you must be wearing armor with the Bulwark property.
% CHAIN
\subsubsection{NAME} \label{feat::name}
    DESCRIPTION

    To gain the benefits of this feat, you must be wearing armor with the Chain property.
\subsubsection{NAME} \label{feat::name}
    DESCRIPTION

    To gain the benefits of this feat, you must be wearing armor with the Chain property.
\subsubsection{NAME (2 FP)} \label{feat::name}
    DESCRIPTION

    To gain the benefits of this feat, you must be wearing armor with the Chain property.
    % use noise of chains to distract nearby creatures?
% CLOTH
\subsubsection{NAME} \label{feat::name}
    DESCRIPTION

    To gain the benefits of this feat, you must be wearing armor with the Cloth property.
\subsubsection{NAME} \label{feat::name}
    DESCRIPTION

    To gain the benefits of this feat, you must be wearing armor with the Cloth property.
\subsubsection{NAME (2 FP)} \label{feat::name}
    DESCRIPTION

    To gain the benefits of this feat, you must be wearing armor with the Cloth property.
% COMPOSITE
\subsubsection{NAME} \label{feat::name}
    DESCRIPTION

    To gain the benefits of this feat, you must be wearing armor with the Composite property.
\subsubsection{NAME} \label{feat::name}
    DESCRIPTION

    To gain the benefits of this feat, you must be wearing armor with the Composite property.
\subsubsection{NAME (2 FP)} \label{feat::name}
    DESCRIPTION

    To gain the benefits of this feat, you must be wearing armor with the Composite property.
% LEATHER
\subsubsection{NAME} \label{feat::name}
    DESCRIPTION

    To gain the benefits of this feat, you must be wearing armor with the Leather property.
\subsubsection{NAME} \label{feat::name}
    DESCRIPTION

    To gain the benefits of this feat, you must be wearing armor with the Leather property.
\subsubsection{NAME (2 FP)} \label{feat::name}
    DESCRIPTION

    To gain the benefits of this feat, you must be wearing armor with the Leather property.
% PLATE
\subsubsection{Heavy Hitter} \label{feat::heavyhitter}
    If you are wearing plate gauntlets, add a +2 bonus to your damage rolls with unarmed attacks.
\subsubsection{NAME} \label{feat::name}
    DESCRIPTION

    To gain the benefits of this feat, you must be wearing armor with the Plate property.
\subsubsection{NAME (2 FP)} \label{feat::name}
    DESCRIPTION

    To gain the benefits of this feat, you must be wearing armor with the Plate property.

% === SHIELD TYPES =================================================================================
% SMALL SHIELDS bucklertraining defensiveduelist quickshielding strappedandsecured swiftdeflection
\subsubsection{Buckler Training} \label{feat::bucklertraining}
    You learn how to use small shields in combat.
\subsubsection{Quick Shielding} \label{feat::quickshielding}
    You can don or doff a light shield as a free object interaction.
    \paragraph{Requirements} Proficiency with Small Shields.
\subsubsection{Defensive Duelist} \label{feat::defensiveduelist}
    When fighting with a one-handed melee weapon and a small shield, you can add half your shield's AC (rounded up) to the weapon's attack damage.
    \paragraph{Requirements} Proficiency with Small Shields.
\subsubsection{Swift Deflection} \label{feat::swiftdeflection}
    You learn the Parry reaction (see page \pageref{act::parry}).

    You can take this feat two additional times.
    Each time you increase the bonus to your AC by one die, so a d6 turns into a d8, and a d8 into a d10.
    \paragraph{Requirements} Proficiency with Small Shields.
\subsubsection{Strapped and Secured (2 FP)} \label{feat::strappedandsecured}
    By attaching a leather strap to a light shield, you can carry it without using your off-hand.
    Due to the shield you still cannot carry a weapon in that hand, but your hand is free to use items, hold spellcasting focus, reload a crossbow, etc.

    In addition, your shield cannot be removed by the Disarm action.
    \paragraph{Requirements} Proficiency with Small Shields.
% MEDIUM SHIELDS bladeandboard shieldtraining shieldmaster stalwartshield thebestdefense
\subsubsection{Shield Training} \label{feat::shieldtraining}
    Always balanced in combat, you learn how to use medium shields in combat.
\subsubsection{Blade \& Board} \label{feat::bladeandboard}
    While you're wielding a medium shield and a creature misses you with a melee attack, you can use your reaction to attack it with a melee attack.
    \paragraph{Requirements} Proficiency with Medium Shields.
\subsubsection{The Best Defense...} \label{feat::thebestdefense}
    As an action, you can bash a creature with your shield.
    Your attack bonus for this attack is equal to the shield's AC bonus + your Strength modifier.
    On a hit, the creature takes 1d4 + your Strength modifier in bludgeoning damage.

    You can use this attack only once per turn, and it doesn't count towards your multiple attack penalty.

    In addition, if you hit a creature with this action using a medium shield, your next melee attack on this turn to the creature is made with advantage.
    \paragraph{Requirements} Proficiency with Medium Shields.
\subsubsection{Shield Master} \label{feat::shieldmaster}
    You can take this feat three times, gaining an additional effect each time:
    \begin{itemize}
        \item The first time, you gain half cover against ranged attacks while using a medium shield.
        \item The second time, you can add your shield's AC bonus to any Dexterity saving throw you make against a spell or other harmful effect that targets only you.
        \item The third time, if you are subjected to an effect that allows you to make a Dexterity saving throw to take only half damage, you can use your reaction to take no damage if you succeed on the saving throw, interposing your shield between yourself and the source of the effect.
    \end{itemize}
    \paragraph{Requirements} Proficiency with Medium Shields.
\subsubsection{Stalwart Shield (2 FP)} \label{feat::stalwartshield}
    You learn the Push action.

    In addition, you can use this action as a free action after making an attack with a melee weapon against it.
    You must be wielding a medium shield to use the action in this way.
    \paragraph{Requirements} Proficiency with Medium Shields.
% HEAVY SHIELDS bulwarktraining consistentefense heavyshieldtraining immovableobject takecover
\subsubsection{Heavy Shield Training} \label{feat::heavyshieldtraining}
    Prioritizing a solid defense above everything, you learn how to use heavy shields in combat.
\subsubsection{Consistent Defense} \label{feat::consistentefense}
    You gain three-fourths cover instead of half cover against ranged attacks while using a heavy shield.
    \paragraph{Requirements} Proficiency with Heavy Shields.
\subsubsection{Bulwark Training} \label{feat::bulwarktraining}
    You halve the movement debuff associated to heav shields.
    \paragraph{Requirements} Proficiency with Heavy Shields.
\subsubsection{Take Cover!} \label{feat::takecover}
    You can duck behind a heavy shield as an action, granting you full cover against ranged attacks until the start of your next turn.

    You can take this feat two additional times, granting you a new bonus each time:
    The second time, creatures standing behind you from the attacker's perspective have three-fourths cover against ranged attacks.

    The third time, you have advantage on Strength and Dexterity saving throws while you are ducking behind your shield.
    \paragraph{Requirements} Proficiency with Heavy Shields.
\subsubsection{Immovable Object (2 FP)} \label{feat::immovableobject}
    You are an obstacle.

    Creatures can't use a Tumble action to move through your space.

    In addition, you have advantage on Strength (Athletics) checks to prevent being moved or grappled if you have a heavy shield equipped.
    \paragraph{Requirements} Proficiency with Heavy Shields.

% === SIMPLE WEAPONS ===============================================================================
% SIMPLE WEAPONS
\subsubsection{Simple Fighter} \label{feat::simplefighter}
    No one learned how to use a sword without first swinging a stick.

    You increase your proficiency level with simple weapons.
    This feat can be taken three times, or until you reach Expert proficiency with the weapon type.
\subsubsection{Combat Improviser} \label{feat::combatimproviser}
    If you can grab it, you can swing it.

    You can apply your proficiency modifier with simple weapons to attacks made with improvised weapons.

    In addition, you have advantage on all attack rolls using improvised weapons during the first round of combat.
    \paragraph{Requirements} Competent proficiency with Simple Weapons.
\subsubsection{Cutthroat} \label{feat::cutthroat}
    You have advantage on any check to conceal a small weapon on your person.
    \paragraph{Requirements} Competent proficiency with Simple Weapons.
\subsubsection{Adaptable Fighter} \label{feat::adaptablefighter}
    You learn one action of your choice between % Bonk, Block, Buttstroke, Feint, Lunge, Parry, Pushing Attack, Protect, Quick Draw, Riposte, Rush, Sweep, and Trip.
    \paragraph{Requirements} Skilled proficiency with Simple Weapons.
\subsubsection{Quick Action (2 FP)} \label{feat::quickaction}
    You can make one melee weapon attack with a simple weapon with the light property as part of a Grapple, Escape a Grapple, Overrun, or Tumble action.

    You can use this ability only once per turn.
    \paragraph{Requirements} Skilled proficiency with Simple Weapons.
\subsubsection{NAME} \label{feat::name}
    DESCRIPTION
    \paragraph{Requirements} Expert proficiency with Simple Weapons.
\subsubsection{Solid Start (2 FP)} \label{feat::solidstart}
    You roll for initiative with advantage.

    In addition, any hit you score against a creature that hasn't taken any actions in combat is a critical hit.
    \paragraph{Requirements} Expert proficiency with Simple Weapons.

% === MARTIAL WEAPONS ==============================================================================
% AXES axemaster forcefuldisarm griefer maimingstrikes murderousintent rush treefeller
\subsubsection{Axemaster} \label{feat::axemaster}
    More than a simple tool, you master the use of axes in combat.

    You increase your proficiency level with axes.
    This feat can be taken three times, or until you reach Expert proficiency with the weapon type.
\subsubsection{Griefer} \label{feat::griefer}
    While wielding an axe, you deal double damage to structures and items.
    In addition, you can cut a rope as one action using your trusty axe.
    \paragraph{Requirements} Competent proficiency with Axes.
\subsubsection{Tree Feller} \label{feat::treefeller}
    Nature is your foe, and you use your axe to fell both trees and enemies.
    You have advantage on attack rolls against plant-based creatures and any structure made of wood.
    \paragraph{Requirements} Competent proficiency with Axes.
\subsubsection{Forceful Disarm} \label{feat::forcefuldisarm}
    You reduce the action cost of the Disarm action to one action.
    If you are holding an axe, you have advantage on your attack roll with the Disarm action to force a creature to drop its shield.
    \paragraph{Requirements} Skilled proficiency with Axes.
\subsubsection{Maiming Strikes (2 FP)} \label{feat::maimingstrikes}
    When you hit a creature with the attack action, the next attack against that creature before the end of your next turn is rolled with advantage.
    \paragraph{Requirements} Skilled proficiency with Axes.
\subsubsection{Rush} \label{feat::rush}
    As part of your attack, you can move up to half your movement speed towards the target of your attack.
    You can use this ability only once per turn.
    \paragraph{Requirements} Expert proficiency with Axes or Hammers.
\subsubsection{Murderous Intent (2 FP)} \label{feat::murderousintent}
    When a creature rolls on the minor or major injury charts from one of your attacks made with an axe, it must roll the d20 twice and take the higher result.
    \paragraph{Requirements} Expert proficiency with Axes.
% BOWS
\subsubsection{Bowmaster} \label{feat::bowmaster}
    You increase your proficiency level with bows.
    This feat can be taken three times, or until you reach Expert proficiency with the weapon type.
\subsubsection{Sudden Attack} \label{feat::suddenattack}
    You can use your bow to make melee weapon attacks.
    Melee attacks with your bow deal 1d4 bludgeoning damage plus your Strength modifier.

    When you hit with a melee attack with your bow, the attacked creature can't make opportunity attacks against you until the start of your next turn.
    \paragraph{Requirements} Competent proficiency with Bows.
\subsubsection{Covered Shooter} \label{feat::coveredshooter}
    If you miss with a ranged attack while hidden, the attack doesn't reveal your location, and you remain hidden.
    \paragraph{Requirements} Competent proficiency with Bows.
\subsubsection{NAME - ACTION} \label{feat::name}
    DESCRIPTION
    \paragraph{Requirements} Skilled proficiency with Bows.
\subsubsection{NAME (2 FP)} \label{feat::name}
    DESCRIPTION
    \paragraph{Requirements} Skilled proficiency with Bows.
\subsubsection{Still Aim} \label{feat::stillaim}
    You gain a bonus equal to half your proficiency bonus with bows to damage rolls with a bow if you don't move during your turn or are riding a mount.
    \paragraph{Requirements} Expert proficiency with Bows.
\subsubsection{Quick Shot (2 FP)} \label{feat::quickshot}
    You learn the Quick Draw reaction.

    In addition, when you hit a creature with an opportunity attack with a bow, the creature cannot continue moving during that turn.
    \paragraph{Requirements} Expert proficiency with Bows.
% CROSSBOWS AND GUNS (one feat is only learning guns)
\subsubsection{Crossbow Enthusiast} \label{feat::crossbowenthusiast}
    You increase your proficiency level with crossbows.
    This feat can be taken three times, or until you reach Expert proficiency with the weapon type.
\subsubsection{NAME} \label{feat::name}
    DESCRIPTION
    \paragraph{Requirements} Competent proficiency with Crossbows.
\subsubsection{Quick Loading} \label{feat::quickloading}
    You ignore the loading property of crossbows.
    % In addition, you can load or reload a one-handed ranged weapon without having a free hand.
    \paragraph{Requirements} Competent proficiency with Crossbows.
\subsubsection{Prepared Shot} \label{feat::preparedshot}
    Right after rolling for initiative, you can don a crossbow or firearm and make one ranged attack with it.
    \paragraph{Requirements} Skilled proficiency with Crossbows.
\subsubsection{Modern Shooter (2 FP)} \label{feat::modernshooter}
    You gain proficiency with firearms, using your proficiency bonus with crossbows for your attack rolls with them.
    \paragraph{Requirements} Skilled proficiency with Crossbows.
\subsubsection{Puncture Wound} \label{feat::puncturewound}
    When you get a critical hit on a creature using a ranged weapon, all attacks against that creature are made with advantage until the start of your next turn.
    \paragraph{Requirements} Expert proficiency with Crossbows.
\subsubsection{Visceral Attack (2 FP)} \label{feat::visceralattack}
    Being within 1.5 meters of a hostile creature doesn't impose disadvantage on your ranged attack rolls, and you don't provoke attacks of opportunity when using the Attack action with a ranged weapon when within the reach of a creature.

    In addition, when you successfully hit a creature with a crossbow or firearm while within 1.5 meters of it, your next melee attack against the creature is made with advantage.
    \paragraph{Requirements} Expert proficiency with Crossbows.
% CURVED SWORDS impairingattack
\subsubsection{Bent Blade Master} \label{feat::bentblademaster}
    You increase your proficiency level with curved swords.
    This feat can be taken three times, or until you reach Expert proficiency with the weapon type.
\subsubsection{Minced Meat} \label{feat::mincedmeat}
    You add a +2 bonus on your attack damage made with curved swords against unarmored creatures and creatures wearing armor with the Cloth property.
    \paragraph{Requirements} Competent proficiency with Curved Swords.
\subsubsection{Graceful Disarm} \label{feat::gracefuldisarm}
    You reduce the action cost of the Disarm action to one action.
    If you are holding a curved sword, you have advantage on your attack roll with the Disarm action to force a creature to drop its shield.
    \paragraph{Requirements} Skilled proficiency with Curved Swords.
\subsubsection{Parrying Stance (2 FP)} \label{feat::parryingstance}
    You learn the Parry reaction (see page \pageref{act::parry}).

    In addition, when you use the Dodge action you assume a parrying stance.
    When you do so, you can use the Parry reaction without expending your reaction until the start of your next turn.
    \paragraph{Requirements} Skilled proficiency with Curved Swords.
\subsubsection{NAME} \label{feat::name}
    DESCRIPTION
    \paragraph{Requirements} Expert proficiency with Curved Swords.
\subsubsection{Shield Sweep (2 FP)} \label{feat::shieldsweep}
    Your melee attacks with curved swords ignore the AC provided by shields.
    In addition, whenever you get a critical hit against a target wearing a shield, you can forfeit your brutal critical roll and injure the target's shield-bearing arm as if you had rolled a 4 in the Minor Injury chart (see page \pageref{ssec::injuriesandinsanity}).
    \paragraph{Requirements} Expert proficiency with Curved Swords.
% HAMMERS rush
\subsubsection{Blunt Fighter} \label{feat::bluntfighter}
    You increase your proficiency level with hammers.
    This feat can be taken three times, or until you reach Expert proficiency with the weapon type.
\subsubsection{Armor Breaker} \label{feat::armorbreaker}
    You add a +2 bonus on your attack damage made with hammers against creatures wearing armor with the Chain or Plate properties.
    \paragraph{Requirements} Competent proficiency with Hammers.
\subsubsection{NAME} \label{feat::name}
    DESCRIPTION
    \paragraph{Requirements} Competent proficiency with Hammers.
\subsubsection{NAME - ACTION} \label{feat::name}
    DESCRIPTION
    \paragraph{Requirements} Skilled proficiency with Hammers.
\subsubsection{NAME (2 FP)} \label{feat::name}
    DESCRIPTION
    \paragraph{Requirements} Skilled proficiency with Hammers.
\subsubsection{Bend Metal (2 FP)} \label{feat::bendmetal}
    When you hit a creature with a hammer, you can choose to not damage it; instead focusing in its armor.
    If the creature is wearing armor with the Chain, Plate, or Natural property, it applies a -1 penalty to its AC.

    Damaged armor can be repaired by someone Skilled with Smith's tools, and it usually costs a quarter or half of the cost of the armor, to the DM's discretion.
    If you reduce the AC bonus given by the armor to 0, the armor is destroyed irreparably.

    Natural armor damaged by this feat heals over time, regaining its original AC over the course of a long rest.
    \paragraph{Requirements} Expert proficiency with Hammers.
% POLEARMS
\subsubsection{Polearm Master} \label{feat::polearmmaster}
    You increase your proficiency level with polearms.
    This feat can be taken three times, or until you reach Expert proficiency with the weapon type.
\subsubsection{Rondel Hit} \label{feat::rondelhit}
    When you take the Attack action and attack with a polearm, you can use another action to make a melee attack with the opposite end of the weapon, using the same ability modifier as the main attack.
    The weapon's damage die for this attack is a d4, and the attack deals bludgeoning damage.

    This attack is unaffected by your multiple attack penalty, and doesn't count toward it.
    \paragraph{Requirements} Competent proficiency with Polearms.
\subsubsection{NAME} \label{feat::name}
    DESCRIPTION
    \paragraph{Requirements} Competent proficiency with Polearms.
\subsubsection{Trip} \label{feat::trip}
    Using two actions, you can attack a creature with the intention of knocking it down.
    Make a normal attack roll against the creature.
    On a hit, the creature must make a Strength saving throw of a DC equal to 8 + your proficiency bonus with polearms + your Strength modifier in addition to taking damage.
    On a failed save, you knock the target prone.

    The creature has advantage on this roll if it is at least 2 size categories larger than you.
    \paragraph{Requirements} Skilled proficiency with Polearms.
\subsubsection{Sweep (2 FP)} \label{feat::sweep}
    Using two actions, you can choose to attack all creatures in a 180$\degree$ arc in front of you within the range of your equipped weapon.
    You make a melee attack roll once, hitting all creatures with an AC lower than the number rolled.
    You roll damage once, which is applied on all the struck creatures.
    \paragraph{Requirements} Skilled proficiency with Polearms.
\subsubsection{Not on my Watch!} \label{feat::notonmywatch}
    While you are wielding a polearm, other creatures provoke an opportunity attack from you when they enter your reach.
    \paragraph{Requirements} Expert proficiency with Polearms.
\subsubsection{NAME (2 FP)} \label{feat::name}
    DESCRIPTION
    \paragraph{Requirements} Expert proficiency with Polearms.
% RAPIERS appropiateresponse dignifiedmiss elegantstriker fencer focusattacks impairingattack
\subsubsection{Elegant Striker} \label{feat::elegantstriker}
    You increase your proficiency level with rapiers.
    This feat can be taken three times, or until you reach Expert proficiency with the weapon type.
\subsubsection{Quick Stabs} \label{feat::quickstabs}
    You reduce the multiple attack penalty by 2 with rapiers.
    \paragraph{Requirements} Competent proficiency with Rapiers.
\subsubsection{Impairing Attack} \label{feat::impairingattack}
    During your turn, if you make a melee attack against a creature, that creature can't make opportunity attacks against you for the rest of your turn.
    \paragraph{Requirements} Competent proficiency with Curved Swords or Rapiers.
\subsubsection{Dignified Miss} \label{feat::dignifiedmiss}
    Whenever you miss with a melee weapon attack, you can use your reaction to make another melee weapon attack.
    The multiple attack penalty applies to this attack normally.
    \paragraph{Requirements} Skilled proficiency with Rapiers.
\subsubsection{Appropiate Response (2 FP)} \label{feat::appropiateresponse}
    You learn the Riposte reaction (see page \pageref{act::riposte}).

    In addition, when you use this reaction while wielding a rapier, you attack twice instead of once.
    \paragraph{Requirements} Skilled proficiency with Rapiers.
\subsubsection{Fencer} \label{feat::fencer}
    You gain a +1 bonus to melee attack rolls and your AC when only one creature is within 1.5 meters of you.
    \paragraph{Requirements} Expert proficiency with Rapiers.
\subsubsection{Focus Attacks (2 FP)} \label{feat::focusattacks}
    Using an action, you can search your opponent for vulnerabilities.
    Choose a creature within 1.5 meters of you.
    Until the end of your next turn, your melee attack damage with rapiers against that creature gains a bonus equal to your proficiency modifier with rapiers.
    \paragraph{Requirements} Expert proficiency with Rapiers.
% SPEARS?
% \subsubsection{Spears Proficiency} \label{feat::name}
%     DESCRIPTION
% \subsubsection{NAME} \label{feat::name}
%     DESCRIPTION
%     \paragraph{Requirements} Competent proficiency with Spears.
% \subsubsection{NAME} \label{feat::name}
%     DESCRIPTION
%     \paragraph{Requirements} Competent proficiency with Spears.
% \subsubsection{NAME - ACTION} \label{feat::name}
%     DESCRIPTION
%     \paragraph{Requirements} Skilled proficiency with Spears.
% \subsubsection{NAME (2 FP)} \label{feat::name}
%     DESCRIPTION
%     \paragraph{Requirements} Skilled proficiency with Spears.
% \subsubsection{NAME} \label{feat::name}
%     DESCRIPTION
%     \paragraph{Requirements} Expert proficiency with Spears.
% \subsubsection{NAME (2 FP)} \label{feat::name}
%     DESCRIPTION
%     \paragraph{Requirements} Expert proficiency with Spears.
% SWORDS
\subsubsection{Swordmaster} \label{feat::swordmaster}
    You increase your proficiency level with straight swords.
    This feat can be taken three times, or until you reach Expert proficiency with the weapon type.
\subsubsection{NAME} \label{feat::name}
    DESCRIPTION
    \paragraph{Requirements} Competent proficiency with Straight Swords.
\subsubsection{NAME} \label{feat::name}
    DESCRIPTION
    \paragraph{Requirements} Competent proficiency with Straight Swords.
\subsubsection{NAME - ACTION} \label{feat::name}
    DESCRIPTION
    \paragraph{Requirements} Skilled proficiency with Straight Swords.
\subsubsection{NAME (2 FP)} \label{feat::name}
    DESCRIPTION
    \paragraph{Requirements} Skilled proficiency with Straight Swords.
\subsubsection{NAME} \label{feat::name}
    DESCRIPTION
    \paragraph{Requirements} Expert proficiency with Straight Swords.
\subsubsection{Stalwart Stance (2 FP)} \label{feat::stalwartstance}
    You learn the Block reaction.

    In addition, when you are wielding a straight sword and no other weapons, you can a +1 bonus to your AC.
    \paragraph{Requirements} Expert proficiency with Straight Swords.

% === WEAPON TYPES =================================================================================
% BLUDGEONING
\subsubsection{Fell Hand} \label{feat::fellhand}
    When you score a critical hit that deals bludgeoning damage to a creature, attack rolls against that creature are made with advantage until the start of your next turn.
\subsubsection{Bonk} \label{feat::bonk}
    Using two actions, you can attempt to hit a creature directly in its head.
    Make a normal attack roll.
    On a successful hit, apart from taking damage, the creature must roll a Constitution saving throw with a DC equal to 8 + your proficiency bonus with the weapon you're wielding + your Strength modifier.
    On a failed save, the creature has disadvantage on all attack rolls and ability checks until the start of your next turn.

    To use this ability, you must be wielding a bludgeoning weapon and be able to reach the creature's head --- or one of them.
\subsubsection{Off-balance} \label{feat::offbalance}
    Once per turn, when you hit a creature with an attack that deals bludgeoning damage, you can move it 1.5 meters to an unoccupied space, provided that the target is no more than one size larger than you.

    You can take this feat two additional times, increasing the range you move the creature by 1.5 meters each time.
\subsubsection{Crusher (2 FP)} \label{feat::crusher}
    DESCRIPTION
% PIERCING
\subsubsection{Critical Injury} \label{feat::criticalinjury}
    Once per turn, when you hit a creature with an attack that deals piercing damage, you can reroll one of the attack's damage dice, and you must use the new roll.
\subsubsection{NAME} \label{feat::name}
    DESCRIPTION
\subsubsection{Pierce a Lung!} \label{feat::piercealung}
    You learn or improve the Sneak Attack action (see page \pageref{act:sneakattack}).
    You can take this feat three times, adding a d6 to the action's damage each time.

    You can only add these d6s to your Sneak Attack if you attack with a piercing weapon.
\subsubsection{Piercer (2 FP)} \label{feat::piercer}
    DESCRIPTION
% SLASHING
\subsubsection{NAME} \label{feat::name}
    DESCRIPTION
\subsubsection{NAME} \label{feat::name}
    DESCRIPTION
\subsubsection{Aimed Cut} \label{feat::aimedcut}
    You can take this feat three times, learning a different specialized cut each time:
    \begin{itemize}
        \item The first time you take this feat, you learn how to perform a leg cut.
        When you hit a creature with an attack that deals slashing damage, you halve the creature's movement speed until the start of your next turn.
        \item The second time, you learn how to perform an arm cut.
        Pick one of the creature's arms.
        You aim for that arm when you make an attack that deals slashing damage, and all attacks made with that hand are made with disadvantage until the start of your next turn.
        \item The third time, you learn how to cause a hemorrhage.
        For the next minute after being hit by your attack with a slashing weapon, the creature takes 1d4 slashing damage at the start of each of its turns.
        It can staunch this wound using two actions, ending the effect early.
    \end{itemize}

    You can only perform any of these cuts once per turn.
\subsubsection{Slasher (2 FP)} \label{feat::slasher}
    DESCRIPTION

% === FIGHTING STYLES ==============================================================================
\subsubsection{Fighting Initiate} \label{feat::fightinginitiate}
    You learn one fighting style of your choice (see page \pageref{ssec::fightingstyles}).
    You can take this feat only once, but you can learn more fighting styles as a major character improvement.
% ARCHERY
\subsubsection{NAME} \label{feat::name}
    DESCRIPTION
    \paragraph{Requirements} Fighting Style: Archery 1.
\subsubsection{Bullseye} \label{feat::bullseye}
    By focusing your aim, you can choose to cause an additional effect when you hit a creature with a ranged attack.
    You learn one effect when learning this technique, but you can learn additional ones by taking it again:
    \begin{itemize}
        \item \textbf{Leg Shot.} The creature's movement speed is divided by half (rounded up) until the start of your next turn.
        \item \textbf{Hand Shot.} The creature has disatvantage on the first melee attack roll it makes during its next turn.
        \item \textbf{Slayer's Shot.} The creature takes an extra 1d8 damage if its below its hit point maximum.
    \end{itemize}

    You can use one of these effects only once per turn.
    \paragraph{Requirements} Fighting Style: Archery 1 or Thrown Weapons Fighting 1.
\subsubsection{NAME (2 FP)} \label{feat::name}
    DESCRIPTION
    \paragraph{Requirements} Fighting Style: Archery 1.
\subsubsection{Sniper} \label{feat::sniper}
    Attacking at long range doesn't impose disadvantage on your ranged weapon attack rolls.
    \paragraph{Requirements} Fighting Style: Archery 2.
\subsubsection{Mark} \label{feat::mark}
    You learn the Mark action (see page \pageref{act::mark}).
    \paragraph{Requirements} Fighting Style: Archery 2.
\subsubsection{NAME (2 FP)} \label{feat::name}
    DESCRIPTION
    \paragraph{Requirements} Fighting Style: Archery 2.
% DUELING
\subsubsection{NAME} \label{feat::name}
    DESCRIPTION
    \paragraph{Requirements} Fighting Style: Dueling 1.
\subsubsection{Purposeful Strike} \label{feat::purposefulstrike}
    You learn or improve the Sneak Attack action (see page \pageref{act:sneakattack}).
    You can take this feat three times, adding a d6 to the action's damage each time.

    You can only add these d6s to your Sneak Attack if you are holding a melee weapon on one hand and no other weapons.
    \paragraph{Requirements} Fighting Style: Dueling 1.
\subsubsection{NAME (2 FP)} \label{feat::name}
    DESCRIPTION
    \paragraph{Requirements} Fighting Style: Dueling 1.
\subsubsection{NAME} \label{feat::name}
    DESCRIPTION
    \paragraph{Requirements} Fighting Style: Dueling 2.
\subsubsection{NAME - ACTION} \label{feat::name}
    DESCRIPTION - UPGRADABLE
    \paragraph{Requirements} Fighting Style: Dueling 2.
\subsubsection{NAME (2 FP)} \label{feat::name}
    DESCRIPTION
    \paragraph{Requirements} Fighting Style: Dueling 2.
% GREAT WEAPON FIGHTING
\subsubsection{NAME} \label{feat::name}
    DESCRIPTION
    \paragraph{Requirements} Fighting Style: Great Weapon Fighting 1.
\subsubsection{Cleaving} \label{feat::cleaving}
    When fighting with a weapon with the Heavy property, you can attack two creatures instead of one with a melee attack.
    You can attack in this way only once during your turn.

    You can take this feat two additional times, increasing the number of creatures you can hit with this attack by 1 each time.
    \paragraph{Requirements} Fighting Style: Great Weapon Fighting 1.
\subsubsection{NAME (2 FP)} \label{feat::name}
    DESCRIPTION
    \paragraph{Requirements} Fighting Style: Great Weapon Fighting 1.
\subsubsection{NAME} \label{feat::name}
    DESCRIPTION
    \paragraph{Requirements} Fighting Style: Great Weapon Fighting 2.
\subsubsection{NAME} \label{feat::name}
    DESCRIPTION
    \paragraph{Requirements} Fighting Style: Great Weapon Fighting 2.
\subsubsection{NAME (2 FP)} \label{feat::name}
    DESCRIPTION
    \paragraph{Requirements} Fighting Style: Great Weapon Fighting 2.
% MONASTIC FIGHTING
\subsubsection{Monkey's Fist} \label{feat::monkeysfist}
    The damage of your unarmed strikes is equal to d4 + your Dexterity modifier instead of the normal damage for an unarmed strike.

    In addition, when you use the attack action on your turn you can expend one use of your monastic die to make an unarmed attack as a free action, adding the result of the die to the damage roll.
    \paragraph{Requirements} Fighting Style: Monastic Fighting 1.
\subsubsection{Agile Grappler} \label{feat::agilegrappler}
    You can take this feat three times, gaining different benefits each time:
    \begin{itemize}
        \item When you take the Grapple action, you can use a monastic die to roll on the contest using your Dexterity (Acrobatics) instead of your Strength (Athletics), adding the result of the die to the check.
        \item When you attempt to grapple a creature, you choose what the creature rolls in the contest between Strength (Athletics) and Dexterity (Acrobatics).
        \item When you successfully grapple a creature, you can use another monastic die to make one melee attack against the creature as a free action, adding the result of the die to the damage roll.
    \end{itemize}
    \paragraph{Requirements} Fighting Style: Monastic Fighting 1.
\subsubsection{Flurry of Blows (2 FP)} \label{feat::flurryofblows}
    The multiple attack penalty related to your unarmed weapon attacks is reduced by 3.

    In addition, you can spend one use of your monastic die to make two unarmed weapon attacks instead of one using one action.
    You add the result of the die to the attack roll of both attacks.
    \paragraph{Requirements} Fighting Style: Monastic Fighting 1.
\subsubsection{Blind Fighter} \label{feat::blindfighter}
    You have blindsight with a range of 3 meters.
    Within that range, you can effectively see anything that isn't behind total cover, even if you're blinded or in darkness.
    Moreover, you can see an invisible creature within that range, unless the creature successfully hides from you.
    \paragraph{Requirements} Fighting Style: Monastic Fighting 2.
\subsubsection{Martial Artist} \label{feat::martialartist}
    When you take this feat, you learn one of the following abilities.
    You can take this feat three times, learning a different ability each time.
    \begin{itemize}
        \item \textbf{Deflect Missiles.} You can use your reaction to deflect or catch the missile when you are hit by a ranged weapon attack.
        When you do so, the damage you take from the attack is reduced by 1d10 + your Dexterity modifier.

        If you reduce the damage to 0, you can catch the missile if it is small enough for you to hold in one hand and you have at least one hand free.
        If you catch a missile in this way, you can spend 1 monastic die to make a ranged attack (range 6/18 meters) with the weapon or piece of ammunition you just caught, as part of the same reaction.
        You make this attack with proficiency, regardless of your weapon proficiencies.

        \item \textbf{Diamond Soul.} Whenever you make a saving throw and fail, you can spend one monastic die.
        You add a the number rolled to your saving throw, potentially turning a failure into a success.

        \item \textbf{Stunning Strike.} When you hit a creature with an unarmed attack, you can spend two monastic dice to attempt a stunning strike.
        The target must succeed on a Constitution Saving throw with DC equal to 8 + the result of your monastic dice roll or be stunned until the end of your next turn.
    \end{itemize}

    \paragraph{Requirements} Fighting Style: Monastic Fighting 2.
\subsubsection{Quickened Healing (2 FP)} \label{feat::quickenedhealing}
    Using an actions, you can spend a monastic die to heal yourself.
    You regain a number of hit points equal to the number rolled + your Wisdom modifier.
    \paragraph{Requirements} Fighting Style: Monastic Fighting 2.
% MOUNTED FIGHTING
\subsubsection{Height Superiority} \label{feat::heightsuperiority}
    You have advantage on melee attack rolls against any unmounted creature that is smaller than your mount.
    \paragraph{Requirements} Fighting Style: Mounted Fighting 1.
\subsubsection{NAME - ACTION} \label{feat::name}
    DESCRIPTION - UPGRADABLE
    \paragraph{Requirements} Fighting Style: Mounted Fighting 1.
\subsubsection{NAME (2 FP)} \label{feat::name}
    DESCRIPTION
    \paragraph{Requirements} Fighting Style: Mounted Fighting 1.
\subsubsection{NAME} \label{feat::name}
    DESCRIPTION
    \paragraph{Requirements} Fighting Style: Mounted Fighting 2.
\subsubsection{NAME - ACTION} \label{feat::name}
    DESCRIPTION - UPGRADABLE
    \paragraph{Requirements} Fighting Style: Mounted Fighting 2.
\subsubsection{NAME (2 FP)} \label{feat::name}
    DESCRIPTION
    \paragraph{Requirements} Fighting Style: Mounted Fighting 2.
% PROTECTION
\subsubsection{NAME} \label{feat::name}
    DESCRIPTION
    \paragraph{Requirements} Fighting Style: Protection 1.
\subsubsection{Interception} \label{feat::interception}
    You can take this feat three times, gaining different benefits each time:
    \begin{itemize}
        \item When a creature you can see attacks a target other than you that is within 1.5 meters of you, you can use your reaction to impose disadvantage on the attack roll.
        \item If the attack hits the target, the damage the target takes is reduced by 1d10 + 3 (to a minimum of 0 damage).
        \item When you use this ability, you can make a melee weapon attack against the attacking creature.
    \end{itemize}
    You must be wearing a shield or a martial wepaon to use this reaction.
    \paragraph{Requirements} Fighting Style: Protection 1.
\subsubsection{Sentinel (2 FP)} \label{feat::sentinel}
    When you hit a creature with an opportunity attack, the creature's speed becomes 0 for the rest of its turn.

    In addition, creatures provoke opportunity attacks from you even if they take the Disengage action before leaving your reach.
    \paragraph{Requirements} Fighting Style: Protection 1.
\subsubsection{NAME} \label{feat::name}
    DESCRIPTION
    \paragraph{Requirements} Fighting Style: Protection 2.
\subsubsection{NAME - ACTION} \label{feat::name}
    DESCRIPTION - UPGRADABLE
    \paragraph{Requirements} Fighting Style: Protection 2.
\subsubsection{NAME (2 FP)} \label{feat::name}
    DESCRIPTION
    \paragraph{Requirements} Fighting Style: Protection 2.
% THROWN WEAPON FIGHTING
\subsubsection{Quick Draw} \label{feat::quickdraw}
    You learn the Quick Draw reaction (see page \pageref{act::quickdraw}).
    \paragraph{Requirements} Fighting Style: Thrown Weapons Fighting 1.
\subsubsection{Fast Fingers (2 FP)} \label{feat::fastfingers}
    You don't attack with disadvantage with ranged attacks when a creature is within 1.5 meters of you.

    In addition, you don't provoke attacks of opportunity when unsheathing a weapon or attacking with a ranged weapon when you are within 1.5 of a creature.
    \paragraph{Requirements} Fighting Style: Thrown Weapons Fighting 1.
\subsubsection{Throwing Arm} \label{feat::throwingarm}
    Attacking at long range doesn't impose disadvantage on your thrown weapon attack rolls.
    \paragraph{Requirements} Fighting Style: Thrown Weapons Fighting 2.
\subsubsection{NAME - ACTION} \label{feat::name}
    DESCRIPTION
    \paragraph{Requirements} Fighting Style: Thrown Weapons Fighting 2.
\subsubsection{NAME (2 FP)} \label{feat::name}
    DESCRIPTION
    \paragraph{Requirements} Fighting Style: Thrown Weapons Fighting 2.
% TWO-WEAPON FIGHTING
\subsubsection{Ready for Action} \label{feat::readyforaction}
    You can draw or stow two one-handed weapons when you would normally be able to draw or stow only one.
    \paragraph{Requirements} Fighting Style: Two-Weapon Fighting 1.
\subsubsection{NAME - ACTION} \label{feat::name}
    DESCRIPTION - UPGRADABLE
    \paragraph{Requirements} Fighting Style: Two-Weapon Fighting 1.
\subsubsection{Lock (2 FP)} \label{feat::lock}
    As an action, you can attempt to lock a creature's weapon with yours to prevent it from attacking.
    Using one of your weapons, you try to seize one of your target's weapons by making a lock check, a Dexterity (Sleight of Hand) check contested by the target's Dexterity (Sleight of Hand) or Strength (Athletics) check (the target chooses the ability to use).
    If you succeed, the target can't make attack rolls with the locked weapon.

    At the end of the creature's turns, it can try to end this conditions by succeeding on a Dexterity (Sleight of Hand) or a Strength (Athletics) check contested by your Dexterity (Sleight of Hand) check at the start of its turn.
    The creature can also end the lock at any time by letting go of the weapon.

    In addition, the condition ends if you are incapacitated, if you are removed from the reach of the creature, or if you choose to end the effect (no action required).
    \paragraph{Requirements} Fighting Style: Two-Weapon Fighting 1.
\subsubsection{Offense \& Defense} \label{feat::offenseanddefense}
    You gain a +1 bonus to AC while you are wielding a separate melee weapon in each hand.
    \paragraph{Requirements} Fighting Style: Two-Weapon Fighting 2.
\subsubsection{NAME - ACTION} \label{feat::name}
    DESCRIPTION
    \paragraph{Requirements} Fighting Style: Two-Weapon Fighting 2.
\subsubsection{Two-weapon Master (2 FP)} \label{feat::twoweaponmaster}
    You can use two-weapon fighting even when the one-handed melee weapons you are wielding aren't light.

    In addition, when you are fighting with one weapon in each hand, you reduce all Multiple Attack Penalties applied by 1.
    \paragraph{Requirements} Fighting Style: Two-Weapon Fighting 2.
% UNARMED FIGHTING
\subsubsection{NAME} \label{feat::name}
    DESCRIPTION
    \paragraph{Requirements} Fighting Style: Unarmed Fighting 1.
\subsubsection{NAME - ACTION} \label{feat::name}
    DESCRIPTION - UPGRADABLE
    \paragraph{Requirements} Fighting Style: Unarmed Fighting 1.
\subsubsection{NAME (2 FP)} \label{feat::name}
    DESCRIPTION
    \paragraph{Requirements} Fighting Style: Unarmed Fighting 1.
\subsubsection{Unarmed Artist} \label{feat::unarmedartist}
    You are a painter.
    Your brush is your fist, and your canvas is that guy's face.

    When you get a critical hit with an unarmed strike, you can take the Disarm, Disengage, or Shove action as a free action.
    \paragraph{Requirements} Fighting Style: Unarmed Fighting 2.
\subsubsection{NAME - ACTION} \label{feat::name}
    DESCRIPTION - UPGRADABLE
    \paragraph{Requirements} Fighting Style: Unarmed Fighting 2.
\subsubsection{NAME (2 FP)} \label{feat::name}
    DESCRIPTION
    \paragraph{Requirements} Fighting Style: Unarmed Fighting 2.
% BATTLE MASTERY assessthesituation commander
\subsubsection{NAME} \label{feat::name}
    DESCRIPTION
    \paragraph{Requirements} Fighting Style: Battle Mastery 1.
\subsubsection{Assess the Situation} \label{feat::assessthesituation}
    Right after rolling initiative, you can quickly assess a creature of your choice.
    Make an Intelligence (Investigation) check contested by a Charisma (Deception) check made by the creature.
    On a success, you learn the creature's vulnerabilities, resistances, and immunities, if it has any.

    You can take this feat three times, increasing the number of creature you can use this ability with by 1 the second and third times.
    \paragraph{Requirements} Fighting Style: Battle Mastery 1.
\subsubsection{NAME (2 FP)} \label{feat::name}
    DESCRIPTION
    \paragraph{Requirements} Fighting Style: Battle Mastery 1.
\subsubsection{Commander} \label{feat::commander}
    You have advantage on all Charisma (Deception), Charisma (Intimidation), Charisma (Performance), and Charisma (Persuasion) checks while in combat.
    \paragraph{Requirements} Fighting Style: Battle Mastery 2.
\subsubsection{NAME - ACTION} \label{feat::name}
    DESCRIPTION - UPGRADABLE
    \paragraph{Requirements} Fighting Style: Battle Mastery 2.
\subsubsection{NAME (2 FP)} \label{feat::name}
    DESCRIPTION
    \paragraph{Requirements} Fighting Style: Battle Mastery 2.

% === EXOTIC WEAPONS ===============================================================================
% BLOWGUNS blowgunmaster hiddenshooter poisoneddart stronglungs
\subsubsection{Blowgun Master} \label{feat::blowgunmaster}
    You increase your proficiency level with blowguns.
    This feat can be taken three times, or until you reach Expert proficiency with the weapon type.
\subsubsection{Hidden Shooter} \label{feat::hiddenshooter}
    Making a ranged attack with a blowgun while hidden doesn't reveal your location, even if the attack is a hit.
    \paragraph{Requirements} Competent proficiency with Blowguns.
\subsubsection{Poisoned Dart} \label{feat::poisoneddart}
    When you hit with a blowgun with a dart that has poison applied, the target has disadvantage on saving throws against the poison.
    \paragraph{Requirements} Skilled proficiency with Blowguns.
\subsubsection{Strong Lungs (2 FP)} \label{feat::stronglungs}
    Your blowgun deals 1d4 piercing damage instead of 1.

    In addition, attacking with a blowgun at long range doesn't impose disadvantage on your ranged weapon attack rolls.
    \paragraph{Requirements} Expert proficiency with Blowguns.
% FLAILS
\subsubsection{Flails Proficiency} \label{feat::name}
    DESCRIPTION
\subsubsection{NAME} \label{feat::name}
    DESCRIPTION
    \paragraph{Requirements} Competent proficiency with Flails.
\subsubsection{NAME - ACTION} \label{feat::name}
    DESCRIPTION
    \paragraph{Requirements} Skilled proficiency with Flails.
\subsubsection{NAME (2 FP)} \label{feat::name}
    DESCRIPTION
    \paragraph{Requirements} Expert proficiency with Flails.
% NETS & BOLAS
\subsubsection{Nets \& Bolas Proficiency} \label{feat::name}
    DESCRIPTION
\subsubsection{NAME} \label{feat::name}
    DESCRIPTION
    \paragraph{Requirements} Competent proficiency with Nets \& Bolas.
\subsubsection{NAME - ACTION} \label{feat::name}
    DESCRIPTION
    \paragraph{Requirements} Skilled proficiency with Nets \& Bolas.
\subsubsection{NAME (2 FP)} \label{feat::name}
    DESCRIPTION
    \paragraph{Requirements} Expert proficiency with Nets \& Bolas.
% WHIPS
\subsubsection{Whips Proficiency} \label{feat::name}
    DESCRIPTION
\subsubsection{NAME} \label{feat::name}
    DESCRIPTION
    \paragraph{Requirements} Competent proficiency with Whips.
\subsubsection{NAME - ACTION} \label{feat::name}
    DESCRIPTION
    \paragraph{Requirements} Skilled proficiency with Whips.
\subsubsection{NAME (2 FP)} \label{feat::name}
    DESCRIPTION
    \paragraph{Requirements} Expert proficiency with Whips.

% CHAMPION: get critical hits from rolls of 19 or 20 instead of only 20. (2 FP)

% % === ARMOR ==================================================================== %
% \subsubsection{Stealthy} \label{feat::stealthy}
% \small{\textcolor{gray}{Stealth}}
%
% \normalsize
% You know how best to hide, and use your dyed light armor to benefit your stealth.
% \paragraph{REQUIREMENTS} Lightly Armored 2 and Sly 2.
% \paragraph{RANK 1} You can take the Hide action as a bonus action on each of your turns.
% \paragraph{RANK 2} If you are hidden, you can move up to 3 meters in the open without revealing yourself if you end the move in a position where you're not clearly visible.
% \paragraph{RANK 3}
% % ============================================================================== %
% \subsubsection{Drunken Brawler} \label{feat::drunkenbrawler} %
% \small{\textcolor{gray}{}}
% % Skills emulating drunken fighting AND actual bonus when drunk.
%
% \normalsize
% Description.
% \paragraph{REQUIREMENTS}
% \paragraph{RANK 1}
% \paragraph{RANK 2} When you suffer from the effects of drunkenness, you have advantage on Constitution and Strength saving throws.
% \paragraph{RANK 3} You learn the Careless Deflect technique.

% ============================================================================== %
\subsubsection{Ball-and-Chain Master} \label{feat::ballandchainmaster}
\small{\textcolor{gray}{Strength}}

\normalsize
Your increased awareness works in tandem to your flail mastery, allowing you to hit enemies behind obstacles and shields.
\paragraph{REQUIREMENTS} Flail Adept 2 and Insightful 2.
\paragraph{RANK 1} When you use a ball-and-chain, its damage die changes from a d6 to a d8.
\paragraph{RANK 2} When you hit with an opportunity attack using a ball-and-chain, the target must succeed on a Strength saving throw (DC 8 + your proficiency bonus + your Strength modifier) or be knocked prone.
\paragraph{RANK 3} You learn the Shield Sweep technique.

% ============================================================================== %
\subsubsection{Blade Breaker} \label{feat::bladebreaker}
\small{\textcolor{gray}{Dexterity}}

\normalsize
You know that the perfect counter against any sword is simply a long pole.
\paragraph{REQUIREMENTS} Staff Fighter 2.
\paragraph{RANK 1} Staves have the Reach property when wielded by you.
\paragraph{RANK 2} You have advantage on the Disarm action and the Parry technique against creatures using any type of sword.
\paragraph{RANK 3} While using a polearm, you can use the Disarm action as your opportunity attack.

% ============================================================================== %
\subsubsection{Blunt Master} \label{feat::bluntmaster}
\small{\textcolor{gray}{Dexterity}}

\normalsize
Your staff is an extension of your body, and it has become an essential part of your movement and defense.
\paragraph{REQUIREMENTS} Staff Fighter 2 and Unarmed Artist 2.
\paragraph{RANK 1} As a bonus action, you can use your staff to propel yourself into the air, making a high jump up to your Dexterity modifier times 1.5 meters.
Additionally, if you have moved at least 3 meters during this round, you can jump in a straight line up to a distance equal to your Dexterity modifier times 3 meters.
\paragraph{RANK 2} You gain a +1 bonus to your AC while using a staff.
Additionally, if you use the weapon with two hands you gain another +1 bonus to your AC.

% ============================================================================== %
\subsubsection{Bullet Tinkerer} \label{feat::bullettinkerer}
\small{\textcolor{gray}{Science}}

\normalsize
Firearm ammunition is near impossible to find or purchase in most parts of Yuadrem.
Due to this, learning how to craft it yourself is an essential aspect for using firearms.
\paragraph{REQUIREMENTS} Musket Adept 2 and Educated 2, or Pistol Adept 2 and Educated 2.
\paragraph{RANK 1} You can craft ammunition using a set of Tinker's Tools at half the cost.
\paragraph{RANK 2} If you roll a misfire, you can use your reaction to roll a d20 with disadvantage.
If the number rolled is higher than the weapon's misfire score, the weapon does not misfire.
\paragraph{RANK 3} You learn the Violent Shot technique.

% ============================================================================== %
\subsubsection{Far Thruster} \label{feat::farthruster}
\small{\textcolor{gray}{Strength or Dexterity}}

\normalsize
You take advantage of your long weapon to keep your distance from foes.
\paragraph{REQUIREMENTS} Halberd Adept 2 and Spear Adept 2.
\paragraph{RANK 1} After successfully attacking a creature, you can use the Shove action as a bonus action.
\paragraph{RANK 2} You gain a +2 to your attack rolls made with a polearm against a creature larger than you.
\paragraph{RANK 3} You learn the Extend Reach technique.

% ============================================================================== %
\subsubsection{Fell Hand} \label{feat::fellhand}
\small{\textcolor{gray}{Strength}}

\normalsize
You are an expert with the greatclub, and can use its raw power to pancake even the hardiest foes.
\paragraph{REQUIREMENTS} Hammer Adept 2.
\paragraph{RANK 1} When you use a greatclub, its damage die changes from a d8 to a d10.
\paragraph{RANK 2} Whenever you miss with a melee attack using a greatclub, the target takes bludgeoning damage equal to your Strength modifier (minimum of 2).
\paragraph{RANK 3} When your target fails its saving throw against your Bonk technique, it also falls prone.

% ============================================================================== %
\subsubsection{Firearm Specialist} \label{feat::firearmspecialist}
\small{\textcolor{gray}{Dexterity}}

\normalsize
You are particularly well at combat with pistols, and are a shining example of this new weapon's viability in combat.
\paragraph{REQUIREMENTS} Pistol Adept 2.
\paragraph{RANK 1} You can reload any weapon as a bonus action.
\paragraph{RANK 2} If you are using a pistol in one hand and nothing on the other, you get a +2 bonus to your ranged weapon attacks with it.
\paragraph{RANK 3} You learn the Rapid Repair technique.

% ============================================================================== %
\subsubsection{Flail Adept} \label{feat::flailadept}
\small{\textcolor{gray}{Strength or Dexterity}}

\normalsize
The flail is a tricky weapon to use, but you have spent countless hours mastering it.
\paragraph{REQUIREMENTS} Armed Fighter 2.
\paragraph{RANK 1} You are proficient with flails.
\paragraph{RANK 2} When wielding flails, enemies provoke opportunity attacks from you when they enter your reach.
\paragraph{RANK 3} You learn the Sweep technique.

% ============================================================================== %
\subsubsection{Giant Slayer} \label{feat::giantslayer}
\small{\textcolor{gray}{Strength}}

\normalsize
Partaking or not in Krudzal's Eternal War, you use your incredible strength to wield their massive weapons.
\paragraph{REQUIREMENTS} Strength 16. Great Weapon User 2.
\paragraph{RANK 1} You are able to use Krudzal's giant-slayers.
\paragraph{RANK 2} You gain a +1 bonus to damage rolls with weapons with the Great property.

% ============================================================================== %
\subsubsection{Glaive Master} \label{feat::glaivemaster}
\small{\textcolor{gray}{Dexterity}}

\normalsize
Different than the common polearm, you know how deadly a glaive can be in capable hands.
\paragraph{REQUIREMENTS} Halberd Adept 2.
\paragraph{RANK 1} When you use a glaive, its damage die changes from a d10 to a d12.
\paragraph{RANK 2} The glaive has the finesse property for you.
\paragraph{RANK 3} When you miss an attack, you can attempt to strike your enemy with the rondel of your glaive as a free action.
On a hit, the target takes 1d4 + your Strength modifier bludgeoning damage.
You can do this once per turn.

% ============================================================================== %
\subsubsection{Grappler} \label{feat::grappler}
\small{\textcolor{gray}{Strength}}

\normalsize
You've developed the skills necessary to hold your own in close-quarters grappling.
\paragraph{REQUIREMENTS} Pugilist 2.
\paragraph{RANK 1} You have advantage on attack rolls against creatures you are grappling.
\paragraph{RANK 2} You benefit from three-quarters cover while you grapple a creature of a size equal or greater than yours.
\paragraph{RANK 3} You learn the Pin technique.

% ============================================================================== %
\subsubsection{Great Weapon User} \label{feat::greatweaponuser}
\small{\textcolor{gray}{Strength}}

\normalsize
You've learned to put the weight of a weapon to your advantage, letting its momentum empower your strikes.
\paragraph{REQUIREMENTS} Strength 13. Armed Fighter 2.
\paragraph{RANK 1} You can use weapons with the great property.
Some of these, like the zweihander and the greataxe, also require you to have proficiency with the corresponding weapon type.
\paragraph{RANK 2} On your turn, when you score a critical hit with a melee weapon or reduce a creature to 0 hit points with one, you can make one melee weapon attack as a bonus action.
\paragraph{RANK 3} You learn the Reckless Strike technique.

% ============================================================================== %
\subsubsection{Gunner} \label{feat::gunner}
\small{\textcolor{gray}{Dexterity}}

\normalsize
Being a new breed of weapon, firearms are constantly subject to change and refinement.
New weapons appear by the minute, and you are an expert at experimenting with them.
\paragraph{REQUIREMENTS} Pistol Adept 2 and Musket Adept 2.
\paragraph{RANK 1} As long as you can examine the weapon for 30 seconds, you are proficient with any kind of firearm, even if it is new or experimental.
\paragraph{RANK 2} Your firearm attacks score a critical hit on a roll of 19-20.
\paragraph{RANK 3} You learn the Reckless Shot technique.

% ============================================================================== %
\subsubsection{Halberd Adept} \label{feat::halberdadept}
\small{\textcolor{gray}{Strength or Dexterity}}

\normalsize
You know that both farmers and nobles use halberds as their weapon of choice for a reason.
\paragraph{REQUIREMENTS} Armed Fighter 2 and Staff Fighter 2.
\paragraph{RANK 1} You are proficient with the many varieties of halberds.
\paragraph{RANK 2} When a creature within 1.5 meters of you makes an attack against a target other than you, you can use your reaction to make a melee weapon attack against the attacking creature.
\paragraph{RANK 3} You learn the Trip technique.

% ============================================================================== %
\subsubsection{Hammer Adept} \label{feat::hammeradept}
\small{\textcolor{gray}{Strength}}

\normalsize
If a hammer will do, why complicate things?
\paragraph{REQUIREMENTS} Armed Fighter 2.
\paragraph{RANK 1} You gain proficiency with hammer weapons.
\paragraph{RANK 2} If you use the Help action to aid an ally's melee attack while you're wielding a hammer weapon, you knock the target's shield aside momentarily.
In addition to the ally gaining advantage on the attack roll, the ally gains a bonus to the roll equal to the shield's AC bonus.
\paragraph{RANK 3} You learn the Bonk technique.

% ============================================================================== %
\subsubsection{Heavy Lifter} \label{feat::heavylifter}
\small{\textcolor{gray}{Strength}}

\normalsize
You can easily lift and hurl objects that others can barely move.
\paragraph{REQUIREMENTS} Combat Improviser 2 and Thrown Weapon Master 2.
\paragraph{RANK 1} You can hurl any object or creature you can carry or grapple as you would an improvised weapon.
If you hurl a creature, both it and the target of your attack take 1d6 + your Strength modifier bludgeoning damage.
\paragraph{RANK 2} You count as if you were one size larger for the purpose of determining your carrying capacity.
\paragraph{RANK 3} You learn the Suplex technique.

% ============================================================================== %
\subsubsection{Horse Killer} \label{feat::horsekiller}
\small{\textcolor{gray}{Animal Handling}}

\normalsize
Mounts are one of the most common advantages in war, so you've learned to deal with them appropriately.
\paragraph{REQUIREMENTS} Spear Adept 2 and Animal Handler 2.
\paragraph{RANK 1} You can always target a rider's mount with an attack made with a spear, even if the rider can normally deflect this attack to it.
\paragraph{RANK 2} You have a +2 to your attack rolls made against mounts.
\paragraph{RANK 3} You learn the Charge Stopper technique.

% ============================================================================== %
\subsubsection{Insistent Fighter} \label{feat::insistentfighter}
\small{\textcolor{gray}{Strength or Dexterity}}

\normalsize
You always maintain your foes at a cozy distance by masterfully manipulating your halberd.
\paragraph{REQUIREMENTS} Halberd Adept 2 and Observant 2.
\paragraph{RANK 1} Creature's provoke opportunity attacks from you even if they take the Disengage action before leaving your reach.
\paragraph{RANK 2} When you hit a creature with an opportunity attack, the creature's speed becomes 0 for the rest of the turn.
\paragraph{RANK 3} Your reach for opportunity attacks with polearms is of at least 3 meters.
Additionally, if you do hit a creature with an opportunity attack, you can choose to pull it towards you by 1.5 meters.

% ============================================================================== %
\subsubsection{Kama Master} \label{feat::kamamaster}
\small{\textcolor{gray}{Dexterity}}

\normalsize
From among the common people weapon, you specialize in the use of the kama.
\paragraph{REQUIREMENTS} Pick Adept 2.
\paragraph{RANK 1} When you use a kama, its damage die changes from a d6 to a d8
\paragraph{RANK 2} When you successfully attack a foe with a piercing attack using your kama, you also deal 1d4 slashing damage, unmodified by your dexterity.
\paragraph{RANK 3} You can use the Trip technique as part of an opportunity attack.

% ============================================================================== %
\subsubsection{Longsword Master} \label{feat::longswordmaster}
\small{\textcolor{gray}{Strength or Dexterity}}

\normalsize
Your proficiency with the longsword is unparalleled, and you regularly prove to your foes that the old sword is as valid today as it was on its golden age.
\paragraph{REQUIREMENTS} Straight Sword Adept 2.
\paragraph{RANK 1} When you use a longsword, its damage die changes from a d10 to a d12.
\paragraph{RANK 2} When a creature misses you with a melee weapon attack, you can roll to use the Disarm action on it as your reaction.
\paragraph{RANK 3} When you use the Block technique and an attack misses you, you can use the Riposte technique without using your reaction.

% ============================================================================== %
\subsubsection{Medium Haft Master} \label{feat::mediumhaftmaster}
\small{\textcolor{gray}{Strength}}

\normalsize
Why learn how to fight with a sword when your tools will do the job?
\paragraph{REQUIREMENTS} Axe Adept 2, Hammer Adept 2, and Pick Adept 2.
\paragraph{RANK 1} You gain a +1 bonus to attack rolls you make with axes, hammers, and picks.
\paragraph{RANK 2} You can treat any weapon as if it had the Heavy property.
\paragraph{RANK 3} You learn the Shield Break technique.

% ============================================================================== %
\subsubsection{Modern Soldier} \label{feat::modernsoldier}
\small{\textcolor{gray}{Strength or Dexterity}}

\normalsize
Always up-to-date, you are ready to fight your foes with both a melee and a ranged weapon.
\paragraph{REQUIREMENTS} Straight Sword Adept 2 and Crossbow Adept 2.
\paragraph{RANK 1} When you use the Attack action and attack with a one-handed weapon, you can use your bonus action to attack with a hand crossbow or loaded firearm you are holding.
\paragraph{RANK 2} When using a backsword or a sabre on one hand and your other hand is free, you can hold the back of the blade with your free hand to increase its cutting power.
You add a +2 to your melee weapon attack damage while wielding your weapon in this way.

% ============================================================================== %
\subsubsection{Musket Adept} \label{feat::musketadept}
\small{\textcolor{gray}{Dexterity}}

\normalsize
Quickly picking up on new trends, you gambled on muskets and their stopping power early on.
\paragraph{REQUIREMENTS} Armed Fighter 2.
\paragraph{RANK 1} You are proficient with muskets.
\paragraph{RANK 2} While you have a two-handed firearm equipped, you have advantage on rolls against effects that would move you.
\paragraph{RANK 3} You learn the Buttstroke technique.

% ============================================================================== %
\subsubsection{Pick Adept} \label{feat::pickadept}
\small{\textcolor{gray}{Strength or Dexterity}}

\normalsize
For you, there is no difference between mining ore and piercing flesh.
Apart from the screams at least.
% You pierce through metal, flesh, and bone just as you would ore.
\paragraph{REQUIREMENTS} Armed Fighter 2.
\paragraph{RANK 1} You are proficient with picks.
\paragraph{RANK 2} All picks have the versatile property for you.
When you wield a pick with two hands, increase its damage die by one (d6 to d8, d8 to d10, etc), and have the heavy property.
\paragraph{RANK 3} You learn the Trip technique.

% ============================================================================== %
\subsubsection{Pike Master} \label{feat::pikemaster}
\small{\textcolor{gray}{Strength}}

\normalsize
You prefer to maintain distance from your opponents by wielding an amusingly long pole.
\paragraph{REQUIREMENTS} Spear Adept 2.
\paragraph{RANK 1} When you use a pike, its damage die changes from a d10 to a d12.
\paragraph{RANK 2} When you use the Feint technique you can choose to not attack with advantage.
If you do this and still hit your target, it's a critical hit.
\paragraph{RANK 3} When wielding a pike, its reach property adds 3 meters instead of 1.5 meters to the weapon's range.

% ============================================================================== %
\subsubsection{Polearm Master} \label{feat::polerarmmaster} %
\small{\textcolor{gray}{Dexterity}}

\normalsize
You keep your enemies at bay with your long weapons.
\paragraph{REQUIREMENTS} Staff Fighter 2, Spear Adept 2, and Halberd Adept 2.
\paragraph{RANK 1} You gain a +1 bonus to attack rolls you make with staves, spears, and halberds.
\paragraph{RANK 2} While wielding a stave, spear, or halberd, other creatures provoke an opportunity attack from you when they enter your reach.
\paragraph{RANK 3} You learn the Rondel Hit technique.

% ============================================================================== %
\subsubsection{Pugilist} \label{feat::pugilist}
\small{\textcolor{gray}{Strength}}

\normalsize
Your fighting style has only three components:
A fist, another fist, and a face to punch.
\paragraph{RANK 1} Your unarmed strike uses a d4 + your Strength modifier for damage.
\paragraph{RANK 2} When you hit a creature with an unarmed strike, you can use a bonus action to attempt to grapple the target.
\paragraph{RANK 3} You learn the Bonk technique.

% ============================================================================== %
\subsubsection{Revenant Blade} \label{feat::revenantblade}
\small{\textcolor{gray}{Dexterity}}

\normalsize
You are a expert of the double-bladed scimitar, and use the technique as proficiently as any master from the ancient Janchu'ut school of hunters.
\paragraph{REQUIREMENTS} Curved Blade Adept 2.
\paragraph{RANK 1} A double-bladed scimitar has the finesse property when you wield it.
\paragraph{RANK 2} While you are holding a double-bladed scimitar with two hands, you gain a +1 bonus to Armor Class.
\paragraph{RANK 3} Instead of dealing 1d4 slashing damage, the extra attack you can do with a double-bladed scimitar deals 2d4 slashing damage.

% ============================================================================== %
\subsubsection{Sharpshooter} \label{feat::sharpshooter}
\small{\textcolor{gray}{Dexterity}}

\normalsize
You can strike a foe with an arrow or bullet no matter where they are in the battlefield.
\paragraph{REQUIREMENTS} Bow Adept 2 or Musket Adept 2.
\paragraph{RANK 1} If you do not move during your turn, you can make an additional ranged weapon attack as a bonus action.
\paragraph{RANK 2} Both your normal and long ranges are multiplied by 1.5 for all ranged weapons..
\paragraph{RANK 3} Whenever you have advantage on a ranged attack roll using Dexterity, you can reroll one of the dice once.

% ============================================================================== %
\subsubsection{Shooter} \label{feat::shooter}
\small{\textcolor{gray}{Dexterity}}

\normalsize
You have mastered ranged weapons and can make shots that others find impossible.
\paragraph{REQUIREMENTS} Proficiency with at least two martial ranged weapon types.
\paragraph{RANK 1} Your ranged attacks ignore half cover and three-quarters covers.
\paragraph{RANK 2} Increase your Dexterity by 1, to a maximum of 20.
\paragraph{RANK 3} You learn the Leg Shot technique.

% ============================================================================== %
\subsubsection{Spear Adept} \label{feat::spearadept}
\small{\textcolor{gray}{Strength or Dexterity}}

\normalsize
Trained in the art of the spear, you know that nothing beats a long pole with a pointy end.
\paragraph{REQUIREMENTS} Armed Fighter 2.
\paragraph{RANK 1} You are proficient with spears.
\paragraph{RANK 2} Spears have the finesse property for you.
\paragraph{RANK 3} You learn the Feint technique.

% ============================================================================== %
\subsubsection{Staff Fighter} \label{feat::stafffighter}
\small{\textcolor{gray}{Dexterity}}

\normalsize
You have mastered the nuances of fighting with a staff.
\paragraph{REQUIREMENTS} Armed Fighter 1.
\paragraph{RANK 1} The staff counts as a finesse weapon for you.
\paragraph{RANK 2} While wielding a staff, you can use your bonus action to give yourself advantage on your next Dexterity (Acrobatics) or Strength (Athletics) check related to climbing, jumping, or otherwise bypassing obstacles.
This benefit lasts until the end of your turn.
\paragraph{RANK 3} You learn the Sweep technique.

% ============================================================================== %
\subsubsection{Stone Piercer} \label{feat::stonepiercer}
\small{\textcolor{gray}{Strength}}

\normalsize
You know that the difference between piercing stone and crushing bone lies only in the morals of the aggressor.
\paragraph{REQUIREMENTS} Pick Adept 2 and Athlete 2.
\paragraph{RANK 1} You double your damage dice against objects and structures while attacking with a pick.
Additionally, you have advantage on checks made with a climber's kit and pitons.
\paragraph{RANK 2} When you score a critical hit that deals piercing damage to a creature, you can roll one additional damage die when determining the extra piercing damage the target takes.
\paragraph{RANK 3} You learn the Injure technique.

% ============================================================================== %
\subsubsection{Sword Master} \label{feat::swordmaster}
\small{\textcolor{gray}{Strength or Dexterity}}

\normalsize
You have an insight unique to those who master the many types of swords.
Your skill with the long blades is unparalleled, and your resoluteness in combat scares even the bravest of warriors.
\paragraph{REQUIREMENTS} Straight Sword Adept 2, Curved Sword Adept 2, and Light Sword Adept 2.
\paragraph{RANK 1} You gain a +1 bonus to attack rolls you make with swords.
\paragraph{RANK 2} When you make an opportunity attack with a sword, you have advantage on the attack roll.

% ============================================================================== %
\subsubsection{Unarmed Artist} \label{feat::unarmedartist}
\small{\textcolor{gray}{Dexterity}}

\normalsize
You are a painter.
Your hands are your brush, your opponent your canvas.
\paragraph{RANK 1} Your unarmed strike uses a d4 + your Dexterity modifier for damage.
You can only use this ability if you're not using a fist weapon.
\paragraph{RANK 2} When you use the Attack action, you can make an unarmed strike as a bonus action.
\paragraph{RANK 3} You get access to one Martial Arts technique of your choice.
You may buy this rank again to attain new Martial Arts techniques as long as you have not learned all available techniques.

% ============================================================================== %
\subsubsection{Way of the Fist} \label{feat::wayofthefist}
\small{\textcolor{gray}{Strength or Dexterity}}

\normalsize
Either by meditative training or drunken fighting, you are a master with your fists.
\paragraph{REQUIREMENT} Pugilist 2 and Unarmed Artist 2.
\paragraph{RANK 1} Your unarmed strike uses a d6 + your Strength or Dexterity modifier for damage.

% ============================================================================== %
\subsubsection{Whip Adept} \label{feat::whipadept}
\small{\textcolor{gray}{Dexterity}}

\normalsize
You are learned in the use of the whip, using your unique weapon for both combat and support.
\paragraph{REQUIREMENTS} Armed Fighter 2.
\paragraph{RANK 1} You can use your bonus action to make one melee attack with an equipped whip.
\paragraph{RANK 2} You can use a whip to extend the range of the Grapple and Shove actions to the range of the weapon.
\paragraph{RANK 3} You can use your whip as an elongated appendage to perform actions that do not require fine motor skills.
This includes object interactions, such as grabbing a sword or pulling a bag towards you, and ability checks, such as an acrobatics roll to grab a nearby support beam to swing from.

% ============================================================================== %
\subsubsection{Wrestler} \label{feat::wrestler}
\small{\textcolor{gray}{Strength}}

\normalsize
Every part of your body is a weapon and you wield it expertly.
\paragraph{REQUIREMENTS} Grappler 2 and Combat Improviser 2.
\paragraph{RANK 1} Once per round, as part of your action, you many deal your Strength modifier in damage to any creature you are grappling.
\paragraph{RANK 2} You can use a grappled creature of a size equal or lower to yours as an improvised weapon.
\paragraph{RANK 3} You learn the Charge technique.

\subsubsection{Sniper} \label{feat::sniper}
\paragraph{RANK 3} You learn the Hemorrhaging Attack technique.

\paragraph{RANK 3} You learn the Armor Puncture technique.

% TODO: Raise a shield as an action/reaction, increasing its AC bonus by +2.
% TODO: Take cover behind a heavy shield, giving 1/2 or 3/4 cover.
% TODO: For monastic combat:
% \subsubsection{Light as a Feather} \label{mtec::lightasafeather}
% You gain the ability to move across liquids on your turn without falling during the move.

% % !TEX root = ../main.tex
\addcontentsline{toc}{section}{Spellcasting Feats}
\subsection*{Spellcasting Feats}

% === MAGIC SCHOOLS ============================================================ %
% All schools of magic require at least one rank in SPELLCASTING and one rank in a skill or tool (depends on the school).
% Spellcasting
% Bonereading (bonecarving tools)
% Wordbinding (two ranks in standard language - second is basic true speech)
% Windherding (acrobatics)
% Sigaldry (two ranks in naenk tongue - first to learn language, second to learn shinerunes)
% Psionics (two ranks in mind speech - first is listening to zaloths that don't necessarily want to be heard, second is feelspeech, and third is actual mind speech)
% Thaumaturgy (science)
% Tidal Manipulation (religion)
% Fleshshaping (nature)

% NOTE: FROM SPELLCASTING STYLES
% \subsubsection{Spell Sniper}
% You have learned techniques to enhance your attacks with ranged spells, gaining the following benefits:
% \begin{itemize}
%     \item When you cast a spell that requires you to make an attack roll, the spell's range is doubled.
%     \item Your ranged spell attacks ignore half cover and three-quarters cover.
%     \item You learn one cantrip that requires an attack roll. Choose the cantrip from the bard, cleric, druid, sorcerer, warlock, or wizard spell list. Your spellcasting ability for this cantrip depends on the spell list you chose from: Charisma for bard, sorcerer, or warlock; Wisdom for cleric or druid; or Intelligence for wizard.
% \end{itemize}

\subsubsection{Spellcaster (2 FP)} \label{feat::spellcaster}
    By joining one spellcasting school of your choice, you learn how to cast spells.
    Pick a spellcasting school from the Spellcasting chapter (page \pageref{ch::spellcasting}).
    The benefits associated to each school are denoted on their respective sections.
    For the rules of spellcasting, see page \pageref{sec::spellcastingrules}.

    When you take this feat, you learn one cantrip and two level 1 spells of your choice.
    The spells must be from your chosen spellcasting doctrine.
    You learn one additional cantrip when you reach levels 4 and 10.

    Whenever you reach a level that grants an Ability Score Improvement, you can replace one cantrip gained from this feat with another cantrip.
    Cantrips from your spellcasting school cannot be replaced.

\subsubsection{Avid Spellcaster (2 FP)} \label{feat::avidspellcaster}
    You increase your spellcasting ability bonus by 2.
    You can take this feat two times.


% TODO: Add a feat to extend the range of the Help action.
% TODO: Feat that allows you to temporarily learn a spell after identifying it.
% TODO: Feat to do a 1 meters roll before standing up, movement doesn't produce opportunity attacks but standing up does.

% % ============================================================================== %
% \subsection*{Heroic Feats}
% \subsubsection{Demented Insight} \label{feat::dementedinsight}
% \small{\textcolor{gray}{-}}
%
% \normalsize
% A true master of the mind, you are able to retain your sentience without a qualar.
% You are immune to the effects of dementia.
% \paragraph{REQUIREMENTS} Succesfully recover from the dementia status (page \pageref{ssec::dementia}).

% --- NOTE: OLD FEAT SCHEMATIC ---------------------------------------------------------------------
% \subsubsection{Feat Name}
% \small{\textcolor{gray}{Relevant Ability Score or Skill}}
%
% \normalsize
% Description.
% \paragraph{REQUIREMENTS} Other ranks that need to be attained before attaining this one.
% Left blank if feat doesn't require anything.
% --------------------------------------------------------------------------------------------------


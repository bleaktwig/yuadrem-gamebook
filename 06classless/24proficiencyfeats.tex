% !TEX root = ../main.tex
\addcontentsline{toc}{section}{Proficiency Feats}
\subsection*{Proficiency Feats}

\subsection*{Tool Proficiencies}
    You can use a set of Artisan's Tools to craft items as part of a long rest.
    Make an ordered list of the items you want to make.
    The items you have access to depend on your proficiency level with the tools, and are detailed in each tools' associated proficiency feat.
    Each item in the list has an associated DC and cost, which is equal to its DC and cost plus these values for all the previous items on the list.

    The DC of an item depends on its rarity and its associated proficiency.
    They are listed in the tools' proficiency feat.
    Common items have a DC of 6, Uncommon items have a DC of 9, Rare items a DC of 12, Very Rare a DC of 15, and legendary a DC of 20.

    After you finish the list, make an ability check with the relevant toolset.
    The number you roll determines how far down the list you are able to craft, and the cost you need to pay for components.

% ALCHEMIST'S SUPPLIES alchemist experimentalelixir fromgreattobest potionperuser restorativereagents
\subsubsection{Alchemist} \label{feat::alchemist}
    Increase your level of proficiency with alchemist's supplies.
    This feat can be re-taken until you are an expert with the toolset.

    You add your Intelligence modifier to checks with Alchemists' Supplies.

    The cost and DC of potions relates to one vial or flask of the substance, as is detailed in the corresponding section.
\subsubsection{Potion Peruser} \label{feat::potionperuser}
    Using two actions, you can identify one potion within 1 meters of you, as if you had tasted it.
    You must see the liquid for this benefit to work.
    \paragraph{Requirements} Competent proficiency with alchemist's supplies.
\subsubsection{Restorative Reagents} \label{feat::restorativereagents}
    Whenever a creature drinks a potion you created, the creature gains temporary hit points equal to 2d6 + your proficiency bonus with alchemist's supplies.
    \paragraph{Requirements} Skilled proficiency with alchemist's supplies.
\subsubsection{From Great to Best} \label{feat::fromgreattobest}
    Over the course of a short rest, you can temporarily improve the potency of one potion of any rarity.
    % To use this benefit, you must have alchemist's supplies with you, and the potion must be within reach.
    If the potion is drunk during the day after the short rest ends, ignore the potion's die roll (if it has any), automatically rolling the maximum possible number.

    You can take this feat three times, increasing the number of potions you can improve as part of the short rest by one each time.
    \paragraph{Requirements} Skilled proficiency with alchemist's supplies.
\subsubsection{Experimental Elixir (2 FP)} \label{feat::experimentalelixir}
    Whenever you finish a short rest, you can produce two experimental elixirs in an empty flask you touch.
    Roll on the Experimental Elixir table for the elixir's effect, which is triggered when someone drinks the elixir.
    Using two actions, a creature can drink the elixir or administer it to an incapacitated creature.

    Creating an experimental elixir requires you to have alchemist's supplies on your person, and any elixir you create with this feature lasts until it is drunk or until the end of your next short rest.

    If you gain Legendary proficiency with alchemist's supplies, can make one more elixir with this ability.

    \begin{DndTable}[width=\linewidth, header=Experimental Elixir]{lX}
        \textbf{d6} & \textbf{Effect} \\
        1 & \textbf{Healing.}
        The drinker regains a number of hit points equal to 2d4 + half your proficiency bonus with alchemist's supplies. \\
        2 & \textbf{Swiftness.}
        The drinker's walking speed increases by 2 meters for 1 hour. \\
        3 & \textbf{Resilience.}
        The drinker gains a +1 bonus to AC for 10 minutes. \\
        4 & \textbf{Boldness.}
        The drinker can roll a d4 and add the number rolled to every attack roll and saving throw they make for the next minute. \\
        5 & \textbf{Flight.}
        The drinker gains a flying speed of 2 meters for 10 minutes. \\
        6 & \textbf{Transformation.}
        The drinker's body is transformed as if by the alter self spell (see page \pageref{spell::alterself}).
        The drinker determines the transformation caused by the spell, the effects of which last for 10 minutes.
    \end{DndTable}
    \paragraph{Requirements} Expert proficiency with alchemist's supplies.

% BREWER'S SUPPLIES brewer chug mixeddrinks purify theenhancer
\subsubsection{Brewer} \label{feat::brewer}
    Increase your level of proficiency with brewer's supplies.
    This feat can be re-taken until you are an expert with the toolset.

    You add your Constitution modifier to checks with brewer's supplies.

    The cost associated to each item only accounts for the required reagents, not the barrels used to contain the liquid or the tankard used to drink it.
    Unless specified otherwise in the item's name, the DC of a brew relates to one small keg of the stuff, which fills 8 tankards before being drained.
\subsubsection{Purify} \label{feat::purify}
    Using brewing techniques you can purify dirty water with your brewer's supplies.
    As part of a short rest, you can purify doses of water equal to 4 + your proficiency bonus with brewer's supplies.
    \paragraph{Requirements} Competent proficiency with brewer's supplies.
\subsubsection{Chug!} \label{feat::chug}
    Instead of a minute, you can drink a brew using 3 actions, gaining the normal benefits associated to it.
    \paragraph{Requirements} Skilled proficiency with brewer's supplies.
\subsubsection{The Enhancer} \label{feat::theenhancer}
    As part of a short rest, you can quickly brew a tankard worth of The Enhancer, a brew unique to master brewers.
    Upon drinking, the drinker gains the effect associated to the drink for an hour.

    The six variants of The Enhancer are listed.
    \paragraph{Bear's Endurance} The target has advantage on Constitution checks.
    It also gains 2d6 temporary hit points, which are lost when the effect ends.
    \paragraph{Bull's Strength} The target has advantage on Strength checks, and their carrying capacity doubles.
    \paragraph{Cat's Grace} The target has advantage on Dexterity checks.
    It also doesn't take damage from falling 4 meters or less if it isn't incapacitated.
    \paragraph{Eagle's Splendor} The target has advantage on Charisma checks.
    \paragraph{Fox's Cunning} The target has advantage on Intelligence checks.
    \paragraph{Owl's Wisdom} The target has advantage on Wisdom checks.

    You can take this feat two additional times.
    The second, you extend the duration of the effect to 8 hours.

    The third, you extend the duration of the effect to 24 hours, and can brew two tankards worth of The Enhancer per short rest.
    These tankards can be different variants of the brew.
    \paragraph{Requirements} Skilled proficiency with brewer's supplies.
\subsubsection{Mixed Drinks (2 FP)} \label{feat::mixeddrinks}
    As part of a short rest, you can mix two brews to combine their effects.
    To do this, you must succeed on a DC 15 check with your brewer's supplies.
    On a successful check you combine the brews' effects, and are left with an amount of the combined brew equal to half the sum of the two brews --- the remainder is lost in the mixing process.
    \paragraph{Requirements} Expert proficiency with brewer's supplies.

% CALLIGRAPHER'S SUPPLIES calligrapher bookminded encriptedwriting guentsuetattoos forger
\subsubsection{Calligrapher} \label{feat::calligrapher}
    Increase your level of proficiency with calligrapher's supplies.
    This feat can be re-taken until you are an expert with the toolset.

    You add your Intelligence modifier to checks with calligrapher's supplies.
\subsubsection{Book-Minded} \label{feat::bookminded}
    You have advantage on any check made to identify someone's handwriting or the source of a document, paper, or ink.
    \paragraph{Requirements} Competent proficiency with caligrapher's supplies.
\subsubsection{Encripted Writing} \label{feat::encriptedwriting}
    A master of codes, you are able to create written ciphers.
    Other can't decipher a code you create unless you teach them, they succeed on an Intelligence check (DC equal to 8 + your proficiency bonus with caligrapher's supplies), or they use magic to decipher it.

    If you already have the ability to create ciphers from another feat, add a +2 to the DC required to decipher your codes.
    \paragraph{Requirements} Skilled proficiency with caligrapher's supplies.
\subsubsection{Forger} \label{feat::forger}
    You gain proficiency with a Forgery kit, using your proficiency bonus with caligrapher's supplies for checks using it.
    You can take this feat two additional times, gaining different effects each time:
    \begin{itemize}
        \item The second, you learn to protect yourself with multiple layers of identities and misdirection.
        When you forge a document, you can leave behind false clues that suggest a different person was the forger.
        When a reader spots that the document is forged, it also discovers these clues.
        Identifying that these clues are false requires an additional Intelligence (Investigation) check against the same DC used to identify the original forgery.
        \item The third, you can permanently commit a person's handwriting to memory if you spend a long rest studying it.
        You can imitate this handwriting to perfection, and it is impossible to discern it from the original.
        You can only have one handwriting commited to memory in this way.
    \end{itemize}
    \paragraph{Requirements} Skilled proficiency with caligrapher's supplies.
\subsubsection{Guen Tsue Tattoos (2 FP)} \label{feat::guentsuetattoos}
    Using your calligrapher's supplies, you are able to inscribe tattoos into a creature as part of a long rest.
    The tattoos are based on runes from the Guen Tsue school of magic, and contain awesome effects for the tattooed creature to invoke.
    A Guen Tsue tattoo requires attunement from the tattooed creature.

    The effects available to you creature:
    \begin{itemize}
        \item \textbf{Absorbing Tattoo}.
        This colored tattoo grants you resistance to one damage type between acid, cold, fire, force, lightning, necrotic, poison, psychic, radiant, and thunder.
        The damage type is chosen by the tattoo artist while applying it.
        When you take damage of the chosen type, you can use your reaction to gain immunity against that instance of the damage, and you regain a number of hit points equal to half the damage you would have taken.
        Once this reaction is used, it can't be used again until the next dawn.
        \item \textbf{Barrier Tattoo}.
        While you aren't wearing armor, this tattoo depicting liquid metal grants you an Armor Class of 12 + your dexterity bonus.
        You can use a shield and still gain this benefit.
        \item \textbf{Blood Fury Tattoo}.
        This tattoo evokes fury in its shape and color.
        The tattoo has 5 charges, and it regains all expended charges daily at dawn.
        When you hit a creature with a weapon attack, you can expend a charge to deal an extra 4d6 necrotic damage to the target, and you regain a number of hit points equal to half the necrotic damage dealt.

        In addition, When a creature you can see damages you, you can expend a charge and use your reaction to make a melee attack against that creature, with advantage on your attack roll.
        \item \textbf{Coiling Grasp Tattoo}.
        While this intertwining tattoo is on your skin, you can, as an action, cause the tattoo to extrude into inky tendrils, which reach for a creature you can see within 3 meters of you.
        The creature must succeed on a DC 14 Strength saving throw or take 3d6 force damage and be grappled by you.
        As an action, the creature can escape the grapple by succeeding on a DC 14 Strength (Athletics) or Dexterity (Acrobatics) check.
        The grapple also ends if you halt it (no action required), if the creature is ever more than 3 meters away from you, or if you use this tattoo on a different creature.
        \item \textbf{Eldritch Claw Tattoo}.
        While this jagged tattoo is on your skin, you gain a +1 bonus to attack and damage rolls with unarmed strikes.
        In addition, you can use an action to empower the tattoo for 1 minute.
        For the duration, each of your melee attacks with a weapon or an unarmed strike can reach a target up to 3 meters away from you, as inky tendrils launch toward the target.
        In addition, your melee attacks deal an extra 1d6 force damage on a hit.
        Once used, this action can't be used again until the next dawn.
        \item \textbf{Lifewell Tattoo}.
        When you would be reduced to 0 hit points, you drop to 1 hit point instead.
        Once used, this property can't be used again until the next dawn.
        \item \textbf{Spellwrought Tattoo}.
        This coin-sized tattoo contains a single cantrip, chosen by the tattoo artist while applying it.
        You can cast this cantrip requiring no material components.
        The ability modifier for this spell is +3, the save DC is 13 and the attack bonus is +5.
    \end{itemize}

    A creature can remove a tattoo from its body as part of a long rest, ending its attunement to it.
    \paragraph{Requirements} Expert proficiency with caligrapher's supplies.

% CARPENTER'S TOOLS carpenter fortify instantfortress laminararmor
\subsubsection{Carpenter} \label{feat::carpenter}
    Increase your level of proficiency with carpenter's tools.
    This feat can be re-taken until you are an expert with the toolset.

    You add your Strength modifier to checks with carpenter's tools.

    From mastery with these tools, you have access to large items made of wood, like barrels, battering rams, boats, etc.
    The cost associated to each item only accounts for the required wood.
\subsubsection{Fortify} \label{feat::fortify}
    You can temporarily fortify a door, wall, or window using two actions.
    The DC needed to open or damage it in any way increases by 5.
    An object can be fortified in this way only once, and the fortifications fall apart 8 hours after being applied.
    \paragraph{Requirements} Competent proficiency with carpenter's tools.
\subsubsection{Instant Fortress} \label{feat::instantfortress}
    With one minute of work and raw materials, you can create a wooden barrier no more than 2 meters in any dimension.
    The barrier has an AC equal to your proficiency with carpenter's tools and an HP equal to three times that.
    It is immune to poison and psychic damage, and is vulnerable to fire damage.
    The barrier works as full cover against projectiles, and if it measures less than 1 by 1 meters, it can be used as a tower shield.

    The barrier collapses 8 hours after being assembled.
    \paragraph{Requirements} Skilled proficiency with carpenter's tools.
\subsubsection{Laminar Armor (2 FP)} \label{feat::laminararmor}
    You gain access to craft all heavy armor and shields by yourself --- even if they would normally require other toolsets.
    All items you craft in this way are made of wood, and weigh twice their normal weight.

    In addition, all armor crafted in this way has the Composite and Noisy properties, overriding the properties they would normally have.
    \paragraph{Requirements} Expert proficiency with carpenter's tools.

% CARTOGRAPHER'S SUPPLIES cartographer copyist everpresent proliferator
\subsubsection{Cartographer} \label{feat::cartographer}
    Increase your level of proficiency with cartographer's supplies.
    This feat can be re-taken until you are an expert with the toolset.

    You add your Intelligence modifier to checks with cartographer's supplies.

    While you travel, you can draw a map as you go in addition engaging in other activity.
\subsubsection{Copyist} \label{feat::copyist}
    You gain the ability to copy another map as part of a long rest.
    To do this, you consume materials that cost half the cost of the map you're copying, and must succeed on an ability check using the Cartographer's Kit with a DC proportional to the scale and the level of detail of the map, at the DM's discretion.
    If you have spent a month of more exploring the region in the map, you make this ability check with advantage.
    \paragraph{Requirements} Competent proficiency with cartographer's supplies.
\subsubsection{Proliferator} \label{feat::proliferator}
    While drawing a map you can simultaneously draw a second copy of it without spending any additional time.
    \paragraph{Requirements} Skilled proficiency with cartographer's supplies.
\subsubsection{Ever-Present (2 FP)} \label{feat::everpresent}
    Your time studying maps has given you the ability to understand vantage points and points of interest even without having seen them before.
    You cannot become lost by any means, and you can always accurately tell where you are in the world as part of a short rest.
    \paragraph{Requirements} Expert proficiency with cartographer's supplies.

% COBBLER'S TOOLS cobbler bootenhancements reinforcedsole travellingshoes
\subsubsection{Cobbler} \label{feat::cobbler}
    Increase your level of proficiency with cobbler's tools.
    This feat can be re-taken until you are an expert with the toolset.

    You add your Dexterity modifier to checks with cobbler's tools.
\subsubsection{Travelling Shoes} \label{feat::travellingshoes}
    You can repair and prepare the party's footwear as part of a short rest.
    When you do so, the party is granted a bonus to their travel pace.
    They are able to use stealth while travelling at normal pace, and don't take a -5 penalty to passive perception when travelling at a fast pace.

    This bonus lasts for one week of travel.
    \paragraph{Requirements} Competent proficiency with cobbler's tools.
\subsubsection{Reinforced Sole} \label{feat::reinforcedsole}
    Any creature that wears shoes you have worked on take half damage (rounded down) from items and effect that specifically target their feet, such as the grease spell or a spread bag of caltrops.
    \paragraph{Requirements} Skilled proficiency with cobbler's tools.
\subsubsection{Boot Enhancements (2 FP)} \label{feat::bootenhancements}
    As part of a long rest, you can work on a pair of boots or shoes to add the following effects to them:
    \subparagraph{Comfortable Soles} If the creature travels for more than 8 hours in one day, they roll the Constitution saving throw against exhaustion with advantage.
    \subparagraph{Padded Soles} The creature's steps don't produce any sound, and they cannot be found from the sounds of their walking alone.
    In addition, rolls to identify tracks left by these shoes are made with disadvantage.
    \subparagraph{Spiked Soles} The creature has advantage on any saving throw made to avoid being pushed or being knocked prone.

    A weapon can only have one of these effects applied at the same time.
    \paragraph{Requirements} Expert proficiency with cobbler's tools.

% COOKING UTENSILS gourmand chef sweettreat warmmeal
\subsubsection{Gourmand} \label{feat::gourmand}
    Increase your level of proficiency with cooking utensils.
    This feat can be re-taken until you are an expert with the toolset.

    You add your Constitution modifier to checks with cooking utensils.

    The cost associated to each item only accounts for the required reagents.
    Unless specified otherwise in the item's name, the DC of food relates to 8 rations worth of the stuff.
\subsubsection{Warm Meal} \label{feat::warmmeal}
    By cooking them, you can make rations more effective.
    As part of a short rest, you can cook a number of rations equal to 4 + your proficiency bonus with cooking utensils.
    The number of cooked rations produced is equal to double the original number of rations, but spoil if left untouched for more than 24 hours.
    \paragraph{Requirements} Competent proficiency with cooking utensils.
\subsubsection{Sweet Treat} \label{feat::sweettreat}
    With one hour of work or as part of a short rest, you can cook a number of treats equal to your proficiency bonus with cooking utensils.
    These special treats last 8 hours after being made.
    A creature can use a bonus action to eat one of those treats to gain temporary hit points equal to your proficiency bonus with cooking utensils.
    \paragraph{Requirements} Skilled proficiency with cooking utensils.
\subsubsection{Chef (2 FP)} \label{feat::chef}
    As part of a short rest, you can cook special food, provided you have ingredients and cooking utensils on hand.
    You can prepare enough of this food for a number of creatures equal to 4 + your proficiency bonus with cooking utensils.
    At the end of the short rest, any creature who eats the food and spends one or more Hit Dice to regain hit points regains extra hit points equal to 2d8 + their Constitution modifier.
    \paragraph{Requirements} Expert proficiency with cooking utensils.

% GAMING KIT player
\subsubsection{Player} \label{feat::player}
    You become proficient with a gaming kit of your choice.
    You add a +4 bonus to any check related to the gaming kit.

    This feat can be re-taken any time, choosing a gaming kit each time.

% GLASSBLOWER'S TOOLS glassblower fineflask glassenhancement glassgrenade
\subsubsection{Glassblower} \label{feat::glassblower}
    Increase your level of proficiency with Glassblower's Tools.
    This feat can be re-taken until you are an expert with the toolset.

    You add your Dexterity modifier to checks with Glassblower's Tools.

    From mastery with these tools, you have access to any item made with glass, like vials, cups, and flasks.
    The cost associated to each item only accounts for the required glass.

    % To work with your glassblowers tools you need access to a furnace capable of reaching temperatures of more than 1,500$\degree$ Celcius.
\subsubsection{Glass Grenade} \label{feat::glassgrenade}
    As part of a short rest, you can bundle up discarded glass dust in a bag to craft a makeshift ``glass grenade''.
    You can throw this glass grenade at a range of 4/12 mt., and it violently explodes out in a 1 mt. radius upon hitting the floor.
    All creatures in its radius must succeed on a DC 15 Dexterity saving throw or take 1d4 piercing damage.
    Furthermore, the target loses 1d4 hit points at the start of each of its turns due to the shattered glass in the wound.
    Any creature can take two actions to staunch the wound with a successful DC 12 Wisdom (Medicine) check.

    This effect is not cummulative.
    \paragraph{Requirements} Competent proficiency with Glassblower's Tools.
\subsubsection{Fine Flask} \label{feat::fineflask}
    Flasks and vials you make are especially effective, making a great addition to an already fine liquid.
    Any potion drank from one flask or vial you make has its duration doubled.
    Bloodwell vials made in this fashion have their number of uses doubled.

    In addition, you learn how to make glass lining inside tankards, which adds this same effects to brews drank from them.
    \paragraph{Requirements} Skilled proficiency with Glassblower's Tools.
\subsubsection{Glass Enhancements (2 FP)} \label{feat::glassenhancement}
    You can enhance a weapon's effect by adding glass into the mix.
    As part of a long rest, you can work with a weaponsmith to add one of the following effects to a slashing or piercing weapon:
    \subparagraph{Hollowed Weapon} When a poison is applied to the weapon, the time it remains applied is doubled, and it can apply one additional dose of the poison before being drained.
    \subparagraph{Grainy Edge} By adding glass grain to a weapons edge it becomes more effective.
    The weapon gains a +1 bonus to attack and damage rolls if it didn't already have a bonus.
    \subparagraph{Glass Shards} This weapon contains loosely glued shards of glass on its faces.
    On a successful hit, the target takes 1d4 damage on the beginning of each of its turns.
    Any creature can take two actions to staunch the wound with a successful DC 12 Wisdom (Medicine) check.
    This effect is not cummulative.

    A weapon can only have one of these effects applied at the same time.
    \paragraph{Requirements} Expert proficiency with Glassblower's Tools.

% HEALER'S KIT healer combatmedic physician quickmender travellingdoctor
\subsubsection{Healer} \label{feat::healer}
    Increase your level of proficiency with the healer's kit.
    This feat can be re-taken until you are an expert with the kit.

    You add your Wisdom modifier to checks with the healer's kit.
\subsubsection{Quick Mender} \label{feat::quickmender}
    Used to working in the field, you can heal minor injuries as part of a short rest expending one use of your healer's kit.

    A creature healed with this feat also gains a number of temporary hit points equal to its Constitution modifier (minimum of one).
    \paragraph{Requirements} Competent proficiency with healer's kit.
\subsubsection{Physician} \label{feat::physician}
    As an avid physician, you can expend 3 uses of a healer's kit to heal a major injury as part of a long rest.
    The creature must succeed on a DC 12 Constitution saving throw for the treatment to work.

    A creature healed with this feat also gains a number of temporary hit points equal to double its Constitution modifier (minimum of two).
    \paragraph{Requirements} Skilled proficiency with healer's kit.
\subsubsection{Combat Medic} \label{feat::combatmedic}
    You are able to mend wounds quickly and get your allies back in the fight.
    Using two actions, you can spend one use of a healer's kit to tend to a creature and restore 1d6 + your proficiency modifier with the healer's kit hit points to it, plus additional hit points equal to the creature's maximum number of Hit Dice.
    The creature can't regain hit points from this feat again until it finishes a short rest.

    You can take this feat two additional times, increasing the number of dice rolled to 2d6 the second time and 3d6 the third.
    \paragraph{Requirements} Skilled proficiency with Healer's kit.
\subsubsection{Travelling Doctor (2 FP)} \label{feat::travellingdoctor}
    You keep your allies in top shape during your travels.
    The minimum number of hit points you and your allies regain from a hit die roll is equal to their Constitution modifier (minimum of 2).
    \paragraph{Requirements} Expert proficiency with Healer's kit.

% HERBALISM KIT herbalist adeptpoisoner healingoinments naturalremedies
\subsubsection{Herbalist} \label{feat::herbalist}
    Increase your level of proficiency with herbalism kit.
    This feat can be re-taken until you are an expert with the toolset.

    You add your Wisdom modifier to checks with a the herbalism kit.

    You can use your proficiency with this kit to gather reagents from nature as part of a long rest.
    Make a list with the rarity of reagents you want to find during the short rest, then roll a skill check using the herbalism kit.
    The DC of one material is the same as the one for crafting an object of the same rarity.

    Reagents you gather can be used to craft brews, food, poisons, and potions.
\subsubsection{Adept Poisoner} \label{feat::adeptpoisoner}
    You can make basic poison (see page \pageref{item::basicpoison} using your herbalism kit as part of a long rest.
    If you gather the reagents yourself, you only need half the normal amount of materials to make the poison, and can make it as part of the long rest you used to gather them.
    \paragraph{Requirements} Competent proficiency with herbalism kit.
\subsubsection{Healing Ointments} \label{feat::healingoinments}
    As part of a short rest, you can use one material worth of gathered reagents to cure one or more creatures' wounds.
    The amount healed depends on the rarity of the reagents, and is specified in the Healing Ointment table below.
    You add your Wisdom modifier to the healing applied.

    \begin{DndTable}[width=\linewidth, header=Healing Oinment]{lX}
        \textbf{Rarity} & \textbf{Healing} \\
        Mundane         & 1d4              \\
        Plain           & 1d6              \\
        Common          & 1d8              \\
        Uncommon        & 2d8              \\
        Rare            & 3d8              \\
        Very Rare       & 4d8              \\
        Legendary       & 5d8
    \end{DndTable}
    \paragraph{Requirements} Skilled proficiency with Herbalism Kit.
\subsubsection{Natural Remedies (2 FP)} \label{feat::naturalremedies}
    You gain advantage on saving throws against poisons and diseases.

    In addition, you can use your herbalism kit to administer a herbal remedy to help a creature recover from their ailments as part of a short rest.
    The creature gains advantage on saving throws against poison and disease for the next day.
    \paragraph{Requirements} Expert proficiency with Herbalism Kit.

% JEWELER'S TOOLS jeweler adornedgarments fakejewels shatterpoint
\subsubsection{Jeweler} \label{feat::jeweler}
    Increase your level of proficiency with jeweler's tools.
    This feat can be re-taken until you are an expert with the toolset.

    You add your Dexterity modifier to checks with a jeweler's tools.
\subsubsection{Adorned Garments} \label{feat::adornedgarments}
    As part of a long rest, you can use one material worth of jewels to adorn an item of your choice.
    The jewels' rarity must be of at least the same rarity as the item.
    The value of the item is increased by the double of the cost of the added jewels.
    \paragraph{Requirements} Competent proficiency with jeweler's tools.
\subsubsection{Shatter Point} \label{feat::shatterpoint}
    Your knowledge of gems and gem-based objects allows you to quickly identify their shatter points.
    You deal double damage to such objects and gem-based creatures with weapon attacks.
    \paragraph{Requirements} Skilled proficiency with jeweler's tools.
\subsubsection{Fake Jewels (2 FP)} \label{feat::fakejewels}
    As part of a long rest, you can create one or more fake gemstones.
    When you create them, you can set a monetary value for them to appear to have; this value can't be more than your proficiency bonus with jeweler's tools times 100 (minimum of 300 agnomas).
    In addition, make an ability check using your jeweler's tools; the total of your check becomes the DC for someone else's attempt to identify the fake gems for what they are.
    \paragraph{Requirements} Expert proficiency with jeweler's tools.

% LEATHERWORKER'S TOOLS leatherworker hardpatches silencedjoints slayandskin
\subsubsection{Leatherworker} \label{feat::leatherworker}
    Increase your level of proficiency with leatherworker's tools.
    This feat can be re-taken until you are an expert with the toolset.

    You add your Dexterity modifier to checks with a leatherworker's tools.
\subsubsection{Slay \& Skin} \label{feat::slayandskin}
    You are particularly proficient at scavenging leather off a dead creature.
    No more than one minute after slaying a creature with the beast type, you can roll with your leatherworker's tools to skin it.
    With a minute of work, you produce a number of leatherworking materials equal to the roll divided by 10 (rounded down).
    You may obtain less or more based on the creature's size at the DMs discretion.

    The rarity of the materials gathered depend on the CR of the creature, as is given in the Leather Rarity table below.

    \begin{DndTable}[width=\linewidth, header=Leather Rarity]{lX}
        \textbf{CR} & \textbf{Rarity} \\
        < 1         & Plain           \\
         1          & Common          \\
         2 -  7     & Uncommon        \\
         8 - 13     & Rare            \\
        14 - 20     & Very Rare       \\
        21+         & Legendary
    \end{DndTable}
    \paragraph{Requirements} Competent proficiency with leatherworker's tools.
\subsubsection{Silenced Joints} \label{feat::silencedjoints}
    As part of a short rest, you can add small leather pieces to the joints of medium or heavy armor.
    This armor loses the Noisy property until the end of the following day.
    \paragraph{Requirements} Skilled proficiency with leatherworker's tools.
\subsubsection{Hard Patches (2 FP)} \label{feat::hardpatches}
    As part of a short rest, you can craft three patches and apply them to up to three armors with the Leather property.
    Patches retain potency during the day after they are applied or until their number of uses is exhausted.
    You can choose to apply multiple patches to one armor.
    \subparagraph{Padded Patch} This patch has three uses.
    When the wearer of this armor takes fall damage, that damage is calculated as if they fell 4 less meters (2 less dice) and it loses a use.
    \subparagraph{Slippery Patch} This patch has one use.
    When the wearer of this armor becomes grappled or restrained, the effect immediately ends.
    \subparagraph{Impact Patch} This patch has two uses.
    When the wearer takes bludgeoning damage from a weapon attack, the damage is reduced to 0.
    \paragraph{Requirements} Expert proficiency with leatherworker's tools.

% LANGUAGES linguist
\subsubsection{Linguist} \label{feat::linguist}
    You learn a language of your choice, with the exception of Jantherlin.

    This feat can be re-taken any time, choosing a different language each time.

% MASON'S TOOLS mason breakingpoint miner reinforcedequipment
\subsubsection{Mason} \label{feat::mason}
    Increase your level of proficiency with mason's tools.
    This feat can be re-taken until you are an expert with the toolset.

    You add your Strength modifier to checks with mason's tools.
\subsubsection{Miner} \label{feat::miner}
    You can use your proficiency with this kit to mine from a cliff face, mountain, or cavern as part of a long rest.
    Make a list with the rarity of reagents you want to find during the short rest, then roll a skill check using the mason's tools.
    The DC of one material is the same as the one for crafting an object of the same rarity.

    Materials gathered in this way can be used by jewelers, masons, and smiths.
    \paragraph{Requirements} Competent proficiency with mason's tools.
\subsubsection{Breaking Point} \label{feat::breakingpoint}
    Your knowledge of masonry allows you to spot weak points in any stone construction, rather than just brick walls.
    You deal double damage to objects and structures made from brick or stone and creatures made from these materials.
    \paragraph{Requirements} Skilled proficiency with mason's tools.
\subsubsection{Reinforced Equipment (2 FP)} \label{feat::reinforcedequipment}
    You can reinforce a heavy armor or bludgeoning weapon as part of a long rest.
    When you do so, you increase the AC of the piece of armor or the damage of the bludgeoning weapon by +1.
    A piece of equipment can be reinforced only once with this effect, and its weight is doubled.
    \paragraph{Requirements} Expert proficiency with mason's tools.

% MUSICAL INSTRUMENTS musician
\subsubsection{Musician} \label{feat::musician}
    You become proficient with a musical instrument of your choice.
    You add a +4 bonus to any check related to the musical instrument.

    This feat can be re-taken any time, choosing a different instrument each time.

% NAVIGATOR'S TOOLS navigator allterrain improvisedcharter perfectguide
\subsubsection{Navigator} \label{feat::navigator}
    Increase your level of proficiency with navigator's tools.
    This feat can be re-taken until you are an expert with the toolset.

    You add your Wisdom modifier to checks with navigator's tools.

    You need the skies to be clear to use your set of navigator's tools.
\subsubsection{All-Terrain} \label{feat::allterrain}
    You can use navigator's tools to plot a course or to determine your location on land just as well as you can at sea.
    \paragraph{Requirements} Competent proficiency with navigator's tools.
\subsubsection{Improvised Charter} \label{feat::improvisedcharter}
    As part of a short rest, you can use scavenged or acquired materials to build a temporary set of navigator's tools for when you lack access to a permanent one. This temporary set ceases to function after 1 hour.
    \paragraph{Requirements} Skilled proficiency with navigator's tools.
\subsubsection{Perfect Guide (2 FP)} \label{feat::perfectguide}
    You don't need a clear sky to use your set of navigator's tools --- you can use them in any situation.
    \paragraph{Requirements} Expert proficiency with navigator's tools.

% PAINTER'S SUPPLIES painter masterartist paintedscrolls precisehands
\subsubsection{Painter} \label{feat::painter}
    Increase your level of proficiency with painter's supplies.
    This feat can be re-taken until you are an expert with the toolset.

    You add your Dexterity modifier to checks with painter's supplies.
\subsubsection{Precise Hands} \label{feat::precisehands}
    You have advantage on Dexterity (Sleight of Hand) checks that require the use of precise or gentle hand movements.
    \paragraph{Requirements} Competent proficiency with painter's supplies.
\subsubsection{Painted Scrolls} \label{feat::paintedscrolls}
    You can produce spell scrolls (see page \pageref{item::spellscroll}) as if you were proficient with calligrapher's supplies.
    Scrolls produced by you are picture-based, and thus do not need the reader to share a language with you.
    \paragraph{Requirements} Skilled proficiency with painter's supplies.
\subsubsection{Master Artist (2 FP)} \label{feat::masterartist}
    As part of a long rest, you can paint a masterwork painting.
    When you do this, you can set the monetary value for it to have.
    This value can't be more than your proficiency bonus with painter's supplies times 200 (minimum of 600 agnomas).
    \paragraph{Requirements} Expert proficiency with painter's supplies.

% POISONER'S KIT poisoner mixedpoisons perfectedpoison quickapplication
\subsubsection{Poisoner} \label{feat::poisoner}
    Increase your level of proficiency with the poisoner's kit.
    This feat can be re-taken until you are an expert with the toolset.

    You add your Intelligence modifier to checks with the poisoner's kit.
\subsubsection{Quick Application} \label{feat::quickapplication}
    You can coat a weapon or three pieces of ammunition in poison using one action instead of two.
    \paragraph{Requirements} Competent proficiency with the poisoner's kit.
\subsubsection{Mixed Poisons} \label{feat::mixedpoisons}
    You can combine two doses of poison of different varieties into one.
    You produce two doses of a new poison which combines the qualities of the two ingredientes.
    The damage done by the poison is equal to half the sum of the damage of the two poisons, and the duration of the effects inherited by the poison is halved.
    \paragraph{Requirements} Skilled proficiency with the poisoner's kit.
\subsubsection{Perfected Poison (2 FP)} \label{feat::perfectedpoison}
    The saving throws of the poisons you make are increased by half proficiency bonus with the poisoner's kit (rounded down).
    In addition, your poisons ignore resistance to poison damage.
    \paragraph{Requirements} Expert proficiency with the poisoner's kit.

% POTTER'S TOOLS potter augmentingpottery ceramicweapons quicksharpening
\subsubsection{Potter} \label{feat::potter}
    Increase your level of proficiency with potter's tools.
    This feat can be re-taken until you are an expert with the toolset.

    You add your Dexterity modifier to checks with potter's tools.
\subsubsection{Quick Sharpening} \label{feat::quicksharpening}
    Taking one minute, you can sharpen a piece of ceramic using a set of potter's tools.
    If used as a weapon, improvised or otherwise, it deals an additional 1d4 damage on hit.
    The object becomes dull after an hour, losing this benefit.
    \paragraph{Requirements} Competent proficiency with potter's tools.
\subsubsection{Ceramic Weapons} \label{feat::ceramicweapons}
    You can craft any slashing weapon using your set of potter's tools and enough materials related to your profession.
    Ceramic weapons made in this way deal an extra 1d4 slashing damage on each attack, but quickly lose sharpness.
    They have a total of 8d10 charges, and lose one charge for each successful hit made with them.
    Upon reaching zero charges they become too dull to be used and cannot be resharpened.
    \paragraph{Requirements} Skilled proficiency with potter's tools.
\subsubsection{Augmenting Pottery (2 FP)} \label{feat::augmentingpottery}
    As part of a long rest, you can make a ceramic vial, flask, or tankard using one piece of material of at least uncommon rarity.
    Any dice roll related to the brew, poison, or potion drank from this container gains a bonus equal to half your proficiency bonus with potter's tools (rounded down).
    \paragraph{Requirements} Expert proficiency with potter's tools.

% SMITH'S TOOLS blacksmith hardenedmetal patcher weakpoint
\subsubsection{Blacksmith} \label{feat::blacksmith}
    Increase your level of proficiency with smith's tools.
    This feat can be re-taken until you are an expert with the toolset.

    You add your Strength modifier to checks with smith's tools.
\subsubsection{Patcher} \label{feat::patcher}
    You can repair a number of weapons, shields, and armor equal to your proficiency bonus with smith's tools as part of a long rest.
    Any penalty the items suffer from damage or an object-targetting attack end at the end of the long rest.
    You don't need any materials to do this, just your set of smith's tools.
    \paragraph{Requirements} Competent proficiency with smith's tools.
\subsubsection{Weak Point} \label{feat::weakpoint}
    When you roll a 20 on a wepaon attack against a target that is wearing metal armor, the armor suffers a -1 penalty to AC.
    The lost AC can only be recovered under a smith's hand, process that costs a quarter of the armor's total cost.
    If an armor loses 4 AC in this way, it is broken beyond repair.
    \paragraph{Requirements} Skilled proficiency with smith's tools.
\subsubsection{Hardened Metal (2 FP)} \label{feat::hardenedmetal}
    As part of a long rest, you can harden a number of metal items equal to half your proficiency bonus with smith's tools.
    All weapons and armor hardened in this way gain a bonus to their damage or AC equal to 1 until the end of the day.
    \paragraph{Requirements} Expert proficiency with smith's tools.

% THIEVES' TOOLS thief improvisedtools thievescant
\subsubsection{Thief} \label{feat::thief}
    Increase your level of proficiency with thieves' tools.
    This feat can be re-taken until you are an expert with the toolset.

    You add your Strength modifier to checks with smith's tools.

    You can use your thieves' tools to make checks to open locks or disarm traps using two actions.
\subsubsection{Thieves' Cant} \label{feat::thievescant}
    You learn thieves' cant, a secret mix of dialect, jargon, and code that allows you to hide messages in seemingly normal conversation.
    Only another creature that knows thieves' cant understands such messages.
    It takes four times longer to convey such a message than it does to speak the same idea plainly.

    In addition, you understand a set of secret signs and symbols used to convey short, simple messages, such as whether an area is dangerous or the territory of a thieves' guild, whether loot is nearby, or whether the people in an area are easy marks or will provide a safe house for thieves on the run.
    \paragraph{Requirements} Competent proficiency with thieves' tools.
\subsubsection{Improvised Tools} \label{feat::improvisedtools}
    As part of a short rest, you can use scavenged or acquired materials to build a temporary set of thieves' tools for when you lack access to a permanent one.
    This temporary set can be used 5 times before it breaks down into useless components.
    \paragraph{Requirements} Skilled proficiency with thieves' tools.
\subsubsection{Quick Hands (2 FP)} \label{feat::quickhands}
    The action cost of the Use an Object and the Search actions is reduced to one for you.
    \paragraph{Requirements} Expert proficiency with thieves' tools.

% TINKER'S TOOLS tinkerer artificersinfusion fixer therighttoolforthejob unconventionalcrafter
\subsubsection{Tinkerer} \label{feat::tinkerer}
    Increase your level of proficiency with tinker's tools.
    This feat can be re-taken until you are an expert with the toolset.

    You add your Intelligence modifier to checks with tinker's tools.
\subsubsection{Fixer} \label{feat::fixer}
    Repairing an item takes half as much time and only requires half as many raw materials.
    \paragraph{Requirements} Competent proficiency with tinker's tools.
\subsubsection{The Right Tool for the Job} \label{feat::therighttoolforthejob}
    With artisan's tools in hand, you can create one set of artisan's tools.
    This creation requires 1 hour of uninterrupted work, which can coincide with a short or long rest.
    The tools break down when you use this feature again.
    \paragraph{Requirements} Skilled proficiency with tinker's tools.
\subsubsection{Unconventional Crafter (2 FP)} \label{feat::unconventionalcrafter}
    Using materials that you forage, store among your other possessions, or harvest from disassembled items, you can create any item of Common or less rarity.
    As part of a long rest, you can use these materials and your tinker's tools to create one or more items.
    The combined value of these items must be worth 100 agnomas or less.
    Items created using this feature cease to function and return to their base components in 3 days after the long rest.
    \paragraph{Requirements} Expert proficiency with tinker's tools.

% LAND VEHICLES rider exoticrider notmysteed shieldedmount
\subsubsection{Rider} \label{feat::rider}
    Increase your level of proficiency with land vehicles.
    This feat can be re-taken until you are an expert with the proficiency.

    You add your Wisdom modifier to checks with land vehicles.

    Mounts act on the same initiative as you, but they can only take the Disengage, Dodge, and Move actions.
    Mounting or dismounting a creature requires two actions.
\subsubsection{Not my Steed!} \label{feat::notmysteed}
    If a creature targets your mount with a melee attack, you can use your reaction to make an opportunity attack against the creature.
    This feat works even if you are able to redirect the attack to yourself via another feat.
    \paragraph{Requirements} Competent proficiency with land vehicles.
\subsubsection{Shielded Mount} \label{feat::shieldedmount}
    If you are wielding a shield, your mount gains a bonus to its AC equal to the shield's AC bonus.
    \paragraph{Requirements} Skilled proficiency with land vehicles.
\subsubsection{Exotic Rider (2 FP)} \label{feat::exoticrider}
    You gain proficiency with air vehicles, using your proficiency bonus with land vehicles for checks using them.
    \paragraph{Requirements} Expert proficiency with land vehicles.

% WATER VEHICLES captain caredforship loudvoice speedysailor
\subsubsection{Captain} \label{feat::captain}
    Increase your level of proficiency with water vehicles.
    This feat can be re-taken until you are an expert with the proficiency.

    You add your Wisdom modifier to checks with water vehicles.
\subsubsection{Loud Voice} \label{feat::loudvoice}
    Any Charisma (Intimidation) and Charisma (Persuasion) check you perform while riding a ship is made with advantage.
    \paragraph{Requirements} Competent proficiency with water vehicles.
\subsubsection{Cared-for Ship} \label{feat::caredforship}
    A master captain, any maintenance cost for a ship you own or otherwise command is reduced to half its normal value.
    \paragraph{Requirements} Skilled proficiency with water vehicles.
\subsubsection{Speedy Sailor (2 FP)} \label{feat::speedysailor}
    Using modern sailing techniques, the speed of any vessel you command is increased by half its normal value.
    \paragraph{Requirements} Expert proficiency with water vehicles.

% WEAVER'S TOOLS weaver salvagedclothing softpatches steadyhands
\subsubsection{Weaver} \label{feat::weaver}
    Increase your level of proficiency with weaver's tools.
    This feat can be re-taken until you are an expert with the toolset.

    You add your Dexterity modifier to checks with weaver's tools.
\subsubsection{Steady Hands} \label{feat::steadyhands}
    You have advantage on Dexterity (Sleight of Hand) checks that require the use of a steady or completely stationary hand.
    \paragraph{Requirements} Competent proficiency with weaver's tools.
\subsubsection{Salvaged Clothing} \label{feat::salvagedclothing}
    You can repair any piece of clothing --- no matter the damage --- as part of a short rest, and using only residuals you keep available with your set of weaver's tools.
    \paragraph{Requirements} Skilled proficiency with weaver's tools.
\subsubsection{Disguise Maker} \label{feat::disguisemaker}
    You gain proficiency with disguise kits, using your proficiency bonus with weaver's tools for checks using them.
    You can take this feat two additional times, gaining different effects each time:
    \begin{itemize}
        \item The second, you learn to observe the most distinguishing attributes of a person.
        If you spend one hour observing or interacting with a creature, you can then craft a disguise that can be used to mimic that creature during a long rest.
        \item The third, you learn how to procure a convincing disguise with stuff you find around you.
        Taking an hour, you can make a disguise that makes sense with the materials you have around you (at the DM's discretion), which is as effective as one made with a disguise kit with similar materials.
    \end{itemize}
    \paragraph{Requirements} Skilled proficiency with weaver's tools.
\subsubsection{Soft Patches (2 FP)} \label{feat::softpatches}
    As part of a short rest, you can craft three patches and apply them to up to three pieces of clothing or armors with the Cloth property.
    Patches retain potency during the day after they are applied or until their number of uses is exhausted.
    You can choose to apply multiple patches to one armor.
    \subparagraph{Padded Patch} This patch has three uses.
    When the wearer of this armor takes fall damage, that damage is calculated as if they fell 4 less meters (2 less dice) and it loses a use.
    \subparagraph{Slippery Patch} This patch has one use.
    When the wearer of this armor becomes grappled or restrained, the effect immediately ends.
    \subparagraph{Impact Patch} This patch has two uses.
    When the wearer takes bludgeoning damage from a weapon attack, the damage is reduced to 0.
    \paragraph{Requirements} Expert proficiency with weaver's tools.

% WOODCARVER woodcarver adhocweaponry bonecarver chiseledhandle
\subsubsection{Woodcarver} \label{feat::woodcarver}
    Increase your level of proficiency with woodcarver's tools.
    This feat can be re-taken until you are an expert with the toolset.

    You add your Dexterity modifier to checks with woodcarver's tools.

    From mastery with these tools, you have access to small items made with wood and to apply engravings on wooden items and surfaces.
    These items include arrows, bolts, rods, etc.
\subsubsection{Ad-hoc Weaponry} \label{feat::adhocweaponry}
    Taking an hour of work, you can create a number of improvised weapons equal to your proficiency bonus with woodcarver's tools.
    The weapons available to you are clubs, daggers, greatclubs, javelins, light hammers, maces, quarterstaves, and spears.
    The weapons quickly lose their effectivity, and can only be used for one combat encounter.

    To use this feat, you must have your woodcarver tools at hand and enough wood to make all the weapons.
    \paragraph{Requirements} Competent proficiency with Woodcarver's Tools.
\subsubsection{Chiseled Handle} \label{feat::chiseledhandle}
    Using your woodcarver's tools, you can chisel the wooden handle of a weapon or shield as part of a short rest.
    Equipment worked in this way fits nicely into their user's hand, and any check that attempts to disarm them is made with disadvantage.
    \paragraph{Requirements} Skilled proficiency with Woodcarver's Tools.
\subsubsection{Bonecarver (2 FP)} \label{feat::bonecarver}
    You can use your woodcarving tools to work with bone.
    Provided you can get the raw materials, you can craft bone charms and qualars as you would craft wooden charms.
    \paragraph{Requirements} Expert proficiency with Woodcarver's Tools.

% NOTE. PHARIKA SHIT. All weapons have some poison in them. Critical hits gain a damage bonus equal to double your Wisdom modifier.

% !TEX root = ../main.tex
\addcontentsline{toc}{section}{Proficiency Feats}
\subsection*{Proficiency Feats}
% NOTE: DIVIDED INTO THRRE, TOOL PROFICIENCIES, LAND VEHICLES, AND KITS.

\subsection*{Tool Proficiencies}
    You can use a set of Artisan's Tools to craft items as part of a long rest.
    Make an ordered list of the items you want to make.
    The items you have access to depend on your proficiency level with the tools, and are detailed in each tools' associated proficiency feat.
    Each item in the list has an associated DC and cost, which is equal to its DC and cost plus these values for all the previous items on the list.

    The DC of an item depends on its rarity and its associated proficiency.
    They are listed in the tools' proficiency feat.
    Common items have a DC of 6, Uncommon items have a DC of 9, Rare items a DC of 12, Very Rare a DC of 15, and legendary a DC of 20.

    After you finish the list, make an ability check with the relevant toolset.
    The number you roll determines how far down the list you are able to craft, and the cost you need to pay for components.

% ALCHEMIST'S SUPPLIES alchemist experimentalelixir fromgreattobest potionperuser restorativereagents
\subsubsection{Alchemist} \label{feat::alchemist}
    Increase your level of proficiency with Alchemist's Supplies.
    This feat can be re-taken until you are an expert with the toolset.

    You add your Intelligence modifier to checks with Alchemists' Supplies.

    The cost and DC of potions relates to one vial or flask of the substance, as is detailed in the corresponding section.
\subsubsection{Potion Peruser} \label{feat::potionperuser}
    Using two actions, you can identify one potion within 1 meters of you, as if you had tasted it.
    You must see the liquid for this benefit to work.
    \paragraph{Requirements} Competent proficiency with Alchemist's Supplies.
\subsubsection{Restorative Reagents} \label{feat::restorativereagents}
    Whenever a creature drinks a potion you created, the creature gains temporary hit points equal to 2d6 + your proficiency bonus with Alchemist's Supplies.
    \paragraph{Requirements} Skilled proficiency with Alchemist's Supplies.
\subsubsection{From Great to Best} \label{feat::fromgreattobest}
    Over the course of a short rest, you can temporarily improve the potency of one potion of any rarity.
    % To use this benefit, you must have alchemist's supplies with you, and the potion must be within reach.
    If the potion is drunk during the day after the short rest ends, ignore the potion's die roll (if it has any), automatically rolling the maximum possible number.

    You can take this feat three times, increasing the number of potions you can improve as part of the short rest by one each time.
    \paragraph{Requirements} Skilled proficiency with Alchemist's Supplies.
\subsubsection{Experimental Elixir (2 FP)} \label{feat::experimentalelixir}
    Whenever you finish a short rest, you can produce two experimental elixirs in an empty flask you touch.
    Roll on the Experimental Elixir table for the elixir's effect, which is triggered when someone drinks the elixir.
    Using two actions, a creature can drink the elixir or administer it to an incapacitated creature.

    Creating an experimental elixir requires you to have alchemist's supplies on your person, and any elixir you create with this feature lasts until it is drunk or until the end of your next short rest.

    If you gain Legendary proficiency with Alchemist's Supplies, can make one more elixir with this ability.

    \begin{DndTable}[width=\linewidth, header=Experimental Elixir]{lX}
        \textbf{d6} & \textbf{Effect} \\
        1 & \textbf{Healing.}
        The drinker regains a number of hit points equal to 2d4 + half your proficiency bonus with Alchemist's Supplies. \\
        2 & \textbf{Swiftness.}
        The drinker's walking speed increases by 2 meters for 1 hour. \\
        3 & \textbf{Resilience.}
        The drinker gains a +1 bonus to AC for 10 minutes. \\
        4 & \textbf{Boldness.}
        The drinker can roll a d4 and add the number rolled to every attack roll and saving throw they make for the next minute. \\
        5 & \textbf{Flight.}
        The drinker gains a flying speed of 2 meters for 10 minutes. \\
        6 & \textbf{Transformation.}
        The drinker's body is transformed as if by the alter self spell (see page \pageref{spell::alterself}).
        The drinker determines the transformation caused by the spell, the effects of which last for 10 minutes.
    \end{DndTable}
    \paragraph{Requirements} Expert proficiency with Alchemist's Supplies.

% BREWER'S SUPPLIES brewer chug mixeddrinks purify theenhancer
\subsubsection{Brewer} \label{feat::brewer}
    Increase your level of proficiency with Brewer's Supplies.
    This feat can be re-taken until you are an expert with the toolset.

    You add your Constitution modifier to checks with Brewer's Supplies.

    The cost associated to each item only accounts for the required reagents, not the barrels used to contain the liquid or the tankard used to drink it.
    Unless specified otherwise in the item's name, the DC of a brew relates to one small keg of the stuff, which fills 8 tankards before being drained.
\subsubsection{Purify} \label{feat::purify}
    Using brewing techniques you can purify dirty water with your Brewer's Supplies.
    As part of a short rest, you can purify doses of water equal to 4 + your proficiency bonus with Brewer's Supplies.
    \paragraph{Requirements} Competent proficiency with Brewer's Supplies.
\subsubsection{Chug!} \label{feat::chug}
    Instead of a minute, you can drink a brew using 3 actions, gaining the normal benefits associated to it.
    \paragraph{Requirements} Skilled proficiency with Brewer's Supplies.
\subsubsection{The Enhancer} \label{feat::theenhancer}
    As part of a short rest, you can quickly brew a tankard worth of The Enhancer, a brew unique to master brewers.
    Upon drinking, the drinker gains the effect associated to the drink for an hour.

    The six variants of The Enhancer are listed.
    \paragraph{Bear's Endurance} The target has advantage on Constitution checks.
    It also gains 2d6 temporary hit points, which are lost when the effect ends.
    \paragraph{Bull's Strength} The target has advantage on Strength checks, and their carrying capacity doubles.
    \paragraph{Cat's Grace} The target has advantage on Dexterity checks.
    It also doesn't take damage from falling 4 meters or less if it isn't incapacitated.
    \paragraph{Eagle's Splendor} The target has advantage on Charisma checks.
    \paragraph{Fox's Cunning} The target has advantage on Intelligence checks.
    \paragraph{Owl's Wisdom} The target has advantage on Wisdom checks.

    You can take this feat two additional times.
    The second, you extend the duration of the effect to 8 hours.

    The third, you extend the duration of the effect to 24 hours, and can brew two tankards worth of The Enhancer per short rest.
    These tankards can be different variants of the brew.
    \paragraph{Requirements} Skilled proficiency with Brewer's Supplies.
\subsubsection{Mixed Drinks (2 FP)} \label{feat::mixeddrinks}
    As part of a short rest, you can mix two brews to combine their effects.
    To do this, you must succeed on a DC 15 check with your Brewer's Supplies.
    On a successful check you combine the brews' effects, and are left with an amount of the combined brew equal to half the sum of the two brews --- the remainder is lost in the mixing process.
    \paragraph{Requirements} Expert proficiency with Brewer's Supplies.

% CALLIGRAPHER'S SUPPLIES calligrapher bookminded encriptedwriting guentsuetattoos forger
\subsubsection{Calligrapher} \label{feat::calligrapher}
    Increase your level of proficiency with calligrapher's supplies.
    This feat can be re-taken until you are an expert with the toolset.

    You add your Intelligence modifier to checks with calligrapher's supplies.
\subsubsection{Book-Minded} \label{feat::bookminded}
    You have advantage on any check made to identify someone's handwriting or the source of a document, paper, or ink.
    \paragraph{Requirements} Competent proficiency with Calligrapher's Supplies.
\subsubsection{Encripted Writing} \label{feat::encriptedwriting}
    A master of codes, you are able to create written ciphers.
    Other can't decipher a code you create unless you teach them, they succeed on an Intelligence check (DC equal to 8 + your proficiency bonus with Calligrapher's Supplies), or they use magic to decipher it.

    If you already have the ability to create ciphers from another feat, add a +2 to the DC required to decipher your codes.
    \paragraph{Requirements} Skilled proficiency with Calligrapher's Supplies.
\subsubsection{Forger} \label{feat::forger}
    You gain proficiency with a Forgery kit, using your proficiency bonus with Calligrapher's Supplies for checks using it.
    You can take this feat two additional times, gaining different effects each time:
    \begin{itemize}
        \item The second, you learn to protect yourself with multiple layers of identities and misdirection.
        When you forge a document, you can leave behind false clues that suggest a different person was the forger.
        When a reader spots that the document is forged, it also discovers these clues.
        Identifying that these clues are false requires an additional Intelligence (Investigation) check against the same DC used to identify the original forgery.
        \item The third, you can permanently commit a person's handwriting to memory if you spend a long rest studying it.
        You can imitate this handwriting to perfection, and it is impossible to discern it from the original.
        You can only have one handwriting commited to memory in this way.
    \end{itemize}
    \paragraph{Requirements} Skilled proficiency with Calligrapher's Supplies.
\subsubsection{Guen Tsue Tattoos (2 FP)} \label{feat::guentsuetattoos}
    Using your calligrapher's supplies, you are able to inscribe tattoos into a creature as part of a long rest.
    The tattoos are based on runes from the Guen Tsue school of magic, and contain awesome effects for the tattooed creature to invoke.
    A Guen Tsue tattoo requires attunement from the tattooed creature.

    The effects available to you creature:
    \begin{itemize}
        \item \textbf{Absorbing Tattoo}.
        This colored tattoo grants you resistance to one damage type between acid, cold, fire, force, lightning, necrotic, poison, psychic, radiant, and thunder.
        The damage type is chosen by the tattoo artist while applying it.
        When you take damage of the chosen type, you can use your reaction to gain immunity against that instance of the damage, and you regain a number of hit points equal to half the damage you would have taken.
        Once this reaction is used, it can't be used again until the next dawn.
        \item \textbf{Barrier Tattoo}.
        While you aren't wearing armor, this tattoo depicting liquid metal grants you an Armor Class of 12 + your dexterity bonus.
        You can use a shield and still gain this benefit.
        \item \textbf{Blood Fury Tattoo}.
        This tattoo evokes fury in its shape and color.
        The tattoo has 5 charges, and it regains all expended charges daily at dawn.
        When you hit a creature with a weapon attack, you can expend a charge to deal an extra 4d6 necrotic damage to the target, and you regain a number of hit points equal to half the necrotic damage dealt.

        In addition, When a creature you can see damages you, you can expend a charge and use your reaction to make a melee attack against that creature, with advantage on your attack roll.
        \item \textbf{Coiling Grasp Tattoo}.
        While this intertwining tattoo is on your skin, you can, as an action, cause the tattoo to extrude into inky tendrils, which reach for a creature you can see within 3 meters of you.
        The creature must succeed on a DC 14 Strength saving throw or take 3d6 force damage and be grappled by you.
        As an action, the creature can escape the grapple by succeeding on a DC 14 Strength (Athletics) or Dexterity (Acrobatics) check.
        The grapple also ends if you halt it (no action required), if the creature is ever more than 3 meters away from you, or if you use this tattoo on a different creature.
        \item \textbf{Eldritch Claw Tattoo}.
        While this jagged tattoo is on your skin, you gain a +1 bonus to attack and damage rolls with unarmed strikes.
        In addition, you can use an action to empower the tattoo for 1 minute.
        For the duration, each of your melee attacks with a weapon or an unarmed strike can reach a target up to 3 meters away from you, as inky tendrils launch toward the target.
        In addition, your melee attacks deal an extra 1d6 force damage on a hit.
        Once used, this action can't be used again until the next dawn.
        \item \textbf{Lifewell Tattoo}.
        When you would be reduced to 0 hit points, you drop to 1 hit point instead.
        Once used, this property can't be used again until the next dawn.
        \item \textbf{Spellwrought Tattoo}.
        This coin-sized tattoo contains a single cantrip, chosen by the tattoo artist while applying it.
        You can cast this cantrip requiring no material components.
        The ability modifier for this spell is +3, the save DC is 13 and the attack bonus is +5.
    \end{itemize}

    A creature can remove a tattoo from its body as part of a long rest, ending its attunement to it.
    \paragraph{Requirements} Expert proficiency with Calligrapher's Supplies.

% CARPENTER'S TOOLS carpenter fortify instantfortress laminararmor
\subsubsection{Carpenter} \label{feat::carpenter}
    Increase your level of proficiency with Carpenter's Tools.
    This feat can be re-taken until you are an expert with the toolset.

    You add your Strength modifier to checks with Carpenter's Tools.

    From mastery with these tools, you have access to large items made of wood, like barrels, battering rams, boats, etc.
    The cost associated to each item only accounts for the required wood.
\subsubsection{Fortify} \label{feat::fortify}
    You can temporarily fortify a door, wall, or window using two actions.
    The DC needed to open or damage it in any way increases by 5.
    An object can be fortified in this way only once, and the fortifications fall apart 8 hours after being applied.
    \paragraph{Requirements} Competent proficiency with Carpenter's Tools.
\subsubsection{Instant Fortress} \label{feat::instantfortress}
    With one minute of work and raw materials, you can create a wooden barrier no more than 2 meters in any dimension.
    The barrier has an AC equal to your proficiency with carpenter's tools and an HP equal to three times that.
    It is immune to poison and psychic damage, and is vulnerable to fire damage.
    The barrier works as full cover against projectiles, and if it measures less than 1 by 1 meters, it can be used as a tower shield.

    The barrier collapses 8 hours after being assembled.
    \paragraph{Requirements} Skilled proficiency with Carpenter's Tools.
\subsubsection{Laminar Armor (2 FP)} \label{feat::laminararmor}
    You gain access to craft all heavy armor and shields by yourself --- even if they would normally require other toolsets.
    All items you craft in this way are made of wood, and weigh twice their normal weight.

    In addition, all armor crafted in this way has the Composite and Noisy properties, overriding the properties they would normally have.
    \paragraph{Requirements} Expert proficiency with Carpenter's Tools.

% CARTOGRAPHER'S SUPPLIES cartographer copyist everpresent proliferator
\subsubsection{Cartographer} \label{feat::cartographer}
    Increase your level of proficiency with cartographer's supplies.
    This feat can be re-taken until you are an expert with the toolset.

    You add your Intelligence modifier to checks with cartographer's supplies.

    While you travel, you can draw a map as you go in addition engaging in other activity.
\subsubsection{Copyist} \label{feat::copyist}
    You gain the ability to copy another map as part of a long rest.
    To do this, you consume materials that cost half the cost of the map you're copying, and must succeed on an ability check using the Cartographer's Kit with a DC proportional to the scale and the level of detail of the map, at the DM's discretion.
    If you have spent a month of more exploring the region in the map, you make this ability check with advantage.
    \paragraph{Requirements} Competent proficiency with cartographer's supplies.
\subsubsection{Proliferator} \label{feat::proliferator}
    While drawing a map you can simultaneously draw a second copy of it without spending any additional time.
    \paragraph{Requirements} Skilled proficiency with cartographer's supplies.
\subsubsection{Ever-Present (2 FP)} \label{feat::everpresent}
    Your time studying maps has given you the ability to understand vantage points and points of interest even without having seen them before.
    You cannot become lost by any means, and you can always accurately tell where you are in the world as part of a short rest.
    \paragraph{Requirements} Expert proficiency with cartographer's supplies.

% COBBLER'S TOOLS cobbler
%   - Dexterity : Shoes and boots
\subsubsection{Cobbler} \label{feat::cobbler}
    Increase your level of proficiency with cobbler's tools.
    This feat can be re-taken until you are an expert with the toolset.
\subsubsection{NAME} \label{feat::name}
    DESCRIPTION
    \paragraph{Requirements} Competent proficiency with cobbler's tools.
\subsubsection{NAME} \label{feat::name}
    DESCRIPTION
    \paragraph{Requirements} Skilled proficiency with cobbler's tools.
\subsubsection{NAME (2 FP)} \label{feat::name}
    DESCRIPTION
    \paragraph{Requirements} Expert proficiency with cobbler's tools.

Whenever a cobbler spends a long rest repairing and preparing the party’s footwear for the journey ahead, they are granted a bonus to their travel pace. Using the RAW travel pace table below, the party can move at a travel pace that is one level higher, while retaining their normal benefits. For example a party travelling at a fast pace is not subjected to the penalty of -5 to passive perception. This bonus lasts for one week of travel.

% COOKING UTENSILS chef cook sweettreat warmmeal
\subsubsection{Cook} \label{feat::cook}
    Increase your level of proficiency with cooking utensils.
    This feat can be re-taken until you are an expert with the toolset.

    You add your Constitution modifier to checks with cooking utensils.

    The cost associated to each item only accounts for the required reagents.
    Unless specified otherwise in the item's name, the DC of food relates to 8 rations worth of the stuff.
\subsubsection{Warm Meal} \label{feat::warmmeal}
    By cooking them, you can make rations more effective.
    As part of a short rest, you can cook a number of rations equal to 4 + your proficiency bonus with Cooking Utensils.
    The number of cooked rations produced is equal to double the original number of rations, but spoil if left untouched for more than 24 hours.
    \paragraph{Requirements} Competent proficiency with Cooking Utensils.
\subsubsection{Sweet Treat} \label{feat::sweettreat}
    With one hour of work or as part of a short rest, you can cook a number of treats equal to your proficiency bonus with Cooking Utensils.
    These special treats last 8 hours after being made.
    A creature can use a bonus action to eat one of those treats to gain temporary hit points equal to your proficiency bonus with Cooking Utensils.
    \paragraph{Requirements} Skilled proficiency with Cooking Utensils.
\subsubsection{Chef (2 FP)} \label{feat::chef}
    As part of a short rest, you can cook special food, provided you have ingredients and Cooking Utensils on hand.
    You can prepare enough of this food for a number of creatures equal to 4 + your proficiency bonus with Cooking Utensils.
    At the end of the short rest, any creature who eats the food and spends one or more Hit Dice to regain hit points regains extra hit points equal to 2d8 + their Constitution modifier.
    \paragraph{Requirements} Expert proficiency with Cooking Utensils.

% GAMING KIT
%   - Intelligence?
% \subsubsection{Player} \label{feat::player}
%     Increase your level of proficiency with Gaming Kits.
%     This feat can be re-taken until you are an expert with the item.
%
%     When you take this feat, choose a specific Gaming Kit with which you are proficient.
%     This doesn't apply for other Gaming Kits, but spending a long rest practicing with one allows you to apply your proficiency bonus to rolls related to it.
% \subsubsection{NAME} \label{feat::name}
%     DESCRIPTION
%     \paragraph{Requirements} Competent proficiency with Gaming kit.
% \subsubsection{NAME} \label{feat::name}
%     DESCRIPTION
%     \paragraph{Requirements} Skilled proficiency with Gaming kit.
% \subsubsection{NAME (2 FP)} \label{feat::name}
%     DESCRIPTION
%     \paragraph{Requirements} Expert proficiency with Gaming kit.

% GLASSBLOWER'S TOOLS glassblower fineflask glassenhancement glassgrenade
\subsubsection{Glassblower} \label{feat::glassblower}
    Increase your level of proficiency with Glassblower's Tools.
    This feat can be re-taken until you are an expert with the toolset.

    You add your Dexterity modifier to checks with Glassblower's Tools.

    From mastery with these tools, you have access to any item made with glass, like vials, cups, and flasks.
    The cost associated to each item only accounts for the required glass.

    % To work with your glassblowers tools you need access to a furnace capable of reaching temperatures of more than 1,500$\degree$ Celcius.
\subsubsection{Glass Grenade} \label{feat::glassgrenade}
    As part of a short rest, you can bundle up discarded glass dust in a bag to craft a makeshift ``glass grenade''.
    You can throw this glass grenade at a range of 4/12 mt., and it violently explodes out in a 1 mt. radius upon hitting the floor.
    All creatures in its radius must succeed on a DC 15 Dexterity saving throw or take 1d4 piercing damage.
    Furthermore, the target loses 1d4 hit points at the start of each of its turns due to the shattered glass in the wound.
    Any creature can take two actions to staunch the wound with a successful DC 12 Wisdom (Medicine) check.

    This effect is not cummulative.
    \paragraph{Requirements} Competent proficiency with Glassblower's Tools.
\subsubsection{Fine Flask} \label{feat::fineflask}
    Flasks and vials you make are especially effective, making a great addition to an already fine liquid.
    Any potion drank from one flask or vial you make has its duration doubled.
    Bloodwell vials made in this fashion have their number of uses doubled.

    In addition, you learn how to make glass lining inside tankards, which adds this same effects to brews drank from them.
    \paragraph{Requirements} Skilled proficiency with Glassblower's Tools.
\subsubsection{Glass Enhancements (2 FP)} \label{feat::glassenhancement}
    You can enhance a weapon's effect by adding glass into the mix.
    As part of a long rest, you can work with a weaponsmith to add one of the following effects to a slashing or piercing weapon:
    \subparagraph{Hollowed Weapon} When a poison is applied to the weapon, the time it remains applied is doubled, and it can apply one additional dose of the poison before being drained.
    \subparagraph{Grainy Edge} By adding glass grain to a weapons edge it becomes more effective.
    The weapon gains a +1 bonus to attack and damage rolls.
    \subparagraph{Glass Shards} This weapon contains loosely glued shards of glass on its faces.
    On a successful hit, the target takes 1d4 damage on the beginning of each of its turns.
    Any creature can take two actions to staunch the wound with a successful DC 12 Wisdom (Medicine) check.
    This effect is not cummulative.

    A weapon can only have one of these effects applied at the same time.
    \paragraph{Requirements} Expert proficiency with Glassblower's Tools.

% HEALER'S KIT healer combatmedic physician quickmender travellingdoctor
\subsubsection{Healer} \label{feat::healer}
    Increase your level of proficiency with the healer's kit.
    This feat can be re-taken until you are an expert with the kit.

    You add your Wisdom modifier to checks with the healer's kit.
\subsubsection{Quick Mender} \label{feat::quickmender}
    Used to working in the field, you can heal minor injuries as part of a short rest expending one use of your healer's kit.

    A creature healed with this feat also gains a number of temporary hit points equal to its Constitution modifier (minimum of one).
    \paragraph{Requirements} Competent proficiency with healer's kit.
\subsubsection{Physician} \label{feat::physician}
    As an avid physician, you can expend 3 uses of a healer's kit to heal a major injury as part of a long rest.
    The creature must succeed on a DC 12 Constitution saving throw for the treatment to work.

    A creature healed with this feat also gains a number of temporary hit points equal to double its Constitution modifier (minimum of two).
    \paragraph{Requirements} Skilled proficiency with healer's kit.
\subsubsection{Combat Medic} \label{feat::combatmedic}
    You are able to mend wounds quickly and get your allies back in the fight.
    Using two actions, you can spend one use of a healer's kit to tend to a creature and restore 1d6 + your proficiency modifier with the healer's kit hit points to it, plus additional hit points equal to the creature's maximum number of Hit Dice.
    The creature can't regain hit points from this feat again until it finishes a short rest.

    You can take this feat two additional times, increasing the number of dice rolled to 2d6 the second time and 3d6 the third.
    \paragraph{Requirements} Skilled proficiency with Healer's kit.
\subsubsection{Travelling Doctor (2 FP)} \label{feat::travellingdoctor}
    You keep your allies in top shape during your travels.
    The minimum number of hit points you and your allies regain from a hit die roll is equal to their Constitution modifier (minimum of 2).
    \paragraph{Requirements} Expert proficiency with Healer's kit.

% HERBALISM KIT herbalist adeptpoisoner healingoinments naturalremedies
%   - Wisdom : Reagents, this includes stuff for alchemy, brewing, cooking, and poisonmaking.
\subsubsection{Herbalist} \label{feat::herbalist}
    Increase your level of proficiency with herbalism kit.
    This feat can be re-taken until you are an expert with the toolset.

    You add your Wisdom modifier to checks with a the herbalism kit.

    You can use your proficiency with this kit to gather reagents from nature as part of a long rest.
    Make a list with the rarity of reagents you want to find during the short rest, then roll a skill check using the herbalism kit.
    The DC of one material is the same as the one for crafting an object of the same rarity.

    Reagents you gather can be used to craft brews, food, poisons, and potions.
\subsubsection{Adept Poisoner} \label{feat::adeptpoisoner}
    You can make basic poison (see page \pageref{item::basicpoison} using your herbalism kit as part of a long rest.
    If you gather the reagents yourself, you only need half the normal amount of materials to make the poison, and can make it as part of the long rest you used to gather them.
    \paragraph{Requirements} Competent proficiency with herbalism kit.
\subsubsection{Healing Ointments} \label{feat::healingoinments}
    As part of a short rest, you can use gathered reagents to cure one or more creatures' wounds.
    The amount healed depends on the rarity of the reagents, and is specified in the Healing Ointment table below.

    \begin{DndTable}[width=\linewidth, header=Healing Oinment]{lX}
        \textbf{Rarity} & \textbf{Healing} \\
        Mundane         & 1d4              \\
        Plain           & 1d6              \\
        Common          & 1d8              \\
        Uncommon        & 2d8              \\
        Rare            & 3d8              \\
        Very Rare       & 4d8              \\
        Legendary       & 5d8
    \end{DndTable}
    \paragraph{Requirements} Skilled proficiency with Herbalism Kit.
\subsubsection{Natural Remedies (2 FP)} \label{feat::naturalremedies}
    You gain advantage on saving throws against poisons and diseases.

    In addition, you can use your herbalism kit to administer a herbal remedy to help a creature recover from their ailments as part of a short rest.
    The creature gains advantage on saving throws against poison and disease for the next day.
    \paragraph{Requirements} Expert proficiency with Herbalism Kit.

% JEWELER'S TOOLS
%   - Dexterity : Wondrous items
% \subsubsection{Jeweler} \label{feat::jeweler}
%     Increase your level of proficiency with Jeweler's Tools.
%     This feat can be re-taken until you are an expert with the toolset.
% \subsubsection{NAME} \label{feat::name}
%     DESCRIPTION
%     \paragraph{Requirements} Competent proficiency with Jeweler's Tools.
% \subsubsection{NAME} \label{feat::name}
%     DESCRIPTION
%     \paragraph{Requirements} Skilled proficiency with Jeweler's Tools.
% \subsubsection{NAME (2 FP)} \label{feat::name}
%     DESCRIPTION
%     \paragraph{Requirements} Expert proficiency with Jeweler's Tools.

% LEATHERWORKER'S TOOLS
%   - Dexterity : Light armour, hide armour?
% \subsubsection{Leatherworker} \label{feat::leatherworker}
%     Increase your level of proficiency with Leatherworker's Tools.
%     This feat can be re-taken until you are an expert with the toolset.
% \subsubsection{NAME} \label{feat::name}
%     DESCRIPTION
%     \paragraph{Requirements} Competent proficiency with Leatherworker's Tools.
% \subsubsection{NAME} \label{feat::name}
%     DESCRIPTION
%     \paragraph{Requirements} Skilled proficiency with Leatherworker's Tools.
% \subsubsection{NAME (2 FP)} \label{feat::name}
%     DESCRIPTION
%     \paragraph{Requirements} Expert proficiency with Leatherworker's Tools.

% MASON'S TOOLS
%   - Strength : Weapons and armour made of stone (e.g. obsidian)
% \subsubsection{Mason} \label{feat::mason}
%     Increase your level of proficiency with Mason's Tools.
%     This feat can be re-taken until you are an expert with the toolset.
% \subsubsection{NAME} \label{feat::name}
%     DESCRIPTION
%     \paragraph{Requirements} Competent proficiency with Mason's Tools.
% \subsubsection{NAME} \label{feat::name}
%     DESCRIPTION
%     \paragraph{Requirements} Skilled proficiency with Mason's Tools.
% \subsubsection{NAME (2 FP)} \label{feat::name}
%     DESCRIPTION
%     \paragraph{Requirements} Expert proficiency with Mason's Tools.

% MUSICAL INSTRUMENTS
%   - ???
% \subsubsection{Musician} \label{feat::musician}
%     Increase your level of proficiency with Musical Instruments.
%     This feat can be re-taken until you are an expert with the toolset.
%
%     When you take this feat, choose a specific instrument with which you are proficient.
%     This doesn't apply for other instruments, but spending a long rest practicing with one allows you to apply your proficiency bonus to rolls related to it.
% \subsubsection{NAME} \label{feat::name}
%     DESCRIPTION
%     \paragraph{Requirements} Competent proficiency with Musical Instruments.
% \subsubsection{NAME} \label{feat::name}
%     DESCRIPTION
%     \paragraph{Requirements} Skilled proficiency with Musical Instruments.
% \subsubsection{NAME (2 FP)} \label{feat::name}
%     DESCRIPTION
%     \paragraph{Requirements} Expert proficiency with Musical Instruments.

% NAVIGATOR'S TOOLS
%   - ???
% \subsubsection{Navigator} \label{feat::navigator}
%     Increase your level of proficiency with Navigator's Tools.
%     This feat can be re-taken until you are an expert with the toolset.
% \subsubsection{NAME} \label{feat::name}
%     DESCRIPTION
%     \paragraph{Requirements} Competent proficiency with Navigator's Tools.
% \subsubsection{NAME} \label{feat::name}
%     DESCRIPTION
%     \paragraph{Requirements} Skilled proficiency with Navigator's Tools.
% \subsubsection{Seafarer} \label{feat::seafarer}
%     You gain proficiency with Water Vehicles, using your proficiency bonus with Navigator's Tools for checks using them.
%     You can take this feat two additional times, gaining different effects each time:
%     \begin{itemize}
%         \item The second, ...
%         \item The third, ... % you cool and increase a ship's speed
%     \end{itemize}
%     \paragraph{Requirements} Skilled proficiency with Navigator's Tools.
% \subsubsection{NAME (2 FP)} \label{feat::name}
%     DESCRIPTION
%     \paragraph{Requirements} Expert proficiency with Navigator's Tools.

% PAINTER'S SUPPLIES
%   - Dexterity : Scrolls
% \subsubsection{Painter} \label{feat::painter}
%     Increase your level of proficiency with Painter's Supplies.
%     This feat can be re-taken until you are an expert with the toolset.
% \subsubsection{NAME} \label{feat::name}
%     DESCRIPTION
%     \paragraph{Requirements} Competent proficiency with Painter's Supplies.
% \subsubsection{NAME} \label{feat::name}
%     DESCRIPTION
%     \paragraph{Requirements} Skilled proficiency with Painter's Supplies.
% \subsubsection{NAME (2 FP)} \label{feat::name}
%     DESCRIPTION
%     \paragraph{Requirements} Expert proficiency with Painter's Supplies.

% POISONER'S KIT
%   - Intelligence - Poisons
% \subsubsection{Poisoner} \label{feat::poisoner}
%     You gain proficiency with Poisoner's Kits, using your proficiency bonus with Herbalism Kits for checks using them.
%     You can take this feat two additional times, gaining different effects each time:
%     \begin{itemize}
%         \item The second, your poisons ignore resistance to poison damage.
%         Additionally, you can coat a weapon in poison using one action instead of two.
%         \item The third, the Constitution saving throws of the poisons you make are increased by half proficiency bonus with the Herbalism Kit.
%     \end{itemize}
%     \paragraph{Requirements} Skilled proficiency with Herbalism Kit.

% POTTER'S TOOLS
%   - Dexterity : ...?
% \subsubsection{Potter} \label{feat::potter}
%     Increase your level of proficiency with Potter's Tools.
%     This feat can be re-taken until you are an expert with the toolset.
% \subsubsection{NAME} \label{feat::name}
%     DESCRIPTION
%     \paragraph{Requirements} Competent proficiency with Potter's Tools.
% \subsubsection{NAME} \label{feat::name}
%     DESCRIPTION
%     \paragraph{Requirements} Skilled proficiency with Potter's Tools.
% \subsubsection{NAME (2 FP)} \label{feat::name}
%     DESCRIPTION
%     \paragraph{Requirements} Expert proficiency with Potter's Tools.

% SMITH'S TOOLS
%   - Strength : Metal weapons, heavy armour, medium armour (except hide)
% \subsubsection{Smith} \label{feat::smith}
%     Increase your level of proficiency with Smith's Tools.
%     This feat can be re-taken until you are an expert with the toolset.
% \subsubsection{NAME} \label{feat::name}
%     DESCRIPTION
%     \paragraph{Requirements} Competent proficiency with Smith's Tools.
% \subsubsection{NAME} \label{feat::name}
%     DESCRIPTION
%     \paragraph{Requirements} Skilled proficiency with Smith's Tools.
% \subsubsection{NAME (2 FP)} \label{feat::name}
%     DESCRIPTION
%     \paragraph{Requirements} Expert proficiency with Smith's Tools.

% Locksmithing Tools
%   - Dexterity : ???
% \subsubsection{Locksmith} \label{feat::locksmith}
%     Increase your level of proficiency with Locksmithing Tools.
%     This feat can be re-taken until you are an expert with the toolset.
%
%     Locksmithing Tools can be used to make checks to open locks or disarm traps, which are rolled after a minute fidgeting with the lock.
% \subsubsection{NAME} \label{feat::name}
%     DESCRIPTION
%     \paragraph{Requirements} Competent proficiency with Locksmithing Tools.
% \subsubsection{NAME} \label{feat::name}
%     DESCRIPTION
%     \paragraph{Requirements} Skilled proficiency with Locksmithing Tools.
% \subsubsection{Thief} \label{feat::thief}
%     You gain proficiency with Thieves' Tools, using your proficiency bonus with Locksmithing Tools for checks using them.
%     Thieves' tools reduce the time it takes you to open a lock or disarm a trap to two actions.
%
%     You can take this feat two additional times, gaining different effects each time:
%     \begin{itemize}
%         \item The second, ...
%         \item The third, the time it takes you to try to disarm a trap or open a lock is reduced from two to one action, provided you use your thieves' tools to do this.
%     \end{itemize}
%     \paragraph{Requirements} Skilled proficiency with Locksmithing Tools.
% \subsubsection{NAME (2 FP)} \label{feat::name}
%     DESCRIPTION
%     \paragraph{Requirements} Expert proficiency with Locksmithing Tools.

% TINKER'S TOOLS
%   - Dexterity : Wondrous items, firearms
% \subsubsection{Tinkerer} \label{feat::tinkerer}
%     Increase your level of proficiency with Tinker's Tools.
%     This feat can be re-taken until you are an expert with the toolset.
% \subsubsection{NAME} \label{feat::name}
%     DESCRIPTION
%     \paragraph{Requirements} Competent proficiency with Tinker's Tools.
% \subsubsection{NAME} \label{feat::name}
%     DESCRIPTION
%     \paragraph{Requirements} Skilled proficiency with Tinker's Tools.
% \subsubsection{NAME} \label{feat::name}
%     DESCRIPTION - UPGRADABLE
%     \paragraph{Requirements} Skilled proficiency with Tinker's Tools.
% \subsubsection{NAME (2 FP)} \label{feat::name}
%     DESCRIPTION
%     \paragraph{Requirements} Expert proficiency with Tinker's Tools.

% VEHICLES (LAND)
% \subsubsection{Rider} \label{feat::rider}
%     Increase your level of proficiency with Land Vehicles.
%     This feat can be re-taken until you are an expert with the proficiency.
%
%     Mounts act on the same initiative as you, but they can only take the Disengage, Dodge, and Move actions.
% \subsubsection{NAME} \label{feat::name}
%     DESCRIPTION
%     \paragraph{Requirements} Competent proficiency with Land Vehicles.
% \subsubsection{NAME} \label{feat::name}
%     DESCRIPTION
%     \paragraph{Requirements} Skilled proficiency with Land Vehicles.
% \subsubsection{Exotic Rider} \label{feat::exoticrider}
%     You gain proficiency with Air Vehicles, using your proficiency bonus with Land Vehicles for checks using them.
%     You can take this feat two additional times, gaining different effects each time:
%     \begin{itemize}
%         \item The second, ...
%         \item The third, ... % Half falling speed and reduce falling damage die from d6s to d4s.
%     \end{itemize}
%     \paragraph{Requirements} Skilled proficiency with Land Vehicles.
% \subsubsection{NAME (2 FP)} \label{feat::name}
%     DESCRIPTION
%     \paragraph{Requirements} Expert proficiency with Land Vehicles.

% WEAVER'S TOOLS
%   - Dexterity : Cloaks, robes, and clothing
% \subsubsection{Weaver} \label{feat::weaver}
%     Increase your level of proficiency with Weaver's Tools.
%     This feat can be re-taken until you are an expert with the toolset.
% \subsubsection{NAME} \label{feat::name}
%     DESCRIPTION
%     \paragraph{Requirements} Competent proficiency with Weaver's Tools.
% \subsubsection{NAME} \label{feat::name}
%     DESCRIPTION
%     \paragraph{Requirements} Skilled proficiency with Weaver's Tools.
% \subsubsection{Disguise Maker} \label{feat::disguisemaker}
%     You gain proficiency with Disguise Kits, using your proficiency bonus with Weaver's Tools for checks using them.
%     You can take this feat two additional times, gaining different effects each time:
%     \begin{itemize}
%         \item The second, you learn to observe the most distinguishing attributes of a person.
%         If you spend one hour observing or interacting with a creature, you can then craft a disguise that can be used to mimic that creature during a long rest.
%         \item The third, you learn how to procure a convincing disguise with stuff you find around you.
%         Taking an hour, you can make a disguise that makes sense with the materials you have around you (at the DM's discretion), which is as effective as one made with a Disguise Kit with similar materials.
%     \end{itemize}
%     \paragraph{Requirements} Skilled proficiency with Weaver's Tools.
% \subsubsection{NAME (2 FP)} \label{feat::name}
%     DESCRIPTION
%     \paragraph{Requirements} Expert proficiency with Weaver's Tools.

% WOODCARVER woodcarver adhocweaponry bonecarver chiseledhandle
\subsubsection{Woodcarver} \label{feat::woodcarver}
    Increase your level of proficiency with woodcarver's tools.
    This feat can be re-taken until you are an expert with the toolset.

    You add your Dexterity modifier to checks with woodcarver's tools.

    From mastery with these tools, you have access to small items made with wood and to apply engravings on wooden items and surfaces.
    These items include arrows, bolts, rods, etc.
\subsubsection{Ad-hoc Weaponry} \label{feat::adhocweaponry}
    Taking an hour of work, you can create a number of improvised weapons equal to your proficiency bonus with woodcarver's tools.
    The weapons available to you are clubs, daggers, greatclubs, javelins, light hammers, maces, quarterstaves, and spears.
    The weapons quickly lose their effectivity, and can only be used for one combat encounter.

    To use this feat, you must have your woodcarver tools at hand and enough wood to make all the weapons.
    \paragraph{Requirements} Competent proficiency with Woodcarver's Tools.
\subsubsection{Chiseled Handle} \label{feat::chiseledhandle}
    Using your woodcarver's tools, you can chisel the wooden handle of a weapon or shield as part of a short rest.
    Equipment worked in this way fits nicely into their user's hand, and any check that attempts to disarm them is made with disadvantage.
    \paragraph{Requirements} Skilled proficiency with Woodcarver's Tools.
\subsubsection{Bonecarver (2 FP)} \label{feat::bonecarver}
    You can use your woodcarving tools to work with bone.
    Provided you can get the raw materials, you can craft bone charms and qualars as you would craft wooden charms.
    \paragraph{Requirements} Expert proficiency with Woodcarver's Tools.

% NOTE: Dunno where to place these so here, they're your problem now.
% \subsubsection{Linguist} \label{feat::linguist}
% \paragraph{RANK 2} You have advantage on Charisma (Languages) checks to determine the language that a creature is speaking.
% \paragraph{RANK 2} If you see a creature's mouth while it is speaking a language you understand, you can interpret what it's saying by reading its lips.
%
% \subsubsection{Educated} \label{feat::educated}
% \paragraph{RANK 3} You have an intuitive understanding of how to use all sorts of machinery, and can operate siege weapons and heavy machines without performing any ability checks.

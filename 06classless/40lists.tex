% !TEX root = ../main.tex
\section{Lists} % TODO: This needs a better name urgently.
This section contains relevant lists for character advancement.
This is temporary until the author finds a better way to sort this stuff out.

\subsection*{Actions} \label{ssec::actions}
    Some feats provide with new actions that can be done.
    Some of them are listed here for easy reference.

    % === USED ============================================================== %
    \subsubsection{Aim $\circ$} \label{act::aim}
    You give yourself advantage on your next attack roll, which can be made until the end of your next turn.
    \subsubsection{Block $\diamond$} \label{act::block}
    You can use your reaction to prepare yourself against one melee or ranged attack.
    You increase your AC by 1d6 against the attack.
    If the attack does hit you, you take half damage from it (rounded down).
    \subsubsection{Distracting Strike $\circ\circ$} \label{act::distractingstrike}
    You use your action to distract a creature with an attack, giving your allies an opening.
    You roll damage normally, and the next attack roll against the target by an creature other than you has advantage if the attack is made before the start of your next turn.
    \subsubsection{Quick Draw $\diamond$} \label{act::quickdraw}
    You can make an attack of opportunity with an equipped thrown or ranged weapon when a creature moves out of a 9-meter radius around you.
    \subsubsection{Mark $\circ\circ$} \label{act::mark}
    You choose a creature you can see within 27 meters and mark it as your quarry.
    For up to one hour, you deal an extra 1d6 damage to the target whenever you hit it with a weapon attack, and you have advantage on any Intelligence (Investigation), Wisdom (Perception), and Wisdom (Survival) checks you make to find it.
    You can use this action a number of times equal to your Wisdom modifier, and you restore all expended uses on a short rest.
    \subsubsection{Parry $\diamond$} \label{act::parry}
    When another creature attacks you with a melee attack, you can use your reaction to increase your AC by 1d6 + your Dexterity modifier against the attack.
    \subsubsection{Push $\circ$} \label{act::push}
    You try to push a target back.
    The target must be no more than one size larger than you and be within 1.5 meters of you.
    You make a Strength (Athletics) check contested by the target's Strength (Athletics) or Dexterity (Acrobatics) check (the target chooses the ability to use).
    If you win, you push the target up to 4.5 meters away from you.
    If you win by 10 or more, the target is also knocked prone.
    \subsubsection{Reckless Attack $\circ$} \label{act::recklessattack}
    You make a special melee weapon attack against a creature using a weapon with the heavy property.
    Apply a -5 penalty to the attack roll.
    If the attack hits, you add +10 to the attack's damage.
    \subsubsection{Reckless Shot $\circ$} \label{act::recklessshot}
    You make a special ranged weapon attack against a creature with a ranged weapon.
    Apply a -5 penalty to the attack roll.
    If the attack hits, you add +10 to the attack's damage.
    \subsubsection{Riposte $\diamond$} \label{act::riposte}
    When a creature misses you with a melee attack, you can use your reaction to make a melee weapon attack against the creature.
    \subsubsection{Sneak Attack $\circ\circ$} \label{act::sneakattack}
    You know how to strike subtly and exploit a foe's distraction.
    You do an attack that deal an extra 2d6 damage to one creature you hit with an attack if you have advantage on the attack roll.
    The attack must use a finesse or a ranged weapon.

    You don't need advantage on the attack roll if another enemy of the target is within 5 feet of it, that enemy isn't incapacitated, and you don't have disadvantage on the attack roll.
    You can use this action only once per turn.

    You can take this action various times.
    Each time after the first increases the number of d6s rolled by one.
    \subsubsection{Steal $\circ\circ$} \label{act::steal}
    As an action, you can make a Dexterity (Sleight of Hand) check contested by a creature's Wisdom (Perception) to plant something on someone else, conceal an object on a creature, lift a purse, or take something from a pocket.
    You can do this in the middle of an encounter.
    % === UNUSED ============================================================ %
    % \subsubsection{Careless Deflect} \label{tec::carelessdeflect}
    %     When a creature misses you with a melee attack roll, you can use your reaction to cause that attack to hit one creature of your choice, other than the attacker, that you can see within 1.5 meters of you.
    %
    % \subsubsection{Charge Stopper} \label{tec::chargestopper}
    %     You can set your polearm to receive a charge.
    %     As a bonus action, choose a creature you can see that is at least 6 meters away from you.
    %     If that creature moves within you polearm's reach on its next turn, you can make a melee attack against it with your polearm as a reaction.
    %     If the attack hits, the targets takes an extra damage die.
    %     You can't use this technique if the creature used the Disengage action before moving.
    %
    % \subsubsection{Commander's Strike} \label{tec::commandersstrike}
    %     When you take the Attack action on your turn, you can choose to attack only once and use a bonus action to direct one of your companions to strike.
    %     When you do so, choose a friendly creature who can see or hear you.
    %     That creature can immediately use its reaction to make one weapon attack, adding a d8 to the attack's damage roll.
    %
    % \subsubsection{Disarming Attack} \label{tec::disarmingattack}
    %     You use your action to attempt to disarm the target with an attack, forcing it to drop one item of your choice that it's holding.
    %     The target must make a Strength saving throw with a DC of 8 + your proficiency bonus + your Strength modifier.
    %     On a failed save, it drops the object you choose.
    %     The object lands at its feet.
    %
    % \subsubsection{Evasive Footwork} \label{tec::evasivefootwork}
    %     When you move, you can use a bonus action to add a d6 to your AC until you stop moving.
    %
    % \subsubsection{Goading Attack} \label{tec::goadingattack}
    %     You use your action to attack a creature and goad it into attacking you.
    %     The target must make a Wisdom saving throw with a DC of 8 + your proficiency bonus + your Charisma modifier.
    %     On a failed save, the target has disadvantage on all attack rolls against targets other than you until the end of your next turn.
    %
    % \subsubsection{Injure} \label{tec::injure}
    %     As an action, you can attack with the sole purpose of injuring a creature.
    %     Roll your weapon attack normally.
    %     On a hit, you deal only half damage, but the creature rolls on the minor injury chart.
    %     If the attack is a critical hit, the creature rolls on the major injury chart.
    %
    % \subsubsection{Maneuvering Attack} \label{tec::maneuveringattack}
    %     When you hit a creature with a melee attack, you can use your bonus action to maneuver one of your comrades into a more advantageous position.
    %     You choose a friendly creature who can see or hear you.
    %     That creature can use its reaction to move up to half its speed without provoking opportunity attacks from the target of your attack.
    %
    % \subsubsection{Menacing Attack} \label{tec::menacingattack}
    %     When you hit a creature with a weapon attack, you can use your bonus action to attempt to frighten the target.
    %     The target must make a Wisdom saving throw of a DC of 8 + your proficiency bonus + your Charisma modifier.
    %     On a failed save, it is frightened of you until the end of your next turn.
    %
    % \subsubsection{Precise Attack} \label{tec::preciseattack}
    %     When you make a weapon attack roll against a creature, you can use your bonus action to add a d8 to the attack roll.
    %     You can use this technique before or after making the attack roll, but before any effects of the attack are applied.
    %
    % \subsubsection{Rapid Repair} \label{tec::rapidrepair}
    %     You can attempt to repair a misfired (but not broken) firearm as a bonus action.
    %
    % \subsubsection{Violent Shot} \label{tec::violentshot}
    %     When you make a firearm attack against a creature, you can tune your weapon to enhance the volatility of the attack.
    %     The attack gains a +2 to the firearm's misfire score.
    %     If the attack hits, you can roll one additional weapon damage die.
    % \subsubsection{Predetermined Fate} \label{mtec::predeterminedfate}
    %     Whenever you make a saving throw and fail, you can reroll it and take the second result.
    %
    % \subsubsection{Quivering Palm} \label{mtec::quiveringpalm}
    %     This is the ultimate technique of the TODO:SCHOOL school, and can only be learned after you learn all other martial arts techniques.
    %     When you hit a creature with an unarmed strike, you can use the quivering palm technique.
    %     The creature must make a Constitution saving throw or take 10d10 bludgeoning damage, taking half damage on a success.
    %     You can use this technique only once per short rest.

\subsection*{Fighting Styles} \label{ssec::fightingstyles}
    \subsubsection{Archery 1}
        You gain a +2 bonus to attack rolls you make with ranged weapons.
    \subsubsection{Archery 2}
        Both your normal and long ranges are multiplied by 1.5 for all ranged weapons.
    \subsubsection{Battle Mastery 1}
        You gain a number a Maneuver Dice equal to half your level (rounded down).
        Your Maneuver Dice are d6s, and are special dice that you can use to enhance the effect of certain actions.
        You regain all expended Maneuver Dice at the end of a short rest.

        Whenever you take an action that involves a d20 roll which isn't the Attack or Cast a Spell action, you can choose to roll a Maneuver Die, adding the result of the die to the d20.
    \subsubsection{Battle Mastery 2}
        You increase the die associated to your Maneuver Dice from a d6 to a d8.

        In addition, when you roll for initiative and have no Maneuver Dice remaining, you regain 4 Maneuver Dice.
    \subsubsection{Dueling 1}
        When you are wielding a melee weapon in one hand and no other weapons, you gain a +2 bonus to damage rolls with that weapon.
    \subsubsection{Dueling 2}
        When you are wielding a melee weapon in one hand and no other weapons, you gain a +1 bonus to AC.
    \subsubsection{Great Weapon Fighting 1}
        When you roll a 1 or 2 on a damage die for an attack you make with a melee weapon that you are wielding with two hands, you can reroll the die and must use the new roll, even if the new roll is a 1 or a 2.
        The weapon must have the two-handed or versatile property for you to gain this benefit.
    \subsubsection{Great Weapon Fighting 2}
        You gain a +1 bonus to attack rolls you make with weapons with the two-handed or versatile property.

        In addition, you gain a +1 bonus to attack rolls you make with weapons with the heavy property.
    \subsubsection{Monastic Fighting 1}
        You gain a number of monastic dice equal to half your level (rounded down).
        Your monastic dice are d6s, and are special dice you can spend to use special abilities related to this fighting style.
        You regain all expended monastic dice at the end of a short rest.

        In addition, you learn three abilities when you learn this fighting style:
        \subparagraph{Concentrated Attack} After you hit a creature with an unarmed attack but before you roll damage for the attack, you can spend one monastic die and add the number rolled to the damage done.
        \subparagraph{Patient Defense} You can spend one monastic die and take the Dodge action using only one action.
        When you do this, add the number rolled to your AC until the start of your next turn.
        \subparagraph{Step of the Wind} You can spend one monastic die to take the Disengage action using only one action.
        When you do this, add the 1.5 times the number rolled to your movement speed (in meters) until the end of your turn.
    \subsubsection{Monastic Fighting 2}
        You increase the die associated to your monastic dice from a d6 to a d8.

        In addition, when you roll for initiative and have no monastic dice remaining, you regain 4 monastic dice.
    \subsubsection{Mounted Fighting 1}
        You have advantage on saving throws made to avoid falling off your mount.
        If you fall off your mount and descend no more than 3 meters, you can land on your feet if you're not incapacitated.

        Finally, mounting or dismounting a creature requires one action rather than two for you.

        To take this fighting style, you must be proficient with land vehicles.
    \subsubsection{Mounted Fighting 2}
        Your steed gains a number of hit points equal to half your maximum hit points.
    \subsubsection{Protection 1}
        While you are wearing armor, you gain a +1 bonus to AC.
    \subsubsection{Protection 2}
        You gain a bonus to all Strength, Dexterity, and Constitution saving throws equal to the AC granted by your shield.
    \subsubsection{Thrown Weapon Fighting 1}
        You can draw a weapon that has the thrown property as part of the attack you make with the weapon.
    \subsubsection{Thrown Weapon Fighting 2}
        When you hit with a ranged attack using a thrown weapon, you add a +2 bonus to the damage roll.
    \subsubsection{Two-Weapon Fighting 1}
        You suffer from the multiple attack penalty for each wielded weapon independently --- Attack rolls made with a weapon held in one hand don't impose a penalty on attack rolls made with a weapon held in your other hand.
    \subsubsection{Two-Weapon Fighting 2}
        When you engage in two-weapon fighting, you can add your ability modifier to the damage of the second attack.
    \subsubsection{Unarmed Fighting 1}
        Your unarmed strikes can deal bludgeoning damage equal to 1d6 + your Strength modifier on a hit.

        In addition, the multiple attack penalty is reduced by 1 when you make unarmed attacks.
    \subsubsection{Unarmed Fighting 2}
        If you aren't wielding any weapons or a shield, the damage die from your unarmed strikes becomes a d8.

        In addition, you can deal 1d4 bludgeoning damage to one creature grappled by you at the start of each of your turns.

% SPELLCASTING STYLES
\subsection*{Spellcasting Styles} \label{ssec::spellcastingstyles}
    \textbf{TODO.}

% MANEUVERS TODO: Dunno if we're using this?

\subsection*{Metamagic} \label{ssec::metamagic}
    \textbf{TODO.} % metamagic involves using one additional action on the spell to change some effect of it. Same as sorcerer metamagic, but unlimited use.

\subsection*{Artificer Infusions} \label{ssec::artificerinfusions}
    \subsubsection{Armor of Strength}
        \textit{Prerequisite: A suit of armor (requires attunement)}.

        This armor has 6 charges. The wearer can expend the armor's charges in the following ways:
        \begin{itemize}
            \item When the wearer makes a Strength check or a Strength saving throw, it can expend 1 charge to add a bonus to the roll equal to its Intelligence modifier.
            \item If the creature would be knocked prone, it can use its reaction to expend 1 charge to avoid being knocked prone.
        \end{itemize}
        The armor regains 1d6 expended charges daily at dawn.
    \subsubsection{Enhanced Focus}
        \textit{Prerequisite: An arcane focus (requires attunement)}.

        While holding this item, a creature gains a +1 bonus to spell attack rolls. In addition, the creature ignores half cover when making a spell attack.

        The bonus increases to +2 when you reach 10th level.
    \subsubsection{Enhanced Defense}
        \textit{Prerequisite: A suit of armor or a shield}.

        A creature gains a +1 bonus to Armor Class while wearing (armor) or wielding (shield) the infused item.

        The bonus increases to +2 when you reach 10th level.
    \subsubsection{Enhanced Weapon}
        \textit{Prerequisite: A simple or martial weapon}.

        This weapon grants a +1 bonus to attack and damage rolls made with it.

        The bonus increases to +2 when you reach 10th level.
    \subsubsection{Mind Sharpener}
        \textit{Prerequisite: A suit of armor or robes}.

        The infused item can send a jolt to the wearer to refocus their mind.
        The item has 4 charges.
        When the wearer fails a Constitution saving throw to maintain concentration on a spell, the wearer can use its reaction to expend 1 of the item's charges to succeed instead.
        The item regains 1d4 expended charges daily at dawn.
    \subsubsection{Repeating Shot}
        \textit{Prerequisite: A simple or martial weapon with the ammunition property (requires attunement)}.

        This weapon grants a +1 bonus to attack and damage rolls made with it when it's used to make a ranged attack, and it ignores the loading property if it has it.

        If you load no ammunition in the weapon, it produces its own, automatically creating one piece of ammunition when you make a ranged attack with it.
        The ammunition created by the weapon vanishes the instant after it hits or misses a target.
    \subsubsection{Returning Weapon}
        \textit{Prerequisite: A simple or martial weapon with the thrown property}.

        This weapon grants a +1 bonus to attack and damage rolls made with it, and it returns to the wielder's hand immediately after it is used to make a ranged attack.

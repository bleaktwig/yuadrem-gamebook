% !TEX root = ../main.tex
\addcontentsline{toc}{section}{Background Feats}
\subsection*{Background Feats}
\begin{DndTable}[width=\linewidth, header=Background Feat List 1/2]{ll}
    \textbf{Background} & \textbf{Feat} \\
    Acolyte & \textbf{Divine Healing} (page \pageref{feat::divinehealing})                 \\
    Acolyte & \textbf{Divine Inspiration} (page \pageref{feat::divineinspiration})         \\
    Acolyte & \textbf{Guidance} (page \pageref{feat::guidance})                            \\
    Acolyte & \textbf{Incite Respect} (page \pageref{feat::inciterespect})                 \\
    Acolyte & \textbf{Inspiring Leader} (page \pageref{feat::inspiringleader})             \\
    Acolyte & \textbf{Shelter of the Faithful} (page \pageref{feat::shelterofthefaithful}) \\

    Artisan & \textbf{Artificer's Infusions} (page \pageref{feat::artificersinfusion})         \\
    Artisan & \textbf{Beyond Mastery} (page \pageref{feat::beyondmastery})                     \\
    Artisan & \textbf{Flash of Genius} (page \pageref{feat::flashofgenius})                    \\
    Artisan & \textbf{Guild Membership} (page \pageref{feat::guildmembership})                 \\
    Artisan & \textbf{Known Crafter} (page \pageref{feat::knowncrafter})                       \\
    Artisan & \textbf{The Right Tool for the Job} (page \pageref{feat::therighttoolforthejob}) \\

    Criminal & \textbf{Cunning Action} (page \pageref{feat::cunningaction})             \\
    Criminal & \textbf{Criminal Contact} (page \pageref{feat::criminalcontact})         \\
    Criminal & \textbf{Down Low} (page \pageref{feat::downlow})                         \\
    Criminal & \textbf{Dual Personalities} (page \pageref{feat::dualpersonalities})     \\
    Criminal & \textbf{Undecity Paths} (page \pageref{feat::undercitypaths})            \\
    Criminal & \textbf{Urban Infrastructure} (page \pageref{feat::urbaninfrastructure}) \\

    Entertainer & \textbf{Bardic Inspiration} (page \pageref{feat::bardicinspiration}) \\
    Entertainer & \textbf{Courtier} (page \pageref{feat::courtier})                    \\
    Entertainer & \textbf{Jack of All Trades} (page \pageref{feat::jackofalltrades})   \\
    Entertainer & \textbf{Rakish Audacity} (page \pageref{feat::rakishaudacity})       \\
    Entertainer & \textbf{Rustic Hospitality} (page \pageref{feat::rustichospitality}) \\
    Entertainer & \textbf{Song of Rest} (page \pageref{feat::songofrest})              %\\
\end{DndTable}
\begin{DndTable}[width=\linewidth, header=Background Feat List 2/2]{ll}
    \textbf{Background} & \textbf{Feat} \\
    Laborer & \textbf{Commoner's Toughness} (page \pageref{feat::commonerstoughness})  \\

    Laborer & \textbf{Deep Miner} (page \pageref{feat::deepminer})                     \\
    Laborer & \textbf{Powerful Build} (page \pageref{feat::powerfulbuild_bg})             \\
    Laborer & \textbf{Resilient} (page \pageref{feat::resilient})                      \\
    Laborer & \textbf{Rustic Hospitality} (page \pageref{feat::rustichospitality})     \\
    Laborer & \textbf{Urban Infrastructure} (page \pageref{feat::urbaninfrastructure}) \\

    Merchant & \textbf{Down Low} (page \pageref{feat::downlow})                          \\
    Merchant & \textbf{Guild Membership} (page \pageref{feat::guildmembership})          \\
    Merchant & \textbf{Master Negotiator} (page \pageref{feat::masternegotiator})        \\
    Merchant & \textbf{Never Tell Met he Odds} (page \pageref{feat::nevertellmetheodds}) \\
    Merchant & \textbf{Silver Tongue} (page \pageref{feat::silvertongue})                \\
    Merchant & \textbf{Unsettling Words} (page \pageref{feat::unsettlingwords})          \\

    Noble & \textbf{Courtier} (page \pageref{feat::courtier})                 \\
    Noble & \textbf{Early Training} (page \pageref{feat::earlytraining})      \\
    Noble & \textbf{Kept in Style} (page \pageref{feat::keptinstyle})         \\
    Noble & \textbf{Legalese} (page \pageref{feat::legalese})                 \\
    Noble & \textbf{Promise of Status} (page \pageref{feat::promiseofstatus}) \\
    Noble & \textbf{Retainers} (page \pageref{feat::retainers})               \\

    Sailor & \textbf{Drunken Resilience} (page \pageref{feat::drunkenresilience}) \\
    Sailor & \textbf{Expert Stirrer} (page \pageref{feat::expertstirrer})         \\
    Sailor & \textbf{Harvest the Water} (page \pageref{feat::harvestthewater})    \\
    Sailor & \textbf{I'll Patch It} (page \pageref{feat::illpatchit})             \\
    Sailor & \textbf{Sailor's Balance} (page \pageref{feat::sailorsbalance})      \\
    Sailor & \textbf{Wisdom of the Wilds} (page \pageref{feat::wisdomofthewilds}) \\

    Scholar & \textbf{Adept Linguist} (page \pageref{feat::adeptlinguist})             \\
    Scholar & \textbf{Flash of Genius} (page \pageref{feat::flashofgenius})            \\
    Scholar & \textbf{Historical Knowledge} (page \pageref{feat::historicalknowledge}) \\
    Scholar & \textbf{Legalese} (page \pageref{feat::legalese})                        \\
    Scholar & \textbf{Metamagic Adept} (page \pageref{feat::metamagicadept})           \\
    Scholar & \textbf{Skill Versatility} (page \pageref{feat::skillversatility})       \\

    Soldier & \textbf{Inspiring Leader} (page \pageref{feat::inspiringleader})      \\
    Soldier & \textbf{Know your Enemy} (page \pageref{feat::knowyourenemy})         \\
    Soldier & \textbf{Martial Adept} (page \pageref{feat::martialadept})            \\
    Soldier & \textbf{Official Inquirt} (page \pageref{feat::officialinquiry})      \\
    Soldier & \textbf{Soldier's Fortitude} (page \pageref{feat::soldiersfortitude}) \\
    Soldier & \textbf{Steady} (page \pageref{feat::steady})                         \\

    Outlander & \textbf{All Eyes on You} (page \pageref{feat::alleyesonyou})         \\
    Outlander & \textbf{Land's Stride} (page \pageref{feat::landsstride})            \\
    Outlander & \textbf{Nature's Heritage} (page \pageref{feat::naturesheritage})    \\
    Outlander & \textbf{Roving} (page \pageref{feat::roving})                        \\
    Outlander & \textbf{Steady} (page \pageref{feat::steady})                        \\
    Outlander & \textbf{Wisdom of the Wilds} (page \pageref{feat::wisdomofthewilds}) %\\
\end{DndTable}

% A
\subsubsection{All Eyes on You} \label{feat::alleyesonyou}
    Your accent, mannerisms, figures of speech, and perhaps even your appearance all mark you as foreign.
    You gain the friendly interest of scholars and others intrigued by far-removed lands, to say nothing of everyday folk who are eager to hear stories of your origin.

    You can parley this attention into access to people and places you might not otherwise have, for you and your traveling companions.
    Noble lords, scholars, and merchant princes, to name a few, might be interested in hearing about your homeland.
    \paragraph{Requirements} Outlander background.
\subsubsection{Adept Linguist} \label{feat::adeptlinguist}
    You can communicate with people who don't speak any language you know.
    You must observe the people interacting with one another for at least 1 day, after which you learn a handful of important words, expressions, and gestures --- enough to communicate on a rudimentary level.
    \paragraph{Requirements} Scholar background.
\subsubsection{Artificer's Infusion} \label{feat::artificersinfusion}
    You gain the ability to imbue mundane items with infusions.

    When you gain this feature, pick four infusions to learn, choosing from the Infusions list in page \pageref{ssec::infusions}.
    The description of each of the following infusions details the type of item that can receive it, along with wether the resulting magic item requires attunement.
    Unless an infusion's description says otherwise, you can't learn an infusion more than once.

    Whenever you finish a long rest, you can touch a nonmagical object and imbue it with one of your artificer infusions.
    An infusion works on only certain kinds of objects, as specified in the infusion's description.
    If the item requires attunement, you can attune yourself to it the instant you infuse the item.
    If you decide to attune to the item later, you must do so using the normal process for attunement (see ``Attunement'' in chapter 7 of the Dungeon Master's Guide).

    Your infusion remains in an item indefinitely, but only one object can be infused at a time.
    If you try to exceed your maximum number of infusions, the oldest infusion immediately ends, and then the new infusion applies.

    You can learn this feat 3 times.
    The second and third time, you learn an additional infusion, and increase the number of objects that can be infused at the same time by one.

    \paragraph{Requirements} Artisan background or Skilled proficiency with tinker's tools.
% B
\subsubsection{Bardic Inspiration} \label{feat::bardicinspiration}
    You can inspire others through stirring words, actions, or music.
    To do so, you use an action on your turn to choose one creature other than yourself within 12 meters of you who can hear you.
    That creature gains one Bardic Inspiration die, a d6.

    Once within the next 10 minutes, the creature can roll the die and add the number rolled to one ability check, attack roll, or saving throw it makes.
    The creature can wait until after it rolls the d20 before deciding to use the Bardic Inspiration die, but must decide before the DM says whether the roll succeeds or fails.
    Once the Bardic Inspiration die is rolled, it is lost.
    A creature can have only one Bardic Inspiration die at a time.

    You can use this feature a number of times equal to your Charisma modifier (a minimum of once).
    You regain any expended uses when you finish a short rest.

    You can take this feat additional times to improve your Bardic Inspiration die.
    On a second time it becomes a d8, on a third a d10, and on a fourth a d12.

    \paragraph{Requirements} Entertainer background.
\subsubsection{Beyond Mastery (2 FP)} \label{feat::beyondmastery}
    Practicing for your whole life, you can take your skill as an artisan further than most mortals.
    You increase your proficiency with a set of artisan's tools from Expert to Legendary, increasing your proficiency bonus to +12.
    \paragraph{Requirements} Artisan background, Expert proficiency with a set of artisan's tools.
% C
\subsubsection{Commoner's Toughness (2 FP)} \label{feat::commonerstoughness}
    Your hit point maximum increases by an amount equal to your level when you take this feat.
    Whenever you gain a level thereafter, your hit point maximum increases by an additional hit point.
    \paragraph{Requirements} Laborer background.
\subsubsection{Courtier} \label{feat::courtier}
    Your knowledge of how bureaucracies function lets you gain access to the records and inner workings of any noble court or government you encounter.
    You know who the movers and shakers are, whom to go to for the favors you seek, and what the current intrigues of interest in the group are.
    When encountering a new court, you need to spend at least a short rest among their ranks to map their inner workings, which you may do by performing for them or using your status to gain an invitation to their dwellings.
    \paragraph{Requirements} Entertainer or Noble background.
\subsubsection{Criminal Contact} \label{feat::criminalcontact}
    Based on your connections, you can gain a criminal contact in a settlement as part of a long rest.
    This contact is reliable and trustworthy, and acts as your liaison to a network of other criminals.
    You know how to get messages to and from your contacts, even over great distances; specifically, you know the local messengers, corrupt caravan masters, and seedy sailors who can deliver messages for you.
    \paragraph{Requirements} Criminal background.
\subsubsection{Cunning Action} \label{feat::cunningaction}
    Your quick thinking and agility allow you to move and act quickly.
    Choose an action between Disengage, Hide, or Search.
    The chosen action only costs you one action to perform instead of two.

    You can take this feat up to three times, selecting a different action each time.
    \paragraph{Requirements} Criminal background.
% D
\subsubsection{Deep Miner} \label{feat::deepminer}
    You are used to navigating the deep places of the earth.
    You never get lost in caves or mines if you have either seen an accurate map of them or have been through them before.
    Furthermore, you are able to scrounge fresh water and food for yourself and as many as five other people each day if you are in a mine or natural caves.
    \paragraph{Requirements} Laborer background.
\subsubsection{Divine Healing (2 FP)} \label{feat::divinehealing}
    When you roll a 1 or 2 on a die related to healing or giving temporary hit points to one or more creatures, you can reroll the die and must use the new roll, even if the new is a 1 or a 2.
    Additionally, whenever you heal a creature (including yourself) you heal yourself by 1.
    \paragraph{Requirements} Acolyte background.
\subsubsection{Divine Inspiration} \label{feat::divineinspiration}
    As an action, you can touch your holy symbol, utter a prayer, and regain one expended spell slot, the level of which can be no higher than 1/5th your level, rounded up.
    You can use this feat once per short rest.
    \paragraph{Requirements} Acolyte background.
\subsubsection{Down Low} \label{feat::downlow}
    After spending a short rest in a settlement, you acquaint yourself with a network of smugglers who are willing to help you out of tight situations.
    While in a town or larger community, you and your companions can stay for free in safe houses.
    Safe houses provide a poor lifestyle.
    While staying at a safe house, you can choose to keep your presence (and that of your companions) a secret.
    \paragraph{Requirements} Criminal or Merchant background.
\subsubsection{Drunken Resilience (2 FP)} \label{feat::drunkenresilience}
    You can drink enough alcohol to knock an elephant.
    You have advantage on saving throws against poison, and you have resistance against poison damage.
    \paragraph{Requirements} Sailor background.
\subsubsection{Dual Personalities (2 FP)} \label{feat::dualpersonalities}
    By carefully managing your connections and your appearance, you have effectively created a second persona for yourself.
    Roll on the Faceless Persona table to determine your persona, or work with the DM to create a persona that's unique to your character and suits the tone of your game.

    \begin{DndTable}[width=\linewidth, header=Persona]{cX}
        \textbf{d10} & \textbf{Faceless Persona}                \\
        1  & A flamboyant spy or brigand.                       \\
        2  & The incarnation of a nation or people.             \\
        3  & A scoundrel with a masked guise.                   \\
        4  & A vengeful spirit.                                 \\
        5  & The manifestation of a deity of your faith.        \\
        6  & One whose beauty is greatly accented using makeup. \\
        7  & An impersonation of another hero.                  \\
        8  & The embodiment of a school of magic.               \\
        9  & A warrior with distinctive armor.                  \\
        10 & A disguise with animalistic or monstrous characteristics, meant to inspire fear.
    \end{DndTable}

    Most of the people who know you do so as your persona.
    Those who seek to learn more about you --- your weaknesses, your origins, your purpose --- find themselves stymied by your disguise.
    Upon donning a disguise and behaving as your persona, you are unidentifiable as your true self.
    By removing your disguise and revealing your true face, you are no longer identifiable as your persona.
    This allows you to change appearances between your two personalities as often as you wish, using one to hide the other or serve as convenient camouflage.
    However, should someone realize the connection between your persona and your true self, your deception might lose its effectiveness.
    \paragraph{Requirements} Criminal background.
% E
\subsubsection{Early Training} \label{feat::earlytraining}
    From an early age, your family procured only the best education for you.
    You learn one Fighting Style (see page \pageref{ssec::fightingstyles}) or one Casting Style (see page \pageref{ssec::castingstyle}) of your choice.
    If you already have a style, the one you choose must be different.
    \paragraph{Requirements} Noble background.
\subsubsection{Expert Stirrer} \label{feat::expertstirrer}
    Years of arguing and interacting at seas have made your insults extraordinarily effective.
    As an action, you can hurl a terrible insult to a creature within 12 meters of you who can hear you and understand your language.
    The creature must roll a DC 12 Wisdom saving throw.
    On a failure, the creature has disadvantage on attack rolls against targets other than you and can't make opportunity attacks against targets other than you.
    This effect lasts for 1 minute, until one of your companions attacks the target or affects it with a spell, or until you and the target are more than 12 meters apart.

    You can use this ability a number of times per short rest equal to your Charisma modifier (Minimum of one).

    You can learn this feat a total of three times, increasing the DC to 15 the second time and to 18 the third.
    \paragraph{Requirements} Sailor background.
% F
\subsubsection{Flash of Genius} \label{feat::flashofgenius}
    You gain the ability to come up with solutions under pressure.
    When you or another creature you can see within 6 meters of you makes an ability check or a saving throw, you can use an action or your reaction to add your Intelligence modifier to the roll.

    You can use this feature a number of times equal to your Intelligence modifier (minimum of once).
    You regain all expended uses when you finish a short rest.
    \paragraph{Requirements} Artisan or Scholar background.
% G
\subsubsection{Guidance} \label{feat::guidance}
    As an action, you give words of encouragement to one willing creature.
    Once during the next minute, the target can roll a d4 and add the number rolled to one ability check of its choice.
    It can roll the die before or after making the ability check.

    Only one creature can be affected by this ability at a time.
    \paragraph{Requirements} Acolyte background
\subsubsection{Guild Membership} \label{feat::guildmembership}
    As an known figure of your trade, you can become a member of a guild associated to it.
    This membership provides you with certain benefits.
    Your fellow guild members will provide you with lodging and food if necessary, and pay for your funeral if needed.
    In some cities and towns, a guildhall offers a central place to meet other members of your profession, which can be a good place to meet potential patrons, allies, or hirelings.

    Guilds often wield tremendous political power.
    If you are accused of a crime, your guild will support you if a good case can be made for your innocence or the crime is justifiable.
    You can also gain access to powerful political figures through the guild, if you are a member in good standing.
    Such connections might require the donation to the guild's coffers.

    To maintain your membership, you must pay dues of 5 agnomas per month to the guild.
    If you miss payments, you must make up back dues to remain in the guild's good graces.
    \paragraph{Requirements} Artisan or Merchant background.
% H
\subsubsection{Harvest the Water} \label{feat::harvestthewater}
    You gain advantage on ability checks made using fishing tackle.
    If you have access to a body of water that sustains marine life, you can maintain a moderate lifestyle while working as a fisher, and you can catch enough food to feed yourself and up to ten other people each day.
    \paragraph{Requirements} Sailor background.
\subsubsection{Historical Knowledge} \label{feat::historicalknowledge}
    When you enter a ruin or dungeon, you can correctly ascertain its original purpose and determine its builders, whether those were gats, oths, ets, thri-kreen, or some other known kin.
    In addition, you can determine the monetary value of art objects more than a century old.
    \paragraph{Requirements} Scholar background.
% I
\subsubsection{I'll Patch It!} \label{feat::illpatchit}
    Provided you have carpenter's tools and wood, you can perform repairs on a water vehicle.
    You don't need to be competent with carpenter's tools to gain this feat.
    When you use this ability, you restore a number of hit points to the hull of a water vehicle equal to twice your level.
    A vehicle cannot be patched by you in this way again until you finish a short rest.
    \paragraph{Requirements} Sailor background.
\subsubsection{Incite Respect} \label{feat::inciterespect}
    Using two actions, you present your holy symbol, and each creature of your choice that can see or hear you within 6 meters of you must succeed on a Wisdom saving throw of DC 12 or be charmed by you until the end of your next turn or until the charmed creature takes any damage.
    You can also cause any of the charmed creatures to drop what they are holding when they fail the saving throw.

    You can use this ability a number of times per short rest equal to your Wisdom modifier (Minimum of one).

    You can learn this feat a total of three times, increasing the DC to 15 the second time and to 18 the third.
    \paragraph{Requirements} Acolyte background
\subsubsection{Inpiring Leader} \label{feat::inspiringleader}
    You can spend 10 minutes inspiring your companions, shoring up their resolve to fight.
    When you do so, choose up to six friendly creatures (which can include yourself) within 6 meters of you who can see or hear you and who can understand you.
    Each creature can gain temporary hit points equal to your level + your Charisma modifier.
    A creature can't gain temporary hit points from this feat again until it has finished a short rest.
    \paragraph{Requirements} Acolyte or Soldier background.
% \subsubsection{Inheritance} \label{feat::inheritance}
%     Choose or randomly determine your inheritance from the possibilities in the table below.
%     Work with your Dungeon Master to come up with details: Why is your inheritance so important, and what is its full story?
%     You might prefer for the DM to invent these details as part of the game, allowing you to learn more about your inheritance as your character does.
%
%     The Dungeon Master is free to use your inheritance as a story hook, sending you on quests to learn more about its history or true nature, or confronting you with foes who want to claim it for themselves or prevent you from learning what you seek.
%     The DM also determines the properties of your inheritance and how they figure into the item's history and importance.
%     For instance, the object might be a minor magic item, or one that begins with a modest ability and increases in potency with the passage of time.
%     Or, the true nature of your inheritance might not be apparent at first and is revealed only when certain conditions are met.
%
%     You can decide whether or when to tell your companions about your inheritance.
%     Rather than attracting attention to yourself, you might want to keep your inheritance a secret until you learn more about what it means to you and what it can do for you.
%
%     \begin{DndTable}[width=\linewidth, header=Inheritance]{cX}
%         \textbf{d8} & \textbf{Object or item}                                  \\
%         1   & A document such as a map, a letter, or a journal                 \\
%         2-3 & a trinket (see "Trinkets" in chapter 5 of the Player's Handbook) \\
%         4   & an article of clothing                                           \\
%         5   & a piece of jewelry                                               \\
%         6   & an arcane book or formulary                                      \\
%         7   & a written story, song, poem, or secret                           \\
%         8   & a tattoo or other body marking
%     \end{DndTable}
%     \paragraph{Requirements} Noble background.
% J
\subsubsection{Jack of All Trades (2 FP)} \label{feat::jackofalltrades}
    You can add a bonus equal to 1/10th of your level, rounded up, to any ability check with which you are not already proficient.
    \paragraph{Requirements} Entertainer background.
% K
\subsubsection{Kept in Style (2 FP)} \label{feat::keptinstyle}
    Your family is known across the realm, and your attitude reflects this.
    Your name and signet are sufficient to cover most of your expenses, since everyone would prefer to be in your family's good graces.

    This advantage enables you and your companions to receive most services for free, as long as the cost doesn't rise above 10 agnomas.
    For services up to 50 agnomas, roll a Charisma (Persuasion) check contested by the target's Wisdom (Insight).
    On a success, you can enjoy the service for free.

    At the DM discretion, this feat may be used for more expensive or rarer services, perhaps with disadvantage on the roll or by promising a reward to the service provider.
    \paragraph{Requirements} Noble background.
\subsubsection{Know your Enemy} \label{feat::knowyourenemy}
    If you spend at least 1 minute observing or interacting with another creature outside combat, you can learn certain information about its capabilities compared to your own.
    The DM tells you if the creature is your equal, superior, or inferior in regard to two of the following characteristics of your choice:
    \begin{itemize}
        \item Strength score
        \item Dexterity score
        \item Constitution score
        \item Armor Class
        \item Current hit points
        \item Total levels
    \end{itemize}
    \paragraph{Requirements} Soldier background.
\subsubsection{Known Crafter} \label{feat::knowncrafter}
    Knowing the local trade like the back of your hand, you know where to find the best ingredients for the cheapest prices in a settlement in which you are familiar.
    You can buy components and ingredients related to your craft for half their normal price, and you can tell the quality of a raw material just by looking at it.
    To use this feat you must first gain familiarity with a settlement as part of a long rest.
    \paragraph{Requirements} Artisan background.
% L
\subsubsection{Land's Stride (2 FP)} \label{feat::landsstride}
    Moving through difficult terrain costs you no extra movement.
    You can also pass through plants without being slowed by them and without taking damage from them if they have thorns, spines, or a similar hazard.

    In addition, you have advantage on saving throws against plants that are magically created or manipulated to impede movement, such those created by the entangle spell.
    \paragraph{Requirements} Outlander background.
\subsubsection{Legalese} \label{feat::legalese}
    Your experience with your local legal system has given you a firm knowledge of the ins and outs of that system.
    Even when the law is not on your side, you can use complex terms like ``ex injuria jus non oritur'' and ``cogitationis poenam nemo patitur'' to frighten people into thinking you know what you're talking about.
    With common folks who don't know any better, you might be able to intimidate or deceive to get favors or special treatment.
    \paragraph{Requirements} Noble or Scholar background.
% \subsubsection{Library Access} \label{feat::libraryaccess}
%     You are a well-known scholar (or good enough at pretending to be one), and have free and easy access to the majority of libraries.
%     This doesn't give you access to repositories of lore that are too valuable or secret to permit anyone immediate access.
%
%     Additionally, you are likely to gain preferential treatment at libraries and universities accross Yuadrem, as professional courtesy to a fellow scholar.
%     \paragraph{Requirements} Scholar background.
% M
\subsubsection{Martial Adept} \label{feat::martialadept}
    You have martial training that allows you to perform special combat maneuvers.
    You learn one maneuver of your choice from among those available in the Combat Maneuvers section (see page \pageref{ssec::combatmaneuvers}).
    If a maneuver you use requires your target to make a saving throw to resist the maneuver's effects, the saving throw DC equals 8 + twice your Strength or Dexterity modifier (your choice).

    % You gain two maneuver points, which are added to any maneuver points you have from other sources.
    % These are used to fuel maneuvers, and are expended when you use one.
    % You regain your expended maneuver points when you finish a short rest.

    You can take this feat 3 times, learning a new maneuver the second and third time.
    \paragraph{Requirements} Soldier background.
\subsubsection{Master Negotiator (2 FP)} \label{feat::masternegotiator}
    An expert trader, haggling is your second nature.
    When you interact with a fellow merchant, you can roll a Charisma (Persuasion) check contested by the merchant's Charisma (Persuasion).
    If you succeed, the price for any item you buy from them is reduced by 25\%.
    \paragraph{Requirements} Merchant background.
% \subsubsection{Mercenary Life} \label{feat::mercenarylife}
%     You know local mercenaries as only someone who has worked with them can.
%     You are able to identify mercenary companies by their emblems, and you know a little about any such company, including who has hired them recently.
%     You can find the taverns and festhalls where mercenaries abide in any area, as long as you speak the language.
%     You can find mercenary work between adventures sufficient to maintain a comfortable lifestyle.
%     \paragraph{Requirements} Soldier background.
\subsubsection{Metamagic Adept} \label{feat::metamagicadept}
    By understanding the relation between a variety of doctrines, you've learned how to exert your will on your spells to alter how they function.
    You learn one metamagic option of your choice (see page \pageref{ssec::metamagic}).
    You can use only one metamagic option on a spell when you cast it, unless the option says otherwise.
    % You gain 2 metamagic points to spend on Metamagic (these points are added to any metamagic points you have from another source but can be used only on Metamagic).
    % You regain all spent metamagic points when you finish a short rest.

    You can take this feat 3 times, learning a new metamagic option the second and third time.
    \paragraph{Requirements} Scholar background and Spellcasting feat.
% N
\subsubsection{Nature's Heritage} \label{feat::naturesheritage}
    You are familiar enough with any wilderness area that you can find twice as much food and water as you normally would when you forage.
    \paragraph{Requirements} Outlander background.
\subsubsection{Never Tell Me the Odds} \label{feat::nevertellmetheodds}
    Odds and probability are your bread and butter.
    During downtime activities that involve games of chance or figuring odds on the best plan, you can get a solid sense of which choice is likely the best one and which opportunities seem too good to be true, at the DM's determination.
    \paragraph{Requirements} Merchant background.
% O
\subsubsection{Official Inquiry} \label{feat::officialinquiry}
    You're experienced at gaining access to people and places to get the information you need.
    Through a combination of fast-talking, determination, and official-looking documentation, you can gain access to a place or an individual related to a something you're investigating.
    Those who aren't involved in your investigation avoid impeding you or pass along your requests.
    \paragraph{Requirements} Soldier background.
% P
\subsubsection{Powerful Build} \label{feat::powerfulbuild_bg}
    Hardened by work, you count as one size larger when determining your carrying capacity and the weight you can push, drag, or lift.
    \paragraph{Requirements} Laborer background.
\subsubsection{Promise of Status} \label{feat::promiseofstatus}
    Using two actions, you shout your family name to inspire fear from your enemies.
    Each creature that can hear and understand you within 6 meters of you must succeed on a DC 12 Charisma saving throw.
    On a failure, the creature is frightened of you until the end of your next turn.
    At the DM's discretion, a creature might alternatively cease fighting altogether or even join your cause, expecting retribution from your family afterwards.

    You can use this ability a number of times per short rest equal to your Charisma modifier (Minimum of one).

    You can learn this feat a total of three times, increasing the DC to 15 the second time and to 18 the third.
    \paragraph{Requirements} Noble background
% Q
% R
\subsubsection{Rakish Audacity} \label{feat::rakishaudacity}
    Your confidence propels you into battle.
    You can give yourself a bonus to your initiative rolls equal to your Charisma modifier.
    \paragraph{Requirements} Entertainer background.
\subsubsection{Resilient} \label{feat::resilient}
    Stout beyond belief, you are living proof of the hardiness of the common folk.
    Choose an ability score.
    You gain competence in saving throws using the chosen ability.

    You can take this feat 3 times, improving your proficiency level with the saving throws of the chosen ability each time.
    \paragraph{Requirements} Laborer background.
\subsubsection{Retainers} \label{feat::retainers}
    You gain the service of three retainers loyal to your family.
    These retainers can be attendants or messengers, and one might be a majordomo.
    One of your retainers can serve as your squire, aiding you in exchange for training on their own path to knighthood.
    Another retainer might be a groom to care for your horse or a servant who polishes your armor and helps you put it on.

    Your retainers can perform mundane tasks for you, but they do not fight for you, follow you into obviously dangerous areas (such as dungeons).
    Your retainers will leave if they are frequently endangered or abused, and it will take your family 1d6 weeks to send you a new group.
    \paragraph{Requirements} Noble background.
\subsubsection{Roving} \label{feat::roving}
    A deft explorer, your movement is unimpeded by water or mountain.
    You can take this feat three times, gaining different abilities each time:
    \begin{itemize}
        \item The first time, you gain a swimming speed equal to your walking speed.
        \item The second time, you can a climbing speed equal to your walking speed.
        \item The third time, your walking speed is increased by 1 meter.
    \end{itemize}
    After gaining the first or second abilities from this feat, Any effect that increases your movement speed also increases your swimming and climbing speed by the same amount.
    \paragraph{Requirements} Outlander background.
\subsubsection{Rustic Hospitality} \label{feat::rustichospitality}
    You've spent your life in company of the masses, and the common folk thank you for it.
    You can find a place to hide, rest, or recuperate among other commoners, unless you have shown yourself to be a danger to them.
    They will shield you from the law or anyone else searching for you, though they will not risk their lives for you.
    For this feat to work, you need to have worked or performed in the settlement at least once.
    \paragraph{Requirements} Entertainer or Laborer background.
% S
\subsubsection{Sailor's Balance} \label{feat::sailorsbalance}
    Used to the ever-present swing of a ship, you have advantage on any ability checks and saving throws made to avoid being moved or being knocked prone.
    \paragraph{Requirements} Sailor background.
\subsubsection{Shelter of the Faithful} \label{feat::shelterofthefaithful}
    Your piety inspires the respect of those who share your faith.
    After performing a religious ceremony of your deity, you and your companions can expect to receive free healing and care at a temple, shrine, or other established presence of your faith, though you must provide any material components needed for spells.
    Those who share your religion will support you (but only you) at a modest lifestyle.

    Additionally, your devotion might rouse members from other religions to look kindly to you.
    After performing a religious ceremony or act of kindness, you can roll an Intelligence (Religion) check contested by the creature's Intelligence (Religion).
    On a success, you gain the benefits from this feat from the creature's creed.
    This check is made with advantage if yours and the target deity share tide, and automatically fails if they are enemy deities from the same pantheon.

    % You might also have ties to a specific temple dedicated to your chosen deity or pantheon, and you have a residence there.
    % This could be the temple where you used to serve, if you remain on good terms with it, or a temple where you have found a new home.
    % While near your temple, you can call upon the priests for assistance, provided the assistance you ask for is not hazardous and you remain in good standing with your temple.
    \paragraph{Requirements} Acolyte background.
\subsubsection{Silver Tongue} \label{feat::silvertongue}
    You are a master at saying the right thing at the right time.
    When you make a Charisma (Persuasion) or Charisma (Deception) check, you can treat a d20 roll of 9 or lower as a 10.
    \paragraph{Requirements} Merchant background.
\subsubsection{Skill Versatility (2 FP)} \label{feat::skillversatility}
    Your life of studies has given you the capacity to quickly learn new subjects.
    You gain two proficiency levels on a skill of your choice.
    This skill must be related to Intelligence or Wisdom.

    After a long rest, you can chance this proficiency to another skill, also related to Intelligence or Wisdom.
    \paragraph{Requirements} Scholar background.
\subsubsection{Soldier's Fortitude (2 FP)} \label{feat::soldiersfortitude}
    Whenever you take the Dodge action in combat, you can spend one hit die to heal yourself.
    Roll the die, add your Constitution modifier, and regain a number of hit points equal to the total (minimum of 1).
    \paragraph{Requirements} Soldier background.
\subsubsection{Song of Rest} \label{feat::songofrest}
    You can use soothing music, oration, or a performance to help revitalize your wounded allies during a short rest.
    If you or any friendly creatures who can see or hear your performance regain hit points by spending Hit Dice at the end of the short rest, each of those creatures regains an extra 1d6 hit points.

    The extra hit points increase when you reach certain levels: to 1d8 at 6th level, to 1d10 at 11th level, and to 1d12 at 16th level.
    \paragraph{Requirements} Entertainer background.
\subsubsection{Steady} \label{feat::steady}
    You can move twice the normal amount of time (up to 16 hours) each day before being subject to the effect of a forced march (see ``Travel Pace'' in chapter 8 of the Player's Handbook).
    \paragraph{Requirements} Outlander or Soldier background.
% T
\subsubsection{The Right Tool for the Job} \label{feat::therighttoolforthejob}
    In times of need, your vast experience allows you to improvise solutions, adapt to any adversity, and overcome insurmountable challenges.
    You learn how to produce exactly the tool you need.
    By gathering resources from any environment, you can create one set of artisan's tools with which you are already competent with.
    This creation requires 1 hour of uninterrupted work, which can coincide with a short or long rest.
    Tools conjured via this feat are flimsy at best, and no reasonable person would buy them.
    \paragraph{Requirements} Artisan background.
% U
\subsubsection{Undercity Paths} \label{feat::undercitypaths}
    You know hidden, underground pathways that you can use to bypass crowds, obstacles, and observation as you move through the city.
    When you aren't in combat, you and companions you lead can travel between any two locations in the city twice as fast as your speed would normally allow.
    The paths of the undercity are haunted by dangers that rarely brave the light of the surface world, so your journey isn't guaranteed to be safe.
    \paragraph{Requirements} Criminal background.
\subsubsection{Unsettling Words} \label{feat::unsettlingwords}
    You can spin words laced with malice that unsettle a creature and cause it to doubt itself.
    Using one action, you can choose one creature you can see within 12 meters of you.
    Roll a d6.
    The creature must subtract the number rolled from the next saving throw it makes before the start of your next turn.
    A creature can only be affected by this once per round.

    You can use this feature a number of times equal to your Charisma modifier (a minimum of once).
    You regain any expended uses when you finish a short rest.

    You can take this feat additional times to improve the die.
    On a second time it becomes a d8, on a third a d10, and on a fourth a d12.
    \paragraph{Requirements} Merchant background.
\subsubsection{Urban Infrastructure} \label{feat::urbaninfrastructure}
    You have a basic knowledge of the structure of buildings, including the stuff behind the walls.
    You can also find blueprints of a specific building in order to learn the details of its construction.
    Such blueprints might provide knowledge of entry points, structural weaknesses, or secret spaces.
    Your access to such information isn't unlimited, and obtaining it can sometimes get you in trouble with the law.
    \paragraph{Requirements} Criminal or Laborer background.
% V
% W
% \subsubsection{Watcher's Eye} \label{feat::watcherseye}
%     Your experience in enforcing the law, and dealing with lawbreakers, gives you a feel for local laws and criminals.
%     You can easily find the local outpost of the watch or a similar organization, and just as easily pick out the dens of criminal activity in a community, although you're more likely to be welcome in the former locations rather than the latter.
%     \paragraph{Requirements} Soldier background.
\subsubsection{Wisdom of the Wilds} \label{feat::wisdomofthewilds}
    Your extended travels have given you a special appreciation for the stillness of the world.
    After finishing a short rest, you gain temporary hit points equal to your level + your Wisdom modifier, and you have advantage on the first ability check you make during the day.
    \paragraph{Requirements} Sailor or Outlander background.
% X
% Y
% Z

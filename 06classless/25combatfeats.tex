% !TEX root = ../main.tex
\addcontentsline{toc}{section}{Combat Feats}
\subsection*{Combat Feats}
TODO: DESCRIPTION

% === ARMOR TYPES ==================================================================================
% UNARMORED
\subsubsection{NAME} \label{feat::name}
    DESCRIPTION

    To gain the benefits of this feat, you must not be wearing armor.
\subsubsection{NAME} \label{feat::name}
    DESCRIPTION

    To gain the benefits of this feat, you must not be wearing armor.
\subsubsection{NAME} \label{feat::name}
    DESCRIPTION - UPGRADABLE

    To gain the benefits of this feat, you must not be wearing armor.
\subsubsection{NAME (2 FP)} \label{feat::name}
    DESCRIPTION

    To gain the benefits of this feat, you must not be wearing armor.
% LIGHT
\subsubsection{Lightly Armored} \label{feat::lightlyarmored}
    You learn how to wear light armor in combat.
\subsubsection{Light on your Feet} \label{feat::lightonyourfeet}
    Equipped light armor and cloth accessories don't add to your encumbrance.
    \paragraph{Requirements} Proficiency with Light Armor.
\subsubsection{NAME} \label{feat::name}
    DESCRIPTION

    To gain the benefits of this feat, you must be wearing light armor.
    \paragraph{Requirements} Proficiency with Light Armor.
\subsubsection{NAME} \label{feat::name}
    DESCRIPTION - UPGRADABLE

    To gain the benefits of this feat, you must be wearing light armor.
    \paragraph{Requirements} Proficiency with Light Armor.
\subsubsection{Light Armor Master (2 FP)} \label{feat::lightarmormaster}
    If you aren't incapacitated, you can use your reaction to gain advantage on any Dexterity saving throw you make against a spell or other harmful effect that targets only you.
    In addition, you can use the Dodge action even if your speed is 0.

    To gain the benefits of this feat, you must be wearing light armor.
    \paragraph{Requirements} Proficiency with Light Armor.
% MEDIUM
\subsubsection{Moderately Armored} \label{feat::moderatelyarmored}
    You learn how to wear medium armor in combat.
\subsubsection{Silent Despite Everything} \label{feat::silentdespiteeverything}
    Wearing medium armor doesn't impose disadvantage on your Dexterity (Stealth) checks.
    \paragraph{Requirements} Proficiency with Medium Armor.
\subsubsection{Agile in a Suit} \label{feat::agileinasuit}
    When you wear medium armor, you can add 3, rather than 2, to your AC if you have a Dexterity of 16 or higher.
    \paragraph{Requirements} Proficiency with Medium Armor.
\subsubsection{NAME} \label{feat::name}
    While wearing medium armor, you may add a +1 to Strength, Dexterity, and Constitution saving throws in which you don't have any proficiency level.

    You can take this feat two additional times, increasing this bonus by +1 each time.
    \paragraph{Requirements} Proficiency with Medium Armor.
\subsubsection{NAME (2 FP)} \label{feat::name}
    DESCRIPTION

    To gain the benefits of this feat, you must be wearing medium armor.
    \paragraph{Requirements} Proficiency with Medium Armor.
% HEAVY
\subsubsection{Heavily Armored} \label{feat::heavilyarmored}
    You learn how to wear heavy armor in combat.
\subsubsection{NAME} \label{feat::name}
    DESCRIPTION

    To gain the benefits of this feat, you must be wearing heavy armor.
    \paragraph{Requirements} Proficiency with Heavy Armor.
\subsubsection{Imposing Figure} \label{feat::imposingfigure}
    DESCRIPTION

    To gain the benefits of this feat, you must be wearing heavy armor.
    \paragraph{Requirements} Proficiency with Heavy Armor.
\subsubsection{NAME} \label{feat::name}
    DESCRIPTION - UPGRADABLE

    To gain the benefits of this feat, you must be wearing heavy armor.
    \paragraph{Requirements} Proficiency with Heavy Armor.
\subsubsection{NAME (2 FP)} \label{feat::name}
    DESCRIPTION

    To gain the benefits of this feat, you must be wearing heavy armor.
    \paragraph{Requirements} Proficiency with Heavy Armor.

% === ARMOR PROPERTIES =============================================================================
% BULWARK
\subsubsection{NAME} \label{feat::name}
    DESCRIPTION

    To gain the benefits of this feat, you must be wearing armor with the Bulwark property.
\subsubsection{NAME} \label{feat::name}
    DESCRIPTION

    To gain the benefits of this feat, you must be wearing armor with the Bulwark property.
\subsubsection{NAME (2 FP)} \label{feat::name}
    DESCRIPTION

    To gain the benefits of this feat, you must be wearing armor with the Bulwark property.
% CHAIN
\subsubsection{NAME} \label{feat::name}
    DESCRIPTION

    To gain the benefits of this feat, you must be wearing armor with the Chain property.
\subsubsection{NAME} \label{feat::name}
    DESCRIPTION

    To gain the benefits of this feat, you must be wearing armor with the Chain property.
\subsubsection{NAME (2 FP)} \label{feat::name}
    DESCRIPTION

    To gain the benefits of this feat, you must be wearing armor with the Chain property.
    % use noise of chains to distract nearby creatures?
% CLOTH
\subsubsection{NAME} \label{feat::name}
    DESCRIPTION

    To gain the benefits of this feat, you must be wearing armor with the Cloth property.
\subsubsection{NAME} \label{feat::name}
    DESCRIPTION

    To gain the benefits of this feat, you must be wearing armor with the Cloth property.
\subsubsection{NAME (2 FP)} \label{feat::name}
    DESCRIPTION

    To gain the benefits of this feat, you must be wearing armor with the Cloth property.
% COMPOSITE
\subsubsection{NAME} \label{feat::name}
    DESCRIPTION

    To gain the benefits of this feat, you must be wearing armor with the Composite property.
\subsubsection{NAME} \label{feat::name}
    DESCRIPTION

    To gain the benefits of this feat, you must be wearing armor with the Composite property.
\subsubsection{NAME (2 FP)} \label{feat::name}
    DESCRIPTION

    To gain the benefits of this feat, you must be wearing armor with the Composite property.
% LEATHER
\subsubsection{NAME} \label{feat::name}
    DESCRIPTION

    To gain the benefits of this feat, you must be wearing armor with the Leather property.
\subsubsection{NAME} \label{feat::name}
    DESCRIPTION

    To gain the benefits of this feat, you must be wearing armor with the Leather property.
\subsubsection{NAME (2 FP)} \label{feat::name}
    DESCRIPTION

    To gain the benefits of this feat, you must be wearing armor with the Leather property.
% PLATE
\subsubsection{NAME} \label{feat::name}
    DESCRIPTION

    To gain the benefits of this feat, you must be wearing armor with the Plate property.
\subsubsection{NAME} \label{feat::name}
    DESCRIPTION

    To gain the benefits of this feat, you must be wearing armor with the Plate property.
\subsubsection{NAME (2 FP)} \label{feat::name}
    DESCRIPTION

    To gain the benefits of this feat, you must be wearing armor with the Plate property.

% === SHIELD TYPES =================================================================================
% SMALL SHIELDS
\subsubsection{Buckler Training} \label{feat::bucklertraining}
    You learn how to use small shields in combat.
\subsubsection{Quick Shielding} \label{feat::quickshielding}
    You can don or doff a light shield as a free object interaction.
    \paragraph{Requirements} Proficiency with Small Shields.
\subsubsection{Swift Deflection} \label{feat::swiftdeflection}
    You learn the Parry reaction (see page \pageref{act::parry}).
    When you use this reaction while holding a small shield, increase the bonus to your AC from a d6 to a d8.
    \paragraph{Requirements} Proficiency with Small Shields.
\subsubsection{NAME} \label{feat::name}
    DESCRIPTION - UPGRADABLE

    To gain the benefits of this feat, you must be wielding a small shield.
    \paragraph{Requirements} Proficiency with Small Shields.
\subsubsection{Strapped and Secured (2 FP)} \label{feat::strappedandsecured}
    By attaching a leather strap to a light shield, you can carry it without using your off-hand.
    Due to the shield you still cannot carry a weapon in that hand, but your hand is free to use items, hold spellcasting focus, reload a crossbow, etc.

    In addition, your shield cannot be removed by the Disarm action.
    \paragraph{Requirements} Proficiency with Small Shields.
% MEDIUM SHIELDS
\subsubsection{Shield Training} \label{feat::shieldtraining}
    Always balanced in combat, you learn how to use medium shields in combat.
\subsubsection{NAME} \label{feat::name}
    DESCRIPTION

    To gain the benefits of this feat, you must be wielding a medium shield.
    \paragraph{Requirements} Proficiency with Medium Shields.
\subsubsection{The Best Defense...} \label{feat::thebestdefense}
    You learn the Bash action (see page \pageref{act::bash}).
    In addition, if you hit a creature with this action using a medium shield, your next melee attack on this turn to the it is made with advantage.
    \paragraph{Requirements} Proficiency with Medium Shields.
\subsubsection{NAME} \label{feat::name}
    DESCRIPTION - UPGRADABLE

    To gain the benefits of this feat, you must be wielding a medium shield.
    \paragraph{Requirements} Proficiency with Medium Shields.
\subsubsection{NAME (2 FP)} \label{feat::name}
    DESCRIPTION

    To gain the benefits of this feat, you must be wielding a medium shield.
    \paragraph{Requirements} Proficiency with Medium Shields.
% HEAVY SHIELDS
\subsubsection{Heavy Shields Training} \label{feat::heavyshieldtraining}
    Prioritizing a solid defense above everything, you learn how to use heavy shields in combat.
\subsubsection{NAME} \label{feat::name}
    DESCRIPTION

    To gain the benefits of this feat, you must be wielding a heavy shield.
    \paragraph{Requirements} Proficiency with Heavy Shields.
\subsubsection{NAME} \label{feat::name}
    DESCRIPTION

    To gain the benefits of this feat, you must be wielding a heavy shield.
    \paragraph{Requirements} Proficiency with Heavy Shields.
\subsubsection{NAME} \label{feat::name}
    DESCRIPTION - UPGRADABLE

    To gain the benefits of this feat, you must be wielding a heavy shield.
    \paragraph{Requirements} Proficiency with Heavy Shields.
\subsubsection{NAME (2 FP)} \label{feat::name}
    DESCRIPTION

    To gain the benefits of this feat, you must be wielding a heavy shield.
    \paragraph{Requirements} Proficiency with Heavy Shields.

% === SIMPLE WEAPONS ===============================================================================
% SIMPLE WEAPONS
\subsubsection{Simple Weapons Proficiency} \label{feat::name}
    No one learned how to use a sword without first swinging a stick.
\subsubsection{NAME} \label{feat::name}
    DESCRIPTION
    \paragraph{Requirements} Competent proficiency with Simple Weapons.
\subsubsection{NAME} \label{feat::name}
    DESCRIPTION
    \paragraph{Requirements} Competent proficiency with Simple Weapons.
\subsubsection{Adaptable Fighter} \label{feat::adaptablefighter}
    You learn one action of your choice between % Bonk, Block, Buttstroke, Feint, Lunge, Parry, Pushing Attack, Protect, Quick Draw, Riposte, Rush, Sweep, and Trip.
    \paragraph{Requirements} Skilled proficiency with Simple Weapons.
\subsubsection{NAME (2 FP)} \label{feat::name}
    DESCRIPTION
    \paragraph{Requirements} Skilled proficiency with Simple Weapons.
\subsubsection{NAME} \label{feat::name}
    DESCRIPTION
    \paragraph{Requirements} Expert proficiency with Simple Weapons.
\subsubsection{NAME (2 FP)} \label{feat::name}
    DESCRIPTION
    \paragraph{Requirements} Expert proficiency with Simple Weapons.

% === MARTIAL WEAPONS ==============================================================================
% AXES
\subsubsection{Axemaster} \label{feat::axemaster}
    More than a simple tool, you master the use of axes in combat.

    You increase your proficiency level with axes.
    This feat can be taken three times, or until you reach Expert proficiency with the weapon type.
\subsubsection{Griefer} \label{feat::griefer}
    While wielding an axe, you deal double damage to structures and items.
    In addition, you can cut a rope as one action using your trusty axe.
    \paragraph{Requirements} Competent proficiency with Axes.
\subsubsection{Tree Feller} \label{feat::treefeller}
    Nature is your foe, and you use your axe to fell both trees and enemies.
    You have advantage on attack rolls against any creature made of wood, regardless their nature.
    \paragraph{Requirements} Competent proficiency with Axes.
\subsubsection{Forceful Disarm} \label{feat::forcefuldisarm}
    You reduce the action cost of the Disarm action to one action.
    If you are holding an axe, you have advantage on your attack roll with the Disarm action to force a creature to drop its shield.
    \paragraph{Requirements} Skilled proficiency with Axes.
\subsubsection{NAME (2 FP)} \label{feat::name}
    DESCRIPTION
    \paragraph{Requirements} Skilled proficiency with Axes.
\subsubsection{Rush} \label{feat::rush}
    As part of your attack, you can move up to half your movement speed towards the target of your attack.
    You can use this ability only once per turn.
    \paragraph{Requirements} Expert proficiency with Axes.
\subsubsection{NAME (2 FP)} \label{feat::name}
    DESCRIPTION
    \paragraph{Requirements} Expert proficiency with Axes.
% BOWS
\subsubsection{Bowmaster} \label{feat::bowmaster}
    You increase your proficiency level with bows.
    This feat can be taken three times, or until you reach Expert proficiency with the weapon type.
\subsubsection{Sudden Attack} \label{feat::suddenattack}
    You can use your bow to make melee weapon attacks.
    Melee attacks with your bow deal 1d4 bludgeoning damage plus your Strength modifier.

    When you hit with a melee attack with your bow, the attacked creature can't make opportunity attacks against you until the start of your next turn.
    \paragraph{Requirements} Competent proficiency with Bows.
\subsubsection{Covered Shooter} \label{feat::coveredshooter}
    If you miss with a ranged attack with a bow or thrown weapon while hidden, the attack doesn't reveal your location, and you remain hidden.
    \paragraph{Requirements} Competent proficiency with Bows.
\subsubsection{NAME - ACTION} \label{feat::name}
    DESCRIPTION
    \paragraph{Requirements} Skilled proficiency with Bows.
\subsubsection{NAME (2 FP)} \label{feat::name}
    DESCRIPTION
    \paragraph{Requirements} Skilled proficiency with Bows.
\subsubsection{Still Aim} \label{feat::stillaim}
    You gain a bonus equal to half your proficiency bonus with bows to damage rolls with a bow if you don't move during your turn or are riding a mount.
    \paragraph{Requirements} Expert proficiency with Bows.
\subsubsection{Quick Shot (2 FP)} \label{feat::quickshot}
    You learn the Quick Draw reaction.

    In addition, when you hit a creature with an opportunity attack with a bow, the creature cannot continue moving during that turn.
    \paragraph{Requirements} Expert proficiency with Bows.
% CROSSBOWS AND GUNS (one feat is only learning guns)
\subsubsection{Crossbow Enthusiast} \label{feat::crossbowenthusiast}
    You increase your proficiency level with crossbows.
    This feat can be taken three times, or until you reach Expert proficiency with the weapon type.
\subsubsection{NAME} \label{feat::name}
    DESCRIPTION
    \paragraph{Requirements} Competent proficiency with Crossbows.
\subsubsection{Quick Loading} \label{feat::quickloading}
    You ignore the loading property of crossbows.
    % In addition, you can load or reload a one-handed ranged weapon without having a free hand.
    \paragraph{Requirements} Competent proficiency with Crossbows.
\subsubsection{Prepared Shot} \label{feat::preparedshot}
    Right after rolling for initiative, you can don a crossbow or firearm and make one ranged attack with it.
    \paragraph{Requirements} Skilled proficiency with Crossbows.
\subsubsection{Modern Shooter (2 FP)} \label{feat::modernshooter}
    You gain proficiency with firearms, using your proficiency bonus with crossbows for your attack rolls with them.
    \paragraph{Requirements} Skilled proficiency with Crossbows.
\subsubsection{Puncture Wound} \label{feat::puncturewound}
    When you get a critical hit on a creature using a ranged weapon, all attacks against that creature are made with advantage until the start of your next turn.
    \paragraph{Requirements} Expert proficiency with Crossbows.
\subsubsection{Visceral Attack (2 FP)} \label{feat::visceralattack}
    Being within 1.5 meters of a hostile creature doesn't impose disadvantage on your ranged attack rolls, and you don't provoke attacks of opportunity when using the Attack action with a ranged weapon when within the reach of a creature.

    In addition, when you successfully hit a creature with a crossbow or firearm while within 1.5 meters of it, your next melee attack against the creature is made with advantage.
    \paragraph{Requirements} Expert proficiency with Crossbows.
% CURVED SWORDS impairingattack
\subsubsection{Bent Blade Master} \label{feat::bentblademaster}
    You increase your proficiency level with curved swords.
    This feat can be taken three times, or until you reach Expert proficiency with the weapon type.
\subsubsection{Minced Meat} \label{feat::mincedmeat}
    You add a +2 bonus on your attack damage made with curved swords against unarmored creatures and creatures wearing armor with the Cloth property.
    \paragraph{Requirements} Competent proficiency with Curved Swords.
\subsubsection{Graceful Disarm} \label{feat::gracefuldisarm}
    You reduce the action cost of the Disarm action to one action.
    If you are holding a curved sword, you have advantage on your attack roll with the Disarm action to force a creature to drop its shield.
    \paragraph{Requirements} Skilled proficiency with Curved Swords.
    DESCRIPTION
    \paragraph{Requirements} Skilled proficiency with Curved Swords.
\subsubsection{Parrying Stance (2 FP)} \label{feat::parryingstance}
    You learn the Parry reaction (see page \pageref{act::parry}).

    In addition, when you use the Dodge action you assume a parrying stance.
    When you do so, you can use the Parry reaction without expending your reaction until the start of your next turn.
    \paragraph{Requirements} Skilled proficiency with Curved Swords.
\subsubsection{NAME} \label{feat::name}
    DESCRIPTION
    \paragraph{Requirements} Expert proficiency with Curved Swords.
\subsubsection{Shield Sweep (2 FP)} \label{feat::shieldsweep}
    Your melee attacks with curved swords ignore the AC provided by shields.
    In addition, whenever you get a critical hit against a target wearing a shield, you can forfeit your brutal critical roll and injure the target's shield-bearing arm as if you had rolled a 4 in the Minor Injury chart (see page \pageref{ssec::injuriesandinsanity}).
    \paragraph{Requirements} Expert proficiency with Curved Swords.
% HAMMERS
\subsubsection{Blunt Fighter} \label{feat::bluntfighter}
    You increase your proficiency level with hammers.
    This feat can be taken three times, or until you reach Expert proficiency with the weapon type.
\subsubsection{Armor Breaker} \label{feat::armorbreaker}
    You add a +2 bonus on your attack damage made with hammers against creatures wearing armor with the Chain or Plate properties.
    \paragraph{Requirements} Competent proficiency with Hammers.
\subsubsection{NAME} \label{feat::name}
    DESCRIPTION
    \paragraph{Requirements} Competent proficiency with Hammers.
\subsubsection{NAME - ACTION} \label{feat::name}
    DESCRIPTION
    \paragraph{Requirements} Skilled proficiency with Hammers.
\subsubsection{NAME (2 FP)} \label{feat::name}
    DESCRIPTION
    \paragraph{Requirements} Skilled proficiency with Hammers.
\subsubsection{NAME} \label{feat::name}
    DESCRIPTION
    \paragraph{Requirements} Expert proficiency with Hammers.
\subsubsection{NAME (2 FP)} \label{feat::name}
    DESCRIPTION
    \paragraph{Requirements} Expert proficiency with Hammers.
% POLEARMS
\subsubsection{Polearms Proficiency} \label{feat::name}
    DESCRIPTION
\subsubsection{NAME} \label{feat::name}
    DESCRIPTION
    \paragraph{Requirements} Competent proficiency with Polearms.
\subsubsection{NAME} \label{feat::name}
    DESCRIPTION
    \paragraph{Requirements} Competent proficiency with Polearms.
\subsubsection{NAME - ACTION} \label{feat::name}
    DESCRIPTION
    \paragraph{Requirements} Skilled proficiency with Polearms.
\subsubsection{NAME (2 FP)} \label{feat::name}
    DESCRIPTION
    \paragraph{Requirements} Skilled proficiency with Polearms.
\subsubsection{NAME} \label{feat::name}
    DESCRIPTION
    \paragraph{Requirements} Expert proficiency with Polearms.
\subsubsection{NAME (2 FP)} \label{feat::name}
    DESCRIPTION
    \paragraph{Requirements} Expert proficiency with Polearms.
% RAPIERS appropiateresponse dignifiedmiss elegantstriker fencer focusattacks impairingattack
\subsubsection{Elegant Striker} \label{feat::elegantstriker}
    You increase your proficiency level with rapiers.
    This feat can be taken three times, or until you reach Expert proficiency with the weapon type.
\subsubsection{Quick Stabs} \label{feat::quickstabs}
    You reduce the Multiple Attack Penalty by 2 with rapiers.
    \paragraph{Requirements} Competent proficiency with Rapiers.
\subsubsection{Impairing Attack} \label{feat::impairingattack}
    During your turn, if you make a melee attack against a creature, that creature can't make opportunity attacks against you for the rest of your turn.
    \paragraph{Requirements} Competent proficiency with Curved Swords or Rapiers.
\subsubsection{Dignified Miss} \label{feat::dignifiedmiss}
    Whenever you miss with a melee weapon attack, you can use your reaction to make another melee weapon attack.
    The Multiple Attack Penalty applies to this attack normally.
    \paragraph{Requirements} Skilled proficiency with Rapiers.
\subsubsection{Appropiate Response (2 FP)} \label{feat::appropiateresponse}
    You learn the Riposte reaction (see page \pageref{act::riposte}).

    In addition, when you use this reaction while wielding a rapier, you attack twice instead of once.
    \paragraph{Requirements} Skilled proficiency with Rapiers.
\subsubsection{Fencer} \label{feat::fencer}
    You gain a +1 bonus to melee attack rolls and your AC when only one creature is within 1.5 meters of you.
    \paragraph{Requirements} Expert proficiency with Rapiers.
\subsubsection{Focus Attacks (2 FP)} \label{feat::focusattacks}
    Using an action, you can search your opponent for vulnerabilities.
    Choose a creature within 1.5 meters of you.
    Until the end of your next turn, your melee attack damage with rapiers against that creature gains a bonus equal to your proficiency modifier with rapiers.
    \paragraph{Requirements} Expert proficiency with Rapiers.
% SPEARS?
% \subsubsection{Spears Proficiency} \label{feat::name}
%     DESCRIPTION
% \subsubsection{NAME} \label{feat::name}
%     DESCRIPTION
%     \paragraph{Requirements} Competent proficiency with Spears.
% \subsubsection{NAME} \label{feat::name}
%     DESCRIPTION
%     \paragraph{Requirements} Competent proficiency with Spears.
% \subsubsection{NAME - ACTION} \label{feat::name}
%     DESCRIPTION
%     \paragraph{Requirements} Skilled proficiency with Spears.
% \subsubsection{NAME (2 FP)} \label{feat::name}
%     DESCRIPTION
%     \paragraph{Requirements} Skilled proficiency with Spears.
% \subsubsection{NAME} \label{feat::name}
%     DESCRIPTION
%     \paragraph{Requirements} Expert proficiency with Spears.
% \subsubsection{NAME (2 FP)} \label{feat::name}
%     DESCRIPTION
%     \paragraph{Requirements} Expert proficiency with Spears.
% SWORDS
\subsubsection{Swordmaster} \label{feat::swordmaster}
    You increase your proficiency level with straight swords.
    This feat can be taken three times, or until you reach Expert proficiency with the weapon type.
\subsubsection{NAME} \label{feat::name}
    DESCRIPTION
    \paragraph{Requirements} Competent proficiency with Straight Swords.
\subsubsection{NAME} \label{feat::name}
    When you are wielding a straight sword in one hand and no other weapons, you gain a +1 bonus to AC.
    \paragraph{Requirements} Competent proficiency with Straight Swords.
\subsubsection{NAME - ACTION} \label{feat::name}
    DESCRIPTION
    \paragraph{Requirements} Skilled proficiency with Straight Swords.
\subsubsection{NAME (2 FP)} \label{feat::name}
    DESCRIPTION
    \paragraph{Requirements} Skilled proficiency with Straight Swords.
\subsubsection{NAME} \label{feat::name}
    DESCRIPTION
    \paragraph{Requirements} Expert proficiency with Straight Swords.
\subsubsection{Stalwart Stance (2 FP)} \label{feat::stalwartstance}
    You learn the Block reaction.

    In addition, when you are wielding a straight sword in one hand and no other weapons, you can a +1 bonus to your AC.
    \paragraph{Requirements} Expert proficiency with Straight Swords.

% === WEAPON TYPES =================================================================================
% BLUDGEONING
\subsubsection{Fell Hand} \label{feat::fellhand}
    When you score a critical hit that deals bludgeoning damage to a creature, attack rolls against that creature are made with advantage until the start of your next turn.
\subsubsection{NAME} \label{feat::name}
    DESCRIPTION
\subsubsection{Off-balance} \label{feat::offbalance}
    Once per turn, when you hit a creature with an attack that deals bludgeoning damage, you can move it 1.5 meters to an unoccupied space, provided that the target is no more than one size larger than you.

    You can take this feat two additional times, increasing the range you move the creature by 1.5 meters each time.
\subsubsection{Crusher (2 FP)} \label{feat::crusher}
    DESCRIPTION
% PIERCING
\subsubsection{Critical Injury} \label{feat::criticalinjury}
    Once per turn, when you hit a creature with an attack that deals piercing damage, you can reroll one of the attack's damage dice, and you must use the new roll.
\subsubsection{NAME} \label{feat::name}
    DESCRIPTION
\subsubsection{Pierce a Lung!} \label{feat::piercealung}
    You learn or improve the Sneak Attack action (see page \pageref{act:sneakattack}).
    You can take this feat three times, adding a d6 to the action's damage each time.

    You can only add these d6s to your Sneak Attack if you attack with a piercing weapon.
\subsubsection{Piercer (2 FP)} \label{feat::piercer}
    DESCRIPTION
% SLASHING
\subsubsection{NAME} \label{feat::name}
    DESCRIPTION
\subsubsection{NAME} \label{feat::name}
    DESCRIPTION
\subsubsection{Aimed Cut} \label{feat::aimedcut}
    You can take this feat three times, learning a different specialized cut each time:
    \begin{itemize}
        \item The first time you take this feat, you learn how to perform a leg cut.
        When you hit a creature with an attack that deals slashing damage, you halve the creature's movement speed until the start of your next turn.
        \item The second time, you learn how to perform an arm cut.
        Pick one of the creature's arms.
        You aim for that arm when you make an attack that deals slashing damage, and all attacks made with that hand are made with disadvantage until the start of your next turn.
        \item The third time, you learn how to cause a hemorrhage.
        For the next minute after being hit by your attack with a slashing weapon, the creature takes 1d4 slashing damage at the start of each of its turns.
        It can staunch this wound using two actions, ending the effect early.
    \end{itemize}

    You can only perform any of these cuts once per turn.
\subsubsection{Slasher (2 FP)} \label{feat::slasher}
    DESCRIPTION

% === FIGHTING STYLES ==============================================================================
\subsubsection{NAME} \label{feat::name}
    DESCRIPTION --- TAKE A FIGHTING STYLE (ONLY ONCE, FOLLOWING TIMES ARE MAJOR CHARACTER ADV.)
% ARCHERY
\subsubsection{NAME} \label{feat::name}
    DESCRIPTION
    \paragraph{Requirements} Fighting Style: Archery 1.
\subsubsection{NAME - ACTION} \label{feat::name}
    DESCRIPTION - UPGRADABLE
    \paragraph{Requirements} Fighting Style: Archery 1.
\subsubsection{NAME (2 FP)} \label{feat::name}
    DESCRIPTION
    \paragraph{Requirements} Fighting Style: Archery 1.
\subsubsection{Sniper} \label{feat::sniper}
    Attacking at long range doesn't impose disadvantage on your ranged weapon attack rolls.
    \paragraph{Requirements} Fighting Style: Archery 2.
\subsubsection{NAME - ACTION} \label{feat::name}
    DESCRIPTION - UPGRADABLE
    \paragraph{Requirements} Fighting Style: Archery 2.
\subsubsection{NAME (2 FP)} \label{feat::name}
    DESCRIPTION
    \paragraph{Requirements} Fighting Style: Archery 2.
% DUELING
\subsubsection{NAME} \label{feat::name}
    DESCRIPTION
    \paragraph{Requirements} Fighting Style: Dueling 1.
\subsubsection{Purposeful Strike} \label{feat::purposefulstrike}
    You learn or improve the Sneak Attack action (see page \pageref{act:sneakattack}).
    You can take this feat three times, adding a d6 to the action's damage each time.

    You can only add these d6s to your Sneak Attack if you are holding a melee weapon on one hand and no other weapons.
    \paragraph{Requirements} Fighting Style: Dueling 1.
\subsubsection{NAME (2 FP)} \label{feat::name}
    DESCRIPTION
    \paragraph{Requirements} Fighting Style: Dueling 1.
\subsubsection{NAME} \label{feat::name}
    DESCRIPTION
    \paragraph{Requirements} Fighting Style: Dueling 2.
\subsubsection{NAME - ACTION} \label{feat::name}
    DESCRIPTION - UPGRADABLE
    \paragraph{Requirements} Fighting Style: Dueling 2.
\subsubsection{NAME (2 FP)} \label{feat::name}
    DESCRIPTION
    \paragraph{Requirements} Fighting Style: Dueling 2.
% GREAT WEAPON FIGHTING
\subsubsection{NAME} \label{feat::name}
    DESCRIPTION
    \paragraph{Requirements} Fighting Style: Great Weapon Fighting 1.
\subsubsection{NAME - ACTION} \label{feat::name}
    DESCRIPTION - UPGRADABLE
    \paragraph{Requirements} Fighting Style: Great Weapon Fighting 1.
\subsubsection{NAME (2 FP)} \label{feat::name}
    DESCRIPTION
    \paragraph{Requirements} Fighting Style: Great Weapon Fighting 1.
\subsubsection{NAME} \label{feat::name}
    DESCRIPTION
    \paragraph{Requirements} Fighting Style: Great Weapon Fighting 2.
\subsubsection{Cleaving} \label{feat::cleaving}
    When fighting with a weapon with the Heavy property, you can attack two creatures instead of one with a melee attack.
    You can attack in this way only once during your turn.

    You can take this feat two additional times, increasing the number of creatures you can hit with this attack by 1 each time.
    \paragraph{Requirements} Fighting Style: Great Weapon Fighting 2.
\subsubsection{NAME (2 FP)} \label{feat::name}
    DESCRIPTION
    \paragraph{Requirements} Fighting Style: Great Weapon Fighting 2.
% MONASTIC FIGHTING?
% \subsubsection{NAME} \label{feat::name}
%     DESCRIPTION
%     \paragraph{Requirements} Fighting Style: Monastic Fighting 1.
% \subsubsection{NAME - ACTION} \label{feat::name}
%     DESCRIPTION - UPGRADABLE
%     \paragraph{Requirements} Fighting Style: Monastic Fighting 1.
% \subsubsection{NAME (2 FP)} \label{feat::name}
%     DESCRIPTION
%     \paragraph{Requirements} Fighting Style: Monastic Fighting 1.
% \subsubsection{NAME} \label{feat::name}
%     DESCRIPTION
%     \paragraph{Requirements} Fighting Style: Monastic Fighting 2.
% \subsubsection{NAME - ACTION} \label{feat::name}
%     DESCRIPTION - UPGRADABLE
%     \paragraph{Requirements} Fighting Style: Monastic Fighting 2.
% \subsubsection{NAME (2 FP)} \label{feat::name}
%     DESCRIPTION
%     \paragraph{Requirements} Fighting Style: Monastic Fighting 2.
% MOUNTED FIGHTING
\subsubsection{NAME} \label{feat::name}
    DESCRIPTION
    \paragraph{Requirements} Fighting Style: Mounted Fighting 1.
\subsubsection{NAME - ACTION} \label{feat::name}
    DESCRIPTION - UPGRADABLE
    \paragraph{Requirements} Fighting Style: Mounted Fighting 1.
\subsubsection{NAME (2 FP)} \label{feat::name}
    DESCRIPTION
    \paragraph{Requirements} Fighting Style: Mounted Fighting 1.
\subsubsection{NAME} \label{feat::name}
    DESCRIPTION
    \paragraph{Requirements} Fighting Style: Mounted Fighting 2.
\subsubsection{NAME - ACTION} \label{feat::name}
    DESCRIPTION - UPGRADABLE
    \paragraph{Requirements} Fighting Style: Mounted Fighting 2.
\subsubsection{NAME (2 FP)} \label{feat::name}
    DESCRIPTION
    \paragraph{Requirements} Fighting Style: Mounted Fighting 2.
% PROTECTION
\subsubsection{NAME} \label{feat::name}
    DESCRIPTION
    \paragraph{Requirements} Fighting Style: Protection 1.
\subsubsection{NAME - ACTION} \label{feat::name}
    DESCRIPTION - UPGRADABLE
    \paragraph{Requirements} Fighting Style: Protection 1.
\subsubsection{NAME (2 FP)} \label{feat::name}
    DESCRIPTION
    \paragraph{Requirements} Fighting Style: Protection 1.
\subsubsection{NAME} \label{feat::name}
    DESCRIPTION
    \paragraph{Requirements} Fighting Style: Protection 2.
\subsubsection{NAME - ACTION} \label{feat::name}
    DESCRIPTION - UPGRADABLE
    \paragraph{Requirements} Fighting Style: Protection 2.
\subsubsection{NAME (2 FP)} \label{feat::name}
    DESCRIPTION
    \paragraph{Requirements} Fighting Style: Protection 2.
% THROWN WEAPON FIGHTING
\subsubsection{NAME} \label{feat::name}
    DESCRIPTION
    \paragraph{Requirements} Fighting Style: Thrown Weapons Fighting 1.
\subsubsection{NAME - ACTION} \label{feat::name}
    DESCRIPTION - UPGRADABLE
    \paragraph{Requirements} Fighting Style: Thrown Weapons Fighting 1.
\subsubsection{NAME (2 FP)} \label{feat::name}
    DESCRIPTION
    \paragraph{Requirements} Fighting Style: Thrown Weapons Fighting 1.
\subsubsection{NAME} \label{feat::name}
    DESCRIPTION
    \paragraph{Requirements} Fighting Style: Thrown Weapons Fighting 2.
\subsubsection{NAME - ACTION} \label{feat::name}
    DESCRIPTION - UPGRADABLE
    \paragraph{Requirements} Fighting Style: Thrown Weapons Fighting 2.
\subsubsection{NAME (2 FP)} \label{feat::name}
    DESCRIPTION
    \paragraph{Requirements} Fighting Style: Thrown Weapons Fighting 2.
% TWO-WEAPON FIGHTING
\subsubsection{NAME} \label{feat::name}
    DESCRIPTION
    \paragraph{Requirements} Fighting Style: Two-Weapon Fighting 1.
\subsubsection{NAME - ACTION} \label{feat::name}
    DESCRIPTION - UPGRADABLE
    \paragraph{Requirements} Fighting Style: Two-Weapon Fighting 1.
\subsubsection{NAME (2 FP)} \label{feat::name}
    DESCRIPTION
    \paragraph{Requirements} Fighting Style: Two-Weapon Fighting 1.
\subsubsection{NAME} \label{feat::name}
    DESCRIPTION
    \paragraph{Requirements} Fighting Style: Two-Weapon Fighting 2.
\subsubsection{NAME - ACTION} \label{feat::name}
    DESCRIPTION - UPGRADABLE
    \paragraph{Requirements} Fighting Style: Two-Weapon Fighting 2.
\subsubsection{NAME (2 FP)} \label{feat::name}
    DESCRIPTION
    \paragraph{Requirements} Fighting Style: Two-Weapon Fighting 2.
% UNARMED FIGHTING
\subsubsection{NAME} \label{feat::name}
    DESCRIPTION
    \paragraph{Requirements} Fighting Style: Unarmed Fighting 1.
\subsubsection{NAME - ACTION} \label{feat::name}
    DESCRIPTION - UPGRADABLE
    \paragraph{Requirements} Fighting Style: Unarmed Fighting 1.
\subsubsection{NAME (2 FP)} \label{feat::name}
    DESCRIPTION
    \paragraph{Requirements} Fighting Style: Unarmed Fighting 1.
\subsubsection{Unarmed Artist} \label{feat::unarmedartist}
    You are a painter.
    Your brush is your fist, and your canvas is that guy's face.

    When you get a critical hit with an unarmed strike, you can take the Disarm, Disengage, or Shove action as a free action.
    \paragraph{Requirements} Fighting Style: Unarmed Fighting 2.
\subsubsection{NAME - ACTION} \label{feat::name}
    DESCRIPTION - UPGRADABLE
    \paragraph{Requirements} Fighting Style: Unarmed Fighting 2.
\subsubsection{NAME (2 FP)} \label{feat::name}
    DESCRIPTION
    \paragraph{Requirements} Fighting Style: Unarmed Fighting 2.
% BATTLE MASTERY assessthesituation commander
\subsubsection{NAME} \label{feat::name}
    DESCRIPTION
    \paragraph{Requirements} Fighting Style: Battle Mastery 1.
\subsubsection{Assess the Situation} \label{feat::assessthesituation}
    Right after rolling initiative, you can quickly assess a creature of your choice.
    Make an Intelligence (Investigation) check contested by a Charisma (Deception) check made by the creature.
    On a success, you learn the creature's vulnerabilities, resistances, and immunities, if it has any.

    You can take this feat three times, increasing the number of creature you can use this ability with by 1 the second and third times.
    \paragraph{Requirements} Fighting Style: Battle Mastery 1.
\subsubsection{NAME (2 FP)} \label{feat::name}
    DESCRIPTION
    \paragraph{Requirements} Fighting Style: Battle Mastery 1.
\subsubsection{Commander} \label{feat::commander}
    You have advantage on all Charisma (Deception), Charisma (Intimidation), Charisma (Performance), and Charisma (Persuasion) checks while in combat.
    \paragraph{Requirements} Fighting Style: Battle Mastery 2.
\subsubsection{NAME - ACTION} \label{feat::name}
    DESCRIPTION - UPGRADABLE
    \paragraph{Requirements} Fighting Style: Battle Mastery 2.
\subsubsection{NAME (2 FP)} \label{feat::name}
    DESCRIPTION
    \paragraph{Requirements} Fighting Style: Battle Mastery 2.

% === EXOTIC WEAPONS ===============================================================================
% BLOWGUNS
\subsubsection{Blowguns Proficiency} \label{feat::name}
    DESCRIPTION
\subsubsection{NAME} \label{feat::name}
    DESCRIPTION
    \paragraph{Requirements} Competent proficiency with Blowguns.
\subsubsection{NAME - ACTION} \label{feat::name}
    DESCRIPTION
    \paragraph{Requirements} Skilled proficiency with Blowguns.
\subsubsection{NAME (2 FP)} \label{feat::name}
    DESCRIPTION
    \paragraph{Requirements} Expert proficiency with Blowguns.
% FLAILS
\subsubsection{Flails Proficiency} \label{feat::name}
    DESCRIPTION
\subsubsection{NAME} \label{feat::name}
    DESCRIPTION
    \paragraph{Requirements} Competent proficiency with Flails.
\subsubsection{NAME - ACTION} \label{feat::name}
    DESCRIPTION
    \paragraph{Requirements} Skilled proficiency with Flails.
\subsubsection{NAME (2 FP)} \label{feat::name}
    DESCRIPTION
    \paragraph{Requirements} Expert proficiency with Flails.
% NETS & BOLAS
\subsubsection{Nets \& Bolas Proficiency} \label{feat::name}
    DESCRIPTION
\subsubsection{NAME} \label{feat::name}
    DESCRIPTION
    \paragraph{Requirements} Competent proficiency with Nets \& Bolas.
\subsubsection{NAME - ACTION} \label{feat::name}
    DESCRIPTION
    \paragraph{Requirements} Skilled proficiency with Nets \& Bolas.
\subsubsection{NAME (2 FP)} \label{feat::name}
    DESCRIPTION
    \paragraph{Requirements} Expert proficiency with Nets \& Bolas.
% WHIPS
\subsubsection{Whips Proficiency} \label{feat::name}
    DESCRIPTION
\subsubsection{NAME} \label{feat::name}
    DESCRIPTION
    \paragraph{Requirements} Competent proficiency with Whips.
\subsubsection{NAME - ACTION} \label{feat::name}
    DESCRIPTION
    \paragraph{Requirements} Skilled proficiency with Whips.
\subsubsection{NAME (2 FP)} \label{feat::name}
    DESCRIPTION
    \paragraph{Requirements} Expert proficiency with Whips.

% CHAMPION: get critical hits from rolls of 19 or 20 instead of only 20. (2 FP)

% % === ARMOR ==================================================================== %
% \subsubsection{Stealthy} \label{feat::stealthy}
% \small{\textcolor{gray}{Stealth}}
%
% \normalsize
% You know how best to hide, and use your dyed light armor to benefit your stealth.
% \paragraph{REQUIREMENTS} Lightly Armored 2 and Sly 2.
% \paragraph{RANK 1} You can take the Hide action as a bonus action on each of your turns.
% \paragraph{RANK 2} If you are hidden, you can move up to 3 meters in the open without revealing yourself if you end the move in a position where you're not clearly visible.
% \paragraph{RANK 3}
% % ============================================================================== %
% \subsubsection{Drunken Brawler} \label{feat::drunkenbrawler} %
% \small{\textcolor{gray}{}}
% % Skills emulating drunken fighting AND actual bonus when drunk.
%
% \normalsize
% Description.
% \paragraph{REQUIREMENTS}
% \paragraph{RANK 1}
% \paragraph{RANK 2} When you suffer from the effects of drunkenness, you have advantage on Constitution and Strength saving throws.
% \paragraph{RANK 3} You learn the Careless Deflect technique.

% ============================================================================== %
\subsubsection{Armor Breaker} \label{feat::armorbreaker}
\small{\textcolor{gray}{Strength}}

\normalsize
You use your extensive knowledge of heavy armor to improve your effectiveness against it, turning your bludgeoning attacks into devastating metal bending strikes.
\paragraph{REQUIREMENTS} Hammer Adept 2 and Heavily Armored 1.
\paragraph{RANK 1} You can use the Bonk technique as an opportunity attack.
\paragraph{RANK 2}
\paragraph{RANK 3} You learn the Fell Strike technique.

% ============================================================================== %
\subsubsection{Ball-and-Chain Master} \label{feat::ballandchainmaster}
\small{\textcolor{gray}{Strength}}

\normalsize
Your increased awareness works in tandem to your flail mastery, allowing you to hit enemies behind obstacles and shields.
\paragraph{REQUIREMENTS} Flail Adept 2 and Insightful 2.
\paragraph{RANK 1} When you use a ball-and-chain, its damage die changes from a d6 to a d8.
\paragraph{RANK 2} When you hit with an opportunity attack using a ball-and-chain, the target must succeed on a Strength saving throw (DC 8 + your proficiency bonus + your Strength modifier) or be knocked prone.
\paragraph{RANK 3} You learn the Shield Sweep technique.

% ============================================================================== %
\subsubsection{Blade \& Board} \label{feat::bladeandboard}
\small{\textcolor{gray}{Strength}}

\normalsize
You know the perfect balance between defense and attack, making you a formidable foe in melee combat.
\paragraph{REQUIREMENTS} Straight Blade Adept 2 and Shield Training 2, or Axe Adept 2 and Shield Training 2.
\paragraph{RANK 1} When you're wielding a slashing weapon and a shield, you gain a +1 to AC and to your attack damage.
\paragraph{RANK 2} While you're wielding a shield and a creature misses you with a melee attack, you can use your reaction to attack it with a melee attack.

% ============================================================================== %
\subsubsection{Blade Breaker} \label{feat::bladebreaker}
\small{\textcolor{gray}{Dexterity}}

\normalsize
You know that the perfect counter against any sword is simply a long pole.
\paragraph{REQUIREMENTS} Staff Fighter 2.
\paragraph{RANK 1} Staves have the Reach property when wielded by you.
\paragraph{RANK 2} You have advantage on the Disarm action and the Parry technique against creatures using any type of sword.
\paragraph{RANK 3} While using a polearm, you can use the Disarm action as your opportunity attack.

% ============================================================================== %
\subsubsection{Blowgun Adept} \label{feat::blowgunadept}
\small{\textcolor{gray}{Dexterity}}

\normalsize
While unconventional, you understand the effectiveness of the blowgun when remaining hidden is of the essence.
\paragraph{REQUIREMENTS} Armed Fighter 1.
\paragraph{RANK 1} You are proficient with blowguns.
\paragraph{RANK 2} You gain a +1 bonus to attack rolls with a blowgun.
Additionally, attacking with a blowgun at long range doesn't impose disadvantage on your ranged weapon attack rolls.
\paragraph{RANK 3} When you hit with a blowgun while hidden, your location is not revealed.

% ============================================================================== %
\subsubsection{Blunt Thrower} \label{feat::bluntthrower}
\small{\textcolor{gray}{Strength}}

\normalsize
You take advantage of the raw practicality of a good hammer, and not even faraway foes can escape the wrath of its face.
\paragraph{REQUIREMENTS} Hammer Adept 2 and Thrown Weapon Master 2.
\paragraph{RANK 1} You don't provoke opportunity attacks when picking weapons up from the floor.
\paragraph{RANK 2} When you hit a creature with an thrown bludgeoning weapon, you divide its moving speed by half (rounded up) during its next turn.
This effect does not stack.
\paragraph{RANK 3} You learn the Hammer Fling technique.

% ============================================================================== %
\subsubsection{Blunt Master} \label{feat::bluntmaster}
\small{\textcolor{gray}{Dexterity}}

\normalsize
Your staff is an extension of your body, and it has become an essential part of your movement and defense.
\paragraph{REQUIREMENTS} Staff Fighter 2 and Unarmed Artist 2.
\paragraph{RANK 1} As a bonus action, you can use your staff to propel yourself into the air, making a high jump up to your Dexterity modifier times 1.5 meters.
Additionally, if you have moved at least 3 meters during this round, you can jump in a straight line up to a distance equal to your Dexterity modifier times 3 meters.
\paragraph{RANK 2} You gain a +1 bonus to your AC while using a staff.
Additionally, if you use the weapon with two hands you gain another +1 bonus to your AC.
\paragraph{RANK 3} You learn the Defensive Stance technique.

% ============================================================================== %
\subsubsection{Bullet Tinkerer} \label{feat::bullettinkerer}
\small{\textcolor{gray}{Science}}

\normalsize
Firearm ammunition is near impossible to find or purchase in most parts of Yuadrem.
Due to this, learning how to craft it yourself is an essential aspect for using firearms.
\paragraph{REQUIREMENTS} Musket Adept 2 and Educated 2, or Pistol Adept 2 and Educated 2.
\paragraph{RANK 1} You can craft ammunition using a set of Tinker's Tools at half the cost.
\paragraph{RANK 2} If you roll a misfire, you can use your reaction to roll a d20 with disadvantage.
If the number rolled is higher than the weapon's misfire score, the weapon does not misfire.
\paragraph{RANK 3} You learn the Violent Shot technique.
\subsubsection{Combat Improviser} \label{feat::combatimproviser}
\small{\textcolor{gray}{Strength or Dexterity}}

\normalsize
Anything can be your weapon.
If you can grab it, you can swing it.
\paragraph{REQUIREMENTS} Armed Fighter 1.
\paragraph{RANK 1} You are proficient with improvised weapons.
An improvised weapon is anything that is somewhat similar to a simple weapon.
\paragraph{RANK 2} After rolling for initiative, you make your first attack with an improvised weapon at advantage.
\paragraph{RANK 3} You learn one technique of your choice between Bonk, Block, Buttstroke, Feint, Lunge, Parry, Pushing Attack, Protect, Quick Draw, Riposte, Rush, Sweep, and Trip.

% ============================================================================== %
\subsubsection{Cutthroat} \label{feat::cutthroat}
\small{\textcolor{gray}{Dexterity}}

\normalsize
A master at concealing your dagger, you act quick and strike hard in combat.
\paragraph{REQUIREMENTS} Dagger Savant 2.
\paragraph{RANK 1} You have advantage on any check to conceal a small weapon on your person.
\paragraph{RANK 2} You gain a +2 bonus to initiative. % when wielding a light weapon.
\paragraph{RANK 3} You gain or improve the Sneak Attack technique.

% ============================================================================== %
\subsubsection{Dagger Savant} \label{feat::daggersavant}
\small{\textcolor{gray}{Dexterity}}

\normalsize
You wield the lightest of blades with the deadliest of skills.
\paragraph{REQUIREMENTS} Armed Fighter 1.
\paragraph{RANK 1} You can perform one melee attack with a dagger as part of a Grapple, Escape a Grapple, Overrun, or Tumble action.
\paragraph{RANK 2} You gain or improve the Sneak Attack technique.
\paragraph{RANK 3} You learn the Riposte technique.

% ============================================================================== %
\subsubsection{Defensive Duelist} \label{feat::defensiveduelist}
\small{\textcolor{gray}{Dexterity}}

\normalsize
A master of the buckler, you have learned to use it as an addition to your main weapon.
\paragraph{REQUIREMENTS} Buckler Training 2.
\paragraph{RANK 1} When you use the Parry technique, you can roll a d8 instead of a d6 when increasing your AC.
\paragraph{RANK 2} When fighting with a one-handed melee weapon and a light shield, you can add half your shield's AC (rounded up) to the weapon's attack damage.
\paragraph{RANK 3} When you use your Parry technique and the attacking creature misses, you can use the Disarm action against it as part of your reaction.

% ============================================================================== %
\subsubsection{Dual Wielder} \label{feat::dualwielder}
\small{\textcolor{gray}{Dexterity}}

\normalsize
You master fighting with two weapons.
\paragraph{REQUIREMENTS} Dexterity 13. Armed Fighter 2.
\paragraph{RANK 1} When you engage in two-weapon fighting, you can add your ability modifier to the damage of the second attack.
\paragraph{RANK 2} You can draw or stow two one-handed weapons when you would normally be able to draw or stow only one.
\paragraph{RANK 3} You learn the Lock technique.

% ============================================================================== %
\subsubsection{Far Thruster} \label{feat::farthruster}
\small{\textcolor{gray}{Strength or Dexterity}}

\normalsize
You take advantage of your long weapon to keep your distance from foes.
\paragraph{REQUIREMENTS} Halberd Adept 2 and Spear Adept 2.
\paragraph{RANK 1} After successfully attacking a creature, you can use the Shove action as a bonus action.
\paragraph{RANK 2} You gain a +2 to your attack rolls made with a polearm against a creature larger than you.
\paragraph{RANK 3} You learn the Extend Reach technique.

% ============================================================================== %
\subsubsection{Fell Hand} \label{feat::fellhand}
\small{\textcolor{gray}{Strength}}

\normalsize
You are an expert with the greatclub, and can use its raw power to pancake even the hardiest foes.
\paragraph{REQUIREMENTS} Hammer Adept 2.
\paragraph{RANK 1} When you use a greatclub, its damage die changes from a d8 to a d10.
\paragraph{RANK 2} Whenever you miss with a melee attack using a greatclub, the target takes bludgeoning damage equal to your Strength modifier (minimum of 2).
\paragraph{RANK 3} When your target fails its saving throw against your Bonk technique, it also falls prone.

% ============================================================================== %
\subsubsection{Firearm Specialist} \label{feat::firearmspecialist}
\small{\textcolor{gray}{Dexterity}}

\normalsize
You are particularly well at combat with pistols, and are a shining example of this new weapon's viability in combat.
\paragraph{REQUIREMENTS} Pistol Adept 2.
\paragraph{RANK 1} You can reload any weapon as a bonus action.
\paragraph{RANK 2} If you are using a pistol in one hand and nothing on the other, you get a +2 bonus to your ranged weapon attacks with it.
\paragraph{RANK 3} You learn the Rapid Repair technique.

% ============================================================================== %
\subsubsection{Flail Adept} \label{feat::flailadept}
\small{\textcolor{gray}{Strength or Dexterity}}

\normalsize
The flail is a tricky weapon to use, but you have spent countless hours mastering it.
\paragraph{REQUIREMENTS} Armed Fighter 2.
\paragraph{RANK 1} You are proficient with flails.
\paragraph{RANK 2} When wielding flails, enemies provoke opportunity attacks from you when they enter your reach.
\paragraph{RANK 3} You learn the Sweep technique.

% ============================================================================== %
\subsubsection{Giant Slayer} \label{feat::giantslayer}
\small{\textcolor{gray}{Strength}}

\normalsize
Partaking or not in Krudzal's Eternal War, you use your incredible strength to wield their massive weapons.
\paragraph{REQUIREMENTS} Strength 16. Great Weapon User 2.
\paragraph{RANK 1} You are able to use Krudzal's giant-slayers.
\paragraph{RANK 2} You gain a +1 bonus to damage rolls with weapons with the Great property.

% ============================================================================== %
\subsubsection{Glaive Master} \label{feat::glaivemaster}
\small{\textcolor{gray}{Dexterity}}

\normalsize
Different than the common polearm, you know how deadly a glaive can be in capable hands.
\paragraph{REQUIREMENTS} Halberd Adept 2.
\paragraph{RANK 1} When you use a glaive, its damage die changes from a d10 to a d12.
\paragraph{RANK 2} The glaive has the finesse property for you.
\paragraph{RANK 3} When you miss an attack, you can attempt to strike your enemy with the rondel of your glaive as a free action.
On a hit, the target takes 1d4 + your Strength modifier bludgeoning damage.
You can do this once per turn.

% ============================================================================== %
\subsubsection{Grappler} \label{feat::grappler}
\small{\textcolor{gray}{Strength}}

\normalsize
You've developed the skills necessary to hold your own in close-quarters grappling.
\paragraph{REQUIREMENTS} Pugilist 2.
\paragraph{RANK 1} You have advantage on attack rolls against creatures you are grappling.
\paragraph{RANK 2} You benefit from three-quarters cover while you grapple a creature of a size equal or greater than yours.
\paragraph{RANK 3} You learn the Pin technique.

% ============================================================================== %
\subsubsection{Great Weapon User} \label{feat::greatweaponuser}
\small{\textcolor{gray}{Strength}}

\normalsize
You've learned to put the weight of a weapon to your advantage, letting its momentum empower your strikes.
\paragraph{REQUIREMENTS} Strength 13. Armed Fighter 2.
\paragraph{RANK 1} You can use weapons with the great property.
Some of these, like the zweihander and the greataxe, also require you to have proficiency with the corresponding weapon type.
\paragraph{RANK 2} On your turn, when you score a critical hit with a melee weapon or reduce a creature to 0 hit points with one, you can make one melee weapon attack as a bonus action.
\paragraph{RANK 3} You learn the Reckless Strike technique.

% ============================================================================== %
\subsubsection{Greatshield Training} \label{feat::greatshieldtraining}
\small{\textcolor{gray}{Strength}}

\normalsize
When it comes to protecting your life or that of others, you know that size matters.
\paragraph{REQUIREMENTS} Shield Training 1.
\paragraph{RANK 1} You gain proficiency with heavy shields
\paragraph{RANK 2} Creatures standing behind you get three-fourths cover against ranged attacks.
\paragraph{RANK 3} You learn the Protect technique.

% ============================================================================== %
\subsubsection{Gunner} \label{feat::gunner}
\small{\textcolor{gray}{Dexterity}}

\normalsize
Being a new breed of weapon, firearms are constantly subject to change and refinement.
New weapons appear by the minute, and you are an expert at experimenting with them.
\paragraph{REQUIREMENTS} Pistol Adept 2 and Musket Adept 2.
\paragraph{RANK 1} As long as you can examine the weapon for 30 seconds, you are proficient with any kind of firearm, even if it is new or experimental.
\paragraph{RANK 2} Your firearm attacks score a critical hit on a roll of 19-20.
\paragraph{RANK 3} You learn the Reckless Shot technique.

% ============================================================================== %
\subsubsection{Halberd Adept} \label{feat::halberdadept}
\small{\textcolor{gray}{Strength or Dexterity}}

\normalsize
You know that both farmers and nobles use halberds as their weapon of choice for a reason.
\paragraph{REQUIREMENTS} Armed Fighter 2 and Staff Fighter 2.
\paragraph{RANK 1} You are proficient with the many varieties of halberds.
\paragraph{RANK 2} When a creature within 1.5 meters of you makes an attack against a target other than you, you can use your reaction to make a melee weapon attack against the attacking creature.
\paragraph{RANK 3} You learn the Trip technique.

% ============================================================================== %
\subsubsection{Hammer Adept} \label{feat::hammeradept}
\small{\textcolor{gray}{Strength}}

\normalsize
If a hammer will do, why complicate things?
\paragraph{REQUIREMENTS} Armed Fighter 2.
\paragraph{RANK 1} You gain proficiency with hammer weapons.
\paragraph{RANK 2} If you use the Help action to aid an ally's melee attack while you're wielding a hammer weapon, you knock the target's shield aside momentarily.
In addition to the ally gaining advantage on the attack roll, the ally gains a bonus to the roll equal to the shield's AC bonus.
\paragraph{RANK 3} You learn the Bonk technique.

% ============================================================================== %
% ============================================================================== %
\subsubsection{Heavily Armored} \label{feat::heavilyarmored}
\small{\textcolor{gray}{Strength}}

\normalsize
You are trained to use heavy armor.
\paragraph{REQUIREMENTS} Moderately Armored 1.
\paragraph{RANK 1} You gain proficiency with heavy armor.
\paragraph{RANK 2} Equipped heavy armor and metal accessories only add half their weight to your encumbrance.
\paragraph{RANK 3} Increase your Strength by 1, to a maximum of 20.

% ============================================================================== %
\subsubsection{Heavy Armor Master} \label{feat::heavyarmormaster}
\small{\textcolor{gray}{Strength}}

\normalsize
You can use your armor to deflect strikes that would easily kill others.
\paragraph{REQUIREMENTS} Heavily Armored 2 and Athletics 2.
\paragraph{RANK 1} Your hit point maximum increases by an amount equal to your number of hit dice.
Whenever you earn a hit die thereafter, your hit point maximum increases by an additional 1 hit point.
\paragraph{RANK 2} While you are wearing heavy armor, piercing and slashing damage that you take is reduced by 3.
\paragraph{RANK 3} You learn the Defend technique.

% ============================================================================== %
\subsubsection{Heavy Lifter} \label{feat::heavylifter}
\small{\textcolor{gray}{Strength}}

\normalsize
You can easily lift and hurl objects that others can barely move.
\paragraph{REQUIREMENTS} Combat Improviser 2 and Thrown Weapon Master 2.
\paragraph{RANK 1} You can hurl any object or creature you can carry or grapple as you would an improvised weapon.
If you hurl a creature, both it and the target of your attack take 1d6 + your Strength modifier bludgeoning damage.
\paragraph{RANK 2} You count as if you were one size larger for the purpose of determining your carrying capacity.
\paragraph{RANK 3} You learn the Suplex technique.

% ============================================================================== %
\subsubsection{Heavy-Weight Combatant} \label{feat::heavyweightcombatant}
\small{\textcolor{gray}{Athletics}}

\normalsize
You can use the weight of your body and your armor to greatly aid you in combat.
\paragraph{REQUIREMENTS} Heavily Armored 2 and Pugilist 2.
\paragraph{RANK 1} If you are wearing metal gauntlets, your unarmed strike uses a d6 + your Strength modifier for damage.
\paragraph{RANK 2} While you are wearing heavy armor, you roll with advantage when taking the Shove action.
You also roll Strength (Athletics) with advantage when you are targeted by a Shove action.
\paragraph{RANK 3} You have learned to strategically position the crevices in your armor to hurt creatures too close to you.
When you grapple a creature, the target takes 1d6 piercing damage if your grapple check succeeds.
If another creature grapples you, they take 1d6 piercing damage.
If you start your turn grappling or being grappled by a creature, it takes 1d6 piercing damage.

% ============================================================================== %
\subsubsection{Horse Killer} \label{feat::horsekiller}
\small{\textcolor{gray}{Animal Handling}}

\normalsize
Mounts are one of the most common advantages in war, so you've learned to deal with them appropriately.
\paragraph{REQUIREMENTS} Spear Adept 2 and Animal Handler 2.
\paragraph{RANK 1} You can always target a rider's mount with an attack made with a spear, even if the rider can normally deflect this attack to it.
\paragraph{RANK 2} You have a +2 to your attack rolls made against mounts.
\paragraph{RANK 3} You learn the Charge Stopper technique.

% ============================================================================== %
\subsubsection{Immovable Object} \label{feat::immovableobject}
\small{\textcolor{gray}{Strength}}

\normalsize
You are an obstacle.
\paragraph{REQUIREMENTS} Greatshield Training 2.
\paragraph{RANK 1} Creatures can't use a Tumble action to move through your space.
Additionally, you have advantage on Strength (Athletics) checks against being shoved or pushed.
\paragraph{RANK 2} While using a heavy shield, you get three-fourths cover against ranged attacks.
Additionally, you can duck behind your shield as a reaction, granting you full cover against ranged attacks until the start of your next turn.
\paragraph{RANK 3} The movement speed debuff from using a tower shield is halved for you.

% ============================================================================== %
\subsubsection{Insistent Fighter} \label{feat::insistentfighter}
\small{\textcolor{gray}{Strength or Dexterity}}

\normalsize
You always maintain your foes at a cozy distance by masterfully manipulating your halberd.
\paragraph{REQUIREMENTS} Halberd Adept 2 and Observant 2.
\paragraph{RANK 1} Creature's provoke opportunity attacks from you even if they take the Disengage action before leaving your reach.
\paragraph{RANK 2} When you hit a creature with an opportunity attack, the creature's speed becomes 0 for the rest of the turn.
\paragraph{RANK 3} Your reach for opportunity attacks with polearms is of at least 3 meters.
Additionally, if you do hit a creature with an opportunity attack, you can choose to pull it towards you by 1.5 meters.

% ============================================================================== %
\subsubsection{Kama Master} \label{feat::kamamaster}
\small{\textcolor{gray}{Dexterity}}

\normalsize
From among the common people weapon, you specialize in the use of the kama.
\paragraph{REQUIREMENTS} Pick Adept 2.
\paragraph{RANK 1} When you use a kama, its damage die changes from a d6 to a d8
\paragraph{RANK 2} When you successfully attack a foe with a piercing attack using your kama, you also deal 1d4 slashing damage, unmodified by your dexterity.
\paragraph{RANK 3} You can use the Trip technique as part of an opportunity attack.

% ============================================================================== %
\subsubsection{Longsword Master} \label{feat::longswordmaster}
\small{\textcolor{gray}{Strength or Dexterity}}

\normalsize
Your proficiency with the longsword is unparalleled, and you regularly prove to your foes that the old sword is as valid today as it was on its golden age.
\paragraph{REQUIREMENTS} Straight Sword Adept 2.
\paragraph{RANK 1} When you use a longsword, its damage die changes from a d10 to a d12.
\paragraph{RANK 2} When a creature misses you with a melee weapon attack, you can roll to use the Disarm action on it as your reaction.
\paragraph{RANK 3} When you use the Block technique and an attack misses you, you can use the Riposte technique without using your reaction.

% ============================================================================== %
\subsubsection{Maiming Strikes} \label{feat::maimingstrikes}
\small{\textcolor{gray}{Strength}}

\normalsize
Your increased proficiency cleaving through wood and stone translates to devastating attacks against flesh and bone.
\paragraph{REQUIREMENTS} Axe Adept 2 and Pick Adept 2.
\paragraph{RANK 1} The last attack you make with the Attack action is rolled with advantage if you hit the target at least once in your turn.
\paragraph{RANK 2} When a creature rolls on the minor \& major injury charts from one of your attacks, they roll with advantage.
\paragraph{RANK 3} You learn the Cleave technique.

% ============================================================================== %
\subsubsection{Medium Haft Master} \label{feat::mediumhaftmaster}
\small{\textcolor{gray}{Strength}}

\normalsize
Why learn how to fight with a sword when your tools will do the job?
\paragraph{REQUIREMENTS} Axe Adept 2, Hammer Adept 2, and Pick Adept 2.
\paragraph{RANK 1} You gain a +1 bonus to attack rolls you make with axes, hammers, and picks.
\paragraph{RANK 2} You can treat any weapon as if it had the Heavy property.
\paragraph{RANK 3} You learn the Shield Break technique.

% ============================================================================== %
\subsubsection{Modern Soldier} \label{feat::modernsoldier}
\small{\textcolor{gray}{Strength or Dexterity}}

\normalsize
Always up-to-date, you are ready to fight your foes with both a melee and a ranged weapon.
\paragraph{REQUIREMENTS} Straight Sword Adept 2 and Crossbow Adept 2.
\paragraph{RANK 1} When you use the Attack action and attack with a one-handed weapon, you can use your bonus action to attack with a hand crossbow or loaded firearm you are holding.
\paragraph{RANK 2} When using a backsword or a sabre on one hand and your other hand is free, you can hold the back of the blade with your free hand to increase its cutting power.
You add a +2 to your melee weapon attack damage while wielding your weapon in this way.

% ============================================================================== %
\subsubsection{Musket Adept} \label{feat::musketadept}
\small{\textcolor{gray}{Dexterity}}

\normalsize
Quickly picking up on new trends, you gambled on muskets and their stopping power early on.
\paragraph{REQUIREMENTS} Armed Fighter 2.
\paragraph{RANK 1} You are proficient with muskets.
\paragraph{RANK 2} While you have a two-handed firearm equipped, you have advantage on rolls against effects that would move you.
\paragraph{RANK 3} You learn the Buttstroke technique.

% ============================================================================== %
\subsubsection{Pick Adept} \label{feat::pickadept}
\small{\textcolor{gray}{Strength or Dexterity}}

\normalsize
For you, there is no difference between mining ore and piercing flesh.
Apart from the screams at least.
% You pierce through metal, flesh, and bone just as you would ore.
\paragraph{REQUIREMENTS} Armed Fighter 2.
\paragraph{RANK 1} You are proficient with picks.
\paragraph{RANK 2} All picks have the versatile property for you.
When you wield a pick with two hands, increase its damage die by one (d6 to d8, d8 to d10, etc), and have the heavy property.
\paragraph{RANK 3} You learn the Trip technique.

% ============================================================================== %
\subsubsection{Pike Master} \label{feat::pikemaster}
\small{\textcolor{gray}{Strength}}

\normalsize
You prefer to maintain distance from your opponents by wielding an amusingly long pole.
\paragraph{REQUIREMENTS} Spear Adept 2.
\paragraph{RANK 1} When you use a pike, its damage die changes from a d10 to a d12.
\paragraph{RANK 2} When you use the Feint technique you can choose to not attack with advantage.
If you do this and still hit your target, it's a critical hit.
\paragraph{RANK 3} When wielding a pike, its reach property adds 3 meters instead of 1.5 meters to the weapon's range.

% ============================================================================== %
\subsubsection{Polearm Master} \label{feat::polerarmmaster} %
\small{\textcolor{gray}{Dexterity}}

\normalsize
You keep your enemies at bay with your long weapons.
\paragraph{REQUIREMENTS} Staff Fighter 2, Spear Adept 2, and Halberd Adept 2.
\paragraph{RANK 1} You gain a +1 bonus to attack rolls you make with staves, spears, and halberds.
\paragraph{RANK 2} While wielding a stave, spear, or halberd, other creatures provoke an opportunity attack from you when they enter your reach.
\paragraph{RANK 3} You learn the Rondel Hit technique.

% ============================================================================== %
\subsubsection{Precise Thrower} \label{feat::precisethrower}
\small{\textcolor{gray}{Dexterity}}

\normalsize
Either from luck or rigorous training, you have the capacity to hit the hardest of targets when throwing a weapon.
\paragraph{REQUIREMENTS} Dagger Savant 2 and Thrown Weapon Master 2.
\paragraph{RANK 1} Attacking at long range doesn't impose disadvantage on your thrown weapon attack rolls.
\paragraph{RANK 2} Whenever you hit a creature with an thrown slashing or piercing weapon, it has disadvantage on the first melee attack it performs on its next turn.
\paragraph{RANK 3} You learn the Bull's Eye technique.

% ============================================================================== %
\subsubsection{Pugilist} \label{feat::pugilist}
\small{\textcolor{gray}{Strength}}

\normalsize
Your fighting style has only three components:
A fist, another fist, and a face to punch.
\paragraph{RANK 1} Your unarmed strike uses a d4 + your Strength modifier for damage.
\paragraph{RANK 2} When you hit a creature with an unarmed strike, you can use a bonus action to attempt to grapple the target.
\paragraph{RANK 3} You learn the Bonk technique.

% ============================================================================== %
\subsubsection{Revenant Blade} \label{feat::revenantblade}
\small{\textcolor{gray}{Dexterity}}

\normalsize
You are a expert of the double-bladed scimitar, and use the technique as proficiently as any master from the ancient Janchu'ut school of hunters.
\paragraph{REQUIREMENTS} Curved Blade Adept 2.
\paragraph{RANK 1} A double-bladed scimitar has the finesse property when you wield it.
\paragraph{RANK 2} While you are holding a double-bladed scimitar with two hands, you gain a +1 bonus to Armor Class.
\paragraph{RANK 3} Instead of dealing 1d4 slashing damage, the extra attack you can do with a double-bladed scimitar deals 2d4 slashing damage.

% ============================================================================== %
\subsubsection{Sharpshooter} \label{feat::sharpshooter}
\small{\textcolor{gray}{Dexterity}}

\normalsize
You can strike a foe with an arrow or bullet no matter where they are in the battlefield.
\paragraph{REQUIREMENTS} Bow Adept 2 or Musket Adept 2.
\paragraph{RANK 1} If you do not move during your turn, you can make an additional ranged weapon attack as a bonus action.
\paragraph{RANK 2} Both your normal and long ranges are multiplied by 1.5 for all ranged weapons..
\paragraph{RANK 3} Whenever you have advantage on a ranged attack roll using Dexterity, you can reroll one of the dice once.

% ============================================================================== %
\subsubsection{Shield Master} \label{feat::shieldmaster}
\small{\textcolor{gray}{Strength or Dexterity}}

\normalsize
You know that fighting to the death leads to a short life.
You trust your shield with your life, and you are an expert at avoiding death.
\paragraph{REQUIREMENTS} Shield Training 2.
\paragraph{RANK 1} If you aren't incapacitated, you can add your shield's AC bonus to any Dexterity saving throw you make against a spell or other harmful effect that targets only you.
\paragraph{RANK 2} If you are subjected to an effect that allows you to make a Dexterity saving throw to take only half damage, you can use your reaction to take no damage if you succeed on the saving throw, interposing your shield between yourself and the source of the effect.
\paragraph{RANK 3} When you use the Shove action or the Push technique with your shield, you can choose to force the Prone condition on your target instead of pushing it.

% ============================================================================== %
\subsubsection{Shield Training} \label{feat::shieldtraining}
\small{\textcolor{gray}{Strength or Dexterity}}

\normalsize
Your prefer to live a long life, and choose to carry your trusty shield with you at all times.
\paragraph{RANK 1} You gain proficiency with medium shields.
\paragraph{RANK 2} In combat, you can don or doff a medium shield as a bonus action instead of an action.
\paragraph{RANK 3} You learn the Push technique.

% ============================================================================== %
\subsubsection{Shooter} \label{feat::shooter}
\small{\textcolor{gray}{Dexterity}}

\normalsize
You have mastered ranged weapons and can make shots that others find impossible.
\paragraph{REQUIREMENTS} Proficiency with at least two martial ranged weapon types.
\paragraph{RANK 1} Your ranged attacks ignore half cover and three-quarters covers.
\paragraph{RANK 2} Increase your Dexterity by 1, to a maximum of 20.
\paragraph{RANK 3} You learn the Leg Shot technique.

% ============================================================================== %
\subsubsection{Spear Adept} \label{feat::spearadept}
\small{\textcolor{gray}{Strength or Dexterity}}

\normalsize
Trained in the art of the spear, you know that nothing beats a long pole with a pointy end.
\paragraph{REQUIREMENTS} Armed Fighter 2.
\paragraph{RANK 1} You are proficient with spears.
\paragraph{RANK 2} Spears have the finesse property for you.
\paragraph{RANK 3} You learn the Feint technique.

% ============================================================================== %
\subsubsection{Staff Fighter} \label{feat::stafffighter}
\small{\textcolor{gray}{Dexterity}}

\normalsize
You have mastered the nuances of fighting with a staff.
\paragraph{REQUIREMENTS} Armed Fighter 1.
\paragraph{RANK 1} The staff counts as a finesse weapon for you.
\paragraph{RANK 2} While wielding a staff, you can use your bonus action to give yourself advantage on your next Dexterity (Acrobatics) or Strength (Athletics) check related to climbing, jumping, or otherwise bypassing obstacles.
This benefit lasts until the end of your turn.
\paragraph{RANK 3} You learn the Sweep technique.

% ============================================================================== %
\subsubsection{Stone Piercer} \label{feat::stonepiercer}
\small{\textcolor{gray}{Strength}}

\normalsize
You know that the difference between piercing stone and crushing bone lies only in the morals of the aggressor.
\paragraph{REQUIREMENTS} Pick Adept 2 and Athlete 2.
\paragraph{RANK 1} You double your damage dice against objects and structures while attacking with a pick.
Additionally, you have advantage on checks made with a climber's kit and pitons.
\paragraph{RANK 2} When you score a critical hit that deals piercing damage to a creature, you can roll one additional damage die when determining the extra piercing damage the target takes.
\paragraph{RANK 3} You learn the Injure technique.

% ============================================================================== %
\subsubsection{Sword Master} \label{feat::swordmaster}
\small{\textcolor{gray}{Strength or Dexterity}}

\normalsize
You have an insight unique to those who master the many types of swords.
Your skill with the long blades is unparalleled, and your resoluteness in combat scares even the bravest of warriors.
\paragraph{REQUIREMENTS} Straight Sword Adept 2, Curved Sword Adept 2, and Light Sword Adept 2.
\paragraph{RANK 1} You gain a +1 bonus to attack rolls you make with swords.
\paragraph{RANK 2} When you make an opportunity attack with a sword, you have advantage on the attack roll.

% ============================================================================== %
\subsubsection{Thrown Weapon Master} \label{feat::thrownweaponmaster}
\small{\textcolor{gray}{Strength or Dexterity}}

\normalsize
Born with quick fingers and good aim, you're an expert at piercing your enemies with thrown objects.
\paragraph{REQUIREMENTS} Armed Fighter 1.
\paragraph{RANK 1} Attacking at long range doesn't impose disadvantage on your thrown weapon attack rolls.
\paragraph{RANK 2} When you throw a thrown weapon as an attack roll, you can draw another thrown weapon freely.
\paragraph{RANK 3} You learn the Quick Draw technique.

% ============================================================================== %
\subsubsection{Two-weapon Master} \label{feat::twoweaponmaster}
\small{\textcolor{gray}{Dexterity}}

\normalsize
Your ability to wield two weapons is unparalleled, and you can surprise any foe with your quick steps and unrelenting strikes.
\paragraph{REQUIREMENTS} Dexterity 16. Dual Wielder 2.
\paragraph{RANK 1} You can use use two-weapon fighting even when the one-handed melee weapons you are wielding aren't light.
\paragraph{RANK 2} You gain a +1 bonus to AC while you are wielding a separate melee weapon in each hand.
\paragraph{RANK 3} You can add your ability modifier to the damage of the bonus attack of your two-weapon fighting action.

% ============================================================================== %
\subsubsection{Unarmed Artist} \label{feat::unarmedartist}
\small{\textcolor{gray}{Dexterity}}

\normalsize
You are a painter.
Your hands are your brush, your opponent your canvas.
\paragraph{RANK 1} Your unarmed strike uses a d4 + your Dexterity modifier for damage.
You can only use this ability if you're not using a fist weapon.
\paragraph{RANK 2} When you use the Attack action, you can make an unarmed strike as a bonus action.
\paragraph{RANK 3} You get access to one Martial Arts technique of your choice.
You may buy this rank again to attain new Martial Arts techniques as long as you have not learned all available techniques.

% ============================================================================== %
\subsubsection{Unarmored} \label{feat::unarmored}
\small{\textcolor{gray}{Dexterity}}

\normalsize
Your extensive physical training allows to forfeit armor when fighting.
Your body is the only armor you need.
\paragraph{REQUIREMENTS} Acrobat 2 and Athlete 2.
\paragraph{RANK 1} As long as you are not wearing any armor, your speed increases by 3 meters.
\paragraph{RANK 2} If you are subjected to an effect that allows you to make a Dexterity saving throw to take only half damage, you can use your reaction to take no damage if you succeed on the saving throw, and half damage if you fail.
You only get this bonus while wearing no armor or light armor.
\paragraph{RANK 3} You learn the Rush technique.

% ============================================================================== %
\subsubsection{Way of the Fist} \label{feat::wayofthefist}
\small{\textcolor{gray}{Strength or Dexterity}}

\normalsize
Either by meditative training or drunken fighting, you are a master with your fists.
\paragraph{REQUIREMENT} Pugilist 2 and Unarmed Artist 2.
\paragraph{RANK 1} Your unarmed strike uses a d6 + your Strength or Dexterity modifier for damage.

% ============================================================================== %
\subsubsection{Whip Adept} \label{feat::whipadept}
\small{\textcolor{gray}{Dexterity}}

\normalsize
You are learned in the use of the whip, using your unique weapon for both combat and support.
\paragraph{REQUIREMENTS} Armed Fighter 2.
\paragraph{RANK 1} You can use your bonus action to make one melee attack with an equipped whip.
\paragraph{RANK 2} You can use a whip to extend the range of the Grapple and Shove actions to the range of the weapon.
\paragraph{RANK 3} You can use your whip as an elongated appendage to perform actions that do not require fine motor skills.
This includes object interactions, such as grabbing a sword or pulling a bag towards you, and ability checks, such as an acrobatics roll to grab a nearby support beam to swing from.

% ============================================================================== %
\subsubsection{Wrestler} \label{feat::wrestler}
\small{\textcolor{gray}{Strength}}

\normalsize
Every part of your body is a weapon and you wield it expertly.
\paragraph{REQUIREMENTS} Grappler 2 and Combat Improviser 2.
\paragraph{RANK 1} Once per round, as part of your action, you many deal your Strength modifier in damage to any creature you are grappling.
\paragraph{RANK 2} You can use a grappled creature of a size equal or lower to yours as an improvised weapon.
\paragraph{RANK 3} You learn the Charge technique.

\subsubsection{Sniper} \label{feat::sniper}
\paragraph{RANK 3} You learn the Hemorrhaging Attack technique.

\paragraph{RANK 3} You learn the Armor Puncture technique.

% TODO: Raise a shield as an action/reaction, increasing its AC bonus by +2.
% TODO: Take cover behind a heavy shield, giving 1/2 or 3/4 cover.
% TODO: For monastic combat:
% \subsubsection{Light as a Feather} \label{mtec::lightasafeather}
% You gain the ability to move across liquids on your turn without falling during the move.

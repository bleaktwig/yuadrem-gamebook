% !TEX root = ../main.tex
\addcontentsline{toc}{section}{Combat Feats}
\subsection*{Combat Feats}
For simplicity in searching, combat feats are separated into broad categories.
These categories pertain to what the feats involve, and are: Armor, Shield, Simple Weapons, Martial Weapons, Special Weapons, Damage Types, and Fighting Styles.

Special weapons are simple or martial weapons that require special training to be used effectively, and thus they have feats of their own to reflect this.

% TODO. Add proficiency with spears.

\begin{DndTable}[width=\linewidth, header=Armor]{ll}
    Unarmored    & \textbf{Fortitude of Body} (page \pageref{feat::fortitudeofbody}) \\
    Unarmored    & \textbf{Unarmored Agility} (page \pageref{feat::unarmoredagility}) \\
    Unarmored    & \textbf{Unarmored Master} (page \pageref{feat::unarmoredmaster}) \\
    Unarmored    & \textbf{Uncanny Dodge} (page \pageref{feat::uncannydodge}) \\

    \textbf{Light Armor} & \textbf{Lightly Armored} (page \pageref{feat::lightlyarmored}) \\
    Light Armor  & \textbf{Light Armor Master} (page \pageref{feat::lightarmormaster}) \\
    Light Armor  & \textbf{Light on your Feet} (page \pageref{feat::lightonyourfeet}) \\
    Light Armor  & \textbf{Uncanny Dodge} (page \pageref{feat::uncannydodge}) \\
    Light Armor  & \textbf{Uncertain Location} (page \pageref{feat::uncertainlocation}) \\

    \textbf{Medium Armor} & \textbf{Moderately Armored} (page \pageref{feat::moderatelyarmored}) \\
    Medium Armor & \textbf{Agile in a Suit} (page \pageref{feat::agileinasuit}) \\
    Medium Armor & \textbf{Balanced Defense} (page \pageref{feat::balanceddefense}) \\
    Medium Armor & \textbf{Silent Despite Everything} (page \pageref{feat::silentdespiteeverything}) \\
    Medium Armor & \textbf{Vital Protection} (page \pageref{feat::vitalprotection}) \\

    \textbf{Heavy Armor} & \textbf{Heavily Armored} (page \pageref{feat::heavilyarmored}) \\
    Heavy Armor  & \textbf{Heavy Armor Master} (page \pageref{feat::heavyarmormaster}) \\
    Heavy Armor  & \textbf{Heavy Weight} (page \pageref{feat::heavyweight}) \\
    Heavy Armor  & \textbf{Imposing Figure} (page \pageref{feat::imposingfigure}) \\
    Heavy Armor  & \textbf{Tough} (page \pageref{feat::tough})
\end{DndTable}
\begin{DndTable}[width=\linewidth, header=Shields]{ll}
    \textbf{Small Shields} & \textbf{Buckler Training} (page \pageref{feat::bucklertraining}) \\
    Small Shields  & \textbf{Defensive Duelist} (page \pageref{feat::defensiveduelist}) \\
    Small Shields  & \textbf{Quick Shielding} (page \pageref{feat::quickshielding}) \\
    Small Shields  & \textbf{Strapped and Secured} (page \pageref{feat::strappedandsecured}) \\
    Small Shields  & \textbf{Swift Deflection} (page \pageref{feat::swiftdeflection}) \\

    \textbf{Medium Shields} & \textbf{Shield Training} (page \pageref{feat::shieldtraining}) \\
    Medium Shields & \textbf{Blade and Board} (page \pageref{feat::bladeandboard}) \\
    Medium Shields & \textbf{Shield Master} (page \pageref{feat::shieldmaster}) \\
    Medium Shields & \textbf{Stalwart Shield} (page \pageref{feat::stalwartshield}) \\
    Medium Shields & \textbf{The Best Defense} (page \pageref{feat::thebestdefense}) \\

    \textbf{Heavy Shields} & \textbf{Heavy Shield Training} (page \pageref{feat::heavyshieldtraining}) \\
    Heavy Shields  & \textbf{Bulwark Training} (page \pageref{feat::bulwarktraining}) \\
    Heavy Shields  & \textbf{Consistent Defense} (page \pageref{feat::consistentdefense}) \\
    Heavy Shields  & \textbf{Immovable Object} (page \pageref{feat::immovableobject}) \\
    Heavy Shields  & \textbf{Take Cover} (page \pageref{feat::takecover})
\end{DndTable}
\begin{DndTable}[width=\linewidth, header=Simple Weapons]{ll}
    \textbf{Simple} \textbf{Weapons} & \textbf{Simple Fighter} (page \pageref{feat::simplefighter}) \\
    Simple Weapons & \textbf{Adaptable Fighter} (page \pageref{feat::adaptablefighter}) \\
    Simple Weapons & \textbf{Combat Improviser} (page \pageref{feat::combatimproviser}) \\
    Simple Weapons & \textbf{Cutthroat} (page \pageref{feat::cutthroat}) \\
    Simple Weapons & \textbf{Dexterous Hand} (page \pageref{feat::dexteroushand}) \\
    Simple Weapons & \textbf{Quick Action} (page \pageref{feat::quickaction}) \\
    Simple Weapons & \textbf{Solid Start} (page \pageref{feat::solidstart})
\end{DndTable}
\begin{DndTable}[width=\linewidth, header=Martial Weapons]{ll}
    \textbf{Axes}   & \textbf{Axemaster} (page \pageref{feat::axemaster}) \\
    Axes            & \textbf{Forceful Disarm} (page \pageref{feat::forcefuldisarm}) \\
    Axes            & \textbf{Griefer} (page \pageref{feat::griefer}) \\
    Axes            & \textbf{Maiming Strikes} (page \pageref{feat::maimingstrikes}) \\
    Axes            & \textbf{Murderous Intent} (page \pageref{feat::murderousintent}) \\
    Axes            & \textbf{Rush} (page \pageref{feat::rush}) \\
    Axes            & \textbf{Tree Feller} (page \pageref{feat::treefeller}) \\

    \textbf{Bows}   & \textbf{Bowmaster} (page \pageref{feat::bowmaster}) \\
    Bows            & \textbf{Covered Shooter} (page \pageref{feat::coveredshooter}) \\
    Bows            & \textbf{Horde Breaker} (page \pageref{feat::hordebreaker}) \\
    Bows            & \textbf{Quick Shot} (page \pageref{feat::quickshot}) \\
    Bows            & \textbf{Still Aim} (page \pageref{feat::stillaim}) \\
    Bows            & \textbf{Sudden Attack} (page \pageref{feat::suddenattack}) \\
    Bows            & \textbf{Volley} (page \pageref{feat::volley}) \\

    \textbf{Crossbows} & \textbf{Crossbow Enthusiast} (page \pageref{feat::crossbowenthusiast}) \\
    Crossbows       & \textbf{Modern Shooter} (page \pageref{feat::modernshooter}) \\
    Crossbows       & \textbf{Prepared Shot} (page \pageref{feat::preparedshot}) \\
    Crossbows       & \textbf{Puncture Wound} (page \pageref{feat::puncturewound}) \\
    Crossbows       & \textbf{Sudden Attack} (page \pageref{feat::suddenattack}) \\
    Crossbows       & \textbf{Quick Loading} (page \pageref{feat::quickloading}) \\
    Crossbows       & \textbf{Visceral Attack} (page \pageref{feat::visceralattack}) \\

    \textbf{Curved Swords} & \textbf{Bent Blade Master} (page \pageref{feat::bentblademaster}) \\
    Curved Swords   & \textbf{Graceful Disarm} (page \pageref{feat::gracefuldisarm}) \\
    Curved Swords   & \textbf{Impairing Attack} (page \pageref{feat::impairingattack}) \\
    Curved Swords   & \textbf{Minced Meat} (page \pageref{feat::mincedmeat}) \\
    Curved Swords   & \textbf{Parrying Stance} (page \pageref{feat::parryingstance}) \\
    Curved Swords   & \textbf{Shield Sweep} (page \pageref{feat::shieldsweep}) \\
    Curved Swords   & \textbf{Whirlwind Attack} (page \pageref{feat::whirlwindattack}) \\

    \textbf{Hammers} & \textbf{Blunt Fighter} (page \pageref{feat::bluntfighter}) \\
    Hammers         & \textbf{Armor Breaker} (page \pageref{feat::armorbreaker}) \\
    Hammers         & \textbf{Bend Metal} (page \pageref{feat::bendmetal}) \\
    Hammers         & \textbf{Rush} (page \pageref{feat::rush}) \\
    Hammers         & \textbf{Heavy Hitter} (page \pageref{feat::heavyhitter}) \\
    Hammers         & \textbf{Lower that Guard!} (page \pageref{feat::lowerthatguard}) \\
    Hammers         & \textbf{Mashed Brain} (page \pageref{feat::mashedbrain}) \\

    \textbf{Polearms} & \textbf{Polearm Master} (page \pageref{feat::polearmmaster}) \\
    Polearms        & \textbf{Far Thrusts} (page \pageref{feat::farthrusts}) \\
    Polearms        & \textbf{Not on my Watch} (page \pageref{feat::notonmywatch}) \\
    Polearms        & \textbf{Rondel Hit} (page \pageref{feat::rondelhit}) \\
    Polearms        & \textbf{Sweep} (page \pageref{feat::sweep}) \\
    Polearms        & \textbf{Trip} (page \pageref{feat::trip}) \\
    Polearms        & \textbf{Unapproachable} (page \pageref{feat::unapproachable}) \\

    \textbf{Rapiers} & \textbf{Fencer} (page \pageref{feat::fencer}) \\
    Rapiers         & \textbf{Appropiate Response} (page \pageref{feat::appropiateresponse}) \\
    Rapiers         & \textbf{Dignified Miss} (page \pageref{feat::dignifiedmiss}) \\
    Rapiers         & \textbf{Elegant Striker} (page \pageref{feat::elegantstriker}) \\
    Rapiers         & \textbf{Focus Attacks} (page \pageref{feat::focusattacks}) \\
    Rapiers         & \textbf{Impairing Attack} (page \pageref{feat::impairingattack}) \\

    \textbf{Spears} & \textbf{Spearmaster} (page \pageref{feat::spearmaster}) \\
    Spears          & \textbf{Charge Stopper} (page \pageref{feat::chargestopper}) \\
    Spears          & \textbf{Far Thrusts} (page \pageref{feat::farthrusts}) \\
    Spears          & \textbf{Staff Fighter} (page \pageref{feat::stafffighter}) \\
    Spears          & \textbf{Swordbreaker} (page \pageref{feat::swordbreaker}) \\
    Spears          & \textbf{Subtle Thrusts} (page \pageref{feat::subtlethrusts}) \\
    Spears          & \textbf{Unapproachable} (page \pageref{feat::unapproachable}) \\

    \textbf{Straight Swords} & \textbf{Swordmaster} (page \pageref{feat::swordmaster}) \\
    Straight Swords & \textbf{Classic Defense} (page \pageref{feat::classicdefense}) \\
    Straight Swords & \textbf{Deflection} (page \pageref{feat::deflection}) \\
    Straight Swords & \textbf{Frenzied Slashes} (page \pageref{feat::frenziedslashes}) \\
    Straight Swords & \textbf{Parrying Atance} (page \pageref{feat::parryingstance}) \\
    Straight Swords & \textbf{Revenant Blade} (page \pageref{feat::revenantblade}) \\
    Straight Swords & \textbf{Stalwart Stance} (page \pageref{feat::stalwartstance})
\end{DndTable}
\begin{DndTable}[width=\linewidth, header=Special Weapons]{ll}
    \textbf{Blowguns} & \textbf{Blowgun Master} (page \pageref{feat::blowgunmaster}) \\
    Blowguns & \textbf{Hidden Shooter} (page \pageref{feat::hiddenshooter}) \\
    Blowguns & \textbf{Poisoned Dart} (page \pageref{feat::poisoneddart}) \\
    Blowguns & \textbf{Strong Lungs} (page \pageref{feat::stronglungs}) \\

    \textbf{Flails} & \textbf{Flail Adept} (page \pageref{feat::flailadept}) \\
    Flails   & \textbf{Down You Go} (page \pageref{feat::downyougo}) \\
    Flails   & \textbf{Shield Sweep} (page \pageref{feat::shieldsweep}) \\
    Flails   & \textbf{Steel Resolve} (page \pageref{feat::steelresolve}) \\

    \textbf{Nets} & \textbf{Snatcher} (page \pageref{feat::snatcher}) \\
    Nets     & \textbf{Fisher} (page \pageref{feat::fisher}) \\
    Nets     & \textbf{Reel} (page \pageref{feat::reel}) \\
    Nets     & \textbf{Strengthened Nets} (page \pageref{feat::strengthenednets}) \\

    \textbf{Whips} & \textbf{Whip Master} (page \pageref{feat::whipmaster}) \\
    Whips    & \textbf{Extended Reach} (page \pageref{feat::extendedreach}) \\
    Whips    & \textbf{Flagellation} (page \pageref{feat::flagellation}) \\
    Whips    & \textbf{Wrap Around} (page \pageref{feat::wraparound})
\end{DndTable}
\begin{DndTable}[width=\linewidth, header=Damage Types]{ll}
    Bludgeoning & \textbf{Bonk} (page \pageref{feat::bonk}) \\
    Bludgeoning & \textbf{Crusher} (page \pageref{feat::crusher}) \\
    Bludgeoning & \textbf{Fell Hand} (page \pageref{feat::fellhand}) \\
    Bludgeoning & \textbf{Off-balance} (page \pageref{feat::offbalance}) \\

    Piercing    & \textbf{Critical Injury} (page \pageref{feat::criticalinjury}) \\
    Piercing    & \textbf{Effective Puncture} (page \pageref{feat::effectivepuncture}) \\
    Piercing    & \textbf{Piercer} (page \pageref{feat::piercer}) \\
    Piercing    & \textbf{Stab-a-Lung!} (page \pageref{feat::stabalung}) \\

    Slashing    & \textbf{Aimed Cut} (page \pageref{feat::aimedcut}) \\
    Slashing    & \textbf{Quick Slashes} (page \pageref{feat::quickslashes}) \\
    Slashing    & \textbf{Slasher} (page \pageref{feat::slasher}) \\
    Slashing    & \textbf{Sustained Impetus} (page \pageref{feat::sustainedimpetus})
\end{DndTable}
\begin{DndTable}[width=\linewidth, header=Fighting Styles]{ll}
    ---                    & \textbf{Fighting Initiate} (page \pageref{feat::fightinginitiate}) \\

    Archery                & \textbf{Bullseye} (page \pageref{feat::bullseye}) \\
    Archery                & \textbf{Mark} (page \pageref{feat::mark}) \\
    Archery                & \textbf{Reckless Shot} (page \pageref{feat::recklessshot}) \\
    Archery                & \textbf{Sharpshooter} (page \pageref{feat::sharpshooter}) \\
    Archery                & \textbf{Shooter} (page \pageref{feat::shooter}) \\
    Archery                & \textbf{Supernatural Accuracy} (page \pageref{feat::supernaturalaccuracy}) \\

    Battle Mastery         & \textbf{Action Surge} (page \pageref{feat::actionsurge}) \\
    Battle Mastery         & \textbf{Assess the Situation} (page \pageref{feat::assessthesituation}) \\
    Battle Mastery         & \textbf{Commander} (page \pageref{feat::commander}) \\
    Battle Mastery         & \textbf{Dynamic Fighter} (page \pageref{feat::dynamicfighter}) \\
    Battle Mastery         & \textbf{Mage Slayer} (page \pageref{feat::mageslayer}) \\
    Battle Mastery         & \textbf{Second Wind} (page \pageref{feat::secondwind}) \\

    Dueling                & \textbf{Bastion} (page \pageref{feat::bastion}) \\
    Dueling                & \textbf{Execute} (page \pageref{feat::execute}) \\
    Dueling                & \textbf{Fighters Resilience} (page \pageref{feat::fightersresilience}) \\
    Dueling                & \textbf{Heightened Senses} (page \pageref{feat::heightenedsenses}) \\
    Dueling                & \textbf{Improved Critical} (page \pageref{feat::improvedcritical}) \\
    Dueling                & \textbf{Purposeful Strike} (page \pageref{feat::purposefulstrike}) \\

    Great Weapon Fighting  & \textbf{Cleaving} (page \pageref{feat::cleaving}) \\
    Great Weapon Fighting  & \textbf{Improved Critical} (page \pageref{feat::improvedcritical}) \\
    Great Weapon Fighting  & \textbf{Reckless Attack} (page \pageref{feat::recklessattack}) \\
    Great Weapon Fighting  & \textbf{Savage Attacker} (page \pageref{feat::savageattacker}) \\
    Great Weapon Fighting  & \textbf{Superior Critical} (page \pageref{feat::superiorcritical}) \\
    Great Weapon Fighting  & \textbf{Taken Aback} (page \pageref{feat::takenaback}) \\

    Monastic Fighting      & \textbf{Agile Grappler} (page \pageref{feat::agilegrappler}) \\
    Monastic Fighting      & \textbf{Blind Fighter} (page \pageref{feat::blindfighter}) \\
    Monastic Fighting      & \textbf{Flurry of Blows} (page \pageref{feat::flurryofblows}) \\
    Monastic Fighting      & \textbf{Martial Artist} (page \pageref{feat::martialartist}) \\
    Monastic Fighting      & \textbf{Monkey's Fist} (page \pageref{feat::monkeysfist}) \\
    Monastic Fighting      & \textbf{Quickened Healing} (page \pageref{feat::quickenedhealing}) \\

    Mounted Fighting       & \textbf{Charger} (page \pageref{feat::charger}) \\
    Mounted Fighting       & \textbf{Height Superiority} (page \pageref{feat::heightsuperiority}) \\
    Mounted Fighting       & \textbf{Protected Mount} (page \pageref{feat::protectedmount}) \\
    Mounted Fighting       & \textbf{Strike a Pose} (page \pageref{feat::strikeapose}) \\
    Mounted Fighting       & \textbf{Trained Mount} (page \pageref{feat::trainedmount}) \\
    Mounted Fighting       & \textbf{Trample} (page \pageref{feat::trample}) \\

    Protection             & \textbf{Aura of Protection} (page \pageref{feat::auraofprotection}) \\
    Protection             & \textbf{Courageous Visage} (page \pageref{feat::courageousvisage}) \\
    Protection             & \textbf{Durable} (page \pageref{feat::durable}) \\
    Protection             & \textbf{Interception} (page \pageref{feat::interception}) \\
    Protection             & \textbf{Protector} (page \pageref{feat::protector}) \\
    Protection             & \textbf{Sentinel} (page \pageref{feat::sentinel}) \\

    Thrown Weapon Fighting & \textbf{Bullseye} (page \pageref{feat::bullseye}) \\
    Thrown Weapon Fighting & \textbf{Deflect Missiles} (page \pageref{feat::deflectmissiles}) \\
    Thrown Weapon Fighting & \textbf{Fast Fingers} (page \pageref{feat::fastfingers}) \\
    Thrown Weapon Fighting & \textbf{Opportunist} (page \pageref{feat::opportunist}) \\
    Thrown Weapon Fighting & \textbf{Quick Draw} (page \pageref{feat::quickdraw}) \\
    Thrown Weapon Fighting & \textbf{Throwing Arm} (page \pageref{feat::throwingarm}) \\

    Two-Weapon Fighting    & \textbf{Crowd Breaker} (page \pageref{feat::crowdbreaker}) \\
    Two-Weapon Fighting    & \textbf{Dual Wielder} (page \pageref{feat::dualwielder}) \\
    Two-Weapon Fighting    & \textbf{Lock} (page \pageref{feat::lock}) \\
    Two-Weapon Fighting    & \textbf{Offense and Defense} (page \pageref{feat::offenseanddefense}) \\
    Two-Weapon Fighting    & \textbf{Ready for Action} (page \pageref{feat::readyforaction}) \\
    Two-Weapon Fighting    & \textbf{Two-weapon Master} (page \pageref{feat::twoweaponmaster}) \\

    Unarmed Fighting       & \textbf{Charger} (page \pageref{feat::charger}) \\
    Unarmed Fighting       & \textbf{Grappler} (page \pageref{feat::grappler}) \\
    Unarmed Fighting       & \textbf{Meat Hammer} (page \pageref{feat::meathammer}) \\
    Unarmed Fighting       & \textbf{Meat Shield} (page \pageref{feat::meatshield}) \\
    Unarmed Fighting       & \textbf{Pugilist} (page \pageref{feat::pugilist}) \\
    Unarmed Fighting       & \textbf{Unarmed Artist} (page \pageref{feat::unarmedartist})
\end{DndTable}


\subsubsection{Action Surge (2 FP)} \label{feat::actionsurge}
    Thanks to rigorous martial training, you can push yourself beyond your normal limits for a limited time.
    On your turn, you can take 3 additional actions.
    In addition, you can ignore the multiple attack penalty during this turn.
    \paragraph{Requirements} Fighting Style: Battle Mastery 1.
\subsubsection{Adaptable Fighter} \label{feat::adaptablefighter}
    You learn one action of your choice between the Aim, Block, Distracting Strike, Parry, Push, Reckless Attack, Reckless Shot, Riposte, and Steal actions.
    \paragraph{Requirements} Skilled proficiency with Simple Weapons.
\subsubsection{Agile Grappler} \label{feat::agilegrappler}
    You can take this feat three times, gaining different benefits each time:
    \begin{itemize}
        \item When you take the Grapple action, you can use a monastic die to roll on the contest using your Dexterity (Acrobatics) instead of your Strength (Athletics), adding the result of the die to the check.
        \item When you attempt to grapple a creature, you choose what the creature rolls in the contest between Strength (Athletics) and Dexterity (Acrobatics).
        \item When you successfully grapple a creature, you can use another monastic die to make one melee attack against the creature as a free action, adding the result of the die to the damage roll.
    \end{itemize}
    \paragraph{Requirements} Fighting Style: Monastic Fighting 1.
\subsubsection{Agile in a Suit} \label{feat::agileinasuit}
    When you wear medium armor, you can add 3, rather than 2, to your AC if you have a Dexterity of 16 or higher.
    \paragraph{Requirements} Proficiency with Medium Armor.
\subsubsection{Aimed Cut} \label{feat::aimedcut}
    You can take this feat three times, learning a different specialized cut each time:
    \begin{itemize}
        \item The first time you take this feat, you learn how to perform a leg cut.
        When you hit a creature with an attack that deals slashing damage, you halve the creature's movement speed until the start of your next turn.
        \item The second time, you learn how to perform an arm cut.
        Pick one of the creature's arms.
        You aim for that arm when you make an attack that deals slashing damage, and all attacks made with that hand are made with disadvantage until the start of your next turn.
        \item The third time, you learn how to cause a hemorrhage.
        For the next minute after being hit by your attack with a slashing weapon, the creature takes 1d4 slashing damage at the start of each of its turns.
        It can staunch this wound using two actions, ending the effect early.
        This effect is not cummulative.
    \end{itemize}

    You can only perform any of these cuts once per turn.
\subsubsection{Appropiate Response (2 FP)} \label{feat::appropiateresponse}
    You learn the Riposte reaction (see page \pageref{act::riposte}).

    In addition, when you use this reaction while wielding a rapier, you attack twice instead of once.
    \paragraph{Requirements} Skilled proficiency with Rapiers.
\subsubsection{Armor Breaker} \label{feat::armorbreaker}
    You add a +2 bonus on your attack damage made with hammers against creatures wearing armor with the Chain or Plate properties.
    \paragraph{Requirements} Competent proficiency with Hammers.
\subsubsection{Assess the Situation} \label{feat::assessthesituation}
    Right after rolling initiative, you can quickly assess a creature of your choice.
    Make an Intelligence (Investigation) check contested by a Charisma (Deception) check made by the creature.
    On a success, you learn the creature's vulnerabilities, resistances, and immunities, if it has any.

    You can take this feat three times, increasing the number of creature you can use this ability with by 1 the second and third times.
    \paragraph{Requirements} Fighting Style: Battle Mastery 1.
\subsubsection{Aura of Protection} \label{feat::auraofprotection}
    Allies around you are emboldened by your presence.
    Whenever a friendly creature (excluding you) within 2 meters of you must make a saving throw, the creature gains a bonus to the saving throw equal to your Charisma modifier (with a minimum bonus of +1).
    You must be conscious to grant this bonus.
    \paragraph{Requirements} Fighting Style: Protection 2.
\subsubsection{Axemaster} \label{feat::axemaster}
    More than a simple tool, you master the use of axes in combat.

    You increase your proficiency level with axes.
    This feat can be taken three times, or until you reach Expert proficiency with the weapon type.
\subsubsection{Balanced Defense (2 FP)} \label{feat::balanceddefense}
    While wearing medium armor, you can roll Dexterity and Strength saving throws using your proficiency on Strength or Dexterity saving throws respectively.
    \paragraph{Requirements} Proficiency with Medium Armor.
\subsubsection{Bend Metal (2 FP)} \label{feat::bendmetal}
    When you hit a creature with a hammer, you can choose to not damage it; instead focusing in its armor.
    If the creature is wearing armor with the Chain, Plate, or Natural property, it applies a -1 penalty to its AC.

    Damaged armor can be repaired by someone Skilled with Smith's tools, and it usually costs a quarter or half of the cost of the armor, to the DM's discretion.
    If you reduce the AC bonus given by the armor to 0, the armor is destroyed irreparably.

    Natural armor damaged by this feat heals over time, regaining its original AC over the course of a long rest.
    \paragraph{Requirements} Expert proficiency with Hammers.
\subsubsection{Bent Blade Master} \label{feat::bentblademaster}
    You increase your proficiency level with curved swords.
    This feat can be taken three times, or until you reach Expert proficiency with the weapon type.
\subsubsection{Blade \& Board} \label{feat::bladeandboard}
    While you're wielding a medium shield and a creature misses you with a melee attack, you can use your reaction to attack it with a melee attack.
    \paragraph{Requirements} Proficiency with Medium Shields.
\subsubsection{Blind Fighter} \label{feat::blindfighter}
    You have blindsight with a range of 2 meters.
    Within that range, you can effectively see anything that isn't behind total cover, even if you're blinded or in darkness.
    Moreover, you can see an invisible creature within that range, unless the creature successfully hides from you.
    \paragraph{Requirements} Fighting Style: Monastic Fighting 2.
\subsubsection{Blowgun Master} \label{feat::blowgunmaster}
    You increase your proficiency level with blowguns.
    This feat can be taken three times, or until you reach Expert proficiency with the weapon type.
\subsubsection{Blunt Fighter} \label{feat::bluntfighter}
    If a hammer will do, why complicate things?

    You increase your proficiency level with hammers.
    This feat can be taken three times, or until you reach Expert proficiency with the weapon type.
\subsubsection{Bonk} \label{feat::bonk}
    Using two actions, you can attempt to hit a creature directly in its head.
    Make a normal attack roll.
    On a successful hit, apart from taking damage, the creature must roll a Constitution saving throw with a DC equal to 8 + your proficiency bonus with the weapon you're wielding + your Strength modifier.
    On a failed save, the creature has disadvantage on all attack rolls and ability checks until the start of your next turn.

    To use this ability, you must be wielding a bludgeoning weapon and be able to reach the creature's head --- or one of them.
\subsubsection{Bowmaster} \label{feat::bowmaster}
    You increase your proficiency level with bows.
    This feat can be taken three times, or until you reach Expert proficiency with the weapon type.
\subsubsection{Bastion} \label{feat::bastion}
    When you are wielding a shield, you gain an additional +1 bonus to your AC.
    \paragraph{Requirements} Fighting Style: Dueling 1.
\subsubsection{Buckler Training} \label{feat::bucklertraining}
    You learn how to use small shields in combat.
\subsubsection{Bullseye} \label{feat::bullseye}
    By focusing your aim, you can choose to cause an additional effect when you hit a creature with a ranged attack.
    You learn one effect when learning this technique, but you can learn additional ones by taking it again:
    \begin{itemize}
        \item \textbf{Leg Shot.} The creature's movement speed is divided by half (rounded up) until the start of your next turn.
        \item \textbf{Hand Shot.} The creature has disatvantage on the first melee attack roll it makes during its next turn.
        \item \textbf{Slayer's Shot.} The creature takes an extra 1d8 damage if its below its hit point maximum.
    \end{itemize}

    You can use one of these effects only once per turn.
    \paragraph{Requirements} Fighting Style: Archery 1 or Thrown Weapons Fighting 1.
\subsubsection{Bulwark Training} \label{feat::bulwarktraining}
    You halve the movement debuff associated to heav shields.
    \paragraph{Requirements} Proficiency with Heavy Shields.
\subsubsection{Charge Stopper} \label{feat::chargestopper}
    You can set your spear to receive a charge.
    As an action, choose a creature you can see that is at least 4 meters away from you.
    If that creature moves within your spear's reach on its next turn, you can make a melee attack against it with your spear as a reaction.
    If the attack hits, the target takes an extra 1d8 piercing damage, or an extra 1d10 piercing damage if you wield the spear with two hands.
    You can't use this ability if the creature used the Disengage action before moving.
    \paragraph{Requirements} Skilled proficiency with spears.
\subsubsection{Charger} \label{feat::charger}
    Mounted or not, if you moved at least 2 meters in a straight line towards a creature during your turn, you have advantage on your next melee attack roll made against the creature.
    In addition, add a +5 to the attack's damage roll on a hit.
    \paragraph{Requirements} Fighting Style: Mounted Fighting 1 or Unarmed Fighting 1.
\subsubsection{Classic Defense} \label{feat::classicdefense}
    When you are wielding a straight sword and no other weapons, you can add a +1 bonus to your AC.
    \paragraph{Requirements} Expert proficiency with Straight Swords.
\subsubsection{Cleaving} \label{feat::cleaving}
    When fighting with a weapon with the heavy property, you can attack two creatures instead of one with a melee attack.
    You can attack in this way only once during your turn.

    You can take this feat two additional times, increasing the number of creatures you can hit with this attack by 1 each time.
    \paragraph{Requirements} Fighting Style: Great Weapon Fighting 1.
\subsubsection{Combat Improviser} \label{feat::combatimproviser}
    If you can grab it, you can swing it.

    You can apply your proficiency modifier with simple weapons to attacks made with improvised weapons.

    In addition, you have advantage on all attack rolls using improvised weapons during the first round of combat.
    \paragraph{Requirements} Competent proficiency with Simple Weapons.
\subsubsection{Commander} \label{feat::commander}
    You have advantage on all Charisma (Deception), Charisma (Intimidation), Charisma (Performance), and Charisma (Persuasion) checks while in combat.
    \paragraph{Requirements} Fighting Style: Battle Mastery 2.
\subsubsection{Consistent Defense} \label{feat::consistentdefense}
    You gain three-fourths cover instead of half cover against ranged attacks while using a heavy shield.
    \paragraph{Requirements} Proficiency with Heavy Shields.
\subsubsection{Courageous Visage} \label{feat::courageousvisage}
    You and friendly creatures within 2 meters of you have advantage on saving throws against being frightened while you are conscious.
    \paragraph{Requirements} Fighting Style: Protection 1.
\subsubsection{Covered Shooter} \label{feat::coveredshooter}
    If you miss with a ranged attack while hidden, the attack doesn't reveal your location, and you remain hidden.
    \paragraph{Requirements} Competent proficiency with Bows.
\subsubsection{Critical Injury} \label{feat::criticalinjury}
    When you score a critical hit that deals piercing damage to a creature, you can roll one additional damage die when determining the extra piercing damage the target takes.
\subsubsection{Crossbow Enthusiast} \label{feat::crossbowenthusiast}
    You increase your proficiency level with crossbows.
    This feat can be taken three times, or until you reach Expert proficiency with the weapon type.
\subsubsection{Crowd Breaker} \label{feat::crowdbreaker}
    Once on each of your turns when you make a melee weapon attack, you can make another attack against a different creature that is within 1 meter of you.

    You can take this feat two additional times.
    The second time you take it, you can make attack rolls against two creatures rather than one when you use this ability.
    The third, you can target three additional creatures.
    \paragraph{Requirements} Fighting Style: Two-Weapon Fighting 1.
\subsubsection{Crusher (2 FP)} \label{feat::crusher}
    Whenever you miss with a melee attack using a bludgeoning weapon, the target of your attack still takes bludgeoning damage equal to your Strength modifier (minimum of 2).

    A creature can take damage from this feat only once per round.
\subsubsection{Cutthroat} \label{feat::cutthroat}
    You have advantage on any check to conceal a small weapon on your person.
    \paragraph{Requirements} Competent proficiency with Simple Weapons.
\subsubsection{Defensive Duelist} \label{feat::defensiveduelist}
    When fighting with a one-handed melee weapon and a small shield, you can add half your shield's AC (rounded up) to the weapon's attack damage.
    \paragraph{Requirements} Proficiency with Small Shields.
\subsubsection{Deflect Missiles} \label{feat::deflectmissiles}
    You can use your reaction to deflect or catch the missile when you are hit by a ranged weapon attack.
    When you do so, the damage you take from the attack is reduced by 1d10 + your Dexterity modifier.

    If you reduce the damage to 0, you can catch the missile if it is small enough for you to hold in one hand and you have at least one hand free.
    On your turn, you can attack with the missile as a ranged weapon attack with an improvised thrown weapon.
    \paragraph{Requirements} Fighting Style: Thrown Weapons Fighting 2.
\subsubsection{Deflection} \label{feat::deflection}
    Whenever you are hit by an attack roll or an effect targeting only you that forces you to make a Dexterity saving throw, you can use your reaction to add your Strength or Dexterity modifier (your choce) to your AC or to the saving throw roll.
    \paragraph{Requirements} Skilled proficiency with Straight Swords.
\subsubsection{Dexterous Hand} \label{feat::dexteroushand}
    You can add the finesse property to any simple melee weapon that does not already have it.
    \paragraph{Requirements} Expert proficiency with Simple Weapons.
\subsubsection{Dignified Miss} \label{feat::dignifiedmiss}
    Whenever you miss with a melee weapon attack, you can use your reaction to make another melee weapon attack.
    The multiple attack penalty applies to this attack normally.
    \paragraph{Requirements} Skilled proficiency with Rapiers.
\subsubsection{Down You Go!} \label{feat::downyougo}
    When you hit a creature with an opportunity attack using a flail, the creature must succeed on a Strength saving throw with a DC equal to 8 + your proficiency bonus with flails + your Strength modifier.
    Upon failure, the creature is knocked prone.
    \paragraph{Requirements} Skilled proficiency with Flails.
\subsubsection{Dual Wielder} \label{feat::dualwielder}
    You can use two-weapon fighting even when the one-handed melee weapons you are wielding aren't light.
    \paragraph{Requirements} Fighting Style: Two-Weapon Fighting 2.
\subsubsection{Durable} \label{feat::durable}
    When you roll a hit die to regain hit points, the minimum number of hit points you can regain from the roll equals 1 + your Constitution modifier (minimum of 2).
    \paragraph{Requirements} Fighting Style: Protection 2.
\subsubsection{Dynamic Fighter} \label{feat::dynamicfighter}
    Adaptable to any weapon, you learn one action of your choice among the Aim, Block, Distracting Strike, Parry, Push, Quick Draw, Reckless Attack, Reckless Shot, and Riposte actions.
    \paragraph{Requirements} Fighting Style: Battle Mastery 2.
\subsubsection{Effective Puncture} \label{feat::effectivepuncture}
    Once per turn, when you hit a creature with an attack that deals piercing damage, you can reroll one of the attack's damage dice, and you must use the new roll.
\subsubsection{Elegant Striker} \label{feat::elegantstriker}
    You increase your proficiency level with rapiers.
    This feat can be taken three times, or until you reach Expert proficiency with the weapon type.
\subsubsection{Execute} \label{feat::execute}
    When a creature is knocked prone within 1 meter of you, you can use your reaction to make a melee attack against it.
    If the attack hits, you deal an additional 2d6 damage of your weapon's damage type.
    \paragraph{Requirements} Fighting Style: Dueling 2.
\subsubsection{Extended Reach} \label{feat::extendedreach}
    You can use your whip as an elongated appendage to perform actions that do not require fine motor skills.
    This includes object interactions, such as grabbing a sword or pulling a bag towards you, and ability checks, such as an acrobatics roll to grab a nearby support beam to swing from.
    \paragraph{Requirements} Skilled proficiency with Whips.
\subsubsection{Far Thrusts} \label{feat::farthrusts}
    As an action, you can extend the reach of a spear or polearm by 1 meter.
    \paragraph{Requirements} Expert proficiency with Polearms or Expert proficiency with spears.
\subsubsection{Fast Fingers (2 FP)} \label{feat::fastfingers}
    You don't attack with disadvantage with ranged attacks when a creature is within 1 meter of you.

    In addition, you don't provoke attacks of opportunity when unsheathing a weapon or attacking with a ranged weapon when you are within 1.5 of a creature.
    \paragraph{Requirements} Fighting Style: Thrown Weapons Fighting 1.
\subsubsection{Fell Hand} \label{feat::fellhand}
    When you score a critical hit that deals bludgeoning damage to a creature, attack rolls against that creature are made with advantage until the start of your next turn.
\subsubsection{Fencer} \label{feat::fencer}
    You gain a +1 bonus to melee attack rolls and your AC when only one creature is within 1 meter of you.
    \paragraph{Requirements} Expert proficiency with Rapiers.
\subsubsection{Fighter's Resilience (2 FP)} \label{feat::fightersresilience}
    If you aren't incapacitated, you can use your reaction to gain advantage on any Constitution saving throw you make.
    \paragraph{Requirements} Fighting Style: Dueling 2.
\subsubsection{Fighting Initiate (2 FP)} \label{feat::fightinginitiate}
    You learn one fighting style of your choice (see page \pageref{ssec::fightingstyles}).
    You can take this feat only once, but you can learn more fighting styles as a major character improvement.
\subsubsection{Fisher} \label{feat::fisher}
    You increase your range with nets to 4/8 meters.
    In addition, you can make an attack with a net more than once per turn.
    \paragraph{Requirements} Competent proficiency with Nets \& Bolas.
\subsubsection{Flagellation (2 FP)} \label{feat::flagellation}
    When you attack a creature with a whip, the next attack made against the creature is made with advantage.

    In addition, when you get a critical attack against a creature with a whip, the creature is stunned until the end of your next turn.
    \paragraph{Requirements} Expert proficiency with Whips.
\subsubsection{Flail Adept} \label{feat::flailadept}
    The flail is a tricky weapon to use, but the benefit that a simple ball and a chain provides is worth it.

    You increase your proficiency level with flails.
    This feat can be taken three times, or until you reach Expert proficiency with the weapon type.
\subsubsection{Flurry of Blows (2 FP)} \label{feat::flurryofblows}
    The multiple attack penalty related to your unarmed weapon attacks is reduced by 3.

    In addition, you can spend one use of your monastic die to make two unarmed weapon attacks instead of one using one action.
    You add the result of the die to the attack roll of both attacks.
    \paragraph{Requirements} Fighting Style: Monastic Fighting 1.
\subsubsection{Focus Attacks (2 FP)} \label{feat::focusattacks}
    Using an action, you can search your opponent for vulnerabilities.
    Choose a creature within 1 meter of you.
    Until the end of your next turn, your melee attack damage with rapiers against that creature gains a bonus equal to your proficiency modifier with rapiers.
    \paragraph{Requirements} Expert proficiency with Rapiers.
\subsubsection{Forceful Disarm} \label{feat::forcefuldisarm}
    You reduce the action cost of the Disarm action to one action.
    If you are holding an axe, you have advantage on your attack roll with the Disarm action to force a creature to drop its shield.
    \paragraph{Requirements} Skilled proficiency with Axes.
\subsubsection{Fortitude of Body} \label{feat::fortitudeofbody}
    As long as you are unarmored, you add a +2 bonus to all Strength, Dexterity, and Constitution saving throws.
\subsubsection{Frenzied Slashes} \label{feat::frenziedslashes}
    When you land a critical hit with a sword on a creature, you can choose to roll damage as normal.
    The creature is frightened of you until the start of your next turn.
    \paragraph{Requirements} Competent proficiency with Straight Swords.
\subsubsection{Graceful Disarm} \label{feat::gracefuldisarm}
    You reduce the action cost of the Disarm action to one action.
    If you are holding a curved sword, you have advantage on your attack roll with the Disarm action to force a creature to drop its shield.
    \paragraph{Requirements} Skilled proficiency with Curved Swords.
\subsubsection{Grappler} \label{feat::grappler}
    You can take this feat three times, learning different effects each time:
    \begin{itemize}
        \item The first time, when you hit a creature with an unarmed strike or an improvised weapon on your turn, you can attempt to grapple the target as a free action.
        \item The second time, you have advantage on attack rolls against creatures you are grappling.
        \item The third, you learn how to pin a creature your are grappling.
        To do so, make another grapple check.
        If you succeed, you and the creature are both restrained until the grapple ends.
    \end{itemize}
    \paragraph{Requirements} Fighting Style: Unarmed Fighting 1.
\subsubsection{Griefer} \label{feat::griefer}
    While wielding an axe, you deal double damage to structures and items.
    In addition, you can cut a rope as one action using your trusty axe.
    \paragraph{Requirements} Competent proficiency with Axes.
\subsubsection{Heavy Armor Master} \label{feat::heavyarmormaster}
    While you are wearing heavy armor, bludgeoning, piercing, and slashing damage that you take is reduced by 1.

    You can take this feat two additional times, increasing this defense by 1 each time.
    \paragraph{Requirements} Proficiency with Heavy Armor.
\subsubsection{Heavy Hitter (2 FP)} \label{feat::heavyhitter}
    You learn the Reckless Attack action (see page \pageref{act::recklessattack}).

    In addition, you can treat any hammer as if it had the Heavy property.
    \paragraph{Requirements} Skilled proficiency with Hammers.
\subsubsection{Heavily Armored} \label{feat::heavilyarmored}
    You learn how to wear heavy armor in combat.
\subsubsection{Heavy Shield Training} \label{feat::heavyshieldtraining}
    Prioritizing a solid defense above everything, you learn how to use heavy shields in combat.
\subsubsection{Heavy-Weight} \label{feat::heavyweight}
    While you are wearing heavy armor, you roll with advantage when you take the Shove action.
    You also roll Strength (Athletics) with advantage when you are targeted by the Shove action.
    \paragraph{Requirements} Proficiency with Heavy Armor.
\subsubsection{Height Superiority (2 FP)} \label{feat::heightsuperiority}
    You have advantage on melee attack rolls against any unmounted creature that is smaller than your mount.
    \paragraph{Requirements} Fighting Style: Mounted Fighting 2.
\subsubsection{Heightened Senses} \label{feat::heightenedsenses}
    At the start of your turn, you can choose to heighten your senses for combat.
    During the next two rounds, you gain a +1 bonus to AC and damage rolls, and you can take one extra action.

    You can use this ability only once between short rests.
    \paragraph{Requirements} Fighting Style: Dueling 2.
\subsubsection{Hidden Shooter} \label{feat::hiddenshooter}
    Making a ranged attack with a blowgun while hidden doesn't reveal your location, even if the attack is a hit.
    \paragraph{Requirements} Competent proficiency with Blowguns.
\subsubsection{Horde Breaker} \label{feat::hordebreaker}
    Once on each of your turns when you make a ranged weapon attack, you can make another attack with the same weapon against a different creature that is within 1 meter of the original target and within range of your weapon.
    \paragraph{Requirements} Skilled proficiency with Bows.
\subsubsection{Immovable Object (2 FP)} \label{feat::immovableobject}
    You are an obstacle.

    Creatures can't use a Tumble action to move through your space.

    In addition, you have advantage on Strength (Athletics) checks to prevent being moved or grappled if you have a heavy shield equipped.
    \paragraph{Requirements} Proficiency with Heavy Shields.
\subsubsection{Impairing Attack} \label{feat::impairingattack}
    During your turn, if you make a melee attack against a creature, that creature can't make opportunity attacks against you for the rest of your turn.
    \paragraph{Requirements} Competent proficiency with Curved Swords or Rapiers.
\subsubsection{Imposing Figure} \label{feat::imposingfigure}
    When you make a Charisma (Intimidation) check, you can add a +5 bonus if you are wearing heavy armor.
    \paragraph{Requirements} Proficiency with Heavy Armor.
\subsubsection{Improved Critical (2 FP)} \label{feat::improvedcritical}
    Your weapon attacks score a critical hit on a roll of 19 or 20.
    \paragraph{Requirements} Fighting Style: Dueling 1 or Great Weapon Fighting 1.
\subsubsection{Interception} \label{feat::interception}
    You can take this feat three times, gaining different benefits each time:
    \begin{itemize}
        \item When a creature you can see attacks a target other than you that is within 1 meter of you, you can use your reaction to impose disadvantage on the attack roll.
        \item If the attack hits the target, the damage the target takes is reduced by 1d10 + 3 (to a minimum of 0 damage).
        \item When you use this ability, you can make a melee weapon attack against the attacking creature.
    \end{itemize}
    You must be wearing a shield or a weapon to use this reaction.
    \paragraph{Requirements} Fighting Style: Protection 1.
\subsubsection{Light Armor Master (2 FP)} \label{feat::lightarmormaster}
    If you aren't incapacitated, you can use your reaction to gain advantage on any Dexterity saving throw you make against a spell or other harmful effect that targets only you.
    In addition, you can use the Dodge action even if your speed is 0.

    To gain the benefits of this feat, you must be wearing light armor.
    \paragraph{Requirements} Proficiency with Light Armor.
\subsubsection{Light on your Feet} \label{feat::lightonyourfeet}
    Equipped light armor and cloth accessories don't add to your encumbrance.
    \paragraph{Requirements} Proficiency with Light Armor.
\subsubsection{Lightly Armored} \label{feat::lightlyarmored}
    You learn how to wear light armor in combat.
\subsubsection{Lock (2 FP)} \label{feat::lock}
    As an action, you can attempt to lock a creature's weapon with yours to prevent it from attacking.
    Using one of your weapons, you try to seize one of your target's weapons by making a lock check, a Dexterity (Sleight of Hand) check contested by the target's Dexterity (Sleight of Hand) or Strength (Athletics) check (the target chooses the ability to use).
    If you succeed, the target can't make attack rolls with the locked weapon.

    At the end of the creature's turns, it can try to end this conditions by succeeding on a Dexterity (Sleight of Hand) or a Strength (Athletics) check contested by your Dexterity (Sleight of Hand) check at the start of its turn.
    The creature can also end the lock at any time by letting go of the weapon.

    In addition, the condition ends if you are incapacitated, if you are removed from the reach of the creature, or if you choose to end the effect (no action required).
    \paragraph{Requirements} Fighting Style: Two-Weapon Fighting 1.
\subsubsection{Lower that Guard!} \label{feat::lowerthatguard}
    If you use the Help action to aid an ally's melee attack while you're wielding a hammer, you knock the target's shield aside momentarily.
    In addition to the ally gaining advantage on the attack roll, the ally gains a bonus to the roll equal to the shield's AC bonus.
    \paragraph{Requirements} Skilled proficiency with Hammers.
\subsubsection{Mage Slayer (2 FP)} \label{feat::mageslayer}
    You have practiced techniques useful in melee combat against spellcaster, gaining the following benefits:
    \begin{itemize}
        \item When a creature within 1 meter of you casts a spell, you can use your reaction to make a melee weapon attack against that creature.
        \item When you damage a creature that is concentrating on a spell, that creature has disadvantage on the saving throw it makes to maintain its concentration.
        \item You have advantage on saving throws against spells cast by creatures within 1 meter of you.
    \end{itemize}
    \paragraph{Requirements} Fighting Style: Battle Mastery 2.
\subsubsection{Maiming Strikes (2 FP)} \label{feat::maimingstrikes}
    When you hit a creature with the attack action, the next attack against that creature before the end of your next turn is rolled with advantage.
    \paragraph{Requirements} Skilled proficiency with Axes.
\subsubsection{Mark} \label{feat::mark}
    You learn the Mark action (see page \pageref{act::mark}).
    \paragraph{Requirements} Fighting Style: Archery 2.
\subsubsection{Martial Artist} \label{feat::martialartist}
    When you take this feat, you learn one of the following abilities.
    You can take this feat three times, learning a different ability each time.
    \begin{itemize}
        \item \textbf{Deflect Missiles.} You can use your reaction to deflect or catch the missile when you are hit by a ranged weapon attack.
        When you do so, the damage you take from the attack is reduced by 1d10 + your Dexterity modifier.

        If you reduce the damage to 0, you can catch the missile if it is small enough for you to hold in one hand and you have at least one hand free.
        If you catch a missile in this way, you can spend 1 monastic die to make a ranged attack (range 4/12 meters) with the weapon or piece of ammunition you just caught, as part of the same reaction.
        You make this attack with proficiency, regardless of your weapon proficiencies.

        \item \textbf{Diamond Soul.} Whenever you make a saving throw and fail, you can spend one monastic die.
        You add a the number rolled to your saving throw, potentially turning a failure into a success.

        \item \textbf{Stunning Strike.} When you hit a creature with an unarmed attack, you can spend two monastic dice to attempt a stunning strike.
        The target must succeed on a Constitution Saving throw with DC equal to 8 + the result of your monastic dice roll or be stunned until the end of your next turn.
    \end{itemize}

    \paragraph{Requirements} Fighting Style: Monastic Fighting 2.
\subsubsection{Mashed Brain} \label{feat::mashedbrain}
    When you land a critical hit with a hammer against a creature, you can choose to roll damage normally.
    If you do this, the creature is stunned until the start of your next turn.
    \paragraph{Requirements} Competent proficiency with Hammers.
\subsubsection{Master Snatcher} \label{feat::snatcher}
    You increase your proficiency level with nets.
    This feat can be taken three times, or until you reach Expert proficiency with the weapon type.
\subsubsection{Meat Hammer (2 FP)} \label{feat::meathammer}
    You can use a grappled creature of a size equal or lower to yours as an improvised weapon.
    Use the stats for the greatclub for attacks with this creature.

    In addition, whenever you successfully hit another creature with a grappled creature, the latter takes half the damage done.
    \paragraph{Requirements} Fighting Style: Unarmed Fighting 2.
\subsubsection{Meat Shield (2 FP)} \label{feat::meatshield}
    You benefit from three-quarters cover while you grapple a creature of a size equal or greater than yours.

    In addition, if a ranged attack misses you while you are grappling a creature, the attack hits the creature you are grappling.
    \paragraph{Requirements} Fighting Style: Unarmed Fighting 1.
\subsubsection{Minced Meat} \label{feat::mincedmeat}
    You add a +2 bonus on your attack damage made with curved swords against unarmored creatures and creatures wearing armor with the Cloth property.
    \paragraph{Requirements} Competent proficiency with Curved Swords.
\subsubsection{Moderately Armored} \label{feat::moderatelyarmored}
    You learn how to wear medium armor in combat.
\subsubsection{Modern Shooter (2 FP)} \label{feat::modernshooter}
    You gain proficiency with firearms, using your proficiency bonus with crossbows for your attack rolls with them.
    \paragraph{Requirements} Skilled proficiency with Crossbows.
\subsubsection{Monkey's Fist} \label{feat::monkeysfist}
    The damage of your unarmed strikes is equal to d4 + your Dexterity modifier instead of the normal damage for an unarmed strike.

    In addition, when you use the attack action on your turn you can expend one use of your monastic die to make an unarmed attack as a free action, adding the result of the die to the damage roll.
    \paragraph{Requirements} Fighting Style: Monastic Fighting 1.
\subsubsection{Murderous Intent (2 FP)} \label{feat::murderousintent}
    When a creature rolls on the minor or major injury charts from one of your attacks made with an axe, it must roll the d20 twice and take the higher result.
    \paragraph{Requirements} Expert proficiency with Axes.
\subsubsection{Not on my Watch!} \label{feat::notonmywatch}
    When a creature within your opportunity attack range makes an attack against a target other than you, you can use your reaction to make a melee weapon attack against the creature.
    \paragraph{Requirements} Competent proficiency with Polearms.
\subsubsection{Offense \& Defense} \label{feat::offenseanddefense}
    You gain a +1 bonus to AC while you are wielding a separate melee weapon in each hand.
    \paragraph{Requirements} Fighting Style: Two-Weapon Fighting 2.
\subsubsection{Off-balance} \label{feat::offbalance}
    Once per turn, when you hit a creature with an attack that deals bludgeoning damage, you can move it 1 meter to an unoccupied space, provided that the target is no more than one size larger than you.

    You can take this feat two additional times, increasing the range you move the creature by 1 meter each time.
\subsubsection{Opportunist (2 FP)} \label{feat::opportunist}
    Whenever a creature is hit by an attack made by another creature other than you within range of your weapon, you can use your reaction to make an opportunity attack with a thrown weapon against it.

    On a hit, the creature has disadvantage on all subsequent attack rolls until the start of its next turn in addition to taking damage.
    \paragraph{Requirements} Fighting Style: Thrown Weapons Fighting 2.
\subsubsection{Parrying Stance (2 FP)} \label{feat::parryingstance}
    You learn the Parry reaction (see page \pageref{act::parry}).

    In addition, when you use the Dodge action you assume a parrying stance.
    When you do so, you can use the Parry reaction without expending your reaction until the start of your next turn.
    \paragraph{Requirements} Skilled proficiency with Curved Swords or Straight Swords.
\subsubsection{Piercer (2 FP)} \label{feat::piercer}
    When a creature rolls on the minor or major injury charts from one of your attacks made with a piercing weapon, it must roll the d20 twice and take the higher result.
\subsubsection{Poisoned Dart} \label{feat::poisoneddart}
    When you hit with a blowgun with a dart that has poison applied, the target has disadvantage on saving throws against the poison.
    \paragraph{Requirements} Skilled proficiency with Blowguns.
\subsubsection{Polearm Master} \label{feat::polearmmaster}
    You increase your proficiency level with polearms.
    This feat can be taken three times, or until you reach Expert proficiency with the weapon type.
\subsubsection{Prepared Shot} \label{feat::preparedshot}
    Right after rolling for initiative, you can don a crossbow or firearm and make one ranged attack with it.
    \paragraph{Requirements} Skilled proficiency with Crossbows.
\subsubsection{Protected Mount} \label{feat::protectedmount}
    While you are mounted and aren't incapacitated, you gain the following effects:
    \begin{itemize}
        \item You can force an attack targeted at your mount to target you instead.
        \item If your mount is subjected to an effect that allows it to make a Dexterity saving throw to take only half damage, it instead takes no damage if it succeeds on the saving throw, and only half damage if it fails.
    \end{itemize}
    \paragraph{Requirements} Fighting Style: Mounted Fighting 2.
\subsubsection{Protector (2 FP)} \label{feat::protector}
    You impose disadvantage on all melee attack rolls made against a friendly creature within 1 meter of you.
    \paragraph{Requirements} Fighting Style: Protection 2.
\subsubsection{Pugilist} \label{feat::pugilist}
    Once per turn when you roll damage for an unarmed melee attack, you can reroll the damage dice and use either total.
    \paragraph{Requirements} Fighting Style: Unarmed Fighting 2.
\subsubsection{Puncture Wound} \label{feat::puncturewound}
    When you get a critical hit on a creature using a ranged weapon, all attacks against that creature are made with advantage until the start of your next turn.
    \paragraph{Requirements} Expert proficiency with Crossbows.
\subsubsection{Purposeful Strike} \label{feat::purposefulstrike}
    You learn or improve the Sneak Attack action (see page \pageref{act::sneakattack}).
    You can take this feat three times, adding a d6 to the action's damage each time.

    You can only add these d6s to your Sneak Attack if you are holding a melee weapon on one hand and no other weapons.
    \paragraph{Requirements} Fighting Style: Dueling 1.
\subsubsection{Quickened Healing (2 FP)} \label{feat::quickenedhealing}
    Using an actions, you can spend a monastic die to heal yourself.
    You regain a number of hit points equal to the number rolled + your Wisdom modifier.
    \paragraph{Requirements} Fighting Style: Monastic Fighting 2.
\subsubsection{Quick Action (2 FP)} \label{feat::quickaction}
    You can make one melee weapon attack with a simple weapon with the light property as part of a Grapple, Escape a Grapple, Overrun, or Tumble action.

    You can use this ability only once per turn.
    \paragraph{Requirements} Skilled proficiency with Simple Weapons.
\subsubsection{Quick Draw} \label{feat::quickdraw}
    You learn the Quick Draw reaction (see page \pageref{act::quickdraw}).
    \paragraph{Requirements} Fighting Style: Thrown Weapons Fighting 1.
\subsubsection{Quick Loading} \label{feat::quickloading}
    You ignore the loading property of crossbows.
    % In addition, you can load or reload a one-handed ranged weapon without having a free hand.
    \paragraph{Requirements} Competent proficiency with Crossbows.
\subsubsection{Quick Shielding} \label{feat::quickshielding}
    You can don or doff a light shield as a free object interaction.
    \paragraph{Requirements} Proficiency with Small Shields.
\subsubsection{Quick Shot (2 FP)} \label{feat::quickshot}
    You learn the Quick Draw reaction (see page \pageref{act::quickdraw}).

    In addition, when you hit a creature with an opportunity attack with a bow, the creature cannot continue moving during that turn.
    \paragraph{Requirements} Skilled proficiency with Bows.
\subsubsection{Quick Slashes} \label{feat::quickslashes}
    When you make an opportunity attack with a slashing weapon, you have advantage on the attack roll.
\subsubsection{Quick Stabs} \label{feat::quickstabs}
    You reduce the multiple attack penalty by 2 with rapiers.
    \paragraph{Requirements} Competent proficiency with Rapiers.
\subsubsection{Ready for Action} \label{feat::readyforaction}
    You can draw or stow two one-handed weapons when you would normally be able to draw or stow only one.
    \paragraph{Requirements} Fighting Style: Two-Weapon Fighting 1.
\subsubsection{Reckless Attack} \label{feat::recklessattack}
    You learn the Reckless Attack action (see page \pageref{act::recklessattack}).
    \paragraph{Requirements} Fighting Style: Great Weapon Fighting 2.
\subsubsection{Reckless Shot} \label{feat::recklessshot}
    You learn the Reckless Shot action (see page \pageref{act::recklessshot}).
    \paragraph{Requirements} Fighting Style: Archery 2.
\subsubsection{Reel} \label{feat::reel}
    When you successfully have restrained another creature in your net, you may reel them up to 2 meters closer to you, costing no action on your part.

    You can also retrieve your net after a failed throw as an action.
    \paragraph{Requirements} Skilled proficiency with Nets \& Bolas.
\subsubsection{Revenant Blade} \label{feat::revenantblade}
    All straigth swords have the finesse property when you wield them.
    \paragraph{Requirements} Competent proficiency with Straight Swords.
\subsubsection{Rondel Hit} \label{feat::rondelhit}
    When you take the Attack action and attack with a polearm, you can use another action to make a melee attack with the opposite end of the weapon, using the same ability modifier as the main attack.
    The weapon's damage die for this attack is a d4, and the attack deals bludgeoning damage.

    This attack is unaffected by your multiple attack penalty, and doesn't count toward it.
    \paragraph{Requirements} Competent proficiency with Polearms.
\subsubsection{Rush} \label{feat::rush}
    As part of your attack, you can move up to half your movement speed towards the target of your attack.
    You can use this ability only once per turn.
    \paragraph{Requirements} Expert proficiency with Axes or Hammers.
\subsubsection{Savage Attacker} \label{feat::savageattacker}
    On your turn, when you score a critical hit with a melee weapon or reduce a creature to 0 hit points with one, you can make one melee weapon attack as a free action.
    \paragraph{Requirements} Fighting Style: Great Weapon Fighting 1.
\subsubsection{Second Wind} \label{feat::secondwind}
    On your turn, you can use an action to regain hit points equal to 1d10 + your level.

    Once you use this ability, you must finish a short rest before you can use it again.
    \paragraph{Requirements} Fighting Style: Battle Mastery 1.
\subsubsection{Sentinel (2 FP)} \label{feat::sentinel}
    When you hit a creature with an opportunity attack, the creature's speed becomes 0 for the rest of its turn.

    In addition, creatures provoke opportunity attacks from you even if they take the Disengage action before leaving your reach.
    \paragraph{Requirements} Fighting Style: Protection 1.
\subsubsection{Sharpshooter} \label{feat::sharpshooter}
    Attacking at long range doesn't impose disadvantage on your ranged weapon attack rolls.
    \paragraph{Requirements} Fighting Style: Archery 1.
\subsubsection{Shield Training} \label{feat::shieldtraining}
    Always balanced in combat, you learn how to use medium shields in combat.
\subsubsection{Shield Master} \label{feat::shieldmaster}
    You can take this feat three times, gaining an additional effect each time:
    \begin{itemize}
        \item The first time, you gain half cover against ranged attacks while using a medium shield.
        \item The second time, you can add your shield's AC bonus to any Dexterity saving throw you make against a spell or other harmful effect that targets only you.
        \item The third time, if you are subjected to an effect that allows you to make a Dexterity saving throw to take only half damage, you can use your reaction to take no damage if you succeed on the saving throw, interposing your shield between yourself and the source of the effect.
    \end{itemize}
    \paragraph{Requirements} Proficiency with Medium Shields.
\subsubsection{Shield Sweep (2 FP)} \label{feat::shieldsweep}
    Your melee attacks with curved swords or flails ignore the AC provided by shields.
    In addition, whenever you get a critical hit against a target wearing a shield, you can forfeit your brutal critical roll and injure the target's shield-bearing arm as if you had rolled a 4 in the Minor Injury chart (see page \pageref{ssec::injuriesandinsanity}).
    \paragraph{Requirements} Expert proficiency with Curved Swords or Flails.
\subsubsection{Shooter (2 FP)} \label{feat::shooter}
    Your ranged weapon attacks ignore half cover and three-quarters cover.

    In addition, you had advantage on attack rolls with ranged weapons against any creature that is not behind cover and doesn't have any other creature within 1 meter of it.
    \paragraph{Requirements} Fighting Style: Archery 2.
\subsubsection{Silent Despite Everything} \label{feat::silentdespiteeverything}
    Wearing medium armor doesn't impose disadvantage on your Dexterity (Stealth) checks.
    \paragraph{Requirements} Proficiency with Medium Armor.
\subsubsection{Simple Fighter} \label{feat::simplefighter}
    No one learned how to use a sword without first swinging a stick.

    You increase your proficiency level with simple weapons.
    This feat can be taken three times, or until you reach Expert proficiency with the weapon type.
\subsubsection{Slasher (2 FP)} \label{feat::slasher}
    Your vicious aim allows you to always aim for weak spots.
    Whenever you hit a creature with a weapon attack using a slashing weapon, that creature takes an additional 1d4 slashing damage.
\subsubsection{Solid Start (2 FP)} \label{feat::solidstart}
    You roll for initiative with advantage.

    In addition, any hit you score against a creature that hasn't taken any actions in combat is a critical hit.
    \paragraph{Requirements} Expert proficiency with Simple Weapons.
\subsubsection{Spearmaster} \label{feat::spearmaster}
    You increase your proficiency level with spears.
    This feat can be taken three times, or until you reach Expert proficiency with the weapon type.
\subsubsection{Stab-a-Lung!} \label{feat::stabalung}
    You learn or improve the Sneak Attack action (see page \pageref{act::sneakattack}).
    You can take this feat three times, adding a d6 to the action's damage each time.

    You can only add these d6s to your Sneak Attack if you attack with a piercing weapon.
\subsubsection{Staff Fighter} \label{feat::stafffighter}
    You can use your proficiency with spears for making attacks with quarterstaves.
    Any feat that requires proficiency with spears applies for staves too for you, changing any piercing damage to bludgeoning.

    In addition, when you knock a creature unconscious with a quarterstaff, you can choose to leave the creature stable.
    \paragraph{Requirements} Competent proficiency with spears.
\subsubsection{Stalwart Shield (2 FP)} \label{feat::stalwartshield}
    You learn the Push action.

    In addition, you can use this action as a free action after making an attack with a melee weapon against it.
    You must be wielding a medium shield to use the action in this way.
    \paragraph{Requirements} Proficiency with Medium Shields.
\subsubsection{Stalwart Stance (2 FP)} \label{feat::stalwartstance}
    You learn the Block reaction (see page \pageref{act::block}).

    If an attack misses you after using this reaction, you can make an melee weapon attack against the attacking creature if you are wielding a straight sword.
    \paragraph{Requirements} Expert proficiency with Straight Swords.
\subsubsection{Steel Resolve} \label{feat::steelresolve}
    Creatures provoke opportunity attacks from you when you when they enter your reach while you have a flail equipped.
    \paragraph{Requirements} Competent proficiency with Flails.
\subsubsection{Strapped and Secured (2 FP)} \label{feat::strappedandsecured}
    By attaching a leather strap to a light shield, you can carry it without using your off-hand.
    Due to the shield you still cannot carry a weapon in that hand, but your hand is free to use items, hold spellcasting focus, reload a crossbow, etc.

    In addition, your shield cannot be removed by the Disarm action.
    \paragraph{Requirements} Proficiency with Small Shields.
\subsubsection{Strengthened Nets (2 FP)} \label{feat::strengthenednets}
    The escape DC and AC of your nets are increased to 18.
    In addition, a creature that fails on a Strength (Athletics) to escape from your net takes 1d4 slashing damage.
    \paragraph{Requirements} Expert proficiency with Nets \& Bolas.
\subsubsection{Strike a Pose (2 FP)} \label{feat::strikeapose}
    You have advantage in all Charisma (Intimidation) and Charisma (Performance) checks while mounted.

    In addition, you can strike an intimidating pose as an action during your turn.
    Each creature of your choice within 6 meters of you must make a Wisdom saving throw with DC equal to 8 + your proficiency with land vehicles + your Charisma modifier.
    On a failed save, a target becomes frightened of you for 1 minute.
    If the frightened creature takes any damage, it can repeat the saving throw, ending the effect on itself on a success.

    You can only use this ability while mounted.
    \paragraph{Requirements} Fighting Style: Mounted Fighting 1.
\subsubsection{Strong Lungs (2 FP)} \label{feat::stronglungs}
    Your blowgun deals 1d4 piercing damage instead of 1.

    In addition, attacking with a blowgun at long range doesn't impose disadvantage on your ranged weapon attack rolls.
    \paragraph{Requirements} Expert proficiency with Blowguns.
\subsubsection{Still Aim} \label{feat::stillaim}
    You gain a bonus equal to half your proficiency bonus with bows to damage rolls with a bow if you don't move during your turn or are riding a mount.
    \paragraph{Requirements} Expert proficiency with Bows.
\subsubsection{Subtle Thrusts} \label{feat::subtlethrusts}
    When you use a spear with two hands, it gains the finesse property.
    \paragraph{Requirements} Competent proficiency with spears.
\subsubsection{Sudden Attack} \label{feat::suddenattack}
    You can use your bow to make melee weapon attacks.
    Melee attacks with your bow deal 1d4 bludgeoning damage plus your Strength modifier.

    When you hit with a melee attack with your bow, the attacked creature can't make opportunity attacks against you until the start of your next turn.
    \paragraph{Requirements} Competent proficiency with Bows or Crossbows.
\subsubsection{Superior Critical (2 FP)} \label{feat::superiorcritical}
    Your weapon attacks with two-handed or versatile weapons score a critical hit on a roll of 19 or 20.

    If you already have taken the Improved Critical feat (see page \pageref{feat::improvedcritical}), this range is extended to rolls of 18-20.
    \paragraph{Requirements} Fighting Style: Great Weapon Fighting 2.
\subsubsection{Supernatural Accuracy (2 FP)} \label{feat::supernaturalaccuracy}
    Whenever you have advantage on a ranged attack roll using Dexterity, Intelligence, Wisdom, or Charisma, you can reroll one of the dice once.
    \paragraph{Requirements} Fighting Style: Archery 1.
\subsubsection{Sustained Impetus} \label{feat::sustainedimpetus}
    Whenever you miss with a melee weapon attack using a slashing weapon, you ignore the multiple attack penalty caused by that attack.
\subsubsection{Sweep (2 FP)} \label{feat::sweep}
    Using two actions, you can choose to attack all creatures in a 180$\degree$ arc in front of you within the range of your equipped weapon.
    You make a melee attack roll once, hitting all creatures with an AC lower than the number rolled.
    You roll damage once, which is applied on all the struck creatures.
    \paragraph{Requirements} Skilled proficiency with Polearms.
\subsubsection{Swift Deflection} \label{feat::swiftdeflection}
    You learn the Parry reaction (see page \pageref{act::parry}).

    You can take this feat two additional times.
    Each time you increase the bonus to your AC by one die, so a d6 turns into a d8, and a d8 into a d10.
    \paragraph{Requirements} Proficiency with Small Shields.
\subsubsection{Swordbreaker} \label{feat::swordbreaker}
    You learn the Parry action (see page \pageref{act::parry}), and you can take the Disarm action as an opportunity attack.

    In addition, you have advantage on the Disarm and the Parry actions against creatures wielding a sword.
    \paragraph{Requirements} Skilled proficiency with spears.
\subsubsection{Swordmaster} \label{feat::swordmaster}
    You increase your proficiency level with straight swords.
    This feat can be taken three times, or until you reach Expert proficiency with the weapon type.
\subsubsection{Take Cover!} \label{feat::takecover}
    You can duck behind a heavy shield as an action, granting you full cover against ranged attacks until the start of your next turn.

    You can take this feat two additional times, granting you a new bonus each time:
    The second time, creatures standing behind you from the attacker's perspective have three-fourths cover against ranged attacks.

    The third time, you have advantage on Strength and Dexterity saving throws while you are ducking behind your shield.
    \paragraph{Requirements} Proficiency with Heavy Shields.
\subsubsection{Taken Aback} \label{feat::takenaback}
    When you hit a creature with an opportunity attack using two-handed or versatile weapons, the creature cannot take reactions until the start of its next turn.
    \paragraph{Requirements} Fighting Style: Great Weapon Fighting 2.
\subsubsection{The Best Defense...} \label{feat::thebestdefense}
    As an action, you can bash a creature with your shield.
    Your attack bonus for this attack is equal to the shield's AC bonus + your Strength modifier.
    On a hit, the creature takes 1d4 + your Strength modifier in bludgeoning damage.

    You can use this attack only once per turn, and it doesn't count towards your multiple attack penalty.

    In addition, if you hit a creature with this action using a medium shield, your next melee attack on this turn to the creature is made with advantage.
    \paragraph{Requirements} Proficiency with Medium Shields.
\subsubsection{Throwing Arm} \label{feat::throwingarm}
    Attacking at long range doesn't impose disadvantage on your thrown weapon attack rolls.
    \paragraph{Requirements} Fighting Style: Thrown Weapons Fighting 2.
\subsubsection{Tough (2 FP)} \label{feat::tough}
    Your hit point maximum increases by an amount equal to your level when you take this feat.
    Whenever you gain a level thereafter, your hit point maximum increases by an additional hit point.
    \paragraph{Requirements} Proficiency with Heavy Armor.
\subsubsection{Trained Mount} \label{feat::trainedmount}
    As an action, you can issue commands to your mount --- even if you are not mounting it.
    You can take this feat three times, improving the commands you can issue to your mount:
    \begin{itemize}
        \item The first time you take this feat, you can issue a general command to your mount, which it will follow to the best of its ability.
        \item The second time, you can issue your mount to make the Attack, Overrun, or Shove actions during its turn.
        \item The third, you gain full control of your mount during its turn after issuing a command.
    \end{itemize}
    \paragraph{Requirements} Fighting Style: Mounted Fighting 1.
\subsubsection{Trample} \label{feat::trample}
    While mounted, you can try to move through a creature's space by having your mount take the Overrun action.
    Your mount makes its Strength (Athletics) roll with advantage if it has moved at least 2 meters before making this roll.

    If your mount succeeds, the target is knocked prone, and it takes 1d8 + your mount's Strength modifier bludgeoning damage.
    \paragraph{Requirements} Fighting Style: Mounted Fighting 2.
\subsubsection{Tree Feller} \label{feat::treefeller}
    Nature is your foe, and you use your axe to fell both trees and enemies.
    You have advantage on attack rolls against plant-based creatures and any structure made of wood.
    \paragraph{Requirements} Competent proficiency with Axes.
\subsubsection{Trip} \label{feat::trip}
    Using two actions, you can attack a creature with the intention of knocking it down.
    Make a normal attack roll against the creature.
    On a hit, the creature must make a Strength saving throw of a DC equal to 8 + your proficiency bonus with polearms + your Strength modifier in addition to taking damage.
    On a failed save, you knock the target prone.

    The creature has advantage on this roll if it is at least 2 size categories larger than you.
    \paragraph{Requirements} Skilled proficiency with Polearms.
\subsubsection{Two-weapon Master (2 FP)} \label{feat::twoweaponmaster}
    You can make a melee attack as a free action on your turn.

    In addition, when you are fighting with one weapon in each hand, all multiple attack penalties are reduced by 1 for you.
    \paragraph{Requirements} Fighting Style: Two-Weapon Fighting 2.
\subsubsection{Unapproachable (2 FP)} \label{feat::unapproachable}
    While you are wielding a spear or polearm, other creatures provoke an opportunity attack from you when they enter your reach.

    In addition, your reach for opportunity attacks with spears and polearms is extended to the reach of your weapon.
    \paragraph{Requirements} Expert proficiency with Polearms, Expert proficiency with spears.
\subsubsection{Unarmed Artist} \label{feat::unarmedartist}
    You are a painter.
    Your brush is your fist, and your canvas is that guy's face.

    When you get a critical hit with an unarmed strike, you can take the Disarm, Disengage, or Shove action as a free action.
    \paragraph{Requirements} Fighting Style: Unarmed Fighting 2.
\subsubsection{Unarmored Agility} \label{feat::unarmoredagility}
    As long as you are not wearing any armor, your movement speed increases by 1 meter.

    You can take this feat three times, increasing your movement speed while unarmored by 1 meter each time.
\subsubsection{Unarmored Master (2 FP)} \label{feat::unarmoredmaster}
    If you are subjected to an effect that allows you to make a Dexterity saving throw to take only half damage, you can use your reaction to take no damage if you succeed on the saving throw, and half damage if you fail.
    You only get this bonus while wearing no armor or light armor.
\subsubsection{Uncanny Dodge} \label{feat::uncannydodge}
    When an attacker that you can see hits you with an attack, you can use your reaction to halve the attack's damage against you.

    To gain the benefits of this feat, you must be wearing either no armor or light armor.
    \paragraph{Requirements} Proficiency with Light Armor.
\subsubsection{Uncertain Location} \label{feat::uncertainlocation}
    Using the looseness of your light armor to your advantage, you gain a +1 bonus to your AC against ranged attack while wearing light armor.

    You can take this feat three times, increasing this AC bonus to +2 and +3 the second and third time respectively.
    \paragraph{Requirements} Proficiency with Light Armor.
\subsubsection{Visceral Attack (2 FP)} \label{feat::visceralattack}
    Being within 1.5 meters of a hostile creature doesn't impose disadvantage on your ranged attack rolls, and you don't provoke attacks of opportunity when using the Attack action with a ranged weapon when within the reach of a creature.

    In addition, when you successfully hit a creature with a crossbow or firearm while within 1 meter of it, your next melee attack against the creature is made with advantage.
    \paragraph{Requirements} Expert proficiency with Crossbows.
\subsubsection{Vital Protection} \label{feat::vitalprotection}
    While wearing medium armor, you may add a +1 to Strength, Dexterity, and Constitution saving throws.

    You can take this feat two additional times, increasing this bonus by +1 each time.
    \paragraph{Requirements} Proficiency with Medium Armor.
\subsubsection{Volley (2 FP)} \label{feat::volley}
    Using two actions, you can make a ranged attack with a bow against any number of creatures within 2 meters of a point you can see within your bow's range.
    You must have ammunition for each target, as normal, and you make a separate attack roll for each target.
    \paragraph{Requirements} Expert proficiency with Bows.
\subsubsection{Whip Master} \label{feat::whipmaster}
    You increase your proficiency level with whips.
    This feat can be taken three times, or until you reach Expert proficiency with the weapon type.
\subsubsection{Whirlwind Attack} \label{feat::whirlwindattack}
    Using two actions, you can make a melee weapon attack against any number of creatures within 1 meter of you, with a separate attack roll for each target.
    \paragraph{Requirements} Expert proficiency with Curved Swords.
\subsubsection{Wrap Around} \label{feat::wraparound}
    You can use a whip to extend the range of the Grapple and Shove actions to the range of the whip.
    \paragraph{Requirements} Competent proficiency with Whips.

% === UNUSED FEATS ========================================================== %
% \subsubsection{Bullet Tinkerer} \label{feat::bullettinkerer}
%     \paragraph{RANK 1} You can craft ammunition using a set of Tinker's Tools at half the cost.
%     \paragraph{RANK 2} If you roll a misfire, you can use your reaction to roll a d20 with disadvantage.
%     If the number rolled is higher than the weapon's misfire score, the weapon does not misfire.
%     \paragraph{RANK 3} You learn the Violent Shot technique.
%
% \subsubsection{Firearm Specialist} \label{feat::firearmspecialist}
%     \paragraph{RANK 1} You can reload any weapon as a bonus action.
%     \paragraph{RANK 2} If you are using a pistol in one hand and nothing on the other, you get a +2 bonus to your ranged weapon attacks with it.
%     \paragraph{RANK 3} You learn the Rapid Repair technique.
%
% \subsubsection{Gunner} \label{feat::gunner}
%     \paragraph{RANK 1} As long as you can examine the weapon for 30 seconds, you are proficient with any kind of firearm, even if it is new or experimental.
%     \paragraph{RANK 2} Your firearm attacks score a critical hit on a roll of 19-20.
%
% \subsubsection{Kama Master} \label{feat::kamamaster}
%     \paragraph{RANK 1} When you use a kama, its damage die changes from a d6 to a d8
%     \paragraph{RANK 2} When you successfully attack a foe with a piercing attack using your kama, you also deal 1d4 slashing damage, unmodified by your dexterity.
%     \paragraph{RANK 3} You can use the Trip technique as part of an opportunity attack.
%
% \subsubsection{Pick Adept} \label{feat::pickadept}
%     \paragraph{RANK 1} You are proficient with picks.
%     \paragraph{RANK 2} All picks have the versatile property for you.
%     When you wield a pick with two hands, increase its damage die by one (d6 to d8, d8 to d10, etc), and have the heavy property.
%     \paragraph{RANK 3} You learn the Trip technique.
%
% \subsubsection{Revenant Blade} \label{feat::revenantblade}
%     \paragraph{RANK 1} A double-bladed scimitar has the finesse property when you wield it.
%     \paragraph{RANK 2} While you are holding a double-bladed scimitar with two hands, you gain a +1 bonus to Armor Class.
%     \paragraph{RANK 3} Instead of dealing 1d4 slashing damage, the extra attack you can do with a double-bladed scimitar deals 2d4 slashing damage.
%
% \subsubsection{Stone Piercer} \label{feat::stonepiercer}
%     \paragraph{RANK 1} You double your damage dice against objects and structures while attacking with a pick.
%     Additionally, you have advantage on checks made with a climber's kit and pitons.
%     \paragraph{RANK 2} When you score a critical hit that deals piercing damage to a creature, you can roll one additional damage die when determining the extra piercing damage the target takes.
%     \paragraph{RANK 3} You learn the Injure technique.
% === ARMOR PROPERTIES =============================================================================
% BULWARK
% \subsubsection{Virtuous Defender} \label{feat::virtuousdefender}
%     You learn to fend off strikes directed at you, or other creatures nearby.
%     If you or a creature you can see within 1 meter of you is hit by an attack, you use your reaction to switch places with the creature, receiving the attack instead.
%
%     To gain the benefits of this feat, you must be wearing armor with the Bulwark property.
% \subsubsection{NAME} \label{feat::name}
%     DESCRIPTION
%
%     To gain the benefits of this feat, you must be wearing armor with the Bulwark property.
% \subsubsection{NAME (2 FP)} \label{feat::name}
%     DESCRIPTION
%
%     To gain the benefits of this feat, you must be wearing armor with the Bulwark property.
% % CHAIN
% \subsubsection{NAME} \label{feat::name}
%     DESCRIPTION
%
%     To gain the benefits of this feat, you must be wearing armor with the Chain property.
% \subsubsection{NAME} \label{feat::name}
%     DESCRIPTION
%
%     To gain the benefits of this feat, you must be wearing armor with the Chain property.
% \subsubsection{NAME (2 FP)} \label{feat::name}
%     DESCRIPTION
%
%     To gain the benefits of this feat, you must be wearing armor with the Chain property.
%     % use noise of chains to distract nearby creatures?
% % CLOTH
% \subsubsection{NAME} \label{feat::name}
%     DESCRIPTION
%
%     To gain the benefits of this feat, you must be wearing armor with the Cloth property.
% \subsubsection{NAME} \label{feat::name}
%     DESCRIPTION
%
%     To gain the benefits of this feat, you must be wearing armor with the Cloth property.
% \subsubsection{NAME (2 FP)} \label{feat::name}
%     DESCRIPTION
%
%     To gain the benefits of this feat, you must be wearing armor with the Cloth property.
% % COMPOSITE
% \subsubsection{NAME} \label{feat::name}
%     DESCRIPTION
%
%     To gain the benefits of this feat, you must be wearing armor with the Composite property.
% \subsubsection{NAME} \label{feat::name}
%     DESCRIPTION
%
%     To gain the benefits of this feat, you must be wearing armor with the Composite property.
% \subsubsection{NAME (2 FP)} \label{feat::name}
%     DESCRIPTION
%
%     To gain the benefits of this feat, you must be wearing armor with the Composite property.
% % LEATHER
% \subsubsection{NAME} \label{feat::name}
%     DESCRIPTION
%
%     To gain the benefits of this feat, you must be wearing armor with the Leather property.
% \subsubsection{NAME} \label{feat::name}
%     DESCRIPTION
%
%     To gain the benefits of this feat, you must be wearing armor with the Leather property.
% \subsubsection{NAME (2 FP)} \label{feat::name}
%     DESCRIPTION
%
%     To gain the benefits of this feat, you must be wearing armor with the Leather property.
% % PLATE
% \subsubsection{Armor-Assisted Punches} \label{feat::armorassistedpunches}
%     If you are wearing plate gauntlets, add a +2 bonus to your damage rolls with unarmed attacks.
% \subsubsection{NAME} \label{feat::name}
%     DESCRIPTION
%
%     To gain the benefits of this feat, you must be wearing armor with the Plate property.
% \subsubsection{NAME (2 FP)} \label{feat::name}
%     DESCRIPTION
%
%     To gain the benefits of this feat, you must be wearing armor with the Plate property.
% Insofar unused special weapons: spears, staves.

\newpage~\newpage

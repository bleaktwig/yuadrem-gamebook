% !TEX root = ../main.tex
\section{Actions}
Techniques are actions that are unlocked by certain ranks in talents. % WRITE MORE & UPDATE.

% TODO: MOVE TO LISTS!

\subsubsection{Aim $\circ$} \label{act::aim}
    You give yourself advantage on your next attack roll, which can be made until the end of your next turn.

\subsubsection{Block $\diamond$} \label{act::block}
    You can use your reaction to prepare yourself against one melee or ranged attack.
    You increase your AC by 1d6 against the attack.
    If the attack does hit you, you take half damage from it (rounded down).

\subsubsection{Distracting Strike $\circ\circ$} \label{act::distractingstrike}
    You use your action to distract a creature with an attack, giving your allies an opening.
    You roll damage normally, and the next attack roll against the target by an creature other than you has advantage if the attack is made before the start of your next turn.

\subsubsection{Quick Draw $\diamond$} \label{act::quickdraw}
    You can make an attack of opportunity with an equipped thrown or ranged weapon when a creature moves out of a 9-meter radius around you.

\subsubsection{Mark $\circ\circ$} \label{act::mark}
    You choose a creature you can see within 27 meters and mark it as your quarry.
    For up to one hour, you deal an extra 1d6 damage to the target whenever you hit it with a weapon attack, and you have advantage on any Intelligence (Investigation), Wisdom (Perception), and Wisdom (Survival) checks you make to find it.
    You can use this action a number of times equal to your Wisdom modifier, and you restore all expended uses on a short rest.

\subsubsection{Parry $\diamond$} \label{act::parry}
    When another creature attacks you with a melee attack, you can use your reaction to increase your AC by 1d6 + your Dexterity modifier against the attack.

\subsubsection{Push $\circ$} \label{act::push}
    You try to push a target back.
    The target must be no more than one size larger than you and be within 1.5 meters of you.
    You make a Strength (Athletics) check contested by the target's Strength (Athletics) or Dexterity (Acrobatics) check (the target chooses the ability to use).
    If you win, you push the target up to 4.5 meters away from you.
    If you win by 10 or more, the target is also knocked prone.

\subsubsection{Reckless Attack $\circ$} \label{act::recklessattack}
    You make a special melee weapon attack against a creature using a weapon with the heavy property.
    Apply a -5 penalty to the attack roll.
    If the attack hits, you add +10 to the attack's damage.

\subsubsection{Reckless Shot $\circ$} \label{act::recklessshot}
    You make a special ranged weapon attack against a creature with a ranged weapon.
    Apply a -5 penalty to the attack roll.
    If the attack hits, you add +10 to the attack's damage.

\subsubsection{Riposte $\diamond$} \label{act::riposte}
    When a creature misses you with a melee attack, you can use your reaction to make a melee weapon attack against the creature.

\subsubsection{Sneak Attack $\circ\circ$} \label{act::sneakattack}
    You know how to strike subtly and exploit a foe's distraction.
    You do an attack that deal an extra 2d6 damage to one creature you hit with an attack if you have advantage on the attack roll.
    The attack must use a finesse or a ranged weapon.

    You don't need advantage on the attack roll if another enemy of the target is within 5 feet of it, that enemy isn't incapacitated, and you don't have disadvantage on the attack roll.
    You can use this action only once per turn.

    You can take this action various times.
    Each time after the first increases the number of d6s rolled by one.

\subsubsection{Steal $\circ\circ$} \label{act::steal}
    As an action, you can make a Dexterity (Sleight of Hand) check contested by a creature's Wisdom (Perception) to plant something on someone else, conceal an object on a creature, lift a purse, or take something from a pocket.
    You can do this in the middle of an encounter.

% === UNUSED SO FAR ================================================================================
% \subsubsection{Careless Deflect} \label{tec::carelessdeflect}
%     When a creature misses you with a melee attack roll, you can use your reaction to cause that attack to hit one creature of your choice, other than the attacker, that you can see within 1.5 meters of you.
%
% \subsubsection{Charge Stopper} \label{tec::chargestopper}
%     You can set your polearm to receive a charge.
%     As a bonus action, choose a creature you can see that is at least 6 meters away from you.
%     If that creature moves within you polearm's reach on its next turn, you can make a melee attack against it with your polearm as a reaction.
%     If the attack hits, the targets takes an extra damage die.
%     You can't use this technique if the creature used the Disengage action before moving.
%
% \subsubsection{Commander's Strike} \label{tec::commandersstrike}
%     When you take the Attack action on your turn, you can choose to attack only once and use a bonus action to direct one of your companions to strike.
%     When you do so, choose a friendly creature who can see or hear you.
%     That creature can immediately use its reaction to make one weapon attack, adding a d8 to the attack's damage roll.
%
% \subsubsection{Disarming Attack} \label{tec::disarmingattack}
%     You use your action to attempt to disarm the target with an attack, forcing it to drop one item of your choice that it's holding.
%     The target must make a Strength saving throw with a DC of 8 + your proficiency bonus + your Strength modifier.
%     On a failed save, it drops the object you choose.
%     The object lands at its feet.
%
% \subsubsection{Evasive Footwork} \label{tec::evasivefootwork}
%     When you move, you can use a bonus action to add a d6 to your AC until you stop moving.
%
% \subsubsection{Goading Attack} \label{tec::goadingattack}
%     You use your action to attack a creature and goad it into attacking you.
%     The target must make a Wisdom saving throw with a DC of 8 + your proficiency bonus + your Charisma modifier.
%     On a failed save, the target has disadvantage on all attack rolls against targets other than you until the end of your next turn.
%
% \subsubsection{Injure} \label{tec::injure}
%     As an action, you can attack with the sole purpose of injuring a creature.
%     Roll your weapon attack normally.
%     On a hit, you deal only half damage, but the creature rolls on the minor injury chart.
%     If the attack is a critical hit, the creature rolls on the major injury chart.
%
% \subsubsection{Maneuvering Attack} \label{tec::maneuveringattack}
%     When you hit a creature with a melee attack, you can use your bonus action to maneuver one of your comrades into a more advantageous position.
%     You choose a friendly creature who can see or hear you.
%     That creature can use its reaction to move up to half its speed without provoking opportunity attacks from the target of your attack.
%
% \subsubsection{Menacing Attack} \label{tec::menacingattack}
%     When you hit a creature with a weapon attack, you can use your bonus action to attempt to frighten the target.
%     The target must make a Wisdom saving throw of a DC of 8 + your proficiency bonus + your Charisma modifier.
%     On a failed save, it is frightened of you until the end of your next turn.
%
% \subsubsection{Precise Attack} \label{tec::preciseattack}
%     When you make a weapon attack roll against a creature, you can use your bonus action to add a d8 to the attack roll.
%     You can use this technique before or after making the attack roll, but before any effects of the attack are applied.
%
% \subsubsection{Rapid Repair} \label{tec::rapidrepair}
%     You can attempt to repair a misfired (but not broken) firearm as a bonus action.
%
% \subsubsection{Violent Shot} \label{tec::violentshot}
%     When you make a firearm attack against a creature, you can tune your weapon to enhance the volatility of the attack.
%     The attack gains a +2 to the firearm's misfire score.
%     If the attack hits, you can roll one additional weapon damage die.

% === MARTIAL ARTS TECHNIQUES ================================================== %
% \subsubsection{Predetermined Fate} \label{mtec::predeterminedfate}
%     Whenever you make a saving throw and fail, you can reroll it and take the second result.
%
% \subsubsection{Quivering Palm} \label{mtec::quiveringpalm}
%     This is the ultimate technique of the TODO:SCHOOL school, and can only be learned after you learn all other martial arts techniques.
%     When you hit a creature with an unarmed strike, you can use the quivering palm technique.
%     The creature must make a Constitution saving throw or take 10d10 bludgeoning damage, taking half damage on a success.
%     You can use this technique only once per short rest.

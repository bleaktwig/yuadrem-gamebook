% !TEX root = ../main.tex
\section{Techniques} % TODO: CHANGE TO ACTIONS
Techniques are actions that are unlocked by certain ranks in talents. % WRITE MORE?

% TODO: ADD: If you learn an action more than once, add a +2 to the DC or attack roll related to it.

% NOTE: UPDATED:
\subsubsection{Aim $\circ$} \label{act::aim}
    You give yourself advantage on your next attack roll.

\subsubsection{Push $\circ$} \label{act::push}
    You try to push a target back.
    The target must be no more than one size larger than you and be within 1.5 meters of you.
    The target must make a Strength saving throw with a DC of 8 + your Strength modifier.
    On a failed save, you push the target up to 3 meters away from you.
    If the target fails by 10 or more, it is also knocked prone.

\subsubsection{Reckless Attack $\circ$} \label{act::recklessattack}
    You make a special melee weapon attack against a creature with a heavy weapon.
    Apply a -5 penalty to the attack roll.
    If the attack hits, you add +10 to the attack's damage.

\subsubsection{Sneak Attack $\circ\circ$} \label{act::sneakattack}
    You know how to strike subtly and exploit a foe's distraction.
    You do an attack that deal an extra 2d6 damage to one creature you hit with an attack if you have advantage on the attack roll.
    The attack must use a finesse or a ranged weapon.

    You don't need advantage on the attack roll if another enemy of the target is within 5 feet of it, that enemy isn't incapacitated, and you don't have disadvantage on the attack roll.
    You can use this action only once per turn.

    You can take this action various times.
    Each time after the first increases the number of d6s rolled by one.

\subsubsection{Steal $\circ\circ$} \label{act::steal}
    As an action, you can make a Dexterity (Sleight of Hand) check contested by a creature's Wisdom (Perception) to plant something on someone else, conceal an object on a creature, lift a purse, or take something from a pocket.
    You can do this in the middle of an encounter.

% === COMBAT MANEUVERS ========================================================= %
\subsubsection{Bonk} \label{tec::bonk}
As an action, you can attempt to strike a creature's head with a melee weapon attack.
In addition to taking the attack's normal damage, the creature must succeed on a Constitution saving throw (DC 8 + your proficiency bonus + your Strength modifier) or have disadvantage on ability checks and attack rolls during its next turn.
You need to be using a bludgeoning weapon to perform this technique.

\subsubsection{Block} \label{tec::block}
You can use your reaction to prepare yourself against one melee or ranged attack.
You increase you AC by 1d6 against one attack.
If the attack hits you, you only take half damage (rounded down).

\subsubsection{Buttstroke} \label{tec::buttstroke}
You use your action to make one Strength melee attack with the pommel or butt of your weapon against a creature.
If the attack hits, the creature takes 1d6 + your Strength modifier bludgeoning damage, and it can't perform opportunity attacks until the start of your next turn.

\subsubsection{Feint} \label{tec::feint}
You use your bonus action to feint, choosing one creature within melee range as your target.
You have advantage on your next melee attack roll this turn against that creature.
The advantage is lost if not used on the turn you gain it.

\subsubsection{Lunge} \label{tec::lunge}
You can replace one of the attacks in your Attack action with a lunge.
A lunge is a melee attack with a range extended by 1.5 meters.
This attack does not require any movement from your part, and thus doesn't provoke opportunity attacks.

\subsubsection{Parry} \label{tec::parry}
When another creature attacks you with a melee attack, you can use your reaction to increase your AC by 1d6 + your Dexterity modifier.

\subsubsection{Protect} \label{tec::protect}
When a creature you can see attacks a target other than you that is within 1.5 meters of you, you can use your reaction to impose disadvantage on the attack roll.
You must have a shield equipped.

\subsubsection{Quick Draw} \label{tec::quickdraw}
You can make an attack of opportunity with an equipped thrown or ranged weapon when a creature moves out of a 9-meter radius around you.

\subsubsection{Riposte} \label{tec::riposte}
When a creature misses you with a melee attack, you can use your reaction to make a melee weapon attack against the creature.

\subsubsection{Rush} \label{tec::rush}
As a bonus action, you can move up to your speed toward an enemy of your choice that you can see or hear.
You must end this move closer to the enemy than you started.

\subsubsection{Sweep} \label{tec::sweep}
As an action, you can choose to attack all targets within a 180$\degree$ arc in front of you.
You make an melee attack roll normally once, and hit all creatures with an AC lower than the number rolled.
You roll damage once for all the struck creatures.

\subsubsection{Trip} \label{tec::trip}
As an action, you attack a target with the intention of knocking it down.
If the target is Large or smaller, it must make a Strength saving throw of a DC equal to 8 + your proficiency bonus + your Strength modifier.
On a failed save, you knock the target prone.

% === ADVANCED COMBAT MANEUVERS ================================================ %
\subsubsection{Bash} \label{tec::bash}
You use your bonus action to bash a creature's head with your shield.
Your attack bonus for this attack is equal to your proficiency bonus (if you're proficient with the shield type) + your strength modifier.
On a hit, the creature takes 1d6 + your Strength modifier bludgeoning damage, and your next melee attack on this turn to it is made with advantage.

\subsubsection{Beast Command} \label{tec::beastcommand}
You use your bonus action to command a beast within 18 meters of you that can hear you and that isn't currently following the command of someone else.
Make a Wisdom (Animal Handling) check of DC 8.
On a success, you decide now what action the beast will take and where it will move during its next turn, or you issue a general command that lasts for 1 minute, such as to guard a particular area.
You can only have one creature under this effect at a time.

\subsubsection{Bull's Eye} \label{tec::bullseye}
You use your action to perform one lethal attack with a thrown or ranged weapon.
Perform your attack normally.
If your attack is successful, it is an automatic critical hit.

\subsubsection{Careless Deflect} \label{tec::carelessdeflect}
When a creature misses you with a melee attack roll, you can use your reaction to cause that attack to hit one creature of your choice, other than the attacker, that you can see within 1.5 meters of you.

\subsubsection{Charge} \label{tec::charge}
When you use your action to Dash, you can use a bonus action to make one melee weapon attack against, shove, or grapple a creature.
If you move at least 3 meters in a straight line immediately before taking this bonus action, you either gain a +5 to the attack's damage roll (if you chose the former and hit) or push the target up to 10 feet away from you (if you chose the latter and succeed).

\subsubsection{Charge Stopper} \label{tec::chargestopper}
You can set your polearm to receive a charge.
As a bonus action, choose a creature you can see that is at least 6 meters away from you.
If that creature moves within you polearm's reach on its next turn, you can make a melee attack against it with your polearm as a reaction.
If the attack hits, the targets takes an extra damage die.
You can't use this technique if the creature used the Disengage action before moving.

\subsubsection{Charm} \label{tec::charm}
If you spend one minute talking to someone who can understand what you say, you can make a Charisma (Persuasion) check contested by the creature's Wisdom (Insight) check.
If you or your companions are fighting the creature, your check automatically fails.
If your check succeeds, the target is charmed by you as long as it remains within 18 meters of you and for 1 minute thereafter.

\subsubsection{Cheap Shot} \label{tec::cheapshot}
After successfully attacking a target with a melee weapon attack, you can use your bonus action to attack the same target with a crossbow or pistol by shooting in melee range.
This attack is done with advantage.

\subsubsection{Cleave} \label{tec::cleave}
You learn to use the weight of your weapon to destabilize your opponent.
Before making an attack with a heavy weapon, you can decide to apply a -5 to the attack roll.
If the attack hits, you can use a bonus action to shove the creature 1.5 meters.
The creature falls prone.

\subsubsection{Commander's Strike} \label{tec::commandersstrike}
When you take the Attack action on your turn, you can choose to attack only once and use a bonus action to direct one of your companions to strike.
When you do so, choose a friendly creature who can see or hear you.
That creature can immediately use its reaction to make one weapon attack, adding a d8 to the attack's damage roll.

\subsubsection{Deceive} \label{tec::deceive}
When you take the Attack action on your turn, you can replace one attack with an attempt to deceive one humanoid you can see within 9 meters of you that can see and hear you.
Make a Charisma (Deception) check contested by the target's Wisdom (Insight) check.
If your check succeeds, you movement doesn't provoke attacks from the target and your attack rolls against it have advantage; both benefits last until the end of your next turn.
If your check fails, the target can't be deceived by you in this way for 1 hour.

\subsubsection{Defensive Stance} \label{tec::defensivestance}
You use your action and prepare yourself against assailants from all directions.
Until the start of your next turn you gain a d4 bonus to your AC, and you can make opportunity attacks and the Riposte technique without using your reaction.

\subsubsection{Demoralize} \label{tec::demoralize}
When you take the Attack action on your turn, you can replace one attack with an attempt to demoralize one humanoid you can see within 9 meters of you that can see and hear you.
Make a Charisma (Intimidation) check contested by the target's Wisdom (Insight) check.
If your check succeeds, the target is frightened until the end of your next turn.
If your check fails, the target can't be frightened by you in this way for 1 hour.

\subsubsection{Disarming Attack} \label{tec::disarmingattack}
You use your action to attempt to disarm the target with an attack, forcing it to drop one item of your choice that it's holding.
The target must make a Strength saving throw with a DC of 8 + your proficiency bonus + your Strength modifier.
On a failed save, it drops the object you choose.
The object lands at its feet.

\subsubsection{Distract} \label{tec::distract}
While performing, you can try to distract one humanoid you can see who can see and hear you.
Make a Charisma (Performance) check contested by the humanoid's Wisdom (Insight) check.
If you check succeeds, you grab the humanoid's attention enough that it makes Wisdom (Perception), Wisdom (Insight), and Intelligence (Investigation) checks with disadvantage until you stop performing.

\subsubsection{Distracting Strike} \label{tec::distractingstrike}
You use your action to distract a creature with an attack, giving your allies an opening.
Additionally to dealing damage normally, the next attack roll against the target by an attacker other than you has advantage if the attack is made before the start of your next turn.

\subsubsection{Evasive Footwork} \label{tec::evasivefootwork}
When you move, you can use a bonus action to add a d6 to your AC until you stop moving.

\subsubsection{Extend Reach} \label{tec::extendreach}
As a bonus action on your turn, you can increase your reach with a polearm by 1.5 meters until the start of your next turn.

\subsubsection{Fell Strike} \label{tec::fellstrike}
You use your action to attempt to knock the creature down with a strike.
In addition to taking the attack's normal damage, the creature must succeed on a Strength saving throw (DC 8 + your proficiency bonus + your Strength modifier) or be knocked prone.
If the creature is wearing heavy armor, it rolls this saving throw with disadvantage.

\subsubsection{Focus Strikes} \label{tec::focusstrikes}
As a bonus action, you can search your opponent for vulnerabilities.
Choose a creature you can see within 5 feet of you.
Until the start of your next turn, your attacks deal an extra piercing damage equal to half your proficiency bonus (rounded up) to that creature.

\subsubsection{Goading Attack} \label{tec::goadingattack}
You use your action to attack a creature and goad it into attacking you.
The target must make a Wisdom saving throw with a DC of 8 + your proficiency bonus + your Charisma modifier.
On a failed save, the target has disadvantage on all attack rolls against targets other than you until the end of your next turn.

\subsubsection{Hammer Fling} \label{tec::hammerfling}
As part of you attack action, you can throw any one-handed bludgeoning weapon as if it was a thrown weapon.
The range of this throw is 6/18 meters, and the damage is the same as the weapon's one handed normal damage.

\subsubsection{Heal} \label{tec::heal}
As an action, you can spend one use of a healer's kit to tend to a creature and restore 1d6 + 4 hit points to it, plus additional hit points equal to the creature's number of hit dice.
The creature can't regain hit points from this technique again until it finishes a short or long rest.

\subsubsection{Identify} \label{tec::identify}
You choose one object you are holding or can easily touch.
After analyzing it for a minute, you learn its properties and how to use it.
% If it's a magic item, you learn whether it requires attunement to use, and how many charges it has, if any.
You learn whether any spells are affecting the item and what they are.
If the item was created by a spell, you learn which spell created it.

\subsubsection{Injure} \label{tec::injure}
As an action, you can attack with the sole purpose of injuring a creature.
Roll your weapon attack normally.
On a hit, you deal only half damage, but the creature rolls on the minor injury chart.
If the attack is a critical hit, the creature rolls on the major injury chart.

\subsubsection{Lock} \label{tec::lock}
As an action, you can lock a creature's weapon with yours to prevent it from attacking.
Using one of your weapons, you try to seize one of your target's weapons by making a lock check, a Dexterity (Sleight of Hand) check contested by the target's Dexterity (Sleight of Hand) or Strength (Athletics) check (the target chooses the ability to use).
If you succeed, the target can't attack with the locked weapon.
You can use your bonus action to attack with your other weapon using the two-weapon fighting action after using this technique.

Your target can end this condition by succeeding on a Dexterity (Sleight of Hand) or a Strength (Athletics) check contested by your Dexterity (Sleight of Hand) check at the start of its turn, or by letting go of the weapon.
Additionally, the condition ends if you are incapacitated, if you are removed from the reach of the creature, or if you choose to end the effect (no action required).

\subsubsection{Maneuvering Attack} \label{tec::maneuveringattack}
When you hit a creature with a melee attack, you can use your bonus action to maneuver one of your comrades into a more advantageous position.
You choose a friendly creature who can see or hear you.
That creature can use its reaction to move up to half its speed without provoking opportunity attacks from the target of your attack.

\subsubsection{Mark} \label{tec::mark}
You choose a creature you can see within 27 meters and mark it as your quarry.
For up to one hour, you deal an extra 1d6 damage to the target whenever you hit it with a weapon attack, and you have advantage on any Wisdom (Perception) or Wisdom (Survival) checks you make to find it.
You can only use this technique again after completing a short rest.

\subsubsection{Menacing Attack} \label{tec::menacingattack}
When you hit a creature with a weapon attack, you can use your bonus action to attempt to frighten the target.
The target must make a Wisdom saving throw of a DC of 8 + your proficiency bonus + your Charisma modifier.
On a failed save, it is frightened of you until the end of your next turn.

\subsubsection{Mimic} \label{tec::mimic}
You can mimic the speech of another person or the sounds made by other creatures.
You must have heard the person speaking, or heard the creature make the sound, for at least one minute.
A successful Wisdom (Insight) check contested by your Charisma (Deception) check allows a listener to determine that the effect is faked.

\subsubsection{Precise Attack} \label{tec::preciseattack}
When you make a weapon attack roll against a creature, you can use your bonus action to add a d8 to the attack roll.
You can use this technique before or after making the attack roll, but before any effects of the attack are applied.

\subsubsection{Prepared Shot} \label{tec::preparedshot}
Right at the beginning of combat, after rolling for initiative, you can use your reaction to doff a ranged or thrown weapon, and make a ranged weapon attack with it.

\subsubsection{Pin} \label{tec::pin}
You use your action to try to pin a creature grappled by you.
To do so, make another grapple check.
If you succeed, you and the creature are both restrained until the grapple ends.

\subsubsection{Rally} \label{tec::rally}
On your turn, you can use a bonus action to bolster the resolve of one of your companions.
When you do so, choose a friendly creature who can see or hear you.
That creature gains temporary hit points equal to a d8 + your Charisma modifier.
A creature can only gain temporary hit points in this manner once per encounter.

\subsubsection{Rapid Repair} \label{tec::rapidrepair}
You can attempt to repair a misfired (but not broken) firearm as a bonus action.

\subsubsection{Reckless Shot} \label{tec::recklessshot}
Before you make an attack with a ranged weapon that you are proficient with, you can choose to take a -5 penalty to the attack roll.
If the attack hits, you add +10 to the attack's damage.

\subsubsection{Reckless Strike} \label{tec::recklessstrike}
Before you make a melee attack with a heavy weapon that you are proficient with, you can choose to take a -5 penalty to the attack roll.
If the attack hits, you add +10 to the attack's damage.

\subsubsection{Rondel Hit} \label{tec::rondelhit}
When you take the Attack action and attack with only a stave, spear, or halberd, you can use a bonus action to make a melee attack with the opposite end of the weapon, using the same ability modifier as the main attack.
The weapon's damage die for this attack is a d4, and the attack deals bludgeoning damage.

\subsubsection{Shield Sweep} \label{tec::shieldsweep}
As a bonus action, you can prepare yourself to sweep over targets' shields.
Until the end of this turn, your attack rolls with ignore the AC gained by the target's shield.

\subsubsection{Suplex} \label{tec::suplex}
You can use your Attack action against a grappled creature to make a special melee attack to lift and slam that creature against the floor.
You make a grapple check against the creature.
If you succeed, the creature takes damage as if you hit it with a normal unarmed attack, and both you and the creature are prone.

\subsubsection{Taunt} \label{tec::taunt}
As an action, you make a Charisma (Intimidation) or Charisma (Persuasion) check contested by a creature's Wisdom (Insight) check.
The creature must be able to hear you, and the two of you must share a language.

If you succeed on the check, the creature has disadvantage on attack rolls against targets other than you and can't make opportunity attacks against targets other than you.
This effect lasts for 1 minute, until one of your companions attacks the target or affects it with a spell, or until you and the target are more than 18 meters apart.

\subsubsection{Uncanny Insight} \label{tec::uncannyinsight}
You use your action to try to get uncanny insight about one humanoid you can see within 9 meters of you.
Make a Wisdom (Insight) check contested by the target's Charisma (Deception) check.
If your check succeeds, you have advantage on attack rolls and ability checks against the target until the end of your next turn.

\subsubsection{Visceral Attack} \label{tec::visceralattack}
You can use your bonus action to do a melee attack shooting with a crossbow or pistol.
If the attack hits, your next melee attack on your turn against the same target is done with advantage.

\subsubsection{Violent Shot} \label{tec::violentshot}
When you make a firearm attack against a creature, you can tune your weapon to enhance the volatility of the attack.
The attack gains a +2 to the firearm's misfire score.
If the attack hits, you can roll one additional weapon damage die.

% === EXPERT COMBAT MANEUVERS ================================================== %
\subsubsection{Armor Puncture} \label{tec::armorpuncture}
You use your action to attempt a well-place strike onto a creature's armor or carapace.
If your target has natural, medium, or heavy armor, you strike with the sole purpose of destroying it.
You make a weapon attack with a piercing weapon with which you are proficient, applying a -3 penalty to the attack roll.

If the attack hits, the creature's armor suffers a permanent -1 to its AC bonus, in addition to the damage that the attack would normally do.
This effect is cumulative, and can only be reversed by an appropriate armorer.

\subsubsection{Grievous Wound} \label{tec::grievouswound}
As an action, you can choose to aim an attack to your opponent's arms or object manipulation limbs.
You make a weapon attack with a slashing weapon with which you are proficient, applying a -3 penalty to the attack roll.
If the attack hits, the target has disadvantage on all attack rolls until your next turn, and suffers the damage that the attack would normally do.

\subsubsection{Hemorrhaging Attack} \label{tec::hemorrhagingattack}
As an action, you concentrate to perform one deadly shot at a target.
You make a weapon attack with a ranged weapon with which you are proficient, applying a -3 penalty to the attack roll.
If the attack hits, it is automatically a critical hit.
Additionally, the target suffers half of the damage from the attack at the end of its next turn.

\subsubsection{Parrying Stance} \label{tec::parryingstance}
On your turn, you can use a bonus action to assume a parrying stance, provided you have at least one weapon in hand.
Doing so grants you a +1 bonus to your AC until the start of your next turn or until you're not holding the weapon.
Additionally, you can use the Parry technique without using your reaction.

\subsubsection{Shield Break} \label{tec::shieldbreak}
On a successful melee attack against a creature, you can forego dealing damage to it and instead knock its shield aside momentarily.
The creature subtracts the shield's AC bonus from its AC until the start of your next turn.

If the attack is a critical hit, the shield suffers damage depending on its material:
A wooden shield shatters on impact.
A metal shield is dented, becoming useless until a blacksmith repairs it.
A shield reinforced with leather or linen is unaffected by the critical hit.

\subsubsection{Leg Shot} \label{tec::legshot}
As an action, you can attempt to aim a shot at the legs of a creature.
You make a weapon attack with a ranged weapon with which you are proficient, applying a -3 penalty to the attack roll.
If the attack hits, the target's speed is reduced to 0 until the start of your next turn, in addition to taking the damage that the attack would normally do.

\subsubsection{Whack} \label{tec::whack}
As an action, you can attempt to break your opponent's balance with a well-placed strike.
You make a weapon attack with a bludgeoning weapon with which you are proficient, applying a -3 penalty to the attack roll.
On a hit, the attack deals your normal attack damage, and attack rolls against that creature are made with advantage until the end of your next turn.

% ============================================================================== %
% === MARTIAL ARTS TECHNIQUES ================================================== %
\subsection*{Martial Arts Techniques} \label{tec::martialartstechniques}
Martial arts techniques are special set of techniques that are unlocked when a character takes the Unarmed Artist 3 feat.
% TODO: INVENT A SCHOOL WHERE THIS SKILL IS ACTUALLY TAUGHT, AND RELATE THE ABILITIES TO THAT SCHOOL.

Every time the Unarmed Artist 3 rank is taken, you learn a new martial arts technique.
You can use the techniques in this list a number of times equal to your Wisdom modifier + 2 per combat.

\subsubsection{Deflect Missiles} \label{mtec::deflectmissiles}
You can use your reaction to deflect or catch the missile when you are hit by a ranged weapon attack.
When you do so, the damage you take from the attack is reduced by 1d10 + your Dexterity modifier + your number of hit dice.

\subsubsection{Flurry of Blows} \label{mtec::flurryofblows}
Immediately after you take the Attack action on your turn, you can make two unarmed strikes as a bonus action.

\subsubsection{Patient Defense} \label{mtec::patientdefense}
You can take the Dodge action as a bonus action on your turn.

\subsubsection{Predetermined Fate} \label{mtec::predeterminedfate}
Whenever you make a saving throw and fail, you can reroll it and take the second result.

\subsubsection{Quivering Palm} \label{mtec::quiveringpalm}
This is the ultimate technique of the TODO:SCHOOL school, and can only be learned after you learn all other martial arts techniques.
When you hit a creature with an unarmed strike, you can use the quivering palm technique.
The creature must make a Constitution saving throw or take 10d10 bludgeoning damage, taking half damage on a success.
You can use this technique only once per short rest.

\subsubsection{Step of the Wind} \label{mtec::stepofthewind}
You can take the Disengage or Dash action as a bonus action on your turn, and your jump distance is doubled for that turn.

\subsubsection{Stunning Strike} \label{mtec::stunningstrike}
When you hit another creature with a melee weapon attack, you can attempt a stunning strike.
The target must succeed on a Constitution saving throw with a save DC of 8 + your proficiency bonus + your Wisdom modifier or be stunned until the end of your next turn.

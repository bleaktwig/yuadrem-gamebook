% !TEX root = ../main.tex
\section{Beyond First Level} \label{sec::beyondfirstlevel}
% \DndDropCapLine{W}{ith one heavy loss, the group}
% dispatches the fearsome troll.
% They perform a funerary ritual, to then finally get some rest.
% Beaten and tired, they learned their lesson for the day.

% \subsubsection{Feats and Feat Points}
%     As your character adventures, they will gain \textbf{Feat Points} (FP).
%     Feat points are spent to buy feats, which can only be done as part of a long rest.
%     As your character gains feat points they will gain levels, gaining hit dice and improving their ability scores and saving throw proficiencies.
%
%     % TODO: ADD MORE SHIT HERE
%
% \subsubsection{Beyond 1st Level}
%     % TODO: ADD MORE HERE.
%
%     Ability Score Improvements allow you to increase one of you ability scores by 1.
%     Saving Throw Improvements increase your level of proficiency on a saving throw of your choice by 1.
%     Major Character Advancements are character-defining improvements which include hit dice upgrades, combat styles, and proficiency improvements.
%     Major Character Advancements are listed in their own section in page \pageref{sec::majorcharacteradvancements}.


What is learned by your character during the game is measured by Experience Points (XP).
After you finish playing, talk about the session with the whole group and discuss what happened.
Read the questions in the table below.
For each of them that you can reply ``yes'' to, your character gets one XP.

\begin{itemize}
    \item Did you participate in the session?
    \item Did you travel to a new location?
    \item Did you defeat one or more creatures?
    \item Did you loot treasure?
    \item Did you complete a quest?
    \item Did you solve a conflict?
    \item Did you make a new friend, ally, or enemy?
    \item Did you help another PC?
    \item Did your ideals, bonds, or flaws affect any of your decisions during the session?
    \item Did you perform any action related to your kin, background, or origin?
    \item Did you perform an extraordinary action of some kind?
\end{itemize}

In case of any discussion or disagreement, the DM always has the last word.
Additional XP may be awarded when specific story milestones are achieved, but this too is left to the discression of the DM.

% TODO: CONTINUE FROM HERE ONWARD.
\subsection*{Feat Points}
    When you 15 XP, reset your XP counter and add one FP, keeping the remaining XP.
    Feat points are used to learn new feats, which include either an increase in your proficiency level or a feature.

    Proficiency levels follow a linear progression.
    A blank character is considered \textbf{Untrained} in all skills and proficiencies, which means that they apply no proficiency modifier to their ability checks with any of them.






    Ranks in a feat give a character either a proficiency or a feature.
    Each feat has three ranks, and some act as requirements for others.



    % You don't need to spend TP right after earning it, and you can save as many as you may want for as long as you may need.

    % The rules to learn a new feat or increase your rank in one depend on what is gained through the feat.

    If the rank of the feat involves a proficiency, you need to practice it in five different 4-hour training sessions with a teacher who already knows the feat.
    If you don't have a teacher, you must roll an ability check relevant to the feat after each session to see if it was fruitful.

    If the rank involves a new feature, you need to practice it in one 4-hour training session at your convenience.
    You must roll an ability check or saving throw after the session to see if it was fruitful.
    This task cannot be aided by a teacher.

    Attaining first, second, and third ranks require a DC of 12, 15, and 18 respectively.
    A TP is only spent if a rank is attained successfully.

    % TODO: MENTION THAT ALL FEATS REQUIRE A BOOK OR SOURCE OF KNOWLEDGE WHEN YOU DON'T HAVE A TEACHER

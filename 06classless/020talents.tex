% !TEX root = ../main.tex
\section{Feats}

% TODO: Consider changing the ``relevant ability score or skill'' to just Intelligence.
%       This would give more relevance to the skill in classes that don't actively use it.
%       Discuss with Hernán.

% TODO: Add the explanation: ``unless specified otherwise, a feat can only be learned once''.

% TODO: Write list of Fighting Styles and ``Casting Styles'' (Based on this: https://standsinthefire.com/2016/05/31/dd-5e-magical-styles/).

% TODO: Write list of Metamagic Options.

\subsubsection{Feat Name}
\small{\textcolor{gray}{Relevant Ability Score or Skill}}

\normalsize
Description.
\paragraph{REQUIREMENTS} Other ranks that need to be attained before attaining this one.
Left blank if feat doesn't require anything.

\begin{table*}[b]%
    \begin{DndTable}[width=\linewidth, header=General Feats]{clXX} \label{feat::generalfeats}
        \textbf{Page} & \textbf{Name} & \textbf{Tags} & \textbf{Requirements} \\
        \pageref{feat::acrobat} & Acrobat & acrobatics & - \\
        \pageref{feat::actor} & Actor & performance & Performer 3 \\
        \pageref{feat::aerialist} & Aerialist & acrobatics & Acrobat 3 \\
        \pageref{feat::animalhandler} & Animal Handler & animal handling & - \\
        \pageref{feat::armedfighter} & Armed Fighter & simple weapons & - \\
        \pageref{feat::armorbreaker} & Armor Breaker & martial weapons, medium-haft & Hammer Adept 2 and Heavily Armored 1 \\
        \pageref{feat::athlete} & Athlete & athletics & - \\
        \pageref{feat::axeadept} & Axe Adept & martial weapons, medium-haft & Armed Fighter 2 \\
        \pageref{feat::ballandchainmaster} & Ball and Chain Master & martial weapons, insight & Flail Adept 2 and Insightful 2 \\
        \pageref{feat::bentblademaster} & Bent Blade Master & martial weapons, swords, sleight of hand & Curved Sword Adept 2 and Quick Fingers 2
    \end{DndTable}
\end{table*}

\begin{table*}[b]%
    \begin{DndTable}[width=\linewidth, header=Kin Feats]{clXX} \label{feat::kinfeats}
        \textbf{Page} & \textbf{Name} & \textbf{Tags} & \textbf{Requirements} \\
    \end{DndTable}
\end{table*}

\begin{table*}[b]%
    \begin{DndTable}[width=\linewidth, header=Artisan Feats]{clXX}
        \textbf{Page} & \textbf{Name} & \textbf{Tags} & \textbf{Requirements} \\
    \end{DndTable}
\label{feat::artisanfeats}
\end{table*}

\begin{table*}[b]%
    \begin{DndTable}[width=\linewidth, header=Injury Feats]{clXX} \label{feat::injuryfeats}
        \textbf{Page} & \textbf{Name} & \textbf{Tags} & \textbf{Requirements} \\
    \end{DndTable}
\end{table*}

\subsection*{Feats}
% !TEX root = ../main.tex
\subsubsection{Acrobat} \label{feat::acrobat}
\small{\textcolor{gray}{Acrobatics}}

\normalsize
You've had an exceptional flexibility from a very young age.
By regular training you maintain this ability, aware of its usefulness in your adventuring life.
\paragraph{RANK 1} You are proficient with the Acrobatics skill.
\paragraph{RANK 2} You can make a running long jump or a running high jump after moving only 1.5 meters on foot, rather than 3 meters.
\paragraph{RANK 3} Moving through difficult terrain costs you no extra movement.

% ============================================================================== %
\subsubsection{Actor} \label{feat::actor}
\small{\textcolor{gray}{Performer}}

\normalsize
You master performance so that you can command any stage.
\paragraph{REQUIREMENTS} Performer 3.
\paragraph{RANK 1} You double your proficiency modifier in the Performance skill.
\paragraph{RANK 2} You have advantage on Charisma (Deception) and Charisma (Performance) checks when trying to pass yourself off as a different person.
\paragraph{RANK 3} Increase your Charisma score by 1, to a maximum of 20.

% ============================================================================== %
\subsubsection{Aerialist} \label{feat::aerialist}
\small{\textcolor{gray}{Acrobatics}}

\normalsize
High above the ground, you defy falls to earth on a daily basis.
\paragraph{REQUIREMENTS} Acrobat 3.
\paragraph{RANK 1} You double your proficiency modifier in the Acrobatics skill.
\paragraph{RANK 2} Your walking speed increases by 1.5 meters.
Additionally, you gain a climbing speed and a swimming speed equal to your walking speed.
\paragraph{RANK 3} You learn the Light as a Feather technique.

% ============================================================================== %
\subsubsection{Animal Handler} \label{feat::animalhandler}
\small{\textcolor{gray}{Animal Handling}}

\normalsize
Ever since you were a kid, you've always surrounded yourself with animals, pets or otherwise.
Animals are calmer when they're around you, and you yourself get the same feeling when around them.
\paragraph{RANK 1} You are proficient with the Animal Handling skill.
\paragraph{RANK 2} Animals naturally trust you, and you don't need to perform any check to calm an animal or monster that is not already violent toward you.
\paragraph{RANK 3} You learn the Beast Command technique.

% ============================================================================== %
\subsubsection{Armed Fighter} \label{feat::armedfighter}
\small{\textcolor{gray}{Strength or Dexterity}}

\normalsize
No one learned how to use a sword without first swinging a stick.
\paragraph{RANK 1} You are proficient with simple weapons.
\paragraph{RANK 2} You can attack twice, instead of once, whenever you take the Attack action on your turn.
\paragraph{RANK 3} You learn one technique of your choice between Bonk, Block, Buttstroke, Feint, Lunge, Parry, Pushing Attack, Protect, Quick Draw, Riposte, Rush, Sweep, and Trip.

% ============================================================================== %
\subsubsection{Armor Breaker} \label{feat::armorbreaker}
\small{\textcolor{gray}{Strength}}

\normalsize
You use your extensive knowledge of heavy armor to improve your effectiveness against it, turning your bludgeoning attacks into devastating metal bending strikes.
\paragraph{REQUIREMENTS} Hammer Adept 2 and Heavily Armored 1.
\paragraph{RANK 1} You can use the Bonk technique as an opportunity attack.
\paragraph{RANK 2} When fighting a target with plate armor or natural armor, you get a +2 bonus to your melee attacks with hammers.
\paragraph{RANK 3} You learn the Fell Strike technique.

% ============================================================================== %
\subsubsection{Athlete} \label{feat::athlete}
\small \textcolor{gray}{Athletics}

\normalsize
You begin your day by training, pushing the limits of your body ever so slightly.
\paragraph{RANK 1} You are proficient with the Athletics skill.
\paragraph{RANK 2} When you are prone, standing up uses only 1.5 meters of your movement.
\paragraph{RANK 3} You count as if you were one size larger for the purpose of determining your carrying capacity.

% ============================================================================== %
\subsubsection{Axe Adept} \label{feat::axeadept}
\small{\textcolor{gray}{Strength}}

\normalsize
More than a simple tool, you master the use of axes in combat.
\paragraph{REQUIREMENTS} Armed Fighter 2.
\paragraph{RANK 1} You are proficient with axes.
\paragraph{RANK 2} While wielding an axe, you get advantage on your attack rolls with the Disarm action when forcing your target to drop a shield.
\paragraph{RANK 3} You learn the Rush technique.

% ============================================================================== %
\subsubsection{Ball-and-Chain Master} \label{feat::ballandchainmaster}
\small{\textcolor{gray}{Strength}}

\normalsize
Your increased awareness works in tandem to your flail mastery, allowing you to hit enemies behind obstacles and shields.
\paragraph{REQUIREMENTS} Flail Adept 2 and Insightful 2.
\paragraph{RANK 1} When you use a ball-and-chain, its damage die changes from a d6 to a d8.
\paragraph{RANK 2} When you hit with an opportunity attack using a ball-and-chain, the target must succeed on a Strength saving throw (DC 8 + your proficiency bonus + your Strength modifier) or be knocked prone.
\paragraph{RANK 3} You learn the Shield Sweep technique.

% ============================================================================== %
\subsubsection{Bent Blade Master} \label{feat::bentblademaster}
\small{\textcolor{gray}{Dexterity}}

\normalsize
Your extreme skill with curved swords allows you to cut through flesh with effectiveness and ease.
\paragraph{REQUIREMENTS} Curved Sword Adept 2 and Quick Fingers 2.
\paragraph{RANK 1} While you're wielding at least one curved sword and a creature gets an attack of opportunity on you, they attack with disadvantage.
\paragraph{RANK 2} You get a +2 bonus to your melee weapon attack damage with curved blades.
This extra damage is nullified against targets with chainmail, plate armor, or natural armor.
\paragraph{RANK 3} You learn the Shield Sweep technique.

% ============================================================================== %
\subsubsection{Blade \& Board} \label{feat::bladeandboard}
\small{\textcolor{gray}{Strength}}

\normalsize
You know the perfect balance between defense and attack, making you a formidable foe in melee combat.
\paragraph{REQUIREMENTS} Straight Blade Adept 2 and Shield Training 2, or Axe Adept 2 and Shield Training 2.
\paragraph{RANK 1} When you're wielding a slashing weapon and a shield, you gain a +1 to AC and to your attack damage.
\paragraph{RANK 2} While you're wielding a shield and a creature misses you with a melee attack, you can use your reaction to attack it with a melee attack.
\paragraph{RANK 3} You learn the Bash technique.

% ============================================================================== %
\subsubsection{Blade Breaker} \label{feat::bladebreaker}
\small{\textcolor{gray}{Dexterity}}

\normalsize
You know that the perfect counter against any sword is simply a long pole.
\paragraph{REQUIREMENTS} Staff Fighter 2.
\paragraph{RANK 1} Staves have the Reach property when wielded by you.
\paragraph{RANK 2} You have advantage on the Disarm action and the Parry technique against creatures using any type of sword.
\paragraph{RANK 3} While using a polearm, you can use the Disarm action as your opportunity attack.

% ============================================================================== %
\subsubsection{Blowgun Adept} \label{feat::blowgunadept}
\small{\textcolor{gray}{Dexterity}}

\normalsize
While unconventional, you understand the effectiveness of the blowgun when remaining hidden is of the essence.
\paragraph{REQUIREMENTS} Armed Fighter 1.
\paragraph{RANK 1} You are proficient with blowguns.
\paragraph{RANK 2} You gain a +1 bonus to attack rolls with a blowgun.
Additionally, attacking with a blowgun at long range doesn't impose disadvantage on your ranged weapon attack rolls.
\paragraph{RANK 3} When you hit with a blowgun while hidden, your location is not revealed.

% ============================================================================== %
\subsubsection{Blunt Thrower} \label{feat::bluntthrower}
\small{\textcolor{gray}{Strength}}

\normalsize
You take advantage of the raw practicality of a good hammer, and not even faraway foes can escape the wrath of its face.
\paragraph{REQUIREMENTS} Hammer Adept 2 and Thrown Weapon Master 2.
\paragraph{RANK 1} You don't provoke opportunity attacks when picking weapons up from the floor.
\paragraph{RANK 2} When you hit a creature with an thrown bludgeoning weapon, you divide its moving speed by half (rounded up) during its next turn.
This effect does not stack.
\paragraph{RANK 3} You learn the Hammer Fling technique.

% ============================================================================== %
\subsubsection{Blunt Master} \label{feat::bluntmaster}
\small{\textcolor{gray}{Dexterity}}

\normalsize
Your staff is an extension of your body, and it has become an essential part of your movement and defense.
\paragraph{REQUIREMENTS} Staff Fighter 2 and Unarmed Artist 2.
\paragraph{RANK 1} As a bonus action, you can use your staff to propel yourself into the air, making a high jump up to your Dexterity modifier times 1.5 meters.
Additionally, if you have moved at least 3 meters during this round, you can jump in a straight line up to a distance equal to your Dexterity modifier times 3 meters.
\paragraph{RANK 2} You gain a +1 bonus to your AC while using a staff.
Additionally, if you use the weapon with two hands you gain another +1 bonus to your AC.
\paragraph{RANK 3} You learn the Defensive Stance technique.

% ============================================================================== %
\subsubsection{Bow Adept} \label{feat::bowadept}
\small{\textcolor{gray}{Dexterity}}

\normalsize
While a long list of new ranged weapons have recently appeared, you know that nothing beats the long range of the traditional bow.
\paragraph{REQUIREMENTS} Armed Fighter 2.
\paragraph{RANK 1} You are proficient with bows.
\paragraph{RANK 2} After you successfully hit a target with a melee attack using your bow, that creature can't make opportunity attacks against you until the start of your next turn.
\paragraph{RANK 3} You learn the Quick Draw technique.

% ============================================================================== %
\subsubsection{Buckler Training} \label{feat::bucklertraining}
\small{\textcolor{gray}{Dexterity}}

\normalsize
Small shields are much more than a weaker version of the standard shields.
They allow you to remain mobile while still providing a solid defense against even the fastest assailants.
\paragraph{RANK 1} You gain proficiency with light shields.
\paragraph{RANK 2} In combat, you can don or doff a light shield as the free object interaction of your turn.
\paragraph{RANK 3} You learn the Parry technique.

% ============================================================================== %
\subsubsection{Bullet Tinkerer} \label{feat::bullettinkerer}
\small{\textcolor{gray}{Science}}

\normalsize
Firearm ammunition is near impossible to find or purchase in most parts of Yuadrem.
Due to this, learning how to craft it yourself is an essential aspect for using firearms.
\paragraph{REQUIREMENTS} Musket Adept 2 and Educated 2, or Pistol Adept 2 and Educated 2.
\paragraph{RANK 1} You can craft ammunition using a set of Tinker's Tools at half the cost.
\paragraph{RANK 2} If you roll a misfire, you can use your reaction to roll a d20 with disadvantage.
If the number rolled is higher than the weapon's misfire score, the weapon does not misfire.
\paragraph{RANK 3} You learn the Violent Shot technique.

% ============================================================================== %
\subsubsection{Cognizant} \label{feat::cognizant}
\small{\textcolor{gray}{History}}

\normalsize
From background or perchance you've been given a privileged access to books and teaching thorough your life.
\paragraph{RANK 1} You are proficient with the History skill.
\paragraph{RANK 2} You have advantage on any Intelligence (History) checks to recall information of your kin and your country of origin.
\paragraph{RANK 3} You learn the Wizened Advice technique.

% ============================================================================== %
\subsubsection{Combat Improviser} \label{feat::combatimproviser}
\small{\textcolor{gray}{Strength or Dexterity}}

\normalsize
Anything can be your weapon.
If you can grab it, you can swing it.
\paragraph{REQUIREMENTS} Armed Fighter 1.
\paragraph{RANK 1} You are proficient with improvised weapons.
An improvised weapon is anything that is somewhat similar to a simple weapon.
\paragraph{RANK 2} After rolling for initiative, you make your first attack with an improvised weapon at advantage.
\paragraph{RANK 3} You learn one technique of your choice between Bonk, Block, Buttstroke, Feint, Lunge, Parry, Pushing Attack, Protect, Quick Draw, Riposte, Rush, Sweep, and Trip.

% ============================================================================== %
\subsubsection{Covered Shooter} \label{feat::coveredshooter}
\small{\textcolor{gray}{Dexterity}}

\normalsize
You know that stealth and range go hand in hand.
\paragraph{REQUIREMENTS} Bow Adept 2 and Sly 2.
\paragraph{RANK 1} If you miss with a ranged attack with a bow or thrown weapon while hidden, the attack doesn't reveal your location, and you remain hidden.
\paragraph{RANK 2} You can treat ranged attack damage die rolls of 1 as 2.
\paragraph{RANK 3} You learn or improve the Sneak Attack technique.

% ============================================================================== %
\subsubsection{Crossbow Adept} \label{feat::crossbowadept}
\small{\textcolor{gray}{Dexterity}}

\normalsize
You are learned with the use of crossbow, and can use it proficiently both for hunting and combat.
\paragraph{REQUIREMENTS} Armed Fighter 2.
\paragraph{RANK 1} You are proficient with crossbows.
\paragraph{RANK 2} You can change the loading quality of crossbows to reloading.
% You can ignore the loading quality of crossbows.
\paragraph{RANK 3} You learn the Buttstroke technique.

% ============================================================================== %
\subsubsection{Crossbow Expert} \label{feat::crossbowexpert}
\small{\textcolor{gray}{Dexterity}}

\normalsize
While muskets and pistols have largely replaced the old weapons of warfare, you prefer a weapon that won't blow up on your hand every so often.
\paragraph{REQUIREMENTS} Crossbow Adept 2 and Quick Fingers 2.
\paragraph{RANK 1} Being within 1.5 meters of a hostile creature doesn't impose disadvantage on your ranged attack rolls.
Additionally, you don't provoke attacks of opportunity when using the Attack action with a ranged weapon when within the reach of a creature.
\paragraph{RANK 2} You get a +2 bonus to damage rolls when attacking a creature within 1.5 meters of you with a crossbow.
\paragraph{RANK 3} You learn the Prepared Shot technique.

% ============================================================================== %
\subsubsection{Crusher} \label{feat::crusher}
\small{\textcolor{gray}{Strength}}

\normalsize
You are practiced in the art of crushing your enemies.
\paragraph{REQUIREMENTS} Proficiency with at least two martial bludgeoning weapon types.
\paragraph{RANK 1} Once per turn, when you hit a creature with an attack that deals bludgeoning damage, you can move it 1.5 meters to an unoccupied space, provided that the target is no more than one size larger than you.
\paragraph{RANK 2} Increase your Strength by 1, to a maximum of 20.
\paragraph{RANK 3} You learn the Whack technique.

% ============================================================================== %
\subsubsection{Curved Sword Adept} \label{feat::curvedswordadept}
\small{\textcolor{gray}{Strength or Dexterity}}

\normalsize
You forfeited the piercing capacity of the straight blades to maximize your deadliness against the unarmored.
\paragraph{REQUIREMENTS} Armed Fighter 2.
\paragraph{RANK 1} You are proficient with curved swords.
\paragraph{RANK 2} While wielding a curved sword, you get advantage on your attack rolls with the Disarm action when forcing your target to drop a weapon.
\paragraph{RANK 3} You learn the Parry technique.

% ============================================================================== %
\subsubsection{Cutthroat} \label{feat::cutthroat}
\small{\textcolor{gray}{Dexterity}}

\normalsize
A master at concealing your dagger, you act quick and strike hard in combat.
\paragraph{REQUIREMENTS} Dagger Savant 2.
\paragraph{RANK 1} You have advantage on any check to conceal a small weapon on your person.
\paragraph{RANK 2} You gain a +2 bonus to initiative. % when wielding a light weapon.
\paragraph{RANK 3} You gain or improve the Sneak Attack technique.

% ============================================================================== %

% !TEX root = ../main.tex
\subsubsection{Dagger Savant} \label{feat::daggersavant}
\small{\textcolor{gray}{Dexterity}}

\normalsize
You wield the lightest of blades with the deadliest of skills.
\paragraph{REQUIREMENTS} Armed Fighter 1.
\paragraph{RANK 1} You can perform one melee attack with a dagger as part of a Grapple, Escape a Grapple, Overrun, or Tumble action.
\paragraph{RANK 2} You gain or improve the Sneak Attack technique.
\paragraph{RANK 3} You learn the Riposte technique.

% ============================================================================== %
\subsubsection{Defensive Duelist} \label{feat::defensiveduelist}
\small{\textcolor{gray}{Dexterity}}

\normalsize
A master of the buckler, you have learned to use it as an addition to your main weapon.
\paragraph{REQUIREMENTS} Buckler Training 2.
\paragraph{RANK 1} When you use the Parry technique, you can roll a d8 instead of a d6 when increasing your AC.
\paragraph{RANK 2} When fighting with a one-handed melee weapon and a light shield, you can add half your shield's AC (rounded up) to the weapon's attack damage.
\paragraph{RANK 3} When you use your Parry technique and the attacking creature misses, you can use the Disarm action against it as part of your reaction.

% ============================================================================== %
\subsubsection{Deft Explorer} \label{feat::deftexplorer}
\small{\textcolor{gray}{Nature}}

\normalsize
You are an unsurpassed explorer, using your uncanny knowledge of nature to understand and survive in any environment.
\paragraph{REQUIREMENTS} Naturalist 3.
\paragraph{RANK 1} You double your proficiency modifier in the Nature skill.
\paragraph{RANK 2} By almost supernatural perception, you can sense the presence of poisons and poisonous creatures within 9 meters of you.
\paragraph{RANK 3} Increase your Intelligence score by 1, to a maximum of 20.

% ============================================================================== %
\subsubsection{Detective} \label{feat::detective}
\small{\textcolor{gray}{Investigation}}

\normalsize
You excel at rooting out secrets and unraveling mysteries.
\paragraph{REQUIREMENTS} Inquisitive 3.
\paragraph{RANK 1} You double your proficiency modifier in the Investigation skill.
\paragraph{RANK 2} You have a +5 bonus to your passive intelligence (Investigation) score.
\paragraph{RANK 3} Increase your Intelligence score by 1, to a maximum of 20.

% ============================================================================== %
\subsubsection{Dual Wielder} \label{feat::dualwielder}
\small{\textcolor{gray}{Dexterity}}

\normalsize
You master fighting with two weapons.
\paragraph{REQUIREMENTS} Dexterity 13. Armed Fighter 2.
\paragraph{RANK 1} When you engage in two-weapon fighting, you can add your ability modifier to the damage of the second attack.
\paragraph{RANK 2} You can draw or stow two one-handed weapons when you would normally be able to draw or stow only one.
\paragraph{RANK 3} You learn the Lock technique.

% ============================================================================== %
\subsubsection{Educated} \label{feat::educated}
\small{\textcolor{gray}{Science}}

\normalsize
From background or perchance you've been given a privileged access to books and teaching thorough your life.
You have an advanced understanding of the natural laws of the world.
\paragraph{RANK 1} You are proficient with the Science skill.
\paragraph{RANK 2} You have advantage on the Identify a Spell action.
\paragraph{RANK 3} You have an intuitive understanding of how to use all sorts of machinery, and can operate siege weapons and heavy machines without performing any ability checks.

% ============================================================================== %
\subsubsection{Empathetic} \label{feat::empathetic}
\small{\textcolor{gray}{Insight}}

\normalsize
You possess keen insight into how other people think and feel.
\paragraph{REQUIREMENTS} Insightful 3.
\paragraph{RANK 1} You double your proficiency modifier in the Insight skill.
\paragraph{RANK 2} You learn the Uncanny Insight technique.
\paragraph{RANK 3} Increase your Wisdom score by 1, to a maximum of 20.

% ============================================================================== %
\subsubsection{Estoc Master} \label{feat::estocmaster}
\small{\textcolor{gray}{Dexterity}}

\normalsize
A master of thrust, you prefer to focus only on piercing and abandon the slashing capabilities of the rapier.
\paragraph{REQUIREMENTS} Light Sword Adept 2.
\paragraph{RANK 1} Whenever you miss with a melee weapon attack, you can use your reaction to reroll that attack if you're wielding an estoc.
\paragraph{RANK 2} You can attack twice instead of once when you use your reaction to attack with the Riposte technique.
\paragraph{RANK 3} While wielding an estoc, you can attack one additional time when you use the attack action.

% ============================================================================== %
\subsubsection{Far Thruster} \label{feat::farthruster}
\small{\textcolor{gray}{Strength or Dexterity}}

\normalsize
You take advantage of your long weapon to keep your distance from foes.
\paragraph{REQUIREMENTS} Halberd Adept 2 and Spear Adept 2.
\paragraph{RANK 1} After successfully attacking a creature, you can use the Shove action as a bonus action.
\paragraph{RANK 2} You gain a +2 to your attack rolls made with a polearm against a creature larger than you.
\paragraph{RANK 3} You learn the Extend Reach technique.

% ============================================================================== %
\subsubsection{Fearsome} \label{feat::fearsome}
\small{\textcolor{gray}{Intimidation}}

\normalsize
You become fearsome to others, and people are scared to even talk about you.
\paragraph{REQUIREMENTS} Menacing 3.
\paragraph{RANK 1} You double your proficiency modifier in the Intimidation skill.
\paragraph{RANK 2} You learn the Demoralize technique.
\paragraph{RANK 3} Increase your Charisma score by 1, to a maximum of 20.

% ============================================================================== %
\subsubsection{Fell Hand} \label{feat::fellhand}
\small{\textcolor{gray}{Strength}}

\normalsize
You are an expert with the greatclub, and can use its raw power to pancake even the hardiest foes.
\paragraph{REQUIREMENTS} Hammer Adept 2.
\paragraph{RANK 1} When you use a greatclub, its damage die changes from a d8 to a d10.
\paragraph{RANK 2} Whenever you miss with a melee attack using a greatclub, the target takes bludgeoning damage equal to your Strength modifier (minimum of 2).
\paragraph{RANK 3} When your target fails its saving throw against your Bonk technique, it also falls prone.

% ============================================================================== %
\subsubsection{Fencer} \label{feat::fencer}
\small{\textcolor{gray}{Dexterity}}

\normalsize
More than a mere combatant, you are able to dance in the battlefield when using your light blades.
\paragraph{REQUIREMENTS} Light Sword Adept 2 and Performer 2.
\paragraph{RANK 1} You gain a +1 to melee attack rolls and AC when only one creature is within 5 feet of you.
\paragraph{RANK 2} When you use your Parry technique and the attacking creature misses, you can use the Riposte technique without using your reaction.
\paragraph{RANK 3} You learn the Focus Strikes technique.

% ============================================================================== %
\subsubsection{Firearm Specialist} \label{feat::firearmspecialist}
\small{\textcolor{gray}{Dexterity}}

\normalsize
You are particularly well at combat with pistols, and are a shining example of this new weapon's viability in combat.
\paragraph{REQUIREMENTS} Pistol Adept 2.
\paragraph{RANK 1} You can reload any weapon as a bonus action.
\paragraph{RANK 2} If you are using a pistol in one hand and nothing on the other, you get a +2 bonus to your ranged weapon attacks with it.
\paragraph{RANK 3} You learn the Rapid Repair technique.

% ============================================================================== %
\subsubsection{Flail Adept} \label{feat::flailadept}
\small{\textcolor{gray}{Strength or Dexterity}}

\normalsize
The flail is a tricky weapon to use, but you have spent countless hours mastering it.
\paragraph{REQUIREMENTS} Armed Fighter 2.
\paragraph{RANK 1} You are proficient with flails.
\paragraph{RANK 2} When wielding flails, enemies provoke opportunity attacks from you when they enter your reach.
\paragraph{RANK 3} You learn the Sweep technique.

% ============================================================================== %

% !TEX root = ../main.tex
\subsubsection{Giant Slayer} \label{feat::giantslayer}
\small{\textcolor{gray}{Strength}}

\normalsize
Partaking or not in Krudzal's Eternal War, you use your incredible strength to wield their massive weapons.
\paragraph{REQUIREMENTS} Strength 16. Great Weapon User 2.
\paragraph{RANK 1} You are able to use Krudzal's giant-slayers.
\paragraph{RANK 2} You gain a +1 bonus to damage rolls with weapons with the Great property.
\paragraph{RANK 3} When fighting with a melee giant-slayer, you can attack up to three creatures in range with one attack.
Additionally, you can reload the hand ballista as a bonus action instead of an action.

% ============================================================================== %
\subsubsection{Glaive Master} \label{feat::glaivemaster}
\small{\textcolor{gray}{Dexterity}}

\normalsize
Different than the common polearm, you know how deadly a glaive can be in capable hands.
\paragraph{REQUIREMENTS} Halberd Adept 2.
\paragraph{RANK 1} When you use a glaive, its damage die changes from a d10 to a d12.
\paragraph{RANK 2} The glaive has the finesse property for you.
\paragraph{RANK 3} When you miss an attack, you can attempt to strike your enemy with the rondel of your glaive as a free action.
On a hit, the target takes 1d4 + your Strength modifier bludgeoning damage.
You can do this once per turn.

% ============================================================================== %
\subsubsection{Grappler} \label{feat::grappler}
\small{\textcolor{gray}{Strength}}

\normalsize
You've developed the skills necessary to hold your own in close-quarters grappling.
\paragraph{REQUIREMENTS} Pugilist 2.
\paragraph{RANK 1} You have advantage on attack rolls against creatures you are grappling.
\paragraph{RANK 2} You benefit from three-quarters cover while you grapple a creature of a size equal or greater than yours.
\paragraph{RANK 3} You learn the Pin technique.

% ============================================================================== %
\subsubsection{Great Weapon User} \label{feat::greatweaponuser}
\small{\textcolor{gray}{Strength}}

\normalsize
You've learned to put the weight of a weapon to your advantage, letting its momentum empower your strikes.
\paragraph{REQUIREMENTS} Strength 13. Armed Fighter 2.
\paragraph{RANK 1} You can use weapons with the great property.
Some of these, like the zweihander and the greataxe, also require you to have proficiency with the corresponding weapon type.
\paragraph{RANK 2} On your turn, when you score a critical hit with a melee weapon or reduce a creature to 0 hit points with one, you can make one melee weapon attack as a bonus action.
\paragraph{RANK 3} You learn the Reckless Strike technique.

% ============================================================================== %
\subsubsection{Greatshield Training} \label{feat::greatshieldtraining}
\small{\textcolor{gray}{Strength}}

\normalsize
When it comes to protecting your life or that of others, you know that size matters.
\paragraph{REQUIREMENTS} Shield Training 1.
\paragraph{RANK 1} You gain proficiency with heavy shields
\paragraph{RANK 2} Creatures standing behind you get three-fourths cover against ranged attacks.
\paragraph{RANK 3} You learn the Protect technique.

% ============================================================================== %
\subsubsection{Gunner} \label{feat::gunner}
\small{\textcolor{gray}{Dexterity}}

\normalsize
Being a new breed of weapon, firearms are constantly subject to change and refinement.
New weapons appear by the minute, and you are an expert at experimenting with them.
\paragraph{REQUIREMENTS} Pistol Adept 2 and Musket Adept 2.
\paragraph{RANK 1} As long as you can examine the weapon for 30 seconds, you are proficient with any kind of firearm, even if it is new or experimental.
\paragraph{RANK 2} Your firearm attacks score a critical hit on a roll of 19-20.
\paragraph{RANK 3} You learn the Reckless Shot technique.

% ============================================================================== %
\subsubsection{Haggler} \label{feat::haggler}
\small{\textcolor{gray}{Persuasion}}

\normalsize
Your abilities at haggling and persuading are known in your local markets, and you are both feared and revered by merchants and traders.
\paragraph{REQUIREMENTS} Persuasive 3.
\paragraph{RANK 1} You double your proficiency modifier in the Persuasion skill.
\paragraph{RANK 2} Your well-honed haggling skills allow you to shave off 10\% off the price of anything, even if you fail a Charisma (Persuasion) check.
\paragraph{RANK 3} Increase your Charisma score by 1, to a maximum of 20.

% ============================================================================== %
\subsubsection{Halberd Adept} \label{feat::halberdadept}
\small{\textcolor{gray}{Strength or Dexterity}}

\normalsize
You know that both farmers and nobles use halberds as their weapon of choice for a reason.
\paragraph{REQUIREMENTS} Armed Fighter 2 and Staff Fighter 2.
\paragraph{RANK 1} You are proficient with the many varieties of halberds.
\paragraph{RANK 2} When a creature within 1.5 meters of you makes an attack against a target other than you, you can use your reaction to make a melee weapon attack against the attacking creature.
\paragraph{RANK 3} You learn the Trip technique.

% ============================================================================== %
\subsubsection{Hammer Adept} \label{feat::hammeradept}
\small{\textcolor{gray}{Strength}}

\normalsize
If a hammer will do, why complicate things?
\paragraph{REQUIREMENTS} Armed Fighter 2.
\paragraph{RANK 1} You gain proficiency with hammer weapons.
\paragraph{RANK 2} If you use the Help action to aid an ally's melee attack while you're wielding a hammer weapon, you knock the target's shield aside momentarily.
In addition to the ally gaining advantage on the attack roll, the ally gains a bonus to the roll equal to the shield's AC bonus.
\paragraph{RANK 3} You learn the Bonk technique.

% ============================================================================== %
\subsubsection{Hand Crossbow Master} \label{feat::handcrossbowmaster}
\small{\textcolor{gray}{Dexterity}}

\normalsize
You can equip, load, and shoot a hand crossbow in an instant, and with deadly effect.
\paragraph{REQUIREMENTS} Crossbow Adept 2.
\paragraph{RANK 1} You can reload a hand crossbow or one-handed firearm without having a free hand.
\paragraph{RANK 2} You can ignore the loading (or reloading) quality of hand crossbows.
\paragraph{RANK 3} You can don and you can doff a one-handed ranged weapon as your free object interaction during your turn.
The two actions don't need to be done in tandem - for example, you can don a hand crossbow, shoot an arrow, and doff it during one turn.

% ============================================================================== %
\subsubsection{Heavily Armored} \label{feat::heavilyarmored}
\small{\textcolor{gray}{Strength}}

\normalsize
You are trained to use heavy armor.
\paragraph{REQUIREMENTS} Moderately Armored 1.
\paragraph{RANK 1} You gain proficiency with heavy armor.
\paragraph{RANK 2} Equipped heavy armor and metal accessories only add half their weight to your encumbrance.
\paragraph{RANK 3} Increase your Strength by 1, to a maximum of 20.

% ============================================================================== %
\subsubsection{Heavy Armor Master} \label{feat::heavyarmormaster}
\small{\textcolor{gray}{Strength}}

\normalsize
You can use your armor to deflect strikes that would easily kill others.
\paragraph{REQUIREMENTS} Heavily Armored 2 and Athletics 2.
\paragraph{RANK 1} Your hit point maximum increases by an amount equal to your number of hit dice.
Whenever you earn a hit die thereafter, your hit point maximum increases by an additional 1 hit point.
\paragraph{RANK 2} While you are wearing heavy armor, piercing and slashing damage that you take is reduced by 3.
\paragraph{RANK 3} You learn the Defend technique.

% ============================================================================== %
\subsubsection{Heavy Lifter} \label{feat::heavylifter}
\small{\textcolor{gray}{Strength}}

\normalsize
You can easily lift and hurl objects that others can barely move.
\paragraph{REQUIREMENTS} Combat Improviser 2 and Thrown Weapon Master 2.
\paragraph{RANK 1} You can hurl any object or creature you can carry or grapple as you would an improvised weapon.
If you hurl a creature, both it and the target of your attack take 1d6 + your Strength modifier bludgeoning damage.
\paragraph{RANK 2} You count as if you were one size larger for the purpose of determining your carrying capacity.
\paragraph{RANK 3} You learn the Suplex technique.

% ============================================================================== %
\subsubsection{Heavy-Weight Combatant} \label{feat::heavyweightcombatant}
\small{\textcolor{gray}{Athletics}}

\normalsize
You can use the weight of your body and your armor to greatly aid you in combat.
\paragraph{REQUIREMENTS} Heavily Armored 2 and Pugilist 2.
\paragraph{RANK 1} If you are wearing metal gauntlets, your unarmed strike uses a d6 + your Strength modifier for damage.
\paragraph{RANK 2} While you are wearing heavy armor, you roll with advantage when taking the Shove action.
You also roll Strength (Athletics) with advantage when you are targeted by a Shove action.
\paragraph{RANK 3} You have learned to strategically position the crevices in your armor to hurt creatures too close to you.
When you grapple a creature, the target takes 1d6 piercing damage if your grapple check succeeds.
If another creature grapples you, they take 1d6 piercing damage.
If you start your turn grappling or being grappled by a creature, it takes 1d6 piercing damage.

% ============================================================================== %
\subsubsection{Historian} \label{feat::historian}
\small{\textcolor{gray}{History}}

\normalsize
Your extensive knowledge in the social sciences allows you to understand people, societies, and history with ease.
\paragraph{REQUIREMENTS} Cognizant 3.
\paragraph{RANK 1} You double your proficiency modifier in the History skill.
\paragraph{RANK 2} You can accurately recall anything you have seen or heard within the past month.
\paragraph{RANK 3} Increase your Intelligence score by 1, to a maximum of 20.

% ============================================================================== %
\subsubsection{Horse Killer} \label{feat::horsekiller}
\small{\textcolor{gray}{Animal Handling}}

\normalsize
Mounts are one of the most common advantages in war, so you've learned to deal with them appropriately.
\paragraph{REQUIREMENTS} Spear Adept 2 and Animal Handler 2.
\paragraph{RANK 1} You can always target a rider's mount with an attack made with a spear, even if the rider can normally deflect this attack to it.
\paragraph{RANK 2} You have a +2 to your attack rolls made against mounts.
\paragraph{RANK 3} You learn the Charge Stopper technique.

% ============================================================================== %
\subsubsection{Immovable Object} \label{feat::immovableobject}
\small{\textcolor{gray}{Strength}}

\normalsize
You are an obstacle.
\paragraph{REQUIREMENTS} Greatshield Training 2.
\paragraph{RANK 1} Creatures can't use a Tumble action to move through your space.
Additionally, you have advantage on Strength (Athletics) checks against being shoved or pushed.
\paragraph{RANK 2} While using a heavy shield, you get three-fourths cover against ranged attacks.
Additionally, you can duck behind your shield as a reaction, granting you full cover against ranged attacks until the start of your next turn.
\paragraph{RANK 3} The movement speed debuff from using a tower shield is halved for you.

% ============================================================================== %
\subsubsection{Inquisitive} \label{feat::inquisitive}
\small{\textcolor{gray}{Investigation}}

\normalsize
No detail can miss your analytical mind.
Your extensive experience and awareness means you can always tell when something's amiss.
\paragraph{RANK 1} You are proficient with the Investigation skill.
\paragraph{RANK 2} Your inquisitiveness allows you to tell when you are missing a detail or clue in a situation, even after a failed roll.
This doesn't allow you to roll again, but the lingering uneasiness stays in your mind.
\paragraph{RANK 3} You can take the Search action as a bonus action.

% ============================================================================== %
\subsubsection{Insightful} \label{feat::insightful}
\small{\textcolor{gray}{Insight}}

\normalsize
Either by primal intuition or extensive knowledge, you are always in complete awareness of your surroundings.
It's very rare for you to be surprised, and getting lost is something that simply does not happen to you.
\paragraph{RANK 1} You are proficient with the Insight skill.
\paragraph{RANK 2} You always know which way is north, and you always know the number of hours left before the next sunrise or sunset.
\paragraph{RANK 3} You have advantage on Wisdom (Perception) and Intelligence (Investigation) checks made to detect the presence of secret doors and traps.

% ============================================================================== %
\subsubsection{Insistent Fighter} \label{feat::insistentfighter}
\small{\textcolor{gray}{Strength or Dexterity}}

\normalsize
You always maintain your foes at a cozy distance by masterfully manipulating your halberd.
\paragraph{REQUIREMENTS} Halberd Adept 2 and Observant 2.
\paragraph{RANK 1} Creature's provoke opportunity attacks from you even if they take the Disengage action before leaving your reach.
\paragraph{RANK 2} When you hit a creature with an opportunity attack, the creature's speed becomes 0 for the rest of the turn.
\paragraph{RANK 3} Your reach for opportunity attacks with polearms is of at least 3 meters.
Additionally, if you do hit a creature with an opportunity attack, you can choose to pull it towards you by 1.5 meters.

% ============================================================================== %

% !TEX root = ../main.tex
\subsubsection{Kama Master} \label{feat::kamamaster}
\small{\textcolor{gray}{Dexterity}}

\normalsize
From among the common people weapon, you specialize in the use of the kama.
\paragraph{REQUIREMENTS} Pick Adept 2.
\paragraph{RANK 1} When you use a kama, its damage die changes from a d6 to a d8
\paragraph{RANK 2} When you successfully attack a foe with a piercing attack using your kama, you also deal 1d4 slashing damage, unmodified by your dexterity.
\paragraph{RANK 3} You can use the Trip technique as part of an opportunity attack.

% ============================================================================== %
\subsubsection{Knowledgeable} \label{feat::knowledgeable}
\small{\textcolor{gray}{Science}}

\normalsize
Your extensive knowledge in the natural sciences gives you access to a plethora of explanations to the phenomena of the world.
\paragraph{REQUIREMENTS} Educated 3.
\paragraph{RANK 1} You double your proficiency modifier in the Science skill.
\paragraph{RANK 2} You learn the Identify technique.
\paragraph{RANK 3} Increase your Intelligence score by 1, to a maximum of 20.

% ============================================================================== %
\subsubsection{Liar} \label{feat::liar}
\small{\textcolor{gray}{Deception}}

\normalsize
You can easily lie to someone to their face without batting an eye.
While you might not always be proud of this skill, you can't deny that it's saved you in countless situations.
\paragraph{RANK 1} You are proficient with the Deception skill.
\paragraph{RANK 2} When you fail to convince someone with a lie, it'll still take them one turn to realize you're lying, provided the lie isn't something absurd.
\paragraph{RANK 3} You learn the Mimic technique.

% ============================================================================== %
\subsubsection{Light Armor Master} \label{feat::lightarmormaster}
\small{\textcolor{gray}{Dexterity}}

\normalsize
You make effective use of the increased mobility that comes from the use of light armor.
\paragraph{REQUIREMENTS} Lightly Armored 2.
\paragraph{RANK 1} You can use the Dodge action even if you speed is 0.
\paragraph{RANK 2} If you aren't incapacitated, you can add a +2 bonus to any Dexterity saving throw you make against a spell or other harmful effect that targets only you.
\paragraph{RANK 3} Your learn the Lunge technique.

% ============================================================================== %
\subsubsection{Light Sword Adept} \label{feat::lightswordadept}
\small{\textcolor{gray}{Dexterity}}

\normalsize
Focused on piercing throw your opponent's armor, you've learned the art of fighting with light swords.
\paragraph{REQUIREMENTS} Armed Fighter 2.
\paragraph{RANK 1} You are proficient with light swords.
\paragraph{RANK 2} When you are wielding a light sword in one hand and no other weapons, you gain a +2 bonus to damage rolls with that weapon.
\paragraph{RANK 3} You learn the Riposte technique.

% ============================================================================== %
\subsubsection{Lightly Armored} \label{feat::lightlyarmored}
\small{\textcolor{gray}{Strength or Dexterity}}

\normalsize
You are trained in the use of light armor.
\paragraph{RANK 1} You gain proficiency with light armor.
\paragraph{RANK 2} Equipped light armor and cloth accessories don't add to your encumbrance.
\paragraph{RANK 3} Increase your Dexterity by 1, to a maximum of 20.

% ============================================================================== %
\subsubsection{Linguist} \label{feat::linguist}
\small{\textcolor{gray}{Languages}}

\normalsize
Your extensive knowledge in tongues is manifested in an enhanced ability recognizing languages and inventing codes of your own.
\paragraph{RANK 1} You are proficient in the Languages skill.
\paragraph{RANK 2} You have advantage on Charisma (Languages) checks to determine the language that a creature is speaking.
\paragraph{RANK 3} You can ably create ciphers.
Others can't decipher a code you create unless you teach them or they succeed on an Intelligence check (DC equal to your Intelligence score + your proficiency bonus).

% ============================================================================== %
\subsubsection{Lip-reader} \label{feat::lipreader}
\small{\textcolor{gray}{Languages}}

\normalsize
You have an uncanny capacity with languages and understanding speakers even without actually hearing them.
\paragraph{REQUIREMENTS} Linguist 3.
\paragraph{RANK 1} You double your proficiency modifier in the Languages skill.
\paragraph{RANK 2} If you see a creature's mouth while it is speaking a language you understand, you can interpret what it's saying by reading its lips.
\paragraph{RANK 3} Increase your Charisma score by 1, to a maximum of 20.

% ============================================================================== %
\subsubsection{Longsword Master} \label{feat::longswordmaster}
\small{\textcolor{gray}{Strength or Dexterity}}

\normalsize
Your proficiency with the longsword is unparalleled, and you regularly prove to your foes that the old sword is as valid today as it was on its golden age.
\paragraph{REQUIREMENTS} Straight Sword Adept 2.
\paragraph{RANK 1} When you use a longsword, its damage die changes from a d10 to a d12.
\paragraph{RANK 2} When a creature misses you with a melee weapon attack, you can roll to use the Disarm action on it as your reaction.
\paragraph{RANK 3} When you use the Block technique and an attack misses you, you can use the Riposte technique without using your reaction.

% ============================================================================== %

% !TEX root = ../main.tex
\subsubsection{Maiming Strikes} \label{feat::maimingstrikes}
\small{\textcolor{gray}{Strength}}

\normalsize
Your increased proficiency cleaving through wood and stone translates to devastating attacks against flesh and bone.
\paragraph{REQUIREMENTS} Axe Adept 2 and Pick Adept 2.
\paragraph{RANK 1} The last attack you make with the Attack action is rolled with advantage if you hit the target at least once in your turn.
\paragraph{RANK 2} When a creature rolls on the minor \& major injury charts from one of your attacks, they roll with advantage.
\paragraph{RANK 3} You learn the Cleave technique.

% ============================================================================== %
\subsubsection{Medic} \label{feat::medic}
\small{\textcolor{gray}{Medicine}}

\normalsize
You mastered the physician's arts, and are able to heal any wound or bruise with ease.
\paragraph{REQUIREMENT} Mender 3.
\paragraph{RANK 1} You double your proficiency modifier in the Medicine skill.
\paragraph{RANK 2} When you heal a creature by any means, the creature regains additional hit points equal to 2 + your proficiency modifier.
\paragraph{RANK 3} Increase your Wisdom score by 1, to a maximum of 20.

% ============================================================================== %
\subsubsection{Medium Armor Master} \label{feat::mediumarmormaster}
\small{\textcolor{gray}{Strength or Dexterity}}

\normalsize
You have extensively practiced moving in medium armor, and make skilled use of its adaptive capabilities.
\paragraph{REQUIREMENTS} Moderately Armored 2.
\paragraph{RANK 1} Wearing medium armor doesn't impose disadvantage on your Dexterity (Stealth) checks.
\paragraph{RANK 2} When you wear medium armor, you can add 3, rather than 2, to your AC if you have a Dexterity of 16 or higher.
\paragraph{RANK 3} While wearing medium armor, you may add half your proficiency bonus, rounded down, to Strength, Dexterity, and Constitution saving throws that don't already include your proficiency bonus.

% ============================================================================== %
\subsubsection{Medium Haft Master} \label{feat::mediumhaftmaster}
\small{\textcolor{gray}{Strength}}

\normalsize
Why learn how to fight with a sword when your tools will do the job?
\paragraph{REQUIREMENTS} Axe Adept 2, Hammer Adept 2, and Pick Adept 2.
\paragraph{RANK 1} You gain a +1 bonus to attack rolls you make with axes, hammers, and picks.
\paragraph{RANK 2} You can treat any weapon as if it had the Heavy property.
\paragraph{RANK 3} You learn the Shield Break technique.

% ============================================================================== %
\subsubsection{Menacing} \label{feat::menacing}
\small{\textcolor{gray}{Intimidation}}

\normalsize
With good reason or not, people have always been a bit fearful of you.
Your menacing appearance has permitted you to bypass consequence in many situations, allowing you to lead a carefree lifestyle.
\paragraph{RANK 1} You are proficient with the Intimidation skill.
\paragraph{RANK 2} People are naturally more careful around you, and you can get away with petty crimes without repercussion from the law.
\paragraph{RANK 3} You learn the Taunt technique.

% ============================================================================== %
\subsubsection{Mender} \label{feat::mender}
\small{\textcolor{gray}{Medicine}}

\normalsize
You have a natural affinity to healing.
You are uncomfortable with the thought of a friend suffering, and your dedication to mending wounds is valued by anyone around you.
\paragraph{RANK 1} You are proficient with the Medicine skill.
\paragraph{RANK 2} When you successfully stabilize a dying creature, that creature also regains 1 hit point.
\paragraph{RANK 3} During a short rest, you can clean and bind the wounds of up to six willing beasts and humanoids.
Make a DC 15 Wisdom (Medicine) check for each creature.
On a success, if a creature spends a Hit Die during this rest, that creature can forgo the roll and instead regain the maximum number of hit points the die can restore.
A creature can do so only once per rest, regardless of how many Hit Dice it spends.

% ============================================================================== %
\subsubsection{Mobile} \label{feat::mobile}
\small{\textcolor{gray}{Acrobatics}}

\normalsize
You are exceptionally speedy and agile.
\paragraph{REQUIREMENTS} Lightly Armored 2 and Acrobat 2.
\paragraph{RANK 1} When you use the Dash action, difficult terrain doesn't cost you extra movement.
\paragraph{RANK 2} When you make a melee attack against a creature, you don't provoke opportunity attacks from that creature for the rest of the turn, whether you hit or not.
\paragraph{RANK 3} Your speed increases by 3 meters.

% ============================================================================== %
\subsubsection{Moderately Armored} \label{feat::moderatelyarmored}
\small{\textcolor{gray}{Strength or Dexterity}}

\normalsize
You are trained in the use of medium armor, be it leather, chain, or plate.
\paragraph{REQUIREMENTS} Lightly Armored 1.
\paragraph{RANK 1} You gain proficiency with medium armor.
\paragraph{RANK 2} Equipped medium armor and leather or chain accessories don't add to your encumbrance.
\paragraph{RANK 3} Increase your Strength or Dexterity by 1, to a maximum of 20.

% ============================================================================== %
\subsubsection{Modern Soldier} \label{feat::modernsoldier}
\small{\textcolor{gray}{Strength or Dexterity}}

\normalsize
Always up-to-date, you are ready to fight your foes with both a melee and a ranged weapon.
\paragraph{REQUIREMENTS} Straight Sword Adept 2 and Crossbow Adept 2.
\paragraph{RANK 1} When you use the Attack action and attack with a one-handed weapon, you can use your bonus action to attack with a hand crossbow or loaded firearm you are holding.
\paragraph{RANK 2} When using a backsword or a sabre on one hand and your other hand is free, you can hold the back of the blade with your free hand to increase its cutting power.
You add a +2 to your melee weapon attack damage while wielding your weapon in this way.
\paragraph{RANK 3} You learn the Visceral Attack technique.

% ============================================================================== %
\subsubsection{Musket Adept} \label{feat::musketadept}
\small{\textcolor{gray}{Dexterity}}

\normalsize
Quickly picking up on new trends, you gambled on muskets and their stopping power early on.
\paragraph{REQUIREMENTS} Armed Fighter 2.
\paragraph{RANK 1} You are proficient with muskets.
\paragraph{RANK 2} While you have a two-handed firearm equipped, you have advantage on rolls against effects that would move you.
\paragraph{RANK 3} You learn the Buttstroke technique.

% ============================================================================== %
\subsubsection{Naturalist} \label{feat::naturalist}
\small{\textcolor{gray}{Nature}}

\normalsize
Your appreciation for nature has led you to attain an extensive categorical knowledge of plants and animals, giving you a great feat at recognizing and identifying them.
\paragraph{RANK 1} You are proficient with the Nature skill.
\paragraph{RANK 2} Provided you are familiar with the local flora and fauna from experience or a guide, you don't need to make an ability check to identify common plants, fungi, animals, and insects.
\paragraph{RANK 3} Your group can't become lost except by magical means.

% ============================================================================== %
\subsubsection{Observant} \label{feat::observant}
\small{\textcolor{gray}{Perception}}

\normalsize
Easily distracted, you are always attentive of the details in your surroundings.
\paragraph{RANK 1} You are proficient with the Perception skill.
\paragraph{RANK 2} You can easily spot a person you know in a crowd or an object you're familiar with among many.
\paragraph{RANK 3} Being in a lightly obscured area doesn't impose disadvantage on your Wisdom (Perception) checks if you can both see and hear.

% ============================================================================== %

% !TEX root = ../main.tex
\subsubsection{Perceptive} \label{feat::perceptive}
\small{\textcolor{gray}{Perception}}

\normalsize
When you focus your mind on something, you are able to quickly perceive even the smallest details and imperfections.
\paragraph{REQUIREMENTS} Observant 3.
\paragraph{RANK 1} You double your proficiency modifier in the Perception skill.
\paragraph{RANK 2} You have a +5 bonus to your passive Wisdom (Perception) score.
\paragraph{RANK 3} Increase your Wisdom score by 1, to a maximum of 20.

% ============================================================================== %
\subsubsection{Performer} \label{feat::performer}
\small{\textcolor{gray}{Performance}}

\normalsize
A natural performer, you've never failed to impress with your refined acting skills.
You have an easy time imitating the demeanor of others, especially of those that you know well.
\paragraph{RANK 1} You are proficient with the Performance skill.
\paragraph{RANK 2} You have advantage on ability checks trying to pass off as another member of your kin, even without a disguise.
\paragraph{RANK 3} You learn the Distract technique.

% ============================================================================== %
\subsubsection{Persuasive} \label{feat::persuasive}
\small{\textcolor{gray}{Persuasion}}

\normalsize
A skilled negotiator and a master of diplomacy, you have an easy time convincing people to do what you need.
\paragraph{RANK 1} You are proficient with the Persuasion skill.
\paragraph{RANK 2} You have advantage on Charisma (Persuasion) checks when trading with a creature.
\paragraph{RANK 3} You learn the Charm technique.

% ============================================================================== %
\subsubsection{Pick Adept} \label{feat::pickadept}
\small{\textcolor{gray}{Strength or Dexterity}}

\normalsize
For you, there is no difference between mining ore and piercing flesh.
Apart from the screams at least.
% You pierce through metal, flesh, and bone just as you would ore.
\paragraph{REQUIREMENTS} Armed Fighter 2.
\paragraph{RANK 1} You are proficient with picks.
\paragraph{RANK 2} All picks have the versatile property for you.
When you wield a pick with two hands, increase its damage die by one (d6 to d8, d8 to d10, etc), and have the heavy property.
\paragraph{RANK 3} You learn the Trip technique.

% ============================================================================== %
\subsubsection{Pickpocket} \label{feat::pickpocket}
\small{\textcolor{gray}{Sleight of Hand}}

\normalsize
You have taken the time to hone your larcenous skills, making you an accomplished thief.
\paragraph{REQUIREMENTS} Quick Fingers 3.
\paragraph{RANK 1} You double your proficiency modifier in the Sleight of Hand skill.
\paragraph{RANK 2} Other creatures have disadvantage on Intelligence (Investigation) or Wisdom (Perception) checks to notice an object you palmed in your hands.
\paragraph{RANK 3} If you spend at least 1 minute observing or interacting with another creature outside combat, you can learn about anything the creature has that can be stolen.

% ============================================================================== %
\subsubsection{Piercer} \label{feat::piercer}
\small{\textcolor{gray}{Strength or Dexterity}}

\normalsize
You have achieved a penetrating precision in combat.
\paragraph{REQUIREMENTS} Proficiency with at least two martial piercing weapon types.
\paragraph{RANK 1} Once per turn, when you hit a creature with an attack that deals piercing damage, you can reroll one of the attack's damage dice, and you must use the new roll.
\paragraph{RANK 2} Increase your Strength or Dexterity by 1, to a maximum of 20.
\paragraph{RANK 3} You learn the Armor Puncture technique.

% ============================================================================== %
\subsubsection{Pike Master} \label{feat::pikemaster}
\small{\textcolor{gray}{Strength}}

\normalsize
You prefer to maintain distance from your opponents by wielding an amusingly long pole.
\paragraph{REQUIREMENTS} Spear Adept 2.
\paragraph{RANK 1} When you use a pike, its damage die changes from a d10 to a d12.
\paragraph{RANK 2} When you use the Feint technique you can choose to not attack with advantage.
If you do this and still hit your target, it's a critical hit.
\paragraph{RANK 3} When wielding a pike, its reach property adds 3 meters instead of 1.5 meters to the weapon's range.

% ============================================================================== %
\subsubsection{Pious} \label{feat::pious}
\small{\textcolor{gray}{Religion}}

\normalsize
You're a zealot when it comes to your faith.
Your devotion has led to an incredible understanding of your religion and its history, and no connoted figure goes unnoticed in your prayers.
\paragraph{RANK 1} You are proficient with the Religion skill.
\paragraph{RANK 2} You can recall the name and description of any deity and important figure related to a religion of your choice without needing to succeed on any ability check.
\paragraph{RANK 3} You have advantage in Intelligence (Religion) checks made to tell the tide of a creature you can see.

% ============================================================================== %
\subsubsection{Pistol Adept} \label{feat::pistoladept}
\small{\textcolor{gray}{Dexterity}}

\normalsize
Already a master of the crossbow, it isn't hard for you to adapt your use of the handcrossbow to that of the pistol.
\paragraph{REQUIREMENTS} Crossbow Adept 2.
\paragraph{RANK 1} You are proficient with pistols.
\paragraph{RANK 2} You can stow a firearm, then draw another weapon as a single object interaction on your turn.
\paragraph{RANK 3} When you get a critical hit against a creature on a ranged attack with a firearm, all other attacks against the creature are made with advantage until the start of your next turn.

% ============================================================================== %
\subsubsection{Polearm Master} \label{feat::polerarmmaster} %
\small{\textcolor{gray}{Dexterity}}

\normalsize
You keep your enemies at bay with your long weapons.
\paragraph{REQUIREMENTS} Staff Fighter 2, Spear Adept 2, and Halberd Adept 2.
\paragraph{RANK 1} You gain a +1 bonus to attack rolls you make with staves, spears, and halberds.
\paragraph{RANK 2} While wielding a stave, spear, or halberd, other creatures provoke an opportunity attack from you when they enter your reach.
\paragraph{RANK 3} You learn the Rondel Hit technique.

% ============================================================================== %
\subsubsection{Precise Thrower} \label{feat::precisethrower}
\small{\textcolor{gray}{Dexterity}}

\normalsize
Either from luck or rigorous training, you have the capacity to hit the hardest of targets when throwing a weapon.
\paragraph{REQUIREMENTS} Dagger Savant 2 and Thrown Weapon Master 2.
\paragraph{RANK 1} Attacking at long range doesn't impose disadvantage on your thrown weapon attack rolls.
\paragraph{RANK 2} Whenever you hit a creature with an thrown slashing or piercing weapon, it has disadvantage on the first melee attack it performs on its next turn.
\paragraph{RANK 3} You learn the Bull's Eye technique.

% ============================================================================== %
\subsubsection{Pugilist} \label{feat::pugilist}
\small{\textcolor{gray}{Strength}}

\normalsize
Your fighting style has only three components:
A fist, another fist, and a face to punch.
\paragraph{RANK 1} Your unarmed strike uses a d4 + your Strength modifier for damage.
\paragraph{RANK 2} When you hit a creature with an unarmed strike, you can use a bonus action to attempt to grapple the target.
\paragraph{RANK 3} You learn the Bonk technique.

% ============================================================================== %
\subsubsection{Quick Fingers} \label{feat::quickfingers}
\small{\textcolor{gray}{Sleight of Hand}}

\normalsize
While they've given you problems in the past, your quick and sticky fingers have awarded you with many a treasure thorough your life.
\paragraph{RANK 1} You are proficient with the Sleight of Hand skill.
\paragraph{RANK 2} You can use a bonus action to take the Search or the Use an Object actions.
\paragraph{RANK 3} You learn the Steal technique.

% ============================================================================== %
\subsubsection{Revenant Blade} \label{feat::revenantblade}
\small{\textcolor{gray}{Dexterity}}

\normalsize
You are a expert of the double-bladed scimitar, and use the technique as proficiently as any master from the ancient Janchu'ut school of hunters.
\paragraph{REQUIREMENTS} Curved Blade Adept 2.
\paragraph{RANK 1} A double-bladed scimitar has the finesse property when you wield it.
\paragraph{RANK 2} While you are holding a double-bladed scimitar with two hands, you gain a +1 bonus to Armor Class.
\paragraph{RANK 3} Instead of dealing 1d4 slashing damage, the extra attack you can do with a double-bladed scimitar deals 2d4 slashing damage.

% ============================================================================== %
\subsubsection{Robust} \label{feat::robust}
\small \textcolor{gray}{Athletics}

\normalsize
Your physical prowess is a marvel to all.
\paragraph{REQUIREMENTS} Athlete 3.
\paragraph{RANK 1} You double your proficiency modifier in the Athletics skill.
\paragraph{RANK 2} Your walking speed increases by 1.5 meters.
Additionally, you gain a climbing speed and a swimming speed equal to your walking speed.
\paragraph{RANK 3} Your hit point maximum increases by an amount equal to your number of hit dice.
Whenever you earn a hit die thereafter, your hit point maximum increases by an additional 1 hit point.

% ============================================================================== %

% !TEX root = ../main.tex
\subsubsection{Sharpshooter} \label{feat::sharpshooter}
\small{\textcolor{gray}{Dexterity}}

\normalsize
You can strike a foe with an arrow or bullet no matter where they are in the battlefield.
\paragraph{REQUIREMENTS} Bow Adept 2 or Musket Adept 2.
\paragraph{RANK 1} If you do not move during your turn, you can make an additional ranged weapon attack as a bonus action.
\paragraph{RANK 2} Both your normal and long ranges are multiplied by 1.5 for all ranged weapons..
\paragraph{RANK 3} Whenever you have advantage on a ranged attack roll using Dexterity, you can reroll one of the dice once.

% ============================================================================== %
\subsubsection{Shield Master} \label{feat::shieldmaster}
\small{\textcolor{gray}{Strength or Dexterity}}

\normalsize
You know that fighting to the death leads to a short life.
You trust your shield with your life, and you are an expert at avoiding death.
\paragraph{REQUIREMENTS} Shield Training 2.
\paragraph{RANK 1} If you aren't incapacitated, you can add your shield's AC bonus to any Dexterity saving throw you make against a spell or other harmful effect that targets only you.
\paragraph{RANK 2} If you are subjected to an effect that allows you to make a Dexterity saving throw to take only half damage, you can use your reaction to take no damage if you succeed on the saving throw, interposing your shield between yourself and the source of the effect.
\paragraph{RANK 3} When you use the Shove action or the Push technique with your shield, you can choose to force the Prone condition on your target instead of pushing it.

% ============================================================================== %
\subsubsection{Shield Training} \label{feat::shieldtraining}
\small{\textcolor{gray}{Strength or Dexterity}}

\normalsize
Your prefer to live a long life, and choose to carry your trusty shield with you at all times.
\paragraph{RANK 1} You gain proficiency with medium shields.
\paragraph{RANK 2} In combat, you can don or doff a medium shield as a bonus action instead of an action.
\paragraph{RANK 3} You learn the Push technique.

% ============================================================================== %
\subsubsection{Shooter} \label{feat::shooter}
\small{\textcolor{gray}{Dexterity}}

\normalsize
You have mastered ranged weapons and can make shots that others find impossible.
\paragraph{REQUIREMENTS} Proficiency with at least two martial ranged weapon types.
\paragraph{RANK 1} Your ranged attacks ignore half cover and three-quarters covers.
\paragraph{RANK 2} Increase your Dexterity by 1, to a maximum of 20.
\paragraph{RANK 3} You learn the Leg Shot technique.

% ============================================================================== %
\subsubsection{Silver-tongued} \label{feat::silvertongued}
\small{\textcolor{gray}{Deception}}

\normalsize
You develop your conversational skill to better deceive others.
\paragraph{REQUIREMENTS} Liar 3.
\paragraph{RANK 1} You double your proficiency modifier in the Deception skill.
\paragraph{RANK 2} You learn the Deceive technique.
\paragraph{RANK 3} Increase your Charisma score by 1, to a maximum of 20.

% ============================================================================== %
\subsubsection{Skulker} \label{feat::skulker}
\small{\textcolor{gray}{Stealth}}

\normalsize
You are expert at slinking through shadows.
\paragraph{REQUIREMENTS} Sly 3.
\paragraph{RANK 1} You double your proficiency modifier in the Stealth skill.
\paragraph{RANK 2} Making a melee or range attack alerts your target, but doesn't immediately reveal your position.
\paragraph{RANK 3} You gain or improve the Sneak Attack technique.

% ============================================================================== %
\subsubsection{Slasher} \label{feat::slasher}
\small{\textcolor{gray}{Strength or Dexterity}}

\normalsize
Your experience with blades has taught you where to cut to have the greatest results.
\paragraph{REQUIREMENTS} Proficiency with at least two martial slashing weapon types.
\paragraph{RANK 1} Once per turn when you hit a creature with an attack that deals slashing damage, you can reduce the speed of the target by 3 meters until the start of your next turn.
\paragraph{RANK 2} Increase your Strength or Dexterity by 1, to a maximum of 20.
\paragraph{RANK 3} You learn the Grievous Wound technique.

% ============================================================================== %
\subsubsection{Sly} \label{feat::sly}
\small{\textcolor{gray}{Stealth}}

\normalsize
One with shadows, you can always escape from the most unfavorable situations.
You have a natural ease melding with the dark, and intuitively know how to avoid making any noise.
\paragraph{RANK 1} You are proficient with the Stealth skill.
\paragraph{RANK 2} You can try to hide when you are lightly obscured from the creature from which you are hiding.
\paragraph{RANK 3} You gain advantage on any Dexterity (Stealth) check if you move no more than half your speed on the same turn.

% ============================================================================== %
\subsubsection{Sniper} \label{feat::sniper}
\small{\textcolor{gray}{Dexterity}}

\normalsize
Learned in the use of most ranged weapons, you are an expert at piercing foes from afar.
\paragraph{REQUIREMENTS} Bow Adept 2, Crossbow Adept 2, and Musket Adept 2.
\paragraph{RANK 1} You gain a +1 bonus to attack rolls you make with bows, crossbows, pistols, and muskets.
\paragraph{RANK 2} Attacking at long range doesn't impose disadvantage on your ranged weapon attack rolls.
\paragraph{RANK 3} You learn the Hemorrhaging Attack technique.

% ============================================================================== %
\subsubsection{Spear Adept} \label{feat::spearadept}
\small{\textcolor{gray}{Strength or Dexterity}}

\normalsize
Trained in the art of the spear, you know that nothing beats a long pole with a pointy end.
\paragraph{REQUIREMENTS} Armed Fighter 2.
\paragraph{RANK 1} You are proficient with spears.
\paragraph{RANK 2} Spears have the finesse property for you.
\paragraph{RANK 3} You learn the Feint technique.

% ============================================================================== %
\subsubsection{Staff Fighter} \label{feat::stafffighter}
\small{\textcolor{gray}{Dexterity}}

\normalsize
You have mastered the nuances of fighting with a staff.
\paragraph{REQUIREMENTS} Armed Fighter 1.
\paragraph{RANK 1} The staff counts as a finesse weapon for you.
\paragraph{RANK 2} While wielding a staff, you can use your bonus action to give yourself advantage on your next Dexterity (Acrobatics) or Strength (Athletics) check related to climbing, jumping, or otherwise bypassing obstacles.
This benefit lasts until the end of your turn.
\paragraph{RANK 3} You learn the Sweep technique.

% ============================================================================== %
\subsubsection{Stone Piercer} \label{feat::stonepiercer}
\small{\textcolor{gray}{Strength}}

\normalsize
You know that the difference between piercing stone and crushing bone lies only in the morals of the aggressor.
\paragraph{REQUIREMENTS} Pick Adept 2 and Athlete 2.
\paragraph{RANK 1} You double your damage dice against objects and structures while attacking with a pick.
Additionally, you have advantage on checks made with a climber's kit and pitons.
\paragraph{RANK 2} When you score a critical hit that deals piercing damage to a creature, you can roll one additional damage die when determining the extra piercing damage the target takes.
\paragraph{RANK 3} You learn the Injure technique.

% ============================================================================== %
\subsubsection{Straight Sword Adept} \label{feat::straightswordadept}
\small{\textcolor{gray}{Strength or Dexterity}}

\normalsize
You are proficient with the traditional fighting style of straight swords.
\paragraph{REQUIREMENTS} Armed Fighter 2.
\paragraph{RANK 1} You are proficient with straight swords.
\paragraph{RANK 2} When you are wielding a straight sword in one hand and no other weapons, you gain a +1 to AC.
% As a free action at the start of your turn, you can choose to reduce your chance to hit by 2, while increasing your AC by 1 until the start of your next turn.
\paragraph{RANK 3} You learn the Block technique.

% ============================================================================== %
\subsubsection{Survivor} \label{feat::survivor}
\small{\textcolor{gray}{Survival}}

\normalsize
As if raised by wolves, you are a natural survivor in the most harsh of circumstances.
Your enviable endurance and experience allows you to be particularly effective at building shelter, purifying water, and foraging food.
\paragraph{RANK 1} You are proficient with the Survival skill.
\paragraph{RANK 2} When you forage, you find twice as much food as you normally would.
\paragraph{RANK 3} You have advantage on Wisdom (Survival) checks made to track creatures.

% ============================================================================== %
\subsubsection{Sword Master} \label{feat::swordmaster}
\small{\textcolor{gray}{Strength or Dexterity}}

\normalsize
You have an insight unique to those who master the many types of swords.
Your skill with the long blades is unparalleled, and your resoluteness in combat scares even the bravest of warriors.
\paragraph{REQUIREMENTS} Straight Sword Adept 2, Curved Sword Adept 2, and Light Sword Adept 2.
\paragraph{RANK 1} You gain a +1 bonus to attack rolls you make with swords.
\paragraph{RANK 2} When you make an opportunity attack with a sword, you have advantage on the attack roll.
\paragraph{RANK 3} You learn the Parrying Stance technique.

% ============================================================================== %
\subsubsection{Swashbuckler} \label{feat::swashbuckler}
\small{\textcolor{gray}{Dexterity}}

\normalsize
Like modern fighters, you mix your combat style with both melee and ranged attacks.
However, you gain the upper hand on them by adding a fair share of cheap tactics to this mix.
\paragraph{REQUIREMENTS} Curved Sword Adept 2 and Crossbow Adept 2, or Light Sword Adept 2 and Crossbow Adept 2.
\paragraph{RANK 1} When you use the Attack action and attack with a one-handed weapon, you can use a bonus action to attack with a hand crossbow or a loaded firearm you are holding.
\paragraph{RANK 2} During your turn, if you make a melee attack against a creature, that creature can't make opportunity attacks against you for the rest of your turn.
\paragraph{RANK 3} You learn the Cheap Shot technique.

% ============================================================================== %
\subsubsection{Theologian} \label{feat::theologian}
\small{\textcolor{gray}{Religion}}

\normalsize
Your extensive study of religion and the tides rewards you with many benefits.
\paragraph{REQUIREMENTS} Pious 3.
\paragraph{RANK 1} You double your proficiency modifier in the Religion skill.
\paragraph{RANK 2} You have advantage on Charisma (Persuasion) and Charisma (Deception) checks against creatures of your same tide.
You must know the tide of the creature to get this benefit.
\paragraph{RANK 3} Increase your Intelligence score by 1, to a maximum of 20.

% ============================================================================== %
\subsubsection{Thrown Weapon Master} \label{feat::thrownweaponmaster}
\small{\textcolor{gray}{Strength or Dexterity}}

\normalsize
Born with quick fingers and good aim, you're an expert at piercing your enemies with thrown objects.
\paragraph{REQUIREMENTS} Armed Fighter 1.
\paragraph{RANK 1} Attacking at long range doesn't impose disadvantage on your thrown weapon attack rolls.
\paragraph{RANK 2} When you throw a thrown weapon as an attack roll, you can draw another thrown weapon freely.
\paragraph{RANK 3} You learn the Quick Draw technique.

% ============================================================================== %
\subsubsection{Tracker} \label{feat::tracker}
\small{\textcolor{gray}{Survival}}

\normalsize
You have spent enough time hunting to hone your skill to a venerable level.
\paragraph{REQUIREMENTS} Survivor 3.
\paragraph{RANK 1} You double your proficiency modifier in the Survival skill.
\paragraph{RANK 2} You learn the Mark technique.
\paragraph{RANK 3} Increase your Wisdom score by 1, to a maximum of 20.

% ============================================================================== %
\subsubsection{Tree Feller} \label{feat::treefeller}
\small{\textcolor{gray}{Strength}}

\normalsize
Nature is your foe, and you use your broad axe to fell both trees and foes.
\paragraph{REQUIREMENTS} Axe Adept 2.
\paragraph{RANK 1} When you use a broad axe, its damage die changes from a d8 to a d10, and from a d10 to a d12 when wielded with two hands.
\paragraph{RANK 2} While wielding a broad axe, you deal double damage to structures and items.
\paragraph{RANK 3} Whenever you hit a plant or a plant creature with a slashing weapon, the target takes an additional 2 slashing damage.

% ============================================================================== %
\subsubsection{Two-weapon Master} \label{feat::twoweaponmaster}
\small{\textcolor{gray}{Dexterity}}

\normalsize
Your ability to wield two weapons is unparalleled, and you can surprise any foe with your quick steps and unrelenting strikes.
\paragraph{REQUIREMENTS} Dexterity 16. Dual Wielder 2.
\paragraph{RANK 1} You can use use two-weapon fighting even when the one-handed melee weapons you are wielding aren't light.
\paragraph{RANK 2} You gain a +1 bonus to AC while you are wielding a separate melee weapon in each hand.
\paragraph{RANK 3} You can add your ability modifier to the damage of the bonus attack of your two-weapon fighting action.

% ============================================================================== %
\subsubsection{Unarmed Artist} \label{feat::unarmedartist}
\small{\textcolor{gray}{Dexterity}}

\normalsize
You are a painter.
Your hands are your brush, your opponent your canvas.
\paragraph{RANK 1} Your unarmed strike uses a d4 + your Dexterity modifier for damage.
You can only use this ability if you're not using a fist weapon.
\paragraph{RANK 2} When you use the Attack action, you can make an unarmed strike as a bonus action.
\paragraph{RANK 3} You get access to one Martial Arts technique of your choice.
You may buy this rank again to attain new Martial Arts techniques as long as you have not learned all available techniques.

% ============================================================================== %
\subsubsection{Unarmored} \label{feat::unarmored}
\small{\textcolor{gray}{Dexterity}}

\normalsize
Your extensive physical training allows to forfeit armor when fighting.
Your body is the only armor you need.
\paragraph{REQUIREMENTS} Acrobat 2 and Athlete 2.
\paragraph{RANK 1} As long as you are not wearing any armor, your speed increases by 3 meters.
\paragraph{RANK 2} If you are subjected to an effect that allows you to make a Dexterity saving throw to take only half damage, you can use your reaction to take no damage if you succeed on the saving throw, and half damage if you fail.
You only get this bonus while wearing no armor or light armor.
\paragraph{RANK 3} You learn the Rush technique.

% ============================================================================== %

% !TEX root = ../main.tex
\subsubsection{Way of the Fist} \label{feat::wayofthefist}
\small{\textcolor{gray}{Strength or Dexterity}}

\normalsize
Either by meditative training or drunken fighting, you are a master with your fists.
\paragraph{REQUIREMENT} Pugilist 2 and Unarmed Artist 2.
\paragraph{RANK 1} Your unarmed strike uses a d6 + your Strength or Dexterity modifier for damage.
\paragraph{RANK 2} When you get a critical hit with an unarmed strike, you can take the Disarm, Disengage, or Shove action as a free action.
\paragraph{RANK 3} You are never unarmed, for your fists are your weapons.
Your unarmed strikes deal double stamina damage.

% ============================================================================== %
\subsubsection{Whip Adept} \label{feat::whipadept}
\small{\textcolor{gray}{Dexterity}}

\normalsize
You are learned in the use of the whip, using your unique weapon for both combat and support.
\paragraph{REQUIREMENTS} Armed Fighter 2.
\paragraph{RANK 1} You can use your bonus action to make one melee attack with an equipped whip.
\paragraph{RANK 2} You can use a whip to extend the range of the Grapple and Shove actions to the range of the weapon.
\paragraph{RANK 3} You can use your whip as an elongated appendage to perform actions that do not require fine motor skills.
This includes object interactions, such as grabbing a sword or pulling a bag towards you, and ability checks, such as an acrobatics roll to grab a nearby support beam to swing from.

% ============================================================================== %
\subsubsection{Wrestler} \label{feat::wrestler}
\small{\textcolor{gray}{Strength}}

\normalsize
Every part of your body is a weapon and you wield it expertly.
\paragraph{REQUIREMENTS} Grappler 2 and Combat Improviser 2.
\paragraph{RANK 1} Once per round, as part of your action, you many deal your Strength modifier in damage to any creature you are grappling.
\paragraph{RANK 2} You can use a grappled creature of a size equal or lower to yours as an improvised weapon.
\paragraph{RANK 3} You learn the Charge technique.

% ============================================================================== %

% !TEX root = ../main.tex
\subsubsection{Yumi User} \label{feat::yumiuser}
\small{\textcolor{gray}{Dexterity}}

\normalsize
You use the versatility of the yumi to your advantage, improving your aim while kneeling or on horseback.
\paragraph{REQUIREMENTS} Bow Adept 2 and Great Weapon User 1.
\paragraph{RANK 1} You only get disadvantage on ranged attack rolls with a yumi when your target is within 3 meters instead of 6.
\paragraph{RANK 2} At the beginning of your turn, you can reduce your movement to 0 and get advantage on one ranged weapon attack you make during the same turn.
\paragraph{RANK 3} You get a +2 to damage rolls with a yumi if you don't move during your turn or are riding on a mount.

% ============================================================================== %
\subsubsection{Zoophilist} \label{feat::zoophilist}
\small{\textcolor{gray}{Animal Handling}}

\normalsize
You master the techniques needed to train and handle animals.
\paragraph{REQUIREMENTS} Animal Handler 3.
\paragraph{RANK 1} You double your proficiency modifier in the Animal Handling skill.
\paragraph{RANK 2} When you use the Beast Command technique, you can make a Wisdom (Animal Handling) check of DC 20.
If you succeed, you can spend the next 8 hours bonding with the beast, after which it becomes your permanent companion.
You can have only one animal companion at a time.
You can issue commands to your companion using the Beast Command technique without needing to make a Wisdom (Animal Handling) check.
Additionally, your animal companion considers you its kin, and will protect you with its life.
If you fail, the effect of Beast Command works normally, but you cannot try to bond with the same beast again.
\paragraph{RANK 3} Increase your Wisdom score by 1, to a maximum of 20.

% ============================================================================== %


% TODO: Add a feat to do a grapple with acrobatics.
% TODO: Add a feat to force an opponent to roll a grapple with acrobatics.
% TODO: Add a feat to extend the range of the Help action.
% TODO: Add light crossbows as simple crossbows.
% TODO: Medium and Heavy armor no longer requires competence with "lesser" armor.
% TODO: Talent that allows you to temporarily learn a spell after identifying it.
% TODO: Talent to do a 1.5 meters roll before standing up, movement doesn't produce opportunity attacks but standing up does.

% ============================================================================== %
\subsection{Kin Feats} % TODO: Re-think how to sort this! Many kins share feats.
    \subsection*{Gat Feats}
        \subsubsection{Force of Will} \label{feat::forceofwill}
        \small{\textcolor{gray}{Athletics}}

        \normalsize
        You can plant yourself in place with all your weight, making you very difficult to move.
        As long as your feet are on the ground, you have advantage on any ability checks or saving throws made to move you or force you to fall prone.
        \paragraph{REQUIREMENTS} Kin: Gat.

        % ========================================================================== %
        \subsubsection{Goat's Strength} \label{feat::goatsstrength}
        \small{\textcolor{gray}{Strength}}

        \normalsize
        You complement your natural strengths with hard training.
        The saving throw related to the Push technique gains a +2 bonus for you.
        Additionally, any attack made with your horns gains a +1 to its attack and damage rolls.
        \paragraph{REQUIREMENTS} Kin: Gat.

        % ========================================================================== %
        \subsubsection{Efficient Craftsgat} \label{feat::efficientcraftsgat}
        \small{\textcolor{gray}{Set of artisan's tools}}

        \normalsize
        They say that a gat is born with tools in their hands.
        Crafting times with the artisan's tools related to your craftgatship trait are halved for you.
        \paragraph{REQUIREMENTS} Kin: Gat.

        % ========================================================================== %
        \subsubsection{Gat Resilience (2 FP)} \label{feat::gatresilience}
        \small{\textcolor{gray}{Constitution}}

        \normalsize
        The horned kin possesses an almost otherwordly hardiness.
        You have advantage on saving throws against poison and are resistant against poison damage.
        \paragraph{REQUIREMENTS} Kin: Gat.

        % ========================================================================== %
        \subsubsection{Legendary Craftsgat (3 FP)} \label{feat::legendarycraftsgat}
        \small{\textcolor{gray}{Set of artisan's tools}}

        \normalsize
        Above the average gat, you are legendary at the use of your tools.
        Your proficiency with the artisan's tools related to your craftgatship trait is increased to Legendary, incresing your proficiency bonus to +12 with them.
        \paragraph{REQUIREMENTS} Kin: Gat. Expert proficiency with a set of artisan's tools.

        % ========================================================================== %
        \subsubsection{Mountain Born} \label{feat::mountainborn}
        \small{\textcolor{gray}{Constitution}}

        \normalsize
        You are acclimated to high altitude.
        You don't suffer the effects associated to the cold or lack of oxygen of any elevation up to 6,000 meters.
        You also have a climbing speed of 6 meters.
        \paragraph{REQUIREMENTS} Kin: Noves Gat or Thulkraka Ird.

        % ========================================================================== %
        \subsubsection{Hardy} \label{feat::hardy}
        \small{\textcolor{gray}{Constitution}}

        \normalsize
        Resilience comes as natural to you as breathing.
        As long as you are not already a master, you increase your proficiency in the Survival skill.
        Additionally, you can choose to add your Constitution modifier to Survival ability checks   instead of your Wisdom modifier.
        \paragraph{REQUIREMENTS} Kin: Bughna Gat.

        % ========================================================================== %
        \subsubsection{Relentless Striker} \label{feat::relentlessstriker}
        \small{\textcolor{gray}{Strength}}

        \normalsize
        Your hammering horns are your most valuable weapons, and you are trained to use them as a normal    part of your arsenal.
        Melee attack rolls made with your horns don't add to your multiple attack penalty.
        \paragraph{REQUIREMENTS} Kin: Treb Gat.

        % ========================================================================== %
\subsection*{Ird Feats}
    \subsubsection{Perfect Landing} \label{feat::perfectlanding}
    \small{\textcolor{gray}{Acrobatics}}

    \normalsize
    Furthering your natural falling skills, you are trained to handle any fall with ease.
    As long as you are conscious and can freely use your wings, you are immune to fall damage.
    \paragraph{REQUIREMENTS} Kin: Ird.

    % ========================================================================== %
    \subsubsection{Forest Defender} \label{feat::forestdefender}
    \small{\textcolor{gray}{Strength}}

    \normalsize
    You have received martial training to fight among the branches, and are extremely dangerous to climbing opponents.
    You know the Push feat (See page \pageref{feat::push}), and have advantage on the Strength check if you flew at least 3 meters before using pushing your target.
    \paragraph{REQUIREMENTS} Kin: Qulbaba Ird.

    % ========================================================================== %
    \subsubsection{Thulkraka Steel} \label{feat::thulkrakasteel}
    \small{\textcolor{gray}{Set of smith's tools}}

    \normalsize
    You are specially skilled with a set of smith's tools.
    You can craft any metal item for half the normal cost, as long as you have access to a forge at an altitude of at least 3,000 meters to use the special quench-hardening technique unique to your people.
    \paragraph{REQUIREMENTS} Kin: Thulkraka Ird.

    % ========================================================================== %
\subsection*{Marset Feats}
    \subsubsection{Born Climber} \label{feat::bornclimber}
    \small{\textcolor{gray}{Acrobatics}}

    \normalsize
    You can use your tail to grab onto branches, and don't require to pass any ability checks to climb trees and branches.
    Additionally, you are able to freely use your hands while climbing.
    \paragraph{REQUIREMENTS} Kin: Marset.

    % ========================================================================== %
    \subsubsection{Natural Defense} \label{feat::naturaldefense}
    \small{\textcolor{gray}{Constitution}}

    \normalsize
    Through careful training, you know how to position your quills effectively for deadly effect in combat.
    When you grapple a creature, the target takes 3 piercing damage if your grapple check succeeds.
    If another creature grapples you, it takes 3 piercing damage.
    If you start your turn grappling or being grappled by a creature, it takes 3 piercing damage.
    \paragraph{REQUIREMENTS} Kin: Marset.

    % ========================================================================== %
    \subsubsection{Lip Reading} \label{feat::lipreading}
    \small{\textcolor{gray}{Charisma}}

    \normalsize
    If you can see a creature's mouth while it is speaking a language you understand, you can interpret what it's saying by reading its lips.
    \paragraph{REQUIREMENTS} Kin: Marset.

    % ========================================================================== %
    \subsubsection{Trained Forager} \label{feat::trainedforager}
    \small{\textcolor{gray}{Survival}}

    \normalsize
    Due to your nature as a gatherer, all foraging DCs are reduced by 5 for you.
    You can also feed on tree gum, but cannot cook meals for other species with this material.
    \paragraph{REQUIREMENTS} Kin: Marset.

    % ========================================================================== %
\subsection*{Oth Feats}
    \subsubsection{Silent Speech} \label{feat::silentspeech}
    \small{\textcolor{gray}{Intelligence}}

    \normalsize
    You can communicate with other oths and some insectoids using Silent Speech.
    This language can only communicate simple ideas, and it does so via a combination of pheromones and thrumming sounds made with antennae.
    \paragraph{REQUIREMENTS} Kin: Oth.

    % ========================================================================== %
    \subsubsection{Silk Spinning} \label{feat::silkspinning}
    \small{\textcolor{gray}{Set of weaver's tools}}

    \normalsize
    Dust kin's silk is known for its beauty and strength, and oth's know how to craft clothes and various items with it.
    Your silk counts as a cloth material, and armor made with it has a +1 bonus.
    You can produce up to one kg of silk per long rest, which can be stored for future use.
    \paragraph{REQUIREMENTS} Kin: Oth. Competency with weaver's tools.

    % ========================================================================== %
\subsection*{Naenk Feats}
    \subsubsection{Moss Armor} \label{feat::mossarmor}
    \small{\textcolor{gray}{Constitution}}

    \normalsize
    Your plant-based framework provides you with unique resilience.
    You gain resistance to lightning damage.
    \paragraph{REQUIREMENTS} Kin: Naenk.

    % ========================================================================== %
    \subsubsection{Cellular Regeneration (2 FP)} \label{feat::cellularregeneration}
    \small{\textcolor{gray}{Constitution}}

    \normalsize
    As an action, you can stimulate your plant cells to rapidly multiply to quickly regenerate wounds.
    You regain 1 hit point, and regain 1 hit point at the start of each of your turns after, until you've restored an amount equal to twice your number of hit dice.
    After using this trait, you must take a short rest before using it again.
    If you suffer cold, fire, or necrotic damage, this regeneration is cancelled.
    You are also able to regenerate lost limbs, albeit at a very slow pace: it takes you 1d4+2 months to fully recover a lost arm or leg.
    \paragraph{REQUIREMENTS} Kin: Naenk.

    % ========================================================================== %
    \subsubsection{Improved Nuen (2 FP)} \label{feat::improvednuen}
    \small{\textcolor{gray}{Medicine}}

    \normalsize
    Your nuens are stronger than average, and you do not half the creature's hit points or hit dice when you raise one.
    % Additionally, your nuen grows a thick layer of thorns, gaining the \textbf{Natural Defense} trait (page  \pageref{trait::naturaldefense}).

    \paragraph{REQUIREMENTS} Kin: Gannagian Warrior.

    % ========================================================================== %
    \subsubsection{Trained Forager} \label{feat::trainedforager}
    \small{\textcolor{gray}{Survival}}

    \normalsize
    As long as you are not already a master, you increase your proficiency with the herbalism kit.
    Additionally, you do not need to roll anything to tell if a plant is poisonous, even if you've never encountered it before.
    \paragraph{REQUIREMENTS} Kin: Gannagian Gatherer.

    % ========================================================================== %
\subsection*{Tsanek Feats}
    \subsubsection{Improved Spores (2 FP)} \label{feat::improvedspores}
    \small{\textcolor{gray}{Constitution}}

    \normalsize
    You can add a +2 to all of your spore effects' spell save DC.
    Alternatively, you can add your spellcasting modifier if you have one.
    \paragraph{REQUIREMENTS} Kin: Tsanek.

    % ========================================================================== %
\subsection*{Tortle Feats}
    \subsubsection{Reptile Claws} \label{feat::reptileclaws}
    \small{\textcolor{gray}{Strength}}

    \normalsize
    Your claws are natural weapons, which you can use to make unarmed strikes.
    If you hit with them, you deal slashing damage equal to 1d4 + your Strength modifier, instead of the bludgeoning damage normal for an unarmed strike.
    \paragraph{REQUIREMENTS} Kin: Tortle.

    % ========================================================================== %
    \subsubsection{Impaling Carapace} \label{feat::impalingcarapace}
    \small{\textcolor{gray}{Constitution}}

    \normalsize
    Via hard training, you learn how to position the crevices in your natural armor to be especially deadly up close.
    When you grapple a creature, the target takes 3 piercing damage if your grapple check succeeds.
    If another creature grapples you, they take 3 piercing damage.
    If you start your turn grappling or being grappled by a creature, it takes 3 piercing damage.
    \paragraph{REQUIREMENTS} Kin: Tortle.

    % ========================================================================== %
    \subsubsection{Steam Breath (3 FP)} \label{feat::steambreath}
    \small{\textcolor{gray}{Constitution}}

    \normalsize
    Some tortles are born with special abilities, which some say hail from the great dragon turtles of old.
    You can use your action to exhale a 20 feet cone of scalding steam.
    Every target in the area must make a Dexterity saving throw, with a DC of (8 + your Constitution modifier + your proficiency bonus).
    A creature takes 2d6 fire damage on a failed save, or half as much damage on a successful one.
    The fire damage increases to 3d6 with your 6th hit die, 4d6 with your 11th, and 5d6 with your 16th.
    After you use this trait, you can't use it again until you complete a short rest.
    Being underwater doesn't grant resistance to this damage.

    \paragraph{REQUIREMENTS} Kin: Tortle.

    % ========================================================================== %
\subsection*{Grung Feats}
    \subsubsection{Poison Immunity (2 FP)} \label{feat::poisonimmunity}
    \small{\textcolor{gray}{Constitution}}

    \normalsize
    You've developed your poison resistance from constant exposure to poisons in your environment and from your own skin.
    You are immune to poison damage and the poisoned condition.
    \paragraph{REQUIREMENTS} Kin: Grung.

    % ========================================================================== %
\subsection*{Uman Feats}
    \subsubsection{Reptile Claws} \label{feat::reptileclaws}
    \small{\textcolor{gray}{Strength}}

    \normalsize
    Your claws are natural weapons, which you can use to make unarmed strikes.
    If you hit with them, you deal slashing damage equal to 1d4 + your Strength modifier, instead of the bludgeoning damage normal for an unarmed strike.
    \paragraph{REQUIREMENTS} Kin: Tortle.

    % ========================================================================== %
\subsection*{Zaloth Feats}
    \subsubsection{Reptile Claws} \label{feat::reptileclaws}
    \small{\textcolor{gray}{Strength}}

    \normalsize
    Your claws are natural weapons, which you can use to make unarmed strikes.
    If you hit with them, you deal slashing damage equal to 1d4 + your Strength modifier, instead of the bludgeoning damage normal for an unarmed strike.
    \paragraph{REQUIREMENTS} Kin: Tortle.

    % ========================================================================== %
\subsection*{Quies Feats}
    \subsubsection{Integrated Weapon} \label{feat::integratedweapon}
    \small{\textcolor{gray}{Intelligence}}

    \normalsize
    Your understanding of your own body allows you to modify it with ease.
    As part of a short rest, you can integrate a weapon into your form.
    You can only have one weapon integrated at a time, and you need a short rest in order to separate from it.

    You can equip and unequip your integrated weapon as a free action.
    When inactive, the weapon is effectively invisible, requiring no check of any kind to conceal.
    \paragraph{REQUIREMENTS} Kin: Quies.

    % ========================================================================== %
    \subsubsection{Immutable Form} \label{feat::immutableform}
    \small{\textcolor{gray}{Constitution}}

    \normalsize
    Embracing your origin, you become protected by the strange designs of the tall kin.
    You are immune to any effect that would alter your form.
    \paragraph{REQUIREMENTS} Kin: Quies.

    % ========================================================================== %
    \subsubsection{Long-Limbed (2 FP)} \label{feat::longlimbed}
    \small{\textcolor{gray}{Athletics}}

    \normalsize
    You harness your inherited adaptability, learning how to alter your proportions at will.
    When you make a melee attack on your turn, your reach for it is 1.5 meters greater than normal.
    \paragraph{REQUIREMENTS} Kin: Quies.

    % ========================================================================== %
    \subsubsection{Designed with a Purpose (3 FP)} \label{feat::designedwithapurpose}
    \small{\textcolor{gray}{Relevant Skill}}

    \normalsize
    Following your creator's design, your competence at a particular skill is unparalleled.
    Your proficiency with a skill is increased from Master to Legendary, increasing your proficiency bonus to +10 with it.
    You can gain this feat only once.
    \paragraph{REQUIREMENTS} Kin: Quies Operative. Master proficiency with a skill.
    % ========================================================================== %
    \subsubsection{Thermal Manipulation} \label{feat::thermalmanipulation}
    \small{\textcolor{gray}{Intelligence}}

    \normalsize
    The years of hard work have taught you how to manipulate your obsidian extensions, modifying its hardness and properties at will.
    The damage die of your unarmed strikes is increased to a d6.
    Additionally, you can change their damage type to bludgeoning, slashing, fire, or cold as a free action.
    \paragraph{REQUIREMENTS} Kin: Quies Juggernaut.

    % ========================================================================== %
    \subsubsection{Rugged Aspect} \label{feat::ruggedaspect}
    \small{\textcolor{gray}{Stealth}}

    \normalsize
    By manipulating the composition of your natural armor, you are able to better shift into specific environments.
    You have advantage on Stealth checks made to hide in rocky and similar terrain.
    \paragraph{REQUIREMENTS} Kin: Quies Slag Worker.

    % ========================================================================== %

% ============================================================================== %
\subsection{Artisan Feats}
\subsection*{Healer's Kit}
    \subsubsection{Healer} \label{feat::healer}
    \small{\textcolor{gray}{Intelligence}} % TODO: Sure?

    \normalsize
    Increase your level of proficiency with the healer's kit.
    This feat can be re-taken until you are a master with the artisan's tools.
    % ========================================================================== %

    \subsubsection{Quick Mender} \label{feat::quickmender}
    \small{\textcolor{gray}{Prof w/ Healer's Kit}}

    \normalsize
    Used to working in the field, you can heal minor injuries as part of a short rest expending one use of your healer's kit.
    A creature healed with this feat also gains a number of temporary hit points equal to its Constitution modifier.
    \paragraph{REQUIREMENTS} Competent with healer's kit.
    % ========================================================================== %

    \subsubsection{Resuscitator} \label{feat::resuscitator}
    \small{\textcolor{gray}{Prof w/ Healer's Kit}}

    \normalsize
    When you use a healer's kit to stabilize a dying creature, that creature ealso regains 1 hit point.
    \paragraph{REQUIREMENTS} Competent with healer's kit.
    % ========================================================================== %

    \subsubsection{Physician} \label{feat::physician}
    \small{\textcolor{gray}{Prof w/ Healer's Kit}}

    \normalsize
    As an avid physician, you can expend 3 uses of a healer's kit to heal a major injury as part of a long rest.
    The creature must succeed on a DC 12 Constitution check for the treatment to work.
    \paragraph{REQUIREMENTS} Expert with healer's kit.
    % ========================================================================== %

    \subsubsection{Combat Medic} \label{feat::combatmedic}
    \small{\textcolor{gray}{Prof w/ Healer's Kit}}

    \normalsize
    You are able to mend wounds quickly and get your allies back in the fight.
    Using two actions, you can spend one use of a healer's kit to tend to a creature and restore 1d6 + your proficiency modifier with the healer's kit hit points to it, plus additional hit points equal to the creature's maximum number of Hit Dice.
    The creature can't regain hit points from this feat again until it finishes a short rest.
    \paragraph{REQUIREMENTS} Expert with Healer's kit.
    % ========================================================================== %

    \subsubsection{Travelling Doctor} \label{feat::travellingdoctor}
    \small{\textcolor{gray}{Prof w/ Healer's Kit}}

    \normalsize
    You keep your allies in top shape during your travels.
    The minimum number of hit points you and your allies regain from a Hit Die roll is equal to your Constitution modifier (minimum of 2).
    This effect is only valid if the Hit Die is rolled as part of a short rest.
    \paragraph{REQUIREMENTS} Master with Healer's kit.
    % ========================================================================== %

    \subsubsection{Therapist} \label{feat::therapist}
    \small{\textcolor{gray}{Prof in Persuation}}

    \normalsize
    A keen healer and convincing persuader, you can heal a creature's insanity as part of a long rest.
    The creature must succeed on a DC 12 Intelligence check for the treatment to work.
    \paragraph{REQUIREMENTS} Expert in the Persuasion skill and competent with Healer's kit.
    % ========================================================================== %

\subsubsection{Reliever} \label{feat::reliever}
\small{\textcolor{gray}{Prof in Persuasion}}

\normalsize
You are able to motivate and relieve the most stressed of folks.
As part of a short rest, you can reduce an ally's stress by one point.
You can only use this feat once per short rest, and cannot reduce your own stress via this manner.
\paragraph{REQUIREMENTS} Expert in the Persuasion skill.
% ========================================================================== %

\subsubsection{Entertainer} \label{feat::entertainer}
\small{\textcolor{gray}{Prof in Performance}}

\normalsize
A favorite of the masses, you are an expert at making people laugh and relax.
During a long rest, you and your allies reduce stress by one additional point.
\paragraph{REQUIREMENTS} Master in the Performance skill.
% ========================================================================== %

% NOTE:
% Each background has 6 feats associated to it.
% One is related to their initial feat, usually related to acquiring shelter from their people, extra food from the wilds, etc.
% At least one is useful in combat.
% One is upgradeable. If related to a DC, use 12, 15, and 18.
% One required 2 FP, and is a lv 6 ability from a related class.
% Two are shared with other classes, and are considered outside of the rules previously mentioned, but must only need 1 FP.
\subsection*{Background Feats}
ACOLYTE 5/5, 1/1
    \subsubsection{Shelter of the Faithful} \label{feat::shelterofthefaithful}
        Your piety inspires the respect of those who share your faith.
        After performing a religious ceremony of your deity, you and your companions can expect to receive free healing and care at a temple, shrine, or other established presence of your faith, though you must provide any material components needed for spells.
        Those who share your religion will support you (but only you) at a modest lifestyle.

        Additionally, your devotion might rouse members from other religions to look kindly to you.
        After performing a religious ceremony or act of kindness, you can roll an Intelligence (Religion) check contested by the creature's Intelligence (Religion).
        On a success, you gain the benefits from this feat from the creature's creed.
        This check is made with advantage if yours and the target deity share tide, and automatically fails if they are enemy deities from the same pantheon.

        % You might also have ties to a specific temple dedicated to your chosen deity or pantheon, and you have a residence there.
        % This could be the temple where you used to serve, if you remain on good terms with it, or a temple where you have found a new home.
        % While near your temple, you can call upon the priests for assistance, provided the assistance you ask for is not hazardous and you remain in good standing with your temple.
        \paragraph{REQUIREMENTS} Acolyte background.
    % ========================================================================== %

    \subsubsection{Divine Inspiration} \label{feat::divineinspiration}
        As an action, you can touch your holy symbol, utter a prayer, and regain one expended spell slot, the level of which can be no higher than 1/5th your level, rounded up.
        You can use this feat once per short rest.
        \paragraph{REQUIREMENTS} Acolyte background.
    % ========================================================================== %

    \subsubsection{Guidance} \label{feat::guidance}
        As an action, you give words of encouragement to one willing creature.
        Once during the next minute, the target can roll a d4 and add the number rolled to one ability check of its choice.
        It can roll the die before or after making the ability check.

        Only one creature can be affected by this ability at a time.
        \paragraph{REQUIREMENTS} Acolyte background
    % ========================================================================== %

    \subsubsection{Incite Respect} \label{feat::inciterespect}
        As an action, you present your holy symbol, and each creature of your choice that can see or hear you within 9 meters of you must succeed on a Wisdom saving throw of DC 12 or be charmed by you until the end of your next turn or until the charmed creature takes any damage.
        You can also cause any of the charmed creatures to drop what they are holding when they fail the saving throw.

        You can use this ability a number of times per short rest equal to your Wisdom modifier (Minimum of one).

        You can learn this feat a total of three times, increasing the DC to 15 the second time and to 18 the third.
        \paragraph{REQUIREMENTS} Acolyte background
    % ========================================================================== %

    \subsubsection{Divine Healing (2 FP)} \label{feat::divinehealing}
        When you roll a 1 or 2 on a die related to healing or giving temporary hit points to one or more creatures, you can reroll the die and must use the new roll, even if the new is a 1 or a 2.
        Additionally, whenever you heal a creature (including yourself) you heal yourself by 1.
        \paragraph{REQUIREMENTS} Acolyte background.
    % ========================================================================== %

    \subsubsection{Inpiring Leader} \label{feat::inspiringleader}
        You can spend 10 minutes inspiring your companions, shoring up their resolve to fight.
        When you do so, choose up to six friendly creatures (which can include yourself) within 9 meters of you who can see or hear you and who can understand you.
        Each creature can gain temporary hit points equal to your level + your Charisma modifier.
        A creature can't gain temporary hit points from this feat again until it has finished a short rest.
        \paragraph{REQUIREMENTS} Acolyte or Soldier background.
    % ========================================================================== %

ARTISAN 4/4, 2/2
    \subsubsection{Known Crafter} \label{feat::knowncrafter}
        Knowing the local trade like the back of your hand, you know where to find the best ingredients for the cheapest prices in a settlement in which you are familiar.
        You can buy components and ingredients related to your craft for half their normal price, and you can tell the quality of a raw material just by looking at it.
        To use this feat you must first gain familiarity with a settlement as part of a long rest.
        \paragraph{REQUIREMENTS} Artisan background.
    % ========================================================================== %

    \subsubsection{The Right Tool for the Job} \label{feat::therighttoolforthejob}
        In times of need, your vast experience allows you to improvise solutions, adapt to any adversity, and overcome insurmountable challenges.
        You learn how to produce exactly the tool you need.
        By gathering resources from any environment, you can create one set of artisan's tools with which you are already competent with.
        This creation requires 1 hour of uninterrupted work, which can coincide with a short or long rest.
        Tools conjured via this feat are flimsy at best, and no reasonable person would buy them.
        \paragraph{REQUIREMENTS} Artisan background.
    % ========================================================================== %

    \subsubsection{Artificer's Infusion} \label{feat::artificersinfusion}
        You gain the ability to imbue mundane items with infusions.

        When you gain this feature, pick four infusions to learn, choosing from the ``Artificer Infusions'' table.
        The description of each of the following infusions details the type of item that can receive it, along with whether the resulting magic item requires attunement.
        Unless an infusion's description says otherwise, you can't learn an infusion more than once.

        Whenever you finish a long rest, you can touch a nonmagical object and imbue it with one of your artificer infusions.
        An infusion works on only certain kinds of objects, as specified in the infusion's description.
        If the item requires attunement, you can attune yourself to it the instant you infuse the item.
        If you decide to attune to the item later, you must do so using the normal process for attunement (see ``Attunement'' in chapter 7 of the Dungeon Master's Guide).

        Your infusion remains in an item indefinitely, but only one object can be infused at a time.
        If you try to exceed your maximum number of infusions, the oldest infusion immediately ends, and then the new infusion applies.

        You can learn this feat 3 times.
        The second and third time, you learn an additional infusion, and increase the number of objects that can be infused at the same time by one.

        \paragraph{REQUIREMENTS} Artisan background.

        \begin{table*}[!ht]%
            \begin{DndTable}[width=\linewidth, header=Artificer Infusions]{ll}
                \textbf{Name} & \textbf{Description} \\
                Armor of Strength &
                \textit{Prerequisite: A suit of armor (requires attunement)}.

                This armor has 6 charges. The wearer can expend the armor's charges in the following ways:
                \begin{itemize}
                    \item When the wearer makes a Strength check or a Strength saving throw, it can expend 1 charge to add a bonus to the roll equal to its Intelligence modifier.
                    \item If the creature would be knocked prone, it can use its reaction to expend 1 charge to avoid being knocked prone.
                \end{itemize}
                The armor regains 1d6 expended charges daily at dawn. \\

                Enhanced Focus    &
                \textit{Prerequisite: An arcane focus (requires attunement)}.

                While holding this item, a creature gains a +1 bonus to spell attack rolls. In addition, the creature ignores half cover when making a spell attack.

                The bonus increases to +2 when you reach 10th level. \\

                Enhanced Defense  &
                \textit{Prerequisite: A suit of armor or a shield}.

                A creature gains a +1 bonus to Armor Class while wearing (armor) or wielding (shield) the infused item.

                The bonus increases to +2 when you reach 10th level. \\

                Enhanced Weapon   &
                \textit{Prerequisite: A simple or martial weapon}.

                This weapon grants a +1 bonus to attack and damage rolls made with it.

                The bonus increases to +2 when you reach 10th level. \\

                Mind Sharpener    &
                \textit{Prerequisite: A suit of armor or robes}.

                The infused item can send a jolt to the wearer to refocus their mind.
                The item has 4 charges.
                When the wearer fails a Constitution saving throw to maintain concentration on a spell, the wearer can use its reaction to expend 1 of the item's charges to succeed instead.
                The item regains 1d4 expended charges daily at dawn. \\

                Repeating Shot    &
                \textit{Prerequisite: A simple or martial weapon with the ammunition property (requires attunement)}.

                This weapon grants a +1 bonus to attack and damage rolls made with it when it's used to make a ranged attack, and it ignores the loading property if it has it.

                If you load no ammunition in the weapon, it produces its own, automatically creating one piece of ammunition when you make a ranged attack with it.
                The ammunition created by the weapon vanishes the instant after it hits or misses a target. \\

                Returning Weapon  &
                \textit{Prerequisite: A simple or martial weapon with the thrown property}.

                This weapon grants a +1 bonus to attack and damage rolls made with it, and it returns to the wielder's hand immediately after it is used to make a ranged attack.
            \end{DndTable}
        \end{table*}
    % ========================================================================== %

    \subsubsection{Beyond Mastery (2 FP)} \label{feat::beyondmastery}
        Practicing for your whole life, you can take your skill as an artisan further than most mortals.
        You increase your proficiency with a set of artisan's tools from Expert to Master, increasing your proficiency bonus to +9.
        \paragraph{REQUIREMENTS} Artisan background, Expert proficiency with a set of artisan's tools.
    % ========================================================================== %

    \subsubsection{Flash of Genius} \label{feat::flashofgenius}
        You gain the ability to come up with solutions under pressure.
        When you or another creature you can see within 9 meters of you makes an ability check or a saving throw, you can use your reaction to add your Intelligence modifier to the roll.

        You can use this feature a number of times equal to your Intelligence modifier (minimum of once).
        You regain all expended uses when you finish a short rest.
        \paragraph{REQUIREMENTS} Artisan or Scholar background.
    % ========================================================================== %

CRIMINAL 4/4, 2/2
    \subsubsection{Criminal Contact} \label{feat::criminalcontact}
        Based on your connections, you can gain a criminal contact in a settlement as part of a long rest.
        This contact is reliable and trustworthy, and acts as your liaison to a network of other criminals.
        You know how to get messages to and from your contacts, even over great distances; specifically, you know the local messengers, corrupt caravan masters, and seedy sailors who can deliver messages for you.
        \paragraph{REQUIREMENTS} Criminal background.
    % ========================================================================== %

    \subsubsection{Undercity Paths} \label{feat::undercitypaths}
        You know hidden, underground pathways that you can use to bypass crowds, obstacles, and observation as you move through the city.
        When you aren't in combat, you and companions you lead can travel between any two locations in the city twice as fast as your speed would normally allow.
        The paths of the undercity are haunted by dangers that rarely brave the light of the surface world, so your journey isn't guaranteed to be safe.
        \paragraph{REQUIREMENTS} Criminal background.
    % ========================================================================== %

    \subsubsection{Cunning Action} \label{feat::cunningaction}
        Your quick thinking and agility allow you to move and act quickly.
        Choose an action between Dash, Disengage, or Hide.
        The chosen action only costs you one action to perform instead of two.

        You can take this feat up to three times, selecting a different action each time.
        \paragraph{REQUIREMENTS} Criminal background.
    % ========================================================================== %

    \subsubsection{Dual Personalities (2 FP)} \label{feat::dualpersonalities}
        By carefully managing your connections and your appearance, you have effectively created a second persona for yourself.
        Roll on the Faceless Persona table to determine your persona, or work with the DM to create a persona that's unique to your character and suits the tone of your game.

        \begin{DndTable}[width=\linewidth, header=Persona]{cX}
            \textbf{d20} & \textbf{Faceless Persona}                \\
            1  & A flamboyant spy or brigand.                       \\
            2  & The incarnation of a nation or people.             \\
            3  & A scoundrel with a masked guise.                   \\
            4  & A vengeful spirit.                                 \\
            5  & The manifestation of a deity or your faith.        \\
            6  & One whose beauty is greatly accented using makeup. \\
            7  & An impersonation of another hero.                  \\
            8  & The embodiment of a school of magic.               \\
            9  & A warrior with distinctive armor.                  \\
            10 & A disguise with animalistic or monstrous characteristics, meant to inspire fear.
        \end{DndTable}

        Most of the people who know you do so as your persona.
        Those who seek to learn more about you --- your weaknesses, your origins, your purpose --- find themselves stymied by your disguise.
        Upon donning a disguise and behaving as your persona, you are unidentifiable as your true self.
        By removing your disguise and revealing your true face, you are no longer identifiable as your persona.
        This allows you to change appearances between your two personalities as often as you wish, using one to hide the other or serve as convenient camouflage.
        However, should someone realize the connection between your persona and your true self, your deception might lose its effectiveness.
        \paragraph{REQUIREMENTS} Criminal background.
    % ========================================================================== %

    \subsubsection{Urban Infrastructure} \label{feat::urbaninfrastructure}
        You have a basic knowledge of the structure of buildings, including the stuff behind the walls.
        You can also find blueprints of a specific building in order to learn the details of its construction.
        Such blueprints might provide knowledge of entry points, structural weaknesses, or secret spaces.
        Your access to such information isn't unlimited, and obtaining it can sometimes get you in trouble with the law.
        \paragraph{REQUIREMENTS} Criminal or Laborer background.
    % ========================================================================== %

ENTERTAINER 4/4, 2/2
    \subsubsection{Rakish Audacity} \label{feat::rakishaudacity}
        Your confidence propels you into battle.
        You can give yourself a bonus to your initiative rolls equal to your Charisma modifier.
        \paragraph{REQUIREMENTS} Entertainer background.
    % ========================================================================== %

    \subsubsection{Bardic Inspiration} \label{feat::bardicinspiration}
        You can inspire others through stirring words, actions, or music.
        To do so, you use an action on your turn to choose one creature other than yourself within 18 meters of you who can hear you.
        That creature gains one Bardic Inspiration die, a d6.

        Once within the next 10 minutes, the creature can roll the die and add the number rolled to one ability check, attack roll, or saving throw it makes.
        The creature can wait until after it rolls the d20 before deciding to use the Bardic Inspiration die, but must decide before the DM says whether the roll succeeds or fails.
        Once the Bardic Inspiration die is rolled, it is lost.
        A creature can have only one Bardic Inspiration die at a time.

        You can use this feature a number of times equal to your Charisma modifier (a minimum of once).
        You regain any expended uses when you finish a short rest.

        You can take this feat additional times to improve your Bardic Inspiration die.
        On a second time it becomes a d8, on a third a d10, and on a fourth a d12.

        \paragraph{REQUIREMENTS} Entertainer background.
    % ========================================================================== %

    \subsubsection{Song of Rest} \label{feat::songofrest}
        You can use soothing music, oration, or a performance to help revitalize your wounded allies during a short rest.
        If you or any friendly creatures who can see or hear your performance regain hit points by spending Hit Dice at the end of the short rest, each of those creatures regains an extra 1d6 hit points.

        The extra hit points increase when you reach certain levels: to 1d8 at 6th level, to 1d10 at 11th level, and to 1d12 at 16th level.
        \paragraph{REQUIREMENTS} Entertainer background.
    % ========================================================================== %

    \subsubsection{Jack of All Trades (2 FP)} \label{feat::jackofalltrades}
        You can add a bonus equal to 1/10th of your level, rounded up, to any ability check with which you are not already proficient.
        \paragraph{REQUIREMENTS} Entertainer background.
    % ========================================================================== %

    \subsubsection{Courtier} \label{feat::courtier}
        Your knowledge of how bureaucracies function lets you gain access to the records and inner workings of any noble court or government you encounter.
        You know who the movers and shakers are, whom to go to for the favors you seek, and what the current intrigues of interest in the group are.
        When encountering a new court, you need to spend at least a short rest among their ranks to map their inner workings, which you may do by performing for them or using your status to gain an invitation to their dwellings.
        \paragraph{REQUIREMENTS} Entertainer or Noble background.
    % ========================================================================== %

LABORER 4/4, 2/2
    \subsubsection{Deep Miner} \label{feat::deepminer}
        You are used to navigating the deep places of the earth.
        You never get lost in caves or mines if you have either seen an accurate map of them or have been through them before.
        Furthermore, you are able to scrounge fresh water and food for yourself and as many as five other people each day if you are in a mine or natural caves.
        \paragraph{REQUIREMENTS} Laborer background.
    % ========================================================================== %

    \subsubsection{Resilient} \label{feat::resilient}
        Stout beyond belief, you are living proof of the hardiness of the common folk.
        Choose an ability score.
        You gain competence in saving throws using the chosen ability.

        You can take this feat 3 times, improving your proficiency level with the saving throws of the chosen ability each time.
        \paragraph{REQUIREMENTS} Laborer background.
    % ========================================================================== %

    \subsubsection{Powerful Build} \label{feat::powerfulbuild}
        Hardened by work, you count as one size larger when determining your carrying capacity and the weight you can push, drag, or lift.
        \paragraph{REQUIREMENTS} Laborer background.
    % ========================================================================== %

    \subsubsection{Commoner's Toughness (2 FP)} \label{feat::commonerstoughness}
        Your hit point maximum increases by an amount equal to half your level (rounded up) when you gain this feat.
        Whenever you gain an odd level thereafter, your hit point maximum increases by an additional hit point.
        \paragraph{REQUIREMENTS} Laborer background.
    % ========================================================================== %

    \subsubsection{Rustic Hospitality} \label{feat::rustichospitality}
        You've spent your life in company of the masses, and the common folk thank you for it.
        You can find a place to hide, rest, or recuperate among other commoners, unless you have shown yourself to be a danger to them.
        They will shield you from the law or anyone else searching for you, though they will not risk their lives for you.
        For this feat to work, you need to have worked or performed in the settlement at least once.
        \paragraph{REQUIREMENTS} Entertainer or Laborer background.
    % ========================================================================== %

MERCHANT 4/4, 2/2
    \subsubsection{Never Tell Me the Odds} \label{feat::nevertellmetheodds}
        Odds and probability are your bread and butter.
        During downtime activities that involve games of chance or figuring odds on the best plan, you can get a solid sense of which choice is likely the best one and which opportunities seem too good to be true, at the DM's determination.
        \paragraph{REQUIREMENTS} Merchant background.
    % ========================================================================== %

    \subsubsection{Silver Tongue} \label{feat::silvertongue}
        You are a master at saying the right thing at the right time.
        When you make a Charisma (Persuasion) or Charisma (Deception) check, you can treat a d20 roll of 9 or lower as a 10.
        \paragraph{REQUIREMENTS} Merchant background.
    % ========================================================================== %

    \subsubsection{Unsettling Words} \label{feat::unsettlingwords}
        You can spin words laced with malice that unsettle a creature and cause it to doubt itself.
        Using one action, you can choose one creature you can see within 18 meters of you.
        Roll a d6.
        The creature must subtract the number rolled from the next saving throw it makes before the start of your next turn.
        A creature can only be affected by this once per round.

        You can use this feature a number of times equal to your Charisma modifier (a minimum of once).
        You regain any expended uses when you finish a short rest.

        You can take this feat additional times to improve the die.
        On a second time it becomes a d8, on a third a d10, and on a fourth a d12.
        \paragraph{REQUIREMENTS} Merchant background.
    % ========================================================================== %

    \subsubsection{Master Negotiator (2 FP)} \label{feat::masternegotiator}
        An expert trader, haggling is your second nature.
        When you interact with a fellow merchant, you can roll a Charisma (Persuasion) check contested by the merchant's Charisma (Persuasion).
        If you succeed, the price for any item you buy from them is reduced by 25\%.
        \paragraph{REQUIREMENTS} Merchant background.
    % ========================================================================== %

    \subsubsection{Down Low} \label{feat::downlow}
        After spending a short rest in a settlement, you acquaint yourself with a network of smugglers who are willing to help you out of tight situations.
        While in a town or larger community, you and your companions can stay for free in safe houses.
        Safe houses provide a poor lifestyle.
        While staying at a safe house, you can choose to keep your presence (and that of your companions) a secret.
        \paragraph{REQUIREMENTS} Criminal or Merchant background.
    % ========================================================================== %

    \subsubsection{Guild Membership} \label{feat::guildmembership}
        As an known figure of your trade, you can become a member of a guild associated to it.
        This membership provides you with certain benefits.
        Your fellow guild members will provide you with lodging and food if necessary, and pay for your funeral if needed.
        In some cities and towns, a guildhall offers a central place to meet other members of your profession, which can be a good place to meet potential patrons, allies, or hirelings.

        Guilds often wield tremendous political power.
        If you are accused of a crime, your guild will support you if a good case can be made for your innocence or the crime is justifiable.
        You can also gain access to powerful political figures through the guild, if you are a member in good standing.
        Such connections might require the donation to the guild's coffers.

        To maintain your membership, you must pay dues of 5 agnomas per month to the guild.
        If you miss payments, you must make up back dues to remain in the guild's good graces.
        \paragraph{REQUIREMENTS} Artisan or Merchant background.
    % ========================================================================== %

NOBLE 4/4, 2/2
    % \subsubsection{Inheritance} \label{feat::inheritance}
    %     Choose or randomly determine your inheritance from the possibilities in the table below.
    %     Work with your Dungeon Master to come up with details: Why is your inheritance so important, and what is its full story?
    %     You might prefer for the DM to invent these details as part of the game, allowing you to learn more about your inheritance as your character does.
    %
    %     The Dungeon Master is free to use your inheritance as a story hook, sending you on quests to learn more about its history or true nature, or confronting you with foes who want to claim it for themselves or prevent you from learning what you seek.
    %     The DM also determines the properties of your inheritance and how they figure into the item's history and importance.
    %     For instance, the object might be a minor magic item, or one that begins with a modest ability and increases in potency with the passage of time.
    %     Or, the true nature of your inheritance might not be apparent at first and is revealed only when certain conditions are met.
    %
    %     You can decide whether or when to tell your companions about your inheritance.
    %     Rather than attracting attention to yourself, you might want to keep your inheritance a secret until you learn more about what it means to you and what it can do for you.
    %
    %     \begin{DndTable}[width=\linewidth, header=Inheritance]{cX}
    %         \textbf{d8} & \textbf{Object or item}                                  \\
    %         1   & A document such as a map, a letter, or a journal                 \\
    %         2-3 & a trinket (see "Trinkets" in chapter 5 of the Player's Handbook) \\
    %         4   & an article of clothing                                           \\
    %         5   & a piece of jewelry                                               \\
    %         6   & an arcane book or formulary                                      \\
    %         7   & a written story, song, poem, or secret                           \\
    %         8   & a tattoo or other body marking
    %     \end{DndTable}
    %     \paragraph{REQUIREMENTS} Noble background.
    % % ========================================================================== %

    \subsubsection{Retainers} \label{feat::retainers}
        You gain the service of three retainers loyal to your family.
        These retainers can be attendants or messengers, and one might be a majordomo.
        One of your retainers can serve as your squire, aiding you in exchange for training on their own path to knighthood.
        Another retainer might be a groom to care for your horse or a servant who polishes your armor and helps you put it on.

        Your retainers can perform mundane tasks for you, but they do not fight for you, follow you into obviously dangerous areas (such as dungeons).
        Your retainers will leave if they are frequently endangered or abused, and it will take your family 1d6 weeks to send you a new group.
        \paragraph{REQUIREMENTS} Noble background.
    % ========================================================================== %

    \subsubsection{Early Training} \label{feat::earlytraining}
        From an early age, your family procured only the best education for you.
        You learn one Fighting Style (see page \pageref{ssec::fightingstyles}) or one Casting Style (see page \pageref{ssec::castingstyle}) of your choice.
        If you already have a style, the one you choose must be different.
        \paragraph{REQUIREMENTS} Noble background.
    % ========================================================================== %

    \subsubsection{Promise of Status} \label{feat::promiseofstatus}
        As an action, you shout your family name to inspire fear from your enemies.
        Each creature that can hear and understand you within 9 meters of you must succeed on a DC 12 Charisma saving throw.
        On a failure, the creature is frightened of you until the end of your next turn.
        At the DM's discretion, a creature might alternatively cease fighting altogether or even join your cause, expecting retribution from your family afterwards.

        You can use this ability a number of times per short rest equal to your Charisma modifier (Minimum of one).

        You can learn this feat a total of three times, increasing the DC to 15 the second time and to 18 the third.
        \paragraph{REQUIREMENTS} Noble background
    % ========================================================================== %

    \subsubsection{Kept in Style (2 FP)} \label{feat::keptinstyle}
        Your family is known across the realm, and your attitude reflects this.
        Your name and signet are sufficient to cover most of your expenses, since everyone would prefer to be in your family's good graces.

        This advantage enables you and your companions to receive most services for free, as long as the cost doesn't rise above 10 agnomas.
        For services up to 50 agnomas, roll a Charisma (Persuasion) check contested by the target's Wisdom (Insight).
        On a success, you can enjoy the service for free.

        At the DM discretion, this feat may be used for more expensive or rarer services, perhaps with disadvantage on the roll or by promising a reward to the service provider.
        \paragraph{REQUIREMENTS} Noble background.
    % ========================================================================== %

SAILOR 5/5, 1/1
    \subsubsection{I'll Patch It!} \label{feat::illpatchit}
        Provided you have carpenter's tools and wood, you can perform repairs on a water vehicle.
        You don't need to be competent with carpenter's tools to gain this feat.
        When you use this ability, you restore a number of hit points to the hull of a water vehicle equal to twice your level.
        A vehicle cannot be patched by you in this way again until you finish a short rest.
        \paragraph{REQUIREMENTS} Sailor background.
    % ========================================================================== %

    \subsubsection{Harvest the Water} \label{feat::harvestthewater}
        You gain advantage on ability checks made using fishing tackle.
        If you have access to a body of water that sustains marine life, you can maintain a moderate lifestyle while working as a fisher, and you can catch enough food to feed yourself and up to ten other people each day.
        \paragraph{REQUIREMENTS} Sailor background.
    % ========================================================================== %

    \subsubsection{Sailor's Balance} \label{feat::sailorsbalance}
        Used to the ever-present swing of a ship, you have advantage on any ability checks and saving throws made to avoid being moved or being knocked prone.
        \paragraph{REQUIREMENTS} Sailor background.
    % ========================================================================== %

    \subsubsection{Expert Stirrer} \label{feat::expertstirrer}
        Years of arguing and interacting at seas have made your insults extraordinarily effective.
        As an action, you can hurl a terrible insult to a creature within 18 meters of you who can hear and understand your language.
        The creature must roll a DC 12 Wisdom saving throw.
        On a failure, the creature has disadvantage on attack rolls against targets other than you and can't make opportunity attacks against targets other than you.
        This effect lasts for 1 minute, until one of your companions attacks the target or affects it with a spell, or until you and the target are more than 18 meters apart.

        You can use this ability a number of times per short rest equal to your Charisma modifier (Minimum of one).

        You can learn this feat a total of three times, increasing the DC to 15 the second time and to 18 the third.
        \paragraph{REQUIREMENTS} Sailor background.
    % ========================================================================== %

    \subsubsection{Drunken Resilience (2 FP)} \label{feat::drunkenresilience}
        You can drink enough alcohol to knock an elephant.
        You have advantage on saving throws against poison, and you have resistance against poison damage.
        \paragraph{REQUIREMENTS} Sailor background.
    % ========================================================================== %

    \subsubsection{Wisdom of the Wilds} \label{feat::wisdomofthewilds}
        Your extended travels have given you a special appreciation for the stillness of the world.
        After finishing a short rest, you gain temporary hit points equal to your level + your Wisdom modifier, and you have advantage on the first ability check you make during the day.
        \paragraph{REQUIREMENTS} Sailor or Outlander background.
    % ========================================================================== %

SCHOLAR 4/4, 2/2
    % \subsubsection{Library Access} \label{feat::libraryaccess}
    %     You are a well-known scholar (or good enough at pretending to be one), and have free and easy access to the majority of libraries.
    %     This doesn't give you access to repositories of lore that are too valuable or secret to permit anyone immediate access.
    %
    %     Additionally, you are likely to gain preferential treatment at libraries and universities accross Yuadrem, as professional courtesy to a fellow scholar.
    %     \paragraph{REQUIREMENTS} Scholar background.
    % ========================================================================== %

    \subsubsection{Adept Linguist} \label{feat::adeptlinguist}
        You can communicate with people who don't speak any language you know.
        You must observe the people interacting with one another for at least 1 day, after which you learn a handful of important words, expressions, and gestures --- enough to communicate on a rudimentary level.
        \paragraph{REQUIREMENTS} Scholar background.
    % ========================================================================== %

    \subsubsection{Historical Knowledge} \label{feat::historicalknowledge}
        When you enter a ruin or dungeon, you can correctly ascertain its original purpose and determine its builders, whether those were gats, oths, ets, thri-kreen, or some other known kin.
        In addition, you can determine the monetary value of art objects more than a century old.
        \paragraph{REQUIREMENTS} Scholar background.
    % ========================================================================== %

    \subsubsection{Metamagic Adept} \label{feat::metamagicadept}
        By understanding the relation between a variety of doctrines, you've learned how to exert your will on your spells to alter how they function:
        \begin{itemize}
            \item You learn one metamagic option of your choice (see page \pageref{ssec::metamagic}).
            You can use only one metamagic option on a spell when you cast it, unless the option says otherwise.
            \item You gain 2 metamagic points to spend on Metamagic (these points are added to any metamagic points you have from another source but can be used only on Metamagic).
            You regain all spent metamagic points when you finish a short rest.
        \end{itemize}

        You can take this feat 3 times, learning a new metamagic option the second and third time.
        \paragraph{REQUIREMENTS} Scholar background and Spellcasting feat.
    % ========================================================================== %

    \subsubsection{Skill Versatility (2 FP)} \label{feat::skillversatility}
        Your life of studies has given you the capacity to quickly learn new subjects.
        You gain two proficiency levels on a skill of your choice.
        This skill must be related to Intelligence or Wisdom.

        After a long rest, you can chance this proficiency to another skill, also related to Intelligence or Wisdom.
        \paragraph{REQUIREMENTS} Scholar background.
    % ========================================================================== %

    \subsubsection{Legalese} \label{feat::legalese}
        Your experience with your local legal system has given you a firm knowledge of the ins and outs of that system.
        Even when the law is not on your side, you can use complex terms like ``ex injuria jus non oritur'' and ``cogitationis poenam nemo patitur'' to frighten people into thinking you know what you're talking about.
        With common folks who don't know any better, you might be able to intimidate or deceive to get favors or special treatment.
        \paragraph{REQUIREMENTS} Noble or Scholar background.
    % ========================================================================== %

inspiringleader steady
SOLDIER 0/4, 2/2
    \subsubsection{Warrior's Life} \label{feat::warriorslife}
        You know local mercenaries as only someone who has worked with them can.
        You are able to identify mercenary companies by their emblems, and you know a little about any such company, including who has hired them recently.
        You can find the taverns and festhalls where mercenaries abide in any area, as long as you speak the language.
        You can find mercenary work between adventures sufficient to maintain a comfortable lifestyle.
        \paragraph{REQUIREMENTS} Soldier background.
    % ========================================================================== %

    \subsubsection{Watcher's Eye} \label{feat::watcherseye}
        Your experience in enforcing the law, and dealing with lawbreakers, gives you a feel for local laws and criminals.
        You can easily find the local outpost of the watch or a similar organization, and just as easily pick out the dens of criminal activity in a community, although you're more likely to be welcome in the former locations rather than the latter.
        \paragraph{REQUIREMENTS} Soldier background.
    % ========================================================================== %

    \subsubsection{Official Inquiry} \label{feat::officialinquiry}
        You're experienced at gaining access to people and places to get the information you need.
        Through a combination of fast-talking, determination, and official-looking documentation, you can gain access to a place or an individual related to a something you're investigating.
        Those who aren't involved in your investigation avoid impeding you or pass along your requests.
        \paragraph{REQUIREMENTS} Soldier background.
    % ========================================================================== %

    \subsubsection{} \label{feat::NAME}
        DESCRIPTION
        \paragraph{REQUIREMENTS} Soldier background.
    % ========================================================================== %

    \subsubsection{Steady} \label{feat::steady}
        You can move twice the normal amount of time (up to 16 hours) each day before being subject to the effect of a forced march (see ``Travel Pace'' in chapter 8 of the Player's Handbook).
        \paragraph{REQUIREMENTS} Outlander or Soldier background.
    % ========================================================================== %

steady wisdomofthewilds
OUTLANDER 0/4, 2/2
    \subsubsection{All Eyes on You} \label{feat::alleyesonyou}
        Your accent, mannerisms, figures of speech, and perhaps even your appearance all mark you as foreign.
        You gain the friendly interest of scholars and others intrigued by far-removed lands, to say nothing of everyday folk who are eager to hear stories of your origin.

        You can parley this attention into access to people and places you might not otherwise have, for you and your traveling companions.
        Noble lords, scholars, and merchant princes, to name a few, might be interested in hearing about your homeland.
        \paragraph{REQUIREMENTS} Outlander background.
    % ========================================================================== %

    \subsubsection{Nature's Heritage} \label{feat::naturesheritage}
        You are familiar enough with any wilderness area that you can find twice as much food and water as you normally would when you forage there.
        \paragraph{REQUIREMENTS} Outlander background.
    % ========================================================================== %

    \subsubsection{Discovery} \label{feat::discovery} % NOTE: Not sure 'bout this one b0ss.
        The quiet seclusion of your extended hermitage gave you access to a unique and powerful discovery.
        It might be a great truth about the cosmos, the deities, the powerful beings of the outer planes, or the forces of nature.
        It could be a site that no one else has ever seen.
        You might have uncovered a fact that has long been forgotten, or unearthed some relic of the past that could rewrite history.
        It might be information that would be damaging to the people who or consigned you to exile, and hence the reason for your return to society.

        Work with your DM to determine the details of your discovery and its impact on the campaign.
        \paragraph{REQUIREMENTS} Outlander background.
    % ========================================================================== %

    \subsubsection{} \label{feat::NAME}
        DESCRIPTION
        \paragraph{REQUIREMENTS} Outlander background.
    % ========================================================================== %

% === SKILLS =================================================================== %
% \subsubsection{Gifted} \label{feat::gifted} %
% \small{\textcolor{gray}{Arcana}}

% \normalsize
% By chance or mysterious circumstance, you've gained a special attunement with the arcane arts of the world.
% It's easy for you to perceive when magic is used around you, and your extensive studies allow you to even identify the spells being cast.
% \paragraph{RANK 1} You are proficient with the Arcana skill.
% \paragraph{RANK 2} You learn one harmless cantrip of your choice that doesn't belong to any magic school.
% \paragraph{RANK 3} You double your proficiency modifier in the Arcana skill.

% ============================================================================== %
% \subsubsection{} \label{tal::}
% \small{\textcolor{gray}{}}

% \normalsize
% Description.
% \paragraph{REQUIREMENTS}
% \paragraph{RANK 1}
% \paragraph{RANK 2}
% \paragraph{RANK 3}

% % ============================================================================== %
% === ARMOR ==================================================================== %
\subsubsection{Stealthy} \label{feat::stealthy}
\small{\textcolor{gray}{Stealth}}

\normalsize
You know how best to hide, and use your dyed light armor to benefit your stealth.
\paragraph{REQUIREMENTS} Lightly Armored 2 and Sly 2.
\paragraph{RANK 1} You can take the Hide action as a bonus action on each of your turns.
\paragraph{RANK 2} If you are hidden, you can move up to 3 meters in the open without revealing yourself if you end the move in a position where you're not clearly visible.
\paragraph{RANK 3}

\subsubsection{Imposing Figure} \label{feat::imposingfigure}
\small{\textcolor{gray}{Intimidation}}

\normalsize
Description.
\paragraph{REQUIREMENTS} Menacing 2 and Heavily Armored 2.
\paragraph{RANK 1}
\paragraph{RANK 2}
\paragraph{RANK 3}

% ============================================================================== %
% === WEAPONS ================================================================== %
% === UNARMED ================================================================== %
% === SIMPLE WEAPONS =========================================================== %
% === MARTIAL WEAPONS ========================================================== %
% \subsubsection{Nunchaku Master} \label{feat::nunchakumaster}
% \small{\textcolor{gray}{Dexterity}}

% \normalsize
% Description.
% \paragraph{REQUIREMENTS} Flail Adept 2 and Acrobat 2.
% \paragraph{RANK 1}
% \paragraph{RANK 2} extra damage
% \paragraph{RANK 3} ADVANCED TECHNIQUE

% ============================================================================== %
% TODO: exotic shields (spiked, throwing, etc)

% === COMBAT STYLES ============================================================ %
\subsubsection{Numbed Senses} \label{feat::numbedsenses}
\small{\textcolor{gray}{Constitution}}

\normalsize
Your heavy experience in manual labor or melee combat has granted you an increased resilience against strikes of all kind.
\paragraph{REQUIREMENTS} Constitution 13.
\paragraph{RANK 1}
\paragraph{RANK 2}
\paragraph{RANK 3}

% ============================================================================== %
\subsubsection{Desensitized Combatant} \label{feat::desensitizedcombatant}
\small{\textcolor{gray}{Constitution}}

\normalsize
Description.
\paragraph{REQUIREMENTS} Constitution 16. Numbed Senses 2.
\paragraph{RANK 1}
\paragraph{RANK 2}
\paragraph{RANK 3}

% ============================================================================== %
\subsubsection{Flexible Fighter} \label{feat::flexiblefighter} %
\small{\textcolor{gray}{Intelligence}}

\normalsize
Description.
\paragraph{REQUIREMENTS} Intelligence 13. Armed Fighter 2.
\paragraph{RANK 1} You are able to use weapons with the Trick property.
\paragraph{RANK 2}
\paragraph{RANK 3} TECHNIQUE

% ============================================================================== %
\subsubsection{Adaptable} \label{feat::adaptable} %
\small{\textcolor{gray}{Intelligence}}

\normalsize
Description.
\paragraph{REQUIREMENTS} Intelligence 16. Flexible Fighter 2.
\paragraph{RANK 1} % advantage when investigating creatures before combat
\paragraph{RANK 2}
\paragraph{RANK 3}

% ============================================================================== %
\subsubsection{Experienced Combatant} \label{feat::experiencedcombatant} %
\small{\textcolor{gray}{Wisdom}}

\normalsize
Description.
\paragraph{REQUIREMENTS} Wisdom 13. Armed Fighter 2.
\paragraph{RANK 1}
\paragraph{RANK 2}
\paragraph{RANK 3} TECHNIQUE

% ============================================================================== %
\subsubsection{Ready-for-Anything} \label{feat::readyforanything} %
\small{\textcolor{gray}{Wisdom}}

\normalsize
Description.
\paragraph{REQUIREMENTS} Wisdom 16. Experienced Combatant 2.
\paragraph{RANK 1}
\paragraph{RANK 2}
\paragraph{RANK 3}

% ============================================================================== %
\subsubsection{Captain} \label{feat::captain}
\small{\textcolor{gray}{Charisma}}

\normalsize
Description.
\paragraph{REQUIREMENTS} Charisma 13. Armed Fighter 2.
\paragraph{RANK 1}
\paragraph{RANK 2}
\paragraph{RANK 3} TECHNIQUE

% ============================================================================== %
\subsubsection{Commander} \label{feat::commander}
\small{\textcolor{gray}{Charisma}}

\normalsize
Description.
\paragraph{REQUIREMENTS} Charisma 16. Mesmerizing Warrior 2.
\paragraph{RANK 1}
\paragraph{RANK 2}
\paragraph{RANK 3} % Advantage on persuasion, deception, performance, and intimidation checks while fighting

% ============================================================================== %
\subsubsection{Mounted Combatant} \label{feat::mountedcombatant} %
\small{\textcolor{gray}{Animal Handling}}

\normalsize
Description.
\paragraph{REQUIREMENTS} Animal Handler 2.
\paragraph{RANK 1}
\paragraph{RANK 2}
\paragraph{RANK 3}

% ============================================================================== %
\subsubsection{Exotic Rider} \label{feat::exoticrider} %
\small{\textcolor{gray}{Animal Handling}}

\normalsize
Description.
\paragraph{REQUIREMENTS} Mounted Combatant 2.
\paragraph{RANK 1}
\paragraph{RANK 2}
\paragraph{RANK 3} You don't attack with disadvantage on ranged attacks while riding flying creatures.

% ============================================================================== %
\subsubsection{Hidden Striker} \label{feat::hiddenstriker} %
\small{\textcolor{gray}{Stealth}}

\normalsize
Description.
\paragraph{REQUIREMENTS} Sly 2 and Armed Fighter 2.
\paragraph{RANK 1}
\paragraph{RANK 2}
\paragraph{RANK 3} You gain or improve the Sneak Attack technique.

% ============================================================================== %
\subsubsection{Assassin} \label{feat::assassin} %
\small{\textcolor{gray}{Stealth}}

\normalsize
Description.
\paragraph{REQUIREMENTS} Hidden Striker 2.
\paragraph{RANK 1}
\paragraph{RANK 2}
\paragraph{RANK 3} You gain or improve the Sneak Attack technique.

% ============================================================================== %
\subsubsection{Fire-breather} \label{feat::firebreather}
\small{\textcolor{gray}{Science}}

\normalsize
Description.
\paragraph{REQUIREMENTS} Educated 2.
\paragraph{RANK 1} You are proficient with Drer's fire-breathers.
\paragraph{RANK 2}
\paragraph{RANK 3}

% ============================================================================== %
\subsubsection{Drunken Brawler} \label{feat::drunkenbrawler} %
\small{\textcolor{gray}{}}
% Skills emulating drunken fighting AND actual bonus when drunk.

\normalsize
Description.
\paragraph{REQUIREMENTS}
\paragraph{RANK 1}
\paragraph{RANK 2} When you suffer from the effects of drunkenness, you have advantage on Constitution and Strength saving throws.
\paragraph{RANK 3} You learn the Careless Deflect technique.

% ============================================================================== %
% === MAGIC SCHOOLS ============================================================ %
% All schools of magic require at least one rank in SPELLCASTING and one rank in a skill or tool (depends on the school).
% Spellcasting
% Bonereading (bonecarving tools)
% Wordbinding (two ranks in standard language - second is basic true speech)
% Windherding (acrobatics)
% Sigaldry (two ranks in naenk tongue - first to learn language, second to learn shinerunes)
% Psionics (two ranks in mind speech - first is listening to zaloths that don't necessarily want to be heard, second is feelspeech, and third is actual mind speech)
% Thaumaturgy (science)
% Tidal Manipulation (religion)
% Fleshshaping (nature)
% === TOOLS ==================================================================== %
% DISGUISE KIT
% RANK 2: PRODUCE A CONVINCING DISGUISE WITH STUFF YOU FIND AROUND YOU

% ============================================================================== %
\subsubsection{Healer} \label{feat::healer}
\small{\textcolor{gray}{Medicine}}

\normalsize
An able physician, you are able to quickly mend wounds and get your allies back in the fight.
\paragraph{REQUIREMENTS} Mender 2.
\paragraph{RANK 1} You gain proficiency with the healer's kit.
\paragraph{RANK 2} You learn the Heal technique.
\paragraph{RANK 3}

% ============================================================================== %
\subsubsection{Master of Disguise} \label{feat::masterofdisguise}
\small{\textcolor{gray}{Performance}}

\normalsize
Description.
\paragraph{REQUIREMENTS} Performer 2.
\paragraph{RANK 1} You gain proficiency with the disguise kit.
\paragraph{RANK 2}
\paragraph{RANK 3} If you spend one hour observing a creature, you can then spend a long rest crafting a disguise you can quickly don to mimic that creature.
Making the disguise requires a disguise kit.
You must make checks as normal to disguise yourself, but you can assume the disguise as an action.

% ============================================================================== %
\subsubsection{Poisoner} \label{feat::poisoner}
\small{\textcolor{gray}{Nature}}

\normalsize
Description.
\paragraph{REQUIREMENTS} Naturalist 2.
\paragraph{RANK 1} You gain proficiency with the poisoner's kit.
\paragraph{RANK 2} When you make a damage roll, you ignore resistance to poison damage.
Additionally, you can coat a weapon in poison as a bonus action, instead of an action.
\paragraph{RANK 3} The Constitution saving throws of the poisons you make are increased by your proficiency modifier.

% ============================================================================== %
\subsubsection{Thief} \label{feat::thief}
\small{\textcolor{gray}{Sleight of Hand}}

\normalsize
Description.
\paragraph{REQUIREMENTS} Quick Fingers 2.
\paragraph{RANK 1} You gain proficiency with thieves' tools.
\paragraph{RANK 2}
\paragraph{RANK 3} You can use your thieves' tools to disarm a trap or open a lock as a bonus action.

% ============================================================================== %
\subsubsection{Tinkerer} \label{feat::tinkerer}
\small{\textcolor{gray}{Science}}

\normalsize
Description.
\paragraph{REQUIREMENTS} Educated 2.
\paragraph{RANK 1} You gain proficiency with tinker's tools.
\paragraph{RANK 2}
\paragraph{RANK 3}

% ============================================================================== %
\subsubsection{Weaver} \label{feat::weaver}
\small{\textcolor{gray}{Sleight of Hand}}

\normalsize
Description.
\paragraph{RANK 1} You gain proficiency with weaver's tools.
\paragraph{RANK 2}
\paragraph{RANK 3}

% ============================================================================== %
% LANGUAGES
% EXTRA STUFF
\subsubsection{Alert} \label{feat::alert}
\small{\textcolor{gray}{Perception}}

\normalsize
Description.
\paragraph{REQUIREMENTS} Inquisitive 2, Insightful 2, and Observant 2.
\paragraph{RANK 1} You can't be surprised while you are conscious.
\paragraph{RANK 2} Other creatures don't gain advantage on attack rolls against you as a result of being unseen by you.
\paragraph{RANK 3} You gain a +5 bonus to initiative.

% ============================================================================== %
\subsection*{Heroic Feats}
\subsubsection{Demented Insight} \label{feat::dementedinsight}
\small{\textcolor{gray}{-}}

\normalsize
A true master of the mind, you are able to retain your sentience without a qualar.
You are immune to the effects of dementia.
\paragraph{REQUIREMENTS} Succesfully recover from the dementia status (page \pageref{ssec::dementia}).

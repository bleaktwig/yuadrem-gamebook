% !TEX root = ../main.tex
\addcontentsline{toc}{section}{Skill Feats}
\subsection*{Skill Feats}
% TODO: Maybe add one feat per ability score that requires a negative score? Like better Deception checks if str is negative, unpredictability if int is negative, clumsiness if dex is negative, etc.
% STRENGTH brute powerfulbuild_skill powerhouse throwingarm
\subsubsection{Brute} \label{feat::brute}
    You can take the Overrun action (see page \pageref{act::overrun}) as part of your movement.
    \paragraph{Requirements} Strength 12.
\subsubsection{Powerful Build} \label{feat::powerfulbuild_skill}
    You count as one size larger when determining your carrying capacity and the weight you can push, drag, or lift.
    \paragraph{Requirements} Strength 15.
\subsubsection{Throwing Arm} \label{feat::throwingarm}
    Attacking at long range doesn't impose disadvantage on your thrown weapon attack rolls.
    In addition, all weapons and object that weigh less than you have the thrown property for you, and you can throw them up to a range equal to your Strength ability score.
    \paragraph{Requirements} Strength 18.
\subsubsection{Powerhouse (2 FP)} \label{feat::powerhouse}
    For actions that are restricted by the target's size (like the Grapple or Shove actions), you can use them for targets of one size larger than what is defined by the action.
    \paragraph{Requirements} Strength 21.
% DEXTERITY evasion nimble otherwordlyagility uncannydodge
\subsubsection{Nimble} \label{feat::nimble}
    You can take the Tumble action (see page \pageref{act::tumble}) as part of your movement.
    \paragraph{Requirements} Dexterity 12.
\subsubsection{Uncanny Dodge} \label{feat::uncannydodge}
    When an attacker that you can see hits you with an attack, you can use your reaction to halve the attack's damage against you.
    \paragraph{Requirements} Dexterity 15.
\subsubsection{Evasion} \label{feat::evasion}
    You can nimbly dodge out of the way of certain area effects, such as a wyvern's acid breath or an ice storm spell.
    When you are subjected to an effect that allows you to make a Dexterity saving throw to take only half damage, you instead take no damage if you succeed on the saving throw, and only half damage if you fail.
    \paragraph{Requirements} Dexterity 18.
\subsubsection{Otherworldly Agility (2 FP)} \label{feat::otherwordlyagility}
    Increase your moving speed by 3 meters.

    In addition, you double the bonus granted from your Dexterity modifier to your initiative rolls.
    \paragraph{Requirements} Dexterity 21.
% CONSTITUTION
\subsubsection{NAME} \label{feat::name}
    DESCRIPTION
    \paragraph{Requirements} Constitution 12.
\subsubsection{NAME} \label{feat::name}
    DESCRIPTION
    \paragraph{Requirements} Constitution 15.
\subsubsection{Hit die Improvement} \label{feat::hitdieimprovement}
    You increase your hit die from a d6 to a d8, a d8 to a d10, or a d10 to a d12.
    Additionally, you gain a number of hit points equal to your level when you take this feat.
    \paragraph{Requirements} Constitution 18.
\subsubsection{Cling to Life (2 FP)} \label{feat::clingtolife}
    Resilient even in the face of death, you roll death saving throws with advantage.

    In addition, if you are attacked while unconscious, you can roll a Constitution saving throw with a DC equal to 10 or half the damage done, whichever's higher.
    If you succeed, you don't suffer a death saving throw failure.
    \paragraph{Requirements} Constitution 21.
% INTELLIGENCE
\subsubsection{NAME} \label{feat::name}
    DESCRIPTION
    \paragraph{Requirements} Intelligence 12.
\subsubsection{NAME} \label{feat::name}
    DESCRIPTION
    \paragraph{Requirements} Intelligence 15.
\subsubsection{NAME} \label{feat::name}
    DESCRIPTION % Add half you intelligence modifier (rounded down) to all checks with tools.
    \paragraph{Requirements} Intelligence 18.
\subsubsection{NAME (2 FP)} \label{feat::name}
    DESCRIPTION
    \paragraph{Requirements} Intelligence 21.
% WISDOM
\subsubsection{Wizened by Experience} \label{feat::wizenedbyexperience}
    You can add your Wisdom modifier to your initiative rolls.

    In addition, during the first round of combat, if you are not surprised, your AC is increase by an amount equal to your Wisdom modifier against the first attack that targets you.
    \paragraph{Requirements} Wisdom 12.
\subsubsection{NAME} \label{feat::name}
    DESCRIPTION
    \paragraph{Requirements} Wisdom 15.
\subsubsection{NAME} \label{feat::name}
    DESCRIPTION
    \paragraph{Requirements} Wisdom 18.
\subsubsection{NAME (2 FP)} \label{feat::name}
    DESCRIPTION
    \paragraph{Requirements} Wisdom 21.
% CHARISMA
\subsubsection{Saving Face} \label{feat::savingface}
    You are careful not to show weakness in front of their allies, for fear of losing status.
    If you miss with an attack roll or fail an ability check or a saving throw, you can gain a bonus to the roll equal to the number of allies you can see within 9 meters of you (maximum bonus of +5).
    Once you use this trait, you can't use it again until you finish a short rest.
    \paragraph{Requirements} Charisma 12
\subsubsection{NAME} \label{feat::name}
    DESCRIPTION
    \paragraph{Requirements} Charisma 15.
\subsubsection{NAME} \label{feat::name}
    DESCRIPTION
    \paragraph{Requirements} Charisma 18.
\subsubsection{NAME (2 FP)} \label{feat::name}
    DESCRIPTION
    \paragraph{Requirements} Charisma 21.
% ATHLETICS alwaysready athlete jogger robust strongbody
\subsubsection{Athlete} \label{feat::athlete}
    You increase your proficiency level in the Athletics skill.
    This feat can be taken three times, or until you reach Expert proficiency on the skill.
\subsubsection{Always Ready} \label{feat::alwaysready}
    Your physical prowess is a marvel to most.
    When you are prone, standing up doesn't provoke attacks of opportunity.
    \paragraph{Requirements} Competent proficiency in the Athletics skill.
\subsubsection{Strong Body} \label{feat::strongbody}
    You count as if you were one size larger for the purpose of determining your carrying capacity.
    \paragraph{Requirements} Skilled proficiency in the Athletics skill.
\subsubsection{Jogger} \label{feat::jogger}
    Your movement speed is increased by 1.5 meters.
    You can take this feat three times, increasing your speed by the same amount each time.

    The second time you take this feat you also gain a swimming speed equal to your moving speed.

    The third time you gain the ability to move twice the normal amount of time (up to 16 hours) each day before being subject to the effect of a forced march (see ``Travel Pace'' in chapter 8 of the Player's Handbook).
    \paragraph{Requirements} Skilled proficiency in the Athletics skill.
\subsubsection{Robust (2 FP)} \label{feat::robust}
    Your hit point maximum increases by an amount equal to your level when you take this feat.
    Whenever you gain a level thereafter, your hit point maximum increases by an additional hit point.
    \paragraph{Requirements} Expert proficiency in the Athletics skill.
% ACROBATICS acrobat aerialist allterrain fleetfooted mobile
\subsubsection{Acrobat} \label{feat::acrobat}
    You increase your proficiency level in the Acrobatics skill.
    This feat can be taken three times, or until you reach Expert proficiency on the skill.
\subsubsection{Aerialist} \label{feat::aerialist}
    You've had an exceptional flexibility from a very young age.
    You can make a running long jump or a running high jump after moving only 1.5 meters or foot, rather than 3 meters.
    \paragraph{Requirements} Competent proficiency in the Acrobatics skill.
\subsubsection{All-Terrain} \label{feat::allterrain}
    Moving through difficult terrain costs you no extra movement.
    \paragraph{Requirements} Skilled proficiency in the Acrobatics skill.
\subsubsection{Mobile} \label{feat::mobile}
    Your movement speed is increased by 1.5 meters.
    You can take this feat three times, increasing your speed by the same amount each time.

    The second time you take this feat you also gain a climbing speed equal to your moving speed.

    The third time you gain the ability to run along vertical surfaces, falling to the ground if you don't finish your move in a horizontal surface.
    \paragraph{Requirements} Skilled proficiency in the Acrobatics skill.
\subsubsection{Fleet-Footed (2 FP)} \label{feat::fleetfooted}
    You reduce the cost of the Disengage action to one action.
    In addition, opportunity attacks made against you are made with disadvantage.
    % In addition, if an attack of opportunity against you misses, you can use your reaction to make a melee attack against the attacker.
    \paragraph{Requirements} Expert proficiency in the Acrobatics skill.
% SLEIGHT OF HAND fasthands quickfingers phantom pickpocket thief
\subsubsection{Quick Fingers} \label{feat::quickfingers}
    You increase your proficiency level in the Sleight of Hand skill.
    This feat can be taken three times, or until you reach Expert proficiency on the skill.
\subsubsection{Pickpocket} \label{feat::pickpocket}
    Other creatures have disadvantage on Intelligence (Investigation) and Wisdom (Perception) checks to see you stealing or notice an object you have stolen.
    \paragraph{Requirements} Competent proficiency in the Sleight of Hand skill.
\subsubsection{Thief} \label{feat::thief}
    While they've given you problems in the past, your quick and sticky fingers have awarded you with many a treasure thorough your life.
    You learn the Steal action (see page \pageref{act::steal}).
    \paragraph{Requirements} Skilled proficiency in the Sleight of Hand skill.
\subsubsection{Fast Hands} \label{feat::fasthands}
    You can learn this feat three times:
    \begin{itemize}
        \item The first, you reduce the action cost of the Use and Object action to one.
        \item The second, you reduce the action cost of the Search action to one.
        \item The last, you reduce the action cost of the Steal action to one.
    \end{itemize}
    \paragraph{Requirements} Skilled proficiency in the Sleight of Hand skill.
\subsubsection{Phantom (2 FP)} \label{feat::phantom} % NOTE: Consider renaming at some point.
    You have taken the time to hone your larcenous skills, making you an accomplished thief.
    If you spend at least one minute observing or interacting with another creature outside combat, you learn about anything valuable the creature has that can be stolen.
    In addition, any Dexterity (Sleight of Hand) checks you make to steal from the creature are made with advantage.
    \paragraph{Requirements} Expert proficiency in the Sleight of Hand skill.
% STEALTH ghost hiddenstriker skulker sly sneak
\subsubsection{Sneak} \label{feat::sneak}
    You increase your proficiency level in the Stealth skill.
    This feat can be taken three times, or until you reach Expert proficiency on the skill.
\subsubsection{Sly} \label{feat::sly}
    One with shadows, you can always escape from the most unfavorable situations.
    You have a natural ease melding with the dark, and intuitively know how to avoid making any noise.
    You have advantage on any Dexterity (Stealth) check if you move no more than half your speed on the same turn.
    \paragraph{Requirements} Competent proficiency in the Stealth skill.
\subsubsection{Skulker} \label{feat::skulker}
    You are an expert at slinking through shadows.
    Missing a ranged attack against a target doesn't alert your target.
    In addition, you can try to hide when you are lightly obscured from the creature from which you are hiding.
    \paragraph{Requirements} Skilled proficiency in the Stealth skill.
\subsubsection{Hidden Striker} \label{feat::hiddenstriker}
    You learn or improve the Sneak Attack action (see page \pageref{act:sneakattack}).
    You can take this feat three times, improving the action's damage by a d6 each time.
    \paragraph{Requirements} Skilled proficiency in the Stealth skill.
\subsubsection{Ghost (2 FP)} \label{feat::ghost}
    You reduce the action cost of the Hide action to one action.
    In addition, when you make a ranged attack against a creature while hidden, roll Dexterity (Stealth) contested against the creaute's Wisdom (Perception).
    If you win, the attack doesn't immediately reveal your position.
    \paragraph{Requirements} Expert proficiency in the Stealth skill.
% ARCANA arcanist countercaster gifted identify metamagicinitiate
\subsubsection{Arcanist} \label{feat::arcanist}
    You increase your proficiency level in the Arcana skill.
    This feat can be taken three times, or until you reach Expert proficiency on the skill.
\subsubsection{Gifted} \label{feat::gifted}
    By chance or mysterious circumstance, you've gained a special attunement with the arcane arts of the world.
    You learn one harmless cantrip of your choice that doesn't belong to any magic school.
    \paragraph{Requirements} Competent proficiency in the Arcana skill.
\subsubsection{Metamagic Initiate} \label{feat::metamagicinitiate}
    You learn one metamagic option from the metamagic table (see page \pageref{ssec::metamagic}).
    You can use only one metamagic option on a spell when you cast it, unless the option says otherwise.
    \paragraph{Requirements} Skilled proficiency in the Arcana skill.
\subsubsection{Identify} \label{feat::identify}
    % NOTE: Normally, a magic can only be identified by paying an arcanist. This is the feat they use for that.
    You can take this feat multiple times, gaining a different effect each times:
    \begin{itemize}
        \item You choose one object you are holding or can touch.
        After analyzing it for a minute, you learn its properties and how to use it.
        \item As part of the same action, you learn whether any spells are affecting the item and what they are.
        If the item was created by a spell, you learn which spell created it.
        \item As part of a long rest, you can identify magic items without consuming any materials.
        At the DM's discretion, you might not be able to use this ability on a very rare item or lost artifact.
        In this case, you learn the origin of the item and know where you can go to get more information.
    \end{itemize}
    \paragraph{Requirements} Skilled proficiency in the Arcana skill.
\subsubsection{Countercaster (2 FP)} \label{feat::countercaster}
    Educated in all sorts of doctrines, you have advantage in the Identify a Spell action.
    In addition, when you identify a spell you can try to prevent it from being casted.
    If the casting creature is no more than 18 meters away from you, you can roll another Intelligence (Arcana) check with a DC equal to 15 + the spell's level using the same reaction.
    On a success, the creature's spell fails and has no effect.
    \paragraph{Requirements} Expert proficiency in the Arcana skill.
% HISTORY advisor cognizant historian sharpenedmind wizenedadvice
\subsubsection{Historian} \label{feat::historian}
    You increase your proficiency level in the History skill.
    This feat can be taken three times, or until you reach Expert proficiency on the skill.
\subsubsection{Cognizant} \label{feat::cognizant}
    From background or perchance you've been given a privileged access to books and teaching thorough your life.
    You have advantage on Intelligence (History) checks to recall any information of your kin and your country of origin.
    \paragraph{Requirements} Competent proficiency in the History skill.
\subsubsection{Sharpened Mind} \label{feat::sharpenedmind}
    You can accurately recall anything you have seen or heard within the past month.
    \paragraph{Requirements} Skilled proficiency in the History skill.
\subsubsection{Wizened Advice} \label{feat::wizenedadvice}
    You can take this feat three times, obtaining different effects each time:
    \begin{itemize}
        \item You increase the range of the Help action to 9 meters, but you can only use the action in this way if the creature can hear and understand you.
        \item When you use the Help action to aid a creature's attack roll, you can make a DC 15 Intelligence (History) check.
        On a success, the creature gains a bonus on the attack roll equal to your proficiency bonus in History, as you share pertinent advice and historical examples.
        \item Emboldened by your words, the creature also gains a bonus equal to half your proficiency bonus in History (rounded down).
    \end{itemize}
    \paragraph{Requirements} Skilled proficiency in the History skill.
\subsubsection{Advisor (2 FP)} \label{feat::advisor}
    As an action, you can give a creature a bonus equal to your proficiency bonus with the History skill to any ability check or saving throw.
    The creature must be able to hear and understand you.

    You can use this ability a number of times equal to your Intelligence modifier (Minimum of one).
    \paragraph{Requirements} Expert proficiency in the History skill.
% INVESTIGATION detective eyeforweakness inquisitive investigative unerringeye
\subsubsection{Investigative} \label{feat::investigative}
    You increase your proficiency level in the Investigation skill.
    This feat can be taken three times, or until you reach Expert proficiency on the skill.
\subsubsection{Inquisitive} \label{feat::inquisitive}
    No detail can miss your analytical mind.
    Your extensive experience and awareness means you can always tell when something's amiss.

    Your inquisitiveness allows you to tell when you are missing a detail or clue in a situation, even after a failed roll.
    This doesn't allow you to roll again, but the lingering uneasiness stays in your mind.
    \paragraph{Requirements} Competent proficiency in the Investigation skill.
\subsubsection{Detective} \label{feat::detective}
    You reduce the action cost of the Search action to one action.
    \paragraph{Requirements} Skilled proficiency in the Investigation skill.
\subsubsection{Unerring Eye} \label{feat::unerringeye}
    You excel at rooting out secrets and unraveling mysteries.
    You can take this feat three times, obtaining different benefits each time:
    \begin{itemize}
        \item You have advantage on any Wisdom (Perception) or Intelligence (Investigation) check if you move no more than half your speed on the same turn.
        \item You gain a +5 bonus to your passive Intelligence (Investigation) score.
        \item Your senses are almost impossible to foil.
        You sense the presence of illusions, any creature not in their original form, and other magic designed to deceive the senses within 9 meters of you, provided you aren't blinded or deafened.
        You sense that an effect is attempting to trick you, but you gain no insight into what is hidden or into its true nature.
    \end{itemize}
    \paragraph{Requirements} Skilled proficiency in the Investigation skill.
\subsubsection{Eye for Weakness (2 FP)} \label{feat::eyeforweakness}
    You learn to exploit a creature's weaknesses by carefully studying its tactics and movement.
    You learn the Aim action.
    When you successfully attack a creature after using this action, you roll the weapon's damage die one additional time.
    \paragraph{Requirements} Expert proficiency in the Investigation skill.
% NATURE deftexplorer experiencedtraveler hideinplainsight naturalawareness naturalist
\subsubsection{Naturalist} \label{feat::naturalist}
    You increase your proficiency level in the Nature skill.
    This feat can be taken three times, or until you reach Expert proficiency on the skill.
\subsubsection{Deft Explorer} \label{feat::deftexplorer}
    Your appreciation for nature has led you to attain an extensive categorical knowledge of plants and animals, giving you a great feat at recognizing and identifying them.

    Provided you are familiar with the local flora and fauna from experience or a guide, you don't need to make an ability check to identify common plants, fungi, animals, and insects.
    You gain familiarity with a region by taking a long rest in it.
    \paragraph{Requirements} Competent proficiency in the Nature skill.
\subsubsection{Natural Awareness} \label{feat::naturalawareness}
    By almost supernatural perception, you can sense the presence of poisons and poisonous creatures within 9 meters of you.

    \paragraph{Requirements} Skilled proficiency in the Nature skill.
\subsubsection{Experienced Traveler} \label{feat::experiencedtraveler}
    You are an unsurpassed explorer, using your uncanny knowledge of nature to understand and survive in any environment.
    You can take this feat multiple times, obtaining different effects each time:
    \begin{itemize}
        \item Difficult Terrain doesn't slow your group's travel.
        \item When you forage, you find twice as much food as you normally would.
        \item While tracking down creatures, you also learn their exact number, their sizes, and how long ago they passed through the area.
    \end{itemize}
    \paragraph{Requirements} Skilled proficiency in the Nature skill.
\subsubsection{Hide in Plain Sight (2 FP)} \label{feat::hideinplainsight}
    While in the wilds, your proficiency bonus with Intelligence (Nature) and Wisdom (Survival) are doubled.

    In addition, you can spend 1 minute creating camouflage for yourself.
    You must have access to fresh mud, dirt, plants, soot, and other naturally occurring materials with which to create your camouflage.

    Once you are camouflaged in this way, you can try to hide by pressing yourself up against a solid surface, such as a tree or wall, that is at least as tall and wide as you are.
    You gain a +10 bonus to Dexterity (Stealth) checks as long as you remain there without moving or taking actions.
    \paragraph{Requirements} Expert proficiency in the Nature skill.
% RELIGION holyfortitude peaceofmind pious studentofthetides theologian
\subsubsection{Theologian} \label{feat::theologian}
    You increase your proficiency level in the Religion skill.
    This feat can be taken three times, or until you reach Expert proficiency on the skill.
\subsubsection{Pious} \label{feat::pious}
    You're a zealot when it comes to your faith.
    Your devotion has led to an incredible understanding of your religion and its history, and no connoted figure goes unnoticed in your prayers.
    You can recall the name and description of every deity and important figure related to a religion of your choice without needing to succeed on any ability check.

    In addition, you have advantage on Charisma (Persuasion) and Charisma (Deception) checks against creatures that share your faith.
    \paragraph{Requirements} Competent proficiency in the Religion skill.
\subsubsection{Peace of Mind} \label{feat::peaceofmind}
    By sharing a prayer with a creature during a minute, you both reduce your stress levels by 1.
    If you and the creature share the same religion, the stress level reduction is increased to 2.
    You can use this ability once per short rest.
    \paragraph{Requirements} Skilled proficiency in the Religion skill.
\subsubsection{Student of the Tides} \label{feat::studentofthetides}
    You can take this feat three times, gaining different benefits each time:
    \begin{itemize}
        \item You can make an ability check with your Intelligence (Religion) modifier contested against a creature's Charisma (Deception) to tell its tidal alignment.
        You must be able to see the creature to use this ability.
        \item You have advantage on Charisma checks against creatures that shares your tidal alignment, and roll on any attempt made to charm such creature with advantage.
        You must know the tidal alignment of the creature to get this benefit.
        \item By knowing its tidal alignment you have a basic understanding of a creature's goals and ideals.
        As an action, you can roll a Charisma (Persuasion) or Intelligence (Religion) check contested with the creature's Wisdom (Insight).
        On a failure, the creature is charmed by you for a minute.
        This charm ends early if you or your companions attack the creature.
    \end{itemize}
    \paragraph{Requirements} Skilled proficiency in the Religion skill.
\subsubsection{Holy Fortitude (2 FP)} \label{feat::holyfortitude}
    Your piousness provides you with an extreme mental fortitude.
    You roll Constitution saving throws made to maintain concentration on a spell with advantage.

    In addition, you can naturally tell when you are affected by an illusion, but don't know the exact nature of it.
    You have advantage on any Intelligence (Investigation) checks made to identify the illusion.
    \paragraph{Requirements} Expert proficiency in the Religion skill.
% ANIMAL HANDLING animalcompanion animalhandler sympathetic tamer zoophilist
\subsubsection{Animal Handler} \label{feat::animalhandler}
    You increase your proficiency level in the Animal Handling skill.
    This feat can be taken three times, or until you reach Expert proficiency on the skill.
\subsubsection{Sympathetic} \label{feat::sympathetic}
    You can intuit what an animal is feeling and what its intentions are without making a Wisdom (Animal Handling) check.
    \paragraph{Requirements} Competent proficiency in the Animal Handling skill.
\subsubsection{Zoophilist} \label{feat::zoophilist}
    Ever since you were a kid, you've always surrounded yourself with animals, pets or otherwise.
    Animals are calmer when they're around you, and you yourself get the same feeling when around them.

    Animals naturally trust you, and you don't need to perform any check to calm an animal or monster that is not already violent toward you.
    \paragraph{Requirements} Skilled proficiency in the Animal Handling skill.
\subsubsection{Tamer} \label{feat::tamer}
    You can learn this feat three times, gaining different abilities each time:
    \begin{itemize}
        \item Using two actions, you can issue a command to a beast within 18 meters of you that can hear you and you haven't attacked or damaged in any way.
        Make a Wisdom (Animal Handling) check contested by the creature's Wisdom (Insight).
        If you succeed, you can issue a general command that the creature will follow during its next turn.
        \item You gain an increased insight when issuing commands to creatures, and decide the actions that the creature will take during its next turn.
        \item After succeeding on the ability check, you can issue subsequent commands to the creature for the next minute by expending one action per turn.
        The effect ends if you or your companions attack the creature.
        You cannot have more than one creature affected by this ability at once.
    \end{itemize}
    \paragraph{Requirements} Skilled proficiency in the Animal Handling skill.
\subsubsection{Animal Companion (2 FP)} \label{feat::animalcompanion}
    You can try to bond with a wild creature that is not violent towards you and has a number of hit dice equal or lower than yours.
    Make a Wisdom (Animal Handling) check contested by the creature's Wisdom (Insight).
    The creature's roll gains a bonus equal to half its number of hit dice (rounded down).
    If you succeed, you can spend 8 hours bonding with the beast, after which it becomes your animal companion.

    The beast rolls initiative as normal and acts on its own volition during its turns.
    On its turns, it will defend itself and you against aggressors.
    You can issue commands to it as an action without needing to make an ability check, and when you do so you can control the beast during its next turn.
    It never requires your command to use its reaction, such as when making an opportunity attack.

    If you are incapacitated or absent, your beast companion will act on its own, focusing on protecting you and itself.

    If you fail on the check, you cannot try to bond with the same beast again.
    \paragraph{Requirements} Expert proficiency in the Animal Handling skill.
% INSIGHT awesomealertness incredibleintuition insightful keenmind uncannyinsight
\subsubsection{Insightful} \label{feat::insightful}
    You increase your proficiency level in the Insight skill.
    This feat can be taken three times, or until you reach Expert proficiency on the skill.
\subsubsection{Keen Mind} \label{feat::keenmind}
    You always know which way is north, and you always know the number of hours left before the next sunrise or sunset.
    \paragraph{Requirements} Competent proficiency in the Insight skill.
\subsubsection{Uncanny Understanding} \label{feat::uncannyinsight}
    You can use one action to try to get uncanny insight about one humanoid you can see within 9 meters of you.
    Make a Wisdom (Insight) check contested by the target's Charisma (Deception) check.
    If your check succeeds, you have advantage on attack rolls and ability checks against the target until the end of your next turn.
    \paragraph{Requirements} Skilled proficiency in the Insight skill.
\subsubsection{Incredible Intuition} \label{feat::incredibleintuition}
    Either by primal intuition or extensive knowledge, you are always in complete awareness of your surroundings.
    You can learn this feat three times, gaining different effects each time:
    \begin{itemize}
        \item You have advantage on any Wisdom (Perception) or Intelligence (Investigation) checks made to detect the presence of secret doors and traps.
        \item You gain a +5 bonus to your passive Wisdom (Insight) score.
        \item It is nigh-impossible to lie to you.
        You make Wisdom (Insight) checks to determine whether a creature is lying with advantage, and you can treat a roll of 9 or lower on the d20 as a 10.
    \end{itemize}
    \paragraph{Requirements} Skilled proficiency in the Insight skill.
\subsubsection{Awesome Alertness (2 FP)} \label{feat::awesomealertness}
    You are permanently aware of danger, and have advantage on initiative rolls.

    In addition, you and any of your companions within 9 meters of you can't be surprised, except when incapacitated by something other than nonmagical sleep.
    You instantly awaken if you are sleeping naturally when combat begins.
    \paragraph{Requirements} Expert proficiency in the Insight skill.
% MEDICINE guardianangel healersinsight medic refinedhealing restoringrest
\subsubsection{Medic} \label{feat::medic}
    You increase your proficiency level in the Medicine skill.
    This feat can be taken three times, or until you reach Expert proficiency on the skill.
\subsubsection{Guardian Angel} \label{feat::guardianangel}
    When you successfully stabilize a dying creature, that creature also regains 1 hit point.
    \paragraph{Requirements} Competent proficiency in the Medicine skill.
\subsubsection{Healer's Insight} \label{feat::healersinsight}
    You can tell whether a creature is suffering from a wound or injury, where this injury is located, and how severe it is without rolling an ability check.
    \paragraph{Requirements} Skilled proficiency in the Medicine skill.
\subsubsection{Restoring Rest} \label{feat::restoringrest}
    During a short rest, you can clean and bind the wounds of up to six willing beasts and humanoids (including yourself).
    On a success, if a creature spends a hit die during this rest, that creature can forgo the roll and instead regain the maximum number of hit points the die can restore.
    A creature can do so only for one die per rest, regardless of how many hit dice it spends.

    You can take this feat two more times, increasing the number of hit dice affected by this ability by one each time.
    \paragraph{Requirements} Skilled proficiency in the Medicine skill.
\subsubsection{Refined Healing (2 FP)} \label{feat::refinedhealing}
    You mastered the physician's arts, and are able to heal any wound or bruise with ease.
    When you heal a creature by any means, the creature regains additional hit points equal to your proficiency bonus with the Medicine skill.
    \paragraph{Requirements} Expert proficiency in the Medicine skill.
% PERCEPTION blindsense observant perceptive refinedsight reliablesight
\subsubsection{Perceptive} \label{feat::perceptive}
    You increase your proficiency level in the Perception skill.
    This feat can be taken three times, or until you reach Expert proficiency on the skill.
\subsubsection{Reliable Sight} \label{feat::reliablesight}
    Being in a lightly obscured area doesn't impose disadvantage on your Intelligence (Investigation) and Wisdom (Perception) checks if you can see or hear.
    \paragraph{Requirements} Competent proficiency in the Perception skill.
\subsubsection{Observant} \label{feat::observant}
    Based on memory or description alone, you can easily spot a person in a crowd or an object among many.
    \paragraph{Requirements} Skilled proficiency in the Perception skill.
\subsubsection{Refined Sight} \label{feat::refinedsight}
    When you focus your mind on something, you are able to quickly perceive even the smallest details and imperfections.
    You can learn this feat three times, gaining different effects each time:
    \begin{itemize}
        \item You have advantage on any Wisdom (Perception) or Intelligence (Investigation) checks made to detect the presence of creatures.
        \item You gain a +5 bonus to your passive Wisdom (Perception) score.
        \item You can add half your proficiency bonus in Perception (rounded down) to your initiative.
    \end{itemize}
    \paragraph{Requirements} Skilled proficiency in the Perception skill.
\subsubsection{Blindsense (2 FP)} \label{feat::blindsense}
    Unnaturally perceptive, you gain tremorsense with a radius of 3 meters and blindsight with a radius of 9 meters.
    Tremorsense allows you to perceive anything on the ground or below, while blindsight allows you to perceive your surroundings without reyling on sight.
    \paragraph{Requirements} Expert proficiency in the Perception skill.
% SURVIVAL alwaysontrack hunterssense survivor trainedbynature tracker
\subsubsection{Survivor} \label{feat::survivor}
    You increase your proficiency level in the Survival skill.
    This feat can be taken three times, or until you reach Expert proficiency on the skill.
\subsubsection{Always on Track} \label{feat::alwaysontrack}
    You and your group can't become lost except by magical means.
    \paragraph{Requirements} Competent proficiency in the Survival skill.
\subsubsection{Trained by Nature} \label{feat::trainedbynature}
    As if raised by wolves, you are a natural survivor in the most harsh of circumstances.
    Your enviable endurance and experience allows you to be particularly effective at building shelter, purifying water, and foraging food.
    When you forage, you find twice as much food as you normally would, and you are always able to find a shelter to rest protected from the elements.
    \paragraph{Requirements} Skilled proficiency in the Survival skill.
\subsubsection{Tracker} \label{feat::tracker}
    You have spent enough time hunting to hone your skill to a venerable level.
    You can take this feat three times, gaining a different effect each time
    \begin{itemize}
        \item You have advantage on Wisdom (Survival) checks made to track creatures.
        \item You learn the Mark action.
        \item You increase the duration of the Mark action to 8 hours, and you automatically succeed on Wisdom (Survival) checks made to track the marked creature.
    \end{itemize}
    \paragraph{Requirements} Skilled proficiency in the Survival skill.
\subsubsection{Hunter's Sense (2 FP)} \label{feat::hunterssense}
    You gain the ability to peer at a creature and discern how best to hurt it.
    As an action, choose one creature you can see within 18 meters of you.
    You immediately learn whether the creature has any damage immunities, resistances, or vulnerabilities and what they are.
    % If the creature is hidden from divination magic, you sense that it has no damage immunities, resistances, or vulnerabilities.

    You can use this ability a number of times equal to your Wisdom modifier (minimum of once)
     You regain all expended uses of it when you finish a short rest.
    \paragraph{Requirements} Expert proficiency in the Survival skill.
% DECEPTION convincingspeech deceiver feigninnocence liar taunt
\subsubsection{Deceiver} \label{feat::deceiver}
    You increase your proficiency level in the Deception skill.
    This feat can be taken three times, or until you reach Expert proficiency on the skill.
\subsubsection{Liar} \label{feat::liar}
    You can easily lie to someone to their face without batting an eye.
    While you might not always be proud of this skill, you can't deny that it's saved you in countless situations.

    When you fail to convince someone with a lie, it'll still take them two turns to realize you're lying, provided the lie isn't completely absurd.
    \paragraph{Requirements} Competent proficiency in the Deception skill.
% \subsubsection{Vicious Mockery} \label{feat::viciousmockery}
%     Using two actions, you can unleash a string of insults at a creature you can see within 18 meters of you.
%     If the target can hear and understand you, it must succeed on a Wisdom saving throw with DC 8 + your Charisma modifier or take 1d4 psychic damage and have disadvantage on the next attack roll it makes before the end of its next turn.
%
%     The creature knows you insulted it but doesn't noticed that you damaged it.
%
%     This damage increases by 1d4 when you reach 5th level (2d4), 11th level (3d4), and 17th level (4d4).
%
%     You cannot use this ability on the same creature more than once before you take a short rest.
%     \paragraph{Requirements} Skilled proficiency in the Deception skill.
\subsubsection{Taunt} \label{feat::taunt}
    As an action, you make a Charisma (Deception) or Charisma (Persuasion) check contested by a creature's Wisdom (Insight) check.
    The creature must be able to hear and understand you.

    If you succeed on the check, the creature has disadvantage on attack rolls against targets other than you and can't make opportunity attacks against targets other than you.
    This effect lasts for 1 minute, until one of your companions attacks the target or affects it with a spell, or until you and the target are more than 18 meters apart.
    \paragraph{Requirements} Skilled proficiency in Deception.
\subsubsection{Convincing Speech} \label{feat::convincingspeech}
    You can take this feat three times, gaining different effects each time:
    \begin{itemize}
        \item You don't gain disadvantage on Charisma (Deception) or Charisma (Persuasion) checks made in the middle of combat.
        \item Using two actions, you can attempt to deceive one person you can see within 9 meters of you who can see and hear you.
        Make a Charisma (Deception) check contested by the target's Wisdom (Insight) check.
        If your check succeeds, your movement doesn't provoke opportunity attacks from the target and your attack rolls against it have advantage; both benefits last until the end of your next turn or until you use this ability on a different target.
        If your check fails, the target can't be deceived by you in this way for one hour.
        \item If nor you or your allies attack the deceived person, it becomes convinced that you are their allies and will stop attacking you and your companions for a minute.
        The creature rolls a Wisdom saving throw with a DC equal to the number you rolled in your Charisma (Deception) check at the end of its turns.
        Its allies can help it so that it makes this roll with advantage.
        The target can't be deceived by you in this way for 1 hour after winning or failing the contest.
    \end{itemize}
    \paragraph{Requirements} Skilled proficiency in the Deception skill.
\subsubsection{Feign Innocence (2 FP)} \label{feat::feigninnocence}
    You are an expert at convincing people even at the worst of circumstances.
    When you fail a Charisma (Deception) check, you can convince your target that you didn't intend to trick them --- it was just an honest mistake!

    You can use this ability a number of times equal to your Charisma modifier, and restore all expended uses on a short rest.
    \paragraph{Requirements} Expert proficiency in the Deception skill.
% INTIMIDATION demoralize frightfulpresence inspirefear intimidating menacing
\subsubsection{Intimidating} \label{feat::intimidating}
    You increase your proficiency level in the Intimidation skill.
    This feat can be taken three times, or until you reach Expert proficiency on the skill.
\subsubsection{Menacing} \label{feat::menacing}
    Your menacing appearance has permitted you to bypass consequence in many situations, allowing you to lead a carefree lifestyle.
    People are naturally more careful around you, and you can get away with petty crimes without repercussion from the law.
    \paragraph{Requirements} Competent proficiency in the Intimidation skill.
\subsubsection{Demoralize} \label{feat::demoralize}
    You can use two actions to attempt to demoralize one creature within 9 meters of you that can see and hear you.
    Make a Charisma (Intimidation) check contested by the target's Wisdom (Insight) check.
    If you succeed, the target is frightened of you for a minute.
    In addition, it makes all attack rolls with disadvantage for that duration.
    The creature can make a Wisdom saving throw at the end of its turns with a DC equal to the number you rolled to end this effect.

    If you fail, the target can't be frightened by you in this way for one hour.
    \paragraph{Requirements} Skilled proficiency in the Intimidation skill.
\subsubsection{Inspire Fear} \label{feat::inspirefear}
    Your mear sight inspires fear upon others.
    As a reaction, you can glance at a creature right before it makes a melee attack against you.
    The creature rolls the attack with disadvantage.

    You can use this ability a number of times equal to your Charisma modifier (minimum of one).

    You can take this feat three times.
    The second time, you can use this ability on any creature within 9 meters of you that is targetting you or an ally.
    The third, the affected creature rolls all saving throws with disadvantage until the end of its next turn.
    \paragraph{Requirements} Skilled proficiency in the Intimidation skill.
\subsubsection{Frightful Presence (2 FP)} \label{feat::frightfulpresence}
    When you make a critical hit, your target becomes frightened of you until the end of your next turn.
    In addition, any hostile creature within 9 meters that can see you make this attack must roll a DC 12 Wisdom saving throw.
    On a failure, they become frightened of you until the end of your next turn too.
    \paragraph{Requirements} Expert proficiency in the Intimidation skill.
% PERFORMANCE actor enthrallingperformance flourishedduelist performer soothingpresence
\subsubsection{Performer} \label{feat::performer}
    You increase your proficiency level in the Performance skill.
    This feat can be taken three times, or until you reach Expert proficiency on the skill.
\subsubsection{Soothing Presence} \label{feat::soothingpresence}
    A favorite of the masses, you are an expert at making people laugh and relax.
    During a long rest, you and your allies reduce their stress by one additional level.
    \paragraph{Requirements} Competent proficiency in the Performance skill.
\subsubsection{Flourished Duelist} \label{feat::flourishedduelist}
    You learn the Distracting Strike action (see page \pageref{act::distractingstrike}).
    \paragraph{Requirements} Skilled proficiency in the Performance skill.
\subsubsection{Actor} \label{feat::actor}
    A natural performer, you've never failed to impress with your refined acting skills.
    % You have an easy time imitating the demeanor of others, especially of those that you know well.
    You can take this feat three times, gaining different effects each time:
    \begin{itemize}
        \item You have advantage on ability checks trying to pass off as another member of your kin, even without a disguise.
        \item You have advantage on Charisma (Deception) and Charisma (Performance) checks when trying to pass yourself off as a different person of any kin.
        If you are wearing a good quality outfit, you can pass these checks automatically.
        \item You can mimic the speech of another person or the sounds made by other creatures.
        You must have heard the person speaking, or heard the creature make the sound, for at least 1 minute.
        A successful Wisdom (Insight) check contested by your Charisma (Deception) or Charisma (Performance) check allows a listener to determine that the effect is faked.
    \end{itemize}
    \paragraph{Requirements} Skilled proficiency in the Performance skill.
\subsubsection{Enthralling Performance (2 FP)} \label{feat::enthrallingperformance}
    While performing, you can try to distract anyone who can see and hear you.
    Make a Charisma (Performance) check contested by the each person's Wisdom (Insight) check.
    If you check succeeds, you grab the person's attention enough that it automatically fails on Wisdom (Perception) and Intelligence (Investigation).
    You can tell which people have failed and succeeded on the contest.

    Any person enthralled by your performance will remember it for a lifetime.
    They will remember you and can offer you favors as long as they are reasonable.
    \paragraph{Requirements} Expert proficiency in the Performance skill.
% PERSUASION eloquent haggler persuasive reliever therapist
\subsubsection{Persuasive} \label{feat::persuasive}
    You increase your proficiency level in the Persuasion skill.
    This feat can be taken three times, or until you reach Expert proficiency on the skill.
\subsubsection{Reliever} \label{feat::reliever}
    You are able to motivate and relieve the most stressed of folks.
    As part of a short rest, you can reduce an ally's stress by one point.
    You can only use this feat once per short rest, and cannot reduce your own stress via this manner.
    \paragraph{Requirements} Competent proficiency in the Persuasion skill.
\subsubsection{Haggler} \label{feat::haggler}
    Your well-honed haggling skills give you advantage on Charisma (Persuasion) checks made when trading with a creature.
    In addition, you can always reduce the price of an item by a quarter of its price, even if you fail a Charisma (Persuasion) check.
    % NOTE: PRICE REDUCTIONS ARE NOT CUMULATIVE!
    \paragraph{Requirements} Skilled proficiency in the Persuasion.
\subsubsection{Eloquent} \label{feat::eloquent}
    A skilled negotiator and a master of diplomacy, you have an easy time convincing people to do what you need.
    You can take this feat three times, gaining different benefits each time:
    \begin{itemize}
        \item You gain advantage on Charisma (Persuasion) ability checks and contests made against creatures that do not see you as a threat or an enemy in any way.
        \item If you spend one minute talking to someone who can understand what you say, you can make a Charisma (Persuasion) check contested by the creature's Wisdom (Insight).
        If you or your companions are fighting the creature, the check automatically fails.
        If you succeed, the target is charmed by you as long as it remains within 18 meters of you and for 5 minutes thereafter.
        If you or your companions attack the creature, the charm ends.
        \item Your otherwordly charisma is hard to forget.
        Any creature that you have charmed in the past remembers you and can offer you favors as long as they're reasonable.
    \end{itemize}
    \paragraph{Requirements} Skilled proficiency in the Persuasion skill.
\subsubsection{Therapist (2 FP)} \label{feat::therapist}
    A keen healer and convincing persuader, you can heal a creature's insanity as part of a long rest.
    The creature must succeed on a DC 12 Intelligence saving throw for the treatment to work.

    Upon losing an insanity, the creature becomes stalwart, giving them advantage on attack rolls and saving throws until they take a short rest.
    \paragraph{Requirements} Expert proficiency in the Persuasion skill.

% ============================================================================== %
\subsubsection{Linguist} \label{feat::linguist}
\paragraph{RANK 2} You have advantage on Charisma (Languages) checks to determine the language that a creature is speaking.
\paragraph{RANK 2} If you see a creature's mouth while it is speaking a language you understand, you can interpret what it's saying by reading its lips.

\subsubsection{Educated} \label{feat::educated}
\paragraph{RANK 3} You have an intuitive understanding of how to use all sorts of machinery, and can operate siege weapons and heavy machines without performing any ability checks.

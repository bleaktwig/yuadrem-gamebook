% !TEX root = ../main.tex
\section{Fungal Kin} \label{kin::tsanek}
\DndDropCapLine{I}{ done seen some things down there.}
\textit{There be cities grander than any of gat's make, holdin' creatures stranger than the harrowing immensity isself.
There be ungodly abominations that weren't never meant to see the light o' day.
And there be... there be mushrooms! An entire city of mushrooms!}

\hspace*{\fill} --- Blim, the Na'anian chronicles.

% TODO: MOVE TO TSANEK SECTION!

% Apart from its chief, every unit has a designated tsanek shaman.
% This tsanek is mainly in charge of communicating with the other tsaneks in faraway places, aiding in the coordination of the tribe as a whole.
% Apart from this and other ceremonial tasks, the shaman acts as a normal member of the unit.

% The highest ranking members of their society are the sovereigns and elder sovereigns, who are tsaneks that reached their final stage of development.
% The former are huge mobile tsaneks that take root in strategic positions in Drejeck to establish their complex communication network.
% The latter are the eldest in the tribe, and merge with Tekatsae itself.
% They directly speak to the tree, communicating its wishes to the sovereigns and shamans via their root network.

Also known as tsaneks in the naenk tongue, the fungal kin is a species of intelligent fungi creature that inhabit swamps, forests, and caverns.
They are commonly seen in the jungle of Drejeck, as members of the tribes near the tekatsae tree.
Like the naenks, tsaneks grow from the tree itself, starting out as small russet-colored fungi in the tree's base and exposed roots, until they're able to grow legs and emerge.
Unlike the naenks, the fungal kin are capable of reproducing by themselves, and it's very common to find independent tsanek communities in the darker reaches of Yuadrem.

Tsaneks generally deplore violence, and only attack when provoked.
If approached peacefully, they gladly provide shelter or passage through their colonies.

\subsection*{Tribal Life}
    Most tsaneks belong to the tribes of Drejeck, filling the roles of shamans and diplomats that the naenks are less likely to fulfill due to their violent nature.
    They are considered above their mossy companions in their social circles, and are generally treated with respect among them.

    When a tsanek reaches 100 years, it is put through the rite of growth.
    The tsanek must ceremoniosly consume a mixture of the sap of tekatsae, wyvernroot, and water of the boiling river. % Wyvernroot is a strong poisonous plant native to Drejeck.
    Next, it must enter a chamber of awareness, which are small caverns below tekatsae.
    The tsanek is only left out after a month in isolation.
    Most of the tsaneks that go through this ritual die, and are consumed by tekatsae, bringing them back to the tree.
    The ones that don't become the highest ranking members of their tribal societies: Sovereigns.
    Tsanek sovereigns are large, malformed creatures that reign over the tribes.
    They are the only creatures capable of directly speaking with tekatsae, and thus are the only that can communicate its wishes to the tribes.

\subsection*{Circles and Melds}
    Many tsaneks, feeling unprepared, leave the tribes before this ritual.
    Usually many more of their species follow them to start independent communities as exiles.
    Over a timeframe of 300 to 400 years, the eldest from these colonies naturally grow to become sovereigns themselves, presiding over many social groups called circles.
    A circle consists of twenty or more fungal kin that work, live, and meld together.

    Melding is prohibited in the Drejeck tribal communities, but is a regular practice in these circles.
    A meld is a form of communal meditation that allows tsaneks to transcend their sometimes dull existence.
    Their rapport spores bind the participants into a group consciousness, inducing a shared dream that provides entertainment and social interaction.
    Tsaneks use melding in the pursuit of higher consciousness, collective union, and spiritual apotheosis.
    They can also use their rapport spores to communicate telepathically with other sentient creatures.

\subsection*{Tsanek Reproduction}
    Like other fungi, tsanek reproduce by mundane sporing.
    They are the only race that can retain their sentience without qualar, but if their spores grow without the influence of tekatsae or a particularly old sovereign, the sprouting quickly becomes feral, unable to retain sentience.
    Due to this, tsaneks carefully control their spores' release.

    Tsaneks are known to feel very little connection to their offspring.
    Among the tribal tsaneks, the children of their species are taken care of by a few tsanek designated as the spore-caretakers.
    Among the exiles, child rearing becomes a responsbility shared by the entire circle.
    It is rare if young tsaneks can even identify their parents.

\subsection*{Call to Adventure}
    While many tsanek are needed in the tribes near tekatsae for managerial tasks, it is not unusual for some to travel the globe to learn.
    Most focus their study on the qualar and cter'rheth, bringing this knowledge back to their tribes.

    Cavern tsaneks regularly travel the many caves below Yuadrem, and some even settle outside of their colonies, most usually in dark gat cities.
    Some have founded libraries, laboratories, and monasteries, usually along oths, and dedicate their lives to research and education.

\subsection*{Tsanek Names}
    Due to the fact that tsaneks have no verbal language, their names are most appropriately translated as physical descriptions of a particular individual.

    \paragraph{Names}
    Bolete, Brownback, Buttonhead, Greenfoot, Morel, Mossy, Portabelt, Puffball, Redstem, Soft-Step, Stinkhorn, Toad.

\begin{figure}[!t]
    \centering
    \includegraphics[width=0.48\textwidth]{04kins/img/16tsanek_individual.jpg}
\end{figure}

\subsection*{Traits}
    Your tsanek character has a diverse set of skills based on its nature and role on society.

    \subparagraph{Ability Score Increase} Your Wisdom score increases by 1, and your Constitution score increases by 1.

    \subparagraph{Size} A tsanek grows to a wide variety of heights and builds, with the most common being stocky and measuring about 1.9 meters in height, weighing around 65 kg.
    Your size is medium.

    \subparagraph{Speed} Your base walking speed is 6 meters.

    \subparagraph{Age} Individual tsanek are not known to die of old age, and the most elder can live to become a sovereign of one or more circles, living indefinitely longer.
    Sproutings take a long time to fully mature, but it's a continuous process and even the oldest tsaneks seem to continue growing, albeit slowly.

    \subparagraph{Alignment} Most often, a tsanek believes strongly in society and law.
    It is extremely uncommon for a tsanek to directly attack any creature that does not mean it, or its circle, harm.
    Most fungal kin groups and circles dedicate their lives to knowledge, and have a tendency towards the blue tide.

    \subparagraph{Nonverbal Magic} Though you have no conventional language, you can ignore the verbal component of spells.

    \subparagraph{Rapport Spores} All creatures within 3 meters of you with an Intelligence score of 6 or higher that aren't undead, constructs, or elementals can communicate telepathically with you and with each other if you speak at least one language in common.
    You can suppress this ability at will.
    Creatures affected by the spores realize the effect immediately, but those outside of range cannot notice them.
    Affected creatures can communicate telepathically with one another while they remain within 6 meters of each other.

    \subparagraph{Pacifying Spores} \label{kin::tsanek.pacifyingspores}
    As two actions, you can eject spores at one creature you can see within 1 meters of you.
    The target must succeed on a Constitution saving throw or be stunned for 1 minute.
    The spell save DC for this effect is 8 + your Constitution modifier.
    Undead, constructs, and elementals automatically succeed on this save.
    The target can repeat the saving throw at the end of each of its turns, ending the effect on itself on a success.
    After using this trait, you cannot do so again until you finish a short rest.

    \subparagraph{Hallucination Spores} As an action, you can produce spores that affect all creatures within 6 meters of you that aren't undead, constructs, or elementals.
    These creatures are all affected as per the cantrip minor illusion while you concentrate on the effect, which you can do for up to 1 minute.
    The spell save DC is 8 + your Constitution modifier.

    \subparagraph{Languages} You can understand, read and write knaenese and one other language of your choice, but you cannot speak.

\begin{table*}[b]%
    \begin{DndTable}[width=\linewidth]{X}
        \includegraphics[width=0.98\textwidth]{04kins/img/16tsanek_sovereign.jpg}
    \end{DndTable}
\end{table*}

\subsubsection{Gannagian Tsanek}
    Fungal kin that are members of the tribes in Drejeck which surround tekatsae.
    Like their mold kin brothers, they have a very strong sense of community and devotion to their groups, the sovereigns and the tekatsae tree.
    They are the spiritual leaders of the different groups, and focus on coordinating the tribes and communicating with the sovereigns.

    \subparagraph{Ability Score Increase} You focused much of life in study, and your Wisdom score increases by 1.

    \subparagraph{Drug-enhanced Spores} Your Rapport Spores' range is increased to 6 meters.
    The spell save DC of all your spores is increased by 2.

    \subparagraph{Euphoria Spores} Accustomed to fighting with the naenk warriors, you can release a specialized cloud of spores in a 6-meter-radius sphere centered on yourself.
    Other creatures in the area must make a Constitution saving throw of a DC equal to 8 + your Constitution modifier or become poisoned for 1 minute.
    A creature can repeat this saving throw at the end of each of its turns, ending the effect early on itself on a success.
    When the effect ends, the creature gains one level of exhaustion.
    You can produce these spores a number of times equal to your Constitution modifier (minimum of 1) per long rest.

\subsubsection{Na'anian Tsanek}
    Sometimes, a tsanek will decide to abandon the tribe and break its link with the tekatsae tree.
    These tsanek, unable to forgo their community lifestyles, tend to join or form fungus kin communities in the caverns of the world, sometimes spawning huge underground fungus cities.

    \subparagraph{Ability Score Increase} Your meandering in hostile environments has granted you increased resilience, and your Constitution score is increased by 1.

    \subparagraph{Sun Sickness} You become poisoned if you spend more than 1 minute in direct, unobstructed sunlight.
    This conditions ends when you spend 1 minute in dim or dark conditions.

    \subparagraph{Superior Darkvision} Accustomed to the darkness of the deepest of caverns, you have superior darkvision in dark and dim conditions.
    You can see in dim light within 24 meters of you as if it were bright light, and in darkness as if it were dim light.

    \subparagraph{Meld} \label{kin::tsanek.meld}
    When you take a short rest in the presence of one or more other tsaneks, you can meld with them.
    After melding, you and all melding tsaneks regain all expended Hit Dice and gain the following benefits:
    \begin{itemize}
        \item You have advantage on a saving throw you make in the next 24 hours.
        \item You can end one disease or condition affecting you, be it blinded, deafened, paralyzed, or poisoned.
    \end{itemize}

    \subparagraph{Communal Intellect} Your time spent melding with the others of your kin has granted you a deeper understanding of the world and yourself.
    You are competent in the Insight and Religion skills, and you have advantage on Wisdom (survival) checks made to find your way in caverns.

\newpage

\subsection*{Tsanek Feats}
    \begin{DndTable}[width=\linewidth, header=Tsanek Feats]{ll}
        Tsanek           & \textbf{Enhanced Spores} (page \pageref{feat::enhancedspores})             \\
        Tsanek           & \textbf{Fungal Abode} (page \pageref{feat::fungalabode})                   \\
        Tsanek           & \textbf{Fungal Body} (page \pageref{feat::fungalbody})                     \\
        Tsanek           & \textbf{Mycelium Connection} (page \pageref{feat::myceliumconnection})     \\
        Tsanek           & \textbf{Nature's Sanctuary} (page \pageref{feat::naturessanctuary})        \\
        Tsanek           & \textbf{Symbiotic Entity} (page \pageref{feat::symbioticentity})           \\
        Tsanek           & \textbf{Strong Telepathy} (page \pageref{feat::strongtelepathy})           \\
        Gannagian Tsanek & \textbf{Balm of Dreams} (page \pageref{feat::balmofdreams})                \\
        Gannagian Tsanek & \textbf{Fungal Infestation} (page \pageref{feat::fungalinfestation})       \\
        Gannagian Tsanek & \textbf{Narcotic Empowerement} (page \pageref{feat::narcoticempowerement}) \\
        Na'anian Tsanek  & \textbf{Benign Growths} (page \pageref{feat::benigngrowths})               \\
        Na'anian Tsanek  & \textbf{Halo of Spores} (page \pageref{feat::haloofspores})                \\
        Na'anian Tsanek  & \textbf{Observant} (page \pageref{feat::observant})
    \end{DndTable}

    \subsubsection{Balm of Dreams} \label{feat::balmofdreams}
        You learn how to excrete a healing balm from your fungal growths.
        You have a pool of balm represented by a number of d4s equal to your level.

        As an action, you can choose one creature you can see within 1 meters of you and spend a number of those dice equal to half your level or less.
        Roll the spent dice and add them together.
        The target regains a number of hit points equal to the total.
        The target also gains 1 temporary hit point per die spent.

        This balm is uneffective on tsaneks.
        You regain all expended dice when you finish a short rest.
        \paragraph{Requirements} Gannagian Tsanek subrace.
    \subsubsection{Benign Growths (2 FP)} \label{feat::benigngrowths}
        As part of a short rest, you can grow a fungal cornucopia on your back.
        The growths correspond to 3 doses, each of which can be consumed by expending 2 actions.
        Each growth has a random special effect, which is decided randomly by rolling a d6 during the short rest taken:
        \begin{DndTable}[width=\linewidth, header=Benign Growths]{cX}
            \textbf{d6} & \textbf{Effect} \\
            1  & \textbf{Healing}. The creature's regains a number of hit points equal to 2d4 + your Constitution modifier. \\
            2  & \textbf{Swiftness}. The creature's walking speed increases by 2 meters for 10 minutes. \\
            3  & \textbf{Resilience}. The creature gains a +1 bonus to AC for 10 minutes. \\
            4  & \textbf{Boldness}. The creature can roll a d4 and add the number rolled to every attack roll and saving throw they make for the next minute. \\
            5  & \textbf{Sporing} The creature gains the \textbf{Pacifying Spores} trait (see page \pageref{kin::tsanek.pacifyingspores}) for 10 minutes. \\
            5  & \textbf{Melding}. The creature can meld with you on your next short rest.
            You both gain the effect associated to melding (see page \pageref{kin::tsanek.meld}).
        \end{DndTable}
        Apart from their effect, each growth comfortably feeds a creature for a day.
        \paragraph{Requirements} Na'anian Tsanek subrace.
    \subsubsection{Enhanced Spores} \label{feat::enhancedspores}
        You can add a +2 bonus to all of your spore effects' spell save DC.

        You can take this feat three times.
        \paragraph{Requirements} Tsanek kin.
    \subsubsection{Fungal Body} \label{feat::fungalbody}
        Your normal sight and hearing are extended by the spores around you.
        You gain truesight to a range of 2 meters, and become immune to the blinded and deafened conditions.
        \paragraph{Requirements} Tsanek kin.
    \subsubsection{Fungal Infestation (2 FP)} \label{feat::fungalinfestation}
        Your spores gain the ability to infest a corpse and animate it.
        If a beast or a humanoid that is Small or Medium dies within 2 meters of you, you can use your reaction to animate it, causing it to stand up immediately with 1 hit point.
        The creature uses the zombie stat block in the Monster Manual.
        It remains animate for 1 hour, after which time it collapses and dies.

        In combat, the zombie's turn comes immediately after yours.
        It obeys your mental commands, and the only action it can take is the Attack action, making one melee attack.

        You can use this feature a number of times equal to your Wisdom modifier (minimum of once), and you regain all expended uses of it when you finish a short rest.
        \paragraph{Requirements} Gannagian Tsanek subrace.
    \subsubsection{Mycelium Connection} \label{feat::myceliumconnection}
        As an action, you can establish a connection with a creature within 6 meters of you.
        Roll a Wisdom (Insight) check contested by the creature's Wisdom (Insight).
        The creature can choose to fail on this save on purpose.
        If you succeed, you know the location of the creature and can telepathically communicate with it for 24 hours.
        The creature knows that it is connected to you, as it feels your presence pulsating in its head, but it cannot end this connection.

        This ability only works while the creature is touching the ground.
        \paragraph{Requirements} Tsanek kin.
    \subsubsection{Strong Telepathy} \label{feat::strongtelepathy}
        You can speak telepathically to any creature you can see within 12 meters of you.
        Your telepathic utterances are in a language you know, and the creature understands you only if it knows that language.
        Your communication doesn't give the creature the ability to respond to you telepathically.

        Additionally, you can cast the detect thoughts (see page \pageref{spell::detectthoughts}) spell, requiring no spell slot or components, and you must finish a short rest before you can cast it this way again.
        Your spellcasting ability for the spell is your choice between Intelligence and Wisdom.
        \paragraph{Requirements} Tsanek or Zaloth kin.
    \subsubsection{Narcotic Empowerement} \label{feat::narcoticempowerement}
        From your training as a shaman, you are able to release special chemicals as part of your meditation.
        With an hour of rest, you can recover an amount of spell points equal to your level.
        You can't use this feature again until you finish a short rest.
        \paragraph{Requirements} Gannagian Tsanek subrace.

\newpage~\newpage

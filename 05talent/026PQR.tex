% !TEX root = ../main.tex
\subsubsection{Perceptive} \label{tal::perceptive}
\small{\textcolor{gray}{Perception}}

\normalsize
When you focus your mind on something, you are able to quickly perceive even the smallest details and imperfections.
\paragraph{REQUIREMENTS} Observant 3.
\paragraph{RANK 1} You double your proficiency modifier in the Perception skill.
\paragraph{RANK 2} You have a +5 bonus to your passive Wisdom (Perception) score.
\paragraph{RANK 3} Increase your Wisdom score by 1, to a maximum of 20.

% ============================================================================== %
\subsubsection{Performer} \label{tal::performer}
\small{\textcolor{gray}{Performance}}

\normalsize
A natural performer, you've never failed to impress with your refined acting skills.
You have an easy time imitating the demeanor of others, especially of those that you know well.
\paragraph{RANK 1} You are proficient with the Performance skill.
\paragraph{RANK 2} You have advantage on ability checks trying to pass off as another member of your kin, even without a disguise.
\paragraph{RANK 3} You learn the Distract technique.

% ============================================================================== %
\subsubsection{Persuasive} \label{tal::persuasive}
\small{\textcolor{gray}{Persuasion}}

\normalsize
A skilled negotiator and a master of diplomacy, you have an easy time convincing people to do what you need.
\paragraph{RANK 1} You are proficient with the Persuasion skill.
\paragraph{RANK 2} You have advantage on Charisma (Persuasion) checks when trading with a creature.
\paragraph{RANK 3} You learn the Charm technique.

% ============================================================================== %
\subsubsection{Pick Adept} \label{tal::pickadept}
\small{\textcolor{gray}{Strength or Dexterity}}

\normalsize
For you, there is no difference between mining ore and piercing flesh.
Apart from the screams at least.
% You pierce through metal, flesh, and bone just as you would ore.
\paragraph{REQUIREMENTS} Armed Fighter 2.
\paragraph{RANK 1} You are proficient with picks.
\paragraph{RANK 2} All picks have the versatile property for you.
When you wield a pick with two hands, increase its damage die by one (d6 to d8, d8 to d10, etc), and have the heavy property.
\paragraph{RANK 3} You learn the Trip technique.

% ============================================================================== %
\subsubsection{Pickpocket} \label{tal::pickpocket}
\small{\textcolor{gray}{Sleight of Hand}}

\normalsize
You have taken the time to hone your larcenous skills, making you an accomplished thief.
\paragraph{REQUIREMENTS} Quick Fingers 3.
\paragraph{RANK 1} You double your proficiency modifier in the Sleight of Hand skill.
\paragraph{RANK 2} Other creatures have disadvantage on Intelligence (Investigation) or Wisdom (Perception) checks to notice an object you palmed in your hands.
\paragraph{RANK 3} If you spend at least 1 minute observing or interacting with another creature outside combat, you can learn about anything the creature has that can be stolen.

% ============================================================================== %
\subsubsection{Piercer} \label{tal::piercer}
\small{\textcolor{gray}{Strength or Dexterity}}

\normalsize
You have achieved a penetrating precision in combat.
\paragraph{REQUIREMENTS} Proficiency with at least two martial piercing weapon types.
\paragraph{RANK 1} Once per turn, when you hit a creature with an attack that deals piercing damage, you can reroll one of the attack's damage dice, and you must use the new roll.
\paragraph{RANK 2} Increase your Strength or Dexterity by 1, to a maximum of 20.
\paragraph{RANK 3} You learn the Armor Puncture technique.

% ============================================================================== %
\subsubsection{Pike Master} \label{tal::pikemaster}
\small{\textcolor{gray}{Strength}}

\normalsize
You prefer to maintain distance from your opponents by wielding an amusingly long pole.
\paragraph{REQUIREMENTS} Spear Adept 2.
\paragraph{RANK 1} When you use a pike, its damage die changes from a d10 to a d12.
\paragraph{RANK 2} When you use the Feint technique you can choose to not attack with advantage.
If you do this and still hit your target, it's a critical hit.
\paragraph{RANK 3} When wielding a pike, its reach property adds 3 meters instead of 1.5 meters to the weapon's range.

% ============================================================================== %
\subsubsection{Pious} \label{tal::pious}
\small{\textcolor{gray}{Religion}}

\normalsize
You're a zealot when it comes to your faith.
Your devotion has led to an incredible understanding of your religion and its history, and no connoted figure goes unnoticed in your prayers.
\paragraph{RANK 1} You are proficient with the Religion skill.
\paragraph{RANK 2} You can recall the name and description of any deity and important figure related to a religion of your choice without needing to succeed on any ability check.
\paragraph{RANK 3} You have advantage in Intelligence (Religion) checks made to tell the tide of a creature you can see.

% ============================================================================== %
\subsubsection{Pistol Adept} \label{tal::pistoladept}
\small{\textcolor{gray}{Dexterity}}

\normalsize
Already a master of the crossbow, it isn't hard for you to adapt your use of the handcrossbow to that of the pistol.
\paragraph{REQUIREMENTS} Crossbow Adept 2.
\paragraph{RANK 1} You are proficient with pistols.
\paragraph{RANK 2} You can stow a firearm, then draw another weapon as a single object interaction on your turn.
\paragraph{RANK 3} When you get a critical hit against a creature on a ranged attack with a firearm, all other attacks against the creature are made with advantage until the start of your next turn.

% ============================================================================== %
\subsubsection{Polearm Master} \label{tal::polerarmmaster} %
\small{\textcolor{gray}{Dexterity}}

\normalsize
You keep your enemies at bay with your long weapons.
\paragraph{REQUIREMENTS} Staff Fighter 2, Spear Adept 2, and Halberd Adept 2.
\paragraph{RANK 1} You gain a +1 bonus to attack rolls you make with staves, spears, and halberds.
\paragraph{RANK 2} While wielding a stave, spear, or halberd, other creatures provoke an opportunity attack from you when they enter your reach.
\paragraph{RANK 3} You learn the Rondel Hit technique.

% ============================================================================== %
\subsubsection{Precise Thrower} \label{tal::precisethrower}
\small{\textcolor{gray}{Dexterity}}

\normalsize
Either from luck or rigorous training, you have the capacity to hit the hardest of targets when throwing a weapon.
\paragraph{REQUIREMENTS} Dagger Savant 2 and Thrown Weapon Master 2.
\paragraph{RANK 1} Attacking at long range doesn't impose disadvantage on your thrown weapon attack rolls.
\paragraph{RANK 2} Whenever you hit a creature with an thrown slashing or piercing weapon, it has disadvantage on the first melee attack it performs on its next turn.
\paragraph{RANK 3} You learn the Bull's Eye technique.

% ============================================================================== %
\subsubsection{Pugilist} \label{tal::pugilist}
\small{\textcolor{gray}{Strength}}

\normalsize
Your fighting style has only three components:
A fist, another fist, and a face to punch.
\paragraph{RANK 1} Your unarmed strike uses a d4 + your Strength modifier for damage.
\paragraph{RANK 2} When you hit a creature with an unarmed strike, you can use a bonus action to attempt to grapple the target.
\paragraph{RANK 3} You learn the Bonk technique.

% ============================================================================== %
\subsubsection{Quick Fingers} \label{tal::quickfingers}
\small{\textcolor{gray}{Sleight of Hand}}

\normalsize
While they've given you problems in the past, your quick and sticky fingers have awarded you with many a treasure thorough your life.
\paragraph{RANK 1} You are proficient with the Sleight of Hand skill.
\paragraph{RANK 2} You can use a bonus action to take the Search or the Use an Object actions.
\paragraph{RANK 3} You learn the Steal technique.

% ============================================================================== %
\subsubsection{Revenant Blade} \label{tal::revenantblade}
\small{\textcolor{gray}{Dexterity}}

\normalsize
You are a expert of the double-bladed scimitar, and use the technique as proficiently as any master from the ancient Janchu'ut school of hunters.
\paragraph{REQUIREMENTS} Curved Blade Adept 2.
\paragraph{RANK 1} A double-bladed scimitar has the finesse property when you wield it.
\paragraph{RANK 2} While you are holding a double-bladed scimitar with two hands, you gain a +1 bonus to Armor Class.
\paragraph{RANK 3} Instead of dealing 1d4 slashing damage, the extra attack you can do with a double-bladed scimitar deals 2d4 slashing damage.

% ============================================================================== %
\subsubsection{Robust} \label{tal::robust}
\small \textcolor{gray}{Athletics}

\normalsize
Your physical prowess is a marvel to all.
\paragraph{REQUIREMENTS} Athlete 3.
\paragraph{RANK 1} You double your proficiency modifier in the Athletics skill.
\paragraph{RANK 2} Your walking speed increases by 1.5 meters.
Additionally, you gain a climbing speed and a swimming speed equal to your walking speed.
\paragraph{RANK 3} Your hit point maximum increases by an amount equal to your number of hit dice.
Whenever you earn a hit die thereafter, your hit point maximum increases by an additional 1 hit point.

% ============================================================================== %

% !TEX root = ../main.tex
\section{Experience Points}
With one heavy loss, the group manages to dispatch the fearsome troll.
% They perform the correct ritual, to then finally get some rest.
Beaten and tired, they learned their lesson for the day.

What is learned by your character during the game is measured by Experience Points (XP).
Afteryou finish playing, talk about the session with the whole group and discuss what happened.
For each of the questions below that you can reply ``yes'' to, your character gets one XP:

\begin{itemize}
    \item Did you participate in the session?
    \item Did you travel to a new location?
    \item Did you defeat one or more creatures?
    \item Did you loot treasure?
    \item Did you complete a quest?
    \item Did you solve a conflict?
    \item Did you make a new friend, ally, or enemy?
    \item Did you help another PC?
    \item Did your ideals, bonds, or flaws affect any of your decisions during the session?
    \item Did you perform any action related to your kin, nation, or background?
    \item Did you perform an extraordinary action of some kind?
\end{itemize}

In case of any discussion or disagreement, the DM always has the last word.
Additional XP may be awarded when specific story milestones are achieved, but this is always left to the discression of the DM.

\subsection*{Talent Points}
When your character has gained 20 XP you gain one TP and reset your XP counter, keeping the remaining XP.
Talent points are used to increase your rank in a feat.
% Talent points are used to buy new talents, or to upgrade the ones your character already has.

Ranks in a feat give a character either a proficiency or a feature.
Each feat has three ranks, and some act as requirements for others.

% You don't need to spend TP right after earning it, and you can save as many as you may want for as long as you may need.

% The rules to learn a new feat or increase your rank in one depend on what is gained through the feat.

If the rank of the feat involves a proficiency, you need to practice it in five different 4-hour training sessions with a teacher who already knows the feat.
If you don't have a teacher, you must roll an ability check relevant to the feat after each session to see if it was fruitful.

If the rank involves a new feature, you need to practice it in one 4-hour training session at your convenience.
You must roll an ability check or saving throw after the session to see if it was fruitful.
This task cannot be aided by a teacher.

Attaining first, second, and third ranks require a DC of 12, 15, and 18 respectively.
A TP is only spent if a rank is attained successfully.

% TODO: MENTION THAT ALL TALENTS REQUIRE A BOOK OR SOURCE OF KNOWLEDGE WHEN YOU DON'T HAVE A TEACHER

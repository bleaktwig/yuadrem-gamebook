% !TEX root = ../main.tex
\subsection*{Heliod, God of the Sun} \label{ssec::heliod}
    \subparagraph{Domains} Light.

    Heliod is the radiant god of the sun.
    According to myth, they ensure that the sun rises every day to provide light and warmth to the world.
    Every inhabitant of Viphoger acknowledges their dominant presence, and nearly everyone at least pays lip service to the idea of giving them worship and honor.

    Pride and self-assurance radiate from Heliod as light floods from the sun.
    They are cheerful and sociable, enjoying the company of others and forming bonds easily.
    Their friendship can be as easily lost, though, turning them from ally to enemy as the consequence of a single misstep or perceived betrayal.

    Heliod has appeared to mortals in a variety of forms, but they prefers the appearance of a sun-bronzed gat in their forties, dressed in a flowing tunic of golden cloth.
    Their profile is noble, highlighted by a strong chin and a short beard, and they boast the broad chest of a perfectly fit athlete.
    Their hair is glossy black, and their head is crowned with a golden wreath.
    They are also fond of appearing as a brilliant white horse or a radiant golden stag.
    In any guise, they look lit by the sun, even when they travel across the night sky.

    \subsubsection{Worshiping Heliod}
        The brilliance of Heliod's sun is impossible to ignore.
        Thus, virtually everyone on Viphoger pays at least grudging respect to the sun god in forms of worship that range from simple gestures to days-long celebrations.

        Some families, particularly in the polis of Mephetis, follow a practice of bowing in the direction of dawn's first light --- or winking, in a gesture of respect for the sun god's luminous ``eye''.
        More dedicated worshipers offer short litanies at dawn, noon, and dusk, acknowledging the sun's passage across the sky.

\subsection*{Iroas, God of Victory} \label{ssec::iroas}
    \subparagraph{Domains} War.

    Iroas is the steadfast god of honor and victory in war.
    When soldiers march to battle, their voice is the thunder of their footsteps and the crash of spear on shield.
    Soldiers, mercenaries, and athletes all pray for Iroas's favor in securing victory.
    Common folk pray to Iroas for courage and fortitude in times of struggle, for their is the battle nobly fought and won.

    Bold and confident with a soldier's demeanor, Iroas is the pinnacle of martial pride and bearing.
    They are stoic almost to a fault, but also exhibits a wry sense of humor.
    Those who honorably shed blood in Iroas's name can count on their support.
    Cowards and oath breakers are to be despised, and traitors don't deserve mercy in battle.

    Iroas most often appears as a powerfully built gat with a bull-like body, clad in gleaming armor and wielding a spear and shield.
    They speak in a booming baritone that projects power, confidence, and courage.
    They have been known to appear as a burly soldier or a mighty bull before their followers.
    Whatever form they choose, Iroas carries themself with precision and majesty at all times and doesn't tolerate disrespect or undue informality from those who would deal with them.

    \subsubsection{Worshiping Iroas}
        Iroas is interested not in pretty words, but in great deeds.
        The faithful of Iroas show their piety by comporting themselves well in contests of athleticism or skill.
        Swearing an oath to win a battle in Iroas's name and failing to do so is a great shame upon a warrior, thus such a promise is never uttered lightly.

        The fifth month of the Fagalian calendar is Skenion, named for an annual commemoration of the Mephetian conquest of Natumas.
        This victory cemented Mephetis's control over the entire peninsula.
        But in Akros, the month is called Iroagonion, for the Iroan Games.
        These games are the grandest display to honor Iroas.
        To even compete in the Iroan Games is considered noteworthy, as the poleis send only their finest athletes.
        The grand prize, besides a ceremonial wreath, is the opportunity to be visited by Iroas themself.

\subsection*{Karametra, God of Harvests} \label{ssec::karametra}
    \subparagraph{Domains} Life, Nature.

    Karametra is recognized as the serene, maternal god of the harvest, her arms spread wide as she offers bounty to her worshipers or cradles communities in her embrace.
    Almost every uman settlement contains at least a modest shrine to solicit her favor, and she is closely associated with Setesh, the center of her worship.

    Wise and even-tempered, Karametra values community, stability, and the balance of nature.
    She is the god of maternity, family, orphans, domestication, and agriculture, as well as defense of the home and territory.

    Karametra appears to mortals as a motherly figure with hair made of ordered rows of leaves that shroud her eyes from view.
    She is always shown in art (and often seen in the night sky) seated on her throne, which is formed from a tangle of grape vines growing out of a collection of jugs and amphorae that surround her.
    An elaborately carved wooden canopy extends above her, and a giant sable --- her faithful companion --- curls around the base of the throne at her feet.
    In one hand, she holds a harvester's scythe.

    \subsubsection{Worshiping Karametra}
        The earth's fertility is essential for mortal life to continue.
        Those who live in the modern poleis might not be as aware of that fact as those who farm their own food, but even they long for children, know the pinch of hunger, and feel the turn of the seasons.

        Prayers to Karametra focus on asserting Karametra's constancy and bounty, praising the god's love and generosity.
        Worshipers of Karametra gather for a feast once a month, on the evening of the full Fagal moon, that celebrates the god's role in parenthood and community.
        New parents receive gifts and blessings, and young couples sneak away into the woods in hopes of finding sweet berries and sweeter kisses.

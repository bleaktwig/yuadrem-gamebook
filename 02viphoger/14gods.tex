% !TEX root = ../main.tex
\subsection*{Mogis, God of Slaughter} \label{ssec::mogis}
    \subparagraph{Domains} War.

    Mogis is the god of slaughter, violence, and war. He is hatred unrestrained, empathy denied, and mercy forgotten, an entity whose very presence incites mortals to violence. Soldiers fear succumbing to his blood lust lest they dishonor themselves, but the vengeful and forsaken call to him for the gift of his rage. He is the brother of Iroas, god of victory, and his antithesis in matters of warfare.

    The anger and malice radiating from Mogis is almost palpable. He exercises no control over his temper or his urges and lashes out at subordinates at the slightest provocation. Akroan soldiers are warned that to give in to his seductive battle rage is to risk becoming an androphage—a bloodthirsty killer wholly consumed by Mogis's fury.

    Mogis cuts a terrifying figure, appearing as a four-horned minotaur of incredible size clad in spiked bronze armor and wielding a massive ebon greataxe. He doesn't debase himself by appearing in other guises to mortals—to behold him is to behold the cruelty of war personified. He hungers endlessly to defeat his brother Iroas in combat and thus become the sole avatar of war among mortals.

    \subsubsection{Worshiping Mogis}
        Mogis exhorts his followers to channel their hatred and rage into ever greater acts of cruelty and violence. He demands actions over words, making his followers an active and dangerous lot. From the spurned lover thirsting for revenge to the blood-drenched warrior on the battlefield, all honor Mogis with the shedding of blood in anger.

        Minotaurs are the most ardent worshipers of Mogis and regularly hold bloody rites in his honor. Warchanters, the minotaur clergy of Mogis, whip their marauders into a near-mindless frenzy before battle; the ensuing slaughter gives glory to Mogis's name.

        The appearance of the blood moon is a most holy occasion for the faithful of Mogis, since the moon represents his hateful crimson eye. At such times, his followers prepare and consume a feast of meat, either raw or barely cooked, along with copious amounts of intoxicants, followed by ritual self-mutilation—scarring themselves to demonstrate their devotion to Mogis.

\subsection*{Nylea, God of the Hunt} \label{ssec::nylea}
    \subparagraph{Domains} Nature.

    Nylea is the wild, carefree god of the hunt. She claims dominion over the whole of the natural world, particularly hunger and predation, the seasons, metamorphosis and rebirth, and the forest.

    Nylea is among the most gregarious of the gods, and can be spotted frolicking joyfully with her Nyxborn lynx, Halma, or her favorite nymph, Theophilia. But she also savors solitude, and on the hunt she is deadly serious, almost animalistic, in her mood. She is nearly as quick to anger as her brother Purphoros, enacting swift revenge on those who harm the natural realm.

    Nylea usually appears as a green-skinned dryad with woody extremities. Her hair is made of vines and leaves that change with the seasons. She might also appear as a majestic specimen of any animal, most frequently a lynx or a wolf. When she desires stealth or solitude, she might take the form of a tree, usually an oak or an olive.

    \subsubsection{Worshiping Nylea}
        Mortals all over Theros pray to Nylea when they rely on hunting or nature's whims for their livelihood. Her most ardent followers are satyrs, centaurs, humans (particularly those who live in Setessa and in the wilds), and nymphs of all kinds, especially dryads. Few leonin worship any of the gods, but of those who do, many favor Nylea with their prayers.

        Nylea blesses those who are kind to animals, considering such acts as wordless prayers. Those who must kill a dangerous natural animal or cut down trees often pray to Nylea for forgiveness, sometimes leaving food for other animals or planting new trees as atonement.

\subsection*{Pharika, God of Affliction} \label{ssec::pharika}
    \subparagraph{Domains} Death, Knowledge, Life.

    Pharika is a god of affliction and medicine, alchemy and aging. In the earliest days of Theros, Pharika seeded the world with countless secret truths—mysteries of medicine, minerals with strange properties, nexuses of magic, and the like—which she hid among Nylea's wilds and the shadows of Erebos's Underworld, leaving clues where mortals might find them. It isn't altruism that drives her; she studies the innovation and suffering of mortals, deciphering in them ever greater mysteries as she treats Theros as her personal laboratory.

    Pharika typically takes the form of a green-skinned human woman with the lower body of a snake. Her hands are thickly scaled and a pair of bronze-scaled vipers seamlessly emerge from her chest. She is never without her kylix, a drinking cup within which she can produce virtually any medicine or toxin. When her aims require subtlety, Pharika often takes the form of a serpent or a medusa, or sometimes an aged human.

    Little escapes Pharika's cool gaze. Even when outwardly friendly, she is cunning and calculating, watching for the slightest sign of weakness or desire that she can exploit later. Those who offend her rarely recognize their misstep until she strikes.

    \subsubsection{Worshiping Pharika}
        The diseased and the dying alike often make written entreaties to Pharika for a remedy. Prayers are written on scraps of paper or shards of pottery, sealed in small pots, and buried in bogs, leaving them as secrets for others to exhume years later. Many people pray to her before undergoing a medical procedure, picking herbs, or confronting a venomous animal. Nights of a waxing crescent moon (roughly the first week of each month, when a sliver of moon lingers in the early evening) are sacred to Pharika and are thought to be an auspicious time to harvest medicinal plants.

        Pharika's followers include members of several small mystery cults, which embrace varying aspects of her divine nature. The most infamous of these is the Cult of Frozen Faith, led by a medusa. Initiates receive a lethal dose of poison, become petrified, and then are restored to flesh one year later. Petitioners who have Pharika's favor emerge alive and healthy; those she doesn't care for fail to survive the transformation.

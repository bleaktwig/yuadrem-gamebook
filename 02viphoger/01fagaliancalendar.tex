% !TEX root = ../main.tex
\subsection*{The Fagalian Calendar} \label{ssec::thefagaliancalendar}
The oth astronomers and philosophers from oldentimes established a calendar that has found massive adoption in the rest of Yuadrem.
It divides the year into fifteen months of twenty-four days, each beginning with Fagal's new moon.
Every ten years, an extra day is added at the end of the calendar to keep it aligned with the solar year.

The beginning of the year is considered the end of spring, so the new year begins with the summer.
Each month is associated to a tide, and is holy to a specific god.
A major festival in honor of that god is celebrated in Mephetis.
The fifth month (Skenion in Mephetis) is called Iroagonion in Akhosh, after the Iroan Games, which are held in that month every year.

% The Fagalian Calendar table summarizes the months, their lengths, and the god each is associated with.

% !TEX root = ../main.tex
\begin{tikzpicture}[remember picture,overlay]
    \node[anchor=north, yshift=0.10cm] at (current page.north) {\includegraphics[width=\pdfpagewidth]{02viphoger/img/10athreos.png}};
\end{tikzpicture}

\vspace{14.0cm}

\section{Therism} \label{ssec::therism}

A pantheon of fifteen gods guides religious life on Viphoger.
From the sun and agriculture to death and passage into Nyx, the gods oversee the forces of nature and the most important aspects of mortal life.
These gods are quite real to the people of Viphoger, who see them moving across the sky at night and sometimes encounter them face to face.
Thus, most people perform rituals and devotions that honor various gods, hoping to win their favor and stave off their wrath.
They tell and retell the stories of the gods' deeds --- even as they watch those stories continue to play out in the vastness of the night sky.

Therism is the main religion practiced in Viphoger.
While both religions draw from the same pantheon, Tanethism and Therism couldn't have less in common.
The former was designed by a scholar and forced on their by a king, while the latter developed gradually from the gat settlers of the Sylvan Canyon.

\pagebreak~
\vspace{14.5cm}

In their new land, they saw the gods' visage in the sky, equatting them to their ancient deities.
Not every mortal serves or acknowledges the gods, though.
Some philosophers in the schools of Mephetis teach that the gods of the pantheon are subordinate to a higher reality, perhaps the sky itself.
And other people, particularly leonin, believe that the gods are undeserving of mortal reverence.

\subsubsection{Worship}
    The most prevalent form of expressing reverence is the practice of libation, pouring out a splash of wine or water in honor of the gods.
    Pious people perform a simple rite of prayer and libation every morning and evening at a household altar or hearth, while the less devoted might still pour out a splash of wine before drinking the rest.

    The defining feature of a Theran temple is a statue of a god.
    Worshipers kneel before it, touch and kiss it, drape it in garlands and fine cloth, and leave offerings before it.
    % These acts are sometimes spontaneous outpourings of love or gratitude, and sometimes petitions, imploring the god to cure an illness, send rain for crops, guarantee a safe journey, or perform any other favor related to the god's sphere of influence.

    % Most people aren't devoted to a single god, though many prefer some gods over others. Someone might ask Pharika to spare a loved one from disease, then later offer prayers to Karametra for a bountiful harvest or to Thassa for safety on a sea journey.

    % Considering their significance to the people of Viphoger, this section includes a list of the 15 gods of the Therist pantheon, including a short description of each.

\pagebreak

% TODO: Change domains to tidal alignment of each god!
% !TEX root = ../main.tex
\subsection*{Athreos, God of Passage} \label{ssec::athreos}
    \subparagraph{Domains} Death, Grave.

    All mortals are destined to face Athreos, the River Guide, when their lives come to an end.
    The god of passage ferries the dead across the Tartyx River, conveying each mortal soul to its destiny in the Underworld.
    For most people, Athreos embodies the greatest mysteries of existence --- the terror and wonder of life's last moment and the revelation of one's ultimate fate in the afterlife.
    Athreos is no judge, though.
    The veiled, silent god undergoes no deliberations and makes no exceptions.
    The River Guide reads the truth of each soul and bears it unfailingly to its proper place in the Underworld.
    There is no haggling and no sympathy on Athreos's skiff, the god having heard and denied every conceivable mortal plea.

    Athreos appears as a gaunt figure cloaked in ragged robes and a collection of golden masks.
    What little can be seen of their body is unsettling, its gray flesh stretched thin over a barely et skeleton.
    The River Guide is never without their ancient staff, Katabasis, which they transforms into the ferryboat they use to ply the Rivers That Ring the World.
    Athreos can change shape but rarely, if ever, takes on other forms.

    \begin{figure}[t]
        \centering
        \includegraphics[width=0.47\textwidth]{02viphoger/img/10s_athreos.png}
    \end{figure}

    \subsubsection{Worshiping Athreos}
        Most funeral traditions include small offerings and words of reverence to Athreos.
        Predominant among these traditions is burying or burning the dead with a clay funerary mask, to "frame" the identity of the dead for Athreos, and with at least one coin, so a soul might pay Athreos to ferry them to the Underworld.
        Some people are laid to rest with large amounts of grave goods.
        Memorial practices vary widely by culture, from tearful, somber affairs to lively celebrations.
        These rituals serve more as catharsis for the living than as meaningful boons to Athreos, though.
        The River Guide cares only for the single coin they're owed by any who board their skiff.

        During the feast of Necrologion, pious souls silently spend the day reading ancient memoirs or writing messages for their own descendants.

\subsection*{Ephara, God of the Polis} \label{ssec::ephara}
    \subparagraph{Domains} Knowledge, Light.

    As god of the polis, Ephara sees themself as the founder of civilization.
    They watch over cities, protecting them from outside threats.
    They are credited with establishing the first code of law, which Mephetis has preserved and the other poleis have imitated.
    Even more important, they help cities reach their highest potential, becoming centers of scholarship, industry, and art.

    Ephara appears as a huge animated statue wearing a stone crown, resembling the capital of a column.
    When they choose to walk about their cities at a mortal's scale, they often take on the form of a gat.
    In either form, they are always dressed in blue and white, and their expression is usually serious, but not unkind.
    They often carry a large urn on one shoulder, with the dark, star-studded sky pouring from it and dissolving into mist as it hits the ground.

    \begin{figure}[b]
        \centering
        \includegraphics[width=0.47\textwidth]{02viphoger/img/10s_ephara.png}
    \end{figure}

    \subsubsection{Worshiping Ephara}
        To an extent, Ephara's devout show their faith by going about their lives and contributing to society.
        Midday services at Ephara's temples often feature a brief prayer, followed by a longer talk from an industrial or civic leader on a topic of general interest.
        Attendants often bring meals to eat while on a break from their jobs.

        Ephara's face is a common sight in cities.
        Marble buildings, stone walls, and similar surfaces usually feature a sculpture or relief of their visage.
        People often swear oaths or engage in verbal disputes in front of these images, believing they won't let a falsehood told in front of them go unpunished.
        Whether they actually intervenes is unclear, but conflicts that play out this way are often resolved peacefully, without a need for the justice system to get involved.

\subsection*{Erebos, God of the Dead} \label{ssec::erebos}
    \subparagraph{Domains} Death, Trickery.

    \begin{figure}[h]
        \centering
        \includegraphics[width=0.47\textwidth]{02viphoger/img/10s_erebos.png}
    \end{figure}

    Erebos is the god of death and the Underworld, lord of all that has ever lived.
    They preside over the bitterness, envy, and eventual acceptance of those who suffer misfortune.
    Their hoarding of both souls and the treasures the dead carry into the Underworld see them worshiped by those who desire to collect and keep wealth.

    Erebos's very presence is stifling, and those who come face to face with them often depart in despair.
    They are jealous and tyrannical within their realm, but unlike their brother Heliod, they neither bluster nor try to expand their influence.
    They wait patiently, secure in the knowledge that everything belongs to then in the end.

    Erebos most frequently appears as a slender, gray-skinned gat with two large, outward-curving horns, wielding an impossibly long black whip.
    They also appear in the form of a black asp, a cloud of choking smoke, or an animated golden idol.

    \subsubsection{Worshiping Erebos}
        To many mortals, Erebos is primarily concerned not with death, but with gold.
        Most of their followers downplay their association with death and misfortune, instead praying to them for material wealth.
        Others pray to them because they want to be more accepting of their misfortune.
        These individuals see themselves as beyond hope of improving their lot in life, asking only that Erebos grant them the strength to endure until they enter their realm at their predestined time.

        A smaller but more dangerous group of Erebos worshipers are those who actively glorify death.
        These cultists and assassins congregate in secret in communities across Viphoger, engaging in campaigns of violence.

        The only major festival dedicated to Erebos, called the Katabasion or "the Descent," features a ceremony in which worshipers make a symbolic journey into the Underworld.
        The supplicants enter a cave, offer prayers and sacrifices to Erebos in utter darkness, and slowly make their way back to the surface just before sunrise.

% !TEX root = ../main.tex
\subsection*{Heliod, God of the Sun} \label{ssec::heliod}
    \subparagraph{Domains} Light.

    Heliod is the radiant god of the sun.
    According to myth, they ensure that the sun rises every day to provide light and warmth to the world.
    Every inhabitant of Viphoger acknowledges their dominant presence, and nearly everyone at least pays lip service to the idea of giving them worship and honor.

    Pride and self-assurance radiate from Heliod as light floods from the sun.
    They are cheerful and sociable, enjoying the company of others and forming bonds easily.
    Their friendship can be as easily lost, though, turning them from ally to enemy as the consequence of a single misstep or perceived betrayal.

    Heliod has appeared to mortals in a variety of forms, but they prefers the appearance of a sun-bronzed gat in their forties, dressed in a flowing tunic of golden cloth.
    Their profile is noble, highlighted by a strong chin and a short beard, and they boast the broad chest of a perfectly fit athlete.
    Their hair is glossy black, and their head is crowned with a golden wreath.
    They are also fond of appearing as a brilliant white horse or a radiant golden stag.
    In any guise, they look lit by the sun, even when they travel across the night sky.

    \begin{figure}[t]
        \centering
        \includegraphics[width=0.47\textwidth]{02viphoger/img/10s_heliod.png}
    \end{figure}

    \subsubsection{Worshiping Heliod}
        The brilliance of Heliod's sun is impossible to ignore.
        Thus, virtually everyone on Viphoger pays at least grudging respect to the sun god in forms of worship that range from simple gestures to days-long celebrations.

        Some families, particularly in the polis of Mephetis, follow a practice of bowing in the direction of dawn's first light --- or winking, in a gesture of respect for the sun god's luminous ``eye''.
        More dedicated worshipers offer short litanies at dawn, noon, and dusk, acknowledging the sun's passage across the sky.

\subsection*{Iroas, God of Victory} \label{ssec::iroas}
    \subparagraph{Domains} War.

    Iroas is the steadfast god of honor and victory in war.
    % When soldiers march to battle, Iroas's voice is the thunder of their footsteps and the crash of spear on shield.
    Soldiers, mercenaries, and athletes all pray for Iroas's favor in securing victory.
    Common folk pray to Iroas for courage and fortitude in times of struggle, for they'll aid in the battle nobly fought.

    Bold and confident with a soldier's demeanor, Iroas is the pinnacle of martial pride and bearing.
    They are stoic almost to a fault, but also exhibits a wry sense of humor.
    Those who honorably shed blood in Iroas's name can count on their support.
    Cowards and oath breakers are to be despised, and traitors don't deserve mercy in battle.

    Iroas most often appears as a powerfully built gat with a bull-like body, clad in gleaming armor and wielding a spear and shield.
    They speak in a booming baritone that projects power, confidence, and courage.
    They have been known to appear as a burly soldier or a mighty bull before their followers.
    Whatever form they choose, Iroas carries themself with precision and majesty at all times and doesn't tolerate disrespect or undue informality from those who would deal with them.

    \begin{figure}[t]
        \centering
        \includegraphics[width=0.47\textwidth]{02viphoger/img/10s_iroas.png}
    \end{figure}

    \subsubsection{Worshiping Iroas}
        Iroas is interested not in pretty words, but in great deeds.
        The faithful of Iroas show their piety by comporting themselves well in contests of athleticism or skill.
        Swearing an oath to win a battle in Iroas's name and failing to do so is a great shame upon a warrior, thus such a promise is never uttered lightly.

        During Skenion, the annual commemoration of the Mephetian conquest of Natumas is carried.
        This victory cemented Mephetis's control over the entire peninsula.
        But in Akhosh, the month is called Iroagonion, for the Iroan Games.
        These games are the grandest display to honor Iroas.
        To even compete in the Iroan Games is noteworthy, as the poleis send only their finest athletes.
        The grand prize, besides a ceremonial wreath, is the opportunity to be visited by Iroas themself.

\subsection*{Karametra, God of Harvests} \label{ssec::karametra}
    \subparagraph{Domains} Life, Nature.

    Karametra is recognized as the serene, maternal god of the harvest, her arms spread wide as she offers bounty to her worshipers or cradles communities in her embrace.
    Almost every uman settlement contains at least a modest shrine to solicit her favor, and she is closely associated with Setesh, the center of her worship.

    Wise and even-tempered, Karametra values community, stability, and the balance of nature.
    She is the god of maternity, family, orphans, domestication, and agriculture, as well as defense of the home and territory.

    Karametra appears to mortals as a motherly figure with hair made of ordered rows of leaves that shroud her eyes from view.
    She is always shown in art (and often seen in the night sky) seated on her throne, which is formed from a tangle of grape vines growing out of a collection of jugs and amphorae that surround her.
    An elaborately carved wooden canopy extends above her, and a giant sable --- her faithful companion --- curls around the base of the throne at her feet.
    In one hand, she holds a harvester's scythe.

    \begin{figure}[b]
        \centering
        \includegraphics[width=0.47\textwidth]{02viphoger/img/10s_karametra.png}
    \end{figure}

    \subsubsection{Worshiping Karametra}
        The earth's fertility is essential for mortal life to continue.
        Those who live in the modern poleis might not be as aware of that fact as those who farm their own food, but even they long for children, know the pinch of hunger, and feel the turn of the seasons.

        Prayers to Karametra focus on asserting Karametra's constancy and bounty, praising the god's love and generosity.
        Worshipers of Karametra gather for a feast once a month, on the evening of the full Fagal, that celebrates the god's role in parenthood and community.
        New parents receive gifts and blessings, and young couples sneak away into the woods in hopes of finding sweet berries and sweeter kisses.

\pagebreak

% !TEX root = ../main.tex
\subsection*{Keranos, God of Storms} \label{ssec::keranos}
    \subparagraph{Domains} Knowledge, Tempest.

    Keranos is the god of storms and wisdom. Merciless and impatient, Keranos is equally likely to strike out at mortals with a bolt of inspiration or a blast of lightning. To revere Keranos is to exult in the power of wisdom, clarity of purpose, and the fury of the storm. He is favored by tinkerers, inventors, and sailors as well as those seeking solutions to intractable problems. He doesn't tolerate the company (or the worship) of fools, and he despises vapidity and indecision.

    Keranos rarely appears directly to mortals, preferring to communicate through an epiphany or a crashing bolt of lightning. When he does deign to manifest in the mortal world, Keranos prefers the form of a stout, bearded, male human wearing a purple loincloth girdled in a mithral chain belt with a clasp in the form of a dragon's skull. His bearing is upright and stern, with a clipped, brusque way of speaking. Particularly clever plans and observations bring a hint of a smile to his face. When interacting with mortals, Keranos sometimes appears in the form of a great horned owl with lightning strikes flashing in its eyes.

    \subsubsection{Worshiping Keranos}
        Keranos's name is often invoked by those amid a storm who seek safety, or by someone who is faced with a particularly difficult problem. Only the foolhardy call out to Keranos frivolously or in jest, since he might well smite the offender with a bolt from the blue.

        In Akros, where Queen Cymede actively promoted the worship of Keranos, elaborate ceremonies are conducted beginning just before the first summer thunderstorm. Intricate, open-framed sand paintings with complex geometric shapes are created by dancers in flowing blue silken wraps. Then, as the rains fall, the paintings are washed away, symbolizing the impermanence of genius and the power of change. Akroan oracles strive to predict the exact time of the first storm in hopes of allowing enough time to stage the celebration. A similar festival in Meletis, called the Lightning Festival, gives its name (Astrapion) to the third month of the year.

        On the last day of every month, Keranos's priests and laity bring offerings of fish and distilled spirits to his temples. The fish are cooked under a skylight open to the stars, with a shot of spirits thrown on the fire.

\subsection*{Klothys, God of Destiny} \label{ssec::klothys}
    \subparagraph{Domains} Knowledge, War.

    Believed to have sprung into existence during Theros's earliest days, Klothys is the god of destiny and, along with Kruphix, one of the plane's original deities. She oversees the order of the cosmos, ensuring that all things remain in their proper place, knowing how easily the cosmic balance could be undone if she were not vigilant. On the heels of a near-catastrophic upset of the cosmic order—the rise to godhood and subsequent defeat of the satyr Xenagos—Klothys has emerged from the Underworld for the first time in mortal memory to untangle the strands of destiny and set the world right.

    Klothys typically appears as a woman with six curling horns and an impossibly long mane of pale hair that cascades around her horns, drapes over her eyes, and spools into her spear-like weapon and the various other spindles she carries.

    Beneath her outward calm, Klothys seethes at the way mortals and gods alike have pulled apart and rearranged the threads of destiny to feed their petty ambitions. Her peaceful mien falls away in the presence of such villains. In her rage, her red-glowing eyes come into view through the veil of her hair, and she wields burning strands of hair as a devastating weapon.

    \subsubsection{Worshiping Klothys}
        Klothys doesn't trace her origins to mortal devotion, and she has languished in obscurity for almost the whole of human history. Unlike the other gods (except Kruphix), she doesn't need worship to sustain or empower her, and she doesn't seek out reverence or demand it. By and large, mortals are irrelevant to her, except insofar as they have played a role in tangling the strands of destiny by defying nature's order.

\subsection*{Kruphix, God of Horizons} \label{ssec::kruphix}
    \subparagraph{Domains} Knowledge, Trickery.

    Kruphix is the enigmatic god of mysteries, horizons, and the passage of time. His followers claim that he knows not only everything that is known at present, but everything that has ever been known by anyone.

    Quiet surrounds Kruphix like a shroud. Standing apart from the other gods, he speaks rarely, even to his most favored followers. When he does communicate, it is often as a barely audible whisper. Kruphix can speak with a booming voice directly into the minds of all the other gods simultaneously, though, doing so when something threatens the cosmic order.

    Kruphix's true form is more abstract than that of any of the other gods. He appears only in star-filled silhouette, usually as a hooded, four-armed figure of indeterminate species and gender. Two of the stars in his "body" often shine brightly, suggesting eyes. Kruphix's starry silhouette sometimes takes the form of a bird or a whale.

    \subsubsection{Worshiping Kruphix}
        Many pray to Kruphix when they need to find something lost, but few dedicate themselves to his worship. Cults devoted to Kruphix fiercely guard their secrets, and their initiates refrain from drawing attention to themselves. Some followers and champions of Kruphix travel the world in secret, searching for hidden truths. Many use secret signals to enable them to find safe lodging with other worshipers nearly anywhere.

        Rituals honoring Kruphix are usually performed at boundaries, both temporal and spatial: shorelines, riverbanks, equinoxes, and sunsets. One of the god's greatest festivals is the Agrypnion ("the Watching"), which marks the end of winter and the close of the year.

% !TEX root = ../main.tex
\subsection*{Mogis, God of Slaughter} \label{ssec::mogis}
    \subparagraph{Domains} War.

    Mogis is the god of slaughter, violence, and war. He is hatred unrestrained, empathy denied, and mercy forgotten, an entity whose very presence incites mortals to violence. Soldiers fear succumbing to his blood lust lest they dishonor themselves, but the vengeful and forsaken call to him for the gift of his rage. He is the brother of Iroas, god of victory, and his antithesis in matters of warfare.

    The anger and malice radiating from Mogis is almost palpable. He exercises no control over his temper or his urges and lashes out at subordinates at the slightest provocation. Akroan soldiers are warned that to give in to his seductive battle rage is to risk becoming an androphage—a bloodthirsty killer wholly consumed by Mogis's fury.

    Mogis cuts a terrifying figure, appearing as a four-horned minotaur of incredible size clad in spiked bronze armor and wielding a massive ebon greataxe. He doesn't debase himself by appearing in other guises to mortals—to behold him is to behold the cruelty of war personified. He hungers endlessly to defeat his brother Iroas in combat and thus become the sole avatar of war among mortals.

    \subsubsection{Worshiping Mogis}
        Mogis exhorts his followers to channel their hatred and rage into ever greater acts of cruelty and violence. He demands actions over words, making his followers an active and dangerous lot. From the spurned lover thirsting for revenge to the blood-drenched warrior on the battlefield, all honor Mogis with the shedding of blood in anger.

        Minotaurs are the most ardent worshipers of Mogis and regularly hold bloody rites in his honor. Warchanters, the minotaur clergy of Mogis, whip their marauders into a near-mindless frenzy before battle; the ensuing slaughter gives glory to Mogis's name.

        The appearance of the blood moon is a most holy occasion for the faithful of Mogis, since the moon represents his hateful crimson eye. At such times, his followers prepare and consume a feast of meat, either raw or barely cooked, along with copious amounts of intoxicants, followed by ritual self-mutilation—scarring themselves to demonstrate their devotion to Mogis.

\subsection*{Nylea, God of the Hunt} \label{ssec::nylea}
    \subparagraph{Domains} Nature.

    Nylea is the wild, carefree god of the hunt. She claims dominion over the whole of the natural world, particularly hunger and predation, the seasons, metamorphosis and rebirth, and the forest.

    Nylea is among the most gregarious of the gods, and can be spotted frolicking joyfully with her Nyxborn lynx, Halma, or her favorite nymph, Theophilia. But she also savors solitude, and on the hunt she is deadly serious, almost animalistic, in her mood. She is nearly as quick to anger as her brother Purphoros, enacting swift revenge on those who harm the natural realm.

    Nylea usually appears as a green-skinned dryad with woody extremities. Her hair is made of vines and leaves that change with the seasons. She might also appear as a majestic specimen of any animal, most frequently a lynx or a wolf. When she desires stealth or solitude, she might take the form of a tree, usually an oak or an olive.

    \subsubsection{Worshiping Nylea}
        Mortals all over Theros pray to Nylea when they rely on hunting or nature's whims for their livelihood. Her most ardent followers are satyrs, centaurs, humans (particularly those who live in Setessa and in the wilds), and nymphs of all kinds, especially dryads. Few leonin worship any of the gods, but of those who do, many favor Nylea with their prayers.

        Nylea blesses those who are kind to animals, considering such acts as wordless prayers. Those who must kill a dangerous natural animal or cut down trees often pray to Nylea for forgiveness, sometimes leaving food for other animals or planting new trees as atonement.

\subsection*{Pharika, God of Affliction} \label{ssec::pharika}
    \subparagraph{Domains} Death, Knowledge, Life.

    Pharika is a god of affliction and medicine, alchemy and aging. In the earliest days of Theros, Pharika seeded the world with countless secret truths—mysteries of medicine, minerals with strange properties, nexuses of magic, and the like—which she hid among Nylea's wilds and the shadows of Erebos's Underworld, leaving clues where mortals might find them. It isn't altruism that drives her; she studies the innovation and suffering of mortals, deciphering in them ever greater mysteries as she treats Theros as her personal laboratory.

    Pharika typically takes the form of a green-skinned human woman with the lower body of a snake. Her hands are thickly scaled and a pair of bronze-scaled vipers seamlessly emerge from her chest. She is never without her kylix, a drinking cup within which she can produce virtually any medicine or toxin. When her aims require subtlety, Pharika often takes the form of a serpent or a medusa, or sometimes an aged human.

    Little escapes Pharika's cool gaze. Even when outwardly friendly, she is cunning and calculating, watching for the slightest sign of weakness or desire that she can exploit later. Those who offend her rarely recognize their misstep until she strikes.

    \subsubsection{Worshiping Pharika}
        The diseased and the dying alike often make written entreaties to Pharika for a remedy. Prayers are written on scraps of paper or shards of pottery, sealed in small pots, and buried in bogs, leaving them as secrets for others to exhume years later. Many people pray to her before undergoing a medical procedure, picking herbs, or confronting a venomous animal. Nights of a waxing crescent moon (roughly the first week of each month, when a sliver of moon lingers in the early evening) are sacred to Pharika and are thought to be an auspicious time to harvest medicinal plants.

        Pharika's followers include members of several small mystery cults, which embrace varying aspects of her divine nature. The most infamous of these is the Cult of Frozen Faith, led by a medusa. Initiates receive a lethal dose of poison, become petrified, and then are restored to flesh one year later. Petitioners who have Pharika's favor emerge alive and healthy; those she doesn't care for fail to survive the transformation.

% !TEX root = ../main.tex
\subsection*{Phenax, God of Deception} \label{ssec::phenax}
    \subparagraph{Domains} Blue, Silver.

    Phenax is the masked patron of lies and cheats.
    They are Heliod's ethical antithesis, governing the spheres of gambling, deception, and betrayal.
    Phenax was once a mortal who was trapped in Nyx, but they learned how to forsake their identity to prevent Erebos from detecting what they were doing.
    They crossed back over the Rivers That Ring the World wrapped in the tattered cloak of Athreos, the River Guide.
    Hidden by illusion as they were, neither Athreos nor Erebos could find Phenax and bring them back.

    % Able to play whatever role the situation calls for, Phenax is a consummate actor.
    % Their incisive wit and cunning enable them to read the desires of their marks, adjusting their approach to suit the moment.
    % In their rare moments of candor, Phenax is calm and calculating, always looking toward their next scheme.

    Phenax is a shadowy and mysterious figure.
    When appearing before mortals, they prefer the form of a willowy gat with ashen gray skin, clad in elegant robes.
    They have also been known to appear in a variety of animal forms, including the shapes of asps, mockingbirds, or rats.
    Regardless of their shape, a mask forever conceals the blank face of the first Returned.

    \begin{figure}[t]
        \centering
        \includegraphics[width=0.47\textwidth]{02viphoger/img/10s_phenax.png}
    \end{figure}

    \subsubsection{Worshiping Phenax}
        Every lie is an homage to Phenax.
        Because their most devout followers are criminals and gamblers, their influence is keenly felt in gambling halls and dens of thieves.
        But everyone has their own reasons to stray from the truth at times, and thus, they also find small ways to seek Phenax's favor as they go about their daily lives.

        Formal services to Phenax are conducted at night, with the most sacred rituals performed on nights of the new three moons.
        Offerings are made to attract Phenax's favor, with valuables from successful robberies, parchment filled with lies, or loaded dice being thrown into deep crags or buried at crossroads.
        Such sacrifices often vanish soon after, claimed by the god or their servants.
        Devout criminals often offer Phenax stolen goods as part of their preparations for premeditated crimes.

        % Phenax is worshiped openly in the necropoleis of Asphodel and Odunos, though the Returned who are loyal to Erebos's agent, Tymaret, refuse to worship the god they're hunting.
        % Somber ceremonies are intoned to bless the golden funeral masks the Returned wear.
\subsection*{Purphoros, God of the Forge} \label{ssec::purphoros}
    \subparagraph{Domains} Red.

    Purphoros is the god of the forge, the restless earth, and fire.
    They rule the raw creative force that infuses sapient minds.
    Purphoros is also the god of artisans, obsession, and the cycle of creation and destruction.

    As a forge radiates heat in the area around it, Purphoros's influence provides inspiration to mortals.
    They makes exquisitely crafted objects almost constantly, sometimes absentmindedly working while they holds conversations with the other gods, only to destroy the finished product and begin again.
    % Impulsive and mercurial, Purphoros is prone to bouts of either joyous productivity or frustrated anger.
    Purphoros often feels constrained by the limits of imagination, yearning to realize ideas that seem just out of reach.

    Purphoros's preferred form is that of a muscular treb gat whose coal-hued skin is mostly covered in mutable organic bronze.
    They might also appear in the form of a phoenix or a bull made of cooling lava, and for that reason, both of those creatures are associated with them.
    When angered, they might appear as an enormous mass of lava, a blazing fire, or a volcanic eruption.
    Mortals who see Purphoros in one of those forms seldom live to tell about it.

    \begin{figure}[b]
        \centering
        \includegraphics[width=0.47\textwidth]{02viphoger/img/10s_purphoros.png}
    \end{figure}

    \subsubsection{Worshiping Purphoros}
        Purphoros holds dominion over everything that springs from mortal ingenuity.
        Most artisans say a small prayer to them upon beginning or completing the construction of nearly anything, from swords to fortresses to ships.

        Naturally, Purphoros is strongly associated with the forge, and nearly every smithy on Viphoger is a sort of ad hoc temple to them.
        Charms and idols of Purphoros hang from the walls in such places, intended both to inspire the artisans and protect them against accidents.
        Regardless of their professions, worshipers of Purphoros often light small fires in the god's honor, burning wooden crafts or drawings of their inventions to gain their favor.

\pagebreak

\begin{tikzpicture}[remember picture,overlay]
    \node[anchor=north, yshift=0.10cm] at (current page.north) {\includegraphics[width=\pdfpagewidth]{02viphoger/img/10kruphix.png}};
\end{tikzpicture}

\vspace{15.0cm}

\subsection*{Thassa, God of the Sea} \label{ssec::thassa}
    \subparagraph{Domains} Blue.

    Thassa is the god of the sea, aquatic creatures, and the unknown depths.
    She also holds sway over less tangible concepts such as ancient knowledge, long voyages, and gradual change.

    % Impassive and slow to anger, Thassa is secure in the knowledge that there are no mortals and few gods who can threaten her status.
    % Once her ire is aroused, however, it is as unstoppable as a cresting wave.
    % She often speaks in the future tense, referring to what tomorrow will bring.
    % She seldom laughs, and when she does, it is usually out of smugness rather than genuine mirth.

    Thassa usually appears to mortals in the form of a female tortle-like being with octopus-tentacle hair and a crown of crab legs.
    % She seldom adopts the same size as her followers, preferring to be seen from a distance as she towers over the ocean.
    When she moves close to the view of mortals, she takes many other forms, often shifting from one to another: a giant squid, an ocean storm, a school of sharks, a fog bank, or a crab, her favored animal.
    % She sometimes speaks out of the ocean itself, in droplets hissing across the surface of the waves.

    \begin{figure}[b]
        \centering
        \includegraphics[width=0.47\textwidth]{02viphoger/img/10s_thassa.png}
    \end{figure}

    \subsubsection{Worshiping Thassa}
        Most of Thassa's dedicated worshipers are tortles.% , and the vast majority of tortles are wholly devoted to Thassa.
        Tortles spend much of their lives in Thassa's realm, with their god omnipresent.
        They weave prayers to Thassa into nearly everything they do.

        \pagebreak~
        \vspace{15.5cm}

        Among the poleis, Thassa is worshiped by those who rely on bountiful seas for sustenance or calm waters for safety.
        Sailors, fishers, and residents of Viphoger's coasts and islands all pay her at least nominal respect and sacrifice.

        % Her center of worship on land is in the coastal polis of Mephetis, where sailors and philosophers pray to her for guidance.
        % The week-long Lyokymion festival (the Feast of the Melting Swell) marks the start of the new year by celebrating the bounty of the sea.

        % Thassa's most fervent gat worshipers offer prayers at high and low tide.
        % If possible, they do so at the water's edge.
        % At low tide they walk barefoot out onto the tidal flats, relishing the touch of Thassa's seabed.


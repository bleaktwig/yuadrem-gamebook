% !TEX root = ../main.tex
\subsection*{Athreos, God of Passage} \label{ssec::athreos}
    \subparagraph{Domains} Death, Grave.

    All mortals are destined to face Athreos, the River Guide, when their lives come to an end.
    The god of passage ferries the dead across the Tartyx River, conveying each mortal soul to its destiny in the Underworld.
    For most people, Athreos embodies the greatest mysteries of existence --- the terror and wonder of life's last moment and the revelation of one's ultimate fate in the afterlife.
    Athreos is no judge, though.
    The veiled, silent god undergoes no deliberations and makes no exceptions.
    The River Guide reads the truth of each soul and bears it unfailingly to its proper place in the Underworld.
    There is no haggling and no sympathy on Athreos's skiff, the god having heard and denied every conceivable mortal plea.

    Athreos appears as a gaunt figure cloaked in ragged robes and a collection of golden masks.
    What little can be seen of their body is unsettling, its gray flesh stretched thin over a barely et skeleton.
    The River Guide is never without their ancient staff, Katabasis, which they transforms into the ferryboat they use to ply the Rivers That Ring the World.
    Athreos can change shape but rarely, if ever, takes on other forms.

    \subsubsection{Worshiping Athreos}
        Most funeral traditions include small offerings and words of reverence to Athreos.
        Predominant among these traditions is burying or burning the dead with a clay funerary mask, to "frame" the identity of the dead for Athreos, and with at least one coin, so a soul might pay Athreos to ferry them to the Underworld.
        Some people are laid to rest with large amounts of grave goods.
        Memorial practices vary widely by culture, from tearful, somber affairs to lively celebrations.
        These rituals serve more as catharsis for the living than as meaningful boons to Athreos, though.
        The River Guide cares only for the single coin they's owed by any who board their skiff.

        During the feast of Ranion, which gives its name to the eighth month in the calendar of Mephetis, pious souls silently spend the day reading ancient memoirs or writing messages for their own descendants.

\subsection*{Ephara, God of the Polis} \label{ssec::ephara}
    \subparagraph{Domains} Knowledge, Light.

    As god of the polis, Ephara sees themself as the founder of civilization.
    They watch over cities, protecting them from outside threats.
    They are credited with establishing the first code of law, which Mephetis has preserved and the other poleis have imitated.
    Even more important, they help cities reach their highest potential, becoming centers of scholarship, industry, and art.

    Ephara appears as a huge animated statue wearing a stone crown, resembling the capital of a column.
    When they choose to walk about their cities at a mortal's scale, they often take on the form of a gat.
    In either form, they are always dressed in blue and white, and their expression is usually serious, but not unkind.
    They often carry a large urn on one shoulder, with the dark, star-studded sky pouring from it and dissolving into mist as it hits the ground.

    \subsubsection{Worshiping Ephara}
        To an extent, Ephara's devout show their faith by going about their lives and contributing to society.
        Midday services at Ephara's temples often feature a brief prayer, followed by a longer talk from an industrial or civic leader on a topic of general interest.
        Attendants often bring meals to eat while on a break from their jobs.

        Ephara's face is a common sight in cities.
        Marble buildings, stone walls, and similar surfaces usually feature a sculpture or relief of their visage.
        People often swear oaths or engage in verbal disputes in front of these images, believing they won't let a falsehood told in front of them go unpunished.
        Whether they actually intervenes is unclear, but conflicts that play out this way are often resolved peacefully, without a need for the justice system to get involved.

\subsection*{Erebos, God of the Dead} \label{ssec::erebos}
    \subparagraph{Domains} Death, Trickery.

    Erebos is the god of death and the Underworld, lord of all that has ever lived.
    They preside over the bitterness, envy, and eventual acceptance of those who suffer misfortune.
    Their hoarding of both souls and the treasures the dead carry into the Underworld see them worshiped by those who desire to collect and keep wealth.

    Erebos's very presence is stifling, and those who come face to face with them often depart in despair.
    They are jealous and tyrannical within their realm, but unlike their brother Heliod, they neither bluster nor try to expand their influence.
    They wait patiently, secure in the knowledge that everything belongs to then in the end.

    Erebos most frequently appears as a slender, gray-skinned gat with two large, outward-curving horns, wielding an impossibly long black whip.
    They also appear in the form of a black asp, a cloud of choking smoke, or an animated golden idol.

    \subsubsection{Worshiping Erebos}
        To many mortals, Erebos is primarily concerned not with death, but with gold.
        Most of their followers downplay their association with death and misfortune, instead praying to them for material wealth.
        Others pray to them because they want to be more accepting of their misfortune.
        These individuals see themselves as beyond hope of improving their lot in life, asking only that Erebos grant them the strength to endure until they enter their realm at their predestined time.

        A smaller but more dangerous group of Erebos worshipers are those who actively glorify death.
        These cultists and assassins congregate in secret in communities across Viphoger, engaging in campaigns of violence.

        The only major festival dedicated to Erebos, called the Katabasion or "the Descent," features a ceremony in which worshipers make a symbolic journey into the Underworld.
        The supplicants enter a cave, offer prayers and sacrifices to Erebos in utter darkness, and slowly make their way back to the surface just before sunrise.

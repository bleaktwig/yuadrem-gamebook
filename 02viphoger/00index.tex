% !TEX root = ../main.tex
\chapter{Viphoger} \label{ch::viphoger}
\DndDropCapLine{A}{s a new nation, Viphoger is small. Its}
poleis tiny in comparison to the unexplored wilderness beyond.

\begin{table*}[b]
\begin{DndTable}[width=\linewidth, header=Fagalian Calendar]{cXXX}
    \textbf{Month} & \textbf{Name} & \textbf{Tide} & \textbf{God} \\
    1              & Amion         & Blue          & Heliod, God of the Sun \\
    2              & Sianion       & Gold          & Ephara, God of the Polis \\
    3              & Granion       & Magenta       & Iroas, God of Victory \\
    4              & Zmivion       & Silver        & Phenax, God of Deception \\
    5              & Skenion       & Red           & Mogis, God of Slaughter \\
    6              & Danion        & Blue          & Kruphix, God of Horizons \\
    7              & Dibinion      & Gold          & Purphoros, God of the Forge \\
    8              & Ranion        & Magenta       & Athreos, God of Passage \\
    9              & Amelamion     & Silver        & Thassa, God of the Sea \\
    10             & Amelsion      & Red           & Erebos, God of the Dead \\
    11             & Amelgranion   & Blue          & Pharika, God of Affliction \\
    12             & Amelzvion     & Gold          & Nylea, God of the Hunt \\
    13             & Amelskenion   & Magenta       & Klothys, God of Destiny \\
    14             & Ameldanion    & Silver        & Karametra, God of Harvests \\
    15             & Ameldibinion  & Red           & Keranos, God of Storms \\
    16*            & Fisimmas      & White         & The Sorrow
\end{DndTable}
\end{table*}
% * This day only occurs once every ten years.

Viphoger consists of a peninsula forming the south-western edge of the vast Whaler's Sea.
South from the sea, the land rises up to a ridge of mountains.
The lofty peaks of this ridge forms a barrier that few cross, so only rumors of the vast Dead Sea describe the land beyond.

To the west, the coastal lands become small pockets of forests crossed by a labyrinth of arid canyons, with the Zoedrem desert beyond.

The Siren's inlet to the north is studded with islands large and small.
The largest cluster near the mainland, called the Dakra Isles, is poorly charted, and even those sailors who attempt to explore the isles return with contradictory information.
Eastward from Viphoger, the old nations of Yuadrem protect their lands, dominated by the strong influence of the Seven Kingdoms of the Sea.

The heart of Viphoger lies in and around three poleis—cities and their surrounding territories.
Together the three poleis, Akhosh, Mephetis, and Setesh, encompass most of the population of Viphoger.
Mephetis covers the whole territory of the northeastern peninsula, Akhosh forms the frontier to the desert, and Setesh lies at the northern edge of the wild Nessian Wood.

The bands of the Uqhardu, the Vahagha, and the Pheres, roam the hills and badlands between the three poleis.
Remnants of the once great empire of Hulnar, now reduced to bandits and traders.
The mysterious leonin hunt in the valley of Oreskos, nestled between Mt. Kure and Winter's Heart.
Bughna gats and marsets dwell around the Stola Vale and the larger Nessian Wood.
And tortles live primarily in the coastal shallows of the Siren Inlet, though some manage to make comfortable homes among the gats of Mephetis.

The canyon of Katajthon, south of Akhosh, is the frontier where Akhoshian soldiers clash with Treb gats.
Farther south-wwest is the Treb city of Kofos, little known to Viphogerians.

The necropoleis of Asphodel and Odunos are home to the Returned --- zombie-like beings who have contracted the Illness.
Access to these lands is strictly forbidden, and is punishable by death in all three poleis.
The lands around these cities are bleak and barren, as if the Returned brought the pall of the underworld out with them into the mortal realm.

% !TEX root = ../main.tex
\subsection*{The Fagalian Calendar} \label{ssec::thefagaliancalendar}
The oth astronomers and philosophers from oldentimes established a calendar that has found massive adoption in the rest of Yuadrem.
It divides the year into fifteen months of twenty-four days, each beginning with Fagal's new moon.
Every ten years, an extra day is added at the end of the calendar to keep it aligned with the solar year.

The beginning of the year is considered the end of spring, so the new year begins with the summer.
Each month is associated to a tide, and is holy to a specific god.
A major festival in honor of that god is celebrated in Mephetis.
The fifth month (Skenion in Mephetis) is called Iroagonion in Akhosh, after the Iroan Games, which are held in that month every year.

% The Fagalian Calendar table summarizes the months, their lengths, and the god each is associated with.

% !TEX root = ../main.tex
\subsection*{Life in the Poleis} \label{ssec::lifeinthepoleis}
Civilization in Viphoger is centered in three poleis: Akhosh, Mephetis, and Setesh.
These poleis exemplify the kins' drive to settle the land, to shape nature according to their needs, and to organize into political structures that can withstand the changing fortunes of the passing centuries.

Each polis is centered in a city but includes a wide region of surrounding territory, and each one has its own distinct society and culture.
To the people of Viphoger, ``Mephetis'' is more or less synonymous with ``Mephetians'' --- the polis isn't just the people who live in the city of Mephetis or even those who dwell in nearby villages; it is the people who follow the Mephetian way of life, wherever they might be found.

\subsubsection{Citizenship and Government}
In every polis, civic responsibility and full protection are afforded only to citizens.
Citizenship is limited to those whose parents were both citizens of the polis.
Citizens of other poleis, and their children, aren't permitted to participate in the government of the polis.
In Akhosh, citizens must meet one additional requirement: they must serve in the army.

The three poleis have different political structures, but each one has a council elected by popular vote of the citizenry.
The Twelve, Mephetis's council of philosophers, is the democratically elected ruling body of the polis.
Akhosh is ruled by a hereditary monarch who is advised by a council of elders elected by and from among the citizenry.
Similarly, Setesh's Ruling Council is formed by popular vote, and they govern the polis while its queen --- the goddess Karametra --- is absent.

\subsubsection{Trade and Currency}
Trade between Akhosh and Mephetis is constant and productive.
Caravans make the two-week journey between the poleis twice a month, aided by the Tsher river.
They carry fine Akhoshian metalwork and pottery to Mephetis, and Mephetian fabric, stonework, and fish westward.
Both poleis mint coins of copper, silver, and gold, with equivalent value.

Setesh trades with the other poleis as well, but less extensively.
Its Abora Market, just inside the city gates, is open to outsiders only on certain days, and Seteshan merchants prefer to barter goods rather than accept currency.
Despite these restrictions, Seteshan food, woodwork, and trained falcons are highly valued in the other poleis.

Aside from the other poleis, Mephetis and Setesh both trade with the dratl irds of the Vahagha band.
The irds don't work metal, so they trade woodwork, the produce of the plains, and woven blankets to the human poleis in exchange for weapons and armor.

\subsubsection{Recreation}
The people of the poleis enjoy the opportunity for some recreation, as time and money allow.

Gymnasia are popular gathering places, offering athletic training as well as space for philosophical discussion and friendly socializing.
A resident of the city might visit a gymnasium one day to exercise, the next to view a wrestling match between celebrated competitors, and the next to hear a renowned philosopher give a lecture on ethics.

Another important venue for recreation is the theater.
The works of celebrated playwrights, past and present, are regularly produced by casts of professional actors.
On occasion, a storyteller, accompanied by a small orchestra, draws crowds to a theater for a recitation of one of the great epics, such as The Sylvan Wars, The Theriad, or The Callapheia.
Such a performance might stretch over two or three days.


\incgraph[documentpaper,][width=\paperwidth,height=\paperheight]{02viphoger/img/00map.png}
\newpage~

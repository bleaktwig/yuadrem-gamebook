% TODO: CHANGE ALL INSTANCES OF PRIEST TO MONK IN GRONSELAR

\section{Second Landing}

\DndDropCapLine{T}{wice you thought you conquered us,}
\textit{yet twice we were liberated by our fair White Queen El\~na.
Try as much as you might, you will never subjugate us.
We are the people of Gronselar, and we will forever be immortal and free.}

\hspace*{\fill} --- Gulsen's Letter to Krugghom, 381 AS.

% intro
In the year 344 AS, a Krudzalian iron-hulled ship named the Timur set sail to the Wildlands.
It landed at the south-eastern coast of the Elderberry Wilds, where its crew established the port town of Siszkruech.
The town's location was selected strategically, and an open-pit magnetite mine was quickly dug.

While the cold climate was an easy fit for the thulkraka irds of Krudzal, the local fauna was not.
Much like the rest of the Wildlands, the Elderberry Wilds are home to the strange creatures born in the Tidal Sway, dubbed the xuagra.
Stone walls were erected, and many a settler gave their life for the colony.

Not two decades after its establishment, Siszkruech suddenly lost contact with Krudzal.
Lacking the capacity to sail there, conflict quickly stirred in the city.
Two political parties, loyalists and separatists, fiercely fought for control.
The latter, led by the White Queen El\~na, rose up to power and established the independent nation of Gronselar.

% Under the rule of strong queens and kings, the country has thrived against all adversity.
Expanding along the coast, the Shtardym river, and the Tydgrux mountains, Gronselar has thrived against all adversity.
Despite the nation's strong grip on the land, the forest is still a strong adversary.
It is common for hunters and farmers to be lost to the xuagra, and sometimes even entire hamlets are razed by the beasts.

% Despite the circumstances, Siszkruech quickly grew into thriving economy, and the Timur sailed back and forth into the city every two years.
% This relationship however lasted very little, since the Timur was never seen again after its eighth trip to the port.

% Without iron-hulled ships, and lacking the capacity to build them, the colonists were left stranded in Siszkruech.
% Conflict came faster than expected, and two rival political parties attempted to claim power: The loyalists and the separatists.
% A civil war sprung up, which was quickly won by the former in 360 AS.

% Zuleija, the first governor of the colony, was quick to execute as much separatists as possible.
% The general population sided with them notwithstanding, and the opposition only grew with the years as nothing was heard from Krudzal.

% The conflict peaked in 377 AS, when an insurgent group, led by the so-called White Queen El\~na, took the loyalists by surprise and ended Zuleija's reign.
% El\~na hastily anointed king Ald to the throne, and disappeared with the rest of her party into the forest, never to be seen again.
% Only small quarrels ensued, and the loyalists quickly dispersed into anonymity.

% Under the rule of Ald and the following queens and kings the strong nation of Gronselar was born.
% This didn't mean the end of conflict for the nation though, as the forest proves a strong adversary to date.
% It is not uncommon for hunters and farmers to be lost to xuagra, or for entire hamlets to be razed by them.

\subsection*{Life in Gronselar}
Civilization in Gronselar is sparse, but is centered in two cities: Siszgoel and Yregrux.
These bastions exemplify the peoples' drive to settle the land, against all odds and hardships.

Population is distributed into 12 eadsbeers.
Each eadsbeer consists of a considerable number of people and land, along with a small local government, ruled by an ead.
Each eadsbeer is named after the city or town of residence of its ead, and are ---from largest to smallest---: Syszkruech, Yregrux, Sygrunselar, Shtarshlar, Getydzar, \~Narselar, Undatyr, Dymkust, Sternad, Zekretyr, Kated Field, and Do Odo.

\subsubsection{Government}
The highest ruler in Gronselar is the traad.
They live in Syszkruech, and rule the entire nation with a group of people known as the council of five.
The traad is the avatar of the great god Kyyrat, while the council are the avatars of the five lesser gods.

To become the traad, one must first become the avatar.
Avatars are people chosen by the gods among those who do great deeds in their name.
They must proclaim their divinity during the funeral of the previous avatar, and all claiming to be chosen by the same god must compete to decide who is the true avatar.
The nature of the competition is decided by the traad and the remaining members of the council.

When a traad dies, the avatar of the following god in the cycle is anointed as the new traad.
During the coronation, the avatar forfeits their allegiance to the god in question, and pledges allegiance to Kyyrat.
The coronation is celebrated with a competition to decide the following avatar of the forfeited god.

If a traad dies and there is no following avatar then the council becomes a temporary government.
This is a rare occurrence that has only happened twice in Gronselar's history.
The months or years without a traad are usually very unstable, and in both instances small-scale civil wars quickly sprung up.

Each ead has local power in their own eadsbeer, which is shared with a board of four people.
The ead has the final say in all matters, but members of their council can communicate to the traad any misdeeds committed by their ead.

Finally, every village is ruled by its udtere.
Udteres are to settle squabbles and manage villages, but their role is not considered of special importance --- they are another member of the village.
Traads and eads have life-long roles, while udteres are chosen by the people.
They can be changed at any time, as long as the majority agrees.

\subsubsection{Religion and Travel}
Trade in Gronselar is sparse and ineffective.
Roads between eadsbeers remain unused for most of the year, yet not entirely.
Monk parties travel these paths regularly.

It is a gronselarian custom to send the dead to Udbeer, the Nyx.
This is done once every ten years, in a special day called Yg\~naxtad.
Monk groups are constantly traveling all around Gronselar, gathering the mummified dead from each village.

Travelers and merchants commonly join these parties, forming large caravans.
To defend themselves from the xuagra, they employ local hunters as paid guards.
Additionally, all monks are trained in the magic of Rashid, and employ their skill to defend the caravan as well.

\subsubsection{Trade and Currency}
% Trade between eadsbeers is sparse and ineffective.
% Travelling merchants are a very rare sight, and most villages receive at most a couple of caravans each year.
% Merchants are celebrated due to this, and are to be protected and well-attended when they visit.

It's an entirely different story for the main river of Gronselar, the Shtardym.
The waterway splits the nation in half, and is the safest way to access its inner regions.
Small boats can be seen all across the river, trading resources from the coast to the inland eadsbeers.
% While safer than the forests, the rivers are not without their dangers.
% Boats have painted eyes to watch for waterborne xuagra, and a unique branch of the magic of Mevthan focuses on attacking the violent creatures.

The national coin of Gronselar is the yrege, a square iron coin.
Merchants however trade in goods, and the currency is mostly used inside eadsbeers.
Additional to the yrege is the wyrge, a thick wood piece made from kapok and ash wood.
One wyrge is worth about 2500 yrege, yet it is very rarely traded.
Holding even one wyrge is a strong symbol of wealth and honor, since there are only 291 wyrge in circulation.

% \subsubsection{Traditions}
% Apart from merchants, monks also regularly move between eadsbeers to collect corpses and recruit new members.
% Monks are capable of fending for themselves, and use the magic of Rashid to fend off the xuagra.
% They also commonly travel with large numbers of travelers, creating packed travelling parties.

% It is a gronselarian custom to send the dead to Udbeer, The Nyx, during Yg\~naxtad.
% The ritual to get to this is a very elaborate process.
% After death occurs, internal organs and fluids are removed.
% Then, the body is wrapped in flax linen and is temporarily kept in an underground crypt.
% Monk groups are always travelling around Gronselar, and gather the dead from each village.

% The bodies are taken to a very large designated catacomb, which holds all those who died in the current 10-year cycle.
% The most venerable among the monks stay and care for the dead.
% The entire week before Yg\~naxtad a large celebration is held in the catacomb, during which the place is filled with oil, wood, and barley.
% Then, at the last hour of Yg\~naxtad, a sacrifice is made, and the catacomb is set aflame.
% Only in fire can the dead travel to Udbeer, and the eldest monks accompany the new dead to their eternal resting place.

% \subsubsection{Religion}

\subsection*{The Fagalian Calendar}
Even before the Schism, the astronomers and philosophers of Palegna devised the fagalian calendar.
The calendar is based on the movements of both the sun Zash and the gray moon Fagal.
The regularity of the calendar is useful for historians, and it's long-term use has led to consistent records of hundreds of years.
Aside from these, its months track the seasons remarkably well, yielding its use by the general populace.

A year is defined as one rotation around Zash, and lasts about 360 days.
A month is defined as one rotation of Fagal around the planet, Darhoc, and lasts 24 days.
A week is defined as half a rotation of the star Lenbier around Zash, process that lasts 6 days.

The new year is considered the end of autumn, so the year begins with the summer.
The fifteen months are separated into groups of five, where each is dedicated to a different god.
The year begins with the first month of Suata, followed by Etunry's, Aytoon's, Draga's, and Odata's, and then loops three times.
Every ten years there is one additional day.
This day is Yg\~naxtad, and is dedicated to Kyyrat.

The Fagalian Calendar table summarizes the months and the god each is associated with.

\begin{DndTable}[width=\linewidth, header=Fagalian Calendar]{clll}
    \textbf{Month} & \textbf{Name} & \textbf{Length} & \textbf{God} \\
                1  & Gartaor       &         24 days & Suata        \\
                2  & Tlenor        &         24 days & Etunry       \\
                3  & Xaryzor       &         24 days & Aytoon       \\
                4  & Shuldor       &         24 days & Draga        \\
                5  & Tlyor         &         24 days & Odata        \\
                6  & Zdengartaor   &         24 days & Suata        \\
                7  & Zdentlenor    &         24 days & Etunry       \\
                8  & Zdenxaryzor   &         24 days & Aytoon       \\
                9  & Zdenshuldor   &         24 days & Draga        \\
               10  & Zdentlyor     &         24 days & Odata        \\
               11  & Qeresgartaor  &         24 days & Suata        \\
               12  & Qerestlenor   &         24 days & Etunry       \\
               13  & Qeresxaryzor  &         24 days & Aytoon       \\
               14  & Qeresshuldor  &         24 days & Draga        \\
               15  & Qerestlyor    &         24 days & Odata        \\
               16* & Yg\~naxtad    &           1 day & Kyyrat
\end{DndTable}
* This day occurs only once every ten years.



\section{People of the Eadsbeers}
The territories of Gronselar are divided into twelve eadsbeers, each ruled by the local ead.
% Eadsbeers are administered in an individual city or town, after which the region is named.
Eadsbeers vary wildly from one another, culturally and economically.
Most were born from an individual settlement, set with a particular purpose in mind.

\subsection*{Executive Power}
While the traad rules Gronselar in their throne, eads are the ones in charge of actually putting their rule into action.
Eads do not rule alone however, and they anoint four aides: the shartere, the tuatere, the neatere, and the odsh.

The shartere or money-person administrates taxes, valuation of goods, and regulates trade with other eadsbeers.
Additionally, they are the only person allowed to mint money in the eadsbeer.

The tuatere or law-person is an expert on the law who usually studied the craft in an oth monastery.
Their job is to resolve legal conflicts between people, but it's usual and expected for citizens to resolve these issues among themselves.

The neatere or justice-person is in charge of the carrying out of the harshest punishments in Gronselar, left only for crimes against the nation.
All competitions that are expected to lead to severe physical harm or death need to be told in advance to the local neatere, otherwise such competitions are forbidden.

Finally, the odsh or history-person is the expert on local and nation-wide history and the local medium to commune with the gods.
The role is reserved for a zaloth, as they are seen as the oldest and wisest among all kins.

\subsection*{Adulthood}
In Gronselar, the concept of adulthood is unrelated to age.
To become an adult, one must pass through a rite of passage.
This rite depends on the calling of the person, and can be one among four: hunter, fisher, farmer, and monk.
Each rite is celebrated in a specific date dependent on the calling.

The calling of the hunters begins on Shur's Day, which is accentuated by strong tides.
The xuagra, in sync with the tides, are calmer during this period.
This helps the hunters practice their profession in peace, and allows them to travel without fearing for their village.
The calling lasts for 32 days, during which the young are initiated by the experienced.

The calling of the fishers begins on Suata's Day, which marks the calmest tides of the lunar cycle.
Large fishing parties leave to practice their profession for 32 days.
Those who wish to become adults are free to join the fishers and learn their profession.
Upon return, they give a fourth of their catch to the ead and a fourth to the udtere.
City-wide feasts are celebrated, where the new adults are celebrated with the stories of the sailors.

The calling of the farmers begins on Etunry's Day, the day of harvest.
Those who heed the calling must sacrifice an animal and claim a new farmland by feeding the earth its blood.
They spend alone an entire year, working the land, and only become adults on the following day of harvest.

The calling of the monks happens on Kyyrat's Day, the conjunction of the moons.
Tidal effects are the strongest during this day, and those who can show restraint during this day are welcome to join priesthood.
Those who accept the calling pilgrimage to El\~na's tomb, where they learn the way of the monks.

After finishing their training they are given two choices.
If they choose to dedicate their lives to the service of the dead, they join the other monks and are taught Rashid.
Those who shy away from priesthood still become adults, and usually join monk parties as merchants, mercenaries, or travelers.

% \subsection*{Marriage and Allegiance}

% \subsection*{Racial Views}
% The great majority of Gronselar's populace are thulkraka irds.
% Only irds can become eads, and it is very rare for a member of another kin to be seen as an avatar.
% Despite this, all other political roles can be taken by any member of any kin.

% Apart from irds, the population of Gronselar is composed mainly of gats, marsets, and oths.
% While they technically have the same rights as irds, there is a palpable social stigma against them in most eadsbeers.
% This causes a very uneven distribution of the population, and it historically led to the founding of Gronselar's neighbors, Froibias and Glameas.


\section{Cities of Gronselar}
\subsection*{Siszkruech}
The safest place in Gronselar. At least from the xuagra.
% The old capital. Devoid of resources, it mainly survives by leeching resources from the other eadsbeers through taxes. Nobody likes this, but it's also the religious center of Gronselar so bad luck boi.

Gronselar was founded in Siszgoel, and it remained the nation's only city for centuries.
It houses the traad's throne, and 

Religious center of Gronselar.

Its long history means that the burg and its surrounding eadsbeer is devoid or resources.

It survives by leeching money through taxes.

\subsection*{Yregrux}

\subsection*{Dymkust and Shtarshlar}

\section{Gronselarian Forests}
% % outside
% \subsubsection{Outside Gronselar}
% Gronselar is surrounded by the Ironwoods Forest, the Elderberry Wilds, the Manta Sea, and the Burnt Ocean.
% ...
\subsection*{Neighbors}
% Gronselar is neighbored by 3 moonborn oth states known as the Houses of the South.
% From north to south, the first is Froibias, known for...

% Then, there's Glameas, ...

% Finally, the last house is Visilias, which...

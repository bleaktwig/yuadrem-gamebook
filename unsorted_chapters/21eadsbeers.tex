\section{People of the Eadsbeers}
The territories of Gronselar are divided into twelve eadsbeers, each ruled by the local ead.
% Eadsbeers are administered in an individual city or town, after which the region is named.
Eadsbeers vary wildly from one another, culturally and economically.
Most were born from an individual settlement, set with a particular purpose in mind.

\subsection*{Executive Power}
While the traad rules Gronselar in their throne, eads are the ones in charge of actually putting their rule into action.
Eads do not rule alone however, and they anoint four aides: the shartere, the tuatere, the neatere, and the odsh.

The shartere or money-person administrates taxes, valuation of goods, and regulates trade with other eadsbeers.
Additionally, they are the only person allowed to mint money in the eadsbeer.

The tuatere or law-person is an expert on the law who usually studied the craft in an oth monastery.
Their job is to resolve legal conflicts between people, but it's usual and expected for citizens to resolve these issues among themselves.

The neatere or justice-person is in charge of the carrying out of the harshest punishments in Gronselar, left only for crimes against the nation.
All competitions that are expected to lead to severe physical harm or death need to be told in advance to the local neatere, otherwise such competitions are forbidden.

Finally, the odsh or history-person is the expert on local and nation-wide history and the local medium to commune with the gods.
The role is reserved for a zaloth, as they are seen as the oldest and wisest among all kins.

\subsection*{Adulthood}
In Gronselar, the concept of adulthood is unrelated to age.
To become an adult, one must pass through a rite of passage.
This rite depends on the calling of the person, and can be one among four: hunter, fisher, farmer, and monk.
Each rite is celebrated in a specific date dependent on the calling.

The calling of the hunters begins on Shur's Day, which is accentuated by strong tides.
The xuagra, in sync with the tides, are calmer during this period.
This helps the hunters practice their profession in peace, and allows them to travel without fearing for their village.
The calling lasts for 32 days, during which the young are initiated by the experienced.

The calling of the fishers begins on Suata's Day, which marks the calmest tides of the lunar cycle.
Large fishing parties leave to practice their profession for 32 days.
Those who wish to become adults are free to join the fishers and learn their profession.
Upon return, they give a fourth of their catch to the ead and a fourth to the udtere.
City-wide feasts are celebrated, where the new adults are celebrated with the stories of the sailors.

The calling of the farmers begins on Etunry's Day, the day of harvest.
Those who heed the calling must sacrifice an animal and claim a new farmland by feeding the earth its blood.
They spend alone an entire year, working the land, and only become adults on the following day of harvest.

The calling of the monks happens on Kyyrat's Day, the conjunction of the moons.
Tidal effects are the strongest during this day, and those who can show restraint during this day are welcome to join priesthood.
Those who accept the calling pilgrimage to El\~na's tomb, where they learn the way of the monks.

After finishing their training they are given two choices.
If they choose to dedicate their lives to the service of the dead, they join the other monks and are taught Rashid.
Those who shy away from priesthood still become adults, and usually join monk parties as merchants, mercenaries, or travelers.

% \subsection*{Marriage and Allegiance}

% \subsection*{Racial Views}
% The great majority of Gronselar's populace are thulkraka irds.
% Only irds can become eads, and it is very rare for a member of another kin to be seen as an avatar.
% Despite this, all other political roles can be taken by any member of any kin.

% Apart from irds, the population of Gronselar is composed mainly of gats, marsets, and oths.
% While they technically have the same rights as irds, there is a palpable social stigma against them in most eadsbeers.
% This causes a very uneven distribution of the population, and it historically led to the founding of Gronselar's neighbors, Froibias and Glameas.


% !TEX root = ../main.tex
\subsection*{Brutal Criticals} \label{ssec::brutalcriticals}

Presented here are optional rules that expand the scope of critical hits.
Introducing a series of critical hit charts that vary based on the damage type of the attack, this module adds an additional element of uncertainty, suspense, and surprise to combat.

\subsubsection{Revisions to Critical Hits}
    \begin{itemize}
        \item When you land a critical hit on a creature, roll the damage dice twice and add them together as normal.
        Then, roll a d20 and use the corresponding result on the critical hit chart determined by the damage type of your attack.
        \item When you score a critical hit with an attack that does two or more types of damage, choose one of those damage types and roll on that critical chart.
        \item Traits and feats like Savage Attacks and Brutal Critical continue to work as written.
    \end{itemize}

\subsubsection{Injuries \& Insanity}
    Critical hits can cause minor and major injuries.
    When a creature suffers a minor injury it can be healed by someone competent with a healer's kit over the course of a long rest.

    Major injuries and Insanities will not heal on their own, and count as a new Flaw (see page \pageref{ssec::flaws}) when they are applied.
    The text of certain injuries will specify other ways this injuries might resolve.

\subsubsection{Damage Charts}
    \begin{DndTable}[width=\linewidth, header=Bludgeoning]{cX}
        \textbf{d20} & \textbf{Effect} \\
                   1 & No extra effect. \\
                 2-4 & Until the start of your next turn, the next attack against the creature has advantage. \\
                 5-8 & Push the creature up to 15 feet in any direction. \\
                9-12 & The creature is knocked prone. \\
               13-16 & Push the creature up to 15 feet away.
                       The creature is knocked prone. \\
               17-18 & Roll on the minor injury chart.
                       If the creature is wearing heavy armor, roll on the major injury chart instead. \\
                  19 & Roll on the major injury chart. \\
                  20 & The creature is stunned until the end of your next turn.
                       Roll on the major injury chart.
    \end{DndTable}

    \begin{DndTable}[width=\linewidth, header=Piercing]{cX}
        \textbf{d20} & \textbf{Effect} \\
                   1 & No extra effect. \\
                 2-4 & Until the start of your next turn, the creature's attack rolls are made with disadvantage. \\
                 5-8 & Do not roll your damage dice, instead deal the maximum result possible with those dice. \\
                9-12 & Roll on the minor injury chart with disadvantage. \\
               13-16 & Roll on the minor injury chart. \\
               17-18 & Roll on the major injury chart with disadvantage. \\
                  19 & Roll on the major injury chart. \\
                  20 & Roll once on the minor injury chart and once on the major injurt chart.
    \end{DndTable}

    \begin{DndTable}[width=\linewidth, header=Slashing]{cX}
        \textbf{d20} & \textbf{Effect} \\
                   1 & No extra effect. \\
                 2-4 & The creature loses 1d6 hit points at the start of its next turn. \\
                 5-8 & The creature is bleeding.
                       For the next minute the creature loses 1d4 damage at the start of each of its turns until it uses an action to staunch this wound. \\
                9-12 & The creature is bleeding.
                       For the next minute the creature loses 1d8 hit points at the start of each of its turns until it uses an action to staunch this wound. \\
               13-16 & The creature is bleeding.
                       For the next minute the creature loses 1d12 hit points at the start of each of its turns until it uses an action to staunch this wound. \\
               17-18 & Roll on the minor injury chart.
                       If the creature is wearing light or no armor, roll on the major injury chart instead \\
                  19 & Roll on the major injury chart. \\
                  20 & Roll on the major injury chart.
                       The creature is bleeding.
                       For the next minute the creature loses 2d8 hit points at the start of each of its turns until it uses an action to staunch this wound.
    \end{DndTable}

    \begin{DndTable}[width=\linewidth, header=Slashing]{cX}
        \textbf{d20} & \textbf{Effect} \\
                   1 &  \\
                 2-4 &  \\
                 5-8 &  \\
                9-12 &  \\
               13-16 &  \\
               17-18 &  \\
                  19 &  \\
                  20 &
    \end{DndTable}

\subsection*{Dementia} \label{ssec::dementia}
\DndDropCapLine{H}{arsh is the process that awaits the}
foolish or unlucky enough to lose their qualar.
Their sentience slips away slowly as they lose their mental capacities.
Perhaps the worst part is that inward awareness is one of the last attributes lost, forcing them to be fully conscious of the process.

The road towards dementia comes in seven stages, each roughly lasting a week.
At the moment when you lose your qualar, you enter the first stage of dementia.
After a week, you roll a DC 15 Intelligence saving throw.
On a fail, you enter the next stage of dementia.
On a success, you stay on your current stage and roll again at the start of the next week.
Dementia is unavoidable, and the DC increases by 1 after every successful roll.

\subsubsection{First Stage}
No obvious signs of dementia, only minor short-memory loss occurs.
The main symptoms are associated to the anxiety from the loss of the qualar.
You start focusing more on your past, often drifting away into daydreams.

You suffer the following effects:
\subparagraph{Decreased Awareness} You roll for initiative and Dexterity saving throws with disadvantage.
\subparagraph{Restlessness} Roll an DC 12 intelligence saving throw right after a short rest.
On a success, you recover your hit points normally.
On a failure, you only recover half of the hit dice rolled (rounded down).

\subsubsection{Second Stage}
The self realization and awareness that something is wrong settles in.
You refuse to accept that your mind is slipping away.
The more effort you put on remembering the more deterioration your memory suffers.
Confusion starts setting in.

In addition to the effects of the first stage, you suffer the following effects:
\subparagraph{Lack of Recollection} Any ability check made to remember or recollect a memory is done with disadvantage.
All Intelligence (History) checks are made with disadvantage.
\subparagraph{Mood Swings} All Charisma ability checks and saving throws are made with disadvantage.

\subsubsection{Third Stage}
You experience increased forgetfulness and might find concentrating difficult.
You are presented with some of the last coherent memories before confusion fully rolls in.
Some singular memories become more disturbed, isolated, broken, and distant.
These are the last embers of awareness.

In addition to the effects of the last stages, you suffer the following effects:
\subparagraph{Decreased Concentration} All Constitution saving throws made to maintain concentration are made with disadvantage.
\subparagraph{Vagrant Mind} Intelligence and Wisdom ability checks and saving throws are rolled with disadvantage.

\subsubsection{Fourth Stage}
Grey mists form and fade away in your memory.
The ability to recall singular memories gives way to confusion and horror.
You struggle with daily tasks, presenting clear cognitive problems.

In addition to the effects of the last stages, you suffer the following effects:
\subparagraph{Motor Difficulties} Strength and Dexterity ability checks and saving throws are rolled with disadvantage.
Attack rolls are made with disadvantage.
\subparagraph{Drifting Conscience} In combat, roll a DC 8 Intelligence saving throw at the start of every turn.
On a failure, you forget where you are, and cannot take any actions during the turn.

\subsubsection{Fifth Stage}
You have major memory deficiencies.
The few lapses of consciousness you get are filled with dread, as you realize your mind has mostly left you.
The repetition and rupture gives way to calmer moments, as the unfamiliar becomes familiar.

In addition to the effects of the last stages, you suffer the following effects:
\subparagraph{Loss of Fortitude} All rolls are made with disadvantage.
\subparagraph{Complete Unawareness} All Intelligence (Investigation), Wisdom (Insight), and Wisdom (Perception) automatically fail.
Your passive investigation, insight, and perception become 0.

\subsubsection{Sixth Stage}
Your mental state is beyond description.
You struggle to remember your early life, even forgetting the names of your family and fellows.
Your capacity of speech is severely impaired, and you suffer sudden and radical personality and emotional changes.

In addition to the effects of the last stages, you suffer the following effects:
\subparagraph{Declining Speech} You automatically fail all Charisma ability checks and saving throws, and have a very hard time communicating verbally.
\subparagraph{Hope Lost} As your conscious mind fights your natural tendencies, you automatically fail death saving throws.

\subsubsection{Final Stage}
Everywhere, an empty bliss.

You make what will most likely be your last die roll.
Nothing can give you advantage or disadvantage on this roll.
Roll a d20 on the post-dementia table.
You suffer the effect rolled.

\begin{DndTable}[width=\linewidth, header=Post-dementia Effects]{lX}
    \textbf{Roll} & \textbf{Effect} \\
    1 & Your last emotion is rage.
    You become mindless and violent, attacking your companions without reason. \\
    2-5 & Your last emotion is apathy.
    You mindlessly stare to the east until you die of natural causes. \\
    6-18 & Even after brain death, you seek survival.
    You wander off, attacking any who try to stop you.
    You become a lost one. \\
    19 & With memories gone, your mind is fully pulled by its tidal impulses.
    You lose control of your character, who becomes a servant of The Sorrow. \\
    20 & Recovery from dementia is very rare, but not impossible.
    You recover from all effects associated to dementia, and gain the \textbf{Demented Insight} feat at page \pageref{feat::dementedinsight}.
\end{DndTable}

The only conventional way to remove dementia is to obtain a qualar again.
When this happens, you quickly recover your sentience.
Each morning, you reduce your stage of dementia by one, and don't need to roll the Intelligence saving throw associated to the status.

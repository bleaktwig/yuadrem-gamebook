% !TEX root = ../main.tex
\chapter{Mechanic Changes}
\begin{linenumbers}
\DndDropCapLine{S}{trans may break alone, but twisted}
\textit{make a braid.
All are not the same, but they shall be as one.}

\hspace*{\fill} --- Anonymous song.

This book includes major mechanical changes.
These mostly focus on allowing players to embrace the worldbuilding aspects of Yuadrem by freeing up their character building options.
These changes are of course optional, but most of the rules in the book were designed with them in mind.

Then, optional rules are included.
Yuadrem is a harsh continent, and these rules help set the atmosphere to reflect this.
They are completely optional, and a healthy gaming table should discuss which should be included or excluded in their game.

% \begin{DndComment}{Help Wanted!}
%     This entire book was designed with the major mechanical changes considered, yet the author realizes that this can alienate a large part of the community.
%     You may want to play in Yuadrem without these changes.
%     If this is the case, it is recommended that change many mechanical aspect of the book to acommodate this.
%
%     If for some insane reason you want to take on this challenge, you are encouraged to contact the book's author.
%     We could work together on this, discussing appropiate retribution and co-authorship terms.
% \end{DndComment}

\section{3-Action System}
Instead of the movement, action, and bonus action system of D\&D 5e, this book uses Pathfinder 2e's 3-action system.
Simply put, you can use up to three actions in a turn.
Additionally, free actions and reactions are available as usual.

The actions available to you are:
\subsubsection{Attack}
    \textit{Perform a melee or ranged attack with your weapon.}

    You can use more than one Attack action on your turn, but each additional attack after the first becomes less accurate.
    This is reflected by a multiple attack penalty that starts at -5 on the second attack, and increases to -10 on the third.
    This penalty resets at the end of your turn.

    Some conditions give advantage on the attack: attacks against blinded, paralyzed, petrified, restrained, stunned, or unconscious targets; melee attacks against prone targets; attacks by invisible or hidden attackers.

    Some conditions give disadvantage on the attack: attacks against invisible or hidden targets; ranged attacks against prone targets; attacks by blinded, frightened, poisoned, or restrained attackers.
\subsubsection{Cast a Spell}
    \textit{Cast a spell with a casting time of 1 or more actions.}

    The target of a spell must be within the spell's range.
    To target something, you must have a clear path to it, so it can't be behind total cover.

    Spells with material components do not consume the material unless explicitly stated.
    Unless the cost of a material is given, you can assume that the cost is negligible and the material is simply available in a component pouch.

    Some spells require you to maintain concentration in order to keep their magic active.
    If you lose concentration, such a spell ends.
    You lose concentration on a spell if you cast another spell that requires concentration or when you are incapacitated.
    Each time you take damage, you must make a Constitution saving throw to maintain your concentration.
    The DC equals 10 or half the damage you take, whichever number is higher.
\subsubsection{Climb onto a Bigger Creature}
    \textit{Hop atop your target.}

    You can treat a suitably large opponent as terrain for the purpose of jumping onto its back or clinging to a limb.
    After making any ability checks necessary to get into position and onto the larger creature, you use an action to make a Strength (Athletics) or Dexterity (Acrobatics) check contested by the target's Dexterity (Acrobatics) check.
    If you win the contest, you successfully move into the target creature's space.
    You move with the target and have advantage on attack rolls against it.

    You can move around within the larger creature's space, treating it as difficult terrain.
    The larger creature's ability to attack you depends on your location, and is left to the DMs discretion.
    The larger creature can dislodge you as an action—knocking you off, scraping you against a wall, or grabbing and throwing you—by making a Strength (Athletics) check contested by the your Strength (Athletics) or Dexterity (Acrobatics) check.
    You choose which ability to use.
\subsubsection{Disarm}
    \textit{Force your opponent to unequip their weapon.}

    You can use an action to knock a weapon or another item from a target's grasp.
    You make an attack roll contested by the target's Strength (Athletics) check or Dexterity (Acrobatics) check.
    If you win the contest, the attack causes no damage or other ill effect, but the defender drops the item.

    You have disadvantage on the attack roll if your target is holding the item with two or more hands.
    The target has advantage on its ability check if it is larger than you, or disadvantage if it is smaller.
\subsubsection{Disengage}
    \textit{Your movement doesn't provoke opportunity attacks for the rest of the turn.}
\subsubsection{Dodge (2 Actions)}
    \textit{Focus entirely on avoiding attacks.}

    Until the start of your next turn, any attack roll made against you has disadvantage if you can see the attacker, and you make Dexterity saving throws with advantage.

    You lose this benefit if you are incapacitated or if your speed drops to 0.
\subsubsection{Equip Shield}
    \textit{Equip or unequip a shield.}

    A shield always takes an action to equip or unequip.

    Armor takes several minutes to equip or unequip.
\subsubsection{Escape}
    \textit{Escape a grapple.}

    To escape a grapple, you must succeed on a Strength (Athletics) or Dexterity (Acrobatics) check contested by the grappler's Strength (Athletics) check.

    Escaping other conditions that restrain you (such as manacles) may require a Dexterity or Strength check, as specified by the condition.
\subsubsection{Grapple}
    \textit{Attempt to grab a creature or wrestle with it.}

    The target of your grapple must be no more than one size larger than you, and it must be within your reach.

    Using at least one free hand, you try to seize the target by making a grapple check, a Strength (Athletics) check contested by the target's Strength (Athletics) or Dexterity (Acrobatics) check (the target chooses the ability to use).

    If you succeed, you subject the target to the grappled condition (its speed is set to 0).
\subsubsection{Help}
    \textit{Grant an ally advantage on an ability check or attack.}

    The target gains advantage on the next ability check it makes to perform the task you are helping with.

    Alternatively, the target gains advantage on the next attack roll against against a creature within 5 feet of you.

    The advantage lasts until the start of your next turn.
\subsubsection{Hide}
    \textit{Attempt to hide.}

    You can't hide from a creature that can see you.
    You must have total cover, be in a heavily obscured area, be invisible, or otherwise block the enemy's vision.

    If you make noise (such as shouting a warning or knocking over a vase), you give away your position.

    When you try to hide, make a Dexterity (Stealth) check and note the result.
    Until you are discovered or you stop hiding, that check's total is contested by the Wisdom (Perception) check of any creature that actively searches for signs of your presence.

    A creature notices you even if it isn't searching unless your Stealth check is higher than its Passive Perception.

    Out of combat, you may also use a Dexterity (Stealth) check for acts like concealing yourself from enemies, slinking past guards, slipping away without being noticed, or sneaking up on someone without being seen or heard.
\subsubsection{Identify a Spell}
    \textit{Attempt to understand a casted spell.}

    Sometimes you may want to identify a spell that someone else is casting or that was already cast.
    To do so, you can use your reaction to identify a spell as it's being cast, or use an action on your turn to identify a spell by its effect after it is cast.

    If you perceived the casting, the spell's effect, or both, you can make an Intelligence (Arcana) check with the reaction or action.
    The DC equals 15 + the spell's level.
    If the spell casted is from a school of magic in which you have competence, the check is made with advantage.
    Some spells aren't associated with any school, such as when a monster uses its Innate Spellcasting trait.

    This Intelligence (Arcana) check represents the fact that identifying a spell requires a quick mind and familiarity with the theory and practice of casting.
    This is true even for a character whose spellcasting ability is Wisdom or Charisma.
    Being able to cast spells doesn't by itself make you adept at deducing exactly what others are doing when they cast their spells.
\subsubsection{Improvise}
    \textit{Perform any action you can imagine.}

    When you describe an action not detailed elsewhere in the rules, the DM tells you whether that action is possible and what kind of roll you need to make, if any, to determine success or failure.
\subsubsection{Move}
    \textit{Move up to your movement speed.}

    If you have more than one speed, such as your walking speed and a flying speed, you can switch back and forth between your speeds during your move.
    Whenever you switch, subtract the distance you've already moved from the new speed.

    You can move through a nonhostile creature's space.
    You can move through a hostile creature's space only if the creature is at least two sizes larger or smaller than you.
    Another creature's space is difficult terrain for you.
    Whether a creature is a friend or an enemy, you can't willingly end your move in its space.

    Climbing, swimming, and crawling cost an additional 1.5 m per 1.5 m travelled, unless you have an specific climbing or swimming speed.

    This movement can include a high jump and/or a long jump, following the normal rules for each.
\subsubsection{Overrun}
    \textit{Force your way through a creature's space.}

    When you try to move through a hostile creature's space, you try to force your way through by overrunning the hostile creature.
    As an action, you make a Strength (Athletics) check contested by the hostile creature's Strength (Athletics) check.
    You have advantage on this check if you are larger than the hostile creature, or disadvantage if you are smaller.
    If you win the contest, you can move through the hostile creature's space once this turn.
\subsubsection{Ready}
    \textit{Choose a trigger and a response reaction.}

    First, you decide what perceivable circumstance will trigger your reaction.

    Then, you choose the action you will take in response to that trigger, or you choose to move up to your speed in response to it.

    When the trigger occurs, you can either take your reaction right after the trigger finishes or ignore the trigger.

    When you ready a spell, you cast it as normal but hold its energy, which you release with your reaction when the trigger occurs.
    To be readied, a spell must have a casting time of 1 action, and holding onto the spell's magic requires concentration.
\subsubsection{Search}
    \textit{Devote your attention to finding something.}

    Depending on the nature of your search, the DM might have you make a Wisdom (Perception) check or an Intelligence (Investigation) check.
\subsubsection{Shove}
    \textit{Shove a creature, either to knock it prone, push it away from you, or push it to the side.}

    The target of your shove must be no more than one size larger than you, and it must be within your reach.

    You make a Strength (Athletics) check contested by the target's Strength (Athletics) or Dexterity (Acrobatics) check (the target chooses the ability to use).

    If you win the contest, you either knock the target prone or push it 5 feet away from you.
\subsubsection{Stabilize a Creature}
    \textit{Stop a dying creature from needing to make death saving throws.}

    Make a Wisdom (Medicine) check with DC 10.

    On a success, the creature is stable and no longer needs to make death saving throws.

    A stable creature regains 1 hit point after 1d4 hours.
\subsubsection{Stand Up}
    \textit{End the prone condition on yourself by standing up.}
\subsubsection{Tumble}
    \textit{Weave past a hostile creature.}

    You can try to tumble through a hostile creature's space, ducking and weaving past the opponent.
    As an action, you make a Dexterity (Acrobatics) check contested by the hostile creature's Dexterity (Acrobatics) check.
    If you win the contest, you can move through the hostile creature's space once this turn.
\subsubsection{Use Object}
    \textit{Interact with a second object or use special object abilities.}

    You can interact with one object for free during your turn (such as drawing a weapon or opening a door).
    If you want to interact with a second object, use this action.

    When an object requires your action for its use, you also take this action.

\hrulefill % TODO: Seriously improve this separator.

The reactions available to you are:
\subsubsection{Cast a Spell}
    \textit{Cast a spell with a casting time of 1 reaction.}

    Trigger: specified by the spell.

    For further details, see the Cast a spell action.
\subsubsection{Readied Action}
    \textit{Execute the reaction specified by your Ready action.}

    Trigger: specified by your Ready action.
\subsubsection{Opportunity Attack}
    \textit{You can rarely move heedlessly past your foes without putting yourself in danger.}

    Trigger: enemy creature leaves your reach.

    Make one melee attack against the provoking creature.

    The attack interrupts the provoking creature's movement, occurring right before the creature leaves your reach.

    Creatures don't provoke an opportunity attack when they teleport or when someone or something moves them without using their movement, action, or reaction.

    Additional conditions for opportunity attacks are included as an optional rule in page \pageref{rule::pportunityattacks}.


\section{Classless D\&D}

% \section{Dementia}
% \DndDropCapLine{H}{arsh is the process that awaits the}
% foolish or unlucky enough to lose their qualar.
% Their sentience slips away slowly as they lose their mental capacities.
% Perhaps the worst part is that inward awareness is one of the last attributes lost, forcing them to be fully conscious of the process.
%
% The road towards dementia comes in seven stages, each roughly lasting a week.
% At the moment when you lose your qualar, you enter the first stage of dementia.
% After a week, you roll a DC 15 Intelligence saving throw.
% On a fail, you enter the next stage of dementia.
% On a success, you stay on your current stage and roll again at the start of the next week.
% Dementia is unavoidable, and the DC increases by 1 after every successful roll.
%
% \subsubsection{First Stage}
% No obvious signs of dementia, only minor short-memory loss occurs.
% The main symptoms are associated to the anxiety from the loss of the qualar.
% You start focusing more on your past, often drifting away into daydreams.
%
% You suffer the following effects:
% \subparagraph{Decreased Awareness} You roll for initiative and Dexterity saving throws with disadvantage.
% \subparagraph{Restlessness} Roll an DC 12 intelligence saving throw right after a short rest.
% On a success, you recover your hit points normally.
% On a failure, you only recover half of the hit dice rolled (rounded down).
%
% \subsubsection{Second Stage}
% The self realization and awareness that something is wrong settles in.
% You refuse to accept that your mind is slipping away.
% The more effort you put on remembering the more deterioration your memory suffers.
% Confusion starts setting in.
%
% In addition to the effects of the first stage, you suffer the following effects:
% \subparagraph{Lack of Recollection} Any ability check made to remember or recollect a memory is done with disadvantage.
% All Intelligence (History) checks are made with disadvantage.
% \subparagraph{Mood Swings} All Charisma ability checks and saving throws are made with disadvantage.
%
% \subsubsection{Third Stage}
% You experience increased forgetfulness and might find concentrating difficult.
% You are presented with some of the last coherent memories before confusion fully rolls in.
% Some singular memories become more disturbed, isolated, broken, and distant.
% These are the last embers of awareness.
%
% In addition to the effects of the last stages, you suffer the following effects:
% \subparagraph{Decreased Concentration} All Constitution saving throws made to maintain concentration are made with disadvantage.
% \subparagraph{Vagrant Mind} Intelligence and Wisdom ability checks and saving throws are rolled with disadvantage.
%
% \subsubsection{Fourth Stage}
% Grey mists form and fade away in your memory.
% The ability to recall singular memories gives way to confusion and horror.
% You struggle with daily tasks, presenting clear cognitive problems.
%
% In addition to the effects of the last stages, you suffer the following effects:
% \subparagraph{Motor Difficulties} Strength and Dexterity ability checks and saving throws are rolled with disadvantage.
% Attack rolls are made with disadvantage.
% \subparagraph{Drifting Conscience} In combat, roll a DC 8 Intelligence saving throw at the start of every turn.
% On a failure, you forget where you are, and cannot take any actions during the turn.
%
% \subsubsection{Fifth Stage}
% You have major memory deficiencies.
% The few lapses of consciousness you get are filled with dread, as you realize your mind has mostly left you.
% The repetition and rupture gives way to calmer moments, as the unfamiliar becomes familiar.
%
% In addition to the effects of the last stages, you suffer the following effects:
% \subparagraph{Loss of Fortitude} All rolls are made with disadvantage.
% \subparagraph{Complete Unawareness} All Intelligence (Investigation), Wisdom (Insight), and Wisdom (Perception) automatically fail.
% Your passive investigation, insight, and perception become 0.
%
% \subsubsection{Sixth Stage}
% Your mental state is beyond description.
% You struggle to remember your early life, even forgetting the names of your family and fellows.
% Your capacity of speech is severely impaired, and you suffer sudden and radical personality and emotional changes.
%
% In addition to the effects of the last stages, you suffer the following effects:
% \subparagraph{Declining Speech} You automatically fail all Charisma ability checks and saving throws, and have a very hard time communicating verbally.
% \subparagraph{Hope Lost} As your conscious mind fights your natural tendencies, you automatically fail death saving throws.
%
% \subsubsection{Final Stage}
% Everywhere, an empty bliss.
%
% You make what will most likely be your last die roll, this time with no disadvantage.
% Roll a d20 on the post-dementia table.
% You suffer the effect rolled.
%
% \begin{DndTable}[width=\linewidth, header=Post-dementia Effects]{lX}
%     \textbf{Roll} & \textbf{Effect} \\
%     1 & Your last emotion is rage. You become mindless and violent, attacking your companions without reason. \\
%     2-5 & Your last emotion is apathy. You endlessly stare at the east until you die of natural causes. \\
%     6-18 & Even after brain death, you seek survival. You wander off, attacking any who try to stop you. You become a lost one. \\
%     19 & With memories gone, your mind is fully pulled by its tidal impulses. You lose control of your character, who becomes a servant of The Sorrow. \\
%     20 & Go back to the first stage of dementia, and gain a rank in the Dementia Insight heroic talent.
% \end{DndTable}
%
% The only conventional way to remove dementia is to obtain a qualar again.
% When this happens, you go back one stage of dementia per day passed.
% While in this state you suffer a terrible fever.
% At the beginning of each day roll a Constitution saving throw.
% On a failure, you gain one level of exhaustion.
%
% \section{Hardcore Rules}
% \subsection*{Critical Hits}
%
% \subsection*{Encumbrance}
%
% \subsection*{Hunger and Thirst}
%
% \subsection*{Minor and Major Injuries}
%
% \subsection*{Only One Chance}
% As a rule, you only have one chance to succeed in any action.
% Once you have rolled the dice, you many not roll again to achieve the same goal.
% You need to try something different or wait until the circumstances have changed in a substantial way.
% Or let another PC try.
%
% This rule does not apply to combat, where you can attack the same enemy until it becomes a bloody pulp.
%
% \subsubsection*{Opportunity Attacks} \label{rule::opportunityattacks}
% In a fight, everyone is constantly watching for enemies to drop their guard.
% While heedless movement is easily punished, any lack of concentration or unplanned action can have devastating consequences.
% Various actions provoke opportunity attacks from a creature:
% \begin{itemize}
%     \item Moving out of the creature's reach.
%     This doesn't apply when someone or something moves you without using your movement, action, or reaction.
%     \item Picking up an item from the floor while in the creature's reach.
%     \item Unsheathing a weapon while in the creature's reach.
%     \item Donning or doffing a shield while in the creature's reach.
%     \item Using the Attack action with a ranged weapon in a creature's reach.
% \end{itemize}
% You can avoid opportunity attacks during your turn by taking the Disengage action.
%
% \subsection*{Short and Long Rests}
%
% \subsection*{Stamina}

\end{linenumbers}

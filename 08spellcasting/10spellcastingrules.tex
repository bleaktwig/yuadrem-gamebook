% !TEX root = ../main.tex
\section{Spellcasting Rules} \label{sec::spellcastingrules}
\subsection*{Known Spells}
    As a user of magic, you know a number of spells which you can use in any scenario you find appropiate.
    The maximum number of spells you can have prepared at a time is equal to your spellcasting ability modifier + half your level (rounded down).
    If you add a new spell when you already have the maximum number of spells prepared, you forget one spell of your choice.
    To keep these spells available, you can use a spellbook (see page \pageref{item::spellbook}).

    When you take the spellcaster feat (page \pageref{feat::spellcaster}), you have access to one spell and three cantrips --- one from your doctrine, one from your school, and one of your choice.
    You learn one additional cantrip when you reach levels 4 and 10.

    You can learn one new spell as part of a long rest.
    This spell must belong to a spellcasting doctrine available to you.

    \begin{DndTable}[width=\linewidth, header=Spellcasting Ability]{llll}
        \textbf{Level} & \textbf{Cantrips Known} & \textbf{Spell Points} & \textbf{Max Spell Level} \\
         1st &    3 &     2 &    1st \\
         2nd &    3 &     3 &    1st \\
         3rd &    3 &     7 &    1st \\
         4th &    4 &     8 &    1st \\
         5th &    4 &    13 &    2nd \\
         6th &    4 &    16 &    2nd \\
         7th &    4 &    19 &    2nd \\
         8th &    4 &    22 &    2nd \\
         9th &    4 &    28 &    3rd \\
        10th &    5 &    32 &    3rd \\
        11th &    5 &    36 &    3rd \\
        12th &    5 &    36 &    3rd \\
        13th &    5 &    41 &    4th \\
        14th &    5 &    41 &    4th \\
        15th &    5 &    47 &    4th \\
        16th &    5 &    47 &    4th \\
        17th &    5 &    53 &    5th \\
        18th &    5 &    57 &    5th \\
        19th &    5 &    61 &    5th \\
        20th &    5 &    66 &    5th
    \end{DndTable}

\subsection*{Spell Points}
    Your character uses spell points to fuel spells.
    Each spell has a point cost based on its level.
    The spell point cost table summarizes the cost in spell points of spells from 1st to 9th level.
    Cantrips don't require slots and therefore don't require spell points.

    Spells of 6th level and higher are particularly taxing to cast.
    You can use spell points to cast one spell of each level of 6th or higher.
    You can't cast another spell of the same level until you finish a short rest.
    % TODO: Remove this restriction as a feat.

    The number of spell points you have to spend is based on your level, as shown in the spellcasting ability table.
    Your level also determines the maximum-level spell you can cast.
    Even though you might have enough points to create a slot above this maximum, you can't do so.

    \begin{DndTable}[width=\linewidth, header=Spell Point Cost]{ll}
        \textbf{Spell Level} & \textbf{Point Cost} \\
        1st &  2 \\
        2nd &  3 \\
        3rd &  5 \\
        4th &  6 \\
        5th &  7 \\
        6th &  9 \\
        7th & 10 \\
        8th & 11 \\
        9th & 13
    \end{DndTable}

\subsection*{Spellcasting Ability}
    When you take the spellcaster feat, you decide a spellcasting ability, which can be Intelligence, Wisdom, or Charisma.
    This choice reflects your medium for casting spells: Conceptualization for Intelligence, volition for Wisdom, and empathy for Charisma.

    You use this ability score whenever a spell refers to your spellcasting ability.
    In addition, you use its modifier when setting the saving throw DC for a spell you cast and when making an attack roll with one.

    \subparagraph{Spell save DC} = 8 + your spellcasting proficiency bonus + your chosen ability modifier
    \subparagraph{Spell attack modifier} = your spellcasting proficiency bonus + your chosen ability modifier

    After taking the spellcaster feat, your spellcasting proficiency bonus is +2.
    You can increase this modifier further by taking the avid spellcaster feat (see page \pageref{feat::avidspellcaster}).

\subsection*{Spellcasting Focus}
    You can use a spellcasting focus to replace the material components for your spells.
    This focus is related to your magic school, and is explained in its section.

\newpage % TODO. Replace for some fancy image.

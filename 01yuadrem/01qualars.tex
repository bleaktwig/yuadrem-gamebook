% !TEX root = ../main.tex
\subsection*{Qualars} \label{ssec::qualars}
\addcontentsline{toc}{section}{Qualars \& The Tides}

\DndDropCapLine{I}{n Yuadrem, sentience is not an ability}
to grow spontaneously in a being, but is related to a certain object: a qualar.
Qualars, or rluthe qual'yiz in jantherlin, are small bone totems filled with a strange black liquid named qual ichor.
All the peoples of Yuadrem need to keep a qualar close to them, lest they lose their sentience, devolving into ``lost ones''.
Lost ones are incapable to perceive subjectively, losing their sense of self.

Most qualars are made by one of the only surviving ets, a gargantuan being named Lua'Zheshal Qual'yiz Cter-Rheth, or Ctereth.
With bones provided by cultists and its own blood as qual ichor, the tall one tirelessly crafts the totems.

In oldentimes, qualars were assumed to be a limitation created by the tall kin to enslave their children.
However, when it was learned that the spontaneous kins like naenks, tsaneks, and zaloths still needed qualar, the hypothesis had to be quickly discarded.
Further confusing matters are the foreigner kins, who only started needing qualar after arriving at Yuadrem.

The sudden need for a qualar led each of the foreigner kins into two groups.
The first, who raided Ctereth and gained their qualars through effort.
And the second, who quickly lost their sentience and devolved into beasts.

While none can match Ctereth's handiwork, a few master bone carvers can craft imitations that replicate a qualar's properties, provided they can obtain the qual ichor to fill the totems.
Still, merely due to the qualars' value, it is common for the brave to attempt to steal the objects from Ctereth, endeavor that most usually results in new additions to its bone pile.

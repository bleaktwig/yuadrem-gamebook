% !TEX root = ../main.tex
\subsection*{Religions} \label{ssec::religions}

\DndDropCapLine{R}{eligion is an important part of life}
of the many cultures of Yuadrem.
Some worship specific pantheons of gods, others praise unpersonified concepts, and a selected few worship nature itself.
% In the times before the schism there was a wide belief that the tall kin could answer prayers, but their worship is now forbidden in most of the continent.

% The true existence of these divinities is a widely discussed subject, but their worship is undeniable.
From the nature-worshiping folk of Jenkash to the god-birds of Krudzal, each culture performs a set of rituals in the name of their deities, and some even claim to be able to channel their divine power.
While it might be hard to pinpoint the exact number of religions in Yuadrem, a few are built into the fabric of civilizations, and are easy to tell apart.

\subsubsection{Igneism}
The oth scholars hailing from Ignelli were the first to conceptualize the tides.
They learned that the tides are intrinsically woven into sentience.
As tightly tied threads between sentient creatures, a change in the tidal alignment in one has profound effects in that of those nearby.
% Each tide was then assosiated with a symbol and an entity, to give a more concrete face to it and facilitate its worship.

The blue tide is represented by a blue unfinished book, representing the eternal pursuit of knowledge.
The gold tide is symbolized as a seed or an egg, telling of the coming of future life with proper nourishment.
Indicated by a torch, the indigo tide tells of the truth revealed under light, and the punishment exerted upon those who hide it.
A broken compass represents the red tide, representing the roaming of those who walk without roads.
The silver tide's symbol is a bell, denoting the attention obtained by those who seek fame.

Nearing the year 247 AS, the knowledge of the tides split the Ignelli school in two - The Igneists and the Rashiists.
% The Zelseists, who simply sought to further understand this new discovery, and the Rashiists, who attempted to wield and manipulate them.
The latter created a system of magic known as Rashid, with which they could potentiate the tidal alignment of others.
Despite the warnings issued by their sister school, the Rashiists honed their craft to its maximal potency, and suffered severely from it.
Their actions awoke a strange an antique and mysterious creature: The Sorrow.

% The Sorrow is a being of indescribable shape who was summoned to Yuadrem by the Rashiists' folly.
Breaking the mind of any who lay their eyes upon it, it swiftly took the lives of all who corrupted the balance of the tides, thus ending the Rashiist school's folly.
Its precense caused the pale blemish, and with it came horrid creatures known now as the xuagra.
Seeing the destruction caused by their sister school, the Igenists hid the knowledge of their former brethren, forbidding Rashid in any shape or form.
% Finally, they changed their own name to Igneists in an attempt to bury the other school in anonymity.

Igneism is the worship of the tides as a concept, and the active pursuit of keeping the five balanced.
Igneists recognize that sentience cannot exist without the tides, and praise them as thanks for the capacity of independent thought.

\subsubsection{Tanethism}
Across the long history of the Seven Kingdoms, varied deities were gradually associated with many different concepts, and any gat would pray to different gods at different times and circumstances.
For example, one might say a prayer to the Traveler for luck, make an offering to Tamaz before going to the market, and pray to appease Matevos when a severe storm blows in --- all in the same day.

Independent to any institution, the gat scholar Taneth officially published ``The Rituals and Gods of the Gats'' in 511 AS.
The book was an exhaustive compilation of the more than a thousand deities that were praised by the different kins of the Seven Kingdoms, and proposed a reduced pantheon of 15, distilling the chaotic pantheon into the main gods.
Taneth's work was barely known during their life, and they died without due recognition.

Later, in the year 577 AS, the king of Khedrat Olag the Immortal seeked a method to reduce the religious disparity among their people.
Either by divine will or happenstance, they came upon this book by Taneth, and established Tanethism as the official religion of Khedrat, imitating the well-established creed of other nations.
By command of the government, many churches were erected in the name of each god, and the many gods of the gats coalesced into a more sober pantheon of 15.

Many have a favorite among the gods, one whose ideals and teaching they make their own.
A few even dedicate entirely to a single deity, serving as a priest, acolyte, or champion of that god's image.
Famous among these devout beings are the nimrod, an organization of zealous hunters of Phusinhe who pursue all who disturb the balance of Yuadrem.
Well-known as well are the followers of Havetish, a group of gats in golden robes whose goal is to distribute wealth and food to the impoverished hamlets of the inner regions of the Seven Kingdoms.

\begin{table*}[b]%
    \begin{DndTable}[width=\linewidth, header=The Gods of Yuadrem]{p{2cm}p{0.8cm}p{3cm}p{1.8cm}l}
        \textbf{Name} & \textbf{Tides} & \textbf{Domains} & \textbf{Religion} & \textbf{Symbol} \\
        The Scholar  & B  & Reason, Knowledge     & Igneism   & A many-armed blue oth reading multiple books. \\
        The Zealous  & R  & Passion, Zeal         & Igneism   & A red zsek ird standing over a sand dune. \\
        The Star     & S  & Admiration, Fame      & Igneism   & A naked tall one, sometimes replaced by a shadow or a uman. \\
        The Equalist & I  & Justice, Equity       & Igneism   & An indigo gat holding a spear and a coin. \\
        The Altruist & G  & Empathy, Compassion   & Igneism   & A furtive golden marset carrying a basket full of eggs. \\
        The Sorrow   & -  & Balance, Punishment   & Igneism   & An indistinct cloaked figure holding a bloody heart. \\
        Changing God & -  & Secrecy, Manipulation & Rashiism  & A robed oth with a featureless bronze mask. \\
        Febrid       & B  & Intellect, Wood       & Tanethism & A gat forming a crescent moon with its horns. \\
        The Traveler & BR & Luck, Beer            & Tanethism & An indistinct figure cloaked in light brown robes. \\
        Vugar        & BG & Family, Fertility     & Tanethism & A gat prince dressed in a simple silver toga. \\
        Vahagn       & R  & Mountains, Fire       & Tanethism & A red quies holding a colossal mace. \\
        Genadi       & RI & Bravery, Love         & Tanethism & A grung warrior carrying a sword and a lute. \\
        Sakris       & RS & Fun, Wine             & Tanethism & A uman servant carrying cups and wine. \\
        Matevos      & S  & Glory, Water          & Tanethism & An ice zaloth holding a bident and a shield. \\
        Hanutsh      & SB & Teaching, Books       & Tanethism & A tsanek dressed in scrolls and paper. \\
        Tamaz        & SG & Wealth, Silver        & Tanethism & A gray ird eternally flying towards the sun. \\
        Phusinhe     & I  & The Stars, Metal      & Tanethism & A giant tortle with the visage of stars in its shell. \\
        Nadzim       & IB & Justice, the Sky      & Tanethism & A purple oth holding an abacus and a spyglass. \\
        Gathoz       & IS & Secrecy, Murder       & Tanethism & A kinless being with shifting body and face. \\
        Bagrat       & G  & Farming, Earth        & Tanethism & A gat farmer with tools made of gold. \\
        Havetish     & GI & Leadership, Tyranny   & Tanethism & A naenk holding a golden and an indigo spear. \\
        Mziva        & GR & Self Sacrifice        & Tanethism & A blonde marset with a flowered back. \\
        Jua\~nansiz  & G  & Day, Sunlight         & Tsalemism & A rainbow-colored heron followed by northern lights. \\
        Dzadsiz      & R  & Night, Darkness       & Tsalemism & A black raven surrounded by never-dispersing mists. \\
        The Observer & -  & Cosmos, the Unknown   & Cosmism   & A titanic three-eyed slug ridden with tentacles and appendages.
    \end{DndTable}
\end{table*}

\subsubsection{Tsalemism}
It is indubitable that astral concepts are commonly associated to divinities, and no religion reflects this as clearly as Tsalemism.
Tsalemism is a belief that originally gained popularity in the coasts of Krudzal, quickly becoming the official religion of the nation and of many thulkraka irds.
Due to its proximity to the north pole, Krudzal experiences long polar days and nights every year, and this irregular schedule naturally led to the personification of night and day.

Day is associated to Jua\~nansiz, a rainbow-colored heron that brings daylight and colors to the entirety of the polar region.
Jua\~nansiz eternally hunts Dzadsiz, a black raven who in turn seeks to tire the heron and finally feast on its exhausted body.
The birds' duel is unending, and the wreckage of their battle is used to explain the chaotic fjords in the Northern Territories.

The boreal lights seen near the pole are Jua\~nansiz's trail.
The mountainous landscape of the Whitenorth are the places where Dzadsiz fell, struck by the heron.
% The endless mists were created by the raven in an attempt to hide from Jua\~nansiz.
These and many other natural phenomena of the Northern Territories are explained by the birds and their eternal duel.

The two birds are not worshipped equally, but their wrath is feared by all.
A sailor may produce a small temple to appease Dzadsiz before sailing, and a cartographer may sing a praise to Jua\~nansiz before taking flight.

\subsubsection{Cosmism}
While the worship of the tall ones is forbidden, their own religious ideas persist in the form of Cosmism.
Cosmism is related to the search of one's place in the larger scheme of things, dubbed the ``cosmos''.
The Observer is the manifestation of the elusive concept of this cosmos, an omnipresent god observing all of Yuadrem at once.

The ideas behind the doctrine were originally conceived by the ets in time immemorial, and cosmists are generally met with disdain and criticism.
Due to this, many acolytes of the religion practice their rituals in the protection of the darkness, and it's very rare to see a church openly dedicated to cosmism.

Cosmism explains some of the strange phenomena of Yuadrem as the whims and thoughts of The Observer.
The tides are the reactions of The Observer to the actions of each being.
Qualars are the medium by which people can commune with The Observer, granting them some of Their wisdom.

% Cosmists fear the cosmos, and carry strange and surreptitious rituals to appease The Observer or gain its favor.

However, what may be shunned in the surface can always find its place underground.
There seems to be a deep connection between the search of oneself in the larger scheme of things and the ego death experienced in the tsanek melds.
Many temples and ritual places exist in the mushroom cities of the cave-dwelling fungal kin, and cosmism is the official religion of the tsanek nation of Na'ane.
% Tsaneks view the cosmos with curiosity, and seek to understanding through observation and hypotheses.

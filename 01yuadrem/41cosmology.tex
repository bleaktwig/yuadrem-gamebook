% !TEX root = ../main.tex
\subsection*{Cosmology}
As far as is known in the mortal world, Yuadrem is the only continent in the planet of Darhoc, ergo the name ``One Land'' in Jantherlin.
Darhoc is the third planet orbiting Zash --- the white sun in the sky.

Zash is not alone at the center of the solar system, and the failed star Lenbier closely orbits it.
Despite being a second star, Lenbier has barely any effect in the lives of the peoples of Yuadrem --- barely shining more than a moon.
In increasing order of distance to Zash, the planets orbiting these two stars are listed.

\subparagraph{Rhelath} The twin planets, Gelthul and Ethul orbit each other four times a day.
Their violent rotation produces irregular, and noticeable effects on Darhoc's tides, comparable even to its moons.
Every sailor keeps in their calendar the days of Rhelath's closest approach each year, and only the mad or the suicidal set sail during this time.

\subparagraph{Ttinhoc} Darhoc's calmer neighbor, Ttinhoc is a large brown planet covered in rock and sand.
Many religions believe it to be the home of gods, hiding within its sands gates to the three deserts of Yuadrem.

\subparagraph{Badreth} A small, gray rock, Badreth is a planet easily forgotten.
An insignificant mote of dust in the grand scale of the cosmos.

\subparagraph{Rlabier} Along with its sibling, Rerhici, most days Rlabier is a hardly noticeable speck in the sky.
Some believe it to be the source of all windborn creatures, like the zaloth, for its grayish blue color reflects that of the sentient wind.

\subparagraph{Rerhici} The only planet orbitted by belts, Rerhici is believed to be the prison for ancient beasts of yore.
Twin planet-long chains make sure that whatever is trapped inside this land stays there until the end of time.

\subparagraph{Miraqual, Beroqual} Discovered in 662 and 663 AS respectively, Miraqual and Beroqual cannot be perceived by the naked eye.
While unknown to most commoners, they are a common subject of studies in Dnomitian universities, who tell stories of the twin blue giants with great passion.

\subparagraph{Thurbier} Despite its incredible distance from Darhoc, Thurbier is visible to the average person.
Believed to be the largest planet in the system, some claim that this faraway brown eye watches and protects Darhoc from the horrors of the distant cosmos.

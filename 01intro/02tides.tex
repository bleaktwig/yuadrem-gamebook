% !TEX root = ../main.tex
% The elusive white tide isn't explained here. The white tide is inaction, and is associated with observing events rather than taking part on them.
% Neither is the green tide, which concerns animal impulses.
\begin{linenumbers}
\begin{center}
    \includegraphics[width=0.46\textwidth]{01intro/img/02tides.png}
\end{center}

\subsection*{The Tides} \label{ssec::tides}
\DndDropCapLine{N}{o one can tell how old the tides are.}
Some speculate that they were created by the tall kin in their search for individuality.
Others claim that they are simply part of the nature of Yuadrem.
No evidence asserts either of these viewpoints.

The tides represent complicated concepts that aren't entirely definable by language, but maybe are best defined as currents of emotions.
They flow within each individual's mind, deeply connected to qualars and sentience.
There are even ways to influence the tides of others, but they were quickly forbidden after their discovery.

They are as inviolate as air, and only a few are capable of perceiving them.
Their discoverers, the Igneist school of thought, gave them symbolic colors based on how they correspond with emotional reactions:

\subparagraph{Blue Tide} Wisdom, enlightenment, and mysticism.
It is the tide of the ones whose goal is to expand the mind and the spirit.

\subparagraph{Red Tide} Passion, emotion, action, and zeal.
It is the tide of the ones whose goal is to live in the moment, to experience life to its fullest, or to follow their heart wherever it leads them.

\subparagraph{Silver Tide} Admiration of power and the seek of fame.
It is the tide of the ones who seek to influence the lives of others or who actively seek to be remembered.

\subparagraph{Indigo Tide} Justice, compromise, and the greater good.
It is the tide of the ones who view life's difficulties from a broad, global perspective rather than an individual one.

\subparagraph{Gold Tide} Charity, sacrifice, and empathy.
It is the tide of the ones whose primary goal is to help others, especially at a cost to themselves.

These five tides form the basis of how actions are perceived.
None of these concepts are as simple as ``good'' or ``evil''.
A silver-aligned creature might use their fame to help someone.
A gold-aligned being might help someone purely for their own benefit.

The tides are linked to action, not intention.
For instance, someone could be falsely compassionate, but they will still be in the sphere of the gold tide.
Someone could unconsciously give itself to its passions, and they will still be moving along the red tide.
While the tides are linked to all, only a few truly understand their significance.
Even fewer know how to manipulate them, and those who do are severely punished.

The phenomena of the tides was discovered long ago in Ignelli.
Its manipulation led to the devastating tidal sway in 247 AS, and has been banned ever since.
No one can however deny that they are not compelled by the mere concept of the tides, and they are commonly used to describe people and ideas, for better or for worse.
\end{linenumbers}

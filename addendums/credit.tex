% !TEX root = ../main.tex
\section{Credits}

Credit is given where credit is due.
There are at least one metric ton worth of hyperlinks here, so using the PDF version of the book is recommended if you want to check out the source content or the amazing artists behind the images in this book.

\subsection*{Yuadrem}
    \subparagraph{The Tides} The concept is directly taken from the \href{http://numenera.com/}{Numenera} universe, but many flavor twists and related mechanics are added by the author.

    \subparagraph{The Continent} The pictures used in this section where taken by various artists:
    The picture used to represent the Tallwoods is from the talented artist \href{https://www.artstation.com/tyleredlinart}{Tyler Edlin}.
    The one reminiscent of the Dead Sea is drawn by \href{https://www.artstation.com/knightblur}{Kevin Hou}, and the painting used for the Fog Gorge by \href{https://www.artstation.com/brucebrenn}{Bruce Brenn}.
    The picture used to represent Siszgoel is from the amazing swedish artist \href{https://www.artstation.com/davidvason}{David V\"ason}.
    Finally, the picture used to represent Om was drawn by the japanese artist \href{https://www.pixiv.net/en/users/2830609}{You Shimizu}, and the one associated to the forest surrounding the Pale Blemish by the parisian artist \href{https://www.artstation.com/maxfieve}{Max Fieve}.

    \subparagraph{Second Landing} The picture used to represent the Elderberry Wilds is again from \href{https://www.artstation.com/davidvason}{David V\"ason}.

    \subparagraph{Nyx} The image used for Nyx is stolen from ``Mythic Oddysseys of Theros'', as many other images used from this point onward.
    Also, the plane of Nyx is heavily inspired by the Underworld and Nyx from the Theros setting.
    My sincerest apology to Theros' fans if this becomes a bit confusing.

\subsection*{Viphoger}
    The whole Viphoger region is a re-skin of Theros, and the chapter itself is essentially a copy of the third chapter of ``Mythic Oddysseys of Theros''.
    Theros is a take on a Greece-inspired D\&D setting, and is very recommended by the author if you're looking for a campaign in such an environment.

    % \subparagraph{Cyclops} The picture used for the cyclops is taken from Final Fantasy XIV's concept art.
    % It's a pretty rad design, but I couldn't find the original author online to give proper credit.

\subsection*{Mechanic Changes}
    \subparagraph{Capline} The \textit{Anonymous Song} at the chapter's capline are the lyrics to Darren Korb's amazing song ``In the Flame'' from Pyre's soundtrack.
    The song is seriously recommended, along with the videogame if you're into that.

    \subparagraph{3 Action System} The system is clearly inspired by the Pathfinder 2E 3-action system, kudos to them for figuring out how to speed up combats in D\&D5e.
    Additionally, the descriptions for each action are taken from \href{https://crobi.github.io/dnd5e-quickref/}{dnd5e quickref} by Crobi.

    \subparagraph{Brutal Criticals, Injuries, and Insanity} The brutal criticals system is very heavily inspired by ``Critical Hits Revisited'', written by \href{https://www.sterlingvermin.com/}{Benjamin Huffman}.
    The Injuries and Insanity system is taken literally from the same book.

    \subparagraph{Equipment Properties} The armor properties are again taken from Pathfinder 2E, this time from their Armor Specialization system.

    \subparagraph{Dementia} The concept of dementia is heavily inspired off of Alzheimer disease.
    Most of the text and ideas explored come as inspiration from \href{https://thecaretaker.bandcamp.com/}{The Caretaker}, a project by James Leyland Kirby.

    \subparagraph{Only One Chance} While the idea itself is simple, this mechanic is taken from ``The Forbidden Lands'' RPG system.

    \subparagraph{Stress} This mechanic is adapted from the ``Darkest Dungeon'' stress mechanic, and afflictions are heavily inspired from the afflictions in that game.

\subsection*{Kins}
    \subparagraph{The Tall Kin} The picture used is taken from \href{https://www.playstation.com/en-us/games/bloodborne-ps4/}{Bloodborne}'s concept art.

    \subparagraph{The Horned Kin} Inspired by the \href{https://homebrewery.naturalcrit.com/share/BksGVG27b}{goatfolk} by \href{https://www.reddit.com/user/Vagar/}{Vagar}, and the \href{https://www.dandwiki.com/wiki/Geitlan_(5e_Race)}{Geitlan} by TreyBae and \href{https://www.dandwiki.com/wiki/User:ConcealedLight}{ConcealedLight} from the dandwiki.
    The strange mood mechanic is inspired from the strange moods in \href{http://www.bay12games.com/dwarves/}{Dwarf Fortress}, made by Bay12Games.
    All the beautiful illustrations were made by \href{https://www.artstation.com/hiziripro}{Satoshi Matsuura}.

    \subparagraph{The Winged Kin} Inspired by the official Aarakocra race from the Elemental Evil Player's Companion book by the Wizards of the Coast (WotC), but with a twist from older editions.
    The Qulbaba ird picture was drawn (as far as I am able to tell) by \href{https://www.artstation.com/elinvason}{Elin V\"a son}, and the Dratl ird picture by \href{https://zachcunninghamart.weebly.com/}{Zach Cunningham}.
    For the life of me I can't find the original artist behind the Thul'kraka ird picture, but it was found \href{https://www.pinterest.com.au/pin/772859986026889236/?nic_v1=1aMPfjrcEzGBNwm803q3V1cypforv8WVbZ4jUXYP9aDqcJFyrfUt0Ww9rAOEq3SPSw}{here}.

    \subparagraph{The Dust Kin} Inspired by the \href{https://drive.google.com/file/d/1M200-YKAbl-nOLo52W--gkXVO6QpmihE/view}{mothfolk race} by \href{https://twitter.com/aofhaocv}{aofhaocv}, the \href{http://volthorne.wikidot.com/kahakai:races}{mothfolk race} from the \href{http://volthorne.wikidot.com/kahakai}{Kahaki} setting, and the \href{https://drive.google.com/file/d/1hxPW6VRRlcWuK9ukljLCO3ORSNlonwm1/view}{Lunala race} by \href{https://www.reddit.com/user/Alessia45/}{Alessia45}.
    The art was taken from \href{https://www.bungie.net/}{Bungie's Destiny} videogame concept art.

    \subparagraph{The Archer Kin} Taken from \href{https://www.youtube.com/channel/UCncTjqw75krp9j_wRRh5Gvw}{Ewa U}'s video \href{https://www.youtube.com/watch?v=_XCqpZwm39Q}{"River Basin $\mid$ Archer Marmosets} with permission.
    All of the art was drawn by the talented \href{https://jayrockin.tumblr.com/}{jayrockin}.

    \subparagraph{The Moss Kin} Inspired by the vegepygmy race from the Volo's Guide to Monsters and the Tomb of Annihilation books by WotC.
    Some the traits were taken from the \href{https://blackbandos-homebrew.fandom.com/wiki/Vegepygmy_(5e_Race)}{blackbandos-homebrew} page, but most were designed based on the vegepygmies original abilities.
    Additionally, the Seedspeech trait is taken from the awesome \href{https://thedeckofmany.com/collections/humblewood}{Humblewood} campaign setting.
    The first image was taken from \href{http://ericbelisle.com/}{Eric Belistle}, while the second from the Volo's Guide to Monsters official book.

    \subparagraph{The Fungal Kin} Inspired by myconids, and based on their homebrew implementation from \href{https://www.dandwiki.com/wiki/Myconid_(5e_Race)}{dandwiki}, \href{https://www.reddit.com/r/UnearthedArcana/comments/5269hx/race_myconid/}{naturalcrit}, and \href{https://mfov.magehandpress.com/2015/09/myconids.html}{magehandpress}.
    The first image is made by an author that I'm unable to find, and the sovereign image is official WotC art.

    \subparagraph{The Shelled Kin} Pretty much directly taken from the Tortle Package by WotC, with some small twists added for flavor.
    The first tortle image is the cover of this package.
    I can't find the source of the image at the end, with the best lead I've found being this \href{https://avatarko.ru/kartinka/30751}{russian page}.

    \subparagraph{The Poison Kin} Taken from the One Grung Above PDF by WotC, with some extra bits added based on dart frog's anatomy.
    Both images are made by \href{https://www.artstation.com/rafis}{Rafis Khuzin}.

    \subparagraph{The Nomad Kin} Inspired by the variant humans from the Player's Handbook by WotC, with extra stuff added for flavor and to keep consistency with the other kins of Yuadrem.
    The first image is from the \href{https://www.kickstarter.com/projects/acherongames/brancalonia-the-spaghetti-fantasy-rpg?lang=es}{Brancalonia} setting, and the second and third are official art from the amazing \href{https://www.supergiantgames.com/games/pyre/}{Pyre} videogame.

    \subparagraph{The Storm Kin} Inspired by the ethereals from the Warcraft universe, with minor ideas taken from the \href{https://www.dandwiki.com/wiki/Vihar_(5e_Race)}{Vihar} homebrew race.
    The first image was made by \href{https://radikatt.tumblr.com/}{Radikatt}, and the second by \href{https://twitter.com/danscottart}{Dan Scott}.

    \subparagraph{The Retainer Kin} Inspired by the warforged from the Eberron: Rising from the Last War official book by WotC, with many of the descriptions and two of the subraces directly taken from this race.
    The first image was made by \href{https://www.reddit.com/user/Matasmic/}{Matasmic} and provided for free use (you're awesome man!), and the second one was made by \href{https://www.reddit.com/user/captdiablo/}{Captdiablo}.

\subsection*{Walk of Life}
    \subparagraph{Backgrounds} The backgrounds used in this book and the feats related to each are taken word-by-word from many official D\&D5e books, with minor mechanical changes to make them fit into Yuadrem's open playstyle.

    \subparagraph{Laborer} The laborer background is taken from the ``\href{https://standbyforadventure.blogspot.com/2014/08/d-background-laborer.html}{Stand by for Adventure}'' blog by Jon.

    \subparagraph{The Tidal Sway} The rendition of the tidal sway was drawn by the amazing Polish artist \href{https://grzegorzprzybys.artstation.com/}{Grzegorz Przybyś}.

    \subparagraph{Trebos} The image used to portray the king of the gray sands Trebos is drawn as a collaboration between \href{https://www.artstation.com/hiziripro}{Satoshi Matsuura} and \href{https://www.artstation.com/nicodemus}{Nicodemus Yang-Mattisson}.
